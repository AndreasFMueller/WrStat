%
% e06-junkscience.tex -- Die Mathematik hinter einem Spamfilter basieren
%                       auf einem Bayes-Filter
%
% (c) 2006 Prof Dr Andreas Mueller, Hochschule Rapperswil
% $Id: e06-junkscience.tex,v 1.5 2008/09/09 14:22:16 afm Exp $
%
\rhead{Junk Science}
\chapter{Junk Science}
Das Internet hat nicht nur der Wissenschaft neue Kommunikationsm"oglichkeiten
gebracht, sondern auch allen Arten von Pseudwissenschaften,
Aberglauben, Quacksalberei und Betr"ugern, wobei die Grenzen manchmal
fliessen sind. In diesem Abschnitt sollen ein paar Wahrscheinlichkeitstheoretische
"Uberlegungen zeigen, wie wahrscheinlich die von W"unschelruteng"angern,
Gedankenlesern, Hom"oo\-pathen, Graphologen, Astrologen und anderen Betr"ugern
und L"ugnern beanspruchten F"ahigkeiten sind, wenn man nur den Zufall wirken
l"asst.

\section{W"unschelruteng"anger}
In einem Versuch wird einem W"unschelruteng"anger erlaubt, eine einzelne,
unter einer von sieben identischen Abdeckungen verborgene Erzprobe allein
mit der Hilfe seiner W"unschelrute zu finden. Sobald er sich auf eine
der Stellen festgelegt hat, wird sie entfernt. Falls die Erzprobe darunter
nicht gefunden wird, kann er mit den verbleibenden Stellen nochmals
versuchen. Dies kann wiederholt werden, bis die Erzprobe gefunden wird.
Wie viele Versuche sind im Mittel n"otig, wenn der W"unschelruteng"anger
"uber keine besonderen F"ahigkeiten verf"ugt, sondern einfach nur r"at?

Wir betrachten also die Zufallsvariable $R$, die Nummer der Runde, in der
der W"unschelruteng"anger die Probe findet. Wenn der W"unschelruteng"anger
nur r"at, ist die Wahrscheinlichkeit, die Probe in der $k$-ten Runde zu finden
$\frac1{n-k+1}$. Soweit muss er jedoch erst gekommen sein. Dazu muss er in
den vorangehenden Runden daneben geraten haben. In der erste
Runde war die Wahrscheinlichkeit daf"ur $\frac{n-1}n$, in der zweiten
Runde war es $\frac{n-2}{n-1}$ etc. Die Wahrscheinlichkeit, genau in der
$k$-ten Runde die Probe zu finden ist also
$$p_k=\frac{(n-1)(n-2)\cdots (n-k+1)}{n(n-1)\cdots(n-k+2)}\cdot\frac1{n-k+1}=\frac1n$$
Der Erwartungswert von $R$ wird damit
$$E(R)=\sum_{k=1}^nk\frac1n=\frac1n\cdot\sum_{k=1}^n{n(n+1)}2=\frac{n+1}2.$$

Auch dieser Versuch wurde bereits durchgef"uhrt. Dabei fand der W"unschelruteng"anger
die Probe auch nach drei Versuchen nicht. Wie gross ist die Wahrscheinlichkeit,
die Probe in drei Versuchen durch blosses Raten zu finden? Gem"ass obigem ist
$$P(R\le m)= \sum_{k=1}^mp_k=\frac1n\sum_{k=1}^m1=\frac{m}{n}$$
In drei Versuchen kann man also die Probe mit Wahrscheinlichkeit $\frac37$ durch
blosses Raten finden.

\section{Psychometrie}
Ein ``Medium'' behaupten, Gegenst"ande mit Hilfe ihrer "ubersinnlichen F"ahigkeiten
ihren Besitzern zuordnen zu k"onnen. Dazu stellen in einem Versuch f"unf
zuf"allig ausgew"ahlte Personen ihren Schl"usselbund und ihr Portemonnaie zur
Ver\-f"ugung. Das Medium soll bestimmen, welcher Schl"usselbund zu welchem
Portemonnaie geh"ort.
Welche Anzahl von "Ubereinstimmungen kann man erwarten, wenn das ``Medium''
"uber keinerlei "ubersinnliche F"ahigkeiten verf"ugt, sondern einfach nur r"at?

Die Wahrscheinlichkeit, alle Schl"ussel richtig zuzuordnen, ist $\frac1{5!}=\frac1{120}$.
Wir bezeichnen mit $p_k$ die Wahrscheinlichkeit, genau $k$ Schl"ussel richtig zuzuordnen,
wir wissen also bereits $p_5=\frac1{120}$. F"ur $p_k$ mit $k<5$ zu berechnen, 
m"ussen wir die Zahl der M"oglichkeiten z"ahlen, genau $k$ richtige Zuordnungen
und $5-k$ falsche Zuordnungen vorzunehmen. Wenn 4 Zuordnungen richtig waren,
muss offensichtlich auch die f"unfte richtig sein. Wenn 3 Zuordnungen richtig waren,
k"onnen die verbleibenden zwei nur auf eine Art falsch sein, da es nur 2 m"ogliche
Zuordnungen unter den falschen gibt, wovon die eine richtig sein muss.

Mit Hilfe der Technik der erzeugenden Funktionen l"asst sich die Anzahl der
der m"oglichen Zuordnungen mit genau $k$ richtigen bestimmen, sie ist identisch
mit der Zahl der Permutationen von $n$ Objekten, die genau $k$ Fixpunkte haben.
Wir bezeichnen diese Zahlen mit $e_k$.

Es ist einfach zu bestimmen, wieviele Zuordnungen von $n$ Objekten auf $k$
vorgegebenen Objekten richtig sind. Offensichtlich k"onnen nur $n-k$ Objekte
noch verschoben werden, also gibt es $(n-k)!$ M"oglichkeiten.
Die Anzahl der M"oglichkeiten, mindestens $k$ {\it beliebige} Objekte korrekt
zuzuordnen ist jetzt
$$N_k=\binom{n}{k}(n-k)!=\frac{n!}{k!(n-k)!}=\frac{n!}{k!}.$$
Die Zahl $N_k$ l"asst sich auch so bestimmen: f"ur jede $k$-elementige
Teilmenge von Objekten z"ahlen wir die Anzahl der Permutationen, die 
alle diese Objekte richtig zuordnen. Wir k"onnten aber genausogut
f"ur jede Permutation z"ahlen, wie viele $k$-elementige Mengen von Objekten
von ihr richtig zugeordnet werden. Die Anzahl der $k$-elementigen Mengen,
die von einer Permutation richtig zugeordnet werden ist die Anzahl der
M"oglichkeiten, aus den richtig zugeordneten Elementen der Permutation
deren $k$ auszuw"ahlen. Jede Zuordnung mit $t$ richtigen tr"agt also
$\binom{t}{k}$ zu $N_k$ bei. Wenn es $e_k$ Permutationen gibt,
die genau $k$ Objekte richtig zuordnen, bedeutet dies
$$N_k=\sum_{t\ge 0}\binom{t}{k}e_t,\qquad k=0,1,2,\dots.$$
Das Problem ist also gel"ost, wenn die $e_t$ aus den $N_k$ bestimmt werden k"onnen.

Dazu betrachtet man die Funktionen
$$N(x)=\sum_{k}N_kx^k$$
und 
$$E(x)=\sum_{k}e_kx^k.$$
Es folgt
\begin{align*}
N(x)&=\sum_kN_kx^k=\sum_k\sum_t\binom{t}{k}e_tx^k\\
&=\sum_te_t\biggl(\sum_k\binom{t}{k}x^k\biggr)\\
&=\sum_te_t(x+1)^t\\
&=E(x+1)
\end{align*}
Durch Ersetzen von $x$ durch $x-1$ folgt
$$E(x)=N(x-1).$$
Da die Koeffizienten $N_k$ einfach zu bestimmen sind, lassen sich auch
die Koeffizienten von $E(x)$ durch einfaches Einsetzen einfach
bestimmen:
\begin{align*}
N(x)&=\sum_{k=0}^n\frac{n!}{k!}x^k=n!\sum_{k=0}^n\frac{x^k}{k!}\\
E(x)&=n!\sum_{k=0}^n\frac{(x-1)^k}{k!}
\end{align*}
Um die Wahrscheinlichkeit daf"ur zu bestimmen, genau $k$ richtige Zuordnungen nur
durch zu Zufall zu erreichen, muss $e_k$ noch durch $n!$ geteilt werden. Somit
ist der Koeffizient von $x^k$ in
$$\sum_{k=0}^n\frac{(x-1)^k}{k!}$$
die diese gesuchte Wahrscheinlichkeit.

Nun wird aber der Erwartungswert gesucht. Aus
$$\sum_{k=0}^np_kx^k=\sum_{k=0}^n\frac{(x-1)^k}{k!}$$
erh"alt man durch Ableiten
$$
\sum_{k=1}^nkp_kx^{k-1}
=\sum_{k=1}^n\frac{k(x-1)^{k-1}}{k!}
=\sum_{k=1}^n\frac{(x-1)^{k-1}}{(k-1)!}
=\sum_{k=0}^{n-1}\frac{(x-1)^k}{k!}
$$
Setzt man auf der linken Seite $x=1$ ein, entsteht die Summe zur
Berechnung des Erwartungswertes. Auf der rechten Seite hingegen
verschwinden alle Terme bis auf $k=0$, somit ist der Erwartungswert
$1$.
Interessant ist auch, dass der Erwartungswert sich nicht "andert,
wenn man die Zahl der Teilnehmer erh"oht.

Dieser Versuch wurde tats"achlich mit einem Medium durchgef"uhrt, welches
sich diesem Versuch zu unterziehen bereit war. Das Resultat entsprach genau
obigen Vorhersagen.

\section{Eine Variante: Graphologie}
Graphologen behaupten, aus der Handschrift einer Person ablesen zu k"onnen,
f"ur welche Art Beruf sie geeignet w"are, und f"ur welche nicht. Deshalb
werden sie auch immer wieder bei Personalentscheidungen herbeigezogen.
Zwar darf jeder Entscheider f"ur sich beanspruchen, diejenigen Entscheidungshilfen
hinzuzuziehen, die ihm am besten helfen. Wenn jedoch ein objektives
Auswahlverfahren durchgef"uhrt werden muss, wie dies zum Beispiel bei
"offentlich rechtlichen Anstellungen der Fall ist, muss man auch
fordern, dass die verwendeten Entscheidungsgrundlagen einer objektiven
"Uberpr"ufung standhalten. Bei der Graphologie darf das in Zweifel
gezogen werden.

Wie kann man dies testen? Man k"onnte einem Graphologen f"unf Schriftproben
von sehr unterschiedlichen Berufsleuten geben, und ihm auch die Berufe
der Autoren mitteilen. Seine Aufgabe ist sodann, die Schriftproben den
Personen zuzuordnen. Diese Aufgabe ist vollst"andig "aquivalent zur Aufgabe,
pers"onliche Objekte ihrem Besitzer zuzuordnen, die Analyse des vorangehenden
Abschnittes ist daher anwendbar. Wir k"onnen mit einer Erfolgsrate von einem
Treffer rechnen. Auch dieser Versuch ist durchgef"uhrt worden, mit genau
diesem Resultat.

Man k"onnten auch fragen, wie gross die Wahrscheinlichkeit ist, mehr als einen
Treffer durch Raten zu erzielen.
Dazu kann man erneut die erzeugende Funktion
heranziehen. F"ur $n=5$ ergibt die explizite Berechnung
$$\sum_{k=0}^5\frac{(x-1)^k}{k!}
=\frac{x^5}{120}+\frac{x^3}{12}+\frac{x^2}{6}+\frac{3 x}{8}+\frac{11}{30}
$$
Die Wahrscheinlichkeit, mindestens zwei Treffer zu erzielen, ist also
$$1-\left(\frac38+\frac{11}{30}\right)=\frac1{120}+\frac1{12}+\frac16=\frac{1+10+20}{120}=\frac{31}{120}>\frac14.$$
Anders formuliert: bei jedem zweiten Versuch schafft es jeder, h"ochstens $3$
Fehler zu machen. Dies d"urfte der Grund sein, warum den Graphologen
von gewissen Leuten "uberhaupt noch Glauben geschenkt wird.

Zwei Treffer von f"unf m"oglichen sind f"ur eine objektive Anstellungsentscheidung
jedoch nicht wirklich viel, denn immerhin lag der Graphologe in der Mehrzahl
der F"alle falsch! Wie gross ist die Wahrscheinlichkeit, durch blosses Raten
mehr als 50\% richtig zu bestimmen? Gem"ass der gerade
aufgelisteten Wahrscheinlichkeiten ist dies
$\frac1{120}+\frac1{12}=\frac{11}{121}<\frac1{11}$.

Andererseits sagen uns die Graphologen, dass 40\% der Bev"olkerung im falschen
Beruf arbeitet.
Ein Schelm, wer glaubt,
die Graphologen wollten mit diesem
Argument nur erkl"aren, warum sie keine h"ohere Trefferquote als $50\%$ hinkriegen!


\section{Hom"oopathie}
Die Hom"oopathie behauptet, dass ein Stoff, der normalerweise Krankheitssymptome
verursacht, zur Behandlung gegen genau diese Krankheitssymptome verwendet werden
kann, wenn man ihn nur gen"ugen stark verd"unnt. Die dabei verwendeten
Verd"unnungen von mindestens $10^{30}$ bis "uber $10^{1000}$ sind so gering,
dass die Wahrscheinlichkeit, "uberhaupt noch ein Wirkstoffmolek"ul in der
L"osung vorzufinden, kleiner als $10^{-6}$ ist.

In einem Versuch wurden daher in einem Versuch $2N$ Proben hergestellt,
von denen die eine H"alfte den Wirkstoff Histamin enthielten, die anderen 
nicht. Anschliessend wurden die Proben gem"ass den Vorschriften zur
Herstellung hom"oopatischer Medikamente im Verh"altnis $10^{-36}$
verd"unnt, so dass am Ende dieses Prozesses, $2N$ hom"oopatische
Pr"aparate vorliegen, wovon die H"alfte nie mit dem Wirkstoff in
Kontakt waren, w"ahrend die andere H"alfte gem"ass hom"oopathischer
Lehre besonders wirkungsvoll, weil stark verd"unnt sein sollte.

Nun werden alle Proben mit Zellen in Verbindung gebracht, die normalerweise
auf diesen Wirkstoff reagieren. Nach hom"oopatischer Lehre sollten vorwiegend
diejenigen Proben eine Reaktion hervorrufen, welche einst in Verbindung
mit dem Wirkstoff gestanden haben. Man erwartet nat"urlich, dass 
alle Proben gleichermassen eine Wirkung zeigen.

Nehmen wir an, dass $p_0$ die Wahrscheinlichkeit ist, dass eine Probe ohne
Wirkstoff eine Reaktion ausl"ost, und $p_1$ die Wahrscheinlichkeit, dass
eine hom"oopatische Probe eine Wirkung ausl"ost. Wie gross muss der Unterschied
von $p_0$ und $p_1$ sein, damit wir sicher sein k"onnen, dass das hom"oopatische
Pr"aparat eine Wirkung hat?

Da jede einzelne Probe individuell mit Wahrscheinlichkeit $p$ getestet wird, liegt
hier eine Binomialverteilung vor. Getestet werden soll die Hypothese, dass zwei
Binomalverteilung den gleichen Parameter haben. Dazu kann ein $\chi^2$-Test oder
ein Kolmogoroff-Smirnov-Test verwendet werden. Um eine ungef"ahre Gr"ossenordnung
zu bekommen, kann man aber auch die Binomialverteilung durch eine Normalverteilung
approximieren, die beiden Normalverteilungen haben unter der Hypothese den
gleichen Erwartungswert.
Der $t$-Test erm"oglicht zu entscheiden, ob zwei Normalverteilungen
verschiedenen Mittelwert haben, also $Np_0\ne Np_1$. Dazu muss man die
Gr"osse
$$T=\frac{\bar X - \bar Y}{\sqrt{S_X^2+S_Y^2}}\sqrt{\frac{N^2(2N-2)}{{2N}{(N-1)}}}
=
\frac{\bar X - \bar Y}{\sqrt{S_X^2+S_Y^2}}\sqrt{N}$$
Bei $N=10$ Proben verwirft dieser Test die Hypothese
auf dem Niveau $\alpha=5\%$, sobald
$T>2.228$.

Dieser Versuch wurde vor einigen Jahren vom britischen Fernsehen BBC mit der
Unterst"utzung einiger renommierter Forschungslabors durchgef"uhrt, mit dem
erwarteten Resultat.
Ein anderes Resultat war auch nicht zu erwarten. Die Lehre der Hom"oopathie
ist in sich bereits so widerspr"uchlich, dass kein Resultat mit der ganzen
Lehre in Einklang stehen kann.
Da im Meer alle m"oglichen Giftstoffe extrem verd"unnt vorliegen, m"ussten
Leute, die Meerwasser aufnehmen, gem"ass hom"oopatischer Lehre gegen alle
ihre Symptome gesch"utzt sein.

