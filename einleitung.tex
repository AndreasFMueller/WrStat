%SourceDoc ws-skript.tex
% a-einleitung.tex -- Einleitung zum Skript
%
% (c) 2006 Prof. Dr. Andreas Mueller, HSR
% $Id: a-einleitung.tex,v 1.9 2008/09/15 23:49:57 afm Exp $
%
\rhead{Einleitung}
\chapter*{Einleitung}
\section*{Was ist m"oglich?}
Es ist unm"oglich, die unz"ahligen Dichteschwankungen, Wirbel und
schwachen Luftstr"omungen zu beschreiben, die die Flugbahn eines
Golfballes beeinflussen. Trotzdem gelingt es dem guten Golfspieler,
dan Ball sehr nahe an das Loch zu spielen, so dass er ihn mit wenigen
weiteren Schl"agen einlochen kann.

Es ist unm"oglich, die Bewegung aller Atome eines Gases zu kennen.
Im achtzehnten Jahrhundert, als die newtonsche Mechanik mit der
exakten Vorhersage aller Bewegungen im Sonnensystem grosse Triumphe
feierte, herrschte die "Uberzeugung, dass sich die Zukunft verhersagen
liesse, wenn man nur die Anfangsbedingung kennen w"urde.
Der Mathematiker Pierre-Simon Laplace (1749-1827) hat noch geglaubt hat, dass dies
wenigstens im Prinzip m"oglich sein m"usste, als er Napoleon auf die Frage
nach dem Sch"opfer entgegnete: ``Je n'avais pas besoin de cette hypoth\`ese''.

Die Quantenmechanik hat
im 20.~Jahrhundert gezeigt, dass man die notwendige Kenntnis auch
grunds"atzlich nicht haben kann. Trotzdem gelingt es der Thermodynamik,
n"utzliche Gesetzm"assigkeiten zu formulieren. Die
Vorg"ange in der Atmosph"are, in einem Verbrennungsmotor oder die
Str"omung um ein "Uberschallflugzeug oder in einem Raketenmotor
lassen sich genau genug berechnen, dass einigermassen zuverl"assige
Wetterprognosen m"oglich werden, dass man lange nicht mehr so viele
Windkanaltests f"ur die Entwicklung eines Flugzeuges vorsehen muss,
oder dass man die Eigenschaften eines Raketenmotors auf dem
Pr"ufstand nur noch experimentell best"atigen muss.

Technische Grenzen zwingen auch in der Praxis zur Anwendung statistischer
Methoden. Eine Einzelmessung eines Radarger"ates ist prinzipbedingt
von geringer
Genauigkeit. Da sich das beobachtete Flugzeug aber sehr schnell bewegen
kann, kann man nicht einfach nacheinander gemessen Positionen mitteln
und glauben, damit das Resultat zu verbessern. Auch die Positionsbestimmung
eines GPS-Empf"angers wird durch ein aufwendiges statistisches Verfahren
verbessert.

Es ist unm"oglich vorauszusagen, ob ein Haus abbrennen wird, und
trotzdem kann man Gesetzm"assigkeiten finden, die erlauben, das
Risiko bei einer grossen Zahl von H"ausern zu kalkulieren.
Versicherungen bauen ihr Gesch"aft auf der Tatsache auf, dass
sich der zu erwartende Schaden berechnen l"asst, und decken ihn
rechtzeitig durch Einzug gen"ugend hoher Pr"amien.

Es ist unm"oglich, alle Stimmen in einer politischen Abstimmung
schon kurz nach Erf"offnung der Wahllokale zu wissen. Die meisten
W"ahler haben ihre Stimme ja noch gar nicht abgegeben. Trotzdem git
es Verfahren, "uber die grosse Gesamtheit aller Stimmberechtigten
Aussagen zu machen, was medial in endlosen und pseudospannenden
Wahlsendungen ausgeschlachtet wird.

Es ist unm"oglich zu wissen, welche Zahl im Roulette beim n"achsten
Spiel angezeigt wird. Aber das Spielcasino ist in der Lage, die
zu erwartenden Gewinne der Mitspieler bei einer grossen Zahl der
Spiele abzusch"atzen und vorauszusehen, dass auf die Dauer f"ur
das Casino immer ein Gewinn, f"ur den Spieler ein Verlust herausschaut.
Es ist immer noch m"oglich, dass einzelne Spieler vereinzelt
gewinnen, dies ist f"ur Marketing-Zwecke auch wichtig, g"alte
dies jedoch f"ur eine grosse Zahl von Spielern, w"are der Betrieb
eines Spielcasinos wirtschaftlich nicht attraktiv.

\section*{Die Kunst des Vermutens}
Bei einer grossen Zahl von Einfl"ussen auf den Ausgang eines Experimentes
wird es praktisch unm"oglich, den genauen Ausgang vorauszusagen.
Trotzdem ist es m"oglich, Gesetzm"assigkeiten zu formulieren, mit
denen der Ingenieur seine Hypothesen "uber seine Konstruktion
verifizieren kann.

In vielen Bereichen der Wissenschaft und Technik ist es unm"oglich, den
genauen Ausgang eines Experimentes vorherzusagen. Trotzdem werden
Aussagen "uber den zu erwartenden Ausgang ben"otigt. So kann zum Beispiel
eine Fluggesellschaft nicht wissen, ob alle Passagiere f"ur einen
bestimmten Flug erscheinen werden. Sie hat die Wahl, f"ur jedes verkaufte
Ticket auch einen Sitzplatz bereit zu halten, oder eine gewisse Anzahl
Pl"atze doppelt zu verkaufen. Sie w"agt dabei ab zwischen der finanziellen
Ineffizienz eines nicht voll besetzten Flugzeugs und dem m"oglichen
finanziellen Schaden durch einen Passagier, der keinen Sitzplatz mehr
erhalten hat, und der auf einen anderen Flug umgebucht werden muss.
"Uber die Zahl der nicht erscheinenden Flugg"aste ist nat"urlich nichts
genaues bekannt, die Fluggesellschaft kann sich nur in der Kunst "uben,
Vermutungen dar"uber anzustellen.

Genau in diesem Sinne hat im achtzehnten Jahrhundert
Jakob Bernoulli II.~das Problem angepackt und in seinem Buch
{\em Ars conjectandi} (lat. die Kunst des Vermutens) die Kunst,
Vermutungen zur Wissenschaft zu erheben. Er schuf damit das
neue Gebiet der Wahrscheinlichkeitsrechnung. Sie stellt die theoretischen Grundlagen
bereit, auf denen Aussagen "uber eine grosse Zahl von gleichartigen
Ereignissen oder Individuen formuliert werden k"onnen.

Die Wahrscheinlichkeitsrechnung lehrt, wie auf rationale
Art Vermutungen angestellt werden k"onnen. Nat"urlich braucht diese
Kunst des Vermutens auch Fakten, auf die sich die Vermutungen
abst"utzen k"onnen. Der Natur der Sache nach sind auch diese
Fakten nicht unbedingt absolute Wahrheiten, sondern Aussagen,
die in der "uberwiegenden Mehrzahl der F"alle zutreffen werden.
Es ist die Aufgabe der Statistik, experimentelle Verfahren zu finden,
mit denen die Zuverl"assigkeit der Faktengrundlage ermittelt
werden kann. Wahrscheinlichkeit und Statistik bilden also ein
untrennbares wissenschaftliches Geschwisterpaar, welches umgangssprachlichen
Wendungen wie ``meistens'', ``selten'', ``wahrscheinlich'' oder  ``unwahrscheinlich''
einen klar definierten und messbaren Sinn gibt.

\section*{Vorgehen}
Die Kunst, Vermutungen anzustellen, und die Kunst, eine f"ur das Vermuten
geeignete Datengrundlage zu erstellen, werden in diesem Skript
nebeneinander entwickelt. Zun"achst m"ussen einige Grundbegriffe eingef"uhrt
und ihre Eigenschaften verstanden werden. Damit lassen sich sofort
viele Statistiken verstehen. Aber erst der Begriff der Zufallsvariable und
der Verteilungen schafft eine
geeignete Grundlage f"ur die Beurteilung wissenschaftlicher Messdaten.

