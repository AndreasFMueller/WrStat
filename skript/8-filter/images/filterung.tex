%
% faltung.tex
%
% (c) 2020 Prof Dr Andreas Müller, Hochschule Rapperswil
%
\documentclass[tikz]{standalone}
\usepackage{times}
\usepackage{amsmath}
\usepackage{txfonts}
\usepackage[utf8]{inputenc}
\usepackage{graphics}
\usepackage{color}
\usetikzlibrary{arrows,intersections,math}
\usepackage{pgfplots}
\begin{document}

\definecolor{darkgreen}{rgb}{0,0.6,0}
\definecolor{pink}{rgb}{0.8,0.2,0.8}

\def\achsen{
\draw[->] (-0.03,0)--(4.2,0) coordinate[label={$t$}];
\draw[->] (0,-0.03)--(0,1.5);
\draw (1,-0.03)--(1,0.03);
\draw (2,-0.03)--(2,0.03);
\draw (3,-0.03)--(3,0.03);
\draw (4,-0.03)--(4,0.03);
\node at (1,-0.03) [below] {$1$};
\node at (2,-0.03) [below] {$2$};
\node at (3,-0.03) [below] {$3$};
\node at (4,-0.03) [below] {$4$};
\node at (0,-0.03) [below] {$0$};
\draw (-0.03,1)--(0.03,1);
\node at (-0.03,1) [left] {$1$};
}

\def\noisy{\draw[linewidth=1pt,color=red] (0.000,0.139)
--(0.016,0.134)
--(0.031,0.144)
--(0.047,0.217)
--(0.062,0.178)
--(0.078,0.268)
--(0.094,0.303)
--(0.109,0.342)
--(0.125,0.380)
--(0.141,0.360)
--(0.156,0.398)
--(0.172,0.418)
--(0.188,0.472)
--(0.203,0.492)
--(0.219,0.435)
--(0.234,0.585)
--(0.250,0.562)
--(0.266,0.572)
--(0.281,0.632)
--(0.297,0.675)
--(0.312,0.652)
--(0.328,0.646)
--(0.344,0.718)
--(0.359,0.709)
--(0.375,0.722)
--(0.391,0.730)
--(0.406,0.819)
--(0.422,0.788)
--(0.438,0.841)
--(0.453,0.864)
--(0.469,0.859)
--(0.484,0.901)
--(0.500,0.939)
--(0.516,0.967)
--(0.531,0.943)
--(0.547,0.954)
--(0.562,1.011)
--(0.578,1.027)
--(0.594,1.010)
--(0.609,1.001)
--(0.625,1.012)
--(0.641,1.061)
--(0.656,1.037)
--(0.672,1.059)
--(0.688,1.160)
--(0.703,1.093)
--(0.719,1.149)
--(0.734,1.184)
--(0.750,1.196)
--(0.766,1.157)
--(0.781,1.214)
--(0.797,1.283)
--(0.812,1.245)
--(0.828,1.294)
--(0.844,1.287)
--(0.859,1.253)
--(0.875,1.273)
--(0.891,1.285)
--(0.906,1.343)
--(0.922,1.307)
--(0.938,1.422)
--(0.953,1.323)
--(0.969,1.302)
--(0.984,1.279)
--(1.000,1.277)
--(1.016,1.235)
--(1.031,1.242)
--(1.047,1.332)
--(1.062,1.223)
--(1.078,1.179)
--(1.094,1.174)
--(1.109,1.165)
--(1.125,1.122)
--(1.141,1.175)
--(1.156,1.105)
--(1.172,1.094)
--(1.188,1.057)
--(1.203,1.088)
--(1.219,1.036)
--(1.234,1.100)
--(1.250,0.975)
--(1.266,1.046)
--(1.281,1.040)
--(1.297,0.952)
--(1.312,0.952)
--(1.328,0.918)
--(1.344,0.892)
--(1.359,0.906)
--(1.375,0.827)
--(1.391,0.889)
--(1.406,0.853)
--(1.422,0.842)
--(1.438,0.846)
--(1.453,0.816)
--(1.469,0.757)
--(1.484,0.815)
--(1.500,0.797)
--(1.516,0.729)
--(1.531,0.761)
--(1.547,0.760)
--(1.562,0.738)
--(1.578,0.777)
--(1.594,0.731)
--(1.609,0.700)
--(1.625,0.707)
--(1.641,0.705)
--(1.656,0.644)
--(1.672,0.744)
--(1.688,0.572)
--(1.703,0.685)
--(1.719,0.677)
--(1.734,0.629)
--(1.750,0.599)
--(1.766,0.665)
--(1.781,0.552)
--(1.797,0.568)
--(1.812,0.574)
--(1.828,0.560)
--(1.844,0.541)
--(1.859,0.572)
--(1.875,0.564)
--(1.891,0.486)
--(1.906,0.579)
--(1.922,0.546)
--(1.938,0.504)
--(1.953,0.476)
--(1.969,0.461)
--(1.984,0.402)
--(2.000,0.566)
--(2.016,0.479)
--(2.031,0.484)
--(2.047,0.463)
--(2.062,0.428)
--(2.078,0.442)
--(2.094,0.403)
--(2.109,0.409)
--(2.125,0.404)
--(2.141,0.440)
--(2.156,0.430)
--(2.172,0.382)
--(2.188,0.362)
--(2.203,0.348)
--(2.219,0.456)
--(2.234,0.406)
--(2.250,0.327)
--(2.266,0.344)
--(2.281,0.328)
--(2.297,0.332)
--(2.312,0.333)
--(2.328,0.284)
--(2.344,0.374)
--(2.359,0.311)
--(2.375,0.277)
--(2.391,0.338)
--(2.406,0.353)
--(2.422,0.369)
--(2.438,0.292)
--(2.453,0.338)
--(2.469,0.286)
--(2.484,0.282)
--(2.500,0.296)
--(2.516,0.249)
--(2.531,0.318)
--(2.547,0.290)
--(2.562,0.296)
--(2.578,0.230)
--(2.594,0.191)
--(2.609,0.247)
--(2.625,0.245)
--(2.641,0.278)
--(2.656,0.244)
--(2.672,0.242)
--(2.688,0.246)
--(2.703,0.229)
--(2.719,0.217)
--(2.734,0.207)
--(2.750,0.236)
--(2.766,0.258)
--(2.781,0.214)
--(2.797,0.185)
--(2.812,0.250)
--(2.828,0.168)
--(2.844,0.198)
--(2.859,0.195)
--(2.875,0.169)
--(2.891,0.202)
--(2.906,0.191)
--(2.922,0.164)
--(2.938,0.164)
--(2.953,0.217)
--(2.969,0.145)
--(2.984,0.173)
--(3.000,0.141)
--(3.016,0.196)
--(3.031,0.204)
--(3.047,0.128)
--(3.062,0.186)
--(3.078,0.241)
--(3.094,0.075)
--(3.109,0.100)
--(3.125,0.172)
--(3.141,0.149)
--(3.156,0.186)
--(3.172,0.154)
--(3.188,0.171)
--(3.203,0.160)
--(3.219,0.130)
--(3.234,0.101)
--(3.250,0.132)
--(3.266,0.133)
--(3.281,0.143)
--(3.297,0.123)
--(3.312,0.159)
--(3.328,0.093)
--(3.344,0.133)
--(3.359,0.115)
--(3.375,0.132)
--(3.391,0.103)
--(3.406,0.104)
--(3.422,0.162)
--(3.438,0.120)
--(3.453,0.098)
--(3.469,0.128)
--(3.484,0.125)
--(3.500,0.063)
--(3.516,0.108)
--(3.531,0.143)
--(3.547,0.082)
--(3.562,0.136)
--(3.578,0.091)
--(3.594,0.040)
--(3.609,0.131)
--(3.625,0.100)
--(3.641,0.112)
--(3.656,0.179)
--(3.672,0.042)
--(3.688,0.051)
--(3.703,0.071)
--(3.719,0.135)
--(3.734,0.077)
--(3.750,0.071)
--(3.766,0.029)
--(3.781,0.059)
--(3.797,0.130)
--(3.812,0.033)
--(3.828,0.052)
--(3.844,-0.014)
--(3.859,0.066)
--(3.875,0.042)
--(3.891,0.066)
--(3.906,0.085)
--(3.922,0.030)
--(3.938,0.060)
--(3.953,-0.004)
--(3.969,0.074)
--(3.984,0.015)
--(4.000,0.093)
}
\def\reconstructed{\draw[linewidth=1pt,color=blue] (0.000,0.184)
--(0.016,0.178)
--(0.031,0.191)
--(0.047,0.286)
--(0.062,0.235)
--(0.078,0.355)
--(0.094,0.402)
--(0.109,0.453)
--(0.125,0.503)
--(0.141,0.476)
--(0.156,0.527)
--(0.172,0.553)
--(0.188,0.625)
--(0.203,0.651)
--(0.219,0.575)
--(0.234,0.774)
--(0.250,0.744)
--(0.266,0.757)
--(0.281,0.837)
--(0.297,0.894)
--(0.312,0.862)
--(0.328,0.854)
--(0.344,0.950)
--(0.359,0.938)
--(0.375,0.955)
--(0.391,0.966)
--(0.406,1.084)
--(0.422,1.042)
--(0.438,1.112)
--(0.453,1.143)
--(0.469,1.136)
--(0.484,1.192)
--(0.500,1.242)
--(0.516,1.280)
--(0.531,1.248)
--(0.547,1.263)
--(0.562,1.337)
--(0.578,1.359)
--(0.594,1.337)
--(0.609,1.325)
--(0.625,1.339)
--(0.641,1.403)
--(0.656,1.372)
--(0.672,1.401)
--(0.688,1.535)
--(0.703,1.446)
--(0.719,1.520)
--(0.734,1.566)
--(0.750,1.582)
--(0.766,1.530)
--(0.781,1.606)
--(0.797,1.697)
--(0.812,1.647)
--(0.828,1.712)
--(0.844,1.702)
--(0.859,1.658)
--(0.875,1.685)
--(0.891,1.700)
--(0.906,1.777)
--(0.922,1.729)
--(0.938,1.881)
--(0.953,1.751)
--(0.969,1.723)
--(0.984,1.693)
--(1.000,1.689)
--(1.016,1.634)
--(1.031,1.644)
--(1.047,1.762)
--(1.062,1.618)
--(1.078,1.560)
--(1.094,1.553)
--(1.109,1.541)
--(1.125,1.484)
--(1.141,1.554)
--(1.156,1.462)
--(1.172,1.448)
--(1.188,1.398)
--(1.203,1.440)
--(1.219,1.371)
--(1.234,1.456)
--(1.250,1.290)
--(1.266,1.383)
--(1.281,1.377)
--(1.297,1.260)
--(1.312,1.259)
--(1.328,1.215)
--(1.344,1.180)
--(1.359,1.198)
--(1.375,1.094)
--(1.391,1.176)
--(1.406,1.128)
--(1.422,1.114)
--(1.438,1.119)
--(1.453,1.079)
--(1.469,1.001)
--(1.484,1.079)
--(1.500,1.054)
--(1.516,0.964)
--(1.531,1.006)
--(1.547,1.005)
--(1.562,0.977)
--(1.578,1.028)
--(1.594,0.967)
--(1.609,0.927)
--(1.625,0.935)
--(1.641,0.933)
--(1.656,0.852)
--(1.672,0.984)
--(1.688,0.756)
--(1.703,0.906)
--(1.719,0.896)
--(1.734,0.832)
--(1.750,0.792)
--(1.766,0.880)
--(1.781,0.730)
--(1.797,0.751)
--(1.812,0.759)
--(1.828,0.741)
--(1.844,0.716)
--(1.859,0.757)
--(1.875,0.746)
--(1.891,0.643)
--(1.906,0.766)
--(1.922,0.723)
--(1.938,0.667)
--(1.953,0.629)
--(1.969,0.610)
--(1.984,0.532)
--(2.000,0.749)
--(2.016,0.634)
--(2.031,0.641)
--(2.047,0.612)
--(2.062,0.567)
--(2.078,0.584)
--(2.094,0.533)
--(2.109,0.541)
--(2.125,0.534)
--(2.141,0.582)
--(2.156,0.569)
--(2.172,0.505)
--(2.188,0.478)
--(2.203,0.460)
--(2.219,0.603)
--(2.234,0.537)
--(2.250,0.432)
--(2.266,0.455)
--(2.281,0.434)
--(2.297,0.439)
--(2.312,0.441)
--(2.328,0.376)
--(2.344,0.495)
--(2.359,0.412)
--(2.375,0.366)
--(2.391,0.447)
--(2.406,0.468)
--(2.422,0.489)
--(2.438,0.386)
--(2.453,0.448)
--(2.469,0.378)
--(2.484,0.373)
--(2.500,0.392)
--(2.516,0.330)
--(2.531,0.420)
--(2.547,0.383)
--(2.562,0.392)
--(2.578,0.305)
--(2.594,0.253)
--(2.609,0.327)
--(2.625,0.325)
--(2.641,0.368)
--(2.656,0.322)
--(2.672,0.320)
--(2.688,0.325)
--(2.703,0.303)
--(2.719,0.287)
--(2.734,0.274)
--(2.750,0.313)
--(2.766,0.342)
--(2.781,0.283)
--(2.797,0.244)
--(2.812,0.330)
--(2.828,0.222)
--(2.844,0.262)
--(2.859,0.257)
--(2.875,0.224)
--(2.891,0.267)
--(2.906,0.252)
--(2.922,0.217)
--(2.938,0.217)
--(2.953,0.288)
--(2.969,0.192)
--(2.984,0.229)
--(3.000,0.187)
--(3.016,0.259)
--(3.031,0.270)
--(3.047,0.170)
--(3.062,0.246)
--(3.078,0.319)
--(3.094,0.099)
--(3.109,0.133)
--(3.125,0.228)
--(3.141,0.197)
--(3.156,0.247)
--(3.172,0.203)
--(3.188,0.226)
--(3.203,0.212)
--(3.219,0.171)
--(3.234,0.134)
--(3.250,0.174)
--(3.266,0.176)
--(3.281,0.190)
--(3.297,0.163)
--(3.312,0.210)
--(3.328,0.123)
--(3.344,0.176)
--(3.359,0.153)
--(3.375,0.175)
--(3.391,0.136)
--(3.406,0.137)
--(3.422,0.215)
--(3.438,0.159)
--(3.453,0.129)
--(3.469,0.169)
--(3.484,0.166)
--(3.500,0.083)
--(3.516,0.142)
--(3.531,0.189)
--(3.547,0.109)
--(3.562,0.180)
--(3.578,0.121)
--(3.594,0.053)
--(3.609,0.173)
--(3.625,0.133)
--(3.641,0.148)
--(3.656,0.237)
--(3.672,0.056)
--(3.688,0.068)
--(3.703,0.094)
--(3.719,0.178)
--(3.734,0.102)
--(3.750,0.093)
--(3.766,0.039)
--(3.781,0.078)
--(3.797,0.172)
--(3.812,0.044)
--(3.828,0.069)
--(3.844,-0.019)
--(3.859,0.087)
--(3.875,0.055)
--(3.891,0.087)
--(3.906,0.113)
--(3.922,0.039)
--(3.938,0.079)
--(3.953,-0.005)
--(3.969,0.098)
--(3.984,0.020)
--(4.000,0.122)
}


\begin{tikzpicture}[>=latex,thick,scale=3]

\begin{scope}
\achsen
\clean
\noisy
\end{scope}

\begin{scope}[yshift=-1.8cm]
\achsen
\cleanreconstructed
\reconstructed
\end{scope}

\end{tikzpicture}

\end{document}
