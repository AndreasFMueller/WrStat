%
% faltung.tex
%
% (c) 2020 Prof Dr Andreas Müller, Hochschule Rapperswil
%
\documentclass[tikz]{standalone}
\usepackage{times}
\usepackage{amsmath}
\usepackage{txfonts}
\usepackage[utf8]{inputenc}
\usepackage{graphics}
\usepackage{color}
\usetikzlibrary{arrows,intersections,math}
\usepackage{pgfplots}
\begin{document}

\begin{tikzpicture}[>=latex,thick,scale=3]

\draw[->] (-1.1,0)--(3.2,0) coordinate[label={$t$}];
\draw[->] (0,-0.1)--(0,1) coordinate[label={right:$G*X$}];

\draw[line width=1pt,color=red] (-1.05,0)--(0,0)
        --
        plot[domain=0:1,samples=20] ({\x},{1-exp(-\x)})
        --
        plot[domain=1:3.1,samples=20] ({\x},{exp(-\x)*(exp(1)-1)});

\draw (-1,-0.05)--(-1,0.05);
\draw (1,-0.05)--(1,0.05);
\draw (2,-0.05)--(2,0.05);
\draw (3,-0.05)--(3,0.05);
\node at (-1,-0.05) [below] {$-1$};
\node at (1,-0.05) [below] {$1$};
\node at (2,-0.05) [below] {$2$};
\node at (3,-0.05) [below] {$3$};
\node at (0,-0.05) [below right] {$0$};

\end{tikzpicture}

\end{document}
