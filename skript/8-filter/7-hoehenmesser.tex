%
% hoehenmesser.tex -- Hoehenmesser mit Kalman-Filter
%
% (c) 2006-2015 Prof. Dr. Andreas Mueller, Hochschule Rapperswil
%

\section{Beispiel: Höhenmesser mit Scheitelpunktermittlung}
\kopfrechts{Höhenmesser}
Modellraketen verwenden oft einen elektronischen Höhenmesser, der die Höhe
barometrisch mit Hilfe eines Drucksensors bestimmt.
Die fliegende Modellrakete
kann in erster Näherung als Massepunkt behandelt werden, der nur unter dem
Einfluss der konstanten Schwerkraft fliegt.
Der Systemzustand besteht
dann aus Höhe $h$, Vertikalgeschwindigkeit $v$ und Beschleunigung $a$.
Die Kinematik beschreibt die zeitliche Änderung mit Hilfe des
Differentialgleichungssystems
\[
\frac{d}{dt}
\begin{pmatrix}
h\\v\\a
\end{pmatrix}
=
\begin{pmatrix}
0&1&0\\
0&0&1\\
0&0&0
\end{pmatrix}
\begin{pmatrix}
h\\v\\a
\end{pmatrix}
\]
Die Entwicklungsgleichung für einen Zeitschritt der Länge $\Delta t$ ist
\[
\varphi=\begin{pmatrix}
1&\Delta t&\frac{\Delta t^2}2\\
0&1&\Delta t\\
0&0&1\\
\end{pmatrix}
\]
Wir nehmen an, dass die Systemfehler, also die Matrix $Q$ ebenfalls diagonal sind,
und alle Komponenten den gleichen Fehler $\sigma^2$ haben, also
\[
Q=\begin{pmatrix}
\sigma^2&0&0\\
0&\sigma^2&0\\
0&0&\sigma^2\\
\end{pmatrix}
\]

Mit dem Drucksensor kann nur die Höhe bestimmt werden, also
\[
H=\begin{pmatrix}1&0&0\end{pmatrix}
\]
und entsprechend ist die Matrix $R_k$ ebenfalls nur eine Zahl, welche
den Messfehler $\varrho^2$ der Messung wiedergibt.

Zur Berechnung der Kalmanmatrix muss die Matrix $HP_{k+1|k}H^t+R$ invertiert werden.
Aber $HP_{k+1|k}H^t$ ist einfach nur die Zahl in der linken oberen Ecke, also
der mittlere quadratische Schätzfehler der Höhe.
Das Produkt $P_{k+1|k}H^t$
ist die erste Spalte der Kovarianzmatrix.
Die Kalmanmatrix ist also die erste
Spalte der Kovarianzmatrix dividiert durch $(P_{k+1|k})_{11}+\rho^2$.

