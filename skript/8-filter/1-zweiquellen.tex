%
% zweiquellen.tex
%
% (c) 2020 Prof Dr Andreas Müller, Hochschule Rapperswil
%
\section{Optimale Mittelung
\label{section:optimale-mittelung}}
\kopfrechts{Optimale Mittelung}
Ein Filter ist in der Lage, von einem Zufallsprozess gestörte Information 
aus einer oder mehreren Quellen auf optimale Art und Weise wiederherzustellen.
Als protoypisches Beispiel für diese Art von Problem betrachten wir hier
die folgende Aufgabe.

\begin{aufgabe}
\label{filter:aufgabe1}
Eine Grösse wird mit zwei verschiedenen Messgeräten mit unterschiedlichen
Messfehlern gemessen.
Wie müssen die Resultate gemittelt werden, um die bestmögliche Schätzung
für die gemessen Grösse zu bekommen?
\end{aufgabe}

Etwas abstrakter können wir das wie folgt formulieren.
Seien $X_1$ und $X_2$ Zufallsvariablen mit Varianzen
$\operatorname{var}(X_1)=\sigma_1^2$ und
$\operatorname{var}(X_2)=\sigma_2^2$, die beide den gleichen Erwartungswert
$E(X_1)=E(X_2)=\mu$ haben.
Die Zufallsvariable $tX_1 + sX_2$ ist eine Mittelung von der beiden
Messwerte mit verschiedenen Gewichte und eine Schätzung für $\mu$.
Der Erwartungswert davon ist
\[
E(tX_1+sX_2)
=
tE(X_1) + sE(X_2)
=
(t+s)\mu.
\]
Die Mittelwertbildung ist also nur dann ein erwartungstreuer Schätzer
für $\mu$, wenn $t+s=1$ ist oder $s=1-t$.
Damit wird die Aufgabe~\ref{filter:aufgabe1} zur folgenden Aufgabe.

\begin{aufgabe}
\label{filter:aufgabe2}
Seien $X_i$ mit $i=1,2$ unabhängige Zufallsvariablen mit $E(X_i)=\mu$
und $\operatorname{var}(X_i)=\sigma_i^2$.
Wie muss man $t$ wählen, damit die Zufallsvariable $X=tX_1 +(1-t)X_2$
minimale Varianz hat?
\end{aufgabe}

Die Lösung der Aufgaben~\ref{filter:aufgabe2} liefert also ein Verfahren,
aus zwei verschiedenen Messungen das Resultat mit kleinstmöglichem
Fehler zu ermitteln.

Zur Lösung des Problems berechnen wir die Varianz von $X$ und 
ermitteln dann denjeningen Wert von $t$, der die Varianz minimiert.
Die Varianz von $X$ ist
\begin{align*}
\operatorname{var}(X)
&=
\operatorname{var}(tX_1+(1-t)X_2)
\\
&=
t^2\operatorname{var}(X_1) + (1-t)^2\operatorname{var}(X_2).
\\
&=t^2\sigma_1^2 + (1-t)^2\sigma_2^2
\intertext{Die Ableitung nach $t$ ist}
\frac{d}{dt}\operatorname{var}(X)
&=
2t\sigma_1^2 - 2(1-t)\sigma_2^2
\\
\Rightarrow
\qquad
0&=t\sigma_1^2 - (1-t)\sigma_2^2
=t(\sigma_1^2 + \sigma_2^2) -\sigma_2^2.
\end{align*}
Daraus lässt sich $t$ bestimmen als
\begin{equation}
t=\frac{\sigma_2^2}{\sigma_1^2+\sigma_2^2}
\qquad\text{und}\qquad
1-t=\frac{\sigma_1^2}{\sigma_1^2+\sigma_2^2}.
\end{equation}
Mit diesem $t$ lässt sich jetzt auch die minimal mögliche Varianz
bestimmen:
\begin{align*}
\sigma^2=\operatorname{var}(X)
&=
\biggl(\frac{\sigma_2^2}{\sigma_1^2+\sigma_2^2}\biggr)^2\sigma_1^2
+
\biggl(\frac{\sigma_1^2}{\sigma_1^2+\sigma_2^2}\biggr)^2\sigma_2^2
\\
&=
\frac{\sigma_2^4\sigma_1^2}{(\sigma_1^2+\sigma_2^2)^2}
+
\frac{\sigma_1^4\sigma_2^2}{(\sigma_1^2+\sigma_2^2)^2}
\\
&=
\frac{(\sigma_1^2+\sigma_2^2)\sigma_1^2\sigma_2^2}{(\sigma_1^2+\sigma_2^2)^2}
=
\frac{\sigma_1^2\sigma_2^2}{\sigma_1^2+\sigma_2^2}
\\
\text{oder}\qquad
\frac{1}{\sigma^2}
&=
\frac{1}{\sigma_1^2}+\frac1{\sigma_2^2}.
\end{align*}
Wir fassen die Resultate im folgenden Satz zusammen.

\begin{satz}
Sind $X_1$ und $X_2$ unabhängige Zufallsvariablen mit dem Erwartungswert
$E(X_i)=\mu$ und Varianz $\operatorname{var}(X_i)=\sigma_i^2$,
dann ist $X=tX_1+(1-t)X_2$ eine Schätzung für $\mu$ mit minimaler Varianz
genau dann, wenn
\[
t=\frac{\sigma_2^2}{\sigma_1^2+\sigma_2^2}.
\]
Der Schätzfehler ist $\operatorname{var}(X)
=\sigma_1^2\sigma_2^2/(\sigma_1^2+\sigma_2^2)$.
\end{satz}

Man kann das Resultat auch noch etwas verallgemeinern auf mehrere
Messungen.

\begin{satz}
Sind $X_i$ unabhängige Zufallsvariablen mit Erwartungswert $E(X_i)=\mu$
und Varianz $\operatorname{var}(X_i)=\sigma_i^2$, dann hat der 
gewichtete Mittelwert
\[
X=t_1X_1+\dots+t_nX_n = \sum_{i=1}^n t_iX_n
\]
genau dann minimale Varianz, wenn man
\[
t_k= \frac{1}{\sigma_k^2} \biggl(\sum_{i=1} \frac{1}{\sigma_i^2}\biggr)^{-1}
\]
wählt und die minimale Varianz $\sigma^2=\operatorname{var}(X)$ erfüllt
die Gleichung
\[
\frac{1}{\sigma^2}
=
\sum_{i=1}^n \frac{1}{\sigma_i^2}.
\]
\end{satz}

\begin{proof}
Der gewichtete Mittelwert ist nur dann ein erwartungstreuer Schätzer, wenn
\begin{equation}
E\biggl(\sum_{i=1}^n t_iX_i\biggr)
=
\sum_{i=1}^n t_iE(X_i)=\mu\sum_{i=1}^nt_i = \mu
\qquad\Rightarrow\qquad
\sum_{i=1}^n t_i = 1
\label{filter:optimal:nebenbedingung}
\end{equation}
ist.
Wir haben daher das Minimum von $\operatorname{var}(X)$ zu finden unter
der Nebenbedingung~\eqref{filter:optimal:nebenbedingung}.

Die Varianz ist
\begin{align}
\operatorname{var}(X)
&=
\operatorname{var}\biggl(
\sum_{i=1}^n t_iX_i
\biggr)
=
\sum_{i=1}^n t_i^2 \operatorname{var}(X_i).
\label{filter:varianz:n}
\end{align}
Das Minimum unter der Nebenbedingung~\eqref{filter:optimal:nebenbedingung}.
kann mit Hilfe eines Lagrange-Multiplikators $\lambda$ gefunden werden.
Die Gradienten von $\operatorname{var}(X)$ und der Nebenbedingung müssen
proportional sein, also
\begin{equation}
\left.
\begin{aligned}
\frac{\partial}{\partial t_k} \operatorname{var}(X)
&= 2t_k \sigma_k^2
\\
\frac{\partial}{\partial t_k}\sum_{i=1}^nt_i&=1
\end{aligned}
\quad
\right\}
\qquad
\Rightarrow
\qquad
2t_k \sigma_k^2 =\lambda.
\quad
\Rightarrow
\quad
t_k = \frac{\lambda}{2\sigma_k^2}
\end{equation}
Da die Summe der $t_i$ den Wert $1$ ergeben muss, folgt für $\lambda$
\[
1
=
\sum_{k=1}^n t_k
=
\sum_{k=1}^n \frac{\lambda}{2\sigma_k^2}
=
\frac{\lambda}{2}\sum_{k=1}^{n} \frac{1}{\sigma_k^2}
\qquad\Rightarrow\qquad
\lambda = 2\biggl(
\sum_{k=1}^{n} \frac{1}{\sigma_k^2}
\biggr)^{-1}.
\]
Daraus kann man jetzt die Formel für die $t_i$ ableiten:
\[
t_k
=
\frac{1}{\sigma_k^2} \;\biggl(\sum_{i=1}^n \frac{1}{\sigma_i^2}\biggr)^{-1}.
\]
Durch Einsetzen in \eqref{filter:varianz:n} kann man auch den Wert für
die Varianz finden, nämlich
\begin{align*}
\sigma^2
=
\operatorname{var}(X)
&=
\sum_{i=1}^n \frac{1}{\sigma_i^2}
\cdot
\biggl(\sum_{i=1}^n \frac{1}{\sigma_i^2}\biggr)^{-2}
=
\biggl(\sum_{i=1}^n \frac{1}{\sigma_i^2}\biggr)^{-1}
\\
\text{oder}\qquad
\frac{1}{\sigma^2}
&=
\sum_{i=1}^n \frac{1}{\sigma_i^2}.
\qedhere
\end{align*}
\end{proof}
