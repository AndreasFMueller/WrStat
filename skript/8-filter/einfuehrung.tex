%
% einfuehrung.tex -- Einfuehrung Kalman-Filter
%
% (c) 2006-2015 Prof. Dr. Andreas Mueller, Hochschule Rapperswil
%

\section{Dynamische Filterung: Ein Einführungsbeispiel
\label{filter:section:einfuerung}}
Als Motivation für die nachstehend zu entwickelnde Filtertheorie betrachten
wir ein System mit einer einzigen Zustandsvariablen $x$, die sich in diskreten
Zeitschritten entwickelt.
Wir haben also mit eine Folge von Zuständen $x_k$ zu tun, wobei
bis auf eine zufällige Störung $x_{k+1}$ mit $x_k$ übereinstimmt,
also $x_{k+1}=x_k+u_k$.
Die Zufallsvariable beschreibt den Systemfehler, wir
nehmen an, dass $E(u_k)=0$ und $\operatorname{var}(u_k)=\sigma^2$.

Den Zustand $x_k$ versuchen wir nun zu schätzen, für jeden Zeitpunkt
$k$ gibt es also eine Zufallsvariable $\hat x_k$, die ein Schätzer für
$x_k$ sein soll.
Da wir davon ausgehen, dass sich $x$ nicht ändert, können wir $\hat x_k$
als eine nur auf den zum Zeitpunkt $k$ bekannten Informationen basierende
Schätzung $\hat x_{k+1|k}=\hat x_k$ verwenden.
Der Erwartungswert von $\hat x_{k+1|k}$ ist $E(x_{k+1|k})=x_{k+1}$, die Varianz
ist
\[
\operatorname{var}(\hat x_{k+1|k})
=\operatorname{var}(\hat x_k)+\operatorname{var}(u_k)=p_k^2+\sigma^2,
\]
wobei wir $\operatorname{var}(\hat x_k)=p_k^2$ abkürzen.

Nun wird zu jedem Zeitpunkt auch eine Messung $z_{k+1}$ mit einem Messfehler
$\operatorname{var}(z_{k+1})=\rho^2$ durchführt.
Auch hier haben wir als Erwartungswert $E(z_{k+1})=x_{k+1|k}$,
wir haben jetzt also zwei Zufallsvariablen
$\hat x_{k+1|k}$ und $z_{k+1}$, die beide Information über den tatsächlichen
Zustand $x_{k+1}$ liefern, allerdings mit unterschiedlichen Varianzen.
Wir
versuchen daher, die Schätzung $\hat x_{k+1}$ als gewichtetes Mittel 
von $\hat x_{k+1|k}$ und $z_{k+1}$ zu bilden
\begin{equation}
\hat x_{k+1}=(1-K)\hat x_{k+1|k}+Kz_{k+1}=\hat x_{k+1|k}+K( z_{k+1} - \hat x_{k+1|k}).
\label{1dimentwicklung}
\end{equation}
Der Gewichtsfaktor $K$ soll so gewählt werden, dass die Varianz von $\hat x_{k+1}$
minimal wird.
Dazu berechnen wir zunächst die Varianz
\begin{align}
\operatorname{var}(\hat x_{k+1})
&=
(1-K)^2\operatorname{var}(x_{k+1|k})+K^2\operatorname{var}(z_{k+1})\notag\\
&=(1-K)^2(p_k^2+\sigma^2)+K^2\rho^2.\label{varianzentwicklung}
\end{align}
Das Minimum kann durch Ableiten nach $K$ und Nullsetzen gefunden werden:
\begin{align*}
\frac{d}{dK}\operatorname{var}(\hat x_{k+1})
=0&=
-2(1-K)(p_k^2+\sigma^2)+2K\rho^2\\
p_k^2+\sigma^2&=K(p_k^2+\sigma^2+\rho^2)\\
K&=\frac{p_k^2+\sigma^2}{p_k^2+\sigma^2+\rho^2}
\end{align*}
Damit wird die Schätzung $\hat x_{k+1}$
\[
\hat x_{k+1}=\frac{\rho^2}{p_k^2+\sigma^2+\rho^2}\hat x_k+\frac{p_k^2+\sigma^2}{p_k^2+\sigma^2+\rho^2}z_{k+1}
\]
Ebenso kann man jetzt die Varianz $\operatorname{var}(\hat x_{k+1})=p_{k+1}^2$
mit (\ref{varianzentwicklung}) berechnen.
Insgesamt haben wir also folgenden
Satz gewonnen

\begin{satz}
Sei $x_k$ eine Folge von Zufallsvariablen, für die $x_{k+1}=x_k+u_k$ gilt, wobei
$u_k$ eine Zufallsvariable mit Erwartungswert $0$ und Varianz $\sigma^2$ ist.
Sei
ausserdem $z_{k+1}$ eine Folge von Messungen von $x_{k+1}$, d.~h.~$E(x_{k+1})=E(z_{k+1})$
und $\operatorname{var}(z_{k+1})=\rho^2$.
Weiter sei der Anfangszustand $x_0=x$ bekannt.
Die Folgen $\hat x_k$ und $p_k^2$, definiert durch die Anfangswerte
$\hat x_0=x$ und $p_0^2=0$ und die Iterationsgleichungen
\begin{align*}
K&=\frac{p_k^2+\sigma^2}{p_k^2+\sigma^2+\rho^2}\\
\hat x_{k+1}&=\hat x_k + K ( z_{k+1} - \hat x_k)\\
p_{k+1}^2&=(1-K)^2(p_k^2+\sigma^2)+K^2\rho^2,
\end{align*}
ergeben die bestmögliche Schätzung $\hat x_k$ von $x_k$ mit
einem mittleren quadratischen Fehler $p_k^2$.
\end{satz}
Der Gewichtsfaktor $K$ heisst Kalman Filter Faktor.
Ist $K$ gross, wird der
Schätzwert $\hat x_{k+1}$ vor allem durch die Messung bestimmt, ist er
klein, hat die bisherige Schätzung $\hat x_k$ mehr Gewicht.
$K$ wird gross,
also nahe bei $1$, wenn, $\rho^2$ sehr klein ist, wenn also die Messung
sehr genau ist.
In diesem Fall wird auch $p_{k+1}^2$ vor allem durch die
die Grösse $K^2\rho^2$ bestimmt, die Schätzung $\hat x_k$  ist also
ähnlich präzise Messung.
Ist die Messgenauigkeit dagegen gering,
wird die Iteration von einem einmal gefundenen Schätzwert $\hat x_k$
nur in kleinen Schritten abrücken.
Dabei besteht die Gefahr, dass
der Fehler $p_k^2$ stark anwächst, wen nämlich $\sigma^2$ zu gross
ist, kann es geschehen, dass $p_{k+1}^2>p_{k}^2$, so dass der Filter
mit der Zeit immer ungenauer wird.
Daher ist es wichtig, einerseits
eine möglichst gut zutreffende Systembeschreibung zu haben und andererseits
den Systemzustand möglichst genau zu messen.

