%
% wiener.tex
%
% (c) 2020 Prof Dr Andreas Müller, Hochschule Rapperswil
%
\section{Wiener-Filter
\label{section:wiener-filter}}
Ein ähnliches Problem findet man in der Übertragungstechnik.
Wir gehen davon aus, dass ein Übertragungskanal ein Signal abschwächt
und mit einem Rauschsignal überlagert.
Die Abschwächung können wir selbstverständlich wieder rückgängig
machen, dabei wird aber auch das Rauschen verstärkt, so dass das
verstärkte Signal wenig mit dem ursprünglichen Signal zu tun hat.
Je stärker das Rauschen ist, desto vorsichtiger müssen wir mit
der Verstärkung sein.
Es dürfte also einen optimalen Verstärkungsfaktor geben, der 
das ursprüngliche Signal zwar nicht vollständig wiederherstellt, 
aber doch auf eine Weise, dass der Fehler möglichst klein ist.

Eine mathematischere Formulierung dieser Aufgabe ist die folgende.

\begin{aufgabe}
Sei $X$ eine Zufallsvariable mit Varianz $\operatorname{X}=\sigma$,
die von einem Übertragungskanal um den Faktor $a$ gedämpft wird.
Zusätzlich wird ein Rauschsignal $N$ mit Varianz $\sigma_N^2$ überlagert,
$X$ und $N$ sind unabhängig.
Finde den Faktor $b$, mit dem das gestörte Signal $Y=aX+N$ verstärkt werden
muss, damit sich ein Signal ergibt, dessen Abweichung vom ursprünglichen Signal
minimale Varianz hat.
\end{aufgabe}

Die Abweichung des verstärkten Signals $bY$ von $X$ hat die Varianz
\begin{align*}
\operatorname{var}(bY-X)
&=
\operatorname{var}(b(aX+N)-X)
=
\operatorname{var}((ab-1)X+bN)
=
(ab-1)^2 \operatorname{var}(X) + b^2\operatorname{var}(N).
\end{align*}
Um das Minimum zu bestimmen, leiten wir nach $b$ ab
\begin{align*}
0
&=
\frac{\partial}{\partial b} \operatorname{var}(bY-X)
=
2(ab-1)a \operatorname{var}(X) +2b\operatorname{var}(N)
\\
0&=
b(a^2\sigma^2 + \sigma_N^2) - a\sigma^2
\qquad\Rightarrow\qquad
b
=
\frac{a\sigma^2}{a^2\sigma^2+\sigma_N^2}
=
\frac{a}{a^2+\sigma_N^2/\sigma^2}
=
\frac{1}{a+\sigma_N^2/a\sigma^2}.
\end{align*}
Diese Formel lässt sich auch leicht plausiblisieren.
Falls kein Rauschen vorhanden ist, also $\sigma_N=0$, dann kann
man für $b$ den Faktor $1/a$ verwenden, der die Abschwächung
vollständig rückgängig macht.
Falls das Rauschen im Vergleich zum Signal sehr gross ist, wird
$b\simeq 0$, es ist also gar nicht möglich, das Signal sinnvoll
wiederherzustellen.

Wir fassen dieses Resultat im folgenden Satz zusammen.

\begin{satz}
\label{filter:wiener:primitiv}
\end{satz}





