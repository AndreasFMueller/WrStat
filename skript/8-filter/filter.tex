%
% filter.tex -- Kalman-Filter
%
% (c) 2006-2015 Prof. Dr. Andreas Mueller, Hochschule Rapperswil
%
\chapter{Filtern}
\label{chapter:filtern}
\kopflinks{Filtern}

In den vergangenen Kapiteln wurden unter den Voraussetzungen einer
bekannten Verteilung einer einzigen Zufallsvariablen Methoden entwickelt,
die Parameter der Verteilung zu schätzen, und die daraus gewonnenen
Hypothesen gegebenenfalls zu testen.
Überlicherweise musste man dazu
mehrere Beobachtungen der gleichen Zufallsvariablen vornehmen.
In der Praxis kann man jedoch
nicht davon ausgehen, dass man ein System mehrmals beobachten kann.
Im Gegenteil wird sich das System zwischen den Messungen weiterentwickeln,
so dass man jeden Zustand nur ein einziges Mal messen kann.
Die bisherige
Theorie ist auf dieses Problem als zunächst nicht anwendbar.
Da aber in vielen Fällen die zeitliche Entwicklung zum Beispiel durch
Naturgesetze vorgegeben ist, sollte man die bereits erfolgten Messungen
früherer Zustände mit der Systementwicklung verrechnen können und
daraus eine bessere Schätzung des Zustandes ableiten können.
Genau
dies leistet ein Filter: er ermittelt aus neu eintreffenden Messungen
des Systemzustands laufend neue Schätzungen des aktuellen Systemzustandes.
Ziel dieses Kapitels ist zu zeigen, wie man für lineare Systeme\footnote{Diese
Einschränkung ist unbedeutend, da sich praktisch alle Probleme linearisieren lassen.}
einen optimalen Filter berechnen kann.

\input{8-filter/1-zweiquellen.tex}
\input{8-filter/2-wiener.tex}
\input{8-filter/3-einfuehrung.tex}
%
% problemstellung.tex -- Problemstellung Kalman-Filter
%
% (c) 2006-2015 Prof. Dr. Andreas Mueller, Hochschule Rapperswil
%

\section{Problemstellung}
\kopfrechts{Problemstellung}
Im Einführungsbeispiel wurde ein eindimensionales System betrachtet,
welches sich nicht verändert.
Im Allgemeinen wird ein System durch
mehrere Variablen beschrieben, die sich in der Zeit verändern.
in diesem Abschnitt wollen wir die Problemstellung des eindimensionalen
Problems auf mehrdimensionale Systeme verallgemeinern.

\subsection{System und Beobachtung}
Zwei Positionsmessung im zeitlichen Abstand $\Delta t$
bei einem Fahrzeug, welches selbständig
in seiner Fahrbahn bleiben soll, können sich bis auf Messfehler um höchstens
$v\Delta t$ unterscheiden, wenn sich das Fahrzeug mit Geschwindigkeit $V(t)$
quer zur Fahrtrichtung bewegt.
Insbesondere würden wir für die zweite
Messung ungefähr den Wert $E(X(t+\Delta t)) = X(t) + V(t)\Delta t$
erwarten.
Hier haben wir unser Wissen über das System einfliessen lassen.

Offensichtlich ist es also möglich, eine gute Voraussage über den
zukünftigen Zustand $X(t+\Delta t)$ zu machen, wenn $X(t)$ und $V(t)$
bekannt sind.
Natürlich wird diese Voraussage nicht mit der tatsächlichen
Messung zur Zeit $t+\Delta t$ übereinstimmen.
Zum einen werden unweigerlich
neue Messfehler auftreten, zum anderen basiert die Voraussage ja bereits
auf Daten, die mit einem Messfehler behaftet sind.
Die Voraussage ist also
ebenfalls eine Zufallsvariable, die jedoch auf der Kenntnis eines früheren
Zustands des Systems basiert.

Wir modellieren dieses Problem jetzt für den Fall konstanter Zeitschritte.
Den Zustand des Systems betrachten wir als Vektor $x_k\in\mathbb{R}^n$,
wobei $k\in\mathbb{N}$.
Diese Vektoren sind nicht vollständig beobachtbar,
nur ein Abbild davon ist messbar.
Grundsätzlich müssten wir also eine
bestimmte Funktion auf $x_k$ anwenden, um die Messgrössen zu finden,
für die meisten praktischen Anwendungen genügt es, eine lineare Abbildung
zu verwenden.
Es gibt also eine Matrix $H_k$, welche die beobachteten
Grössen $z_k=H_kx_k$ bestimmt.

Die Zeitentwicklung des Systems muss aus einem Zustand $x_k$ den nächsten
Zustand $x_{k+1}$ berechnen.
Bei einem
einfachen System mit der Bewegungsgleichung
\[
\frac{dx(t)}{dt}=ax(t)
\]
gilt zum Beispiel $x(t_2)=e^{a(t_2-t_1)}x(t_1)$, d.~h.~der Zeitschritt
kann durch Multiplikation mit einer Zahl $e^{a\Delta t}$ berechnet werden.
ersetzt man $x$ durch einen Vektor $\vec x$ und $a$ durch eine Matrix
$A$, wird die Lösung zu $\vec x(t_2)=e^{A(t_2-t_1)}\vec x(t_1)$,
insbesondere gibt es also eine Matrix $\varphi=e^{A\Delta t}$, mit
der der aktuelle Zustand multipliziert werden muss, um den zukünftigen
Zustand zu ermitteln.
Bei genügend kleinem Zeitschritt kann man jedes
System soweit linearisieren, dass diese Annahme zutrifft, die Matrix
wird aber im Allgemeinen zeitabhängig sein.
Die Bewegungsgleichung
wird damit in linearisierter Form
\[
x_{k+1}=\varphi_kx_k.
\]

\subsection{Messfehler und Systemfehler}
\begin{figure}
\centering
\includegraphics{images/filter-1.pdf}
\caption{Systementwicklung durch die Abbildung $\varphi_k$, der Vektor $u_k$
gibt die Unzulänglichkeiten des durch $\varphi_k$ beschriebenen Modells
wieder.
\label{filter-systementwicklung}}
\end{figure}
In der Realität bewegt sich das System kaum genau nach diesen Gleichungen.
Vielmehr wird es von äusseren Störungen beeinflusst, die sich einer
exakten Vorhersage entziehen, wir tragen dem Rechnung mit Hilfe eines
zusätzlichen Zufallsvektors $u_k$ in den Bewegungsgleichungen
\[
x_{k+1}=\varphi_kx_k+ u_k.
\]
Wir verlangen, dass die Komponenten des Vektors $u_k$ alle unabhängige
Zufallsvariablen sind Erwartungswert 0.
\index{Systemfehler-Kovarianz-Matrix}
\index{Systemfehler!Kovarianz-Matrix}
Die Kovarianz-Matrix $Q_k=E(u_k\cdot u_k^t)$ ist
ein Mass dafür.
$u_k$ heisst auch Prozessrauschen oder Systemfehler.

Die Messfehler modellieren wir durch einen
Zufallsvektor $w_k$, der bei jeder Messung dazukommt, also
\[
z_k=H_kx_k+w_k.
\]
Die Komponenten von $w_k$ sollen unabhängige Zufallsvariablen mit
Erwartungswert $0$ sein, d.~h.~es liegen keine systematischen Messfehler
vor.
Dies bedeutet, dass die Matrix
\index{Messfehler-Kovarianz-Matrix}
\index{Messfehler!Kovarianz-Matrix}
\[
R_k=E(w_kw_k^t)
\]
Diagonalform hat.
Ausserdem sollen die Messfehler unabhängig vom aktuellen Zustand
sein, in Matrix-Form können wir dies durch $E(x_kw_k^t)=0$ ausdrücken.
Man nennt $w_k$ oft auch das Messrauschen.

Die Messfehler $w_k$ und das Prozessrauschen $u_k$ sollen ebenfalls unabhängig
sein, also $E(w_ku_k^t)=0$.

Das Filterproblem besteht darin, aus den bekannten Messwerten von $z_0,\dots,z_k$
die bestmögliche Schätzung $\hat x_k$ zu ermitteln.
Dabei sollen
natürlich die Bewegungsgleichungen berücksichtigt werden.
Da aber
mit der Matrix $R_k$ auch die Grössenordnung der Messfehler bekannt ist,
sollen ``schlechte'' Messwerte geringeres Gewicht erhalten also gute
Messwerte.
Je mehr Werte $z_k$ bekannt werden, desto besser sollte die
Schätzung werden.

In einer praktischen Realisierung muss aber auch die Berechnung der Schätzung
einfach und mit wenig Speicherplatz möglich sein.
Insbesondere sollte zur
Berechnung der neuen Schätzung $\hat x_{k+1}$ nur der letzte bekannte Schätzwert
$\hat x_k$ und die Messung $z_{k+1}$ notwendig sein.
Ausserdem soll der Schätzer
$\hat x_k$ möglichst gut sein, d.~h.~erwartungstreu, linear, iterativ und
mit minimaler Varianz.

\begin{definition}Gegeben ist ein lineares System mit der Zeitentwicklung
\[
x_{k+1}=\varphi_kx_k+u_k,
\]
wobei $x_k$ und $u_k$ unabhängigen Zufallsvariablen sind,
$u_k$ mit Erwartungswert $0$ und bekannter Kovarianzmatrix $Q_k$, d.~h.
\[
E(u_kx_k^t)=0,\qquad E(u_k)=0,\qquad E(u_ku_k^t)=Q_k.
\]
An diesem System werden Beobachtungen
\[
z_k=H_kx_k+w_k
\]
vorgenommen, $w_k$ sind mit $x_k$ und $u_k$ unabhängige Zufallsvariablen
$w_k$ mit Erwartungswert $0$ und bekannter Kovarianzmatrix $R_k$, also
\[
E(w_kx^t)=0,\qquad E(w_k)=0,\qquad E(w_kw_k^t)=R_k.
\]
Das Filterproblem besteht darin, einen Schätzer $\hat x_k$ zu
finden, welcher sich linear aus $\hat x_{k-1}$ und $z_k$ berechnen lässt.
Die Fehler $\tilde x_k=\hat x_k-x_k$ können mit der Fehlerkovarianzmatrix
\[
P_k=E(\tilde x_k\tilde x_k^t)
\]
gemessen werden.
Der Schätzer soll so gewählt werden, dass die Spur
$\operatorname{tr}P_k$ minimal wird.
\end{definition}


\input{8-filter/5-loesung.tex}
%
% positionsmessung.tex -- 
%
% (c) 2006-2015 Prof. Dr. Andreas Mueller, Hochschule Rapperswil
%

\section{Beispiel: Positionsmessung und Geschwindigkeit}
\kopfrechts{Positionsmessung}
In diesem Abschnitt soll gezeigt werden, wie man mit Hilfe der
eben dargestellten Theorie einen Kalman-Filter entwickeln kann.
Er soll folgende Aufgabe lösen:
\begin{quote}
Von einem System wird zehn mal
pro Sekunde die Position mit einer Genauigkeit von $\sigma_z=0.7\text{m}$
gemessen.
Man ermittle aus diesen Messdaten möglichst genau die aktuelle Position
und Geschwindigkeit.
\end{quote}
\subsection{Systembeschreibung}
Wir modellieren das System unter der Annahme, dass die Geschwindigkeit in
ausreichender Näherung konstant ist.
Dazu brauchen wir zwei Zustandsvariablen, den Ort und die Geschwindigkeit.
Die Zustandsvektoren $x_k$ sind also zweidimensionale Vektoren.

Für die Zeitentwicklung verwenden wir die Annahme: über einen Zeitschritt
der Länge $\Delta t$
ändert sich die Geschwindigkeit nicht, die Position wird jedoch 
um das $\Delta t$-fache der Geschwindigkeit geändert.
Die Entwicklungsmatrix $\varphi$ ist daher
\[
\varphi=\begin{pmatrix}
1&\Delta t\\
0&1
\end{pmatrix}.
\]

Die Systemfehler geben wieder, in welchem Mass das System von den Annahmen
abweicht.
Die zentrale Annahme ist, dass die Geschwindigkeit konstant ist, dass es
also keine Beschleunigung gibt.
Da die meisten Systeme auch irgend wann wieder zur Ruhe kommen werden,
dürfen wir annehmen, dass der Erwartungswert der Beschleunigung $0$ ist.
Die Varianz der Beschleunigung wird aber nicht 0 sein, kann uns aber als
Mass für die Systemfehler dienen.
Ist $\Delta a$ die Standardabweichung der Beschleunigung, dann erzeugt
diese einen Geschwindigkeitsfehler in der Grössenordnung von
$\Delta a\cdot \Delta t$, und einen Positionsfehler von der Grössenordnung
$\frac12\Delta a\cdot\Delta t^2$.
Als Systemfehlermatrix können wir daher 
\[
Q=
\begin{pmatrix}
\sigma_x^2&0\\
0&\sigma_v^2
\end{pmatrix}
=
\begin{pmatrix}
(\frac12\Delta a\cdot\Delta t^2)^2&0\\
0&(\Delta a\cdot\Delta t)^2
\end{pmatrix}
\]
verwenden.

\subsection{Messprozess}
Von den Zustandsvariablen des Systems wird nur die Position gemessen,
die Messmatrix ermittelt also die erste Variable des Zustandsvektors:
\[
H=\begin{pmatrix}1&0\end{pmatrix}.
\]
Der Messfehler der Position ist aus der Aufgabenstellung bekannt, die
Messfehlermatrix ist die $1\times 1$-Matrix
\[
R=\begin{pmatrix}
\sigma_z^2
\end{pmatrix}.
\]
Damit ist das System vollständig beschrieben.

\subsection{Filter}
Die Matrizen der Systembeschreibung sind in diesem Problem unabhängig
von der Zeit, wir können daher in den Formeln die Indizes weglassen.

\subsubsection{Initialisierung}
Wir stellen uns vor, dass das System sich zu Beginn der Messung an
bekannter Position, die wir der Einfachheit halber mit $0$ bezeichnen,
aus der Ruhe losgelassen wird.
Zur Zeit $t=0$ ist also der Zustandsvektor bekannt, es ist 
\[
x_0=\begin{pmatrix}0\\0\end{pmatrix}
\]
der Nullvektor.
Da Position und Geschwindigkeit exakt bekannt sind, ist die initiale
Fehlerkovarianzmatrix
\[
P_0=\begin{pmatrix}0&0\\0&0\end{pmatrix}.
\]
Man nennt diese Art der Initialisierung des Filters auch
Fisher-Initialisierung, sie ist in diesem Fall angmessen.

\subsubsection{Vorhersage}
Der Kalmanfilter ermittelt die geschätzte Position und Geschwindigkeit aus
einer Abfolge von Vorhersage- und Korrektur-Schritten.
Für die Vorhersage gelten die Gleichungen
\begin{align*}
\hat x_{k+1|k}&=\varphi \hat x_k\\
P_{k+1|k}&=\varphi P_k\varphi^t + Q.
\end{align*}

\subsubsection{Korrektur}
Der Korrekturschritt bestimmt zuerst die Kalman-Matrix $K_{k+1}$ und
dann daraus die endgültigen Schätzungen für den Zustand
$\hat x_{k+1}$ und die Fehlerkovarianz $P_{k+1}$:
\begin{align*}
K_{k+1}&=P_{k+1|k}H^t(HP_{k+1|k}H^t+R)^{-1}
\\
\hat x_{k+1}&= (I-K_{k+1}H)\hat x_{k+1|k} + K_{k+1}z_{k+1}
\\
P_{k+1}&=(I-K_{k+1}H)P_{k+1|k}
\end{align*}

\subsection{Matlab/Octave Code}
Der Kalman-Filter kann mit wenigen Zeilen Matlab/Octave-Code implementiert
werden.
Die Parameter des Systems sind
\begin{verbatim}
deltat = 0.1;
a = 0.4;
sigmav = a * deltat;
sigmax = 0.5 * sigmav * deltat;
\end{verbatim}
Die Systemmatrizen sind
\begin{verbatim}
phi = [ 1, deltat; 0, 1 ];
Q = [ sigmax^2, 0; 0, sigmav^2 ];
\end{verbatim}
Der Messprozess wird beschrieben durch die Matrizen
\begin{verbatim}
H = [ 1, 0 ];
global sigmaz = 0.7;
R = [ sigmaz^2 ];
\end{verbatim}
Die Konstante \verb+sigmaz+ ist global definiert, weil sie in der
Funktion \verb+messung+ benutzt wird, um Messwerte mit normalverteilten
Messfehlern zu simulieren.

Der Filter wird initialisiert mit
\begin{verbatim}
x = [ 0; 0 ];
P = zeros(2, 2);
t = 0;
tlimit = 30;
\end{verbatim}
Und der eigentliche Filtercode ist
\begin{verbatim}
I = eye(2);
t = 0;
while (t < tlimit)
        # Vorhersage
        xpred = phi * x;
        Ppred = phi * P * phi' + Q;

        # Korrektur
        K = Ppred * H' * pinv(H * Ppred * H' + R);
        z = messung(t);
        x = (I - K * H) * xpred + K * z;
        P = (I - K * H) * Ppred;

        # Zeit
        t = t + deltat;
endwhile
\end{verbatim}
Die Funktion \verb+messung+ muss einen für die Zeit $t$ angemessenen
Messwert liefern.
Die Objekte mit dem Suffix \verb+pred+ bezeichnen Resultate des
Vorhersageschrittes.
Die Funktion \verb+pinv+ berechnet die Pseudoinverse.
Da im vorliegenden Fall das Argument nur eine $1\times 1$-Matrix ist,
könnte man auch die gewöhnliche inverse Matrix nehmen oder direkt
durch das einzige Matrixelement dividieren.

\subsection{Simulationsresultate}
\begin{figure}
\centering
\includegraphics{tacho/graph-1.pdf}
\caption{Kalman-Filter zur Ortsmessung (oben) und Geschwindigkeitsbestimmung
(unten).
Die reale Position ({\color{blue}blau}) wird mit relativ grossem
Fehler gemessen ({\color{green}grün}), doch der Filter ({\color{red}rot})
kann die Position rekonstruieren, und sogar einigermassen brauchbare
Werte für die Geschwindigkeit ableiten.
Ebenfalls dargestellt sind die Diagonalelemente der Fehlerkovarianz,
der Positionsfehler ist allerdings 3-fach überhöht.
\label{tacho-graph}}
\end{figure}
In Abbildung~\ref{tacho-graph} ist eine Simulation des eben entwickelten
Filters dargestellt.
Das System befindet sich von Zeit $t=0$ bis $t=10$ am Nullpunkt, bewegt
sich dann zwischen $t=10$ und $t=20$ mit Geschwindigkeit $1$ in Richtung
zunehmender Positionskoordinate, danach bleibt es bis $t=30$ an Position $10$.
Ausgehend von dieser Bewegung werden dem Filter als Messungen $z_k$ die
Positionen sowie ein normalverteilter Messfehler mit $\sigma_z=0.7\text{m}$
erzeugt.

Da die Messfehler relativ gross sind, ist der entstehende Kalman-Filter
ziemlich träge, er ``traut'' der Messung nicht, und folgt vor allem der
Systementwicklungsgleichung.
Erst einige Sekunden nach $t=10$ akzeptiert der Filter, dass das System
sich jetzt bewegt, die Geschwindigkeitskoordinate des Zustandsvektors beginnt
zuzunehmen.
Auch dauert es nach $t=20$ wieder ein paar Sekunden, bis der Filter
die Geschwindigkeitskoordinate nach $0$ zurückkehren lässt.

Der Filter ermittelt auch die zu erwartende Fehlerkovarianz, sie ist
in Abbildung~\ref{tacho-graph} als hellrote Fläche hinter der
Kurve der gefilterten Daten dargestellt.
Da die angenommenen Systemfehler der Position sehr klein sind, ist auch die
Fehlerkovarianz in dieser Koordinate sehr gering, in der Darstellung
ist der Fehler daher dreifach überhöht dargestellt.
Beim deutlich grösseren Fehler der Geschwindigkeit war dies nicht
nötig.

Man beachte, wie der Fehler zur Zeit $0$ zunächst verschwindet, und
sich dann innerhalb der ersten paar Sekunden auf einen konstanten
Wert einstellt.

\subsection{Tuning}
Dieser erste Versuch eines Kalman-Filters ist möglicherweise für
die vorgesehene Anwendung ungeeignet.
Es stellt sich daher die Frage, wie die einstellbaren Parameter
$\sigma_x$ und $\sigma_v$ verändert werden können,
um ein beabsichtigtes Filterverhalten zu erreichen.

Man beachte, dass $\sigma_z$ kein Tuning-Parameter ist, da $\sigma_z$
den bekannten Messfehler der Positionsbestimmung beschreibt.
Allerdings wird angenommen, dass der Messfehler normalverteilt ist.
Falls diese Annahme nicht zutrifft, kann man $\sigma_z$ modifizieren,
um möglicherweise von der Normalverteilung abweichendes Verhalten
zu berücksichtigen.

Kleine Werte von $\sigma_x$ und $\sigma_v$ bedeuten, dass der Filter
vor allem den Systemgleichungen folgt, und Änderungen der
Messwerte nur sehr träge folgt.
Grosse Werte von $\sigma_x$ oder $\sigma_v$ lassen den Filter vor allem
den Messwerten folgen. 
Als Nebeneffekt sind auch die Geschwindigkeitswerte sehr viel stärker
verrauscht, und kaum brauchbar.

Wählt man $\sigma_v$ besonders klein, zwingt man den Filter, die
Geschwindigkeit nur sehr langsam zu ändern.
Der Filter ist ja konzipiert worden ausgehend von der Annahme,
dass die Geschwindigkeit konstant bleibt, und $\sigma_v$ drückt
die Abweichung von dieser Annahme aus.
Ein grosser Wert von $\sigma_v$ erlaubt dagegen, dass die Geschwindigkeit
sehr schnell den veränderten Messwerten folgt.

Die Grösse $\sigma_x$ beschreibt die Abweichung der Entwicklung
der Position vom Modell.
Ein grosser Wert bedeutet, dass der Filter einer Positionsänderung
folgen kann, auch wenn es keine dazu passende Geschwindigkeitsänderung
gibt.
Der Filter wird ziemlich gut der Position folgen, aber der
Geschwindigkeitswert wird unbrauchbar träge.


\input{8-filter/7-hoehenmesser.tex}
