%
% learning.tex -- Machine Learning
%
% (c) 2017 Prof Dr Andreas Müller, Hochschule Rapperswil
%
\section{Machine Learning}
\index{Machine Learning}
Die lineare Regression aus dem vorangegangenen Abschnitt kann auch aus
der etwas modischeren Perspektive des {\em Maschine Learning} 
betrachtet werden.
Dabei nimmt man wie vorhin an, dass es eine lineare Beziehung zwischen den
Variablen $X$ und $Y$ in der Form
$Y=aX+b$ gibt.
Wir möchten die Konstanten $a$ und $b$ {\em lernen}, d.~h.~wir möchten
eine grosse Menge von Trainingsdatenpaaren $(x_i, y_i)$ verwenden,
um $a$ und $b$ zu bestimmen.
Die Formeln \eqref{lr:a} und \eqref{lr:b} zur Bestimmung von $a$ und $b$
sind so gewählt, dass die Kostenfunktion
\[
C
=
\sum_{i=1}^n (y_i-ax_i-b)^2
\]
minimiert wird,
die misst, wie gut die das lineare Modell mit Parametern $a$ und $b$
den Zusammenhang zwischen $X$ und $Y$ gelernt hat.



