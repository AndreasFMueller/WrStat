\section{Motivation}
Wenn man Aussagen über die (noch nicht definierte) Wahrscheinlichkeit
machen will, mit welcher ein Ereignis eintreten kann, sollte man sich
zunächst einen Überblick darüber verschaffen, was es denn für
Alternativen gibt.
Wenn man mit einem Würfel eine Eins würfelt, ist
das nicht mehr sehr beeindruckend, wenn man erfährt, dass es ein
Tetraederwürfel mit nur vier Seiten war. 

In den letzten Jahren konnte man immer wieder hören, die Naturkonstanten
in unserem Universum seien unwahrscheinlich genau aufeinander abgestimmt,
genau so dass es möglich sei, dass intelligentes Leben entstand.
Natürlich ist das Unsinn, denn unwahrscheinlich würde ja wohl
heissen, dass es sehr viele Alternativen gibt, in denen eben kein
intelligentes Leben entsteht.
Dummerweise haben wir für so eine Aussage
gar nicht genügend Experimente, wir haben nun mal nur ein einziges
Universum, ganz abgesehen davon, dass wir keine Ahnung darüber haben,
was es genau für Bedingungen braucht, damit intelligentes Leben
entstehen kann.

