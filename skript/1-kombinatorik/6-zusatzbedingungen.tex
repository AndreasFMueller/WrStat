\section{Zusatzbedingungen}
Oft muss man Objekte zählen und dabei noch zusätzliche Einschränkungen
berücksichtigen.
Oft sind zu diesem Zweck Symmetriebetrachtungen
nützlich.

\begin{beispiele}
\item Dominosteine enthalten zwei Felder, die mit je einer Augenzahl
aus $\{0,1,2,3,4,5,6\}$ beschriftet sind.
Wieviele verschiedene Dominosteine
gibt es?

\begin{loesung}
Zwar ist jede Kombination möglich, aber die beiden Dominosteine
mit den Zahlen $(x,y)$ und $(y,x)$ sind identisch.
In den nach der
Produktregel berechneten $7\cdot 7=49$ Paaren sind jeweils die Paare
mit verschiedenen Zahlen doppelt gezählt.
Wir müssen also zählen,
auf wieviele Arten wir zwei verschiedene Zahlen auf die Dominosteine
schreiben können.
Dazu müssen offenbar zwei Augenzahlen aus
der Menge $\{0,1,2,3,4,5,6\}$ gezogen werden, was auf 
$\binom{7}{2}=21$ Arten möglich ist.
Die Anzahl der verschiedenen
Dominosteine ist also $49 - \binom{7}{2}=49-21=28$.

Alternativ hätte man dies auch so zählen können.
In den 49 Paaren sind
einzige die sieben Dominosteine mit zwei gleichen Augenzahlen nicht doppelt
gezählt worden.
Nimmt man die nochmals hinzu, erhält man doppelt so viel
wie es mögliche Dominosteine gibt, also gibt es
\[
\frac{49+7}{2}=\frac{56}2=28
\]
Dominosteine.
\end{loesung}

\item Beim Spiel Eile mit Weile darf man nochmals würfeln, wenn
man eine 6 gewürfelt hat, höchstens aber drei mal.
Wieviele
mögliche Würfelszenarien gibt es?

\begin{loesung}
Es gibt 6 Möglichkeiten für den ersten Wurf.
Bei 5 dieser Möglichkeiten endet das Szenario an dieser Stelle.
In einem der
Fälle wird nochmals gewürfelt, wobei es wieder 6 Möglichkeiten
gibt, mit wieder genau einem Fall, in dem nochmals gewürfelt wird.
Man hat also insgesamt 
\[
5 + 5 + 6=16
\]
mögliche Szenarien.
\end{loesung}

\end{beispiele}

Komplizierte Fälle sind ebenfalls denkbar.
Sie entstehen meist dadurch, 
dass eine die zu zählenden Objekte beliebig vielen zusätzlichen
Bedingungen unterworfen werden können.
In solchen Situationen
führt oft folgende Strategie zum Ziel:
\begin{enumerate}
\item Zeichne ein Mengendiagramm für die Gesamtheit aller Fälle.
\item Zeichne in diesem Diagramm alle Teilmengen, die durch 
die zusätzlichen Bedingungen ausgezeichnet werden.
\item Bestimme die Anzahl Möglichkeiten in jeder Schnitt-, Vereinigungs-
und Differenzmenge.
\end{enumerate}
Genau so sind wir in Beispiel~\ref{nachtessen} von
Abschnitt~\ref{kombinatorik-permutation} vorgegangen,
als wir zunächst die Anzahl der unerwünschten Platzierungen
von Mathematikern und Ingenieuren ermittelt haben, und dann daraus
die die ursprünglich gesucht Anzahl der erwünschten Platzierungen
abgeleitet haben.

