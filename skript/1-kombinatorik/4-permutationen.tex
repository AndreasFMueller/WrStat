\section{Permutationen: Reihenfolge}
\label{kombinatorik-permutation}
\kopfrechts{Permutationen}
\index{Permutation}
\index{Reihenfolge}
Die Frage ``Auf wieviele Arten lassen sich $n$ verschieden Objekte anordnen?''
ist gleichbedeutend mit der Frage, wieviele Permutationen von $n$
Objekten es gibt.
Die Zahl  $P_n$ der Permutationen von $n$ Objekten
kann durch den folgenden Abzählprozess gefunden werden.
Offenbar muss für jedes der Objekte ein Platz gewählt werden, dies will
man Schrittweise tun.
Für das erste Objekt stehen $n$ Plätze
zur Verfügung.
Für jede Wahl des Platzes des ersten Objektes 
ist jetzt zu bestimmen, auf wieviele Arten die verbleibenden
$n-1$ Objekte platziert werden können.
Für das zweite
Objekt muss einer der $n-1$ verbleibenden Plätze gewählt werden.
Bisher haben sind also $n(n-1)$ Möglichkeiten gefunden worden.
Für das dritte Objekt verbleiben jetzt noch $n-2$ Plätze,
was die bisher gefundenen Möglichkeiten auf $n(n-1)(n-2)$
erhöht.
So geht es weiter bis zum letzten Objekt, für welches
genau ein Platz übrig bleibt.
Es folgt
\[
P_n=n(n-1)(n-2)\dotsm 2\cdot 1=n!
\]

Man kann für $P_n$ auch eine Rekursionsformel finden.
Um $n$ Objekte
zu platzieren, muss man zunächst einen Platz für das erste Objekt
platzieren, was auf $n$ Arten möglich ist.
Für jede solche Wahl
muss man dann $n-1$ Objekte platzieren, dafür gibt es $P_{n-1}$
Möglichkeiten.
Also muss gelten
\[
P_n=nP_{n-1}.
\]
Löst man die Rekursion auf, ergibt sich wieder $P_n=n!$.


\begin{beispiele}
\item In wievielen verschiedenen Reihenfolgen können die acht Läufer
eines Rennens im Ziel eintreffen?

\begin{loesung}
Jede Reihenfolge ist möglich, also $8!=40320$ mögliche Reihenfolgen.
\end{loesung}

\item Einer der Läufer von Beispiel 1 ist unbestrittener Favorit,
es ist klar, dass er gewinnen wird.
Von einem anderen ist bekannt, dass er keine Chance hat.
Wieviele Reihenfolgen bleiben übrig?

\begin{loesung}
Nur noch sechs der acht Läufer können in beliebiger Reihenfolge
eintreffen, also in $6!=720$ verschiedene Reihenfolgen.
\end{loesung}

\item An einem Wettkampf treten 7 Männer und 7 Frauen an.
Die Männer
ermitteln untereinander einen Sieger, und die Frauen tun dasselbe.
Am Schluss stellen sich alle für ein Erinnerungsfoto auf: links
die Männer rechts die Frauen, beide in der im Wettkampf ermittelten
Rangfolge.
Wieviele Erinnerungsfotos sind möglich?

\begin{loesung}
Es gibt $7!=5040$ Rangfolgen der Männer, und ebensoviele bei den
Frauen.
Die Ermittelung der Rangfolge bei den Männern ist offensichtlich
unabhängig von derjenigen bei den Frauen, so dass die Gesamtzahl der
möglichen Anordnungen auf dem Erinnerungsfoto $7!\cdot 7!=5040^2=25401600$
ist.
\end{loesung}

\item
\label{nachtessen}
Drei Mathematiker und drei Ingenieure treffen sich zu einem
Nachtessen an einem runden Tisch.
\begin{teilaufgaben}
\item
Auf wieviele Arten können die Teilnehmer die Plätze einnehmen?
\item
Auf wieviel Arten können 
die Teilnehmer die Plätze einnehmen, wenn nicht alle drei Mathematiker
oder Ingenieure nebeneinander sitzen dürfen?
\end{teilaufgaben}

\begin{loesung}
\begin{teilaufgaben}
\item
Man muss offenbar herausfinden, in wievielen möglichen Reihenfolgen
man die sechs Teilnehmer auf die sechs Plätze setzen kann.
Dies sind $6!$ Möglichkeiten.
\item
Es spielt keine Rolle, ob wir untersuchen, wann die Mathematiker 
nebeneinander sitzen und wann die Ingenieure, denn sobald die Mathematiker
nebeneinander sitzen tun es auch die Ingenieure.

Wir wissen bereits, dass die Leute auf $6!=720$ Arten platziert werden
können.
Wir müssen also nur herausfinden, wieviele Platzierungen
unerwünscht sind (weil drei Mathematiker nebeneinander sitzen),
dann wissen wir auch, auf wieviele erwünschte Arten die Teilnehmer
sitzen können.
An einem runden Tisch gibt es sechs Plätze, wo im Uhrzeigersinn gezählt 
die Gruppe der Mathematiker beginnen kann.
Und zu jeder dieser
sechs Möglichkeiten gibt es $3!$ Anordnungen der Mathematiker innerhalb
der Gruppe.
Es gibt also $6 \cdot 3!$ Möglichkeiten, die Mathematiker
in einer Dreiergruppe zu platzieren.
Die Ingenieure sitzen zwar garantiert nebeneinander, können aber
innerhalb ihrer Dreiergruppe auch noch beliebig die Plätze
tauschen, und zwar auf $3!$ verschiedene Arten.
Somit gibt es
\[
6\cdot 3!\cdot 3!=6\cdot 6\cdot 6=6^3=216
\]
unerwünschte Platzierungen.
Folglich gibt es $720-216=504$ zulässige
Platzierungen der Gäste.
\end{teilaufgaben}
\end{loesung}

\end{beispiele}

Die Fakultät kann man in Octave/Matlab mit der Funktion {\tt factorial}
berechnen.
Die letzte Aufgabe kann in Octave zum Beispiel so gerechnet
werden:
\begin{verbatim}
octave> factorial(6) - 6 * factorial(3) * factorial(3)
ans =  504
\end{verbatim}
\index{factorial@{\tt factorial}}

