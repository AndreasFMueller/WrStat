\section{Erzeugende Funktionen}
\kopfrechts{Erzeugende Funktionen}
\index{erzeugende Funktion}
Oftmals lassen sich schwierige kombinatorische Fragestellungen in
algebraische oder analytische Probleme umformulieren.
Damit steht
dann der Apparat der Analysis zur Verfügung.

Als einführendes
Beispiel betrachten wir nochmals die Bestimmung von $C^n_k$.
Statt jeden Wert von $C^n_k$ einzeln zu bestimmen, packen wir
die Koeffizienten zusammen in eine Funktion
\[
p_n(z)=C^n_0+C^n_1z +C^n_2z^2+\dots+C^n_{n-1}z^{n-1}+C^n_nz^n.
\]
Interpretieren wir zunächst $p_1(z)$.
Die Koeffizienten dieses
Polynoms geben an, auf wie viele Arten man kein Element aus einer
einelementigen Menge auswählen kann, bzw.~auf wie viele Arten man 
ein Element auswählen kann.
Beides ist genau auf eine Art möglich,
also
\[
p_1(z)=1+z
\]
Betrachten wir jetzt $p_1(z)^n$.
Beim Ausmultiplizieren des
Produktes 
\[
p_1(z)^n= (1+z)(1+z)\dotsm(1+z),
\]
stellt sich die Frage, wie viele Summanden $z^k$ für jedes
$k$ auftreten werden.
Damit $z^k$ entsteht, muss in $k$ Faktoren das $z$ gewählt werden,
und in den anderen $n-k$ Faktoren die $1$.
Dies ist auf $C^n_k$
Arten möglich, wir haben also gezeigt, dass
\[
p_n(z)=(1+z)^n=\sum_{k=1}^n \binom{n}{k}z^k
\]
ist, woraus
\[
C^n_k=\binom{n}{k}
\]
folgt.

Alternativ kann man zu diesem Resultat auch wie folgt gelangen.
Wir kennen bereits die Rekursionsformel
\[
C^n_k=C^{n-1}_{k-1}+C^{n-1}_k
\]
Übersetzt in die erzeugende Funktion heisst dies, dass die Koeffizienten
für den oberen Index $n-1$
um eine Stelle verschoben und dann addiert werden müssen,
um die Koeffizienten für den oberen Index $n$ zu ergeben.
Verschiebt
man die Koeffizienten von $p_{n-1}(z)$ um $1$,
erhält man die erzeugende Funktion
\[
C^{n-1}_0z
+C^{n-1}_1z^2
+C^{n-1}_2z^3
+\dots
+C^{n-1}_{n-2}z^{n-1}
+C^{n-1}_{n-1}z^n
=
zp_{n-1}(z).
\]
Den Koeffizienten $C^{n-1}_{k-1}$ entspricht also die
erzeugende Funktion $zp_{n-1}(z)$,
den Koeffizienten $C^{n-1}_k$ dagegen die erzeugende Funktion $p_{n-1}(z)$.
Die Rekursionsformel für $C^n_k$ lässt sich also in die Formel
\[
p_n(z)=p_{n-1}(z)+zp_{n-1}(z)=(1+z)p_{n-1}(z)
\]
übersetzen.
Daraus folgt wieder $p_n(z)=(1+z)^n$.

Falls die erzeugende Funktion auf dem Wert $1$ ausgewertet werden kann,
liefert sie die Summe aller Glieder der Folge:
\[
\sum_{k=0}^\infty a_k=p(1).
\]
Für die Binomialkoeffizienten ergibt sich:
\[
\sum_{k=0}^n\binom{n}{k}=p_n(1)=(1+1)^n=2^n.
\]

Die erzeugende Funktion einer unendlichen Folge $a_0,a_1,a_2,\dots$ ist
die Potenzreihe
\[
a_0+a_1z+a_2z^2+a_3z^3+\dots=\sum_{k=0}^\infty a_kz^k.
\]
Die erzeugende Funktion der Folge der Binomialkoeffizienten
$\left(\binom{n}{k}\right)_{0\le k\le  n}$ ist also die Funktion $(1+z)^n$.
Die erzeugende Funktion der konstanten Folge $1,1,1,\dots$ ist
\begin{equation}
1+z+z^2+z^3+\dots=\sum_{k=0}^\infty z^k=\frac1{1-z}.
\label{kombinatorik:geometrischereihe}
\end{equation}
In der Analysis lernt man, dass diese Formel nur für $|z|<1$ gilt.
Nur in diesem Fall ist die Reihe konvergent und hat damit einen
wohldefinierten Wert.
Im vorliegenden Fall betrachten wir die Summe aber gar nicht als Zahlenwert,
sondern nur als algebraischen Ausdruck.
Für den Platzhalter $z$ soll gar kein Zahlenwert substituiert werden.
Vielmehr dienen die Potenzen $z^k$ nur dazu, die einzelnen Zahlen $a_k$
auseinander zu halten.
Man spricht von einer {\em formalen Potenzreihe}.
\index{formale Potenzreihe}
\index{Potenzreihe, formale}

Die Begründung für \eqref{kombinatorik:geometrischereihe} in der 
Analysis verwendet eine Formel für die Partialsummen und berechnet
dann den Grenzwert der Partialsummenfolge.
Dieses Argument funktioniert für formale Potenzreihen natürlich nicht
mehr.
Um \eqref{kombinatorik:geometrischereihe} für eine formale Potenzreihe
nachzuweisen, rechnet man das Produkt
\[
(1-z)\sum_{k=0}^\infty z^k
\]
aus, welches $1$ ergeben sollte.
Man bekommt
\begin{align*}
(1-z)\sum_{k=0}^\infty z^k
=
\sum_{k=0}^\infty z^k - z\sum_{k=0}^\infty z^k
&=
1 + z + z^2 + z^3+\dots
\\
&\phantom{\mathstrut=1}
-z-z^2-z^3-\dots
\end{align*}
Formal heben sich alle Terme ausser dem ersten weg.

\begin{beispiele}
\item 
Auf wie viele Arten kann man 90 Rappen mit den verfügbaren Münzen
herausgeben?

Die verfügbaren Münzen sind 5er, 10er, 20er und 50er.
Statt das Problem für 90 Rappen zu lösen, lösen wir es für jeden
denkbaren Betrag.
Wenn wir dazu nur 5er verwenden dürfen, dann können wir ausschliesslich
die durch fünf teilbaren Rappen-Beträge herausgeben, und zwar immer
auf eindeutige Art.
Die Zahl $a_n$ der Möglichkeiten, mit der der Betrag
$n$ herausgegeben werden kann, ist also
\[
a_n = \begin{cases}
1&\qquad n\equiv 0\imod{5}\\
0&\qquad \text{sonst}.
\end{cases}
\]
Die erzeugende Funktion davon ist
\[
C_5(x)=\sum_{k=0}^\infty a_kx^k=1+x^5+x^{10}+x^{15}+x^{20}+x^{25}+\dots
\]
Könnten wir ausschliesslich 10er verwenden, wäre die entsprechende
erzeugende Funktion
\[
C_{10}(x)=1+x^{10}+x^{20}+x^{30}+x^{40}+x^{50}+\dots
\]
Wenn wir 5er und 10er verwenden dürfen, dann muss ein Teil des Betrages
in 5ern, der Rest in 10er herausgegeben werden.
Diese Zusammensetzung entspricht genau dem, was beim Ausmultiplizieren
der beiden Reihen $C_5(x)$ und $C_{10}(x)$ passiert.
Mit einem Computer-Algebra-System kann man die Multiplikation sofort vornehmen:
\begin{align*}
C_5(x)C_{10}(x)&=
 1+x^5+2 x^{10}+2 x^{15}+3 x^{20}+3 x^{25}+4 x^{30}+4 x^{35}+5 x^{40}\\
&\qquad +5 x^{45} +6 x^{50}+6 x^{55}+7 x^{60}+7 x^{65}+8 x^{70}+8 x^{75}+9 x^{80}\\
&\qquad +9 x^{85}+10 x^{90}+10 x^{95}+11 x^{100}+\dots
\end{align*}
Man kann daraus ablesen, dass es zwei Möglichkeiten gibt, 10 Rappen
herauszugeben, entweder zwei 5er oder ein 10er.
Ebenso kann man ablesen,
dass es 10 Möglichkeiten gibt, 90 Rappen zu bilden, mit 0 bis 9 10ern
und der dazu passenden Anzahl 5er.

Nehmen wir jetzt noch die 20er und 50er hinzu, dann kommen die beiden
zusätzlichen Funktionen
\begin{align*}
C_{20}(x)&= 1+x^{20}+x^{40}+x^{60}+x^{80}+x^{100}+\dots\\
C_{50}(x)&=1+x^{50}+x^{100}+\dots
\end{align*}
hinzu.
Die erzeugende Funktion, mit der wir die Möglichkeiten zählen
können, auf wie viele Arten man den Betrag aus allen Münzen bilden kann,
ist dann
\begin{align*}
C_5(x) C_{10}(x) C_{20}(x) C_{50}(x)
&=
1+x^5+2 x^{10}+2 x^{15}+4 x^{20}+4 x^{25}+6 x^{30}\\
&\qquad
+6 x^{35} +9 x^{40} +9 x^{45}+13 x^{50}+13 x^{55}\\
&\qquad
+18 x^{60}+18 x^{65} +24 x^{70}+24 x^{75}+31 x^{80}\\
&\qquad
+31 x^{85}+39 x^{90}+39 x^{95} +49 x^{100}+\dots
\end{align*}
Daraus können wir jetzt ablesen, dass 90 Rappen auf 39 Arten
gebildet werden können.
\end{beispiele}

Man kann diese Produkte nicht nur mit einem Computer-Algebra-System
ausrechnen, sondern auch mit Octave.
Octave stellt Polynome als
Vektoren von Koeffizienten der einzelnen Monome geordnet nach Grad
absteigend.
Als Beispiel versuchen wir auszurechnen, auf wieviele
Arten man 11 Franken zusammensetzen kann, wenn man 2 Fünfliber,
5 Zweifränkler und 11 Einfränkler zur Verfügung hat.
Die Polynome werden
\begin{align*}
C_1&=z^{11}+z^{10}+z^9+z^8+z^7+z^6+z^5+z^4+z^3+z^2+z+1,\\
C_2&=z^{10}+z^8+z^6+z^4+z^2+1,\\
C_5&=z^{10}+z^5+1.
\end{align*}
In Octave können diese Polynome als Zeilenvektoren eingegeben werden:
\begin{verbatim}
octave> c1 = [1,1,1,1,1,1,1,1,1,1,1,1]
c1 =

   1   1   1   1   1   1   1   1   1   1   1   1

octave> c2 = [1,0,1,0,1,0,1,0,1,0,1]
c2 =

   1   0   1   0   1   0   1   0   1   0   1

octave> c5 = [1,0,0,0,0,1,0,0,0,0,1]
c5 =

   1   0   0   0   0   1   0   0   0   0   1
\end{verbatim}
Das Produkt der Polynome berechnet man in Octave mit der Funktion
{\tt conv}, also der Faltung zweier Vektoren.
Im vorliegenden Fall bekommt man
\begin{verbatim}
octave:11> c = conv(c1,conv(c2,c5))
c =

 Columns 1 through 16:

  1  1  2  2  3  4  5  6  7  8 10 11 11 12 12 13

 Columns 17 through 32:

 13 12 12 11 11 10  8  7  6  5  4  3  2  2  1  1
\end{verbatim}
Um jetzt herauszufinden, auf wie viele Arten man 11 Franken
kombinieren kann, muss man den Koeffizienten von $z^{11}$ in diesem
Polynom finden, also den zwölften Koeffizienten vom Ende des
Vektors, zum Beispiel so:
\begin{verbatim}
octave> c(length(c)-11)
ans =  11
\end{verbatim}
Man kann also 11 Franken auf 11 verschiedene Arten aus Fränklern,
Zweifränklern und Fünflibern zusammensetzen.

Im Moment scheint dieses Verfahren keine wirklich Vereinfachung
zu sein.
Doch die Verwendung der geometrischen Reihe erlaubt, die
Funktionen $C_n(x)$ zu vereinfachen:
\[
C_n(x)=1+x^n+x^{2n}+x^{3n}+\dots = \frac1{1-x^n}.
\]
Damit wird die uns interessierende Funktion
\begin{align*}
C_5(x) C_{10}(x) C_{20}(x) C_{50}(x)
&=
\frac1{1-x^5}\cdot
\frac1{1-x^{10}}\cdot
\frac1{1-x^{20}}\cdot
\frac1{1-x^{50}}
\\
&=
\frac1{
(1-x^5)
(1-x^{10})
(1-x^{20})
(1-x^{50})
}.
\end{align*}
Eine solche Funktion kann von einem Computer-Algebra-System sehr effizient
in eine Reihe umgewandelt werden, an der man das Resultat ablesen kann.

Man kann allerdings noch einen Schritt weiter gehen.
Kehren wir dazu zum
einführenden Beispiel zurück, bei dem wir nur 5er und 10er verwenden
dürfen.
Wir suchen eine Reihe
\[
C_5(x)C_{10}(x)=c_0+c_1x+c_2x^2+c_3x^3+\dots=\sum_{k=0}^\infty c_kx^k
\]
derart, dass
\[
(1-x^5)(1-x^{10}) \sum_{k=0}^\infty c_kx^k=1
=
(1-x^5-x^{10}+x^{15}) \sum_{k=0}^\infty c_kx^k=1.
\]
Durch Ausmultiplizieren findet man aus der rechten Seite

{\allowdisplaybreaks
\begin{align*}
C_5(x)C_{10}(x)&=c_0\\
&+c_1x\\
&+c_2x^2\\
&+c_3x^3\\
&+c_4x^4\\
&+(c_5-c_0)x^5\\
&+(c_6-c_1)x^6\\
&+(c_7-c_2)x^7\\
&+(c_8-c_3)x^8\\
&+(c_9-c_4)x^9\\
&+(c_{10} - c_5 -c_0)x^{10}\\
&+(c_{11} - c_6 -c_1)x^{11}\\
&+(c_{12} - c_7 -c_2)x^{12}\\
&+(c_{13} - c_8 -c_3)x^{13}\\
&+(c_{14} - c_9 -c_4)x^{14}\\
&+(c_{15} - c_{10} - c_{5} +c_0)x^{15}\\
&+(c_{16} - c_{11} - c_{6} +c_1)x^{16}\\
&+(c_{17} - c_{12} - c_{7} +c_2)x^{17}\\
&+(c_{18} - c_{13} - c_{8} +c_3)x^{18}\\
&+(c_{19} - c_{14} - c_{9} +c_4)x^{19}\\
&+\dots
\end{align*}
}
Aus $C_{5}(x)C_{10}(x)=1$ folgt, dass $c_0=1$ sein muss, und alle
höheren Koeffizienten müssen verschwinden.
Insbesondere verschwinden
alle Koeffizienten, die nicht durch 5 teilbar sind.
Es bleiben also nur noch
die Gleichungen:
\begin{align*}
c_5&=c_0=1\\
c_{10}&=c_5+c_0=2\\
c_{15}&=c_{10}+c_{5}-c_0=2 + 1 - 1=2\\
c_{5n}&=c_{5(n-1)}+c_{5(n-2)}-c_{5(n-3)}.
\end{align*}
Für die Koeffizienten gibt es also eine Rekursionsgleichung, aus der wir
nacheinander die Koeffizienten berechnen können.
Es ist natürlich nicht
überraschend, dass wir die Lösung
\[
1, 1, 2, 2, 3, 3, 4, 4, 5, 5, 6, 6, 7, 7, 8, 8, \dots
\]
finden.

Für das volle Problem ist die Rekursionsgleichung etwas komplizierter, nämlich
\begin{align*}
c_{5(n+17)} % 0
-c_{5(n+16)} % 5
-c_{5(n+15)} % 10
+c_{5(n+14)} % 15
-c_{5(n+13)} % 20
+c_{5(n+12)} % 25
&
\\
+c_{5(n+11)} % 30
-c_{5(n+10)} % 35
%c_{5(n+9)} % 40
%-c_{5(n+8)} % 45
-c_{5(n+7)} % 50
+c_{5(n+6)} % 55
+c_{5(n+5)} % 60
&
\\
-c_{5(n+4)} % 65
+c_{5(n+3)} % 70
-c_{5(n+2)} % 75
-c_{5(n+1)} % 80
+c_{5(n)} % 85
&=0
\end{align*}
Andererseits gibt es ein allgemeines Verfahren, wie man eine solche
Differenzengleichung lösen kann, es gibt daher auch eine allgemeine
Formel, mit der man die $c_{5n}$ berechnen kann.




