\section{Produktregel: Die Für--jedes--gibt--es--Regel}
\label{kombinatorik-produktregel}
\kopfrechts{Produktregel}
\index{Produktregel}
$k$ Position müssen unabhängig voneinander markiert werden, wobei
$n$ verschiedene Markierungen zur Verfügung stehen.
Da die einzelnen
Entscheidungen unabhängig sind voneinander, ist die Zahl der Möglichkeiten
das Produkt der Möglichkeiten an jeder einzelnen Position, also
$n^k$.

Stehen an den Positionen $i\in\{1,\dots,k\}$ jeweils die $n_i$
Wahlmöglichkeiten
zur Verfügung, ist die Gesamtzahl der Möglichkeiten:
\[
n_1\cdot n_2\dotsm n_k=\prod_{i=1}^kn_i.
\]
Die Produktregel ist immer dann anwendbar wenn man das Entstehen
aller Möglichkeiten aus den Wahlmöglichkeiten an jeder Position
mit einem Satz wie dem folgenden beschreiben kann
\begin{quote}
Für jede der $n_1$ Möglichkeiten an Position 1 gibt es eine
von der ersten Position unabhängige
Anzahl $N$ Möglichkeiten für den Rest.
\end{quote}
Dann ist die Gesamtzahl der Möglichkeiten $n_1\cdot N$. 
Natürlich wird man oft die Grösse $N$ durch eine ähnliche,
verfeinerte Analyse bestimmen müssen.

\begin{beispiele}
\item Wie viele mögliche Augenzahlbilder können entstehen, wenn zwei
verschiedenfarbige Würfel geworfen werden?

\begin{loesung}
Der erste Würfel kann $n_1=6$ verschiedene Augenzahlen anzeigen,
der zweite $n_2=6$.
Die beiden Würfel sind unabhängig voneinander,
also gibt es $n_1\cdot n_2=36$ verschiedene Augenzahlbilder.
\end{loesung}


\item Ein Autohändler bietet 5 verschiedene Fahrzeugtypen in 30
verschiedenen Farben an.
Zu jedem Fahrzeugtyp gibt es 7 verschiedene Extraausstattungen.
Wieviele verschiedene Fahrzeuge kann der Autohändler verkaufen?

\begin{loesung}
Offenbar sind alle die $n_1=5$ Fahrzeugtypen, die $n_2=30$ Farben
und die $n_3=7$ Extraausstattungen unabhängig voneinenander wählbar, also ist
die Gesamtzahl der möglichen Fahrzeuge $n_1n_2n_3=1050$.
\end{loesung}

\item Nehmen wir jetzt an, statt vorgegebener Ausstattungen mit
Extras könne jedes einzelne Extra unabhängig gewählt oder
abgelehnt werden.
Es gebe 7 solche Extras.
Wieviele verschiedene Fahrzeuge können bestellt werden?

\begin{loesung}
Offenbar hat man jetzt für jedes Extra die Wahl, ob man es dazunehmen
will oder nicht, man hat also für jedes Extra zwei Möglichkeiten.
Für jedes Extra kommt also ein Faktor $2$ hinzu, mithin hat man
$5\cdot 30
\cdot 2
\cdot 2
\cdot 2
\cdot 2
\cdot 2
\cdot 2
\cdot 2
=5\cdot 30\cdot 2^7=5\cdot 30 \cdot 128=19200$.
\end{loesung}

\item Als das iPhone 5 neu war, konnte man in weiss oder in schwarz
bestellen und in drei verschiedenen
Grössen des Flashspeichers.
Wieviele verschiedene iPhone 5 gab es?

\begin{loesung}
Offenbar konnte man $n_1=2$ Farben und $n_2=3$ Speicherausstattungen
unabhängig voneinander auswäh\-len, es gab
daher $n_1n_2=6$ verschiedene iPhone 5.
\end{loesung}

\end{beispiele}

