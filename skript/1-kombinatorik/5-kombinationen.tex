\section{Kombinationen: Auswahl}
\index{Kombinationen}
\index{Auswahl}
Auf wieviele Arten kann $k$ aus $n$ verschiedenen Objekten
auswählen.
Auch diese Frage kann ein Abzählargument
beantworten.
Sei $C^n_k$ die Anzahl der Möglichkeiten,
$k$ aus $n$ Objekten auszuwählen.
Es ist also zunächst
$k$ mal eine Auswahl zu treffen.
Für die erste Auswahl stehen $n$ Objekte zur Verfügung.
Für jede solche Wahl bleiben jetzt noch $k-1$ Wahlen zu treffen.
Für die zweite
Auswahl stehen aber nur noch $n-1$ Alternativen zur Verfügung,
bisher haben wir also $n(n-1)$ Möglichkeiten.
Wiederholen wir
dies, finden wir insgesamt $n(n-1)(n-2)\dots(n-k+1)$ Möglichkeiten
(das Produkt hat $k$ Faktoren).
In diesem Prozess werden
aber all möglichen Permutationen der resultierenden Auswahl
unabhängig voneinander erzeugt, obwohl sie uns nicht
interessieren.
Wir erhalten also soviel mal zu viele Lösungen, wie
es Permutationen von $k$ Objekten gibt.
Man muss also noch durch
$P_k=k!$ teilen:
\[
C^n_k=\frac{n(n-1)(n-2)\dotsm(n-k+1)}{k!}.
\]
Durch Erweitern mit $(n-k)!$ bekommt man eine etwas
kompaktere Formel:
\[
C^n_k
=
\frac{n(n-1)(n-2)\dotsm(n-k+1)\cdot(n-k)(n-k-1)\dotsm2\cdot 1}{k!(n-k)!}
=
\frac{n!}{k!(n-k)!}.
\]

Auch für diese Grösse kann man eine Rekursionsformel finden.
Die Zahl der Auswahlen von $k$ Elementen aus $n$ gibt es
offenbar $C^{n-1}_{k}$ Auswahlen, welche das erste Element nicht
enthalten, und $C^{n-1}_{k-1}$ Auswahlen, die dadurch entstehen,
dass man zunächst das erste Element in die Auswahl nimmt, und dann
aus den verbleibenden Elementen deren $k-1$ hinzufügt.
Es muss also
gelten:
\index{Rekursionsformel!der Binomialkoeffizienten}
\[
C^n_k=C^{n-1}_{k-1}+C^{n-1}_{k}.
\]
Diese Rekursionsformel ist aber identisch mit der Rekursionsformel
der Binomialkoeffizienten, also gilt
\index{Binomialkoeffizient}
\[
C^n_k=\binom{n}{k}.
\]

\begin{beispiele}
\item In einen Wald mit 1000 Bäumen schlägt fünfmal der Blitz ein.
Wir dürfen annehmen, dass kein Baum zweimal getroffen, denn der
Einschlag wird den Baum weitgehend zerstören.
Auf wieviele Arten
können die getroffenen Bäume im Wald verteilt sein?

\begin{loesung}
Offenbar geht es darum, 5 von 1000 Bäumen auszuwählen, was 
auf
\[
\binom{1000}{5}=8250291250200
\]
verschiedene Arten möglich ist.
\end{loesung}

\item Für ein Projekt stellt eine Firma mit 30 Mitarbeitern eine Taskforce
aus 5 Leuten zusammen.
Auf wieviele Arten ist das möglich?


\begin{loesung}
Es geht darum $5$ von $30$ Mitarbeiter auszuwählen, was auf
\[
\binom{30}{5}=142506
\]
Arten geht.
\end{loesung}

\item Im Gegensatz zu vorangegangenen Beispiel ist jetzt der
Taskforceleiter schon festgelegt, wieviele Gestaltungen der
Task-Force bleiben?

\begin{loesung}
Jetzt stehen nur noch 29 Mitarbeiter zur Verfügung (der Taskforceleiter
steht nicht mehr zur Auswahl), es müssen aber auch nur noch 4 ausgewählt
werden.
Dies ist auf
\[
\binom{29}{4}=23751
\]
Arten möglich.
\end{loesung}

\item\label{meisterschaft}
Ein Land muss für eine Meisterschaft aus einem Pool von
20 Männern und 10 Frauen eine Delegation von je 5 Männern
und Frauen zusammenstellen.
Auf wieviele Arten ist dies möglich?

\begin{loesung}
Die Auswahl der Männer und Frauen erfolgt unabhängig.
Es gibt
$\binom{20}{5}=15504$ mögliche Auswahlen bei den Männern,
und $\binom{10}{5}=252$ mögliche Auswahlen bei den Frauen.
Insgesamt also
\[
\binom{20}{5}\binom{10}{5}=15504\cdot 252 = 3907008
\]
mögliche Delegationen.
\end{loesung}

\item In welchem Prozentsatz der im vorangegangenen Beispiel
zusammengestellten Delegationen ist eine ganz bestimmte Frau
nicht dabei, in welchem Prozentsatz ein ganz bestimmter Mann?

\begin{loesung}
Um den Prozentsatz der Delegationen zu bestimmen, bei denen
eine ganz bestimmte Frau nicht dabei ist, müssen wir Zählen,
wieviele Delegationen sich ohne diese Frau zusammenstellen 
lassen.
Dazu stehen offenbar nur 9 Frauen zur Verfügung, die 
Anzahl der Delegationen ist also
\[
\binom{20}{5}\binom{9}{5}=15504\cdot 126 = 1953504.
\]
Der Anteil dieser Delegationen ist
\[
\frac{
\binom{20}{5}\binom{9}{5}
}{
\binom{20}{5}\binom{10}{5}
}
=
\frac{ \binom{9}{5} }{ \binom{10}{5} }
=
\frac{126}{252}=\frac12.
\]
\end{loesung}
Bei den Männern führt die analoge Rechnung auf
\[
\frac{\binom{19}{5}}{\binom{20}{5}}=\frac{11628}{15504}=\frac34.
\]
Ein bestimmter Mann ist also bei 3 von 4 Delegationen nicht dabei,
eine bestimmte Frau aber nur bei 2 von 4.
\item Ein Medium behauptet, er können durch Handauflegen erspüren, ob
der Gegenstand in einer Schachtel für ein Gewaltverbrechen verwendet
worden sei.
Daher wird folgender Test durchgeführt.
$n=6$
Schachteln werden präpariert, und eine enthält ein Messer,
welches in einem Mordfall als Tatwaffe verwendet worden war.
Die anderen fünf enthalten je eine Lindor Schoggikugel frisch aus der Packung.
Wir dürfen annehmen, dass die Schoggikugeln nicht für ein Gewaltverbrechen
verwendet worden sind.
Das Medium soll jetzt erfühlen, in welcher
Schachtel die Tatwaffe steckt.
Weil das ziemlich schwierig ist, erlauben
wir dem Medium zwei Tips und sind zufrieden wenn eine der beiden
gewählten Schachteln die Tatwaffe enthält.
Wie viele Möglichkeiten gibt
es für das Medium Erfolg oder Misserfolg zu haben?
Wie sieht es aus, wenn man $n$ grösser macht?

\begin{loesung}
Es gibt offenbar $\binom{6}{2}$ Möglichkeiten, zwei Schachteln auszuwählen,
aber nur $\binom{5}{2}$ Möglichkeiten, zwei Schachteln auszuwählen,
die die Tatwaffe nicht enthalten.
\[
\binom{6}{2}=15,\qquad\binom{5}{2}=10.
\]
Es gibt also 5 Möglichkeiten für Erfolg, und 10 Möglichkeiten
für Misserfolg.

Offenbar kann man das auch für beliebige $n$ ausrechnen.
Aus $n$
Schachteln $2$ auswählen ist auf $\binom{n}2$ Arten möglich, in
$\binom{n-1}2$ Fällen findet man die Tatwaffe nicht.
Die Erfolgsfälle
sind
\[
\binom{n}2-\binom{n-1}2
=
\frac{n(n-1)}2-\frac{(n-1)(n-2)}2
=
\frac{n-1}2(n-(n-2))=n-1.
\]
Die Erfolgsquote ist also
\[
\frac{n-1}{\displaystyle\binom{n}2}
=
\frac{n-1}{\displaystyle\frac{n(n-1)}2}=\frac2n.
\]
Man kann daraus schliessen, dass ein Medium, welches nichts kann,
trotzdem in $2$ von $n$ Fällen die Tatwaffe finden wird.
Wenn wir also einen Betrüger von einem echten Medium unterscheiden
wollen, müssen wir die Zahl $n$ so gross wählen, dass nur ganz
wenige Betrüger eine Chance haben.
Verlangen wir $n=20$, dann
besteht immer noch jeder zehnte Betrüger den Test.
Offenbar ist unser
Test viel zu leicht.
\end{loesung}

\end{beispiele}
Die Zahl der Kombinationen kann in Octave/Matlab mit der Funktion
{\tt nchoosek} berechnet werden.
Die Aufgabe~\ref{meisterschaft}
kann zum Beispiel so gerechnet werden:
\begin{verbatim}
octave> nchoosek(20,5) * nchoosek(10,5)
ans =  3907008
\end{verbatim}
\index{nchoosek@{\tt nchoosek}}

