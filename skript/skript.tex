%
% ws-skript.tex -- Skript zur Vorlesung Wahrscheinlichkeit und Statistik,
%                  gehalten an der Hochschule Rapperswil im Wintersemester
%                  06/07
%
% (c) 2006 Prof. Dr. Andreas Mueller, HSR
% $Id: ws-skript.tex,v 1.34 2008/11/02 22:46:16 afm Exp $
%
\documentclass[a4paper,12pt]{book}
\usepackage{german}
\usepackage{times}
\usepackage{amsmath}
\usepackage{amssymb}
\usepackage{amsfonts}
\usepackage{amsthm}
\usepackage{graphicx}
\usepackage{fancyhdr}
\usepackage{textcomp}
\usepackage{picins}
\usepackage[all]{xy}
\usepackage{txfonts}
\usepackage{alltt}
\usepackage{verbatim}
\makeindex
\begin{document}
\pagestyle{fancy}
\lhead{Wahrscheinlichkeit und Statistik}
\frontmatter
\newcommand\HRule{\noindent\rule{\linewidth}{1.5pt}}
\begin{titlepage}
\vspace*{\stretch{1}}
\HRule
\vspace*{10pt}
\begin{flushright}
{\Huge
Wahrscheinlichkeitsrechnung}
\end{flushright}
\begin{flushright}
{\Huge und Statistik}
\end{flushright}
\HRule
\begin{flushright}
\vspace{30pt}
\LARGE
Andreas M"uller
\end{flushright}
\vspace*{\stretch{2}}
\begin{center}
Hochschule f"ur Technik, Rapperswil, 2009
\end{center}
\end{titlepage}

\newcounter{beispiel}
\newenvironment{beispiele}{
\bgroup\smallskip\parindent0pt\bf Beispiele\egroup

\begin{list}{\arabic{beispiel}.}
  {\usecounter{beispiel}
  \setlength{\labelsep}{5mm}
  \setlength{\rightmargin}{0pt}
}}{\end{list}}
\newenvironment{teilaufgaben}{
\begin{enumerate}
\renewcommand{\labelenumi}{\alph{enumi})}
}{\end{enumerate}}
% Loesung
\def\swallow#1{
%nothing
}
\newenvironment{loesung}{%
\begin{proof}[L"osung]%
\renewcommand{\qedsymbol}{$\bigcirc$}
}{\end{proof}}
\def\keineloesungen{%
\renewenvironment{loesung}{\swallow\begingroup}{\endgroup}%
}

\tableofcontents
\newtheorem{satz}{Satz}[chapter]
\newtheorem{hilfssatz}{Hilfssatz}[chapter]
\newtheorem{definition}{Definition}[chapter]
\newtheorem{annahme}{Annahme}[chapter]
\mainmatter
\input einleitung.tex
\input kombinatorik.tex
\input wahrscheinlichkeit.tex
\input laplace.tex
\input erwartung.tex
\input verteilung.tex
\input stetige-verteilungen.tex
\input diskrete-verteilungen.tex
\input schaetzen.tex
\input testen.tex
\input filter.tex
\appendix
%\input e01-generatoren.tex
\input spamfilter.tex
\input google.tex
\input junkscience.tex
\input tabellen.tex
\input skript.ind
\end{document}
