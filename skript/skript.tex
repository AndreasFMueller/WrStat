%
% skript.tex -- Skript zur Vorlesung Wahrscheinlichkeit und Statistik,
%               gehalten an der Hochschule Rapperswil seit Wintersemester
%               06/07
%
% (c) 2006-2016 Prof. Dr. Andreas Mueller, HSR
%
\documentclass[a4paper,12pt]{book}
\usepackage[ngerman]{babel}
\usepackage[utf8]{inputenc}
\usepackage[T1]{fontenc}
\usepackage{times}
\usepackage{geometry}
\geometry{papersize={210mm,297mm},total={160mm,240mm},top=31mm,bindingoffset=15mm}
\usepackage{amsmath}
\usepackage{amssymb}
\usepackage{amsfonts}
\usepackage{amsthm}
\usepackage{graphicx}
\usepackage{fancyhdr}
\usepackage{textcomp}
\usepackage[all]{xy}
\usepackage{txfonts}
\usepackage{alltt}
\usepackage{verbatim}
\usepackage[colorlinks=true]{hyperref} % changed
\usepackage{epsdice}
\usepackage{color}
\usepackage{array}
\makeindex
\setlength{\headheight}{15pt} % fix for headheight warning

% better modulo spacing http://www.matthewflickinger.com/blog/archives/2005/02/20/latex_mod_spacing.asp
\makeatletter
	\def\imod#1{\allowbreak\mkern10mu({\operator@font mod}\,\,#1)}
\makeatother

\begin{document}
\pagestyle{fancy}
\lhead{Wahrscheinlichkeit und Statistik}
\frontmatter
\newcommand\HRule{\noindent\rule{\linewidth}{1.5pt}}
\begin{titlepage}
\vspace*{\stretch{1}}
\HRule
\vspace*{10pt}
\begin{flushright}
{\Huge
Wahrscheinlichkeitsrechnung}
\end{flushright}
\begin{flushright}
{\Huge und Statistik}
\end{flushright}
\HRule
\begin{flushright}
\vspace{30pt}
\LARGE
Andreas Müller
\end{flushright}
\vspace*{\stretch{2}}
\begin{center}
Hochschule für Technik, Rapperswil, 2006-2017
\end{center}
\end{titlepage}

\newcounter{beispiel}
\newenvironment{beispiele}{
\bgroup\smallskip\parindent0pt\bf Beispiele\egroup

\begin{list}{\arabic{beispiel}.}
  {\usecounter{beispiel}
  \setlength{\labelsep}{5mm}
  \setlength{\rightmargin}{0pt}
}}{\end{list}}
\newenvironment{teilaufgaben}{
\begin{enumerate}
\renewcommand{\labelenumi}{\alph{enumi})}
}{\end{enumerate}}
% Beispiel
\newenvironment{beispiel}[1][Beispiel]{%
\begin{proof}[\bf #1]%
\renewcommand{\qedsymbol}{$\bigcirc$}%
}{\end{proof}}
% Loesung
\def\swallow#1{
%nothing
}
\newenvironment{loesung}{%
\begin{proof}[Lösung]%
\renewcommand{\qedsymbol}{$\bigcirc$}
}{\end{proof}}
\def\keineloesungen{%
\renewenvironment{loesung}{\swallow\begingroup}{\endgroup}%
}

\hypersetup{
    linktoc=all,
    linkcolor=blue
}

\tableofcontents
\newtheorem{satz}{Satz}[chapter]
\newtheorem{hilfssatz}{Hilfssatz}[chapter]
\newtheorem{definition}{Definition}[chapter]
\newtheorem{annahme}{Annahme}[chapter]
\mainmatter
\input einleitung.tex
\input kombinatorik.tex
\input wahrscheinlichkeit.tex
\input spamfilter.tex
\input google.tex
\input erwartung.tex
\input verteilung.tex
\input stetige-verteilungen.tex
\input diskrete-verteilungen.tex
\input schaetzen.tex
\input testen.tex
\input filter.tex
\appendix
%\input junkscience.tex
\input datenblaetter.tex
\input tabellen.tex
\clearpage
\pagebreak
\ifodd\value{page}\else\null\clearpage\fi

\input skript.ind
\end{document}
