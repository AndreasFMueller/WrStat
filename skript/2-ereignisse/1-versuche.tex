\section{Versuche und Versuchsausgänge}
\kopflinks{Versuche und Versuchtsausgänge}
Die Wahrscheinlichkeitsrechnung befasst sich also mit dem Ausgang von
Versuchen.
Das in der Einleitung beschriebene Fussballspiel ist ein solches Experiment.
Natürlich interessiert den Fussballfan viel mehr als das Ergebnis,
für die Zwecke der Sportwetten genügen jedoch die Schlussresultate.
Ähnliche Experimente kann man sich in grosser Zahl ausdenken:
\begin{enumerate}
\item
\index{Munzwurf@Münzwurf}
Der Wurf einer Münze ist ein Experiment mit, wenigstens
im Prinzip, drei Versuchsausgängen: Kopf oder Zahl, ganz selten könnte
die Münze auch auf der Kante stehen bleiben.
\item
\index{Wurfel@Würfel}
Ein weiteres Experiment ist der Wurf eines Würfels.
Klassische Würfel
haben sechs Seitenflächen, die verschiedene Augenzahlen zeigen.
Nach jeder Durchführung des Experimentes zeigt der Würfel ein neues
Resultat, das Experiment hat sechs verschiedene Ausgänge.
\item
Der gleichzeitige Wurf zweier verschiedenfarbiger Würfel ist ein Experiment
mit 36 verschiedenen möglichen Ausgängen:
\begin{center}
\def\e#1#2{\epsdice[black]{#1}\,\epsdice{#2}}
\begin{tabular}{|c|cccccc|}
\hline
&\epsdice{1}&\epsdice{2}&\epsdice{3}&\epsdice{4}&\epsdice{5}&\epsdice{6}\\
\hline
\epsdice[black]{1}&\e{1}{1}&\e{1}{2}&\e{1}{3}&\e{1}{4}&\e{1}{5}&\e{1}{6}\\
\epsdice[black]{2}&\e{2}{1}&\e{2}{2}&\e{2}{3}&\e{2}{4}&\e{2}{5}&\e{2}{6}\\
\epsdice[black]{3}&\e{3}{1}&\e{3}{2}&\e{3}{3}&\e{3}{4}&\e{3}{5}&\e{3}{6}\\
\epsdice[black]{4}&\e{4}{1}&\e{4}{2}&\e{4}{3}&\e{4}{4}&\e{4}{5}&\e{4}{6}\\
\epsdice[black]{5}&\e{5}{1}&\e{5}{2}&\e{5}{3}&\e{5}{4}&\e{5}{5}&\e{5}{6}\\
\epsdice[black]{6}&\e{5}{1}&\e{6}{2}&\e{6}{3}&\e{6}{4}&\e{6}{5}&\e{6}{6}\\
\hline
\end{tabular}
\end{center}
\item 
Der gleichzeitige Wurf zweier ununterscheidbarer Würfel ist ein 
ein ganz ähnliches Experiment, aber die Zahl der Versuchsausgänge
ist kleiner.
Da die Würfel nicht unterscheidbar sind, kann man das Paar
\epsdice{5}\,\epsdice{6} nicht
vom Paar \epsdice{6}\,\epsdice{5} unterschieden.
Man kann beim Auflisten der Versuchsausgänge die Paare immer aufsteigend
ordnen, es sind also nur die folgenden Versuchsausgänge möglich:
\begin{center}
\def\e#1#2{\epsdice{#1}\,\epsdice{#2}}
\begin{tabular}{|c|cccccc|}
\hline
&\epsdice{1}&\epsdice{2}&\epsdice{3}&\epsdice{4}&\epsdice{5}&\epsdice{6}\\
\hline
\epsdice{1}&\e{1}{1}&\e{1}{2}&\e{1}{3}&\e{1}{4}&\e{1}{5}&\e{1}{6}\\
\epsdice{2}&        &\e{2}{2}&\e{2}{3}&\e{2}{4}&\e{2}{5}&\e{2}{6}\\
\epsdice{3}&        &        &\e{3}{3}&\e{3}{4}&\e{3}{5}&\e{3}{6}\\
\epsdice{4}&        &        &        &\e{4}{4}&\e{4}{5}&\e{4}{6}\\
\epsdice{5}&        &        &        &        &\e{5}{5}&\e{5}{6}\\
\epsdice{6}&        &        &        &        &        &\e{6}{6}\\
\hline
\end{tabular}
\end{center}
Es bleiben also nur noch 
\[
1+2+\dots +6=\sum_{i=1}^6=\frac{6\cdot(6+1)}2=21
\]
Versuchsausgänge übrig.
Das heisst aber noch lange nicht, dass der Versuchsausgang
\epsdice{2}\,\epsdice{3} gleich wahrscheinlich ist wie 
\epsdice{3}\,\epsdice{3}, denn letzterer kann nur auf genau eine
Art entstehen, während es für \epsdice{2}\,\epsdice{3}
zwei Möglichkeiten gibt.
\item 
Ein Gewitter tobt über einem Wald. 
Es ist möglich, dass ein Blitz in einen Baum einschlägt,
die meisten Bäume werden jedoch verschont bleiben.
Wir können dieses Experiment zwar nicht jederzeit durchführen,
aber wir können einfach auf das nächste Gewitter warten.
Ist der Baum getroffen, können wir ihn auch nicht wieder verwenden,
aber wir können die Beobachtung an einem anderen, vergleichbaren
Baum wiederholen.
\item
\index{Ebola}
Menschen, die sich mit Ebola-Viren anstecken, sterben sehr häufig
an dieser Krankheit.
Auch hier liegt ein Experiment vor, welches wir nicht nach belieben
durchführen dürfen, doch wir können darauf warten, dass jemand
erkrankt, und dann vergleichbare Krankheitsfälle untersuchen.
\item
Die Messung mit einem Messgerät ist ebenfalls ein wiederholbares
Experiment, der Versuchsausgang ist der vom Messgerät angezeigte
Messwert.
\end{enumerate}

Bevor also Aussagen über die Wahrscheinlichkeit gemacht werden können,
muss definiert werden, welches Experiment genau durchgeführt worden ist.
Nur Experimente, die im Prinzip wiederholbar sind, sind der Untersuchung
durch die Wahrscheinlichkeitsrechnung zugänglich.
Die Aussage: ``die Wahrscheinlichkeit, dass die Naturkonstanten des
Universums so sind, dass intelligentes Leben entstehen kann, ist\dots''
ist nicht nur deshalb Unsinn, weil wir nicht wissen, welche
Voraussetzungen zu intelligentem Leben führen, sondern vor allem wegen
der Tatsache, dass wir keine anderen Universen zur Verfügung haben, in denen
wir die Frage nach intelligentem Leben erneut stellen könnten.

Ein logisches Argument kann nicht eine Wahrscheinlichkeit haben,
richtig oder falsch zu sein.
Ein Argument ist einfach richtig oder falsch.
Die Voraussetzungen für das logische Argument sind vielleicht Beobachtungen
in der Natur, also Versuchsausgänge anderer Experimente, und sie
können das Argument stützen, aber ein richtiges Argument wird nicht
falsch, wenn die Voraussetzungen nicht mehr erfüllt sind, sondern einfach
nur nicht anwendbar.
Ein besonders krasses Beispiel für völliges Unverständnis
dieser Grundtatsache ist die Conservapedia-Seite {\em Counterexamples
to an Old Earth}\footnote{\url{http://www.conservapedia.com/Counterexamples\_to\_an\_Old\_Earth}}.

Für die Zwecke der Wahrscheinlichkeitsrechnung ist der genaue Ablauf
eines Experimentes gegenstandslos, nur das Resultat interessiert.
Wir führen daher die folgenden Begriffe in:

\begin{definition}
Der Ausgang eines Experimentes heisst {\em Elementarereignis}, die
Menge aller Elementarereignisse wird mit $\Omega$ bezeichnet.
Ein Elementarereignis $\omega$ ist also ein Element von $\Omega$,
$\omega\in\Omega$.
\index{Elementarereignis}
\end{definition}

Die Menge $\Omega$ der Elementarereignisse für jedes der Beispiele weiter
oben ist:
\begin{enumerate}
\item $\Omega=\{\text{Kopf},\text{Zahl},\text{Kante}\}$
\item $\Omega=\{\epsdice{1},
\epsdice{2},
\epsdice{3},
\epsdice{4},
\epsdice{5},
\epsdice{6}\}$
\item $\Omega$ besteht aus allen Paaren von Augenzahlen.
\item $\Omega$ besteht aus allen Paaren von Augenzahlen, in denen die
zweite Zahl mindestens so gross ist wie die erste:
{
\def\e#1#2{\epsdice{#1}\,\epsdice{#2}}
\def\p{\phantom{\epsdice{1}\,\epsdice{1},\,}}
\begin{align*}
\Omega=\{
           &\e{1}{1},\e{1}{2},\e{1}{3},\e{1}{4},\e{1}{5},\e{1}{6},\\
           &\p       \e{2}{2},\e{2}{3},\e{2}{4},\e{2}{5},\e{2}{6},\\
           &\p       \p       \e{3}{3},\e{3}{4},\e{3}{5},\e{3}{6},\\
           &\p       \p       \p       \e{4}{4},\e{4}{5},\e{4}{6},\\
           &\p       \p       \p       \p       \e{5}{5},\e{5}{6},\\
           &\p       \p       \p       \p       \p       \e{6}{6}
\}
\end{align*}
}
\item $\Omega=\{\text{Baum vom Blitz getroffen},\text{Baum verschont}\}$
\item $\Omega=\{\text{an Ebola gestorben},\text{Ebola überlebt}\}$
\item Die Menge der Elementarereignisse ist die Menge aller möglichen
Messwerte, also $\Omega=\mathbb{R}$ oder eine Teilmenge davon.
\end{enumerate}

