\section{Weitere Beispiele von Ereignisalgebren} \label{section-beispiele}
\subsection{Dominosteine}
Die Menge aller möglichen Dominosteine wurde bereits früher untersucht.
Jetzt möchten wir darin einzelne Ereignisse auszeichnen.
Wir beschreiben
einen einzelnen Dominostein als ein paar $(x,y)$, wobei $x\ge y$ sein soll.
Beispiele von Ereignissen:
\begin{align*}
S_k&=\{ \text{Die Augensumme ist $k$}\}\\
S_5&=\{ (5,0), (4,1), (3,2) \}\\
S_4&=\{ (4,0), (3,1), (2,2) \}\\
R&=\{\text{die Augenzahlen sind positiv haben keinen gemeinsamen Teiler $>1$}\}\\
 &=\{ (6,5), (6,1), (5,4), (5,3), (5,2), (5,1), (4,3), (4,1), (3,2), (3,1), (2,1), (1,1) \}
\\
P&=\{\text{beide Augenzahlen sind Primzahlen}\}\\
 &=\{(5,5), (5,3), (5,2), (3,3), (3,2), (2,2) \}\\
\end{align*}

\subsection{Würfeln mit zwei Würfeln}
Bei einem Würfelspiel wirft man jeweils zwei Würfel.
Zeigen beide Würfel
die gleiche Zahl, man nennt dies einen {\em Pasch}, darf man genau ein
weiteres Mal würfeln.
Die Elementarereignisse sind also entweder einfach Paare, also von
der Form $(x,y)$, mit $x\ne y$, oder ein Pasch gefolgt von irgend einem
Würfelresultat, wir schreiben dies $(P_k, (x,y))$, also ein $k$-er-Pasch
gefolgt von einem Paar $(x,y)$, wobei diesmal keine Einschränkungen für
$x$ und $y$ gelten.
Damit kann man jetzt alle möglichen Elementarereignisse auflisten
\begin{align*}
\Omega=\{
&\phantom{(6,6),} (6,5), (6,4), (6,3), (6,2), (6,1)\\
&(5,6), \phantom{(5,5),} (5,4), (5,3), (5,2), (5,1)\\
&(4,6), (4,5), \phantom{(4,4),} (4,3), (4,2), (4,1)\\
&(3,6), (3,5), (3,4), \phantom{(3,3),} (3,2), (3,1)\\
&(2,6), (2,5), (2,4), (2,3), \phantom{(2,2),} (2,1)\\
&(1,6), (1,5), (1,4), (1,3), (1,2), \phantom{(1,1)}
\}
\\
&\cup
\{(P_6,(x,y))\}
\cup
\{(P_5,(x,y))\}
\cup
\{(P_4,(x,y))\}
\\
&\cup
\{(P_3,(x,y))\}
\cup
\{(P_2,(x,y))\}
\cup
\{(P_1,(x,y))\}.
\end{align*}
Man sieht daraus zum Beispiel, dass es $30 + 6\cdot 36=246$
Elementarereignisse gibt.
Darin enthalten sind die Ereignisse
\begin{align*}
P&=\{\text{Pasch im ersten Wurf}\}\\
&=
\{(P_6,(x,y))\}
\cup
\{(P_5,(x,y))\}
\cup
\{(P_4,(x,y))\}
\\
&\qquad \cup
\{(P_3,(x,y))\}
\cup
\{(P_2,(x,y))\}
\cup
\{(P_1,(x,y))\}
\\
\tilde P_k&=\{\text{$k$-er Pasch im ersten Wurf}\}\\
   &=\{(P_k,(x,y))\}
\\
Q&=\{\text{totale Augensumme $\ge 10$}\}\\
&=\{\phantom{(6,6),} (6,5), (6,4), \\
&\phantom{\;=\{}(5,6), \phantom{(5,5), (5,4)} \\
&\phantom{\;=\{}(4,6)\phantom{, (4,5), (4,4)}
\}
\\
&\qquad\cup
\{(P_6,(x,y))\}
\cup
\{(P_5,(x,y))\}
\cup
\{(P_4,(x,y))\,|\, x+y \ge 2\}
\\
&\qquad\cup
\{(P_3,(x,y))\,|\, x+y \ge 4\}
\cup
\{(P_2,(x,y))\,|\, x+y \ge 6\}
\\
&\qquad
\cup
\{(P_1,(x,y))\,|\, x+y \ge 8\}.
\end{align*}

\subsection{AIDS-Test}
Wir mathematisieren das Beispiel des AIDS-Tests.
Die frühere Diskussion führt uns vor Augen,
dass wir hier mit zwei verschiedenen Ereignissen zu tun haben. 
Das Experiment besteht offenbar darin, dass wir jemanden untersuchen,
die Menge der Elementarereignisse ist also die Menge aller betrachteten
Personen.
Darin unterscheiden wir:
\begin{align*}
H&=\{\text{Person hat HIV}\},\\
T&=\{\text{Person hat positiven AIDS-Test}\}.
\end{align*}
Es ist klar, dass $H\ne T$.
Es gibt Personen, die zwar HIV haben, bei
denen der Test dies aber noch nicht zeigen kann, $H\setminus T\ne \emptyset$.
Andererseits gibt es falsche positive Testresultate,
$T\setminus H\ne\emptyset$. 

\subsection{Messwertalgebra}
\begin{table}
\begin{center}
\begin{tabular}{|cc|cc|}
\hline
1&0.990
&11&0.990\\
2&0.989
&12&0.989\\
3&0.991
&13&0.990\\
4&0.991
&14&0.991\\
5&0.991
&15&0.989\\
6&0.989
&16&0.990\\
7&0.990
&17&0.989\\
8&0.989
&18&0.990\\
9&0.992
&19&0.990\\
10&0.992
&20&0.991\\
\hline
\end{tabular}
\end{center}
\caption{Werte von Stichprobe von 20 1k$\Omega$-Widerständen\label{widerstandswerte}}
\end{table}
Bei einer Messung wird ein Messwert ermittelt, dieser kann jedoch
nur mit einer begrenzten Genauigkeit festgehalten werden.
Die Elementarereignisse sind zwar immer noch alle möglichen reellen
Zahlen $\mathbb R$, aber sinnvolle Ereignisse sind nur bestimmte Teilmengen
$A\subset\mathbb R$.
Wir wollen bestimmen, welche Teilmengen das sind.

Zunächst führen wir dies an einem Beispiel durch.
Bei einer am 1.~November 2006
zufällig bei Pusterla in Zürich gekauften Stichprobe von 20 1k$\Omega$
Widerständen ergaben sich beim Nachmessen die Widerstandswerte in 
Tabelle \ref{widerstandswerte}.
Offensichtlich streuen die Messwerte
zwischen 0.989 und 0.992.
Genauere Informationen kann das Messgerät nicht
anzeigen.
Wir können aus den Daten also eigentlich nur folgendes schliessen:
der Widerstand mit der Nummer eins hat einen Wert $0.990\le R_1<0.991$.
Der Widerstandswert $R_1$ ist ein Versuchsergebnis, also ein Elementarereignis,
liegt aber in der Menge
\[
A=[0.990,0.991)\subset \mathbb{R}=\Omega.
\]
Ausserdem werden durch die weiteren Messungen auch noch folgende
Ereignisse realisiert:
\begin{center}
\begin{tabular}{|c|c|}
\hline
Ereignis&Häufigkeit\\
\hline
$[0.989,0.990)$&6\\
$[0.990,0.991)$&7\\
$[0.991,0.992)$&5\\
$[0.992,0.993)$&2\\
\hline
\end{tabular}
\end{center}
Diese speziellen Ereignisse beinhalten offensichtlich alle Information,
die wir über die Widerstände haben können.
Bei den 20 Versuchen treten aber auch noch andere Ereignisse ein,
zum Beispiel
\begin{center}
\begin{tabular}{|c|c|}
\hline
Ereignis&Häufigkeit\\
\hline
$\{R\ge 0.990\}$&14\\
``5\% Toleranz'' = $\{0.95\le R\le1.05\}$&20\\
``1\% Toleranz'' = $\{0.99\le R\le 1.01\}$&14\\
\hline
\end{tabular}
\end{center}

Etwas formaler: Eine Genauigkeitsangabe bei einem Messwert definiert
ein Intervall, in dem sich ein Messwert befinden muss.
Wir
fordern also, dass $\cal A$ alle Intervalle der Form $[a,b]\subset\mathbb R$
enthalten muss.
Aber auch die Aussage ``der Messwert ist kleiner als $a$''
muss einem Ereignis in $\cal A$ entsprechen, also müssen auch Intervalle,
die sich ins unendliche erstrecken in $\cal A$ sein:
\begin{align*}{}
(-\infty,r]&=\{x\in\mathbb R\;|\;x \le r\} \in \cal A\\
% Hack: \; brauchts, damit eqnarray nicht meint [ leite ein Argument ein
\;[r,\infty)&=\{x\in\mathbb R\;|\;x \ge r\} \in \cal A
\end{align*}

Da wir Komplemente bilden können, müssen auch die offenen Intervalle
\begin{align*}
\overline{[a,\infty)}&=\mathbb R\setminus [a,\infty)=(-\infty,a)\in\cal A\\
\overline{(-\infty,b]}&=\mathbb R\setminus (-\infty,b]= (b,\infty)\in\cal A
\end{align*}
Ereignisse sein.
Ein beliebiges offenes Intervall lässt sich
als Durchschnitt zweier solcher Intervalle ausdrücken:
\[
(a,b)=(-\infty, b)\cap(a,\infty).
\]
Durch Bildung von geeigneten Durchschnitten lassen sich also alle
Intervalle bilden.
Ausserdem müssen alle Teilmengen von $\mathbb R$
zu $\cal A$ hinzugenommen werden, die sich durch Bildung von Komplementen,
Durchschnitten und Vereinigungen bilden lassen.
Offensichtlich ist $\cal A$
in diesem Falle sehr kompliziert.

