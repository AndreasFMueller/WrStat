%
% exponentialverteilung.tex -- abschnitt ueber Exponentialverteilung in Kapitel 5
%
% (c) 2015 Prof Dr Andreas Mueller, Hochschule Rapperswil
%
\subsection{Exponentialverteilung} \label{section-exponentialverteilung}
\begin{table}
\renewcommand{\arraystretch}{2}
\begin{center}
\begin{tabular}{|l|l|}
\hline
Name&Exponentialverteilung\\
\hline
Dichtefunktion&$\displaystyle ae^{-ax}$,\quad$a>0$\\
Verteilungsfunktion&$1-e^{-ax}$\\
Erwartungswert&$\displaystyle \frac1a$\\
Varianz&$\displaystyle \frac1{a^2}$\\
Median&$\displaystyle \frac1a\log 2$\\[8pt]
$P(|X-E(X)|>\varepsilon)$&
\begin{minipage}{3.7in}
$
\begin{cases}
e^{-a\varepsilon-1}&\qquad\text{f"ur $\varepsilon > \displaystyle\frac1a$}\\
1-e^{a\varepsilon-1}+e^{-a\varepsilon-1}&\qquad\text{f"ur $\varepsilon \le \displaystyle\frac1a$}
\end{cases}
$
\end{minipage}
\\[10pt]
\hline
Anwendungen&\begin{minipage}{3.7in}%
\strut
$\bullet$ Prozess ohne Erinnerungsverm"ogen\\
$\bullet$ Radioaktivit"at
\strut
\end{minipage}\\
\hline
\end{tabular}
\end{center}
\caption{Datenblatt der Exponentialverteilung\label{datenblatt:exponentialverteilung}}
\end{table}
\begin{figure}
\begin{center}
%\includegraphics[width=0.8\hsize]{graphics/expphi}
%\includegraphics[width=0.8\hsize]{graphics/expF}
\includegraphics[width=0.8\hsize]{images/verteilungsfunktion-8}
\end{center}
\caption{Verteilungsfunktion (oben) und Dichtefunktion (unten) der
Exponentialverteilung
\label{bildexponentialverteilung}}
\end{figure}
Bei radioaktiven Stoffen und bei gewissen Bauteilen stellt man fest,
dass ihr Zerfall bzw.~ihr Versagen kein Ged"achtnis hat.
Die Wahrscheinlichkeit, dass ein Atomkern innerhalb der Zeit $t$ zerf"allt,
ist gleich gross wie die Wahrscheinlichkeit, dass er zwischen $t_0$
und $t_0+t$ zerf"allt, wenn er bis zur Zeit $t_0$ nicht zerfallen ist.
Auch Bauteile ohne Erm"udungserscheinungen verhalten sich so.

\subsubsection{Verteilungsfunktion und Dichtefunktion}
Etwas formaler sei $X$ eine Zufallsvariable, die die Zeit angibt, zu der
ein Atomkern zerf"allt. Die ``Ged"achtnislosigkeit'' bedeutet, dass die
bedingte Wahrscheinlichkeit f"ur einen Zerfall vor $t_0+t$ unter der
Annahme, dass der Kern bis zur Zeit $t_0$ nicht zerfallen ist, gleich
gross ist wie die Wahrscheinlichkeit eines Zerfalls bis zur Zeit $t$:
\[
P(X \le t) = P(X\le t_0+t\,|\,X > t_0)
\]
Dies ist nat"urlich gleichbedeutend mit der Wahrscheinlichkeit f"ur
die negierten Ereignisse:
\[
P(X > t) = P(X > t_0+t\,|\,X > t_0)
\]
Aus der Definition~\ref{def-bedingte-wahrscheinlichkeit}
der bedingten Wahrscheinlichkeit folgt
\[
P(X> t_0+t\,|\,X>t_0)=\frac{P(X>t_0+t\wedge X>t_0)}{P(X > t_0)}
=\frac{P(X>t_0+t)}{P(X>t_0)}
\]
Die Bedingung an die Verteilung wird damit zu
\[
P(X>t)P(X>t_0)=P(X>t+t_0).
\]
Schreiben wir jetzt $g(t)=P(X>t)=1-F(t)$, werden die Formeln
etwas "ubersichtlicher:
\[
g(t)g(t_0)=g(t+t_0).
\]
Zun"achst leiten wir nach $t_0$ ab, wir nehmen ja an, dass wir
eine stetige Zufallsvariable haben, und dass die Verteilungsfunktion
differenzierbar sein wird.
Wir erhalten
\[
g(t)g'(t_0)=g'(t+t_0).
\]
Jetzt lassen wir $t_0$ gegen $0$ streben, und bekommen
die Differentialgleichung
\[
g'(t)=g'(0)g(t)
\]
f"ur $g(t)$.
Diese lineare Differentialgleichung erster Ordnung
muss als L"osung eine Exponentialfunktion haben. Da $F(t)$
monoton w"achst, muss $g(t)$ monoton fallen, ausserdem
muss $g(t)$ beschr"ankt bleiben. Damit bleibt nur
$g(t)=e^{-at}$ mit einem positiven $a$.

\begin{definition}
Die Wahrscheinlichkeitsverteilung mit Dichtefunktion
\[
\varphi(x)=\begin{cases}
0&\qquad x<0\\
a e^{-a x}&\qquad x\ge 0
\end{cases}
\]
mit $a>0$ heisst Exponentialverteilung.
Ihre Verteilungsfunktion ist
\[
F(x)=\begin{cases}
0&\qquad\text{f"ur $x < 0$}\\
1-e^{-ax}&\qquad\text{f"ur $x\ge 0$}.
\end{cases}
\]
\end{definition}
\index{Exponentialverteilung}
\index{Verteilungsfunktion!Exponentialverteilung}
\index{Wahrscheinlichkeitsdichte!Exponentialverteilung}
Wir sollten noch nachrechnen, dass dies tats"achlich die richtige
Verteilungsfunktion ist. Zun"achst w"achst sie tats"achlich monoton,
und auch der Grenzwert f"ur $t\to\infty$ ist wie gew"unscht. Aber
auch die Ableitung ist richtig:
\[
\frac{d}{dt}(1-e^{-at})=ae^{-at}\qquad\text{f"ur}\quad t>0.
\]

\subsubsection{Erwartungswert und Varianz}
\index{Erwartungswert!der Exponentialverteilung}
\index{Exponentialverteilung!Erwartungswert}
\index{Varianz!der Exponentialverteilung}
\index{Exponentialverteilung!Varianz}
\begin{satz}Eine exponentialverteilte Zufallsvariable $X$ mit Parameter
$a$ hat folgenden Erwartungswert und folgene Varianz:
\begin{align*}
E(X)&=\frac1a\\
\operatorname{var}(X)&=\frac1{a^2}
\end{align*}
\end{satz}
\begin{proof}[Beweis]
Zur Berechnung von Erwartungswert und Varianz ist es n"utzlich die Integrale
$I_n=\int \xi^ne^{-\xi}\,d\xi$
berechnen zu k"onnen. Diese findet man rekursiv durch partielle Integration:
\begin{align*}
I_n&=\int \xi^ne^{-\xi}\,d\xi\\
&=-\xi^ne^{-\xi}+n\int \xi^{n-1}e^{-\xi}\,d\xi\\
&=-\xi^ne^{-\xi}+nI_{n-1}
\end{align*}
Um Erwartungswert und Varianz zu berechnen, verwendet man mit Vorteil die
Variablentransformation $ax=\xi$. F"ur den Erwartungswert ergibt sich:
\begin{align*}
E(X)&=\int_0^{\infty}ae^{-ax}x\,dx
=\frac1a\int_0^{\infty}ax e^{-ax}a\,dx
=\frac1a\int_0^{\infty}\xi e^{-\xi}\,d\xi\\
&=\frac1a\left[-\xi e^{-\xi}-e^{-\xi}\right]_0^\infty=\frac1a
\end{align*}
Analog f"ur die Varianz:
\begin{align*}
E(X^2)
&=
\int_0^{\infty}x^2ae^{-ax}\,dx
=\frac1{a^2}\int_0^{\infty}(ax)^2e^{-ax}a\,dx\\
&=\frac1{a^2}\int_0^{\infty}\xi^2e^{-\xi}\,d\xi
=\frac1{a^2}\left[\xi^2e^{-\xi}-2\xi e^{-\xi}+2e^{-\xi}\right]_0^\infty
=\frac2{a^2}
\\
\operatorname{var}(X)
&=E(X^2)-E(X)^2=\frac2{a^2}-\left(\frac1a\right)^2=\frac1{a^2}
\end{align*}
\end{proof}
Die Gr"osse $\frac1a$ l"asst sich also leicht interpretiren: sie ist die
mittlere ``Lebensdauer'', man findet sie oft unter dem K"urzel MTBF f"ur
mean time between failure.
Und $\sqrt{\operatorname{var}(X)}$ 
ist genau gleich gross. 
\index{Mean time between failure}
\index{MTBF}
\subsubsection{Wahrscheinlichkeit grosser Abweichungen}
\begin{figure}
\begin{center}
\includegraphics{images/exp-1.pdf}
\end{center}
\caption{Wahrscheinlichkeit f"ur eine grosse Abweichung bei einer
Exponentialverteilten Zufallsvariable, oben die durch den Satz von Tschebyscheff
gegebene Schranke (gr"un), unten die exakte Rechnung mit
Hilfe der Exponentialvereteilung (rot)\label{abweichung-exponential}}
\end{figure}
{
\small
Wir k"onnen nun auch die Wahrscheinlichkeit einer grossen Abweichung
berechnen:
\begin{satz} F"ur eine exponentialverteilte Zufallsvariable mit
Erwartungswert $\frac1a$ ist die Wahscheinlichkeit einer Abweichung
$\varepsilon$ vom Erwartungswert
\[
P(|X-{\textstyle\frac1a}|>\varepsilon)=
\begin{cases}
e^{-a\varepsilon-1}&\qquad\text{f"ur $\varepsilon > \frac1a$}\\
1-e^{a\varepsilon-1}+e^{-a\varepsilon-1}&\qquad\text{f"ur $\varepsilon \le \frac1a$}
\end{cases}
\]
\end{satz}
\begin{proof}[Beweis]
Die Wahrscheinlichkeit einer grossen Abweichung ist
\begin{align*}
P(|X-{\textstyle\frac1a}|>\varepsilon)
&=1-\int_{\frac1a-\varepsilon}^{\frac1a+\varepsilon}\varphi_a(x)\,dx\\
&=1-\int_{\max(0,\frac1a-\varepsilon)}^{\frac1a+\varepsilon}ae^{-ax}\,dx\\
&=1-\left[-e^{-ax}\right]_{\max(0,\frac1a-\varepsilon)}^{\frac1a+\varepsilon}\\
&=1+e^{-a\varepsilon-1}-e^{-\max(0,1-a\varepsilon)}\\
&=\begin{cases}
e^{-a\varepsilon-1}&\qquad\text{f"ur $\varepsilon > \frac1a$}\\
1-e^{a\varepsilon-1}+e^{-a\varepsilon-1}&\qquad\text{f"ur $\varepsilon \le \frac1a$}
\end{cases}
\end{align*}
\end{proof}
Der Satz von Tschebyscheff setzt diese Wahrscheinlichkeit in Relation
zur Varianz
\[
P(|X-{\textstyle\frac1a}|>\varepsilon)\le
\frac{\operatorname{var}(X)}{\varepsilon^2}=\frac{1}{a^2\varepsilon^2},
\]
selbstverst"andlich muss diese Ungleichung immer noch erf"ullt sein.
Also
\begin{alignat*}{3}
e^{-a\varepsilon-1}&\le&\frac1{a^2\varepsilon^2}&\qquad\text{f"ur $a\varepsilon>1$}\\
1-e^{a\varepsilon-1}+e^{-a\varepsilon-1}&\le&\frac1{a^2\varepsilon^2}&\qquad\text{f"ur $a\varepsilon\le1$}
\end{alignat*}
In allen Ausdr"ucken kommt immer nur das Produkt $a\varepsilon$ vor,
wir k"onnen daher
abk"urzend $u=a\varepsilon$ schreiben. Die zweite Ungleichung wird dann zu
\[
\frac{1-e^{u}+e^{-u}}e\le\frac1{u^2}\qquad\text{f"ur $u\le1$}
\]
In diesem Fall ist die rechte Seite mindestens 1,
die Tschebyscheff-Ungleichung wird jetzt v"ollig nichtssagend, denn 
gr"osser als 1 kann die Wahrscheinlichkeit f"ur eine Abweichung ohnehin
nicht werden.

Die erste Ungleichung wird zu
\[
e^{-u-1}\le\frac1{u^2}\qquad\text{f"ur $u > 1$}.
\]
Diese Funktion f"allt monoton sehr schnell gegen 0, viel schneller
als der Ausdruck $\frac1{u^2}$. In der Abbildung~\ref{abweichung-exponential}
sieht man beide Schranken, die allgemeine, gegeben durch den
Satz von Tschebyscheff, und die exakte f"ur die Exponentialverteilung.
}

\subsubsection{Anwendung}
Ger"ate aller Art versuchen, die Lebensdauer dadurch zu erh"ohen, dass
kritische Komponenten redundant aufgebaut werden.
Zum Beispiel verwenden Server h"aufig zwei Disks, deren Daten
gespiegelt sind, so dass der Ausfall eines Disks noch keinen Ausfall
des Gesamtsystems verursacht. Wir wollen die erwartete Zeit bis zum
Ausfall des Gesamtsystems berechnen, wenn zwei Disks verwendet werden,
deren Zeit bis zum Ausfall exponentialverteilt ist. Weiter nehmen
wir an, dass die Disks unabh"angig voneinander ausfallen.

Sei also $T_i$ die Zeit bis zum Ausfall von Disk $i$, mit
Verteilungsfunktion $F_{T_i}(t)=1-e^{at}$ f"ur $t\ge 0$. Gesucht
ist die Verteilungsfunktion f"ur die Zeit $T$ bis zum Ausfall
des Gesamtsystems, also die Funktion $F(t)=P(T\le t)$. Das Gesamtsystem
f"allt aus, wenn beide Disks ausgefallen sind, es ist also
\begin{align*}
F(t)
&=P(T\le t)=P(T_1\le t\wedge T_2\le t)\\
&=P(T_1\le t)\cdot P(T_2\le t)
= F_{T_1}(t) F_{T_2}(t)=(1-e^{-at})^2
\end{align*}
Zur Berechnung des Erwartungswertes von $T$ wird die
Wahrscheinlichkeitsdichte ben"otigt, also die Ableitung davon:
\begin{equation}
\varphi_{T}(t)=2(1-e^{-at})ae^{-at}.
\label{disks-wdichte}
\end{equation}
Damit wird der Erwartungswert
\begin{align*}
E(T)
&=
\int_{-\infty}^{\infty}t\varphi_T(t)\,dt
=\int_0^\infty 2t(1-e^{-at})ae^{-at}\,dt
\\
&=2\int_0^\infty ate^{-at}\,dt - \int_0^\infty t (2a)e^{-(2a)t}\,dt
=2\cdot\frac1a-\frac1{2a}=\frac{4-1}2\cdot\frac1a=\frac32\cdot\frac1a.
\end{align*}
Durch Redundanz l"asst sich die mittlere Zeit bis zu einem Ausfall
also nur um 50\% steigern, die Kosten f"ur das Disksystem werden
aber mehr als verdoppelt.
Abbildung~\ref{graph:disksystem} zeigt, dass die Wahrscheinlichkeit
eines ``fr"uhen'' Ausfalls stark reduziert wird, der eigentliche
Nutzen der Redundanz ist also weniger die Verl"angerung der 
Lebensdauer, sondern die Tatsache, dass man darauf verzichten kann,
eine teure F"ahigkeit zur sofortigen Reaktion aufzubauen.
\begin{figure}
\centering
\includegraphics{images/exp-3.pdf}
\caption{Wahrscheinlichkeitsdichte (\ref{disks-wdichte}) 
der Ausfallzeit eines redundanten Disksystems (blau)
im Vergleich zur Verteilung der Ausfallzeit eines einzelnen Disk (gr"un).
\label{graph:disksystem}}
\end{figure}

