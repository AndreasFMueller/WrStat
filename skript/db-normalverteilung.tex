%
% db-normalverteilung.tex -- datenblatt der Normalverteilung
%
% (c) 2015 Prof Dr Andreas Mueller, Hochschule Rapperswil
%
\subsection{Steckbrief}
\begin{center}
\renewcommand{\arraystretch}{2}
\begin{tabular}{|l|l|}
\hline
Name&Normalverteilung\\
\hline
\setlength{\extrarowheight}{2pt}
Dichtefunktion&$\displaystyle\frac{1}{\sqrt{2\pi}\sigma}e^{-\frac{(x-\mu)^2}{2\sigma^2}}$\\
Verteilungsfunktion&keine elementare Funktion\\
Erwartungswert&$\mu$\\
Varianz&$\sigma^2$\\
Median&$\mu$\\
$P(|X-E(X)|>\varepsilon)$&keine einfache Formel\\
\hline
%\setlength{\extrarowheight}{50pt}
Anwendungen&
\begin{minipage}{3.7in}%
\vskip4pt
\strut
$\bullet$ Messwerte\\
$\bullet$ Summe vieler kleiner Einfl"usse vergleichbar grosser Varianz
(Zentraler Grenzwertsatz)
\\
$\bullet$ Approximation der Binomialverteilung
\strut
\end{minipage}\\[21pt]
\hline
\end{tabular}
\end{center}

\subsection{Verteilungsfunktion und Wahrscheinlichkeitsdichte}
Verteilungsfunktion (oben) und Dichtefunktion (unten) der Normalverteilung:
\begin{center}
%\includegraphics[width=0.8\hsize]{graphics/normphi}
%\includegraphics[width=0.8\hsize]{graphics/normF}
\includegraphics[width=\hsize]{images/verteilungsfunktion-9}
\end{center}

\subsection{Wahrscheinlichkeit einer grossen Abweichung}
Vergleich der Wahrscheinlichkeit f"ur eine grosse Abweichung
f"ur die Normalverteilung (rot) und die Schranke von Tschebyscheff (gr"un):
\begin{center}
\includegraphics{images/norm-1.pdf}
\end{center}

\subsection{Parameter sch"atzen}
Die Parameter $\mu$ und $\sigma$ k"onnen mit den erwartungstreuen Sch"atzern
\begin{align*}
\hat\mu(x_1,\dots,x_n)&=\frac{x_1+\dots+x_n}{n}\\
\hat\sigma(x_1,\dots,x_n)^2&=\frac{1}{n-1}\biggl(
\sum_{i=1}^n x_i^2 - \frac1n\biggl(\sum_{i=1}^n x_i\biggr)^2
\biggr)
\end{align*}
gesch"atzt werden.

Der Mittelwert ist $\hat\mu(x_1,\dots,x_n)$ ist normalverteilt mit Erwartungswert
$\mu$ und Varianz $\frac1n\sigma^2$.
Die Stichprobenvarianz $\hat\sigma(x_1,\dots,x_n)^2/\sigma^2$ ist $\chi^2$-verteilt
mit $n-1$ Freiheitsgraden.

\subsection{Zentraler Grenzwertsatz}
Der zentrale Grenzwertsatz besagt, das die Verteilungsfunktion einer Summe
einer grossen Zahl von Zufallsvariablen unter milden Voraussetzungen
gegen die Verteilungsfunktion einer Normalverteilung konvergiert.
Dies rechtfertigt den Einsatz der Normalverteilung als Modell f"ur Prozesse,
in denen eine grosse Zahl von vergleichbar grossen Einfl"ussen zu einem Effekt
beitragen, zum Beispiel bei Messwerten.

\subsection{Standardisierung}
Ist $X$ eine normalverteilte Zufallsvariable, dann ist 
\[
Z=\frac{X-\mu}{\sigma}
\]
eine normalverteilte Zufallsvariable mit Erwartungswert $0$ und Varianz $1$,
d.~h.~eine standardnormalverteilte Zufallsvariable.
F"ur die Standardnormalverteilung ist im Tabellenanhang eine Tabelle der
Verteilungsfunktion sowie einzelner Quantilen zu finden.

Man beachte, dass die Zufallsvariable $Z$ nicht mehr normalverteilt ist, wenn man
$\mu$ und $\sigma$ durch Sch"atzwerte ersetzt, die resultierende Verteilung
ist dann eine $t$-Verteilung.
