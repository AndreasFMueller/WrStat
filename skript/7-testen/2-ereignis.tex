\section{Testen der Wahrscheinlichkeit eines Ereignisses}
\kopfrechts{Testen der Wahrscheinlichkeit eines Ereignisses}
Mit Hilfe von Beobachtungen können wir die relative Häufigkeit ermitteln, mit
welcher ein Ereignis $A$ eintritt, von dem wir glauben, dass es mit
Wahrscheinlichkeit $p=P(A)$ eintreten sollte.
Damit stellt sich automatisch die Frage, ob wir mit Hilfe der
Beobachtungen entscheiden können, ob $p$ tatsächlich die Wahrscheinlichkeit
von $A$ ist.

Wir betrachten dazu unabhängige Zufallsvariable $(X_i)_{1\le i\le n}$,
welche ausschliesslich
die Werte $0$ und $1$ annehmen, und zwar mit Wahrscheinlichkeit
$1-p$ bzw.~$p$.
Die Zufallsvariable $X_i$ soll mit dem Wert $1$ anzeigen,
ob bei der $i$-ten Durchführung des Experiments das Ereignis eingetreten ist.
Die Summe der Zufallsvariablen 
\[
X=\sum_{i=1}^nX_i
\]
ist eine Zufallsvariable, die angibt, wie oft bei $n$ Versuchen das Ereignis
eingetreten ist.
Für Erwartungswert und Varianz von $X_i$ finden
wir folgende Werte
\begin{align*}
E(X_i)
&=
0\cdot (1-p)+1\cdot p=p
\\
\operatorname{var}(X_i)
&=
E(X_i^2)-E(X_i)^2=0^2\cdot (1-p)+1^2\cdot p-p^2 =p-p^2
\\
&=
p(1-p).
\end{align*}
Entsprechend hat $X$ den Erwartungswert $E(X)=np$ und
die Varianz $\operatorname{var}(X)=np(1-p)$.

Nach dem zentralen Grenzwertsatz wird die geeignet standardisierte Summe
von genügend vielen Summanden $X_i$ eine Verteilungsfunktion haben,
die jener einer Normalverteilung sehr nahe kommt, 
\begin{equation}
\frac{X-np}{\sqrt{np(1-p)}}
\label{chi2-primitiv}
\end{equation}
ist also angenähert standardnormalverteilt.
Daraus liesse sich nach den
Ideen des voranstehenden Abschnitts bereits ein Test konstruieren.
Etwas handlicher ist jedoch das Quadrat von (\ref{chi2-primitiv}), der
sich mit Hilfe der Anzahl der Fälle $Y=n-X$, in denen das Ereignis
nicht eingetreten ist, noch etwas symmetrischer Schreiben lässt
\begin{align}
D=\frac{(X-np)^2}{np(1-p)}
&=
\frac{(X-np)^2}{n}\cdot \frac{1}{p(1-p)}\nonumber
\\
&=
\frac{(X-np)^2}{n}\cdot
\biggl(
\frac{1-p}{p(1-p)} +
\frac{p}{p(1-p)}
\biggr)\nonumber
\\
&=
\frac{(X-np)^2}{np}+\frac{(X-np)^2}{n(1-p)}\nonumber
\\
&=
\frac{(X-np)^2}{np}+\frac{(n-X-n(1-p))^2}{n(1-p)}\label{chi2-primitiv2}
\end{align}
Im letzten Ausdruck (\ref{chi2-primitiv2})
steht im ersten Term die Abweichung der
Anzahl der Eintretensfälle des Ereignisses von der erwarteten Häufigkeit,
im zweiten Term die Anzahl der Nichteintretensfälle des Ereignisses
von deren erwarteter Häufigkeit.
Da wir vom Quadrat einer standardnormalverteilten Zufallsvariable ausgegangen
sind, ist die Diskrepanz $D$ $\chi^2$-verteilt mit einem Freiheitsgrad.

Um die Hypothese zu testen, dass das Ereignis mit Wahrscheinlichkeit $p$
eintritt, brauchen wir nur genügend viele Beobachtungen durchzuführen
(so dass die Approximation durch den zentralen Grenzwertsatz gültig wird),
die Diskrepanz $D$ auszurechnen und mit Hilfe der $\chi^2$-Verteilung
zu prüfen, ob dieser Wert von $D$ ``unwahrscheinlich genug'' ist.

Für einen Test auf dem Niveau $\alpha=0.01$ müssten wir jenes
$x$ finden, für welches $F_{\chi^2_1}(x)=0.99$ gilt.
Die $\chi^2$-Tabelle \ref{chi2-tabelle} im Anhang \ref{chapter:tabellen}
liefert $x=6.63$.
Sobald also $D\ge 6.63$ würde die Hypothese verworfen,
dass $p$ nicht die Wahrscheinlichkeit des Ereignisses $A$ ist.

In einem Experiment wurde eine Münze 30 mal geworfen, wobei 13 mal
Kopf erschien und 17 mal Zahl.
Die Hypothese $p=0.5$ führt auf eine
Diskrepanz 
\[
D=\frac{(13-15)^2}{15}+\frac{(17-15)^2}{15}=\frac{8}{15}=0.53333,
\]
also viel zu klein, um an der Hypothese zu zweifeln.
Natürlich beweist das auch nicht, dass tatsächlich $p=0.5$.
Berechnen wird die Diskrepanz für irgend ein $p$, dann gilt
\[
D(p)=\frac{(13-30p)^2}{30p}+\frac{(17-30(1-p))^2}{30(1-p)}
=\frac{(13-30p)^2}{30p(1-p)}.
\]
Die vorhandenen Beobachtungen bringen uns dazu, die Hypothese eines
bestimmten Wertes von $p$ auf dem Niveau $\alpha=0.01$ zu verwerfen,
wenn $D(p)\ge 6.63$ ist.
Selbst für $p=0.3$ findet man immer noch einen
Wert $D(0.3)=2.54$, welcher nicht gestattet,
die Hypothese $P(\text{Kopf})=0.3$ zu verwerfen.
Hingegen liefert 
$D(0.7)=10.16\ge 6.63$ Grund genug, auf dem Niveau $\alpha=0.01$ nicht
länger an die Hypothese $P(\text{Kopf})=0.7$ zu glauben.

Der Test lässt sich übrigens auch umkehren.
Hat der Experimentator
gemogelt, und seine Daten fabriziert, dann können wir erwarten, dass
die Diskrepanz kleiner ist, als wir erwarten dürften.
Eine Diskrepanz
von $0$, also Kopf und Zahl gleich häufig, ist extrem unwahrscheinlich.

