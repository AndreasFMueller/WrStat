%
% testen.tex
%
% (c) 2006-2015 Prof. Dr. Andreas Mueller
%
\rhead{Testen}
\chapter{Hypothesen testen} \label{chapter-hypothesen-testen}
Francis Bacon (1561--1626) hat gegen die im Mittelalter übliche
Art des Spekulierens in den Wissenschaften einen umfassenden Neubau
der Wissenschaften auf der Grundlage der ``unverfälschten Erfahrung''
versucht.
Die Entscheidung über Fakten und kausale Zusammenhänge
sollten nicht durch abgehobenes Debattieren und Berufung auf 
Autoritäten entschieden werden, sondern durch den Ausgang
möglichst eindeutiger Experimente.
Ein solches ``experimentum crucis''
sollte einem unvoreingenommenen Beobachter eine klare Antwort geben
können.
In der modernen Wissenschaft hat sich jedoch gezeigt, dass
die Entscheidung selten wirklich so einfach ist.
Oft bewegt sich
der Experimentator an der Grenze des messbaren, und er läuft
ständig Gefahr, Messfehler für bisher unbekannte Effekte zu halten
oder vor lauter statistischen Schwankungen selbst den theoretisch
gut untermauerten Effekt nicht zu sehen.

Der Hypothesentest ist daher das moderne Gegenstück zu Bacons
experimentum crucis.
Er erlaubt dem Experimentator auf neutrale
Art zu entscheiden, ob in seinen Messresultaten nun der vermutete
Effekt sichtbar ist, oder die Entdeckung auch nur durch die unvermeidlichen
Zufallsschwankungen vorgespiegelt sein könnte.

In Kapitel \ref{chapter-schaetzen} wurde gezeigt, wie Parameter einer
Verteilung geschätzt werden können.
Trotzdem bleiben viele Fragen offen:
\begin{itemize}
\item Obwohl man bei der Wiederholung eines Experimentes andere Werte
gemessen hat, möchte man mit möglichst geringer Fehlerwahrscheinlichkeit
bestätigen können, dass man immer noch den gleichen Wert gemessen hat.
\item Kann man 95\% sicher sein, dass ein neuer Werkstoff wirklich grössere
Festigkeit aufweist als sein Vorgänger?
\item Kann man wirklich 95\% sicher sein, dass die Messresultate
in einem Experiment normalverteilt sind?
\end{itemize}
Die Schätzung von Parametern kann diese Fragen allein noch nicht beantworten.
Da der Schätzer selbst eine Zufallsvariable ist, müssen wir noch
mehr über dessen Verteilung wissen, um die erste oder die äquivalente
zweite Frage zu entscheiden.
In der dritten Frage stellt die
Verteilungsfunktion selbst in Frage, also die Grundlagen, auf denen
die ganze Theorie der Schätzer aufgebaut wird.

Ein Hypothesentest versucht die Frage zu beantworten, ob vorhandene
Messdaten mit der Hypothese vereinbar sind, und falls nicht, mit welcher
Wahrscheinlichkeit zu Unrecht an der Gültigkeit der Hypothese gezweifelt
wurde.

\section{Allgemeines Prinzip}
Als Illustration des allgemeinen Problems betrachten wir folgende Frage.
Von einer Zufallsgrösse $X$ wird behauptet, dass sie normalverteilt ist,
mit Erwartungswert $0$ und Varianz $1$.
Nun wird eine Messung durchgeführt, bei der der Wert $10$ gemessen wird.
Die Normalverteilung verbietet natürlich
nicht, dass ein derart hoher Wert auftritt.
Allerdings ist die
Wahrscheinlichkeit dass bei einer Standardnormalverteilten Zufallsvariable
ein Wert $\ge 10$ auftritt
\[
P(|X|\ge 10)=1.523970\cdot 10^{-23}
\]
das beobachtete Ereignis ist also extrem unwahrscheinlich.
Da die Beobachtung
real ist, muss man es als extrem unwahrscheinlich ansehen, dass die
Zufallsvariable tatsächlich standardnormalverteilt war.

Nach diesem Grundprinzip funktionieren die meisten Testverfahren.
Auf der Basis einer Hypothese (oft auch die Nullhypothese genannt)
wird eine Testgrösse $\varphi(\cdot)$ bestimmt, die 
unter der Hypothese ein gewisses Intervall nur mit einer sehr kleinen
Wahrscheinlichkeit verlässt.
Dann wird eine Beobachtung $X$ durchgeführt, und die Testgrösse
$\varphi(X)$ ausgewertet.
Die Hypothese wird verworfen, wenn die
Testgrösse das Intervall in der Beobachtung verlässt.

Im erwähnten Beispiel besteht die Hypothese in der Annahme, dass die
Messgrösse standardnormalverteilt ist.
Die Testgrösse entspricht dem Betrag einer einzelnen Messung.
Auf Grund der Annahme, dass die Testgrösse
standardnormalverteilt ist, kann die Wahrscheinlichkeit dafür berechnet
werden, dass $\varphi(X)=|X|\ge10$.
Genau dieses Ereignis tritt dann ein.

Hier sind offensichtlich zwei Arten von Fehlern möglich:
\begin{enumerate}
\item Fehler erster Art: Die Hypothese war richtig, und wird trotzdem verworfen.
Das Testverfahren kann genau angeben, mit welcher Wahrscheinlichkeit
dieser Fehler gemacht wird.
\item Fehler zweiter Art: Die Hypothese war falsch, wird aber nicht verworfen.
Da die Hypothese falsch war, muss die Berechnung des Intervalls in Zweifel
gezogen werden, in dem sich die Testgrösse befinden soll.
Damit kann
in vielen Fällen über die Wahrscheinlichkeit eines Fehlers zweiter
Art nichts ausgesagt werden.
\end{enumerate}
Man beachte, dass der Test gar nicht beweist, dass die Hypothese richtig
war, er sagt bestenfalls aus, dass die vorhandenen Daten nicht ausreichen,
an der Hypothese zu zweifeln.

Im genannten Beispiel ist die Wahrscheinlichkeit, einen Fehler erster
Art zu begehen $1.523970\cdot 10^{-23}$.
Meist geht man in der Praxis von der
Wahrscheinlichkeit aus, mit der man einen Fehler erster Art zu begehen
bereit ist, und berechnet daraus das Intervall, in dem sich die
Testgrösse bewegen darf.
Verlangt man zum Beispiel, dass ein Fehler erster
Art höchstens mit Wahrscheinlichkeit $0.01$ eintritt, dann muss die 
Testgrösse in einem Intervall $[-m,m]$ liegen mit $P(-m\le X\le m)=0.99$.
Die Gleichung
\[
0.99 = P(-m\le X\le m)
=\frac12\left(\operatorname{erf}(\frac{m}{\sqrt{2}})-\operatorname{erf}(-\frac{m}{\sqrt{2}})\right)
=\operatorname{erf}(\frac{m}{\sqrt{2}})
\]
kann mit Hilfe der inversen Fehlerfunktion aufgelöst werden:
\[
m=\sqrt{2}\operatorname{erf}^{-1}(0.99)=2.575829
\]
Somit wäre bereits eine Abweichung von mehr als dem $2.58$-fachen einer
Standardabweichung Grund genug, die Hypothese zu verwerfen.

\begin{definition}
Ein Test auf dem Niveau $\alpha$ ist eine Regel, welche zu einer
Beobachtung $X$ entscheidet, ob die Hypothese zu verwerfen ist, wobei
sie höchstens mit Wahrscheinlichkeit $\alpha$ einen Fehler erster
Art begeht.
\end{definition}

Weiter oben haben wir einen sehr primitiven Test auf dem Niveau
$\alpha=0.01$ für die
Hypothese formuliert, dass $X$ standardnormalverteilt ist.

\section{Testen der Wahrscheinlichkeit eines Ereignisses}
Mit Hilfe von Beobachtungen können wir die relative Häufigkeit ermitteln, mit
welcher ein Ereignis $A$ eintritt, von dem wir glauben, dass es mit
Wahrscheinlichkeit $p=P(A)$ eintreten sollte.
Damit stellt sich automatisch die Frage, ob wir mit Hilfe der
Beobachtungen entscheiden können, ob $p$ tatsächlich die Wahrscheinlichkeit
von $A$ ist.

Wir betrachten dazu unabhängige Zufallsvariable $(X_i)_{1\le i\le n}$,
welche ausschliesslich
die Werte $0$ und $1$ annehmen, und zwar mit Wahrscheinlichkeit
$1-p$ bzw.~$p$.
Die Zufallsvariable $X_i$ soll mit dem Wert $1$ anzeigen,
ob bei der $i$-ten Durchführung des Experiments das Ereignis eingetreten ist.
Die Summe der Zufallsvariablen 
\[
X=\sum_{i=1}^nX_i
\]
ist eine Zufallsvariable, die angibt, wie oft bei $n$ Versuchen das Ereignis
eingetreten ist.
Für Erwartungswert und Varianz von $X_i$ finden
wir folgende Werte
\begin{align*}
E(X_i)
&=
0\cdot (1-p)+1\cdot p=p
\\
\operatorname{var}(X_i)
&=
E(X_i^2)-E(X_i)^2=0^2\cdot (1-p)+1^2\cdot p-p^2 =p-p^2
\\
&=
p(1-p).
\end{align*}
Entsprechend hat $X$ den Erwartungswert $E(X)=np$ und
die Varianz $\operatorname{var}(X)=np(1-p)$.

Nach dem zentralen Grenzwertsatz wird die geeignet standardisierte Summe
von genügend vielen Summanden $X_i$ eine Verteilungsfunktion haben,
die jener einer Normalverteilung sehr nahe kommt, 
\begin{equation}
\frac{X-np}{\sqrt{np(1-p)}}
\label{chi2-primitiv}
\end{equation}
ist also angenähert standardnormalverteilt.
Daraus liesse sich nach den
Ideen des voranstehenden Abschnitts bereits ein Test konstruieren.
Etwas handlicher ist jedoch das Quadrat von (\ref{chi2-primitiv}), der
sich mit Hilfe der Anzahl der Fälle $Y=n-X$, in denen das Ereignis
nicht eingetreten ist, noch etwas symmetrischer Schreiben lässt
\begin{align}
D=\frac{(X-np)^2}{np(1-p)}
&=
\frac{(X-np)^2}{n}\cdot \frac{1}{p(1-p)}\nonumber
\\
&=
\frac{(X-np)^2}{n}\cdot
\biggl(
\frac{1-p}{p(1-p)} +
\frac{p}{p(1-p)}
\biggr)\nonumber
\\
&=
\frac{(X-np)^2}{np}+\frac{(X-np)^2}{n(1-p)}\nonumber
\\
&=
\frac{(X-np)^2}{np}+\frac{(n-X-n(1-p))^2}{n(1-p)}\label{chi2-primitiv2}
\end{align}
Im letzten Ausdruck (\ref{chi2-primitiv2})
steht im ersten Term die Abweichung der
Anzahl der Eintretensfälle des Ereignisses von der erwarteten Häufigkeit,
im zweiten Term die Anzahl der Nichteintretensfälle des Ereignisses
von deren erwarteter Häufigkeit.
Da wir vom Quadrat einer standardnormalverteilten Zufallsvariable ausgegangen
sind, ist die Diskrepanz $D$ $\chi^2$-verteilt mit einem Freiheitsgrad.

Um die Hypothese zu testen, dass das Ereignis mit Wahrscheinlichkeit $p$
eintritt, brauchen wir nur genügend viele Beobachtungen durchzuführen
(so dass die Approximation durch den zentralen Grenzwertsatz gültig wird),
die Diskrepanz $D$ auszurechnen und mit Hilfe der $\chi^2$-Verteilung
zu prüfen, ob dieser Wert von $D$ ``unwahrscheinlich genug'' ist.

Für einen Test auf dem Niveau $\alpha=0.01$ müssten wir jenes
$x$ finden, für welches $F_{\chi^2_1}(x)=0.99$ gilt.
Die $\chi^2$-Tabelle \ref{chi2-tabelle} im Anhang \ref{anhang-tabellen}
liefert $x=6.63$.
Sobald also $D\ge 6.63$ würde die Hypothese verworfen,
dass $p$ nicht die Wahrscheinlichkeit des Ereignisses $A$ ist.

In einem Experiment wurde eine Münze 30 mal geworfen, wobei 13 mal
Kopf erschien und 17 mal Zahl.
Die Hypothese $p=0.5$ führt auf eine
Diskrepanz 
\[
D=\frac{(13-15)^2}{15}+\frac{(17-15)^2}{15}=\frac{8}{15}=0.53333,
\]
also viel zu klein, um an der Hypothese zu zweifeln.
Natürlich beweist das auch nicht, dass tatsächlich $p=0.5$.
Berechnen wird die Diskrepanz für irgend ein $p$, dann gilt
\[
D(p)=\frac{(13-30p)^2}{30p}+\frac{(17-30(1-p))^2}{30(1-p)}
=\frac{(13-30p)^2}{30p(1-p)}.
\]
Die vorhandenen Beobachtungen bringen uns dazu, die Hypothese eines
bestimmten Wertes von $p$ auf dem Niveau $\alpha=0.01$ zu verwerfen,
wenn $D(p)\ge 6.63$ ist.
Selbst für $p=0.3$ findet man immer noch einen
Wert $D(0.3)=2.54$, welcher nicht gestattet,
die Hypothese $P(\text{Kopf})=0.3$ zu verwerfen.
Hingegen liefert 
$D(0.7)=10.16\ge 6.63$ Grund genug, auf dem Niveau $\alpha=0.01$ nicht
länger an die Hypothese $P(\text{Kopf})=0.7$ zu glauben.

Der Test lässt sich übrigens auch umkehren.
Hat der Experimentator
gemogelt, und seine Daten fabriziert, dann können wir erwarten, dass
die Diskrepanz kleiner ist, als wir erwarten dürften.
Eine Diskrepanz
von $0$, also Kopf und Zahl gleich häufig, ist extrem unwahrscheinlich.

\section{Testen einer diskreten Wahrscheinlichkeitsverteilung}
\label{section-testen-diskreter-wkeitsverteilung}
Karl Pearson hat im Jahre 1900 erkannt, dass sich der Test für die
Wahrscheinlichkeit eines Ereignisses auf einen Test für eine
diskrete Wahrscheinlichkeitsverteilung erweitern lässt.
Wir nehmen
dazu an, dass in einem Experiment $d+1$ Ausgänge möglich sind, und
bezeichnen diese mit $0,\dots,d$.
Das Experiment wird $n$ Mal
durchgeführt, wobei der Ausgang $i$ $n_i$ Mal beobachtet wurde.
Mit diesen Daten möchten wir die Hypothese testen, dass $p_i$ die
Wahrscheinlichkeit für den Ausgang $i$ ist.

In Analogie zum Test für eine Wahrscheinlichkeit bilden wird die
Diskrepanz:
\begin{definition}Die Grösse
\begin{equation}
D=\sum_{i=0}^d\frac{(n_i-np_i)^2}{np_i} \label{formel-diskrepanz}
\end{equation}
heisst Diskrepanz.
\end{definition}
Es gilt:
\begin{satz}[Pearson] Die Diskrepanz $D$ ist für genügend grosse
$n$ annähernd $\chi^2$-verteilt mit $d$ Freiheitsgraden.
\end{satz}
``Genügend gross'' heisst nach einer verbreiteten Faustregel: so
gross, dass $n_i>5\forall i$.
Damit ist anscheinend sichergestellt,
dass für die Zwecke des Tests die Annäherung an die Normalverteilung
genau genug ist.
\begin{proof}[Beweis]
Wie beim Fall einer Wahrscheinlichkeit können wir die Terme
\[
\delta_i=\frac{n_i-np_i}{\sqrt{np_i(1-p_i)}}
\]
als standardnormalverteilt ansehen.
Sie sind jedoch nicht
unabhängig, da $\sum_{i=0}^d n_i=n$ und $\sum_{i=0}^dp_i=1$
gilt.
Das Hauptproblem des Beweises besteht also darin zu zeigen,
dass die Diskrepanz die Quadrat-Summe von $d$ standardnormalverteilten
Zufallsvariablen ist, die sich aus den $\delta_i$ bilden lassen.
Dieses etwas technische Problem bringt uns keine neuen theoretischen
Einsichten, wir verzichten daher auf die Durchführung des Beweises.
\end{proof}

Mit diesem Satz lässt sich offensichtlich ein Test auf dem Niveau
$\alpha$ für jede
beliebige diskrete Verteilung bilden.
Man geht dazu wie folgt vor:
\begin{enumerate}
\item Bestimme $x$ so, dass die $F_{\chi_{d}^2}(x)=1-\alpha$
\item Berechne 
\[
D=\sum_{i=0}^d\frac{(n_i-np_i)^2}{np_i}
\]
\item Verwerfe die Hypothese, dass der Ausgang $i$ mit Wahrscheinlichkeit
$p_i$ eintritt, wenn $D>x$ gilt.
\end{enumerate}

In 100 Würfen mit einem Würfel wurden die Augenzahlen $1$ bis $6$ gezählt,
und dabei folgende Häufigkeiten ermittelt
\begin{center}
\begin{tabular}{|r|r|r|}
\hline
$i$&$n_i$&$(n_i-np_i)^2/np_i$\\
\hline
1&21&1.12671\\
2&18&0.10668\\
3&16&0.02666\\
4&17&0.00667\\
5&15&0.16665\\
6&13&0.80664\\
\hline
&&$D=2.24001$\\
\hline
\end{tabular}
\end{center}
Die Tabelle der Quantilen der $\chi^2$-Verteilung \ref{chi2-tabelle}
zeigt, dass für fünf Freiheitsgrade (sechs mögliche Versuchsausgänge)
selbst auf dem Niveau $\alpha=0.1$ die Diskrepanz mindestens $9.236$ sein
müsste.
Die vorliegenden experimentellen Daten geben also noch keinen
Anlass, daran zu zweifeln, dass jede Augenzahl mit der gleichen
Wahrscheinlichkeit $\frac16$ auftreten wird.

\section{Testen einer stetigen Verteilung}
\label{section-testen-stetiger-wkeitsverteilung}
Der im vorangehenden Abschnitt beschriebene $\chi^2$-Test kann prinzipbedingt
nur diskrete Wahrscheinlichkeitsverteilungen testen.
Will man eine stetige
Verteilung testen, muss man zunächst Klassen von Werten bilden,
deren Wahrscheinlichkeiten berechnen, und dann prüfen, ob die so
künstlich konstruierte diskrete Verteilung im $\chi^2$-Test Bestand hat.
Durch die Notwendigkeit der Klassenbildung führt man eine künstliche
Diskretisierung in das Problem ein.
Will man den Test verfeinern, muss man
die Diskretisierung ebenfalls ändern.

\subsection{Testprinzip}
Diese unbefriedigende Situation korrigiert der Kolmogoroff-Smirnov-Test,
kurz KS-Test.
Dieser vergleicht direkt die heuristische, d.~h.~aus den Beobachtungen gewonnene
Verteilungsfunktion mit der erwarteten Verteilungsfunktion.
Er legt fest,
wie die Unterschiede zwischen den zwei Funktionen gemessen werden sollen,
und legt fest, wann eine Abweichung zu gross ist.

Wir wollen testen, ob $n$ Beobachtungen $X_1,\dots,X_n$ sich mit
der Hypothese vertragen, dass die Zufallsvariablen $X_i$ die
Verteilungsfunktion $F$ haben.
Dazu wird aus den $n$ Beobachtungen
und die zugehörige empirische Verteilungsfunktion
\[
F_n\colon x\mapsto F_n(x)=\frac{|\{x_i|X_i\le x\}|}{n}
\]
gebildet.
Wenn die Zufallsvariable $X$ tatsächlich die
Verteilungsfunktion $F$ hat, wird man erwarten,
dass $F$ und $F_n$ sich nicht stark unterscheiden.
Insbesondere sollten die Grössen
\begin{align}
K_n^+
&=
\sqrt{n}\max_{-\infty<x<\infty} F_n(x)-F(x)
\\
K_n^-
&=
\sqrt{n}\max_{-\infty<x<\infty} F(x)-F_n(x)
\end{align}
nicht sehr gross werden.
Um den Test durchzuführen braucht man also
die Wahrscheinlichkeit, dass $K_n^{\pm}$ einen gewissen Wert übersteigt.
Dafür gibt es Tabellen, zum Beispiel findet man im Anhang \ref{anhang-tabellen}
die Tabelle \ref{KS-tabelle}, welche zu $n$ und einer Wahrscheinlichkeit
$p$ eine Grösse $t_{n,p}$ angibt, für die
\[
P(K_n^+\le t)=t_{n,p}.
\]

Für die Durchführung des Tests geht man wie folgt vor:
\begin{satz} Sei $X_1,\dots,X_n$ sei eine Stichprobe einer Zufallsvariable $X$.
Der Kolmogorov-Smirnov-Test auf dem Niveau $\alpha$ für die Hypothese,
dass $X$ die Verteilungsfunktion $F$ hat wird wie folgt durchgeführt:
\begin{enumerate}
\item Berechne
\[
K_n^+ = \sqrt{n}\max_{-\infty<x<\infty} (F_n(x)-F(x))
\]
\item Finde $t_{n,1-\alpha}$ in der Tabelle \ref{KS-tabelle}
\item Falls
\[
K_n^+>t_{n,1-\alpha}
\]
verwerfe die Hypothese, dass $X$ die Verteilungsfunktion $F$ hat.
\end{enumerate}
\end{satz}

\subsection{Berechnung von \texorpdfstring{$K_n^{\pm}$}{Kn-plus-minus}}
Das Maximum der Differenz $F_n(x)-F(x)$ tritt zwangsläufig an einer
Sprungstelle von $F_n$ auf, also bei einem der Werte der Stichprobe
$X_1,\dots,X_n$.
Durch sortieren können wir erreichen, dass die
$X_i$  in aufsteigender Reihenfolge angeordnet sind, also $X_i\le X_j$
für $i\le j$.
Dann ist $F_n(X_j)=\frac{j}{n}$, und
die  $K_n^{\pm}$ können mit Hilfe der einfacheren Formeln
\begin{align}
K_n^+
&=
\sqrt{n}\max_{1\le j\le n}\biggl(\frac{j}{n}-F(X_j)\biggr)
\label{knp-berechnungs-formel}
\\
K_n^-
&=
\sqrt{n}\max_{1\le j\le n}\biggl(F(X_j)-\frac{j-1}n\biggr)
\end{align}
berechnet werden.
In dieser Form ist die Berechnung der
Verteilung von $K_n^{\pm}$ leichter möglich.

\subsection{Reduktion auf Gleichverteilung}
Das Problem wäre etwas handlicher, wenn die Verteilungsfunktion $F$
die Verteilungsfunktion einer Gleichverteilung wäre.
Der folgende Satz ermöglicht, eine
beliebige Zufallsvariable mit stetiger Verteilungsfunktion in eine
gleichverteilte Zufallsvariable zu transformieren.
\begin{satz}\label{reduktion-auf-gleichverteilung}
Sei $X$ eine Zufallsvariable mit stetiger Verteilungsfunktion $F$, dann
ist $F(X)$ eine auf $[0,1]$ gleichverteilte Zufallsvariable.
\end{satz}
\begin{proof}[Beweis]
Um die Verteilungsfunktion in $F(X)$ zu bestimmen, müssen wir $P(F(X)\le y)$
für ein beliebiges $y\in[0,1]$ berechnen.
Da $F$ stetig ist, gibt es ein
grösstes $x$, für welches $F(x)= y$, zum Beispiel
$x=\max\{\xi|F(\xi)= y\}$.
Mit diesem $x$ folgt
\[
P(F(X)\le y)=P(X\le x)=F(x)=y.
\]
\end{proof}
Aus dem Satz folgt, dass wir die Zahlen $Y_j=nF(X_j)$ darauf prüfen wollen,
ob sie im Intervall $[0,n]$ gleichverteilt sind.

\subsection{Berechnung der Verteilung von \texorpdfstring{$K_n^{\pm}$}{Kn-plus-minus}}
Die Formel (\ref{knp-berechnungs-formel})
für $K_n^+$ kann mit Hilfe der $Y_j$ wie folgt 
geschrieben werden
\begin{equation}
K_n^+=\frac1{\sqrt{n}}\max(1-Y_1, 2-Y_2,\dots,n-Y_n).
\end{equation}
Die Wahrscheinlichkeit, dass $K_n^+\le t/\sqrt{n}$ ist also die
Wahrscheinlichkeit, dass $j-Y_j\le t$ für alle $j$, oder
$Y_j\ge j-t$ für $1\le j\le n$.
Da die $Y_j$ gleichverteilt sind,
kann man diese Wahrscheinlichkeit berechnen.

Wir berechnen zunächst die Wahrscheinlichkeit, dass die Bedingung
für $Y_n$ erfüllt ist unter der Annahme, dass die übrigen
Bedingungen erfüllt sind.
$Y_n$ muss im Intervall $[0,n]$
liegen, darf aber auch nicht kleiner als $n-t$ sein.
Setzt man $\alpha_j=\max(j-t,0)$, dann heisst das, dass $Y_n$ im
Intervall $[\alpha_n, n]$ liegen muss.
Ausserdem ist $Y_n$ gleichverteilt,
somit ist die Wahrscheinlichkeit, dass alle Bedingungen für $Y_n$ erfüllt
sind
\begin{equation}
P(\alpha_n\le Y_n\le n)=\frac{\int_{\alpha_n}^n dy_n}{\int_0^ndy_n}
\label{kpn-erstes-integral}
\end{equation}
Der Nenner dient dazu, die Wahrscheinlichkeit auf $1$ zu normieren.

Nun betrachten wir die analogen Bedingungen für $Y_{n-1}$.
Offensichtlich kann $Y_{n-1}$ zwischen $\alpha_{n-1}$ und $Y_n$ variieren.
Es ergibt sich
als Wahrscheinlichkeit, dafür, dass die Bedingungen an $Y_{n-1}$ erfüllt
sind, der Ausdruck
\begin{equation}
\frac{\int_{\alpha_{n-1}}^{y_n}dy_{n-1}}{\int_0^{y_n}dy_{n-1}}.
\label{kpn-zweites-integral}
\end{equation}
In (\ref{kpn-erstes-integral}) hatten wir angenommen,
dass die Bedingungen für $Y_{n-1}$
bereits erfüllt sind, also die Wahrscheinlichkeit für deren zutreffen $1$
ist. Inzwischen haben wir gelernt, dass diese Wahrscheinlichkeit in
Wahrheit eher wie (\ref{kpn-zweites-integral}) aussieht.
Zusammen finden wir
\begin{equation}
P(\alpha_n\le Y_n\le n\wedge \alpha_{n-1}\le Y_{n-1}\le Y_n)
=
\frac{\int_{\alpha_n}^n dy_n\int_{\alpha_{n-1}}^{y_n}dy_{n-1}}{\int_0^ndy_n\int_0^{y_n}dy_{n-1}}
\label{kpn-zwei-integrale}
\end{equation}
In diesem Sinne können schrittweise die Bedingungen an $Y_{n-2},\dots,Y_1$
erfüllt werden, wir erhalten
\begin{equation}
\frac{\int_{\alpha_n}^n dy_n\int_{\alpha_{n-1}}^{y_n}dy_{n-1}\dotsi\int_{\alpha_1}^{y_2}dy_1}{\int_0^ndy_n\int_0^{y_n}dy_{n-1}\dotsi\int_0^{y_2}dy_1}
\label{kpn-alle-integrale}
\end{equation}
Diese Integrale sind nicht ganz selbstverständlich auszurechnen, weshalb
wir dies hier schrittweise durchführen.

\begin{satz}
\label{kn-elementarvolumen}
Es gilt
\begin{equation}
\int_0^xdy_n\int_0^{y_n}dy_{n-1}\dotsi\int_0^{y_2}dy_1=\frac{x^n}{n!}
\end{equation}
\end{satz}
\begin{proof}[Beweis] Wir führen den Beweis mit vollständiger
Induktion.
Für $n=1$ muss das Integral
\[
\int_0^xdy_1=[y_1]_0^x=x
\]
berechnet werden, was offensichtlich mit $\frac{x^1}{1!}=x$
übereinstimmt.

Nehmen wir an, dass die Formel für $n-1$ stimmt, dann können wir den
Fall $n$ wie folgt verifizieren:
\[
\int_0^xdy_n\int_0^{y_n}dy_{n-1}\dotsi\int_0^{y_2}dy_1
=
\int_0^x\frac{y_n^{n-1}}{(n-1)!}\,dy_n
=
\left[\frac{y_n^n}{n!}\right]_0^x=\frac{x^n}{n!}
\]
womit die Behauptung bewiesen ist.
\end{proof}
Der Nenner in (\ref{kpn-alle-integrale}) ist also $\frac{n^n}{n!}$,
dabei ist $n^n$ das
Volumen eines $n$-dimensionalen Würfels mit Kantenlänge $n$, dies wird
durch $n!$ geteilt, weil die $Y_j$ solange vertauscht werden müssen, bis
sie aufsteigend geordnet sind.
Nur eine der $n!$ Permutationen der $Y_j$
erfüllt diese Bedingung.

Wir kümmern uns nun um den Zähler von (\ref{kpn-alle-integrale}).
Die unteren Grenzen der Integrale sind $\alpha_j=\max(j-t,0)$.
Bei gegebenem $t$ wird $\alpha_j=j-t$
sein für genügend grosse $j$, für alle anderen wird $\alpha_j=0$ sein.
Es gibt also eine Zahl $k$ so dass der Zähler von (\ref{kpn-alle-integrale})
die Form
\[
P_{nk}(x)
=
\int_{n-t}^xdx_n\int_{n-1-t}^{x_n}dx_{n-1}\dotsi\int_{k+1-t}^{x_{k+2}}dx_{k+1}\int_0^{x_{k+1}}dx_k\dotsi\int_0^{x_2}dx_1
\]
hat, es ist dies das grösste $k$, für welches $k-t\le 0$ ist, also
$k=\lfloor t\rfloor$.
Der Zähler von (\ref{kpn-alle-integrale})
ist also $P_{n\lfloor t\rfloor}(n)$.
Mit Hilfe der folgenden Sätze berechnen wir $P_{nk}(x)$ schrittweise.

Wir halten zunächst noch den Spezialfall
\begin{equation}
P_{nn}(x)=\frac{x^n}{n!}
\label{spezialfall-pnn}
\end{equation}
fest, der unmittelbar aus dem Satz \ref{kn-elementarvolumen} folgt.

\begin{satz}
\label{kn-variablentransformation}
Es gilt
\begin{equation}
P_{nk}(x)=\int_{n}^{x+t}dx_n\int_{n-1}^{x_n}dx_{n-1}\dotsi\int_{k+1}^{x_{k+2}}dx_{k+1}\int_t^{x_{k+1}}dx_k\dotsi\int_t^{x_2}dx_1
\label{kpn-variablen-transformation}
\end{equation}
\end{satz}
\begin{proof}[Beweis]
Wir substituieren $x_i+t=\xi_i$.
Dazu müssen zu den Intervallgrenzen
$t$ hinzuaddiert werden:
\begin{align}
P_{nk}(x)
&=
\int_{n-t}^xdx_n\int_{n-1-t}^{x_n}dx_{n-1}\dotsi\int_{k+1-t}^{x_{k+2}}dx_{k+1}\int_0^{x_{k+1}}dx_k\dotsi\int_0^{x_2}dx_1
\nonumber\\
&=
\int_{n}^{x+t}d\xi_n\int_{n-1}^{x_n+t}d\xi_{n-1}\dotsi\int_{k+1}^{x_{k+2}+t}dx_{k+1}\int_t^{x_{k+1}+t}d\xi_k\dotsi\int_t^{x_2+t}d\xi_1
\nonumber\\
&=
\int_{n}^{x+t}d\xi_n\int_{n-1}^{\xi_n}d\xi_{n-1}\dotsi\int_{k+1}^{\xi_{k+2}}dx_{k+1}\int_t^{\xi_{k+1}}d\xi_k\dotsi\int_t^{\xi_2}d\xi_1
\nonumber\\
&=
\int_{n}^{x+t}dx_n\int_{n-1}^{x_n}dx_{n-1}\dotsi\int_{k+1}^{x_{k+2}}dx_{k+1}\int_t^{x_{k+1}}dx_k\dotsi\int_t^{x_2}dx_1
\label{kpn-umbenennung}
\end{align}
Im letzten Schritt (\ref{kpn-umbenennung}) haben wir die Integrationsvariablen
wieder auf die vertrauten $x_1,\dots,x_n$ umbenannt.
\end{proof}
Besonders einfach sind die Fälle $k=0$, da dann keine Integral mit
unterer Grenze $t$ auftreten.
Der Parameter $t$ tritt in diesem Fall nur
in der Grenze des letzten Integrals auf:

\begin{satz}Für $n>0$ gilt
\begin{equation}
P_{n0}(x)=\frac{(x+t)^n}{n!}-\frac{(x+t)^{n-1}}{(n-1)!}
\end{equation}
\end{satz}
\begin{proof}[Beweis]
Mittels vollständiger Induktion.
Für $n=1$ gilt
\[
P_{10}(x)=\int_1^{x+t}dx_1=[x_1]_1^{x+t}=(x+t)-1=\frac{(x+t)^1}{1!}-\frac{(x+t)^0}{0!}.
\]
Nehmen wir an, die Behauptung wäre für $n-1$ bereits bewiesen, dann
berechnen wir $P_{n0}(x)$ wie folgt:
\begin{align*}
P_{n0}(x)
&=
\int_n^{x+t}dx_n\int_{n-1}^{x_n}dx_{n-1}\dotsi\int_1^{x_2}dx_1
\\
&=
\int_n^{x+t}dx_n\,P_{n-1,0}(x_n-t)
\\
&=
\int_n^{x+t}\frac{x_n^{n-1}}{(n-1)!}-\frac{x_n^{n-2}}{(n-2)!}\,dx_n
\\
&=
\biggl[\frac{x_n^n}{n!}-\frac{x_n^{n-1}}{(n-1)!}\biggr]_n^{x+t}
\\
&=
\frac{(x+t)^n}{n!}-\frac{(x+t)^{n-1}}{(n-1)!}
-\frac{n^n}{n!}+\frac{n^{n-1}}{(n-1)!}
\\
&=
\frac{(x+t)^n}{n!}-\frac{(x+t)^{n-1}}{(n-1)!}
\end{align*}
wegen
\[
\frac{n^n}{n!}=\frac{n^{n-1}n}{(n-1)! n}=\frac{n^{n-1}}{(n-1)!},
\]
womit der Satz bewiesen ist.
\end{proof}

Der Fall $P_{n0}(x)$ bezieht sich auf kleine Werte von $t$.
Sobald 
$t\ge 1$ ist, werden Integrale mit $t$ als unterer Grenze auftreten,
diese lassen sich jedoch rekursiv aus den bereits berechneten
Integralen bestimmen.

\begin{satz}\label{kn-rekursion}
Für $1\le k\le n$ gilt
\begin{equation}
P_{nk}(x)-P_{n,k-1}(x)
=
\frac{(k-t)^k}{k!}P_{n-k,0}(x-k).
\end{equation}
\end{satz}

\begin{proof}[Beweis]
Wir schreiben
\[
\Delta_{nk}(x)= P_{nk}(x)-P_{n,k-1}(x)
\]
dann gilt
\begin{align*}
\Delta_{nk}(x)
&=
\int_n^{x+t}dx_n\int_{n-1}^{x_n}dx_{n-1}\dotsi\int_{k+1}^{x_{k+2}}dx_{k+1}
\int^{x_{k+1}}_{\color{red}t} dx_k\dotsi\int_t^{x_2}dx_1
\\
&\phantom{=}\quad-\int_n^{x+t} dx_n\int_{n-1}^{x_n}dx_{n-1}\dotsi\int_{k+1}^{x_{k+2}}dx_{k+1}
\int^{x_{k+1}}_{\color{red}k} dx_k\dotsi\int_t^{x_2}dx_1
\\
&=
\int_n^{x+t}dx_n\int_{n-1}^{x_n}dx_{n-1}\dotsi\int_{k+1}^{x_{k+2}}dx_{k+1}
\int_{{\color{red}t\mathstrut}}^{\color{red}k}dx_k\dotsi\int_t^{x_2}dx_1.
\end{align*}
Da das Integral über $x_k$ feste Grenzen $t$ und $k$ hat, reduziert es
sich auf eine Konstante $c_k$, welche wir weiter unten berechnen werden.
Die verbleibenden Integrale sind
\begin{align*}
\Delta_{nk}(x)
&=
c_k\int_n^{x+t}dx_n\int_{n-1}^{x_n}dx_{n-1}\dotsi\int_{k+1}^{x_{k+2}}dx_{k+1}
\\
&=
c_k\int_{n-k}^{x-k+t}dx_{n-k}\int_{n-k-1}^{x_{n-k}}dx_{n-k-1}\dotsi\int_1^{x_2}dx_1
\\
&=c_kP_{n-k,0}(x-k),
\end{align*}
dabei haben wir die selbe Variablentransformation vorgenommen wie im
Beweis von Satz \ref{kn-variablentransformation}, und ausserdem haben
wir die Integrationsvariablen von $x_n,\dots,x_{k+1}$ in $x_{n-k},\dots,x_1$
umbenannt.

Wir müssen uns noch um die Konstante $c_k$ kümmern.
Aus der Definition folgt mit Hilfe von Satz \ref{kn-elementarvolumen}
\begin{equation}
c_k = \int_{t}^{k}dx_k\dotsi\int_t^{x_2}dx_1=P_{kk}(k-t)=\frac{(k-t)^k}{k!},
\end{equation}
die Behauptung.
\end{proof}
Mit den eben bewiesenen Sätzen lässt sich jetzt jedes beliebige 
$P_{nk}(x)$ berechnen.

\begin{satz}\label{pnk-allgemein}
\begin{equation}
P_{nk}(x)
=
\sum_{0\le r\le k}\frac{(r-t)^r}{r!}
\cdot
\frac{(x-r+t)^{n-r-1}}{(n-r)!}
(x-n+t)
\end{equation}
\end{satz}
\begin{proof}[Beweis]
Um $P_{nk}(x)$ zu berechnen schreiben wir
\begin{align*}
P_{nk}(x)
&=
\sum_{1\le r\le k}(P_{nr}(x)-P_{n,r-1}(x))+P_{n0}(x)
\\
&=
\sum_{1\le r\le k}\frac{(r-t)^r}{r!}P_{n-r,0}(x-r)+P_{n0}(x)
\\
&=
\sum_{1\le r\le k}\frac{(r-t)^r}{r!}\biggl(
\frac{(x-r+t)^{n-r}}{(n-r)!}-
\frac{(x-r+t)^{n-r-1}}{(n-r-1)!}\biggr)
\\
&
+\frac{(x+t)^{n}}{n!}- \frac{(x+t)^{n-1}}{(n-1)!}
\\
&=
\sum_{0\le r\le k}\frac{(r-t)^r}{r!}\biggl(
\frac{(x-r+t)^{n-r}}{(n-r)!}-
\frac{(x-r+t)^{n-r-1}}{(n-r-1)!}\biggr)
\\
&=
\sum_{0\le r\le k}\frac{(r-t)^r}{r!}
\cdot
\frac{(x-r+t)^{n-r-1}}{(n-r-1)!}
\biggl( \frac{x-r+t}{n-r}-1 \biggr)
\\
&=
\sum_{0\le r\le k}\frac{(r-t)^r}{r!}
\cdot
\frac{(x-r+t)^{n-r-1}}{(n-r-1)!}
\frac{x-n+t}{n-r}
\\
&=
\sum_{0\le r\le k}\frac{(r-t)^r}{r!}
\cdot
\frac{(x-r+t)^{n-r-1}}{(n-r)!}
(x-n+t).
\end{align*}
\end{proof}
Die Verteilungsfunktion für $K_n^+$ findet man jetzt, indem man $x=n$
einsetzt:

\begin{satz}\label{kn-verteilung}
Für die Verteilungsfunktion von $K_n^+$ gilt
\begin{equation}
P(K_n^+\le t/\sqrt{n})=\frac{t}{n^n}\sum_{0\le r\le t}\binom{n}{r}(r-t)^r(t+n-r)^{n-r-1}
\end{equation}
\end{satz}
\begin{proof}[Beweis]
Die gesuchte Wahrscheinlichkeit ist $P_{n\lfloor t\rfloor}(n)/P_{nn}(n)$,
damit folgt die Behauptung aus Satz \ref{pnk-allgemein}
\end{proof}

\section{\texorpdfstring{$t$}{t}-Test} \label{section-t-test}
Bei der Herstellung eines Produktes sind Verbesserungen durchgeführt worden,
so dass sich ein charakteristischer Parameter des Produktes verbessert hat.
In der Qualtitätssicherung wird der Parameter durch Messungen in einer
Stichprobe ermittelt.
Die vor der Verbesserung gemessene Stichprobe
$X_1,\dots,X_n$
einer normalverteilten Zufallsvariable $X$ liefert den Mittelwert
$\bar X=\frac1n(X_1+\dots+X_n)$, nach
der Verbesserung liefert die Stichprobe $Y_1,\dots,Y_m$ der normalverteilten
Zufallsvariable $Y$ den Mittelwert
$\bar Y=\frac1m(Y_1+\dots+Y_m)$.
Selbst wenn $\bar X$ und $\bar Y$
verschieden sind: Kann man daraus wirklich schon schliessen, dass die
Verbesserung eine Änderung des Parameters gebracht hat, oder könnte
das nicht auch einfach nur eine zufällige Schwankung sein?

Offensichtlich geht es hier darum die Hypothese zu testen, dass sich
durch die Verbesserung nichts geändert hat.
Das Beobachtungsmaterial
soll dann entscheiden, ob die Hypothese noch haltbar ist.
Ist sie es
auf dem Niveau $\alpha$ nicht,
dürften wir schliessen, dass die Verbesserung tatsächlich eine Änderung
des Parameters gebracht hat, und wir würden uns höchstens mit 
Wahrscheinlichkeit $\alpha$ irren.

Wir nehmen an, dass die Streuung der Messwerte der Stichprobe durch die
Veränderung des Produktionsprozesses nicht verändert wurde.
Aber durch
die zusätzlichen Messungen dürfte die Varianz genauer bekannt sein.
Daher bildet man aus der Stichprobenvarianz $S_X^2$ und $S_Y^2$
die gepoolte Stichprobenvarianz
\begin{equation}
S_p^2=\frac{(n-1)S_X^2+(m-1)S_Y^2}{m+n-2},
\label{pooled-variance}
\end{equation}
sie ist ein verbessertes Mass für die Messfehler.

Man möchte nun die die Hypothese testen, dass $\delta=\mu-\nu=0$, wobei
$\mu=E(X)$ und $\nu=E(Y)$ ist.
Dazu braucht man offensichtliche einen
Schätzer für $\delta$, und nicht wirklich überraschend ist
$\hat\delta=\bar X-\bar Y$ ein erwartungstreuer Schätzer für $\delta$.
Man kann weiter zeigen, dass $S_p^2$ ein erwartungstreuer Schätzer
für die gemeinsame Varianz von $X$ und $Y$ ist.
Unter der Annahme der
Hypothese ist $\hat\delta$ eine normalverteilte Zufallsvariable
mit Varianz $\sigma^2(1/n+1/m)$ ist.
$S_p^2(1/n+1/m)$ ist $\chi^2$-verteilt
mit $(m+n-2)$ Freiheitsgraden.
Somit ist 
\begin{equation}
T=\frac{\hat\delta}{S_p\sqrt{1/n+1/m}}
=\frac{\bar X-\bar Y}{\sqrt{(n-1)S_X^2+(m-1)S_Y^2}}\sqrt{\frac{nm(n+m-2)}{n+m}}
\label{t-test-ausdruck}
\end{equation}
$t$-verteilt mit $m+n-2$ Freiheitsgraden.

Der $t$-Test auf Gleichheit der Erwartungswerte spielt sich also wie folgt
ab.
Aus den beiden Stichproben $X_1,\dots,X_n$ und $Y_1,\dots,Y_m$ wird
der Ausdruck $T$ (\ref{t-test-ausdruck}) gebildet.
Übersteigt $T$ den $t$-Wert für die $1-\alpha$-Quantile der $t$-Verteilung
mit $n+m-2$ Freiheitsgraden, wird die Hypothese $\mu=\nu$ verworfen.

\subsubsection{War der Juli 2003 wirklich besonders warm?} \label{julitemperaturen}
Aus den Messdaten der Wetterstation in Altendorf soll entschieden werden,
ob der Juli 2003 wirklich signifikant wärmer war als dr Juli 2004.
Dazu werden die im Laufe des Monats gemessenen Temperaturwerte
ermittelt:

\begin{center}
\begin{tabular}{|r|r|r|r|r|}
\hline
Jahr&$T$&$T^2$&$\operatorname{var}(T)$&$n$\\
\hline
2001&20.351&440.855&26.659&44540\\
2002&19.906&415.302&19.051&44086\\
2003&21.638&493.230&25.023&44473\\
2004&19.582&406.576&23.102&44589\\
2005&19.861&419.796&25.319&44637\\
2006&24.009&603.028&26.573&44623\\
\hline
\end{tabular}
\end{center}

Um einen signifikanten Unterschied der Temperatur nachweisen zu können,
müssen wir für jedes Paar von Jahren den Ausdruck (\ref{t-test-ausdruck})
berechnen, wir erhalten dabei folgende Tabelle:

\begin{center}
\begin{tabular}{|r|rrrrrr|}
\hline
    &    2001&    2002&    2003&    2004&    2005&     2006\\
\hline
2001&        &  13.849& -37.767&  23.013&  14.352& -105.860\\
2002& -13.849&        & -54.881&  10.505&   1.422& -127.863\\
2003&  37.767&  54.881&        &  62.542&  52.864&  -69.666\\
2004& -23.013& -10.505& -62.542&        &  -8.469& -132.656\\
2005& -14.352&  -1.422& -52.864&   8.469&        & -121.646\\
2006& 105.860& 127.863&  69.666& 132.656& 121.646&         \\
\hline
\end{tabular}
\end{center}

Da die Zahl der Freiheitsgrade sehr gross ist, müssen wir
uns nicht um die genaue Anzahl kümmern, sondern können stattdessen
die Grenzwerte für unendlich viele Freiheitsgrade verwenden.
Auf
dem Niveau $\alpha=0.01$ finden wir $t_{\infty}^{0.005}=2.576$.
Da alle Einträge in der Tabelle bis auf das Paar $(2002,2005)$
betragsmässig grösser sind, können wir auf dem Niveau $\alpha=0.01$
schliessen, dass die Juli-Durchschnittstemperatur zwischen
2001 und 2006 verschieden war.
Nur für die Jahre 2002 und 2005
können wir den Unterschied nicht auf dem Niveau 0.01 sicherstellen.
Da jedoch $t_{\infty}^{0.1}=1.282$ könnten wir auf dem Niveau
$0.3$ beweisen, dass die Juli-Durchschnittstemperatur in den
Jahren 2002 und 2005 verschieden war, wir würden dabei aber 
mit Wahrscheinlichkeit $0.3$ einen Fehler erster Art machen.

\section{\texorpdfstring{$F$}{F}-Test}
In der Diskussion des $t$-Tests haben wir für die dort beschriebene
Situation postuliert, dass die Varianzen von $X$ und $Y$ gleich sind.
Wie könnte man dies testen? Gibt es einen Test, der die Hypothese
$\operatorname{var}(X)=\operatorname{var}(Y)$ testet? 

Bei der Entwicklung eines Messgerätes wurde eine Verbesserung angebracht,
die angeblich die Messgenauigkeit erhöht.
Der Messung einer Referenzgrösse
mit dem ursprünglichen Messgerät entspreche die normalverteilte
Zufallsvariable $X$ mit Varianz $\sigma_X^2$, die Messung mit dem verbesserten
Messgerät entspreche der ebenfalls normalverteilten Zufallsvariablen
$Y$ mit Varianz $\sigma_Y^2$.
Ob die möglicherweise kostspielige 
Verbesserung tatsächlich etwas gebracht hat, könnte ein Test aufdecken,
der die Hypothese $\sigma_X=\sigma_Y$ testet.

Da die Varianz positiv ist, bietet sich $\gamma=\sigma_X^2/\sigma_Y^2$ als
Testgrösse an.
Der Test für $\sigma_X=\sigma_Y$ ist dann einfach nur
ein Test auf $\gamma=1$.

Offensichtlich brauchen wir einen Schätzer für $\gamma$, ein solcher ist
schnell gefunden:
\begin{equation}
\hat\gamma=\frac{S_X^2}{S_Y^2}.
\label{ftest-gamma-schaetzer}
\end{equation}
Um einen Test zu konstruieren, muss jetzt nur noch die Verteilung von
$\hat\gamma$ ermittelt werden.

Die Grössen $(n-1)S_X^2/\sigma_X^2$ und $(m-1)S_Y^2/\sigma_Y^2$ sind jeweils
$\chi^2$-verteilte Zufallsvariablen mit $n-1$ bzw.~$m-1$ Freiheitsgraden.
Per Definition ist der Quotient zweier $\chi^2$-verteilter Zufallsvariablen
$F$-verteilt:
\begin{definition}
Sind $X$ und $Y$ $\chi^2$-verteilte Zufallsvariable mit $n$ bzw.~$m$
Freiheitsgraden, dann heisst die Verteilung von $X/Y$ eine $F$-Verteilung
mit $n,m$-Freiheitsgraden, auch $F_{n,m}$ geschrieben.
\end{definition}
Der Quotient der genannten Grössen ist also
\begin{equation}
\frac{(n-1)S_X^2\sigma_Y^2}{(m-1)S_Y^2\sigma_X^2}
=\frac{n-1}{m-1}\frac{\hat\gamma}{\gamma}
\end{equation}
$F_{n-1,m-1}$-verteilt.

Unter der Hypothese ist $\sigma_X=\sigma_Y$, also $\gamma=1$.
Der Test auf dem Niveau $\alpha$ wird also wie folgt durchgeführt.
Zunächst berechnet man die Schätzung $\hat\gamma=S_X^2/S_Y^2$.
Dann
findet man in der Tabelle der Quantilen der $F$-Verteilung mit $n-1,m-1$
Freiheitsgraden den kritischen Wert $F_{n-1,m-1}^\alpha$.
Ist $\hat\gamma>F_{n-1,m-1}^\alpha$, wird die Hypothese verworfen,
d.~h.~man muss davon ausgehen, dass die beiden Zufallsvariablen verschiedene
Varianz haben.

\subsubsection{Sind die Temperaturvarianzen gleich?}
Im Abschnitt \ref{julitemperaturen} haben wir die
Juli-Durchschnittstemperaturen mit Hilfe des $t$-Tests untersucht.
Dabei haben wir nicht untersucht, ob die Varianzen, die für den
$t$-Test ja gleich sein müssen, auch wirklich gleich sein könnten.
Inzwischen haben wir gelernt, dass genau dies mit dem $F$-Test getestet
werden kann, und holen dies noch nach.

Zunächst berechnen wir die $F$-Werte aus der Tabelle der Temperatur-Varianzen
aus Abschnitt \ref{julitemperaturen}

\begin{center}
\begin{tabular}{|r|rrrrrr|}
\hline
    &  2001&  2002&  2003&  2004&  2005&  2006\\
\hline
2001&      & 1.414& 1.067& 1.153& 1.051& 1.001\\
2002& 0.707&      & 0.755& 0.815& 0.743& 0.708\\
2003& 0.937& 1.325&      & 1.080& 0.985& 0.939\\
2004& 0.868& 1.226& 0.926&      & 0.911& 0.869\\
2005& 0.952& 1.346& 1.016& 1.097&      & 0.953\\
2006& 0.999& 1.412& 1.066& 1.151& 1.049&      \\
\hline
\end{tabular}
\end{center}

Die Interpretation dieser Zahlen gestaltet sich etwas schwierig: Für derart
viele Freiheitsgrade lässt sich $F_{n,k}$ wegen der darin
vorkommenden $\Gamma$-Funktionen fast nicht berechnen, der kritische Wert
ist jedenfalls kleiner als $1.22$.
Dies bedeutet, dass in etwa fünf der
fünfzehn Paare die Hypothese verworfen werden muss, dass also die Varianzen
verschieden sind.
Damit ist natürlich auch die Gültigkeit der Schlussweise
beim Vergleich der Juli-Durchschnittstemperaturen in Frage gestellt.
