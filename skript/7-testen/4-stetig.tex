\section{Testen einer stetigen Verteilung}
\label{section-testen-stetiger-wkeitsverteilung}
\kopfrechts{Testen einer stetigen Verteilung}
Der im vorangehenden Abschnitt beschriebene $\chi^2$-Test kann prinzipbedingt
nur diskrete Wahrscheinlichkeitsverteilungen testen.
Will man eine stetige
Verteilung testen, muss man zunächst Klassen von Werten bilden,
deren Wahrscheinlichkeiten berechnen, und dann prüfen, ob die so
künstlich konstruierte diskrete Verteilung im $\chi^2$-Test Bestand hat.
Durch die Notwendigkeit der Klassenbildung führt man eine künstliche
Diskretisierung in das Problem ein.
Will man den Test verfeinern, muss man
die Diskretisierung ebenfalls ändern.

\subsection{Testprinzip}
Diese unbefriedigende Situation korrigiert der Kolmogoroff-Smirnov-Test,
kurz KS-Test.
Dieser vergleicht direkt die heuristische, d.~h.~aus den Beobachtungen gewonnene
Verteilungsfunktion mit der erwarteten Verteilungsfunktion.
Er legt fest,
wie die Unterschiede zwischen den zwei Funktionen gemessen werden sollen,
und legt fest, wann eine Abweichung zu gross ist.

Wir wollen testen, ob $n$ Beobachtungen $X_1,\dots,X_n$ sich mit
der Hypothese vertragen, dass die Zufallsvariablen $X_i$ die
Verteilungsfunktion $F$ haben.
Dazu wird aus den $n$ Beobachtungen
und die zugehörige empirische Verteilungsfunktion
\[
F_n\colon x\mapsto F_n(x)=\frac{|\{x_i|X_i\le x\}|}{n}
\]
gebildet.
Wenn die Zufallsvariable $X$ tatsächlich die
Verteilungsfunktion $F$ hat, wird man erwarten,
dass $F$ und $F_n$ sich nicht stark unterscheiden.
Insbesondere sollten die Grössen
\begin{align}
K_n^+
&=
\sqrt{n}\max_{-\infty<x<\infty} F_n(x)-F(x)
\\
K_n^-
&=
\sqrt{n}\max_{-\infty<x<\infty} F(x)-F_n(x)
\end{align}
nicht sehr gross werden.
Um den Test durchzuführen braucht man also
die Wahrscheinlichkeit, dass $K_n^{\pm}$ einen gewissen Wert übersteigt.
Dafür gibt es Tabellen, zum Beispiel findet man im Anhang \ref{chapter:tabellen}
die Tabelle \ref{KS-tabelle}, welche zu $n$ und einer Wahrscheinlichkeit
$p$ eine Grösse $t_{n,p}$ angibt, für die
\[
P(K_n^+\le t)=t_{n,p}.
\]

Für die Durchführung des Tests geht man wie folgt vor:
\begin{satz} Sei $X_1,\dots,X_n$ sei eine Stichprobe einer Zufallsvariable $X$.
Der Kolmogorov-Smirnov-Test auf dem Niveau $\alpha$ für die Hypothese,
dass $X$ die Verteilungsfunktion $F$ hat wird wie folgt durchgeführt:
\begin{enumerate}
\item Berechne
\[
K_n^+ = \sqrt{n}\max_{-\infty<x<\infty} (F_n(x)-F(x))
\]
\item Finde $t_{n,1-\alpha}$ in der Tabelle \ref{KS-tabelle}
\item Falls
\[
K_n^+>t_{n,1-\alpha}
\]
verwerfe die Hypothese, dass $X$ die Verteilungsfunktion $F$ hat.
\end{enumerate}
\end{satz}

\subsection{Berechnung von \texorpdfstring{$K_n^{\pm}$}{Kn-plus-minus}}
Das Maximum der Differenz $F_n(x)-F(x)$ tritt zwangsläufig an einer
Sprungstelle von $F_n$ auf, also bei einem der Werte der Stichprobe
$X_1,\dots,X_n$.
Durch sortieren können wir erreichen, dass die
$X_i$  in aufsteigender Reihenfolge angeordnet sind, also $X_i\le X_j$
für $i\le j$.
Dann ist $F_n(X_j)=\frac{j}{n}$, und
die  $K_n^{\pm}$ können mit Hilfe der einfacheren Formeln
\begin{align}
K_n^+
&=
\sqrt{n}\max_{1\le j\le n}\biggl(\frac{j}{n}-F(X_j)\biggr)
\label{knp-berechnungs-formel}
\\
K_n^-
&=
\sqrt{n}\max_{1\le j\le n}\biggl(F(X_j)-\frac{j-1}n\biggr)
\end{align}
berechnet werden.
In dieser Form ist die Berechnung der
Verteilung von $K_n^{\pm}$ leichter möglich.

\subsection{Reduktion auf Gleichverteilung}
Das Problem wäre etwas handlicher, wenn die Verteilungsfunktion $F$
die Verteilungsfunktion einer Gleichverteilung wäre.
Der folgende Satz ermöglicht, eine
beliebige Zufallsvariable mit stetiger Verteilungsfunktion in eine
gleichverteilte Zufallsvariable zu transformieren.
\begin{satz}\label{reduktion-auf-gleichverteilung}
Sei $X$ eine Zufallsvariable mit stetiger Verteilungsfunktion $F$, dann
ist $F(X)$ eine auf $[0,1]$ gleichverteilte Zufallsvariable.
\end{satz}
\begin{proof}[Beweis]
Um die Verteilungsfunktion in $F(X)$ zu bestimmen, müssen wir $P(F(X)\le y)$
für ein beliebiges $y\in[0,1]$ berechnen.
Da $F$ stetig ist, gibt es ein
grösstes $x$, für welches $F(x)= y$, zum Beispiel
$x=\max\{\xi|F(\xi)= y\}$.
Mit diesem $x$ folgt
\[
P(F(X)\le y)=P(X\le x)=F(x)=y.
\]
\end{proof}
Aus dem Satz folgt, dass wir die Zahlen $Y_j=nF(X_j)$ darauf prüfen wollen,
ob sie im Intervall $[0,n]$ gleichverteilt sind.

\subsection{Berechnung der Verteilung von \texorpdfstring{$K_n^{\pm}$}{Kn-plus-minus}}
Die Formel (\ref{knp-berechnungs-formel})
für $K_n^+$ kann mit Hilfe der $Y_j$ wie folgt 
geschrieben werden
\begin{equation}
K_n^+=\frac1{\sqrt{n}}\max(1-Y_1, 2-Y_2,\dots,n-Y_n).
\end{equation}
Die Wahrscheinlichkeit, dass $K_n^+\le t/\sqrt{n}$ ist also die
Wahrscheinlichkeit, dass $j-Y_j\le t$ für alle $j$, oder
$Y_j\ge j-t$ für $1\le j\le n$.
Da die $Y_j$ gleichverteilt sind,
kann man diese Wahrscheinlichkeit berechnen.

Wir berechnen zunächst die Wahrscheinlichkeit, dass die Bedingung
für $Y_n$ erfüllt ist unter der Annahme, dass die übrigen
Bedingungen erfüllt sind.
$Y_n$ muss im Intervall $[0,n]$
liegen, darf aber auch nicht kleiner als $n-t$ sein.
Setzt man $\alpha_j=\max(j-t,0)$, dann heisst das, dass $Y_n$ im
Intervall $[\alpha_n, n]$ liegen muss.
Ausserdem ist $Y_n$ gleichverteilt,
somit ist die Wahrscheinlichkeit, dass alle Bedingungen für $Y_n$ erfüllt
sind
\begin{equation}
P(\alpha_n\le Y_n\le n)=\frac{\int_{\alpha_n}^n dy_n}{\int_0^ndy_n}
\label{kpn-erstes-integral}
\end{equation}
Der Nenner dient dazu, die Wahrscheinlichkeit auf $1$ zu normieren.

Nun betrachten wir die analogen Bedingungen für $Y_{n-1}$.
Offensichtlich kann $Y_{n-1}$ zwischen $\alpha_{n-1}$ und $Y_n$ variieren.
Es ergibt sich
als Wahrscheinlichkeit, dafür, dass die Bedingungen an $Y_{n-1}$ erfüllt
sind, der Ausdruck
\begin{equation}
\frac{\int_{\alpha_{n-1}}^{y_n}dy_{n-1}}{\int_0^{y_n}dy_{n-1}}.
\label{kpn-zweites-integral}
\end{equation}
In (\ref{kpn-erstes-integral}) hatten wir angenommen,
dass die Bedingungen für $Y_{n-1}$
bereits erfüllt sind, also die Wahrscheinlichkeit für deren zutreffen $1$
ist. Inzwischen haben wir gelernt, dass diese Wahrscheinlichkeit in
Wahrheit eher wie (\ref{kpn-zweites-integral}) aussieht.
Zusammen finden wir
\begin{equation}
P(\alpha_n\le Y_n\le n\wedge \alpha_{n-1}\le Y_{n-1}\le Y_n)
=
\frac{\int_{\alpha_n}^n dy_n\int_{\alpha_{n-1}}^{y_n}dy_{n-1}}{\int_0^ndy_n\int_0^{y_n}dy_{n-1}}
\label{kpn-zwei-integrale}
\end{equation}
In diesem Sinne können schrittweise die Bedingungen an $Y_{n-2},\dots,Y_1$
erfüllt werden, wir erhalten
\begin{equation}
\frac{\int_{\alpha_n}^n dy_n\int_{\alpha_{n-1}}^{y_n}dy_{n-1}\dotsi\int_{\alpha_1}^{y_2}dy_1}{\int_0^ndy_n\int_0^{y_n}dy_{n-1}\dotsi\int_0^{y_2}dy_1}
\label{kpn-alle-integrale}
\end{equation}
Diese Integrale sind nicht ganz selbstverständlich auszurechnen, weshalb
wir dies hier schrittweise durchführen.

\begin{satz}
\label{kn-elementarvolumen}
Es gilt
\begin{equation}
\int_0^xdy_n\int_0^{y_n}dy_{n-1}\dotsi\int_0^{y_2}dy_1=\frac{x^n}{n!}
\end{equation}
\end{satz}
\begin{proof}[Beweis] Wir führen den Beweis mit vollständiger
Induktion.
Für $n=1$ muss das Integral
\[
\int_0^xdy_1=[y_1]_0^x=x
\]
berechnet werden, was offensichtlich mit $\frac{x^1}{1!}=x$
übereinstimmt.

Nehmen wir an, dass die Formel für $n-1$ stimmt, dann können wir den
Fall $n$ wie folgt verifizieren:
\[
\int_0^xdy_n\int_0^{y_n}dy_{n-1}\dotsi\int_0^{y_2}dy_1
=
\int_0^x\frac{y_n^{n-1}}{(n-1)!}\,dy_n
=
\left[\frac{y_n^n}{n!}\right]_0^x=\frac{x^n}{n!}
\]
womit die Behauptung bewiesen ist.
\end{proof}
Der Nenner in (\ref{kpn-alle-integrale}) ist also $\frac{n^n}{n!}$,
dabei ist $n^n$ das
Volumen eines $n$-dimensionalen Würfels mit Kantenlänge $n$, dies wird
durch $n!$ geteilt, weil die $Y_j$ solange vertauscht werden müssen, bis
sie aufsteigend geordnet sind.
Nur eine der $n!$ Permutationen der $Y_j$
erfüllt diese Bedingung.

Wir kümmern uns nun um den Zähler von (\ref{kpn-alle-integrale}).
Die unteren Grenzen der Integrale sind $\alpha_j=\max(j-t,0)$.
Bei gegebenem $t$ wird $\alpha_j=j-t$
sein für genügend grosse $j$, für alle anderen wird $\alpha_j=0$ sein.
Es gibt also eine Zahl $k$ so dass der Zähler von (\ref{kpn-alle-integrale})
die Form
\[
P_{nk}(x)
=
\int_{n-t}^xdx_n\int_{n-1-t}^{x_n}dx_{n-1}\dotsi\int_{k+1-t}^{x_{k+2}}dx_{k+1}\int_0^{x_{k+1}}dx_k\dotsi\int_0^{x_2}dx_1
\]
hat, es ist dies das grösste $k$, für welches $k-t\le 0$ ist, also
$k=\lfloor t\rfloor$.
Der Zähler von (\ref{kpn-alle-integrale})
ist also $P_{n\lfloor t\rfloor}(n)$.
Mit Hilfe der folgenden Sätze berechnen wir $P_{nk}(x)$ schrittweise.

Wir halten zunächst noch den Spezialfall
\begin{equation}
P_{nn}(x)=\frac{x^n}{n!}
\label{spezialfall-pnn}
\end{equation}
fest, der unmittelbar aus dem Satz \ref{kn-elementarvolumen} folgt.

\begin{satz}
\label{kn-variablentransformation}
Es gilt
\begin{equation}
P_{nk}(x)=\int_{n}^{x+t}dx_n\int_{n-1}^{x_n}dx_{n-1}\dotsi\int_{k+1}^{x_{k+2}}dx_{k+1}\int_t^{x_{k+1}}dx_k\dotsi\int_t^{x_2}dx_1
\label{kpn-variablen-transformation}
\end{equation}
\end{satz}
\begin{proof}[Beweis]
Wir substituieren $x_i+t=\xi_i$.
Dazu müssen zu den Intervallgrenzen
$t$ hinzuaddiert werden:
\begin{align}
P_{nk}(x)
&=
\int_{n-t}^xdx_n\int_{n-1-t}^{x_n}dx_{n-1}\dotsi\int_{k+1-t}^{x_{k+2}}dx_{k+1}\int_0^{x_{k+1}}dx_k\dotsi\int_0^{x_2}dx_1
\nonumber\\
&=
\int_{n}^{x+t}d\xi_n\int_{n-1}^{x_n+t}d\xi_{n-1}\dotsi\int_{k+1}^{x_{k+2}+t}dx_{k+1}\int_t^{x_{k+1}+t}d\xi_k\dotsi\int_t^{x_2+t}d\xi_1
\nonumber\\
&=
\int_{n}^{x+t}d\xi_n\int_{n-1}^{\xi_n}d\xi_{n-1}\dotsi\int_{k+1}^{\xi_{k+2}}dx_{k+1}\int_t^{\xi_{k+1}}d\xi_k\dotsi\int_t^{\xi_2}d\xi_1
\nonumber\\
&=
\int_{n}^{x+t}dx_n\int_{n-1}^{x_n}dx_{n-1}\dotsi\int_{k+1}^{x_{k+2}}dx_{k+1}\int_t^{x_{k+1}}dx_k\dotsi\int_t^{x_2}dx_1
\label{kpn-umbenennung}
\end{align}
Im letzten Schritt (\ref{kpn-umbenennung}) haben wir die Integrationsvariablen
wieder auf die vertrauten $x_1,\dots,x_n$ umbenannt.
\end{proof}
Besonders einfach sind die Fälle $k=0$, da dann keine Integral mit
unterer Grenze $t$ auftreten.
Der Parameter $t$ tritt in diesem Fall nur
in der Grenze des letzten Integrals auf:

\begin{satz}Für $n>0$ gilt
\begin{equation}
P_{n0}(x)=\frac{(x+t)^n}{n!}-\frac{(x+t)^{n-1}}{(n-1)!}
\end{equation}
\end{satz}
\begin{proof}[Beweis]
Mittels vollständiger Induktion.
Für $n=1$ gilt
\[
P_{10}(x)=\int_1^{x+t}dx_1=[x_1]_1^{x+t}=(x+t)-1=\frac{(x+t)^1}{1!}-\frac{(x+t)^0}{0!}.
\]
Nehmen wir an, die Behauptung wäre für $n-1$ bereits bewiesen, dann
berechnen wir $P_{n0}(x)$ wie folgt:
\begin{align*}
P_{n0}(x)
&=
\int_n^{x+t}dx_n\int_{n-1}^{x_n}dx_{n-1}\dotsi\int_1^{x_2}dx_1
\\
&=
\int_n^{x+t}dx_n\,P_{n-1,0}(x_n-t)
\\
&=
\int_n^{x+t}\frac{x_n^{n-1}}{(n-1)!}-\frac{x_n^{n-2}}{(n-2)!}\,dx_n
\\
&=
\biggl[\frac{x_n^n}{n!}-\frac{x_n^{n-1}}{(n-1)!}\biggr]_n^{x+t}
\\
&=
\frac{(x+t)^n}{n!}-\frac{(x+t)^{n-1}}{(n-1)!}
-\frac{n^n}{n!}+\frac{n^{n-1}}{(n-1)!}
\\
&=
\frac{(x+t)^n}{n!}-\frac{(x+t)^{n-1}}{(n-1)!}
\end{align*}
wegen
\[
\frac{n^n}{n!}=\frac{n^{n-1}n}{(n-1)! n}=\frac{n^{n-1}}{(n-1)!},
\]
womit der Satz bewiesen ist.
\end{proof}

Der Fall $P_{n0}(x)$ bezieht sich auf kleine Werte von $t$.
Sobald 
$t\ge 1$ ist, werden Integrale mit $t$ als unterer Grenze auftreten,
diese lassen sich jedoch rekursiv aus den bereits berechneten
Integralen bestimmen.

\begin{satz}\label{kn-rekursion}
Für $1\le k\le n$ gilt
\begin{equation}
P_{nk}(x)-P_{n,k-1}(x)
=
\frac{(k-t)^k}{k!}P_{n-k,0}(x-k).
\end{equation}
\end{satz}

\begin{proof}[Beweis]
Wir schreiben
\[
\Delta_{nk}(x)= P_{nk}(x)-P_{n,k-1}(x)
\]
dann gilt
\begin{align*}
\Delta_{nk}(x)
&=
\int_n^{x+t}dx_n\int_{n-1}^{x_n}dx_{n-1}\dotsi\int_{k+1}^{x_{k+2}}dx_{k+1}
\int^{x_{k+1}}_{\color{red}t} dx_k\dotsi\int_t^{x_2}dx_1
\\
&\phantom{=}\quad-\int_n^{x+t} dx_n\int_{n-1}^{x_n}dx_{n-1}\dotsi\int_{k+1}^{x_{k+2}}dx_{k+1}
\int^{x_{k+1}}_{\color{red}k} dx_k\dotsi\int_t^{x_2}dx_1
\\
&=
\int_n^{x+t}dx_n\int_{n-1}^{x_n}dx_{n-1}\dotsi\int_{k+1}^{x_{k+2}}dx_{k+1}
\int_{{\color{red}t\mathstrut}}^{\color{red}k}dx_k\dotsi\int_t^{x_2}dx_1.
\end{align*}
Da das Integral über $x_k$ feste Grenzen $t$ und $k$ hat, reduziert es
sich auf eine Konstante $c_k$, welche wir weiter unten berechnen werden.
Die verbleibenden Integrale sind
\begin{align*}
\Delta_{nk}(x)
&=
c_k\int_n^{x+t}dx_n\int_{n-1}^{x_n}dx_{n-1}\dotsi\int_{k+1}^{x_{k+2}}dx_{k+1}
\\
&=
c_k\int_{n-k}^{x-k+t}dx_{n-k}\int_{n-k-1}^{x_{n-k}}dx_{n-k-1}\dotsi\int_1^{x_2}dx_1
\\
&=c_kP_{n-k,0}(x-k),
\end{align*}
dabei haben wir die selbe Variablentransformation vorgenommen wie im
Beweis von Satz \ref{kn-variablentransformation}, und ausserdem haben
wir die Integrationsvariablen von $x_n,\dots,x_{k+1}$ in $x_{n-k},\dots,x_1$
umbenannt.

Wir müssen uns noch um die Konstante $c_k$ kümmern.
Aus der Definition folgt mit Hilfe von Satz \ref{kn-elementarvolumen}
\begin{equation}
c_k = \int_{t}^{k}dx_k\dotsi\int_t^{x_2}dx_1=P_{kk}(k-t)=\frac{(k-t)^k}{k!},
\end{equation}
die Behauptung.
\end{proof}
Mit den eben bewiesenen Sätzen lässt sich jetzt jedes beliebige 
$P_{nk}(x)$ berechnen.

\begin{satz}\label{pnk-allgemein}
\begin{equation}
P_{nk}(x)
=
\sum_{0\le r\le k}\frac{(r-t)^r}{r!}
\cdot
\frac{(x-r+t)^{n-r-1}}{(n-r)!}
(x-n+t)
\end{equation}
\end{satz}
\begin{proof}[Beweis]
Um $P_{nk}(x)$ zu berechnen schreiben wir
\begin{align*}
P_{nk}(x)
&=
\sum_{1\le r\le k}(P_{nr}(x)-P_{n,r-1}(x))+P_{n0}(x)
\\
&=
\sum_{1\le r\le k}\frac{(r-t)^r}{r!}P_{n-r,0}(x-r)+P_{n0}(x)
\\
&=
\sum_{1\le r\le k}\frac{(r-t)^r}{r!}\biggl(
\frac{(x-r+t)^{n-r}}{(n-r)!}-
\frac{(x-r+t)^{n-r-1}}{(n-r-1)!}\biggr)
\\
&
+\frac{(x+t)^{n}}{n!}- \frac{(x+t)^{n-1}}{(n-1)!}
\\
&=
\sum_{0\le r\le k}\frac{(r-t)^r}{r!}\biggl(
\frac{(x-r+t)^{n-r}}{(n-r)!}-
\frac{(x-r+t)^{n-r-1}}{(n-r-1)!}\biggr)
\\
&=
\sum_{0\le r\le k}\frac{(r-t)^r}{r!}
\cdot
\frac{(x-r+t)^{n-r-1}}{(n-r-1)!}
\biggl( \frac{x-r+t}{n-r}-1 \biggr)
\\
&=
\sum_{0\le r\le k}\frac{(r-t)^r}{r!}
\cdot
\frac{(x-r+t)^{n-r-1}}{(n-r-1)!}
\frac{x-n+t}{n-r}
\\
&=
\sum_{0\le r\le k}\frac{(r-t)^r}{r!}
\cdot
\frac{(x-r+t)^{n-r-1}}{(n-r)!}
(x-n+t).
\end{align*}
\end{proof}
Die Verteilungsfunktion für $K_n^+$ findet man jetzt, indem man $x=n$
einsetzt:

\begin{satz}\label{kn-verteilung}
Für die Verteilungsfunktion von $K_n^+$ gilt
\begin{equation}
P(K_n^+\le t/\sqrt{n})=\frac{t}{n^n}\sum_{0\le r\le t}\binom{n}{r}(r-t)^r(t+n-r)^{n-r-1}
\end{equation}
\end{satz}
\begin{proof}[Beweis]
Die gesuchte Wahrscheinlichkeit ist $P_{n\lfloor t\rfloor}(n)/P_{nn}(n)$,
damit folgt die Behauptung aus Satz \ref{pnk-allgemein}
\end{proof}

