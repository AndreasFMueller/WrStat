\section{Testen einer diskreten Wahrscheinlichkeitsverteilung}
\label{section-testen-diskreter-wkeitsverteilung}
\kopfrechts{Testen einer diskreten Verteilung}
Karl Pearson hat im Jahre 1900 erkannt, dass sich der Test für die
Wahrscheinlichkeit eines Ereignisses auf einen Test für eine
diskrete Wahrscheinlichkeitsverteilung erweitern lässt.
Wir nehmen
dazu an, dass in einem Experiment $d+1$ Ausgänge möglich sind, und
bezeichnen diese mit $0,\dots,d$.
Das Experiment wird $n$ Mal
durchgeführt, wobei der Ausgang $i$ $n_i$ Mal beobachtet wurde.
Mit diesen Daten möchten wir die Hypothese testen, dass $p_i$ die
Wahrscheinlichkeit für den Ausgang $i$ ist.

In Analogie zum Test für eine Wahrscheinlichkeit bilden wird die
Diskrepanz:
\begin{definition}Die Grösse
\begin{equation}
D=\sum_{i=0}^d\frac{(n_i-np_i)^2}{np_i} \label{formel-diskrepanz}
\end{equation}
heisst Diskrepanz.
\end{definition}
Es gilt:
\begin{satz}[Pearson] Die Diskrepanz $D$ ist für genügend grosse
$n$ annähernd $\chi^2$-verteilt mit $d$ Freiheitsgraden.
\end{satz}
``Genügend gross'' heisst nach einer verbreiteten Faustregel: so
gross, dass $n_i>5\forall i$.
Damit ist anscheinend sichergestellt,
dass für die Zwecke des Tests die Annäherung an die Normalverteilung
genau genug ist.
\begin{proof}[Beweis]
Wie beim Fall einer Wahrscheinlichkeit können wir die Terme
\[
\delta_i=\frac{n_i-np_i}{\sqrt{np_i(1-p_i)}}
\]
als standardnormalverteilt ansehen.
Sie sind jedoch nicht
unabhängig, da $\sum_{i=0}^d n_i=n$ und $\sum_{i=0}^dp_i=1$
gilt.
Das Hauptproblem des Beweises besteht also darin zu zeigen,
dass die Diskrepanz die Quadrat-Summe von $d$ standardnormalverteilten
Zufallsvariablen ist, die sich aus den $\delta_i$ bilden lassen.
Dieses etwas technische Problem bringt uns keine neuen theoretischen
Einsichten, wir verzichten daher auf die Durchführung des Beweises.
\end{proof}

Mit diesem Satz lässt sich offensichtlich ein Test auf dem Niveau
$\alpha$ für jede
beliebige diskrete Verteilung bilden.
Man geht dazu wie folgt vor:
\begin{enumerate}
\item Bestimme $x$ so, dass die $F_{\chi_{d}^2}(x)=1-\alpha$
\item Berechne 
\[
D=\sum_{i=0}^d\frac{(n_i-np_i)^2}{np_i}
\]
\item Verwerfe die Hypothese, dass der Ausgang $i$ mit Wahrscheinlichkeit
$p_i$ eintritt, wenn $D>x$ gilt.
\end{enumerate}

In 100 Würfen mit einem Würfel wurden die Augenzahlen $1$ bis $6$ gezählt,
und dabei folgende Häufigkeiten ermittelt
\begin{center}
\begin{tabular}{|r|r|r|}
\hline
$i$&$n_i$&$(n_i-np_i)^2/np_i$\\
\hline
1&21&1.12671\\
2&18&0.10668\\
3&16&0.02666\\
4&17&0.00667\\
5&15&0.16665\\
6&13&0.80664\\
\hline
&&$D=2.24001$\\
\hline
\end{tabular}
\end{center}
Die Tabelle der Quantilen der $\chi^2$-Verteilung \ref{chi2-tabelle}
zeigt, dass für fünf Freiheitsgrade (sechs mögliche Versuchsausgänge)
selbst auf dem Niveau $\alpha=0.1$ die Diskrepanz mindestens $9.236$ sein
müsste.
Die vorliegenden experimentellen Daten geben also noch keinen
Anlass, daran zu zweifeln, dass jede Augenzahl mit der gleichen
Wahrscheinlichkeit $\frac16$ auftreten wird.

