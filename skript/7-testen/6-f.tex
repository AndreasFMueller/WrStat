\section{\texorpdfstring{$F$}{F}-Test}
\kopfrechts{$F$-Test}
In der Diskussion des $t$-Tests haben wir für die dort beschriebene
Situation postuliert, dass die Varianzen von $X$ und $Y$ gleich sind.
Wie könnte man dies testen? Gibt es einen Test, der die Hypothese
$\operatorname{var}(X)=\operatorname{var}(Y)$ testet? 

Bei der Entwicklung eines Messgerätes wurde eine Verbesserung angebracht,
die angeblich die Messgenauigkeit erhöht.
Der Messung einer Referenzgrösse
mit dem ursprünglichen Messgerät entspreche die normalverteilte
Zufallsvariable $X$ mit Varianz $\sigma_X^2$, die Messung mit dem verbesserten
Messgerät entspreche der ebenfalls normalverteilten Zufallsvariablen
$Y$ mit Varianz $\sigma_Y^2$.
Ob die möglicherweise kostspielige 
Verbesserung tatsächlich etwas gebracht hat, könnte ein Test aufdecken,
der die Hypothese $\sigma_X=\sigma_Y$ testet.

Da die Varianz positiv ist, bietet sich $\gamma=\sigma_X^2/\sigma_Y^2$ als
Testgrösse an.
Der Test für $\sigma_X=\sigma_Y$ ist dann einfach nur
ein Test auf $\gamma=1$.

Offensichtlich brauchen wir einen Schätzer für $\gamma$, ein solcher ist
schnell gefunden:
\begin{equation}
\hat\gamma=\frac{S_X^2}{S_Y^2}.
\label{ftest-gamma-schaetzer}
\end{equation}
Um einen Test zu konstruieren, muss jetzt nur noch die Verteilung von
$\hat\gamma$ ermittelt werden.

Die Grössen $(n-1)S_X^2/\sigma_X^2$ und $(m-1)S_Y^2/\sigma_Y^2$ sind jeweils
$\chi^2$-verteilte Zufallsvariablen mit $n-1$ bzw.~$m-1$ Freiheitsgraden.
Per Definition ist der Quotient zweier $\chi^2$-verteilter Zufallsvariablen
$F$-verteilt:
\begin{definition}
Sind $X$ und $Y$ $\chi^2$-verteilte Zufallsvariable mit $n$ bzw.~$m$
Freiheitsgraden, dann heisst die Verteilung von $X/Y$ eine $F$-Verteilung
mit $n,m$-Freiheitsgraden, auch $F_{n,m}$ geschrieben.
\end{definition}
Der Quotient der genannten Grössen ist also
\begin{equation}
\frac{(n-1)S_X^2\sigma_Y^2}{(m-1)S_Y^2\sigma_X^2}
=\frac{n-1}{m-1}\frac{\hat\gamma}{\gamma}
\end{equation}
$F_{n-1,m-1}$-verteilt.

Unter der Hypothese ist $\sigma_X=\sigma_Y$, also $\gamma=1$.
Der Test auf dem Niveau $\alpha$ wird also wie folgt durchgeführt.
Zunächst berechnet man die Schätzung $\hat\gamma=S_X^2/S_Y^2$.
Dann
findet man in der Tabelle der Quantilen der $F$-Verteilung mit $n-1,m-1$
Freiheitsgraden den kritischen Wert $F_{n-1,m-1}^\alpha$.
Ist $\hat\gamma>F_{n-1,m-1}^\alpha$, wird die Hypothese verworfen,
d.~h.~man muss davon ausgehen, dass die beiden Zufallsvariablen verschiedene
Varianz haben.

\subsubsection{Sind die Temperaturvarianzen gleich?}
Im Abschnitt \ref{julitemperaturen} haben wir die
Juli-Durchschnittstemperaturen mit Hilfe des $t$-Tests untersucht.
Dabei haben wir nicht untersucht, ob die Varianzen, die für den
$t$-Test ja gleich sein müssen, auch wirklich gleich sein könnten.
Inzwischen haben wir gelernt, dass genau dies mit dem $F$-Test getestet
werden kann, und holen dies noch nach.

Zunächst berechnen wir die $F$-Werte aus der Tabelle der Temperatur-Varianzen
aus Abschnitt \ref{julitemperaturen}

\begin{center}
\begin{tabular}{|r|rrrrrr|}
\hline
    &  2001&  2002&  2003&  2004&  2005&  2006\\
\hline
2001&      & 1.414& 1.067& 1.153& 1.051& 1.001\\
2002& 0.707&      & 0.755& 0.815& 0.743& 0.708\\
2003& 0.937& 1.325&      & 1.080& 0.985& 0.939\\
2004& 0.868& 1.226& 0.926&      & 0.911& 0.869\\
2005& 0.952& 1.346& 1.016& 1.097&      & 0.953\\
2006& 0.999& 1.412& 1.066& 1.151& 1.049&      \\
\hline
\end{tabular}
\end{center}

Die Interpretation dieser Zahlen gestaltet sich etwas schwierig: Für derart
viele Freiheitsgrade lässt sich $F_{n,k}$ wegen der darin
vorkommenden $\Gamma$-Funktionen fast nicht berechnen, der kritische Wert
ist jedenfalls kleiner als $1.22$.
Dies bedeutet, dass in etwa fünf der
fünfzehn Paare die Hypothese verworfen werden muss, dass also die Varianzen
verschieden sind.
Damit ist natürlich auch die Gültigkeit der Schlussweise
beim Vergleich der Juli-Durchschnittstemperaturen in Frage gestellt.


