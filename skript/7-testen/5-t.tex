\section{\texorpdfstring{$t$}{t}-Test} \label{section-t-test}
Bei der Herstellung eines Produktes sind Verbesserungen durchgeführt worden,
so dass sich ein charakteristischer Parameter des Produktes verbessert hat.
In der Qualtitätssicherung wird der Parameter durch Messungen in einer
Stichprobe ermittelt.
Die vor der Verbesserung gemessene Stichprobe
$X_1,\dots,X_n$
einer normalverteilten Zufallsvariable $X$ liefert den Mittelwert
$\bar X=\frac1n(X_1+\dots+X_n)$, nach
der Verbesserung liefert die Stichprobe $Y_1,\dots,Y_m$ der normalverteilten
Zufallsvariable $Y$ den Mittelwert
$\bar Y=\frac1m(Y_1+\dots+Y_m)$.
Selbst wenn $\bar X$ und $\bar Y$
verschieden sind: Kann man daraus wirklich schon schliessen, dass die
Verbesserung eine Änderung des Parameters gebracht hat, oder könnte
das nicht auch einfach nur eine zufällige Schwankung sein?

Offensichtlich geht es hier darum die Hypothese zu testen, dass sich
durch die Verbesserung nichts geändert hat.
Das Beobachtungsmaterial
soll dann entscheiden, ob die Hypothese noch haltbar ist.
Ist sie es
auf dem Niveau $\alpha$ nicht,
dürften wir schliessen, dass die Verbesserung tatsächlich eine Änderung
des Parameters gebracht hat, und wir würden uns höchstens mit 
Wahrscheinlichkeit $\alpha$ irren.

Wir nehmen an, dass die Streuung der Messwerte der Stichprobe durch die
Veränderung des Produktionsprozesses nicht verändert wurde.
Aber durch
die zusätzlichen Messungen dürfte die Varianz genauer bekannt sein.
Daher bildet man aus der Stichprobenvarianz $S_X^2$ und $S_Y^2$
die gepoolte Stichprobenvarianz
\begin{equation}
S_p^2=\frac{(n-1)S_X^2+(m-1)S_Y^2}{m+n-2},
\label{pooled-variance}
\end{equation}
sie ist ein verbessertes Mass für die Messfehler.

Man möchte nun die die Hypothese testen, dass $\delta=\mu-\nu=0$, wobei
$\mu=E(X)$ und $\nu=E(Y)$ ist.
Dazu braucht man offensichtliche einen
Schätzer für $\delta$, und nicht wirklich überraschend ist
$\hat\delta=\bar X-\bar Y$ ein erwartungstreuer Schätzer für $\delta$.
Man kann weiter zeigen, dass $S_p^2$ ein erwartungstreuer Schätzer
für die gemeinsame Varianz von $X$ und $Y$ ist.
Unter der Annahme der
Hypothese ist $\hat\delta$ eine normalverteilte Zufallsvariable
mit Varianz $\sigma^2(1/n+1/m)$ ist.
$S_p^2(1/n+1/m)$ ist $\chi^2$-verteilt
mit $(m+n-2)$ Freiheitsgraden.
Somit ist 
\begin{equation}
T=\frac{\hat\delta}{S_p\sqrt{1/n+1/m}}
=\frac{\bar X-\bar Y}{\sqrt{(n-1)S_X^2+(m-1)S_Y^2}}\sqrt{\frac{nm(n+m-2)}{n+m}}
\label{t-test-ausdruck}
\end{equation}
$t$-verteilt mit $m+n-2$ Freiheitsgraden.

Der $t$-Test auf Gleichheit der Erwartungswerte spielt sich also wie folgt
ab.
Aus den beiden Stichproben $X_1,\dots,X_n$ und $Y_1,\dots,Y_m$ wird
der Ausdruck $T$ (\ref{t-test-ausdruck}) gebildet.
Übersteigt $T$ den $t$-Wert für die $1-\alpha$-Quantile der $t$-Verteilung
mit $n+m-2$ Freiheitsgraden, wird die Hypothese $\mu=\nu$ verworfen.

\subsubsection{War der Juli 2003 wirklich besonders warm?} \label{julitemperaturen}
Aus den Messdaten der Wetterstation in Altendorf soll entschieden werden,
ob der Juli 2003 wirklich signifikant wärmer war als dr Juli 2004.
Dazu werden die im Laufe des Monats gemessenen Temperaturwerte
ermittelt:

\begin{center}
\begin{tabular}{|r|r|r|r|r|}
\hline
Jahr&$T$&$T^2$&$\operatorname{var}(T)$&$n$\\
\hline
2001&20.351&440.855&26.659&44540\\
2002&19.906&415.302&19.051&44086\\
2003&21.638&493.230&25.023&44473\\
2004&19.582&406.576&23.102&44589\\
2005&19.861&419.796&25.319&44637\\
2006&24.009&603.028&26.573&44623\\
\hline
\end{tabular}
\end{center}

Um einen signifikanten Unterschied der Temperatur nachweisen zu können,
müssen wir für jedes Paar von Jahren den Ausdruck (\ref{t-test-ausdruck})
berechnen, wir erhalten dabei folgende Tabelle:

\begin{center}
\begin{tabular}{|r|rrrrrr|}
\hline
    &    2001&    2002&    2003&    2004&    2005&     2006\\
\hline
2001&        &  13.849& -37.767&  23.013&  14.352& -105.860\\
2002& -13.849&        & -54.881&  10.505&   1.422& -127.863\\
2003&  37.767&  54.881&        &  62.542&  52.864&  -69.666\\
2004& -23.013& -10.505& -62.542&        &  -8.469& -132.656\\
2005& -14.352&  -1.422& -52.864&   8.469&        & -121.646\\
2006& 105.860& 127.863&  69.666& 132.656& 121.646&         \\
\hline
\end{tabular}
\end{center}

Da die Zahl der Freiheitsgrade sehr gross ist, müssen wir
uns nicht um die genaue Anzahl kümmern, sondern können stattdessen
die Grenzwerte für unendlich viele Freiheitsgrade verwenden.
Auf
dem Niveau $\alpha=0.01$ finden wir $t_{\infty}^{0.005}=2.576$.
Da alle Einträge in der Tabelle bis auf das Paar $(2002,2005)$
betragsmässig grösser sind, können wir auf dem Niveau $\alpha=0.01$
schliessen, dass die Juli-Durchschnittstemperatur zwischen
2001 und 2006 verschieden war.
Nur für die Jahre 2002 und 2005
können wir den Unterschied nicht auf dem Niveau 0.01 sicherstellen.
Da jedoch $t_{\infty}^{0.1}=1.282$ könnten wir auf dem Niveau
$0.3$ beweisen, dass die Juli-Durchschnittstemperatur in den
Jahren 2002 und 2005 verschieden war, wir würden dabei aber 
mit Wahrscheinlichkeit $0.3$ einen Fehler erster Art machen.

