%
% spamfilter.tex -- Die Mathematik hinter einem Spamfilter basieren
%                   auf einem Bayes-Filter
%
% (c) 2006-2015 Prof Dr Andreas Mueller, Hochschule Rapperswil
%
\section{Anwendung: Spamfilter} \label{spamfilter}
\index{Spam}
\index{Spamfilter}
Etwa drei Viertel aller auf dem Internet transportierten Email-Meldungen
sind Spam, d.~h.~sie sind unerw"unscht, und haben meist den Charakter
kommerzieller Werbemeldungen. Lange Zeit wurde versucht, Mail-Meldungen
durch Mustererkennung in Spam und Ham zu klassifizieren, so wie es bei
Viren auch erfolgreich ist. Bei Spam war dieses Vorgehen jedoch nur begrenzt
wirksam, denn in der Tat wird hier ein anderes Problem gel"ost.
Bei der Erkennung von Viren geht es darum, bekannte, unver"anderliche
Muster zu erkennen. Bei Spam geht es nicht nur darum, bereits bekannte
Mails als Spam zu erkennen, sondern auch zuk"unftige Mails auf Grund
ihres Inhalts als Spam zu erkennen. Zu dieser Schwierigkeit lieferte
Paul Graham mit seinem Artikel ``A Plan for Spam'' \url{http://www.paulgraham.com/spam.html} eine unerwartete, eher heuristische L"osung, die er auch in
\index{Graham, Paul}
einem Produkt erfolgreich umsetzen konnte.

Sp"ater wurde sein empirisches Verfahren als eine Anwendung des Satzes von
Bayes auch mathematisch verstanden, Ziel dieses Abschnitts ist,
die mathematischen Prinzipien hinter einem auf Bayes-Filterung
basierenden Spamfilter zu beschreiben.

\subsection{Grundprinzip}
Damit ein Spamfilter auf Wahrscheinlichkeitsbasis funktionieren kann, muss
zun"achst eine Datengrundlage geschaffen werden, was der Benutzer dadurch
tut, dass er Spam-Messages in einen eigenen Ordner aussortiert.
Der Spamfilter wird sp"ater versuchen, auf Grund von Kriterien wie in
der Message enthaltenen W"ortern, die Wahrscheinlichkeit daf"ur zu ermitteln,
dass eine neue Message Spam ist. Der Einfachheit nehmen wir an, dass
das Kriterium darin besteht, ob das Wort ``Viagra'' in der Message
vorhanden ist. Dieses Ereignis bezeichnen wir mit $V$, Spam-Messages bilden
das Ereignis $S$.
\index{Viagra}

Mit diesen Bezeichnungen k"onnen auf Grund der Mail-Messages des Benutzers
bereits folgende Gr"ossen bestimmt werden:
\begin{center}
\begin{tabular}{ll}
$P(S)$&Wahrscheinlichkeit, eine Spam-Message zu erhalten\\
$P(V)$&Wahrscheinlichkeit, das Wort Viagra in einer Message zu finden\\
$P(V|S)$&Wahrscheinlichkeit, dass eine Spam-Message ``Viagra'' enth"alt\\
\end{tabular}
\end{center}
Diese Daten n"utzen uns aber nichts f"ur die Beurteilung, ob eine neue
Message Spam ist, denn dazu m"ussen wir wie folgt vorgehen:
\begin{enumerate}
\item feststellen, ob das Wort ``Viagra'' in der Message enthalten ist
\item im positiven Fall $P(S|V)$ bestimmen und entscheiden, ob Spam vorliegt,
im negativen Fall analog mit $P(S|\bar V)$.
\end{enumerate}
Wir m"ussen also $P(S|V)$ ermitteln. Aus den bekannten Gr"ossen ist dies
mit dem Satz von Bayes sofort m"oglich:
\begin{equation}
P(S|V)=\frac{P(V|S)P(S)}{P(V)}.
\end{equation}
Diese Formel zeigt auch, dass W"orter, die in Spam h"aufig, in normalen
Mails aber selten sind, die Wahrscheinlichkeit f"ur Spam hochtreiben.

\subsection{Kombinierte Kriterien}
Ein leistungsf"ahger Spamfilter wird aber nicht nur ein einziges Kriterium
untersuchen, sondern er wird versuchen, Kombinationen von Kriterien, die
auf Spam hinweisen, zu erkennen und entsprechend die Wahrscheinlichkeit
zu erh"ohen, eine Message als Spam zu erkennen.

Das abstrakte mathematische Problem des Spamfilters ist jetzt also,
dass die Wahrscheinlichkeit berechnet werden soll, dass eine Message
Spam ist, wenn bereits zwei Spam-Kriterien $X$ und $Y$ eingetreten
sind. Es ist also $P(S|X\cap Y)$ gesucht.

Bekannt ist wiederum alles, was sich durch Ausz"ahlen von Messages
in einem festen Bestand von Meldungen ermitteln l"asst. Man sagt,
der Filter sei mit diesen Meldungen trainiert worden. Zum Beispiel
sind folgende Gr"ossen bekannt: $P(X)$, $P(X|S)$ und analog f"ur $Y$.
Ausserdem sind auch $P(S)$ und $P(\bar S)$ bekannt.

F"ur die Berechnung von $P(S|X\cap Y)$ verwendet man jetzt wieder
den Satz von Bayes:
\begin{align}
P(S|X\cap Y)&=&\frac{P(X\cap Y|S)P(S)}{P(X\cap Y)}\nonumber
\\
&=
\frac{P(X\cap Y|S)P(S)}{P(X\cap Y|S)P(S)+P(X\cap Y|\bar S)P(\bar S)}
\end{align}
Das Problem mit dieser Formel ist, dass wir $P(X\cap Y|S)$ normalerweise
nicht kennen. Normalerweise wird keine Statistik dar"uber gef"uhrt, mit
welcher H"aufigkeit Spam-Kriterien zusammen in Spam-Messages auftreten.
Daher m"ussen wir an dieser Stelle annehmen, dass die Kriterien $X$ und $Y$
unabh"angig sind. Dies wird im allgemeinen falsch sein, denn wenn Spam
das Wort ``Viagra'' enth"alt, dann enth"alt die gleiche Message sehr h"aufig
auch das Wort ``Cialis'', die Ereignisse sind also abh"angig.

Wir brauchen hier eine etwas st"arkere Form der Unabh"angigkeit von $X$ und $Y$,
n"amlich die Bedingung, dass $X$ und $Y$ bei Einschr"ankung auf $S$
oder $\bar S$ unabh"angig sind. Dann gilt $P(X\cap Y|S)=P(X|S)P(Y|S)$
und sinngem"ass f"ur $\bar S$.

Damit kann nun die Formel umgeformt werden:
\begin{align}
P(S|X\cap Y)
&=
\frac{P(X\cap Y|S)P(S)}{P(X\cap Y|S)P(S)+P(X\cap Y|\bar S)P(\bar S)}\nonumber
\\
&=
\frac{P(X|S)P(Y|S)P(S)}{P(X|S)P(Y|S)P(S)+P(X|\bar S)P(Y|\bar S)P(\bar S)}
\end{align}
Im Prinzip k"onnte man an dieser Stelle aufh"oren, denn man hat ja jetzt
alles, was man braucht. Trotzdem ist diese Formel noch etwas unpraktisch.
Es w"are doch viel besser, wenn man zun"achst alle Wahrscheinlichkeiten
daf"ur ermitteln k"onnte, unter denen eine Message Spam ist, wenn ein
einzelnes Kriterium eingetreten ist, also $P(S|X)$, und dann nur noch
mit diesen rechnen, um $P(S|X\cap Y)$ zu bestimmen.
Dazu muss man offensichtlich den Satz von Bayes nochmals anwenden, um
$P(X|S)$ umzukehren:
\begin{align}
P(S|X\cap Y)
&=
\frac{P(X|S)P(Y|S)P(S)}{P(X|S)P(Y|S)P(S)+P(X|\bar S)P(Y|\bar S)P(\bar S)}\nonumber
\\
&=\frac{\displaystyle
\frac{P(S|X)P(X)}{P(S)}
\frac{P(S|Y)P(Y)}{P(S)}P(S)
}{\displaystyle
\frac{P(S|X)P(X)}{P(S)}
\frac{P(S|Y)P(Y)}{P(S)}P(S)
+
\frac{P(\bar S|X)P(X)}{P(\bar S)}
\frac{P(\bar S|Y)P(Y)}{P(\bar S)}P(\bar S)
}\nonumber
\\
&=
\frac{\displaystyle
\frac{P(S|X)}{P(S)}
\frac{P(S|Y)}{P(S)}P(S)
}{\displaystyle
\frac{P(S|X)}{P(S)}
\frac{P(S|Y)}{P(S)}P(S)
+
\frac{P(\bar S|X)}{P(\bar S)}
\frac{P(\bar S|Y)}{P(\bar S)}P(\bar S)
}
\end{align}
In dieser Formel stehen nur noch die bereits ermittelten Wahrscheinlichkeiten
$P(S|X)$ und die Wahrscheinlichkeiten von $S$ und $\bar S$.

Eine weitere Vereinfachung kann man erreichen, wenn man die Trainingsmenge so
w"ahlt, dass $P(S)=P(\bar S)=\frac12$. Dann erh"alt man
\begin{align}
P(S|X\cap Y)
&=
\frac{P(S|X)P(S|Y)}{P(S|X)P(S|Y)+P(\bar S|X)P(\bar S|Y)}\nonumber
\\
&=
\frac{P(S|X)P(S|Y)}{P(S|X)P(S|Y)+(1-P(S|X))(1-P(S|Y))}
\end{align}
Damit dieser Wert m"oglichst gross wird, also $X\cap Y$ ein m"oglichst
gutes Spam-Kriterium ist, sollte der zweite Term im Nenner m"oglichst klein
sein, d.h. falls $P(S|X)$ nahe bei $1$ ist, ist $X$ ein gutes Kriterium f"ur
Spam, und dann erst recht auch $X\cap Y$ f"ur zwei solche Kriterien.
