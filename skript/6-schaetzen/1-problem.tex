\section{Das Schätzproblem}
\label{section-schaetzproblem}
\kopfrechts{Das Schätzproblem}
In bisherigen Beispielen sind wir bei der Bestimmung von Erwartungswert
oder Varianz jeweils davon ausgegangen, dass einige bekannte Werte der
Zufallsvariablen bereits eine repräsentative Beschreibung der Zufallsvariablen
darstellen.
Wir konnten also aus den Beobachtungen die Wahrscheinlichkeiten
für die beobachteten Werte ableiten und daraus direkt Erwartungswert
und Varianz bestimmen.

Dieses heuristische Vorgehen ist als erste Näherung durchaus akzeptabel,
und funktioniert für den Erwartungswert auch optimal.
Für die Varianz
hingegen zeigt sich, dass die oben skizzierte, simplizistische Rechnung
die Varianz vor allem bei kleinen Stichproben systematisch unterschätzt.
Damit stellt sich automatisch die Frage, wie denn überhaupt ein geeignetes
Schätzverfahren für eine Grösse wie die Varianz gefunden werden kann.

Etwas genauer geht es um folgendes Problem.
Von einer gegebenen Zufallsvariable
$X$ ist die Verteilung nicht bekannt, es wird angenommen, dass die
Verteilungsfunktion die Form $F(x,\vartheta)$ hat, wobei
$\vartheta$ ein vorläufig noch unbekannter Parameter ist.
Bei einer Normalverteilung sind dies zum Beispiel der Erwartungswert und
die Varianz: $\vartheta=(\mu,\sigma^2)$.
Nun wird das
Experiment $n$ mal durchgeführt, dadurch entstehen $n$ neue, unabhängige
Zufallsvariable $(X_i)_{1\le i\le n}$, welche identisch verteilt sind zu
$X$.

\begin{definition}
Eine Folge von Zufallsvariablen $(X_i)_{1\le i\le n}$ heisst eine
Stichprobe der Zufallsvariablen $X$, wenn die $X_i$ unabhängig 
und identisch zu $X$ verteilt.
\end{definition}
Das Problem besteht nun darin, die Parameterwerte $\vartheta$
aus den gefundenen Werten $X_1,\dots,X_n$ zu finden.
Man braucht dazu
eine sogenannten Schätzer, eine Funktion, die $\vartheta$ aus den
Beobachtungen bildet: $\vartheta=\vartheta(X_1,\dots,X_n)$.

Bei der Konstruktion eines Schätzverfahrens kann man sich von verschiedenen,
teils nicht kompatiblen Prinzipien leiten lassen, und man wird verschiedene
``vernünftige'' Eigenschaften von den Schätzern fordern wollen:
\begin{enumerate}
\item {\bf Konsistent:} Vergrössert man die Stichprobe, soll der
Schätzer gegen den wahren Wert des Parameters streben, also
$\lim_{n\to\infty}\vartheta(X_1,\dots,X_n)=\vartheta$.
\item {\bf Erwartungstreu:} Da die $X_i$ auch wieder Zufallsvariable sind,
kann man den Erwartungswert von $\vartheta(X_1,\dots,X_n)$ bilden. 
Das Verfahren ist besonders vertrauenswürdig, wenn dieser Erwartungswert
mit dem wahren Wert des Parameters übereinstimmt.
\item {\bf Minimaler Fehler:} Selbst wenn der Schätzwert im Mittel den
richtigen Parameterwert liefert, möchten wir doch, dass dies mit möglichst
geringem Fehler passiert, so dass auch eine einzelne Bestimmung des
Parameters bereits grosse Zuverlässigkeit geniesst.
Der mittlere
quadratische Fehler, der im Zusammenhang mit dem Erwartungswert bereits
einmal diskutiert wurde, kann hier als Prinzip zur Konstruktion von
Schätzern erhoben werden.
\item {\bf ``Am Wahrscheinlichsten'':} Ein etwas anderes Prinzip der 
Konstruktion eines Schätzers besteht darin, die Beobachtungen $X_1,\dots,X_n$
in die Dichtefunktion der Verteilung einzusetzen und denjenigen Parameterwert
zu wählen, der diese Funktion maximiert.
Diese sogenannten
``maximum likelihood'' Schätzer sind oft vernünftig, haben interessante
asymptotische Eigenschaften, sind aber manchmal nicht erwartungstreu.
\end{enumerate}

Leider zeigt es sich, dass die Forderung nach minimalem quadratischem
Fehler technisch oft nur schwer umzusetzen ist.
Trotzdem liefert das verwandte
Prinzip der kleinsten Quadrate, welches wir schon bei der Charakterisierung
des Erwartungswertes mit Hilfe der Varianz und bei der Regression 
getroffen haben, manchmal nützliche Vorschläge für geeignete
Schätzer.

Sobald ein Schätzverfahren festgelegt ist, muss man sich mit der Frage der
Zuverlässigkeit des Verfahrens auseinandersetzen.
Da das Schätzverfahren
auch ``nur'' eine Zufallsvariable ist, wird es den Parameter auch nur
mit einer gewissen Unsicherheit liefern können.
Es stellt sich damit
automatisch die Frage nach der Grösse einer möglichen Abweichung.
Eine Antwort darauf gibt das sogenannte Konfidenzintervall, ein
Intervall, in dem sich der wahre Wert des Parameters mit einer gewissen
Wahrscheinlichkeit befindet.

In diesem Kapitel betrachten wir zunächst die bereits bekannten 
Schätzer für den Erwartungswert und die Varianz, und untersuchen
sie daraufhin, ob sie erwartungstreu sind.
In den Abschnitten \ref{section-maximum-likelihood-schaetzer}
und \ref{section-weitere-beispiele-von-schaetzern} wird an einigen
Beispielen gezeigt, wie man Schätzer nach dem
Maximum-Likelihood-Prinzip konstruieren kann.
In \ref{section-verteilung-der-schaetzwerte}
wird die Verteilung der Schätzer studiert, eine Grundlage, damit
anschliessend in \ref{section-konfidenzintervalle} Konfidenzintervall
für die geschätzen Werte konstruiert werden können.

