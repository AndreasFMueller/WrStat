\section{Erwartungstreue Schätzer} \label{section-erwartungstreue-schaetzer}
Nach \ref{section-konsistente-schaetzer} ist der Stichprobenmittelwert
ein konsistenter Schätzer,
aber er ist sogar erwartungstreu:
\[
E(\mu(X_1,\dots,X_n))=\frac{E(X_1)+\dots+E(X_n)}{n}=\frac{E(X)+\dots+E(X)}{n}
=E(X),
\]
der Mittelwert ist also durchaus ein Schätzer mit ``guten'' Eigenschaften.

Für die Schätzung der Varianz wurde bisher aus beobachteten Werten
$(x_i)_{1\le i\le n}$ der Zufallsvariable der Ausdruck
\begin{equation}
\frac1n\sum_{i=1}^n x_i^2-\biggl(\frac1n\sum_{i=1}^nx_i\biggr)^2
\end{equation}
verwendet.
Dem entspricht die Schätzformel
\begin{equation}
V=f(X_1,\dots,X_n)=\frac1n\sum_{i=1}^nX_i^2-\biggl(\frac1n\sum_{i=1}^nX_i\biggr)^2.
\label{schaetzer-sstern}
\end{equation}
Die Zufallsvariable $V$ soll angeblich die Varianz möglichst gut
wiedergeben, also sollte auch der Erwartungswert von $V$ mit der Varianz
übereinstimmen.

Ein numerisches Experiment zeigt jedoch ein anderes Bild.
In einer
Simulation wurde jeweils 10000 mal die Varianz einer kleinen Anzahl von
standardnormalverteilten Zufallsvariablen mit obiger Schätzformel ermittelt.
In der Tabelle~\ref{varianzschaetzung} sind die Resultate von acht
Läufen dieser Simulation untereinander dargestellt.
Man kann gut erkennen, 
dass die Schätzformel für geringe Anzahlen von Werten, also für kleine
$n$, deutlich falsch liegt.
Statt dem wahren Wert $1$ der Varianz
werden Werte gefunden, die deutlich kleiner sind, die Zahlenwerte
suggerieren, dass die Schätzformel statt $1$ im Mittel den Wert
$\frac{n-1}n$ liefert.
Dies ist ein Indiz, dass die Schätzformel nicht
erwartungstreu ist.

\begin{table}
\begin{center}
\begin{tabular}{|c|ccccccccc|}
\hline
$n$&2&3&4&5&6&7&8&9&10\\
\hline
&0.502&0.674&0.745&0.802&0.839&0.853&0.871&0.892&0.896\\
&0.488&0.670&0.746&0.800&0.837&0.858&0.875&0.890&0.899\\
&0.501&0.668&0.746&0.800&0.843&0.861&0.872&0.892&0.900\\
&0.487&0.670&0.743&0.805&0.826&0.854&0.868&0.883&0.898\\
&0.506&0.672&0.745&0.797&0.841&0.856&0.873&0.887&0.904\\
&0.506&0.659&0.763&0.794&0.835&0.855&0.874&0.888&0.908\\
&0.513&0.666&0.751&0.795&0.832&0.853&0.868&0.897&0.903\\
&0.499&0.669&0.748&0.799&0.833&0.856&0.873&0.894&0.903\\
\hline
$\frac{n-1}{n}$&0.500&0.667&0.750&0.800&0.833&0.857&0.875&0.889&0.900\\
\hline
\end{tabular}
\end{center}
\caption{Mittelwerte von 10000 Varianzberechnungen von jeweils $n$
Werten einer standardnormalverteilten Zufallsvariable\label{varianzschaetzung}}
\end{table}

Das dies tatsächlich so ist wird deutlich, wenn wir den
Erwartungswert von $V$ berechnen:
\begin{align*}
E(V)
&=
E(f(X_1,\dots,X_n))
\\
&=
E\biggl(
\frac1n\sum_{i=1}^nX_i^2-\biggl(\frac1n\sum_{i=1}^nX_i\biggr)^2
\biggr)
\\
&=
\frac1n\sum_{i=1}^nE(X_i^2)-\frac1{n^2}\sum_{i=1}^n\sum_{j=1}^n E(X_iX_j)
\\
&=
\frac1n\sum_{i=1}^nE(X_i^2)-
\frac1{n^2}\sum_{i=1}^nE(X_i^2)
-\frac1{n^2}\sum_{i\ne j} E(X_iX_j)
\\
&=
\frac{n-1}{n^2}\sum_{i=1}^nE(X_i^2)
-\frac1{n^2}\sum_{i\ne j} E(X_i)E(X_j).
\end{align*}
Nun sind aber alle Zufallsvariable $X_i$ identisch verteilt, insbesondere
sind alle Erwartungswerte der $X_i$ und der $X_i^2$ jeweils identisch,
wir können also die Indizes weglassen:
\begin{align*}
E(V)
&=
\frac{n-1}{n^2}\sum_{i=1}^nE(X^2) -\frac1{n^2}\sum_{i\ne j} E(X)^2
\\
&=
\frac{n-1}{n^2}E(X^2) -\frac{n^2-n}{n^2}E(X)^2
\\
&=
\frac{n-1}{n}(E(X^2) -E(X)^2)=\frac{n-1}n\operatorname{var}(X).
\end{align*}
Folglich ist der Schätzer $V$ für kleine Stichproben eine schlechte
Wahl, man muss erwarten, dass er einen zu kleinen Wert liefert.
Es
wäre besser, ihn mit $\frac{n}{n-1}$ zu multiplizieren, denn der so
gebildete Schätzer hat als Erwartungswert wie gewünscht die Varianz.

\begin{definition}
Ist $X_i$ eine Stichprobe von $X$, dann heisst
\begin{equation}
S^2=\frac1{n-1}\sum_{i=1}^n (X_i -\bar X)^2
\end{equation}
die {\em Stichprobenvarianz} von $X$.
\end{definition}

\begin{satz}Die Stichprobenvarianz ist ein erwartungstreuer Schätzer
für die Varianz einer Zufallsvariable.
\end{satz}

Dem Schätzer
\begin{equation}
S^{*}\mathstrut^{2}=\frac1n\sum_{i=1}^n(X_i-\mu)^2
\end{equation}
für die Varianz fehlt nicht viel zu einem erwartungstreuen Schätzer,
er unterscheidet sich vom eben gefundenen besseren Schätzer
\begin{equation}
S^2=\frac1{n-1}\sum_{i=1}^n(X_i-\bar X)^2
\end{equation}
nur durch einen Faktor $\frac{n}{n-1}$.
Für eine kleine Stichprobe ist
der bisherige Schätzer also nicht erwartungstreu, aber je grösser $n$
wird, desto kleiner wird die Abweichung von der Erwartungstreue.
Man
nennt $S^{*}\mathstrut^{2}$ einen asymptotisch erwartungstreuen Schätzer:
\begin{definition}
Ein Schätzer $\vartheta(X_1,\dots,X_n)$ heisst asymptotisch erwartungstreu,
wenn
\begin{equation}
\lim_{n\to\infty}E(\vartheta(X_1,\dots,X_n))=\vartheta.
\end{equation}
\end{definition}

