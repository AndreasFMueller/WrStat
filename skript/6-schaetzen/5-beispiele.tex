\section{Weitere Beispiele von Schätzern} \label{section-weitere-beispiele-von-schaetzern}
\subsection{Länge eines Intervalls}
Von der Zufallsvariable $X$ ist bekannt, dass sie im Intervall $[0,\vartheta]$
gleichverteilt ist.
Der Parameter $\vartheta$ soll geschätzt werden.
Offensichtlich ist $\vartheta \ge \max(X_1,\dots,X_n)$,
aber $T=\max(X_1,\dots,X_n)$ ist fast sicher kleiner als der wahre
Wert von $\vartheta$, wir können also nicht erwarten, dass $T$ ein
erwartungstreuer Schätzer ist.

In der Hoffnung, später einen erwartungstreuen Schätzer konstruieren
zu können, berechnen wir den Erwartungswert von $T$.
Die Verteilungsfunktion von $X$ auf dem Intervall $[0,\vartheta]$ ist
\begin{equation}
F(x)=P(X\le x)= \frac{x}{\vartheta}.
\end{equation}
Die Verteilungsfunktion von $T(X_1,\dots,X_n)$ ist dann
\begin{align*}
P(\max(X_1,\dots,X_n)\le x)
&=
P(X_1\le x\wedge\dots\wedge X_n\le x)
\\
&=
P(X_1\le x)\dots P(X_n\le x)
\\
&=
F(x)^n=\frac{x^n}{\vartheta^n},
\end{align*}
im Intervall $[0,\vartheta]$ entspricht dies der Dichtefunktion
\begin{equation}
\frac{nx^{n-1}}{\vartheta^n}.
\end{equation}
Damit wird der Erwartungswert
\begin{align*}
E(\max(X_1,\dots,X_n))
&=
\int_0^{\vartheta}x\frac{nx^{n-1}}{\vartheta^n}\,dx
\\
&=
\frac{n}{n+1}\biggl[\frac{x^{n+1}}{\vartheta^n}\biggr]_0^\vartheta
\\
&=
\frac{n}{n+1}\vartheta,
\end{align*}
bis auf den Faktor $\frac{n}{n+1}$ ist das genau die gesuchte Intervalllänge.
\begin{satz}
Ist $X_i$ eine Stichprobe einer auf dem Intervall $[0,\vartheta]$
gleichverteilten Zufallsvariable $X$, dann ist
\begin{equation}
\vartheta(X_1,\dots,X_n)=\frac{n+1}{n}\max(X_1,\dots, X_n)
\end{equation}
ein erwartungstreuer Schätzer für die Intervalllänge $\vartheta$.
\end{satz}

\subsection{Schätzung des Parameters \texorpdfstring{$\lambda$}{lambda} einer Poissonverteilung}
Die Zufallsvariable $X$ sei poissonverteilt, d.~h.
\begin{equation}
p(k, \lambda)=\frac{\lambda^k}{k^!}e^{-\lambda}.
\end{equation}
Der Parameter $\lambda$ soll aus einer Stichprobe geschätzt werden.
Dazu bilden wir die Likelihood Funktion
\begin{equation}
L(k_1,\dots k_n;\lambda)=\frac{\lambda^{k_1+\dots+k_n}}{k_1!\dotsm k_n!}
\,e^{-n\lambda}.
\label{poisson-likelihood-funktion}
\end{equation}
Das Maximum der Likelihood-Funktion wird durch Ableiten von
(\ref{poisson-likelihood-funktion}) nach $\lambda$
bestimmt:
\begin{align}
\frac{d}{d\lambda}L(k_1,\dots,k_n;\lambda)
&=
\frac{1}{k_1!\dotsm k_n!}(K\lambda^{K-1}-n\lambda^K)e^{-n\lambda}\nonumber
\\
&=
\frac{1}{k_1!\dotsm k_n!}(K-n\lambda)\lambda^{K-1}e^{-n\lambda}.
\label{poisson-likelihood-ableitung}
\end{align}
Die Ableitung verschwindet genau dann, wenn der Klammerausdruck in
(\ref{poisson-likelihood-ableitung}) verschwindet, also ist
\begin{equation}
\lambda(k_1,\dots,k_n) =\frac1n\sum_{i=1}^nk_i
\end{equation}
der Maximum Likelihood Schätzer für $\lambda$.
Offensichtlich ist er
konsistent und erwartungstreu.

\subsection{Schätzung von \texorpdfstring{$p$}{p} in einer Binomialverteilung}
Von der Zufallsvariablen $X$ sei bekannt, dass sie binomialverteilt ist
mit Parameter $(m, p)$, der Parameter $p$ ist zu schätzen.
Die Likelihood-Funktion ist in diesem Fall
\begin{equation}
L(k_1,\dots,k_n;p)=\biggl(\prod_{i=1}^n\binom{m}{k_i}\biggr)
p^{K}(1-p)^{nm-K},
\label{binomial-likelihood-funktion}
\end{equation}
wobei wir wieder $K=\sum_{i=1}^nk_i$ setzen. 
Das grosse Produkt in (\ref{binomial-likelihood-funktion})
ist nur ein konstanter Faktor, den wir mit $P$ abkürzen
\begin{equation}
P= \prod_{i=1}^n\binom{m}{k_i}
\end{equation}
Die Ableitung von (\ref{binomial-likelihood-funktion}) ist
\begin{equation}
\frac{d}{dp}L(k_1,\dots,k_n;p)=P
\bigl(Kp^{K-1}(1-p)^{nm-K}-(nm-K)p^K(1-p)^{nm-K-1}\bigr)=0.
\end{equation}
Der zweite Klammerausdruck lässt sich vereinfachen zu
\[
(K(1-p)-(nm-K)p)p^{K-1}(1-p)^{nm-K-1}=
(K-nmp)p^{K-1}(1-p)^{nm-K-1}=0
\]
er verschwindet genau dann, wenn $p=K/nm$.
Somit ist
\begin{equation}
p(k_1,\dots,k_n)=\frac1{nm}\sum_{i=1}^nk_i
\end{equation}
der Maximum-Likelihood-Schätzer für $p$.
Offensichtlich ist er konsistent
und erwartungstreu.
\begin{satz}
Ein erwartungstreuer Schätzer für die Wahrscheinlichkeit einer
Binomialverteilung $\operatorname{Bin}(m,k,p)$ ist
$\frac1{m}\bar X$.
\end{satz}

