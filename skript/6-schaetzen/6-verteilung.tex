\section{Verteilung der Schätzwerte}
\label{section-verteilung-der-schaetzwerte}
\kopfrechts{Verteilung der Schätzwerte}
Schätzwerte entstehen dadurch, dass die Werte einer Stichprobe, also
gewisse Realisierungen von Zufallsvariablen in die Schätzfunktion
eingesetzt werden.
Somit sind die Schätzwerte selbst wieder Zufallsvariablen,
und es stellt sich die Frage, welcher Verteilung sie gehorchen.

Für einen Spezialfall haben wir uns um diese Problem bereits gekümmert.
Bei der Schätzung der Intervalllänge haben wir die Verteilungsfunktion
des Schätzwertes bestimmt, um einen erwartungstreuen Schätzer konstruieren
zu können.

Ein anderer Spezialfall ist die Schätzung der Parameter einer normalverteilten
Zufallsvariable.
Die Stichprobenelemente $X_i$ sind alle normalverteilt
mit dem gleichen Erwartungswert und der gleichen Varianz wie $X$. 
Somit ist der Schätzwert für den Erwartungswert
$\bar X$
ebenfalls normalverteilt mit Erwartungswert $E(X)$ (wir wissen ja bereits,
dass dieser Schätzer erwartungstreu ist), und Varianz
$\frac1n\operatorname{var}(X)$.
Wie ist der
Schätzwert für die Stichprobenvarianz
$S^2=$
verteilt? Dazu gibt der folgende, vielleicht etwas überraschende Satz Auskunft.
\begin{satz}
Seien $X_1,\dots,X_n$ unabhängige, normalverteilte Zufallsvariablen mit
Erwartungswert $\mu$ und Varianz $\sigma^2$.
Dann gilt
\begin{enumerate}
\item $\bar X$ und $S^2$ sind unabhängig.
\item $\bar X$ ist normalverteilt mit Erwartungswert $\mu$ und Varianz
$\frac{\sigma^2}{n}$.
\item $\frac{n-1}{\sigma^2}S^2$ ist $\chi^2_{n-1}$-verteilt.
\end{enumerate}
\end{satz}
\begin{proof}[Beweis]
Wir beweisen zunächst die Unabhängigkeit mit Hilfe der momenterzeugenden
Funktion.
Zunächst gilt
\begin{equation}
M_{X_i}(t_i)=\exp\biggl(\mu t_i+\frac{\sigma^2t_i^2}2\biggr).
\end{equation}
Wir betrachten jetzt den Vektor $U=(U_1,\dots,U_n)$, bestehend aus
den Zufallsvariablen $U_i=X_i-\bar X$, und $V=\bar X$.
Wir setzen $\bar s=(s_1+\dots +s_n)/n$.
Die gemeinsame momenterzeugende Funktione
von $U$ und $V$ ist dann
\begin{align}
M_{U,V}(s_1,\dots,s_n,r)
&=
E\biggl(\exp\biggl(Vr+\sum_{i=1}^nU_is_i\biggr)\biggr)
\nonumber\\
&=
E\biggl(\exp\biggl(\bar X(r-n\bar s)+\sum_{i=1}^nX_is_i\biggr)\biggr)
\nonumber\\
&=
E\biggl(\exp\biggl(\sum_{i=1}^nX_i\bigl(s_i-\bar s+\frac{r}{n}\bigr)\biggr)\biggr)
\nonumber\\
&=
M_{X_i}\biggl(s_1-\bar s+\frac{r}{n}\biggr)\cdots
M_{X_n}\biggl(s_n-\bar s+\frac{r}{n}\biggr)
\nonumber\\
&=
\exp\biggl(
\mu\biggl(\sum_{i=1}^n(s_i-\bar s+r/n)\biggr)+\frac{\sigma^2}2\sum_{i=1}^n(s_i-\bar s+r/n)^2
\biggr)
\nonumber\\
&=
\exp\biggl(\mu r+\frac{\sigma^2r^2}{2n}\biggr)
\exp\biggl(\frac{\sigma^2}{2}\sum_{i=1}^n(s_i-\bar s)^2\biggr)\label{umformung-mittelwert}
\\
&=
M_{U,V}(0,\dots,0,r)M_{U,V}(s_1,\dots,s_n,0).
\nonumber
\end{align}
In (\ref{umformung-mittelwert}) haben wir folgende zwei Identitäten verwendet:
\begin{align}
\sum_{i=1}^n(s_i-\bar s)
&=
\sum_{i=1}^ns_i-n\bar s
\nonumber\\
&=
\sum_{i=1}^ns_i-n\frac1n\sum_{i=1}^ns_i=0 \label{si-mittelwert}
\\
\sum_{i=1}^n(s_i-\bar s+\frac{r}{n})^2
&=
\sum_{i=1}^n(s_i-\bar s)^2
+\frac{2r}{n}\sum_{i=1}^n(s_i-\bar s)
+\sum_{i=1}^n\frac{r^2}{n^2}
\nonumber\\
&=
\sum_{i=1}(s_i-\bar s)^2
+\frac{2r}{n}\sum_{i=1}^ns_i-2r\bar s
+\frac{r^2}{n}
\nonumber\\
&=
\sum_{i=1}^n(s_i-\bar s)^2+\frac{r^2}{n}.
\label{si-varianz}
\end{align}
Da sich die gemeinsame momenterzeugende Funktion faktorisieren
lässt, sind die Zufallsvariablen $U_i$ und $V$ unabhängig,
und damit natürlich auch $S^2=\frac1{n-1}(U_1^2+\dots+U_n^2)$ und
$\bar X=V$.

Die Verteilung von $\bar X$ wurde bereits früher bestimmt, und bei
dieser Gelegenheit alles im zweiten Punkt behauptete nachgerechnet.

Da nach obigem die Zufallsvariablen $Z_i=(X_i-\mu)/\sigma$ unabhängig
standardnormalverteilt sind, ist deren Quadratsumme
\begin{equation}
T=\sum_{i=1}^nZ_i^2=\sum_{i=1}^n\frac{(X_i-\mu)^2}{\sigma^2}
\end{equation}
$\chi^2_{n}$-verteilt.
Da ausserdem $Z=\sqrt{n}(\bar X-\mu)/\sigma$ standardnormalverteilt
ist, ist ihr Quadrat
\begin{equation}
W=Z^2=\frac{n(\bar X-\mu)^2}{\sigma^2}
\end{equation}
$\chi^2_1$-verteilt.
Wir setzen $Y=(n-1)S^2/\sigma^2$.
$W$ und $Y$ sind nach dem eben Bewiesenen unabhängig.
Wir behaupten
\[
T=W+Y.
\]
In der Tat 
\begin{align}
W+Y
&=
\frac{n(\bar X-\mu)^2}{\sigma^2}
+
\frac{1}{\sigma^2}\sum_{i=1}^n(X_i-\bar X)^2
\nonumber\\
&=
\frac{1}{\sigma^2}\sum_{i=1}^n\bigl((\bar X-\mu)^2+(X_i-\bar X)^2\bigr)
\nonumber\\
&=
\frac{1}{\sigma^2}\sum_{i=1}^n(X_i-\mu)^2
\label{wy-chi2}
\end{align}
Im letzten Schritt (\ref{wy-chi2}) haben wir den Hilfssatz
\ref{hilfssatz-varianz-mittelwert}
weiter unten verwendet.
Für die momenterzeugenden Funktionen bedeutet dies
\begin{equation}
M_T(t)=M_W(t)M_Y(t)
\end{equation}
oder nach Satz \ref{chi2}
\begin{equation}
M_Y(t)=\frac{M_T(t)}{M_W(t)}=\frac{(1-2t)^{-n/2}}{(1-2t)^{-1/2}}
=(1-2t)^{-(n-1)/2},
\end{equation}
dies ist die momenterzeugende Funktion einer $\chi^2_{n-1}$-Verteilung.
Somit ist $Y$ $\chi_{n-1}^2$ verteilt, wie behauptet.
\end{proof}

\begin{hilfssatz}
\label{hilfssatz-varianz-mittelwert}
Seien $(x_i)_{1\le i\le n}$ reelle Zahlen und $\bar x$ so gewählt,
dass $\sum_{i=1}^n(x_i-\bar x)^2$ minimal wird.
Dann ist 
\begin{equation}
\bar x=\frac1n\sum_{i=1}^nx_i
\end{equation}
und für jedes $\mu\in\mathbb{R}$ gilt
\begin{equation}
\sum_{i=1}^n\bigl((\bar x-\mu)^2 + (x_i-\bar x)^2\bigr)
=\sum_{i=1}^n(x_i-\mu)^2.
\end{equation}
\end{hilfssatz}
\begin{proof}[Beweis]
Wir betrachten die Vektoren $(t,\dots,t)$, also eine Gerade im
$n$-dimensionalen Raum.
$\bar x$ ist so gewählt, dass $(\bar x,\dots,\bar x)$
der Fusspunkt des Lotes von $(x_1,\dots,x_n)$ auf die Gerade ist.
Insbesondere bilden die Punkte $(\bar x,\dots,\bar x)$, $(\mu,\dots,\mu)$
und $(x_1,\dots,x_n)$ bilden ein rechtwinkliges Dreieck.
Dies erlaubt,
die Entfernung zwischen $(\mu,\dots,\mu)$ und $(x_1,\dots,x_n)$ 
als Hypotenuse mit Hilfe des Satzes von Pythagoras zu berechnen:
\[
\sum_{i=1}^n(x_i-\mu)^2
=\sum_{i=1}^n(\bar x-\mu)^2+\sum_{i=1}^n(x_i-\bar x)^2,
\]
die Behauptung des Hilfssatzes.
\end{proof}

