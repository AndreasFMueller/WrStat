\section{Konfidenzintervalle} \label{section-konfidenzintervalle}
Mit Hilfe eines Schätzers können die Parameter einer Verteilung
geschätzt werden.
Wenn wir zum Beispiel bei einer Messapparatur
wissen, dass die Messwerte normalverteilt sind, können wir durch
Mittelwertbildung eine gute Schätzung für den Erwartungswert finden.
Da der Schätzwert selbst eine Zufallsvariable ist, kann er
im schlimmsten Fall ziemlich weit weg vom Erwartungswert zu liegen
kommen.
Wir möchten herausfinden, wie wahrscheinlich dies ist.

Das Problem wäre offensichtlich gelöst, wenn wir ein Intervall
angeben könnten, in dem der wahre Wert des Erwartungswertes mit
grosser Wahrscheinlichkeit $1-\alpha$ liegen wird.
In der Praxis
wird man für $\alpha$ kleine Werte wählen, zum Beispiel $0.05$ oder $0.01$.
Ein solches Intervall heisst $1-\alpha$-Konfidenzintervall für den
Parameter.

\begin{definition}
Ein Intervall $[L(X_1,\dots,X_n),R(X_1,\dots,X_n)]$
heisst ein $1-\alpha$-Konfidenzintervall
für den Parameter $\vartheta$, wenn der wahre Wert des Parameters
$\vartheta$ mit Wahrscheinlichkeit höchstens $\alpha$ ausserhalb
des Intervalls liegt.
\end{definition}

\subsection{Konfidenzintervall bei bekannter Varianz}
Nehmen wir an, die Varianz der Verteilung $\sigma^2$ sei uns bereits bekannt.
Dann ist $\bar X$ eine normalverteilte Zufallsvariable mit Varianz
$\sigma^2/n$, deren Erwartungswert mit dem gesuchten Erwartungswert von $X$
übereinstimmt. 

Wäre $\mu$ bekannt, wäre es recht einfach, ein Intervall zu finden,
in dem sich der Wert von $\bar X$ mit Wahrscheinlicheit $1-\alpha$ befinden
wird.
Mit Hilfe der Verteilungsfunktion $F$ der Standardnormalverteilung
könnten wir zum Beispiel die Werte $x_-$ und $x_+$ finden, für die
gilt $F(x_-)=\frac{\alpha}{2}$ und $F(x_+)=1-\frac{\alpha}{2}$, für
diesen Zweck gibt es spezielle Tabellen, zum Beispiel
\ref{tabelle-normalquantilen}.
Dann hat das Intervall
$[\mu+\sigma x_-,\mu+\sigma x_+]$ die Eigenschaft, dass $\bar X$
mit Wahrscheinlichkeit $1-\alpha$ darin enthalten sein wird.

Nun ist zwar $\mu$ nicht bekannt, aber wenn
$\bar X\in[\mu+\frac{\sigma}{\sqrt{n}} x_-,\mu+\frac{\sigma}{\sqrt{n}} x_+]$,
dann ist sicher auch
$\mu\in[\bar X+\frac{\sigma}{\sqrt{n}} x_-,\bar X+\frac{\sigma}{\sqrt{n}} x_+]$
d.~h.~wir haben ein
Intervall gefunden, in dem sich der Parameter mit Wahrscheinlichkeit $1-\alpha$
befindet.

\subsection{Konfidenzintervall mit geschätzer Varianz}
Im allgemeinen ist die Varianz jedoch nicht bekannt, und wir müssen
auch für die Varianz eine Schätzung verwenden.
Bei bekannter Varianz
konnte ein geeignetes Intervall gefunden werden, indem die
standardnormalverteilte Zufallsvariable
$\frac{\bar X-\mu}{\sigma/\sqrt{n}}$
untersucht wurde.
Die Normalverteilung definierte Intervallgrenzen,
für die
\begin{equation}
P\biggl(x_-\le
\frac{\bar X-\mu}{\sigma/\sqrt{n}}
\le x_+\biggr)=1-\alpha
\label{konfidenzintervall-bekannte-varianz}
\end{equation}
gilt, woraus sich dann das Konfidenzintervall ergab.

Da nun die Varianz auch geschätzt werden muss, ersetzen wir in
(\ref{konfidenzintervall-bekannte-varianz})
$\sigma$ durch $S$ und versuchen wieder Grenzen
$x_-$ und $x_+$ zu finden, so dass
\begin{equation}
P\biggl(x_-\le
\frac{\bar X-\mu}{S/\sqrt{n}}
\le x_+\biggr)=1-\alpha
\label{konfidenzintervall-geschaetzte-varianz}
\end{equation}
gilt.
Dazu muss die Verteilung der Zufallsvariable
\begin{equation}
\sqrt{n}\frac{\bar X-\mu}{S}
=\frac{\sqrt{n}(\bar X-\mu)/\sigma}{\sqrt{(n-1)S^2/\sigma^2(n-1)}}
\label{konfidenzintervall-verteilung}
\end{equation}
bekannt sein.
Im Nenner ist $(n-1)S^2/\sigma^2$ eine
$\chi_{n-1}^2$-verteilte Zufallsvariable,
der Zähler ist standardnormalverteilt.
Der Quotient ist über den
vorliegenden Fall hinaus von Bedeutung:

\begin{definition}
Ist $Z$ eine standardnormalverteilte Zufallsvariable, und $V$ ein
$\chi_k^2$, dann heisst die Verteilung von
\[
t=\frac{Z}{\sqrt{V/k}}
\]
die $t$-Verteilung mit $k$ Freiheitsgraden.
\end{definition}

\begin{satz}Die Wahrscheinlichkeitsdichte der $t$-Verteilung ist
\begin{equation}
\varphi_t(t)
=
\frac{\Gamma(\frac{k+1}{2})}{\sqrt{\pi k}\Gamma(\frac{k}2)}
\biggl(1+\frac{t^2}{k}\biggr)^{-\frac{k+1}2}.
\end{equation}
\end{satz}

\begin{proof}[Beweis]
Die Dichtefunktion von $Z$ und $V$ sind
\begin{align*}
\varphi_Z(x)
&=
\frac1{\sqrt{2\pi}}e^{-\frac{x^2}2}
\\
\varphi_V(x)
&=
\gamma_{\frac12,\frac{k}2}(x)
=
\frac1{\Gamma(\frac{k}2)}\frac1{2^\frac{k}2}x^{\frac{k-2}2}e^{-\frac12x}.
\end{align*}
Die Dichtefunktion von $V/k$ ist $\varphi_{V/k}(x)=k\varphi_V(kx)$, aus
den Resultaten der Abschnitte \ref{verteilungsfunktion-wurzel}
und \ref{verteilungsfunktion-quotient} lässt sich jetzt die
Dichtefunktion für $T$ berechnen.

Zunächst berechnen wir die Dichtefunktion für den Nenner:
\[
\varphi_{\sqrt{V/k}}(x)=2x\varphi_{V/k}(x^2)=2xk\varphi_V(kx^2).
\]
Die Dichtefunktion für den Quotienten ist dann
\begin{align*}
\varphi_T(t)
&=
\int_0^\infty \varphi_X(ty)y\varphi_Y(y)\,dy
\\
&=
\int_0^\infty \frac1{\sqrt{2\pi}}e^{-\frac{(ty)^2}2}y\cdot
2yk\varphi_V(ky^2)\,dy
\\
&=
\int_0^\infty \frac1{\sqrt{2\pi}}e^{-\frac{(ty)^2}2}y\cdot
2yk
\frac1{\Gamma(\frac{k}2)}\frac1{2^{\frac{k}2}}(ky^2)^{\frac{k-2}2}e^{-\frac12ky^2}
\,dy
\\
&=
\frac1{\sqrt{k\pi}2^{\frac{k-1}2}\Gamma(\frac{k}2)}
\int_0^\infty e^{-\frac12(1+t^2/k)y^2} \,dy.
\end{align*}
Mit Hilfe der Substitution $s=\frac12(1+\frac{t^2}n)y^2$
oder
\[
y=\frac{2^{\frac12}s^{\frac12}}{(1+\frac{t^2}n)^{\frac12}}
\]
wir daraus
\begin{align*}
\varphi_T(t)
&=
\frac{2^{\frac12}2^{\frac{k}2}}{\sqrt{k\pi}2^{\frac{k-1}2}\Gamma(\frac{k}2)(1+\frac{t^2}k)^{\frac{k+1}2}}
\int_0^\infty \frac12e^{-s}s^{\frac{k-1}2}\,ds
\\
&=
\frac{1}{\sqrt{k\pi}\Gamma(\frac{k}2)(1+\frac{t^2}k)^{\frac{k+1}2}}
\int_0^\infty e^{-s}s^{\frac{k+1}2-1}\,ds
\\
&=
\frac{\Gamma(\frac{k+1}2)}{\sqrt{k\pi}\Gamma(\frac{k}2)}\biggl(1+\frac{t^2}k\biggr)^{-\frac{k+1}2}.
\qedhere
\end{align*}
\end{proof}

Auch zur $t$-Verteilung existieren Tabellen ähnlich der Normalverteilung,
mit denen sich Werte $t_-$ und $t_+$ finden lassen, sodass
$P(t_-\le t\le t_+)=1-\alpha$.
Mit diesen Werten lässt sich dann
auch ein Konfidenzintervall für den Erwartungswert geben:

\begin{satz}
Ist $X$ eine normalverteilte Zufallsvariable, und $X_1,\dots,X_n$ eine
Stichprobe, $\bar X$ der Stichprobenmittelwert, 
und $S^2$ die Stichprobenvarianz.
Seien $t_-$ und $t_+$ so bestimmt, dass
$P(t_-\le t_{n-1}\le t_+)=1-\alpha$ für eine $t$-Verteilung $t_{n-1}$ mit
$n-1$ Freiheitsgraden.
Dann ist
\begin{equation}
\biggl[\bar X+t_-\frac{S}{\sqrt{n}},\bar X+t_+\frac{S}{\sqrt{n}}\biggr]
\end{equation}
ein $1-\alpha$-Konfidenzintervall für den Erwartungswert von $X$.
\end{satz}

\subsubsection{Konfidenzintervall für Juli-Durchschnittstemperatur}
Die Wetterstation in Altendorf hat für Juli 2003 eine Durchschnittstemperatur
von $21.638^\circ\text{C}$ gemessen, bei einer Stichprobenvarianz von
$493.230\text{K}^2$ und $n=44473$.
Der Tabelle der $t$-Verteilung
\ref{tabelle-tverteilung} entnimmt man 
für derart
viele Freiheitsgrade und $\alpha=0.01$ die Werte
$t_{\pm}=\pm2.5758$.
Ein $0.99$-Konfidenzintervall für die
Durchschnittstemperatur ist daher
$[21.332, 21.944]$.

