\section{Zufallsvariable}
In einem Experiment, das mit Hilfe des Ohmschen Gesetzes der Widerstand eines
Bauteils ermitteln soll, wird ein Strom durch das Bauteil geleitet und die
darüber abfallende Spannung gemessen.
Der Versuchsausgang ist also mindestens ein Paar $\omega=(U,I)$ von Spannung und
Strom, möglicherweise aber auch noch mehr, zum Beispiel die Temperatur, die
auch noch gemessen wurde.
Der Versuchsausgang ist also keine Zahl, aber es lassen sich daraus zwei
Zahlen ableiten.
Zu jedem Versuchsausgang $\omega$ gibt es die Zahlen $U(\omega)$
und $I(\omega)$, die in diesem Versuch gemessenen Spannung und der Strom.

Roulette als das Glücksspiel par excellence besteht aus einzelnen
Spielen, in denen jeweils eine Kugel im Roulettekessel in eines von
37 Fächern fällt.
Der Versuchsausgang ist also eine ganze Zahl zwischen $0$ und $36$.
Für den Spieler ist diese Zahl jedoch nur mittelbar von Bedeutung.
Ihn interessiert vor allem, ob die Chips, die er auf dem Roulette-Tisch
gesetzt hat, etwas gewonnen haben, ob als für einen Chip das Ereignis
``Einsatz gewinnt'' eingetreten ist.
Nach den Regeln des Spiels kann man dann aus dem Versuchsausgang den
Gewinn des Spielers ableiten.

Zahlen, die am Ende eines Zufallsprozesses stehen, können also durch
folgendes Modell beschrieben werden.
Zunächst wird ein Experiment durchgeführt, bei welchem wie bisher
ein Versuchsausgang $\omega$ aus den möglichen Versuchsausgängen $\Omega$
ausgewählt wird.
Dann wird aus dem Versuchsausgang über einen deterministischen Prozess
eine Zahl $X(\omega)$ abgeleitet.

\begin{definition}
\index{Zufallsvariable}
\index{Zufallsvariable!stetige}
\index{Zufallsvariable!diskrete}
Eine {\em Zufallsvariable} ist eine Funktion $\Omega\to\mathbb R$.
Eine {\em diskrete} Zufallsvariable nimmt nur diskrete Werte in $\mathbb R$ an.
Bei einer {\em stetigen} Zufallsvariable sind beliebige Werte $X(\omega)\in\mathbb R$
möglich.
\end{definition}

Mit einer Zufallsvariablen $X\colon\Omega\to\mathbb R$ kann man wieder
neue Ereignisse definieren
\begin{align*}
\{X=a\}&=\{\omega\in\Omega\,|\,X(\omega)=a\}
\\
\{a<X\le b\}
&=
\{\omega\in\Omega\,|\, a<X(\omega)\le b\}.
\end{align*}
Bei einer stetigen Zufallsvariable sind die Ereignisse der Form
$\{X=a\}$ nur von beschränktem Nutzen, da nur ganz wenige Versuche
dazu führen werden, dass das Ereignis eintritt.

\subsection{Wurf eines Würfels}
Wirft man einen Würfel, zeigt dieser ein Bild mit einem oder mehreren Punkten, 
dies sind die Elementarereignisse:
\[
\Omega = \{
\epsdice{1},
\epsdice{2},
\epsdice{3},
\epsdice{4},
\epsdice{5},
\epsdice{6}
\}.
\]
Leute, die Zahlen kennen, können diese Bilder auch als Zahlen interpretieren.
Sie verwenden die Abbildung
\[
X\colon \Omega \to \{1,2,3,4,5,6\}
\]
mit den Werten
\begin{align*}
\epsdice{1}&\mapsto 1,\\
\epsdice{2}&\mapsto 2,\\
\epsdice{3}&\mapsto 3,\\
\epsdice{4}&\mapsto 4,\\
\epsdice{5}&\mapsto 5,\\
\epsdice{6}&\mapsto 6.
\end{align*}
Dass diese Zuordnung von Elementarereignissen zu Zahlenwerten tatsächlich
ein zusätzlicher Schritt ist, sieht man zum Beispiel auch daran, dass
in Spielen für Kinder, die die Zahlen noch nicht kennen, Farbwürfel
verwendet werden, und keine Zuordnung von Zahlenwerten vorgenommen wird.

Die Menge $\Omega$ der Elementarereignisse beschreibt alle möglichen Würfe
des Würfels.
Wenn wir uns nur für die gezeigte Augenzahl interessieren,
brauchen wir eine Funktion, welche jedem Elementarereignis diesen speziellen
Ausgang zuordnet:
\[
X: \Omega\to{\mathbb Z}: \omega\mapsto X(\omega).
\]
Eine solche Abbildung heisst eine Zufallsvariable.
Mit Hilfe der Zufallsvariable kann man jetzt das Ereignis, dass eine Sechs
gewürfelt wurde, auch so formulieren:
\[
A_6=\{\omega\in\Omega\;|\;X(\omega) = 6\}.
\]
In $\mathbb{Z}\subset\mathbb{R}$ kann natürlich gerechnet werden, und
auch eine intuitive Vorstellung davon, was ``im Mittel'' etwa heissen
könnte, ist bereits vorhanden.

\subsection{Ein technisches Detail}
Wir hatten früher bemerkt, dass die Menge aller Teilmengen nicht unbedingt
als die Menge der Ereignisse geeignet ist, und dazu die Menge
$\cal A$ der Ereignisse eingeführt.
Wenn aber nicht alle Teilmengen von $\Omega$ Ereignisse sind,
dann ist auch nicht automatisch sichergestellt, dass die
Mengen $\{X < a\}$ für jeden Wert von $a$  Ereignisse sind.
Die technisch korrekte Definition einer Zufallsvariablen ist daher:

\begin{definition}
Eine Zufallsvariable $X$ ist eine Funktion $X\colon \Omega\to \mathbb R$,
die zusätzlich die Eigenschaft hat, dass die Mengen 
$\{\omega | X(\omega) < x\}$ Ereignisse sind.
\end{definition}

Für unsere Zwecke hat dieses Detail keine Konsequenzen, da alle Zufallsvariablen,
die wir betrachten, diese Bedingung automatisch erfüllen.

\subsection{Rechnen mit Zufallsvariablen}
Selbstverständlich können Zufallsvariable auch andere Wertebereiche haben.
Eine Abbildung $X:\Omega\to W$ ist eine $W$-wertige Zufallsvariable.
Alle Operationen im Wertebereich sind auch für Zufallsvariable möglich.
Kann man im Wertebereich $W$ addieren, dann ist auch die Summe zweier
Zufallsvariable $X_1$ und $X_2$ definiert durch
\[
X_1+X_2:\Omega\to W:\omega\mapsto(X_1+X_2)(\omega) := X_1(\omega) + X_2(\omega),
\]
und entsprechend für alle anderen Operationen in $W$.

Bei einer analogen Messung ist zum Beispiel jeder beliebige reelle
Wert möglich, eine reelle Zufallsvariable wäre demnach eine Abbildung
\[
X:\Omega\to\mathbb R:\omega\mapsto X(\omega)\in\mathbb R.
\]
Die Messung von Strom und Spannung an einem Verbraucher definiert zwei
Zufallsvariable $I:\Omega\to\mathbb R$ und $U:\Omega\to \mathbb R$.
Die vom Verbraucher verbrauchte Leistung ist die Zufallsvariable
\[
U\cdot I:\Omega\to\mathbb R: \omega\mapsto (U\cdot I)(\omega) = U(\omega)I(\omega).
\]
Eine komplexe Zufallsvariable wäre die Impedanz
\index{Impedanz}
\[
Z:\Omega\to\mathbb C:\omega\mapsto Z(\omega)\in\mathbb C,
\]
Realteil $\operatorname{Re}Z$ und Imaginärteil $\operatorname{Im}Z$
sind ebenfalls zwei Zufallsvariablen,
für die $Z=\operatorname{Re}Z + i \operatorname{Im}Z$ gilt.

