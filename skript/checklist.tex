%
% checklist.tex -- Checkliste zur Pruefungsvorbereitung
%
% (c) 2009 Prof. Dr. Andreas Mueller, HSR
%
\documentclass[a4paper,12pt,twocolumn]{article}
\usepackage{geometry}
\geometry{papersize={200mm,280mm},total={180mm,250mm}}
\usepackage{german}
\usepackage{times}
\usepackage{alltt}
\usepackage{verbatim}
\usepackage{fancyhdr}
\usepackage{amsmath}
\usepackage{amssymb}
\usepackage{amsfonts}
\usepackage{amsthm}
\usepackage{textcomp}
\usepackage{graphicx}
\usepackage{array}
\usepackage{ifthen}
\usepackage{multirow}
\usepackage{txfonts}
\usepackage{paralist}
\begin{document}
\title{Pr"ufungsvorbereitungscheckliste:\\ Wahrscheinlichkeitsrechung und Statistik}
\date{}
\maketitle
\section{Begriffe}
\begin{compactenum}
\item Variationen, Kombinationen, Permutationen
\item Produktregel
\item Ereignis
\item um"ogliches und sicheres Ereignis
\item Wahrscheinlichkeit
\item Laplace-Ereignisse
\item Rechenregeln f"ur Wahrscheinlichkeit
\item Bedingte Wahrscheinlichkeit
\item Unabh"angige Ereignisse
\item Satz von Bayes
\item Satz von der totalen Wahrscheinlichkeit
\item Google-Matrix
\item Zufallsvariable
\item Unabh"angige Zufallsvariablen
\item Erwartungswert
\item Varianz
\item Rechenregeln f"ur Erwartungswert und Varianz
\item Verteilungsfunktion
\item Wahrscheinlichkeitsdichte
\item Wahrscheinlichkeitsdichte und Erwartungswert/Varianz
\item Median
\item Lineare Regression
\item Normalverteilung
\item Exponentialverteilung
\item Gleichverteilung
\item Binomialverteilung
\item Poissonverteilung
\item Potenzverteilung
\item Konfidenzinterval
\item $t$-Verteilung
\item Nullhypothese
\item $\chi^2$-Verteilung
\item Freiheitsgrade
\item Kolmogorov-Smirnov-Test
\item Kalman-Filter
\end{compactenum}
\vfill
\section{Fragen}
\begin{compactenum}
\item Was ist der Unterschied zwischen dem sicheren Ereignis und
einem Ereignis mit Wahrscheinlichkeit 1?
\item Was ist der Unterschied zwischen dem unm"oglichen Ereignis
und einem Ereignis mit Wahrscheinlichkeit 0?
\item Wozu dient der Satz von Bayes?
\item Was besagt die Ungleichung von Chebychev?
\item Was besagt das Gesetz der grossen Zahl von Bernoulli?
\item Was besagt der zentrale Grenzwertsatz?
\item Wie f"uhrt man einen Hypothesentest durch?
\item Wie berechnet man den Erwartungswert einer Zufallsvariablen, deren
Verteilungsfunktion gegeben ist?
\item Wie sch"atzt man $\mu$ und $\sigma$ einer normalverteilten
Zufallsvariable?
\item Wie h"angen Poisson-Verteilung und Exponentialverteilung zusammen?
\item Welche Voraussetzungen m"ussen erf"ullt sein, damit ein $\chi^2$-Test
durchgef"uhrt werden kann?
\item Was bedeutet es, wenn jemand sagt, mit Statistik h"atte man beweisen
k"onnen, dass $X$ gilt?
\item Was ist der Unterschied zwischen mathematischen und
naturwissenschaftlichen Erkenntnissen, und welche Rolle spielt die
Wahrscheinlichkeitsrechnung dabei?
\item F"ur welche Anwendungen ist die Exponentialverteilung geeignet?
\item F"ur welche Anwendungen ist die Normalverteilung geeignet?
\item F"ur welche Anwendungen ist die Binomialverteilung geeignet?
\item Wie kann man die Binomialverteilung approximieren (zwei M"oglichkeiten)?
\item Warum braucht man den Satz von Bayes, um einen Spamfilter zu bauen?
\item Was hat Google-Pagerank mit dem Satz von der totalen Wahrscheinlichkeit
zu tun?
\item Was ist der Gini-Koeffizient?
\item Warum kann der Kalman-Filter bessere Resultate erreichen als viele
andere Arten von Filtern?
\end{compactenum}
\end{document}
