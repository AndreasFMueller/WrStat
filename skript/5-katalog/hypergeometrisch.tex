%
% hypergeometrisch.tex -- Abschnitt über die hypergeometrische Verteilung in Kap 5
%
% (c) 2015 Prof Dr Andreas Mueller, Hochschule Rapperswil
%
\subsection{Hypergeometrische Verteilung} \label{section-hypergeometrischeverteilung}
\index{hypergeometrische Verteilung}
\index{hypergeometrische Verteilung!Erwartungswert}
\index{hypergeometrische Verteilung!Varianz}
\index{Varianz!hypergeometrische Verteilung}
\index{Erwartungswert!hypergeometrische Verteilung}
\begin{table}
\renewcommand{\arraystretch}{1.5}
\begin{center}
\begin{tabular}{|l|l|}
\hline
Name&Hypergeometrische Verteilung\\
\hline
Wahrscheinlichkeit&
\begin{minipage}{3.7in}
\vskip3pt
$\displaystyle
h(k|N;m;n)=
\left.\binom{M}{k}\binom{N-M}{n-k}\right/\binom{N}{n}
$
\end{minipage}
\\[10pt]
Erwartungswert&$\displaystyle n\frac{M}{N}$\\[10pt]
Varianz&$\displaystyle
n\frac{M(N-M)(N-n)}{N^2(N-1)}
$\\[10pt]
\hline
Anwendungen&\begin{minipage}{3.7in}%
\vskip3pt
\strut
$\bullet$ Lotto\\
$\bullet$ Auswahl von $n$ Elementen aus $N$ so dass $k$ von $M$ markierten
Elementen gewählt werden
\strut
\end{minipage}\\[18pt]
\hline
\end{tabular}
\end{center}
\caption{Datenblatt der hypergeometrischen Verteilung\label{datenblatt:hypergeometrischeverteilung}}
\end{table}

Die hypergeometrische Verteilung wurde bereits in
(\ref{hypergeometrische-verteilung}) angetroffen.
$h(k|N;M;n)$ ist die
Wahrscheinlichkeit dass in einer $n$ Elemente umfassenden Stichprobe aus
einer Grundgesamtheit von $N$ Elementen, von denen $M$ eine spezielle
Eigenschaft besitzen, $k$ Elemente mit der Eigenschaft zu finden sind.
Es ist
\[
h(k|N;M;n)=\frac{\binom{M}{k}\binom{N-M}{n-k}}{\binom{N}{n}}.
\]
Da $h$ eine Verteilung ist, ist die Summe aller Wahrscheinlichkeiten
für alle in Frage kommenden $k$
\[
\sum_{k=0}^nh(k|N;M;n)=1.
\]
\subsubsection{Erwartungswert und Varianz}
Der Erwartungswert ist
\begin{align*}
E(X)
&=\sum_{k=0}^n k\cdot\frac{\binom{M}{k}\binom{N-M}{n-k}}{\binom{N}{n}}\\
&=\sum_{k=1}^n \frac{M\binom{M-1}{k-1}\binom{N-M}{n-k}}{\frac{N}{n}\binom{N-1}{n-1}}\\
&=n\frac{M}{N}\sum_{k=0}^{n-1} \frac{\binom{M-1}{k}\binom{N-1-(M-1)}{n-1-k}}{\binom{N-1}{n-1}}\\
&=n\frac{M}{N}\sum_{k=0}^{n-1}h(k|N-1;n-1;M-1)\\
&=n\frac{M}{N}.
\end{align*}
Unter dem letzten Summenzeichen stehen die Wahrscheinlichkeiten für alle
Fälle einer hypergeometrischen Verteilung in Stichproben, die um eins kleiner
sind, deren Summe ist natürlich 1.

Analog kann man bei der Berechnung der Varianz vorgehen:
{\allowdisplaybreaks
\begin{align*}
E(X^2)
&=\sum_{k=0}^n k^2h(k|N;M;n)\\
&=n\frac{M}{N}\sum_{k=0}^{n-1} (k+1)\cdot h(k|N-1;M-1;n-1)\\
&=n\frac{M}{N}(n-1)\frac{M-1}{n-1}\biggl(\sum_{k=0}^{n-2} h(k|N-2;M-2;n-2)+1\biggr)\\
&=n\frac{M}{N}\biggl((n-1)\frac{M-1}{N-1}+1\biggr)\\
\operatorname{var}(X)
&=n\frac{M}{N}\biggl((n-1)\frac{M-1}{N-1}+1-n\frac{M}{N}\biggr)\\
&=n\frac{M(N-M)(N-n)}{N^2(N-1)}.
\end{align*}
}
\begin{satz}
Eine hypergeometrisch mit den Parametern $M$, $N$ und $n$
verteilte Zufallsgrösse $X$
hat Erwartungswert
\[
E(X)=n\frac{M}{N}
\]
und Varianz
\[
\operatorname{var}(X)=n\frac{M(N-M)(N-n)}{N^2(N-1)}.
\]
\end{satz}
