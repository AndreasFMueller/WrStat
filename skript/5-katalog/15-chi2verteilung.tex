%
% chi2verteilung.tex -- \chi^2-Verteilung
%
% (c) 2015 Prof Dr Andreas Mueller, Hochschule Rapperswil
%
\subsection{\texorpdfstring{$\chi^2$}{Chi hoch 2}-Verteilung} \label{chi2verteilung}
\index{chi@{$\chi^2$-Verteilung}}
Der zentrale Grenzwertsatz garantiert, dass in vielen praktisch wichtigen
Fällen annähernd normalverteilte Zufallsvariablen vorliegen.
Wenn man nun also weiss, dass die Zufallsvariablen $(X_i)_{1\le i\le n}$
mit Erwartungswert $\mu_i$ und Varianz $\sigma_i^2$ normalverteilt
sind, dann sind die Zufallsvariablen $Y_i=(X_i-\mu_i)/\sigma_i$
standardnormalverteilt.
Die meisten $Y_i$ werden also Werte nahe
bei $0$ annehmen, die Summe $\sum_{i=1}^nY_i^2$ wird also im
Allgemeinen klein sein.
Grosse Werte der Summe sind sehr unwahrscheinlich.
Treten sie bei einer Beobachtung trotzdem auf, dann ist entweder genau
dieser unwahrscheinliche Fall eingetreten, oder die Annahmen, dass die
$X_i$ normalverteilt waren, ist nicht zutreffend.
Somit lässt sich
aus der Grösse ein praktisch nützlicher Test konstruieren, ob
eine Normalverteilung vorliegt.


\begin{definition}
Sind $(X_i)_{1\le i\le n}$ standardnormalverteilte Zufallsvariable,
dann ist $\sum_{i=1}^nX_i^2$ eine $\chi^2$-verteilte Zufallsvariable
mit $n$ Freiheitsgraden.
\end{definition}

Es ist einigermassen erstaunlich, dass sich trotz all der beim Integrieren
von Normalverteilungsdichten getroffenen Probleme für die $\chi^2$-Verteilung
geschlossene Formeln angeben lassen.
In der Praxis sind diese Formeln zwar von untergeordneter Bedeutung, da
man praktische Probleme mit Tabellen wie im Anhang lösen kann.
Die nachfolgenden Entwicklungen sollen daher vor allem als Illustration
dienen, wie sich aus bekannten Verteilungen für die Anwendungen wichtige
neue Verteilungen konstruieren lassen.

{\small

\begin{satz}\label{chi2}
Die $\chi^2$-Verteilung mit $n$ Freiheitsgraden hat
die Wahrscheinlichkeitsdichte
\[
\gamma_{\frac12,\frac{n}2}(x)
=
\frac1{\Gamma(\frac{n}2)}\sqrt{\frac1{2^n}x^{n-2}}e^{-\frac12x}\qquad,x\ge 0.
\]
Die momenterzeugende Funktion der $\chi^2$-Verteilung ist
\[
\chi_n^2(t)=(1-2t)^{-\frac{n}2}.
\]
\end{satz}

Um diesen Satz zu beweisen, ist es nützlich, mehr Information über die
Dichtefunktionen zu sammeln.
Zunächst sind die Dichtefunktionen
$\gamma_{\frac12,\frac{n}2}$ Elemente einer viel grösseren Familie von
Dichtefunktionen, nämlich den Gamma-Verteilungen.

\subsubsection{Die Gamma-Verteilungen}
\index{gamma@{$\gamma$-Verteilung}}
\begin{definition}
Die Verteilung mit der Dichtefunktion
\[
\gamma_{\alpha,\nu}(x)=\begin{cases}
\displaystyle \frac1{\Gamma(\nu)}\alpha^\nu x^{\nu-1}e^{-\alpha x}&\qquad x>0\\
0&\qquad\text{sonst}
\end{cases}
\]
heist Gamma-Verteilung zum Parameter $(\alpha,\nu)$.
Darin ist
\[
\Gamma(t)=\int_0^\infty x^{t-1}e^{-x}\,dx
\]
die Gamma-Funktion.
\end{definition}

Die Gamma-Verteilungen erfüllen einige interessante Rechenregeln:
\begin{satz} Die Funktionen $\gamma_{\alpha,\nu}$ sind Wahrscheinlichkeitsdichten.
Die Gamma-Verteilungen sind unter Faltung abgeschlossen:
\[
\gamma_{\alpha,\mu}*\gamma_{\alpha,\nu}=\gamma_{\alpha,\mu+\nu}.
\]
Sind $X$ und $Y$ Gamma-verteilte Zufallsvariable, dann ist auch $X+Y$
Gamma-verteilt.
\end{satz}
\begin{proof}[Beweis]
Wir rechnen zunächst nach, dass $\gamma_{\alpha,\nu}$ tatsächlich
Wahrscheinlichkeitsdichten sind.
Dazu ist die Normierung zu prüfen:
\begin{align*}
\int_{-\infty}^\infty \gamma_{\alpha,\nu}(x)\,dx
&=\int_0^\infty\frac1{\Gamma(\nu)}\alpha^\nu x^{\nu-1}e^{-\alpha x}\,dx\\
&=\frac{1}{\Gamma(\nu)}\int_0^\infty (\alpha x)^{\nu-1}e^{-\alpha x}
\alpha\,dx\\
&=\frac1{\Gamma(\nu)}\int_0^\infty \xi^{\nu-1}e^{-\xi}\,d\xi=1.
\end{align*}
Die letzte Gleichung folgt aus der Definition der Gamma-Funktion.

Wir berechnen jetzt das Faltungsprodukt von $\gamma_{\alpha,\mu}$ und
$\gamma_{\alpha,\nu}$, wir wissen bereits, dass das Faltungsprodukt wieder
eine Wahrscheinlichkeitsdichte sein wird.
Es gilt
\begin{align*}
\gamma_{\alpha,\mu}*\gamma_{\alpha,\nu}(x)
&=\int_{-\infty}^\infty \gamma_{\alpha,\mu}(x-t)\gamma_{\alpha,\nu}(t)\,dt\\
&=\int_0^x \gamma_{\alpha,\mu}(x-t)\gamma_{\alpha,\nu}(t)\,dt\\
&=\int_0^x \frac1{\Gamma(\mu)}\alpha^{\mu}(x-t)^{\mu-1}e^{-\alpha (x-t)}
\frac1{\Gamma(\nu)}\alpha^\nu t^{\nu-1}e^{-\alpha t}\,dt\\
&=\frac{\alpha^{\mu+\nu}e^{-\alpha x}}{\Gamma(\mu)\Gamma(\nu)}\int_0^x
(x-t)^{\mu-1}t^{\nu-1}\,dt\\
&=\frac{\alpha^{\mu+\nu}e^{-\alpha x}}{\Gamma(\mu)\Gamma(\nu)}x^{\mu+\nu-1}
\int_0^1 (1-\tau)^{\mu-1}\tau^{\nu-1}\,d\tau\\
&=
\frac1{\Gamma(\mu+\nu)}\alpha^{\mu+\nu}e^{-\alpha x}
x^{\mu+\nu-1}
\cdot
\frac{\Gamma(\mu+\nu)}{\Gamma(\mu)\Gamma(\nu)}
\int_0^1 (1-\tau)^{\mu-1}\tau^{\nu-1}\,d\tau\\
&=\gamma_{\alpha,\mu+\nu}
\cdot
\frac{\Gamma(\mu+\nu)}{\Gamma(\mu)\Gamma(\nu)}
\int_0^1 (1-\tau)^{\mu-1}\tau^{\nu-1}\,d\tau,
\end{align*}
wobei für die Umformung des Integrals die Substitution $t=x\tau$ verwendet
wurde.
Somit ist das Faltungsprodukt bis auf einen konstanten Faktor
wieder eine Gamma-Verteilung, da aber die Gamma-Verteilungen bereits
normiert sind, muss der Faktor eins sein. Als Nebenprodukt erhalten
wir also die Formel
\[
\int_0^1(1-t)^{\mu-1}t^{\nu-1}\,dt
=
\frac{\Gamma(\mu)\Gamma(\nu)}{\Gamma(\mu+\nu)}.
\qedhere
\]
\end{proof}

\subsubsection{Beweis des Satzes \ref{chi2}}
Die $\chi^2$-Verteilung mit einem Freiheitsgrad ist die Verteilung von
$X^2$, wenn $X$ eine standardnormalverteilte Zufallsvariable ist.
Für $x\le 0$ ist die Verteilungsfunktion $F(x)=P(X^2\le x)=0$. 
Für $x>0$ gilt
\begin{align*}
F(x)&=P(X^2\le x)=P(-\sqrt{x}\le X\le\sqrt{x})\\
&=\frac1{\sqrt{2\pi}}\int_{-\sqrt{x}}^{\sqrt{x}}e^{-\frac12 \xi^2}\,d\xi\\
&=\sqrt{\frac{2}{\pi}}\int_0^{\sqrt{x}}e^{-\frac12\xi^2}\,d\xi.
\end{align*}
Mit Hilfe der Substitution $\xi^2=t$ oder $\xi=t^{\frac12}$ können wir
dies umformen zu
\begin{align*}
F(x)&=\frac1{\sqrt{2\pi}}\int_0^xe^{-\frac12t}t^{-\frac12}\,dt\\
&=\frac1{\sqrt{2\pi}}\int_0^x\gamma_{\frac12,\frac12}(t)
\frac{\Gamma(\frac12)}{{\frac12}^{\frac12}}\,dt.
\end{align*}
Die Ableitung nach $x$ ergibt die Dichtefunktion:
\[
\varphi(x)
=
\frac1{\sqrt{\pi}}\Gamma({\textstyle\frac12})\gamma_{\frac12,\frac12}(x).
\]
Der Wert der Gamma-Funktion
\[
\Gamma({\textstyle\frac12})
=\int_0^\infty x^{-\frac12}e^{-x}\,dx
\]
kann mit der Substitution $x=\xi^2$ berechnet werden,
\[
\Gamma({\textstyle\frac12})=\int_0^\infty \xi^{-1}e^{-\xi^2}2\xi\,d\xi
=\int_0^{\infty}e^{-\xi^2}\,d\xi.
\]
Das Integral auf der echten Seite ist bis auf einen Normierungsfaktor
das Integral über die Dichte einer Normalverteilung, also
\[
\Gamma({\textstyle\frac12})=\sqrt{\pi},
\]
woraus sich ablesen lässt, dass die Wahrscheinlichkeitsdichte 
von $X^2$ tatsächlich $\gamma_{\frac12,\frac12}$ ist.

Nun kann man vollständige Induktion zusammen mit der Faltungseigenschaft
verwenden.
Für $n=1$ ist bereits bekannt, dass die Wahrscheinlichkeitsdichte
für $n$ Summanden $X_i^2$ $\gamma_{\frac12,\frac{n}2}$ ist.
Nehmen wir an,
dass für $n-1$ Summanden die Wahrscheinlichkeitsdichte
$\gamma_{\frac12,\frac{n-1}2}$
ist, dann erhalten wir für $n$ Summanden mit der Faltungseigenschaft
\[
\gamma_{\frac12,\frac12}*\gamma_{\frac12,\frac{n-1}2}
=
\gamma_{\frac12,\frac{n}2}.
\]

Es bleibt, die momenterzeugende Funktion von $\chi_n^2$ zu berechnen.
Wegen
\[
M_{\chi_{n-1}^2}(t)M_{\chi_1^2}(t)=M_{\chi_n^2}(t)
\]
genügt es, eine Formel für $M_{\chi_1^2}(t)$ zu finden.
Die Dichte
der $\chi_1^2$-Verteilung ist
\[
\frac1{\sqrt{2\pi}}e^{-\frac{x}2}x^{-\frac12}.
\]
Nach Definition der momenterzeugenden Funktion ist
\begin{align*}
M_{\chi_1^2}(t)
&=\frac1{\sqrt{2\pi}}\int_0^\infty e^{xt}e^{-\frac{x}2}x^{-\frac12}\,dx
%\\
%&
=\frac1{\sqrt{2\pi}}\int_0^\infty x^{-\frac12}e^{x(t-\frac12)}\,dx
%\\
%&
=\frac1{\sqrt{2\pi}}\int_0^\infty x^{-\frac12}e^{-\alpha x}\,dx\\
&=\frac1{\sqrt{2\pi}}\frac1{\sqrt{\alpha}}\int_0^\infty\xi^{-\frac12}e^{-\xi}\,d\xi
%\\
%&
=\frac1{\sqrt{2\pi}}\frac1{\alpha}\Gamma(\frac12)=\frac1{2\alpha}
%\\
%&
=\frac1{\sqrt{1-2t}}=(1-2t)^{-\frac12},
\end{align*}
wobei wir die Substitutionen $-\alpha=t-\frac12$ und $\alpha x=\xi$
verwendet haben.
Somit ist
\[
M_{\chi_1^2}(t)=(1-2t)^{-\frac12},
\]
also auch
\[
M_{\chi_n^2}(t)=(1-2t)^{-\frac{n}2}.
\]
Damit der Satz \ref{chi2} vollständig bewiesen ist.

\subsubsection{Normalverteilungstest}
\index{Normalverteilungstest}
Die Zufallsvariablen $X_i$ waren unabhängig und standardnormalverteilt
vorausgesetzt worden.
Die Grösse $\sum_{i=1}^nX_i^2$ liefert einen Test dafür, ob die Grössen
$X_i$ tatsächlich standardnormalverteilt sind.
Ist zum Beispiel der
Erwartungswert der $X_i$ nicht $0$, dann wird $\sum_{i=1}^nX_i^2$
deutlich grösser.
Die Hypothese, dass die $X_i$ standardnormalverteilt
sind, kann also verworfen werden, wenn $\sum_{i=1}^nX_i^2$ eine gewisse
Schranke $M$ überschreitet.
Die Wahrscheinlichkeit, dass die Hypothese
verworfen wird, obwohl sie eigentlich zutrifft, ist
\[
P\biggl(\sum_{i=1}^nX_i^2>M\biggr).
\]
Will man die Wahrscheinlichkeit eines solchen Irrtums klein halten, muss
$M$ entsprechend gross gewählt werden.
Wie gross definiert die
$\chi^2$-Verteilung.
Soll die Fehlerwahrscheinlichkeit $p$ sein,
dann muss $M$ so gewählt werden, dass 
\[
P\biggl(\sum_{i=1}^nX_i^2>M\biggr)
=
\int_M^\infty \gamma_{\frac12,\frac{n}2}(t)\,dt = p.
\]
Die Schranke $M$ ist für verschiedene Werte von $p$ in Abhängigkeit
von der Anzahl Freiheitsgrade tabelliert, oder kann mit dem Computer
berechnet werden.
Zu diesem Zweck ist zum Beispiel in der GNU Scientific
Library die Funktion \verb+gsl_cdf_chisq_Pinv(double p, double nu)+
vorhanden, der Parameter \verb+p+ ist die Wahrscheinlichkeit eines
Fehlers, {\tt nu} ist die Zahl der Freiheitsgrade.
}
