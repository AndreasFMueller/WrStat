%SourceDoc ws-skript.tex
% a-einleitung.tex -- Einleitung zum Skript
%
% (c) 2006 Prof. Dr. Andreas Mueller, HSR
% $Id: a-einleitung.tex,v 1.9 2008/09/15 23:49:57 afm Exp $
%
\rhead{Einleitung}
\chapter*{Einleitung}
\section*{Ist Mathematik "uberhaupt anwendbar?}
\subsection*{Wahrheit in der Mathematik}
Die Mathematik ist einzigartig unter den Wissenschaften, indem
sie objektive Wahrheiten produziert.
Ausgehend von als wahr angenommenen Axiomen entwickelt sie mit 
Hilfe von Ableitungen und Beweisen neue Aussagen, S"atze, Formeln
und Berechnugnsverfahren, die ebenfalls als wahr gelten.
Ist ist durchaus m"oglich, dass einzelne Aussagen in allen
F"allen, die man bisher probiert hat, richtig sind, zum Beispiel
scheint die Goldbachsche Vermutung, dass man jede Gerade Zahl als
Summe von zwei Primzahlen schreiben kann, immer richtig zu sein.
Trotzdem gelten sie f"ur die Mathematik nicht als wahr, weil
daf"ur kein Beweis existiert.

\subsection*{Wahrheit in den Naturwissenschaften}
Die Naturwissenschaften funktionieren grundlegend anders.
Aufgrund von Naturbeobachtungen wird eine Hypothese aufgestellt,
wie das Universum funktioniert.
Unter Annahme der Hypothese kann man dann Vorhersagen machen
und sie mit Experimenten "uberpr"ufen. 
Experimente, die darauf angelegt sind, die Theorie zu best"atigen,
sind dabei wertlos,
wir glauben ja schon provisorisch an die Theorie.
Nur Experimente, die das Potential haben, die Theorie zu
wiederlegen, bringen die Wissenschaft weiter.

\subsection*{Beispiel: Newtons Graviationstheorie}
Die Newtonsche Gravitationstheorie ist ein treffendes Beispiel.
Ausgehend von Beobachtungen von Kepler vermutete Newton, dass
ein Kraftgesetz der Form
\[
F=G\frac{mM}{r^2}
\]
f"ur die Anziehung zwischen zwei beliebigen Massen $m$ und $M$ im Abstand $r$
gelten k"onnte.
Er ging damit bereits dar"uber hinaus, was man f"ur die Erkl"arung der
keplerschen Planetenbahnen ben"otigt h"atte.
Aber seine Hypothese machte Voraussagen, die man nachpr"ufen konnte.
Zum Beispiel musste man zwischen zwei beliebigen Gegenst"anden eine
Kraft messen k"onnen, was Cavendish 1798 gelang.
Und man kann die Bahnen aller Planeten und Monde im
Sonnensystem mit hoher Genauigkeit berechnen, viel genauer als die
Keplerschen Gesetze dies erm"oglichen.
Die Newtonsche Gravitationstheorie ist auf jeden Fall ein deutlicher
Fortschritt gegen"uber der Keplerschen Beschreibung.
Ist sie deshalb bewiesen? Sicher nicht!

Ist die Graviationstheorie eine naturwissenschaftliche Theorie?
Dazu m"usste sie auch wiederlegbar sein, d.~h.~es m"usste ein
Experiment geben, welches die Theorie umst"urzen kann.
Ein solches Experiment wurde tats"achlich gefunden. Da die Theorie
bei ``kleinen'' Massen und ``vern"unftigen'' Abst"anden hervorragend
funktioniert, braucht man dazu eine grosse Masse, zum Beispiel
die der Sonne, und einen K"orper, der sich sehr nahe daran
vorbei bewegt, zum Beispiel den Planeten Merkur.
Und tats"achlich bleibt nach einer sorgf"altigen Berechnung 
der Merkurbahn eine kleine Drehung derselben, die die
Newtonsche Theorie nicht erkl"aren kann.
Die Newtonsche Theorie ist zwar sehr gut aber eigentlich falsch,
sie ist nur eine Approximation.

Albert Einstein hat mit der allgemeinen Relativit"atstheorie eine
alternative Theorie formuliert, die wesentlich genauer ist.
Ihre spektakul"aren Voraussagen wie die Existenz schwarzer L"ocher,
Gravitationslinsen, die dunkle Energie und den Big Bang zeigen,
dass wir mit Ihrer Hilfe dem Verst"andnis unseres Universums 
wesentlich n"aher gekommen sind. Die gleichen Ph"anomene zeigen
aber auch, wo auch diese Theorie mit grosser Wahrscheinlichkeit
nicht mehr funktionieren wird: im Zentrum eines Schwarzen Loches oder
beim Big Bang.
Auch Einsteins Theorie ist also nur provisorisch, aber im Moment
gibt es kein Experiment und keine Beobachtung, mit welchem wir die
Theorie wiederlegen k"onnten.

Grunds"atzlich sind also naturwissenschaftliche Gesetz niemals
bewiesen, sondern immer nur noch nicht wiederlegt.
Im Gegensatz zur Mathematik k"onnen wir niemals mit
absoluter Sicherheit behaupten, dass eine Theorie wahr ist,
wir k"onnen ihr immer nur soweit trauen, als wir in der Lage sind,
sie zu "uberpr"ufen.

\subsection*{Weitere Beispiele}
Nicht nur f"ur die Wissenschaften ist es wichtig herauszufinden,
wie die Realit"at funktioniert.
Ein falsche Einsch"atzung der Realit"at kann f"ur viele katastrophale
Folgen haben.
\begin{enumerate}
\item Kinder, die geimpft wurden, erkranken kaum an oft t"odlichen
Krankheiten. Kann man dies beweisen? Nein, nat"urlich nicht, es gibt
immer Kinder, die nicht auf die Impfung ansprechen, und auch solche,
wegen einer Impfung krank werden. Wir akzeptieren also die Theorie,
dass Impfungen Krankheiten verhindern, wissen aber auch, dass es
F"alle gibt, in denen die Theorie nicht zutreffen wird.
\item Im Altertum glaubte man, dass D"amonen einen Menschen
befallen und ihn krank machen k"onnen. Entsprechend hat man Methoden
entwickelt, diese D"amonen zu vertreiben, die f"ur die betroffenen
nicht selten schlimmer waren als die Krankheit.
\item Um 1690 glaubten Einwohner der Gemeinde Salem in Massachusetts,
dass einzelne Mitglieder ihrer Gemeine vom Teufel besessen seien.
Das perfide an dieser Theorie ist, dass sie nicht wiederlegbar ist.
Da man eigentlich nicht so genau weiss, was der Teufel ist, kann man
auch kein Experiment beschreiben, mit dem man herausfinden kann, ob
jemand nicht vom Teufel besessen ist. In der Folge wurden 20 Menschen
f"ur etwas hingerichtet, wof"ur es nicht die Spur eines Beweises gibt.
\item Wir wissen heute, dass es biologisch kaum einen Unterschied
zwischen verschiedenen ``Rassen'' von Menschen gibt.
Trotzdem werden seit Jahrtausenden "uberall auf der Welt immer wieder
ganze V"olker mit dem Argument ausgerottet, dass sie eine minderwertige
Rasse darstellen. Auch der V"olkermord der Hutu an den Tutsi 1994 in Ruanda
wurde zum Teil so begr"undet.
\item Wir wissen heute, dass alle Menschein gleichermassen leiden,
wenn sie zu Sklaven gemacht werden.
Vor nicht allzu langer Zeit wurde die Sklaverei damit gerechtfertigt,
dass die versklavten V"olker gar nicht in der Lage w"aren, eine moderne
Gesellschaft zu bilden, und dass es ihnen daher in der Sklaverei besser
gehe. 2012 hat sich ein republikanischer Politiker in den USA
zu einer "ahnlichen Behauptung verstiegen. Gibt es ein Experiment, mit dem
man diese Theorie falsifizieren k"onnte?
\item Christliche Fundamentalisten (in den USA 40\% der Bev"olkerung)
behaupten, die globale Erw"armung k"onne
nicht stattfinden, weil Gott in der Bibel versprochen habe, dass er
nie mehr eine Flut "uber die Erde bringen werde, in der hunderttausende
umkommen werden.
Entsprechend verhindern sie jegliche Gegenmassnahme.
Hier liegt offenbar eine schwerwiegende Fehleinsch"atzung der
Realit"at vor, mit m"oglicherweise gravierenden Folgen auch f"ur
diejenigen, die etwas rationaler sind.
\end{enumerate}


\subsection*{Wahrheit in anderen Gebieten}
Andere Gebiete tun sich mit dem Wahrheitsbegriff deutlich schwerer.
Theologie zum Beispiel ist nicht einmal eine Wissenschaft, da es dort
immer eine unumst"ossliche Hypothese gibt, die nicht falsifiziert
werden kann. 

TV-Gerichtsdramen vermitteln manchmal den Eindruck man k"onnte durch
Anwendung von Wissenschaft zweifelsfrei den T"ater jedes beliebigen
Verbrechens ermitteln. Wenn dem so w"are, k"onnten die Rechtssprechung
ganz anders funktionieren: man l"asst die wissenschaftliche Beschreibung
durch ein Computer-Programm, welches die f"allige Strafe berechnet.

In der Realit"at funktionieren Gerichte anders. Sie entscheiden
"uber schuldig oder nicht schuldig auf Grund einer Abw"agung aller
Fakten, die "uber den Fall gesammelt worden sind. Sie suchen dabei nach
derjenigen Theorie des Tathergangs, welcher zu allen Fakten passt.
Ein einziges Faktum, welches der Theorie widerspricht, zum Beispiel
ein Alibi eines Tatverd"achtigen, vernichtet die Theorie.

\subsection*{Was kann Mathematik zur Wahrheitsfindung beitragen?}
In diesem Lichte ist zun"achst zweifelhaft, warum Mathematik
"uberhaupt etwas zur Wahrheitsfindung in der Realit"at beitragen
kann. Auf den zweiten Blick aber wird deutlich, dass jedes Urteil
"uber wahr oder nicht wahr in der Realit"at mit einer Einsch"atzung
verbunden ist, wie wahrscheinlich eine bestimmte Theorie, so
fehlerhaft sie in gewissen Details auch sein mag, die Realit"at beschreibt.

Wenn Mathematik ein zuverl"assiges Mass definieren kann f"ur die Unsicherheit,
mit der gewisse Ph"anomene in der Welt eintreten k"onnen, und einen
Kalk"ul bereitstellen kann, wie man dieses Unsicherheitsmass
berechnen kann, dann unterst"utzt sie uns bei der Bewertung der
Realit"at. 

Die zentralen Fragen, f"ur die wir von der Mathematik eine Antwort
brauchen ist also:
\begin{itemize}
\item Wie quantifiziert man ein ``Unsicherheitsmass'' auf konsistente
Art und Weise?
\item Wie berechnet man die Wahrscheinlichkeit einzelner Ausg"ange 
in einem Experiment mit vielen m"oglichen Ausg"angen?
\item Ist es m"oglich, Gesetzm"assigkeiten komplexer Prozesse zu
formulieren, selbst wenn wir deren detailierten Abl"aufe nicht
verstehen?
\end{itemize}
Die Wahrscheinlichkeitsrechnung strebt an, genauf auf diese Fragen
Antworten zu geben.

\section*{Was ist m"oglich?}
Es ist unm"oglich, die unz"ahligen Dichteschwankungen, Wirbel und
schwachen Luftstr"omungen zu beschreiben, die die Flugbahn eines
Golfballes beeinflussen. Trotzdem gelingt es dem guten Golfspieler,
den Ball sehr nahe an das Loch zu spielen, so dass er ihn mit wenigen
weiteren Schl"agen einlochen kann.

Es ist unm"oglich, die Bewegung aller Atome eines Gases zu kennen.
Im achtzehnten Jahrhundert, als die newtonsche Mechanik mit der
exakten Vorhersage aller Bewegungen im Sonnensystem grosse Triumphe
feierte, herrschte die "Uberzeugung, dass sich die Zukunft verhersagen
liesse, wenn man nur die Anfangsbedingung kennen w"urde.
Der Mathematiker Pierre-Simon Laplace (1749-1827) hat noch geglaubt hat, dass dies
wenigstens im Prinzip m"oglich sein m"usste, als er Napoleon auf die Frage
nach dem Sch"opfer entgegnete: ``Je n'avais pas besoin de cette hypoth\`ese''.

Die Quantenmechanik hat
im 20.~Jahrhundert gezeigt, dass man die notwendige Kenntnis auch
grunds"atzlich nicht haben kann. Trotzdem gelingt es der Thermodynamik,
n"utzliche Gesetzm"assigkeiten zu formulieren. Die
Vorg"ange in der Atmosph"are, in einem Verbrennungsmotor oder die
Str"omung um ein "Uberschallflugzeug oder in einem Raketenmotor
lassen sich genau genug berechnen, dass einigermassen zuverl"assige
Wetterprognosen m"oglich werden, dass man lange nicht mehr so viele
Windkanaltests f"ur die Entwicklung eines Flugzeuges vorsehen muss,
oder dass man die Eigenschaften eines Raketenmotors auf dem
Pr"ufstand nur noch experimentell best"atigen muss.

Technische Grenzen zwingen auch in der Praxis zur Anwendung statistischer
Methoden. Eine Einzelmessung eines Radarger"ates ist prinzipbedingt
von geringer
Genauigkeit. Da sich das beobachtete Flugzeug aber sehr schnell bewegen
kann, kann man nicht einfach nacheinander gemessen Positionen mitteln
und glauben, damit das Resultat zu verbessern. Auch die Positionsbestimmung
eines GPS-Empf"angers wird durch ein aufwendiges statistisches Verfahren
verbessert.

Es ist unm"oglich vorauszusagen, ob ein Haus abbrennen wird, und
trotzdem kann man Gesetzm"assigkeiten finden, die erlauben, das
Risiko bei einer grossen Zahl von H"ausern zu kalkulieren.
Versicherungen bauen ihr Gesch"aft auf der Tatsache auf, dass
sich der zu erwartende Schaden berechnen l"asst, und decken ihn
rechtzeitig durch Einzug gen"ugend hoher Pr"amien.

Es ist unm"oglich, alle Stimmen in einer politischen Abstimmung
schon kurz nach Erf"offnung der Wahllokale zu wissen. Die meisten
W"ahler haben ihre Stimme ja noch gar nicht abgegeben. Trotzdem git
es Verfahren, "uber die grosse Gesamtheit aller Stimmberechtigten
Aussagen zu machen, was medial in endlosen und pseudospannenden
Wahlsendungen ausgeschlachtet wird.

Es ist unm"oglich zu wissen, welche Zahl im Roulette beim n"achsten
Spiel angezeigt wird. Aber das Spielcasino ist in der Lage, die
zu erwartenden Gewinne der Mitspieler bei einer grossen Zahl der
Spiele abzusch"atzen und vorauszusehen, dass auf die Dauer f"ur
das Casino immer ein Gewinn, f"ur den Spieler ein Verlust herausschaut.
Es ist immer noch m"oglich, dass einzelne Spieler vereinzelt
gewinnen, dies ist f"ur Marketing-Zwecke auch wichtig, g"alte
dies jedoch f"ur eine grosse Zahl von Spielern, w"are der Betrieb
eines Spielcasinos wirtschaftlich nicht attraktiv.

\section*{Die Kunst des Vermutens}
Bei einer grossen Zahl von Einfl"ussen auf den Ausgang eines Experimentes
wird es praktisch unm"oglich, den genauen Ausgang vorauszusagen.
Trotzdem ist es m"oglich, Gesetzm"assigkeiten zu formulieren, mit
denen der Ingenieur seine Hypothesen "uber seine Konstruktion
verifizieren kann.

In vielen Bereichen der Wissenschaft und Technik ist es unm"oglich, den
genauen Ausgang eines Experimentes vorherzusagen. Trotzdem werden
Aussagen "uber den zu erwartenden Ausgang ben"otigt. So kann zum Beispiel
eine Fluggesellschaft nicht wissen, ob alle Passagiere f"ur einen
bestimmten Flug erscheinen werden. Sie hat die Wahl, f"ur jedes verkaufte
Ticket auch einen Sitzplatz bereit zu halten, oder eine gewisse Anzahl
Pl"atze doppelt zu verkaufen. Sie w"agt dabei ab zwischen der finanziellen
Ineffizienz eines nicht voll besetzten Flugzeugs und dem m"oglichen
finanziellen Schaden durch einen Passagier, der keinen Sitzplatz mehr
erhalten hat, und der auf einen anderen Flug umgebucht werden muss.
"Uber die Zahl der nicht erscheinenden Flugg"aste ist nat"urlich nichts
genaues bekannt, die Fluggesellschaft kann sich nur in der Kunst "uben,
Vermutungen dar"uber anzustellen.

Genau in diesem Sinne hat im achtzehnten Jahrhundert
Jakob Bernoulli II.~das Problem angepackt und in seinem Buch
{\em Ars conjectandi} (lat. die Kunst des Vermutens) die Kunst,
Vermutungen zur Wissenschaft zu erheben. Er schuf damit das
neue Gebiet der Wahrscheinlichkeitsrechnung. Sie stellt die theoretischen Grundlagen
bereit, auf denen Aussagen "uber eine grosse Zahl von gleichartigen
Ereignissen oder Individuen formuliert werden k"onnen.

Die Wahrscheinlichkeitsrechnung lehrt, wie auf rationale
Art Vermutungen angestellt werden k"onnen. Nat"urlich braucht diese
Kunst des Vermutens auch Fakten, auf die sich die Vermutungen
abst"utzen k"onnen. Der Natur der Sache nach sind auch diese
Fakten nicht unbedingt absolute Wahrheiten, sondern Aussagen,
die in der "uberwiegenden Mehrzahl der F"alle zutreffen werden.
Es ist die Aufgabe der Statistik, experimentelle Verfahren zu finden,
mit denen die Zuverl"assigkeit der Faktengrundlage ermittelt
werden kann. Wahrscheinlichkeit und Statistik bilden also ein
untrennbares wissenschaftliches Geschwisterpaar, welches umgangssprachlichen
Wendungen wie ``meistens'', ``selten'', ``wahrscheinlich'' oder  ``unwahrscheinlich''
einen klar definierten und messbaren Sinn gibt.

\section*{Vorgehen}
Die Kunst, Vermutungen anzustellen, und die Kunst, eine f"ur das Vermuten
geeignete Datengrundlage zu erstellen, werden in diesem Skript
nebeneinander entwickelt. Zun"achst m"ussen einige Grundbegriffe eingef"uhrt
und ihre Eigenschaften verstanden werden. Damit lassen sich sofort
viele Statistiken verstehen. Aber erst der Begriff der Zufallsvariable und
der Verteilungen schafft eine
geeignete Grundlage f"ur die Beurteilung wissenschaftlicher Messdaten.

