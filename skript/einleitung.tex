%
% einleitung.tex -- Einleitung zum Skript
%
% (c) 2006-2015 Prof. Dr. Andreas Mueller, Hochschule Rapperswil
%
\rhead{Einleitung}
\chapter*{Einleitung}
Neben den Verschwörungstheoretikern jeder Couleur findet man
auf YouTube eine eigentliche Subkultur von Wissenschaftsverweigerern,
Evolutionsgegnern, Tesla-Anbetern (alle heutigen Erfindungen sind im
Prinzip schon von Nikola Tesla gemacht worden), Chemtrailern, 
Ufo-Jägern,
Impfverweigerern, Anhängern der Theorie der flachen Erde
und Klimawandel-Leugnern.
Während man geneigt ist, die Flat-Earther als Spinner nicht weiter
zu beachten, haben Klima-Leugner, vor allem solche in Machtpositionen,
einen direkten Einfluss auf unsere Zukunft, und Impfverweigerer
gefährden durch ihre Nachlässigkeit ganz direkt Mitmenschen, die
nicht auf eine Impfung nicht ansprechen, oder nicht geimpft werden
können.

In vielen Fällen geben diese Leute durchaus rational klingende
Argumente für ihr verantwortungsloses Handeln.
Zum Beispiel behaupten Impfverweigerer, es lasse sich wissenschaftlich
beweisen, dass das Konservierungsmittel Timerosal Autismus verursache.
Oder Klimawandel-Leugner führen wissenschaftliche Studien ins Feld,
welche einen Zusammenhang zwischen Sonnenaktivität und Klima 
nachweisen sollen.
Die Befürchtungen, dass der LHC am CERN schwarze Minilöcher erzeugen
würde, wofür es angeblich auch wissenschaftliche Belege gebe,
sind dabei noch die eher vernünftigen, nach einigen Quellen soll
am 25.~September der Weltuntergang bevorstehen, weil am CERN ein
Portal in eine andere Dimension geöffnet werde, durch die Dämonen
in unsere Welt eindringen würden.
Auch dafür haben die Weltuntergangs-Apologeten wissenschaftlich
klingende Quellen.

Natürlich ändert das nichts daran, dass diese wirren Geschichten
nichts mit der Realität zu tun haben.
Oft sind die betreffenden Personen einfach nur nicht oder falsch
informiert, oder nicht in der Lage, selbst eine logische Überlegung
anzustellen.
Die meisten Flat-Earther zum Beispiel haben von elementarer Geometrie
nicht die geringste Ahnung, und Evolutionsgegner sind dies meist aus
ideologischen Gründen, ganz ohne Kenntnis dessen, was Evolution ist.
Und manchmal ist es auch das Geld, das zum Beispiel im Falle der 
Klimaerwärmung in grossen Mengen von grossen Erdölinteressen zu
jedem fliesst, der ``wissenschaftlich'' für diese Interessen
daherlügt.

Nach Abzug der Spinner, Dummköpfe und Lügner bleiben noch eine
paar übrig, die auf den ersten Blick logisch klingende Argumente
für ihre Ideen angeben können, und von diesen auch ehrlich 
überzeugt sind.
Wie kann man herausfinden, wer recht hat?
Mehrheitsentscheid funktioniert nicht, Jahrtausendelang war
die Mehrheit der Menschheit überzeugt, dass die Erde den Mittelpunkt
des Universums bildet.
Ein Gerichtsentscheid kann auch nicht funktionieren, der berühmte
Prozess gegen Galileo Galilei zeigt exemplarisch und spektatkulär
das Scheitern von Gerichten, wenn es um die Frage naturwissenschaftlicher
Korrektheit geht.
Und Traditionen sind erst recht ungeeignet, denn auch wenn tradierter Rassenhass
dunkelhäutige Menschen als minderwertig behandeln will, ändert das
nichts an der Tatsache, dass alle Menschen afrikanische Affen sind.

Auch die Mathematik kann in vielen solchen Fragen keine Aussage machen.
Sie ist zwar einzigartig unter den Wissenschaften, indem
sie objektive Wahrheiten produziert.
Ausgehend von als wahr angenommenen Axiomen entwickelt sie mit 
Hilfe von Ableitungen und Beweisen neue Aussagen, Sätze, Formeln
und Berechnugnsverfahren, die ebenfalls als wahr gelten.
Alles ist nur reine Logik, ausgehend von ein paar Definitionen.
Die Resultate der Mathematik sind daher für jeden die selben, 
sie sind unabhängig von jeglichen Voraussetzungen in der Realität.
Sie ändern sich daher auch nicht, wenn die Realität verändert wird.
Mathematische Aussagen, die sich auf die Natur beziehen, müssen daher
verifiziert werden.
Nur weil die Formeln des Standardmodells ein Higgs-Boson vorhersagen,
heisst das noch lange nicht, dass das Higgs-Boson tatsächlich existieren
muss, oder auch nur ähnliche Eigenschaften haben muss.
Ein besonderer Witz ist eine Übung in modaler Logik, die auf Gödel
zurückgeht, und die Existenz eines ``maximal grossartigen Wesens'' 
nachweisen soll. 
Selbst wenn das Argument korrekt wäre\footnote{Es behauptet im wesentlichen,
dass ein maximal grossartiges Wesen existieren müsse, weil nicht Existenz
der maximalen Grossartigkeit Abbruch tun würde.
Aber ein offenes Intervall enthält alle Zahlen, die nicht ganz so gross
sind wie das rechte Ende des Intervalls, also alle nicht ganz maximalen
Zahlen, nur eben die maximale Zahl nicht.
Das Argument ist also im wesentlichen ein Wortspiel um den Begriff eines
``maximal grossartigen'' Wesens.}, würde es niemals die Existenz eines
solchen Wesens in der Realität beweisen können, wie religiöse
Apologeten manchmal gerne daraus ableiten möchten.

\section*{Was ist Naturwissenschaft?}
Die Naturwissenschaft stellt offenbar einen Mechanismus zur Verfügung,
nach dem man objektiv herausfinden kann, ob eine Aussage wahr oder falsch ist.
Er muss über logische Argumentation und reine Mathematik hinausgehen,
wie obige Beispiele gezeigt haben.
Wie geht man also vor, um zu entscheiden, ob die Erde eine Scheibe
oder eine Kugel ist?
Oder ob die Erde die Sonne umkreist oder umgekehrt?

Die Lösung dieses Problems muss so funktionieren, dass das schwächste
Glied in der Kette, nämlich der leichtgläubige und voreingenommene
Forscher, daran gehindert wird, Opfer seiner eigenen Unzulänglichkeit
zu werden.
Natürlich kann man auch durch diese Forderung Fehler nicht vermeiden,
aber wir können durch die Forderung, dass alles durch Dritte nachvollziehbar
sein muss, eine zusätzliche Sicherheit einbauen.
Nur weil sogar Stephen Hawking sagt, die Erde sei eine Kugel, würden wir
nicht von unserer Überzeugung der flachen Erde abrücken.

Wir müssen uns zwischen den folgenden zwei Hypothesen entscheiden:
\begin{enumerate}
\item Die Erde ist eine Scheibe, die Sonne kreist etwa 6000km über der
Scheibe einmal pro Tag auf einem Kreis rund um das Zentrum über der
Scheibenmitte. 
Der Kreisradius verändert sich im Laufe das Jahres.
\item Die Erde ist eine Kugel, die sich einmal täglich um ihre
Achse dreht.
\end{enumerate}
Wir könnten uns Argumente von Befürwortern jeder These anhören,
und dann für diejenige Hypothese entscheiden, die überzeugender
gewirkt hat. 
Damit ist aber in keiner Weise sichergestellt, dass die Anwälte der
beiden Hypothesen tatsächlich über etwas gesprochen haben, was
in der Natur ein Gegenstück hat.
Es bleibt uns also nichts anderes übrig, also die Gültigkeit mit
einem Experiment zu überprüfen.

Damit dies überhaupt möglich ist, müssen wir erst aus der Hypothese
Aussagen ableiten, die mit einem Experiment überprüfbar sind.
Die Aussagen müssen für die in Frage kommenden Hypothesen gegensätzlich
sein.
Es reicht also nicht, eine Aussage zu haben und experimentell zu
verifizieren, die für das Kugelmodell der Erde zutrifft,
die Aussage darf gleichzeitig für das Scheibenmodell der Erde
nicht zutreffen.
Eine solche Aussage lässt uns entscheiden, welches der Modelle
sicher falsch ist, wir können eine Hypothese eliminieren.
Das besagt aber noch nicht, dass die andere Hypothese automatisch
``wahr'' ist, denn wir haben ja nur eine einzige Aussage dieses
Modells überprüft.
Die falsche Aussage falsifiziert ein Modell.

Zum Beispiel kann man aus dem Kugelmodell ableiten, dass der Weg
auf einem Breitenkreis auf $80^\circ$ nördlicher Breite
sehr viel kürzer sein muss als ein Weg entlang des Äquators.
Diese Aussage trifft aber auch für das Scheibenmodell zu, sie
ist also nicht geeignet, eines der Modelle zu falsifizieren.
Sie wäre höchstens geeignet, das Zylinder-Modell der Erde
zu falsifizieren.
Wenn wir aber zusätzliche die Länge eines Breitenkreises 
auf $80^\circ$ südlicher Breite vergleichen, dann sagt
das Scheibenmodell voraus, dass dieser Weg viel länger
sein muss als der Äquator, während das Kugelmodell behauptet,
dass dieser Weg gleich lang sein muss wie der Weg entlang des
$80$-ten nördlichen Breitengrades.
Damit haben wir eine Aussage gefunden, mit der die eine oder andere
Theorie falsifiziert werden kann.
Allerdings ist es technisch nicht ganz einfach, das Experiment
durchzuführen.

1838 führte Samuel Birley Rowbotham folgendes Experiment durch,
welches zwischen den beiden Hypothesen entscheiden sollte.
Er beobachtet ein Schiff, welches sich auf einem langsam fliessenden
Kanal von ihm entfernte.
Wäre die Erde eine Kugel, müsste von dem Schiff mit der Zeit
immer weniger zu sehen sein, bis es ganz hinter dem Horizont
verschwindet.
Tatsächlich konnte er das Schiff die ganze Zeit beobachten, es
war sogar im Vergleich zu einer weit entfernten Brücke, die
man der Kugel-Hypothese gar nicht hätte sehen dürfen, immer auf
gleicher Höhe.
Damit war eindeutig bewiesen, dass die Erde eine Scheibe ist.
Oder eben nicht, es war nur bewiesen, dass die Kugelhypothese
nicht zutreffen kann.

Was ist passiert? 
Durch das Design des Experimentes haben wir den ursprünglichen
Hypothesen eine neue Annahme hinzugefügt: Licht breitet sich
immer auf einer Geraden aus.
Das Experiment von Rowbotham hat also die folgenden Hypothesen miteinander
verglichen:
\begin{enumerate}
\item Die Erde ist eine Scheibe, die Sonne kreist etwa 6000km über der
Scheibe einmal pro Tag auf einem Kreis rund um das Zentrum über der
Scheibenmitte. 
Der Kreisradius verändert sich im Laufe das Jahres.
Lichtstrahlen breiten sich entlang von Geraden aus.
\item Die Erde ist eine Kugel, die sich einmal täglich um ihre
Achse dreht.
Lichtstrahlen breiten sich entlang von Geraden aus.
\end{enumerate}
Das Experiment hat die erste Hypothese falsifiziert, aber wir wissen
damit noch nicht, ob die Kugelgestalt widerlegt wurde, oder die
Gradlinigkeit der Lichtausbreitung.

Es stellte sich heraus, wie Alfred Russel Wallace später nachgewiesen,
dass im richtigen Abstand über dem Wasser das Licht durch die
veränderliche Dichte der Luft gebrochen wird.
Führt man die Beobachtung im richtigen Abstand von der Wasseroberfläche
durch, kann ein Lichtstrahl genau so gekrümmt sein wie die Erdoberfläche.
Den Effekt kann man eliminieren, indem man die Messung genügend hoch
über dem Wasser durchführt. 
Wallace stellte seine Beobachtungen in 4m Höhe durch, und konnte
so die Hypothese der flachen Erde falsifizieren.

Wir halten fest: In den Naturwissenschaften wird nichts bewiesen,
sondern es werden Hypothesen formuliert, die so gebaut sein
müssen, dass sie falsifiziert werden können.
Dann werden Experimente durchgeführt, mit denen die Falsifizierung
möglich sein könnte.
Hypothesen, die alle nur erdenklichen Herausforderungen durch
Falsifizierungsversuche überstanden haben, erhalten den Status
einer wissenschaftlichen Theorie.

Demonstrationsexperimente in Physikvorlesungen sind also nicht
``Beweise'' einer Theorie oder eines Naturgesetzes, sondern bestenfalls
Illustrationen.
Berühmte erfolgreiche Theorien sind die Gravitationstheorie, die
Theorie der elektrischen und magnetischen Felder (Elektrodynamik),
die Quantentheorie, das Standardmodell der Teilchenphysik, die
allgemeine Relativitätstheorie, die Evolution.

\section*{Beitrag der Mathematik}
Meistens sind experimentelle Resultate weniger eindeutig als man sich
wünschen würde.
Messungen sind mit Fehlern behaftet, Experimente funktionieren nicht
immer.
Und manche Experimente sind grundsätzlich nicht exakt reproduzierbar,
zum Beispiel der Wurf eines Würfels.
Trotzdem ist intuitiv klar, dass alle Zahlen eines fairen Würfels
ungefähr gleich häufig gewürfelt werden.
Auch dies ist eine Hypothese über real existierende Würfel, die
sich mit einem Experiment verifizieren lässt.
Wir brauchen also eine Mathematik, die uns Häufigkeiten und mögliche
Abweichungen davon vorherzusagen erlaubt.

\section*{Danksagung}
Über die Jahre, in denen dieses Skript seiner jetzigen Form immer näher
gekommen ist, haben viele Studierende immer wieder auf Fehler und andere
Verbesserungsmöglichkeiten hingewiesen, oft sogar in Form von Pull-Requests
im Github Repository \url{https://github.com/AndreasFMueller/WrStat}.
Ihnen sei allen herzlich gedankt.

