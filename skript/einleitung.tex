%
% einleitung.tex -- Einleitung zum Skript
%
% (c) 2006-2015 Prof. Dr. Andreas Mueller, Hochschule Rapperswil
%
\rhead{Einleitung}
\chapter*{Einleitung}
Neben den Verschw"orungstheoretikern jeder Couleur findet man
auf YouTube eine eigentliche Subkultur von Wissenschaftsverweigerern,
Evolutionsgegnern, Tesla-Anbetern (alle heutigen Erfindungen sind im
Prinzip schon von Nikola Tesla gemacht worden), Chemtrailern, 
Ufo-J"agern,
Impfverweigerern, Anh"angern der Theorie der flachen Erde
und Klimawandel-Leugnern.
W"ahrend man geneigt ist, die Flat-Earther als Spinner nicht weiter
zu beachten, haben Klima-Leugner, vor allem solche in Machtpositionen,
einen direkten Einfluss auf unsere Zukunft, und Impfverweigerer
gef"ahrden durch ihre Nachl"assigkeit ganz direkt Mitmenschen, die
nicht auf eine Impfung nicht ansprechen, oder nicht geimpft werden
k"onnen.

In vielen F"allen geben diese Leute durchaus rational klingende
Argumente f"ur ihr verantwortungsloses Handeln.
Zum Beispiel behaupten Impfverweigerer, es lasse sich wissenschaftlich
beweisen, dass das Konservierungsmittel Timerosal Autismus verursache.
Oder Klimawandel-Leugner f"uhren wissenschaftliche Studien ins Feld,
welche einen Zusammenhang zwischen Sonnenaktivit"at und Klima 
nachweisen sollen.
Die Bef"urchtungen, dass der LHC am CERN schwarze Minil"ocher erzeugen
w"urde, wof"ur es angeblich auch wissenschaftliche Belege gebe,
sind dabei noch die eher vern"unftigen, nach einigen Quellen soll
am 25.~September der Weltuntergang bevorstehen, weil am CERN ein
Portal in eine andere Dimension ge"offnet werde, durch die D"amonen
in unsere Welt eindringen w"urden.
Auch daf"ur haben die Weltuntergangs-Apologeten wissenschaftlich
klingende Quellen.

Nat"urlich "andert das nichts daran, dass diese wirren Geschichten
nichts mit der Realit"at zu tun haben.
Oft sind die betreffenden Personen einfach nur nicht oder falsch
informiert, oder nicht in der Lage, selbst eine logische "Uberlegung
anzustellen.
Die meisten Flat-Earther zum Beispiel haben von elementarer Geometrie
nicht die geringste Ahnung, und Evolutionsgegner sind dies meist aus
ideologischen Gr"unden, ganz ohne Kenntnis dessen, was Evolution ist.
Und manchmal ist es auch das Geld, das zum Beispiel im Falle der 
Klimaerw"armung in grossen Mengen von grossen Erd"olinteressen zu
jedem fliesst, der ``wissenschaftlich'' f"ur diese Interessen
daherl"ugt.

Nach Abzug der Spinner, Dummk"opfe und L"ugner bleiben noch eine
paar "ubrig, die auf den ersten Blick logisch klingende Argumente
f"ur ihre Ideen angeben k"onnen, und von diesen auch ehrlich 
"uberzeugt sind.
Wie kann man herausfinden, wer recht hat?
Mehrheitsentscheid funktioniert nicht, Jahrtausendelang war
die Mehrheit der Menschheit "uberzeugt, dass die Erde den Mittelpunkt
des Universums bildet.
Ein Gerichtsentscheid kann auch nicht funktionieren, der ber"uhmte
Prozess gegen Galileo Galilei zeigt exemplarisch und spektatkul"ar
das Scheitern von Gerichten, wenn es um die Frage naturwissenschaftlicher
Korrektheit geht.
Und Traditionen sind erst recht ungeeignet, denn auch wenn tradierter Rassenhass
dunkelh"autige Menschen als minderwertig behandeln will, "andert das
nichts an der Tatsache, dass alle Menschen afrikanische Affen sind.

Auch die Mathematik kann in vielen solchen Fragen keine Aussage machen.
Sie ist zwar einzigartig unter den Wissenschaften, indem
sie objektive Wahrheiten produziert.
Ausgehend von als wahr angenommenen Axiomen entwickelt sie mit 
Hilfe von Ableitungen und Beweisen neue Aussagen, S"atze, Formeln
und Berechnugnsverfahren, die ebenfalls als wahr gelten.
Alles ist nur reine Logik, ausgehend von ein paar Definitionen.
Die Resultate der Mathematik sind daher f"ur jeden die selben, 
sie sind unabh"angig von jeglichen Voraussetzungen in der Realit"at.
Sie "andern sich daher auch nicht, wenn die Realit"at ver"andert wird.
Mathematische Aussagen, die sich auf die Natur beziehen, m"ussen daher
verifiziert werden.
Nur weil die Formeln des Standardmodells ein Higgs-Boson vorhersagen,
heisst das noch lange nicht, dass das Higgs-Boson tats"achlich existieren
muss, oder auch nur "ahnliche Eigenschaften haben muss.
Ein besonderer Witz ist eine "Ubung in modaler Logik, die auf G"odel
zur"uckgeht, und die Existenz eines ``maximal grossartigen Wesens'' 
nachweisen soll. 
Selbst wenn das Argument korrekt w"are\footnote{Es behauptet im wesentlichen,
dass ein maximal grossartiges Wesen existieren m"usse, weil nicht Existenz
der maximalen Grossartigkeit Abbruch tun w"urde.
Aber ein offenes Intervall enth"alt alle Zahlen, die nicht ganz so gross
sind wie das rechte Ende des Intervalls, also alle nicht ganz maximalen
Zahlen, nur eben die maximale Zahl nicht.
Das Argument ist also im wesentlichen ein Wortspiel um den Begriff eines
``maximal grossartigen'' Wesens.}, w"urde es niemals die Existenz eines
solchen Wesens in der Realit"at beweisen k"onnen, wie religi"ose
Apologeten manchmal gerne daraus ableiten m"ochten.

\section*{Was ist Naturwissenschaft?}
Die Naturwissenschaft stellt offenbar einen Mechanismus zur Verf"ugung,
nach dem man objektiv herausfinden kann, ob eine Aussage wahr oder falsch ist.
Er muss "uber logische Argumentation und reine Mathematik hinausgehen,
wie obige Beispiele gezeigt haben.
Wie geht man also vor, um zu entscheiden, ob die Erde eine Scheibe
oder eine Kugel ist?
Oder ob die Erde die Sonne umkreist oder umgekehrt?

Die L"osung dieses Problems muss so funktionieren, dass das schw"achste
Glied in der Kette, n"amlich der leichtgl"aubige und voreingenommene
Forscher, daran gehindert wird, Opfer seiner eigenen Unzul"anglichkeit
zu werden.
Nat"urlich kann man auch durch diese Forderung Fehler nicht vermeiden,
aber wir k"onnen durch die Forderung, dass alles durch Dritte nachvollziehbar
sein muss, eine zus"atzliche Sicherheit einbauen.
Nur weil sogar Stephen Hawking sagt, die Erde sei eine Kugel, w"urden wir
nicht von unserer "Uberzeugung der flachen Erde abr"ucken.

Wir m"ussen uns zwischen den folgenden zwei Hypothesen entscheiden:
\begin{enumerate}
\item Die Erde ist eine Scheibe, die Sonne kreist etwa 6000km "uber der
Scheibe einmal pro Tag auf einem Kreis rund um das Zentrum "uber der
Scheibenmitte. 
Der Kreisradius ver"andert sich im Laufe das Jahres.
\item Die Erde ist eine Kugel, die sich einmal t"aglich um ihre
Achse dreht.
\end{enumerate}
Wir k"onnten uns Argumente von Bef"urwortern jeder These anh"oren,
und dann f"ur diejenige Hypothese entscheiden, die "uberzeugender
gewirkt hat. 
Damit ist aber in keiner Weise sichergestellt, dass die Anw"alte der
beiden Hypothesen tats"achlich "uber etwas gesprochen haben, was
in der Natur ein Gegenst"uck hat.
Es bleibt uns also nichts anderes "ubrig, also die G"ultigkeit mit
einem Experiment zu "uberpr"ufen.

Damit dies "uberhaupt m"oglich ist, m"ussen wir erst aus der Hypothese
Aussagen ableiten, die mit einem Experiment "uberpr"ufbar sind.
Die Aussagen m"ussen f"ur die in Frage kommenden Hypothesen gegens"atzlich
sein.
Es reicht also nicht, eine Aussage zu haben und experimentell zu
verifizieren, die f"ur das Kugelmodell der Erde zutrifft,
die Aussage darf gleichzeitig f"ur das Scheibenmodell der Erde
nicht zutreffen.
Eine solche Aussage l"asst uns entscheiden, welches der Modelle
sicher falsch ist, wir k"onnen eine Hypothese eliminieren.
Das besagt aber noch nicht, dass die andere Hypothese automatisch
``wahr'' ist, denn wir haben ja nur eine einzige Aussage dieses
Modells "uberpr"uft.
Die falsche Aussage falsifiziert ein Modell.

Zum Beispiel kann man aus dem Kugelmodell ableiten, dass der Weg
auf einem Breitenkreis auf $80^\circ$ n"ordlicher Breite
sehr viel k"urzer sein muss als ein Weg entlang des "Aquators.
Diese Aussage trifft aber auch f"ur das Scheibenmodell zu, sie
ist also nicht geeignet, eines der Modelle zu falsifizieren.
Sie w"are h"ochstens geeignet, das Zylinder-Modell der Erde
zu falsifizieren.
Wenn wir aber zus"atzliche die L"ange eines Breitenkreises 
auf $80^\circ$ s"udlicher Breite vergleichen, dann sagt
das Scheibenmodell voraus, dass dieser Weg viel l"anger
sein muss als der "Aquator, w"ahrend das Kugelmodell behauptet,
dass dieser Weg gleich lang sein muss wie der Weg entlang des
$80$-ten n"ordlichen Breitengrades.
Damit haben wir eine Aussage gefunden, mit der die eine oder andere
Theorie falsifiziert werden kann.
Allerdings ist es technisch nicht ganz einfach, das Experiment
durchzuf"uhren.

1838 f"uhrte Samuel Birley Rowbotham folgendes Experiment durch,
welches zwischen den beiden Hypothesen entscheiden sollte.
Er beobachtet ein Schiff, welches sich auf einem langsam fliessenden
Kanal von ihm entfernte.
W"are die Erde eine Kugel, m"usste von dem Schiff mit der Zeit
immer weniger zu sehen sein, bis es ganz hinter dem Horizont
verschwindet.
Tats"achlich konnte er das Schiff die ganze Zeit beobachten, es
war sogar im Vergleich zu einer weit entfernten Br"ucke, die
man der Kugel-Hypothese gar nicht h"atte sehen d"urfen, immer auf
gleicher H"ohe.
Damit war eindeutig bewiesen, dass die Erde eine Scheibe ist.
Oder eben nicht, es war nur bewiesen, dass die Kugelhypothese
nicht zutreffen kann.

Was ist passiert? 
Durch das Design des Experimentes haben wir den urspr"unglichen
Hypothesen eine neue Annahme hinzugef"ugt: Licht breitet sich
immer auf einer Geraden aus.
Das Experiment von Rowbotham hat also die folgenden Hypothesen miteinander
verglichen:
\begin{enumerate}
\item Die Erde ist eine Scheibe, die Sonne kreist etwa 6000km "uber der
Scheibe einmal pro Tag auf einem Kreis rund um das Zentrum "uber der
Scheibenmitte. 
Der Kreisradius ver"andert sich im Laufe das Jahres.
Lichtstrahlen breiten sich entlang von Geraden aus.
\item Die Erde ist eine Kugel, die sich einmal t"aglich um ihre
Achse dreht.
Lichtstrahlen breiten sich entlang von Geraden aus.
\end{enumerate}
Das Experiment hat die erste Hypothese falsifiziert, aber wir wissen
damit noch nicht, ob die Kugelgestalt widerlegt wurde, oder die
Gradlinigkeit der Lichtausbreitung.

Es stellte sich heraus, wie Albert Russel Wallace sp"ater nachgewiesen,
dass im richtigen Abstand "uber dem Wasser das Licht durch die
ver"anderliche Dichte der Luft gebrochen wird.
F"uhrt man die Beobachtung im richtigen Abstand von der Wasseroberfl"ache
durch, kann ein Lichtstrahl genau so gekr"ummt sein wie die Erdoberfl"ache.
Den Effekt kann man eliminieren, indem man die Messung gen"ugend hoch
"uber dem Wasser durchf"uhrt. 
Wallace stellte seine Beobachtungen in 4m H"ohe durch, und konnte
so die Hypothese der flachen Erde falsifizieren.

Wir halten fest: In den Naturwissenschaften wird nichts bewiesen,
sondern es werden Hypothesen formuliert, die so gebaut sein
m"ussen, dass sie falsifiziert werden k"onnen.
Dann werden Experimente durchgef"uhrt, mit denen die Falsifizierung
m"oglich sein k"onnte.
Hypothesen, die alle nur erdenklichen Herausforderungen durch
Falsifizierungsversuche "uberstanden haben, erhalten den Status
einer wissenschaftlichen Theorie.

Demonstrationsexperimente in Physikvorlesungen sind also nicht
``Beweise'' einer Theorie oder eines Naturgesetzes, sondern bestenfalls
Illustrationen.
Ber"uhmte erfolgreiche Theorien sind die Gravitationstheorie, die
Theorie der elektrischen und magnetischen Felder (Elektrodynamik),
die Quantentheorie, das Standardmodell der Teilchenphysik, die
allgemeine Relativit"atstheorie, die Evolution.

\section*{Beitrag der Mathematik}
Meistens sind experimentelle Resultate weniger eindeutig als man sich
w"unschen w"urde.
Messungen sind mit Fehlern behaftet, Experimente funktionieren nicht
immer.
Und manche Experimente sind grunds"atzlich nicht exakt reproduzierbar,
zum Beispiel der Wurf eines W"urfels.
Trotzdem ist intuitiv klar, dass alle Zahlen eines fairen W"urfels
ungef"ahr gleich h"aufig gew"urfelt werden.
Auch dies ist eine Hypothese "uber real existierende W"urfel, die
sich mit einem Experiment verifizieren l"asst.
Wir brauchen also eine Mathematik, die uns H"aufigkeiten und m"ogliche
Abweichungen davon vorherzusagen erlaubt.

\section*{Danksagung}
"Uber die Jahre, in denen dieses Skript seiner jetzigen Form immer n"aher
gekommen ist, haben viele Studierende immer wieder auf Fehler und andere
Verbesserungsm"oglichkeiten hingewiesen, of sogar in Form von Pull-Requests
im Github Repository \url{https://github.com/AndreasFMueller/WrStat}.
Ihnen sei allen herzlich gedankt.

