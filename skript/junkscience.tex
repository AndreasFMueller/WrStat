%
% junkscience.tex -- Junk Science und seine Entlarvung
%
% (c) 2006-2015 Prof Dr Andreas Mueller, Hochschule Rapperswil
%
\rhead{Junk Science}
\chapter{Junk Science}
Das Internet hat nicht nur der Wissenschaft neue Kommunikationsmöglichkeiten
gebracht, sondern auch allen Arten von Pseudwissenschaften,
Aberglauben, Quacksalberei und Betrügern, wobei die Grenzen manchmal
fliessen sind.
In diesem Abschnitt sollen ein paar Wahrscheinlichkeitstheoretische
"Uberlegungen zeigen, wie wahrscheinlich die von Wünschelrutengängern,
Gedankenlesern, Homöo\-pathen, Graphologen, Astrologen und anderen Betrügern
und Lügnern beanspruchten Fähigkeiten sind, wenn man nur den Zufall wirken
lässt.

\section{Wünschelrutengänger}
In einem Versuch wird einem Wünschelrutengänger erlaubt, eine einzelne,
unter einer von sieben identischen Abdeckungen verborgene Erzprobe allein
mit der Hilfe seiner Wünschelrute zu finden.
Sobald er sich auf eine
der Stellen festgelegt hat, wird sie entfernt.
Falls die Erzprobe darunter
nicht gefunden wird, kann er mit den verbleibenden Stellen nochmals
versuchen.
Dies kann wiederholt werden, bis die Erzprobe gefunden wird.
Wie viele Versuche sind im Mittel nötig, wenn der Wünschelrutengänger
über keine besonderen Fähigkeiten verfügt, sondern einfach nur rät?

Wir betrachten also die Zufallsvariable $R$, die Nummer der Runde, in der
der Wünschelrutengänger die Probe findet.
Wenn der Wünschelrutengänger
nur rät, ist die Wahrscheinlichkeit, die Probe in der $k$-ten Runde zu finden
$\frac1{n-k+1}$.
Soweit muss er jedoch erst gekommen sein.
Dazu muss er in
den vorangehenden Runden daneben geraten haben.
In der erste
Runde war die Wahrscheinlichkeit dafür $\frac{n-1}n$, in der zweiten
Runde war es $\frac{n-2}{n-1}$ etc.
Die Wahrscheinlichkeit, genau in der
$k$-ten Runde die Probe zu finden ist also
\[
p_k=\frac{(n-1)(n-2)\cdots (n-k+1)}{n(n-1)\cdots(n-k+2)}\cdot\frac1{n-k+1}=\frac1n
\]
Der Erwartungswert von $R$ wird damit
\[
E(R)=\sum_{k=1}^nk\frac1n=\frac1n\cdot\sum_{k=1}^n{n(n+1)}2=\frac{n+1}2.
\]

Auch dieser Versuch wurde bereits durchgeführt.
Dabei fand der Wünschelrutengänger
die Probe auch nach drei Versuchen nicht.
Wie gross ist die Wahrscheinlichkeit,
die Probe in drei Versuchen durch blosses Raten zu finden? Gemäss obigem ist
$$P(R\le m)= \sum_{k=1}^mp_k=\frac1n\sum_{k=1}^m1=\frac{m}{n}$$
In drei Versuchen kann man also die Probe mit Wahrscheinlichkeit $\frac37$ durch
blosses Raten finden.

\section{Psychometrie}
Ein ``Medium'' behaupten, Gegenstände mit Hilfe ihrer übersinnlichen Fähigkeiten
ihren Besitzern zuordnen zu können.
Dazu stellen in einem Versuch fünf
zufällig ausgewählte Personen ihren Schlüsselbund und ihr Portemonnaie zur
Ver\-fügung.
Das Medium soll bestimmen, welcher Schlüsselbund zu welchem
Portemonnaie gehört.
Welche Anzahl von "Ubereinstimmungen kann man erwarten, wenn das ``Medium''
über keinerlei übersinnliche Fähigkeiten verfügt, sondern einfach nur rät?

Die Wahrscheinlichkeit, alle Schlüssel richtig zuzuordnen, ist $\frac1{5!}=\frac1{120}$.
Wir bezeichnen mit $p_k$ die Wahrscheinlichkeit, genau $k$ Schlüssel richtig zuzuordnen,
wir wissen also bereits $p_5=\frac1{120}$.
Für $p_k$ mit $k<5$ zu berechnen, 
müssen wir die Zahl der Möglichkeiten zählen, genau $k$ richtige Zuordnungen
und $5-k$ falsche Zuordnungen vorzunehmen.
Wenn 4 Zuordnungen richtig waren,
muss offensichtlich auch die fünfte richtig sein.
Wenn 3 Zuordnungen richtig waren,
können die verbleibenden zwei nur auf eine Art falsch sein, da es nur 2 mögliche
Zuordnungen unter den falschen gibt, wovon die eine richtig sein muss.

Mit Hilfe der Technik der erzeugenden Funktionen lässt sich die Anzahl der
der möglichen Zuordnungen mit genau $k$ richtigen bestimmen, sie ist identisch
mit der Zahl der Permutationen von $n$ Objekten, die genau $k$ Fixpunkte haben.
Wir bezeichnen diese Zahlen mit $e_k$.

Es ist einfach zu bestimmen, wieviele Zuordnungen von $n$ Objekten auf $k$
vorgegebenen Objekten richtig sind.
Offensichtlich können nur $n-k$ Objekte
noch verschoben werden, also gibt es $(n-k)!$ Möglichkeiten.
Die Anzahl der Möglichkeiten, mindestens $k$ {\it beliebige} Objekte korrekt
zuzuordnen ist jetzt
$$N_k=\binom{n}{k}(n-k)!=\frac{n!}{k!(n-k)!}=\frac{n!}{k!}.$$
Die Zahl $N_k$ lässt sich auch so bestimmen: für jede $k$-elementige
Teilmenge von Objekten zählen wir die Anzahl der Permutationen, die 
alle diese Objekte richtig zuordnen.
Wir könnten aber genausogut
für jede Permutation zählen, wie viele $k$-elementige Mengen von Objekten
von ihr richtig zugeordnet werden.
Die Anzahl der $k$-elementigen Mengen,
die von einer Permutation richtig zugeordnet werden ist die Anzahl der
Möglichkeiten, aus den richtig zugeordneten Elementen der Permutation
deren $k$ auszuwählen.
Jede Zuordnung mit $t$ richtigen trägt also
$\binom{t}{k}$ zu $N_k$ bei.
Wenn es $e_k$ Permutationen gibt,
die genau $k$ Objekte richtig zuordnen, bedeutet dies
$$N_k=\sum_{t\ge 0}\binom{t}{k}e_t,\qquad k=0,1,2,\dots.$$
Das Problem ist also gelöst, wenn die $e_t$ aus den $N_k$ bestimmt werden können.

Dazu betrachtet man die Funktionen
$$N(x)=\sum_{k}N_kx^k$$
und 
$$E(x)=\sum_{k}e_kx^k.$$
Es folgt
\begin{align*}
N(x)&=\sum_kN_kx^k=\sum_k\sum_t\binom{t}{k}e_tx^k\\
&=\sum_te_t\biggl(\sum_k\binom{t}{k}x^k\biggr)\\
&=\sum_te_t(x+1)^t\\
&=E(x+1)
\end{align*}
Durch Ersetzen von $x$ durch $x-1$ folgt
$$E(x)=N(x-1).$$
Da die Koeffizienten $N_k$ einfach zu bestimmen sind, lassen sich auch
die Koeffizienten von $E(x)$ durch einfaches Einsetzen einfach
bestimmen:
\begin{align*}
N(x)&=\sum_{k=0}^n\frac{n!}{k!}x^k=n!\sum_{k=0}^n\frac{x^k}{k!}\\
E(x)&=n!\sum_{k=0}^n\frac{(x-1)^k}{k!}
\end{align*}
Um die Wahrscheinlichkeit dafür zu bestimmen, genau $k$ richtige Zuordnungen nur
durch zu Zufall zu erreichen, muss $e_k$ noch durch $n!$ geteilt werden.
Somit
ist der Koeffizient von $x^k$ in
$$\sum_{k=0}^n\frac{(x-1)^k}{k!}$$
die diese gesuchte Wahrscheinlichkeit.

Nun wird aber der Erwartungswert gesucht.
Aus
$$\sum_{k=0}^np_kx^k=\sum_{k=0}^n\frac{(x-1)^k}{k!}$$
erhält man durch Ableiten
$$
\sum_{k=1}^nkp_kx^{k-1}
=\sum_{k=1}^n\frac{k(x-1)^{k-1}}{k!}
=\sum_{k=1}^n\frac{(x-1)^{k-1}}{(k-1)!}
=\sum_{k=0}^{n-1}\frac{(x-1)^k}{k!}
$$
Setzt man auf der linken Seite $x=1$ ein, entsteht die Summe zur
Berechnung des Erwartungswertes.
Auf der rechten Seite hingegen
verschwinden alle Terme bis auf $k=0$, somit ist der Erwartungswert
$1$.
Interessant ist auch, dass der Erwartungswert sich nicht ändert,
wenn man die Zahl der Teilnehmer erhöht.

Dieser Versuch wurde tatsächlich mit einem Medium durchgeführt, welches
sich diesem Versuch zu unterziehen bereit war.
Das Resultat entsprach genau
obigen Vorhersagen.

\section{Eine Variante: Graphologie}
Graphologen behaupten, aus der Handschrift einer Person ablesen zu können,
für welche Art Beruf sie geeignet wäre, und für welche nicht.
Deshalb
werden sie auch immer wieder bei Personalentscheidungen herbeigezogen.
Zwar darf jeder Entscheider für sich beanspruchen, diejenigen Entscheidungshilfen
hinzuzuziehen, die ihm am besten helfen.
Wenn jedoch ein objektives
Auswahlverfahren durchgeführt werden muss, wie dies zum Beispiel bei
öffentlich rechtlichen Anstellungen der Fall ist, muss man auch
fordern, dass die verwendeten Entscheidungsgrundlagen einer objektiven
"Uberprüfung standhalten.
Bei der Graphologie darf das in Zweifel
gezogen werden.

Wie kann man dies testen? Man könnte einem Graphologen fünf Schriftproben
von sehr unterschiedlichen Berufsleuten geben, und ihm auch die Berufe
der Autoren mitteilen.
Seine Aufgabe ist sodann, die Schriftproben den
Personen zuzuordnen.
Diese Aufgabe ist vollständig äquivalent zur Aufgabe,
persönliche Objekte ihrem Besitzer zuzuordnen, die Analyse des vorangehenden
Abschnittes ist daher anwendbar.
Wir können mit einer Erfolgsrate von einem
Treffer rechnen.
Auch dieser Versuch ist durchgeführt worden, mit genau
diesem Resultat.

Man könnten auch fragen, wie gross die Wahrscheinlichkeit ist, mehr als einen
Treffer durch Raten zu erzielen.
Dazu kann man erneut die erzeugende Funktion
heranziehen.
Für $n=5$ ergibt die explizite Berechnung
$$\sum_{k=0}^5\frac{(x-1)^k}{k!}
=\frac{x^5}{120}+\frac{x^3}{12}+\frac{x^2}{6}+\frac{3 x}{8}+\frac{11}{30}
$$
Die Wahrscheinlichkeit, mindestens zwei Treffer zu erzielen, ist also
$$1-\left(\frac38+\frac{11}{30}\right)=\frac1{120}+\frac1{12}+\frac16=\frac{1+10+20}{120}=\frac{31}{120}>\frac14.$$
Anders formuliert: bei jedem zweiten Versuch schafft es jeder, höchstens $3$
Fehler zu machen.
Dies dürfte der Grund sein, warum den Graphologen
von gewissen Leuten überhaupt noch Glauben geschenkt wird.

Zwei Treffer von fünf möglichen sind für eine objektive Anstellungsentscheidung
jedoch nicht wirklich viel, denn immerhin lag der Graphologe in der Mehrzahl
der Fälle falsch! Wie gross ist die Wahrscheinlichkeit, durch blosses Raten
mehr als 50\% richtig zu bestimmen? Gemäss der gerade
aufgelisteten Wahrscheinlichkeiten ist dies
$\frac1{120}+\frac1{12}=\frac{11}{121}<\frac1{11}$.

Andererseits sagen uns die Graphologen, dass 40\% der Bevölkerung im falschen
Beruf arbeitet.
Ein Schelm, wer glaubt,
die Graphologen wollten mit diesem
Argument nur erklären, warum sie keine höhere Trefferquote als $50\%$ hinkriegen!


\section{Homöopathie}
Die Homöopathie behauptet, dass ein Stoff, der normalerweise Krankheitssymptome
verursacht, zur Behandlung gegen genau diese Krankheitssymptome verwendet werden
kann, wenn man ihn nur genügen stark verdünnt.
Die dabei verwendeten
Verdünnungen von mindestens $10^{30}$ bis über $10^{1000}$ sind so gering,
dass die Wahrscheinlichkeit, überhaupt noch ein Wirkstoffmolekül in der
Lösung vorzufinden, kleiner als $10^{-6}$ ist.

In einem Versuch wurden daher in einem Versuch $2N$ Proben hergestellt,
von denen die eine Hälfte den Wirkstoff Histamin enthielten, die anderen 
nicht.
Anschliessend wurden die Proben gemäss den Vorschriften zur
Herstellung homöopatischer Medikamente im Verhältnis $10^{-36}$
verdünnt, so dass am Ende dieses Prozesses, $2N$ homöopatische
Präparate vorliegen, wovon die Hälfte nie mit dem Wirkstoff in
Kontakt waren, während die andere Hälfte gemäss homöopathischer
Lehre besonders wirkungsvoll, weil stark verdünnt sein sollte.

Nun werden alle Proben mit Zellen in Verbindung gebracht, die normalerweise
auf diesen Wirkstoff reagieren.
Nach homöopatischer Lehre sollten vorwiegend
diejenigen Proben eine Reaktion hervorrufen, welche einst in Verbindung
mit dem Wirkstoff gestanden haben.
Man erwartet natürlich, dass 
alle Proben gleichermassen eine Wirkung zeigen.

Nehmen wir an, dass $p_0$ die Wahrscheinlichkeit ist, dass eine Probe ohne
Wirkstoff eine Reaktion auslöst, und $p_1$ die Wahrscheinlichkeit, dass
eine homöopatische Probe eine Wirkung auslöst.
Wie gross muss der Unterschied
von $p_0$ und $p_1$ sein, damit wir sicher sein können, dass das homöopatische
Präparat eine Wirkung hat?

Da jede einzelne Probe individuell mit Wahrscheinlichkeit $p$ getestet wird, liegt
hier eine Binomialverteilung vor.
Getestet werden soll die Hypothese, dass zwei
Binomalverteilung den gleichen Parameter haben.
Dazu kann ein $\chi^2$-Test oder
ein Kolmogoroff-Smirnov-Test verwendet werden.
Um eine ungefähre Grössenordnung
zu bekommen, kann man aber auch die Binomialverteilung durch eine Normalverteilung
approximieren, die beiden Normalverteilungen haben unter der Hypothese den
gleichen Erwartungswert.
Der $t$-Test ermöglicht zu entscheiden, ob zwei Normalverteilungen
verschiedenen Mittelwert haben, also $Np_0\ne Np_1$.
Dazu muss man die
Grösse
$$T=\frac{\bar X - \bar Y}{\sqrt{S_X^2+S_Y^2}}\sqrt{\frac{N^2(2N-2)}{{2N}{(N-1)}}}
=
\frac{\bar X - \bar Y}{\sqrt{S_X^2+S_Y^2}}\sqrt{N}$$
Bei $N=10$ Proben verwirft dieser Test die Hypothese
auf dem Niveau $\alpha=5\%$, sobald
$T>2.228$.

Dieser Versuch wurde vor einigen Jahren vom britischen Fernsehen BBC mit der
Unterstützung einiger renommierter Forschungslabors durchgeführt, mit dem
erwarteten Resultat.
Ein anderes Resultat war auch nicht zu erwarten.
Die Lehre der Homöopathie
ist in sich bereits so widersprüchlich, dass kein Resultat mit der ganzen
Lehre in Einklang stehen kann.
Da im Meer alle möglichen Giftstoffe extrem verdünnt vorliegen, müssten
Leute, die Meerwasser aufnehmen, gemäss homöopatischer Lehre gegen alle
ihre Symptome geschützt sein.

