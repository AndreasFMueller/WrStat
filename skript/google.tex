%
% google.tex -- Google Page Rank
%
% (c) 2006-2015 Prof Dr Andreas Mueller, Hochschule Rapperswil
%
\section{Anwendung: Google Page Rank} \label{pagerank}

\subsection{Rangfolge in Suchmaschinen}
\index{Suchmaschine}
Die Aufgabe einer Suchmaschine ist, diejenigen Seiten im Internet zu
finden, welche am besten auf die Suchabfrage passen.
Auf den ersten
Blick ist dies eine einfach Aufgabe: man findet zunächst alle Seiten,
die den Suchbegriff enthalten, und liefert dann in erster Priorität
diejenigen Seiten, die ``am besten passen''.
Genau darin besteht jedoch
das Problem.
Wie misst man, wie gut eine Seite passt.

\subsubsection{Das Problem}
Bei der Suche in einer kleinen Dokumentmengen schränkt die Angabe von
genügend vielen Suchkriterien die Resultatmenge meistens so stark
ein, dass man damit etwas Sinnvolles anfangen kann.
Die Dokumentmenge
des Internet ist jedoch so gross, dass die Suche nach einigermassen
gebräuchlichen Wörtern notwendigerweise zehntausend oder hunderttausende
von Seiten liefert, die das Wort enthalten.
Ohne eine sinnvolle Rangfolge
unter den gefundenen Seiten ist es also für den Benutzer unmöglich, in
vernünftiger Zeit zu sinnvollen Resultaten zu kommen.

Als das Internet noch relativ klein war, konnten Kriterien wie die
Anzahl der Wörter auf einer Seite, die zum Suchbegriff passen,
oder die ``Entfernung'' eines Synonyms von einem Suchwort dazu dienen,
Seiten zu klassifizieren.
Es wurde aber schon bald klar, dass selbst
solche verfeinerten Suchtechniken für das rasch wachsende Internet
zu wenig selektiv sein würden.

In diesem Abschnitt stellen wir uns also die Aufgabe, ein Kriterium
für eine Rangfolge unter den Seiten des Internet zu finden, welches auch
für ein ein sehr grosses Internet mit $N$ Dokumenten erfolgreich sein kann,
wobei $N$ mindestens $10^9$ sein soll.

\subsubsection{Der Lösungsansatz}
Das Internet hat eine Eigenschaft, welche frühere elektronische 
Datenbestände kaum hatten: Jedes Dokument kann mit beliebig vielen anderen
Dokumenten verlinkt sein.
Die Seiten-Autoren verwenden Links, um auf
\index{Link}
andere Dokumente zu verweisen, die einen Begriff verdeutlichen oder
vertiefen, oder in irgend einer Art mit dem Begriff verbunden sind.
Besonders lohnende Seiten werden von vielen Leuten gelesen, und in ihren
eigenen Dokumenten wieder verlinkt.
Die Links zeigen also jeweils zu den
Seiten, die zu lesen sich lohnt.

Ein Internet-Benutzer, der eine Seite gefunden hat, die zu lesen sich lohnt,
wird eine Zeit lang auf dieser Seite verweilen, und dann den Links auf
andere Seiten folgen, den offenbar verweisen die auf weiteres Material zu
dem Thema, über das er etwas lesen möchte.

Eine Suchmaschine ist erfolgreich, wenn Sie den Benutzern diejenigen Seiten
liefert, welche diese erwarten.
Damit ist es nicht mehr die Aufgabe des
Suchmaschinenbetreibers zu beurteilen, welche Seite am besten passt.
Er
kann diese Information direkt aus dem Internet ablesen: die Seiten, auf die
die meisten Links zeigen, enthalten offenbar die beliebtesten Seiten zu diesem
Thema.
Gesucht ist also eine Methode, die Links zwischen den Seiten des
Internet zu bewerten und daraus ein Rangfolge der Seiten abzuleiten.

\subsection{Eine wahrscheinlichkeitstheoretische Lösung}
Die in der Vorlesung entwickelten Begriffe der bedingten Wahrscheinlichkeit
eignen sich hervorragend, dieses Problem zu analysieren.

\subsubsection{Wahrscheinlichkeitsinterpretation}
Nehmen wir zunächst an, die Surfer bekommen zufällig eine Seite zugewiesen,
die Wahrscheinlichkeit, die Seite mit der Nummer $i$ zugewiesen zu erhalten
sei $p_i$.
Mit dieser befassen sich die Surfer eine Weile, bevor sie einem Link auf
der Seite folgen, und damit auf einer anderen Seite landen.
Wir stellen
uns vereinfachend vor, dass die Surfer jeweils nach einer Minute miteinander
auf die nächste Seite wechseln.
Mit welcher Wahrscheinlichkeit finden wir
einen Surfer nach einer Minute auf der Seite mit der Nummer $i$?

Wir bezeichnen das Ereignis, dass ein zufällig ausgewählter Surfer gerade
\index{Surfer}
die Seite $i$ besucht mit $S_i$, das Ereignis, nach einer Minute die Seite
$j$ zu besuchen, bezeichnen wir mit $S'_j$.
Mit diesen Bezeichnungen ist
$p_i=P(S_i)$.
Uns interessiert die Wahrscheinlichkeit $P(S'_j)$, nach dem erste
Wechsel die Seite $j$ zu besuchen.
Wir bezeichnen diese Wahrscheinlichkeit
auch mit mit $p_{j,1} = P(S'_j)$.

Die Wahrscheinlichkeit, auf der Seite $j$ zu landen,
hängt also zum einen davon ab, ob es überhaupt
einen Link von $i$ auf $j$ gibt, und andererseits davon, wie wahrscheinlich
es ist, dass der Surfer einen bestimmten Link wählt.
Ohne weiteres Wissen über die Seiteninhalte müssen wir
annehmen, dass die Links auf einer Seite alle mit gleicher Wahrscheinlichkeit
vom Surfer ausgewählt werden (Laplace-Experiment).
Die Wahrscheinlichkeit, dass ein Surfer, der bereits
auf Seite $i$ ist, auf der Seite $j$ landet, ist die bedingte Wahrscheinlichkeit
$P(S'_j|S_i)$.
Das eben diskutierte Modell bedeutet
\[
P(S'_j|S_i)=\begin{cases}
0&\qquad\text{es gibt keinen Link von $i$ auf $j$}\\
\displaystyle\frac1n&\qquad\text{einer der $n$ Links auf Seite $i$ führt zu $j$}.
\end{cases}
\]
Die Ereignisse $S_i$ sind disjunkt, kein Surfer darf auf mehr als einer
Seite gleichzeitig sein.
Ausserdem überdecken sie ganz $\Omega$, jeder Surfer ist auf irgendeiner
Seite.
Daher gilt nach dem Satz über die totale Wahrscheinlichkeit
\[
P(S'_j)=\sum_{i}P(S'_j|S_i)P(S_i).
\]
Natürlich sind nur gerade diejenigen Indizes $i$ von Bedeutung, für welche $P(S'_j|S_i)$
nicht $0$ ist, also diejenige Seiten $i$, auf welchen ein Link zur Seite $j$
vorhanden ist.

Den Ausdruck kann man auch als Matrizen-Produkt interpretieren: $P(S'_j)$ lässt sich
als Linearkombination der Werte $P(S_i)$ schreiben.
Fassen wir $P(S_i)$ in den Spaltenvektor
$p$ zusammen, und $P(S'_j)$ in den Spaltenvektor $p'$, erhalten wir die Matrizengleichung
\[
p'=
\left(\begin{matrix}p'_1\\p'_2\\p'_3\\\vdots\end{matrix}\right)
=
\left(\begin{matrix}
P(S'_1|S_1)&P(S'_1|S_2)&P(S'_1|S_3)&\dots\\
P(S'_2|S_1)&P(S'_2|S_2)&P(S'_2|S_3)&\dots\\
P(S'_3|S_1)&P(S'_3|S_2)&P(S'_3|S_3)&\dots\\
\vdots&\vdots&\vdots&\ddots\\
\end{matrix}\right)
\left(\begin{matrix}p_1\\p_2\\p_3\\\vdots\end{matrix}\right)
=
H\cdot p
\]
Die Matrix $H$ codiert also die Verlinkung des Internet.

\begin{figure}
\[\UseTips
\entrymodifiers={++[o][F]}
\xymatrix @=1.5cm {
1 \ar[r] \ar[d]
	& 3 \ar[r] \ar@(dl,dl)[dl]
		& 5 \ar[d] \ar@/^/[r] \ar@(dr,dr)[rd]
			& 7 \ar@/^/[d]  \ar@/^/[l]
\ar `ul^l[lll]+/u6mm/`l^dl[lll] [lll]
\\
2 \ar@/^/[r]
	& 4 \ar@/^/[l] \ar[r] \ar[ur]
		& 6 \ar@/^/[r]
			& 8 \ar@/^/[l] \ar@/^/[u]
}
\]
\caption{Beispiel-Internet\label{google-sample}}
\end{figure}
Zur Illustration wollen wir das Beispiel-Internet in Abbildung \ref{google-sample}
untersuchen.
Die daraus abgeleitete $H$-Matrix ist:
\[
H=
\begin{pmatrix}
0&0&0&0&0&0&\frac13&0\\
\frac12&0&\frac12&\frac13&0&0&0&0\\
\frac12&0&0&0&0&0&0&0\\
0&1&0&0&0&0&0&0\\
0&0&\frac12&\frac13&0&0&\frac13&0\\
0&0&0&\frac13&\frac13&0&0&\frac12\\
0&0&0&0&\frac13&0&0&\frac12\\
0&0&0&0&\frac13&1&\frac13&0\\
\end{pmatrix} .
\]
Die Summe der Elemente jeder Spalte ist $1$, denn in jeder Spalte sind ja genau
diejenigen Elemente $n$ von $0$ verschieden und haben den Wert $\frac1n$,
die einem Link auf dieser Seite entsprechen.

Wie gross sind die Wahrscheinlichkeiten $p_i$? Die Wahrscheinlichkeit, einen zufällig
ausgewählten Surfer auf einer Seite zu finden, ändert sich kurzfristig nicht, nach
einer Minute sollten also die Surfer immer noch gleich verteilt sein.
Insbesondere
sollten gelten:
\[
p=Hp.
\]
Der Vektor $p$ ist also ein Eigenvektor der Matrix $H$ zum Eigenwert $1$.

Beginnen wir mit einer beliebigen Verteilung, sollten sich die Surfer dadurch,
dass sie den Links folgen, mit der Zeit gemässt den Wahrscheinlichkeiten $p_i$
auf den Seiten verteilen.
Somit erwarten wir, dass ausgehend von einer beliebigen
Verteilung, also einem beliebigen Vektor $p_0$, sich durch wiederholte Anwendung der
Matrix $H$ der stationäre Zustand $p$ automatisch ausbilden wird, also
\[
p=\lim_{n\to\infty}H^np_0.
\]
Mit dieser Methode findet man für die Beispielmatrix
\[
p=\left(
\begin{matrix}
0.0600\\0.0675\\0.0300\\0.0675\\0.0975\\0.2025\\0.1800\\0.2950
\end{matrix}
\right).
\]

\subsubsection{Freier Wille}
\index{Wille, freier}
Unser Modell hat einen gravierenden Mangel.
Es ist zwar richtig, dass Surfer oft den
Links folgen, aber es kommt auch vor, dass
Surfer spontan einen neuen URL eingeben.
Offensichtlich passt ein solcher Schritt nicht in das bisherige Modell.
Wir können
ihm aber Rechnung tragen, indem wir die zwei Fälle unterscheiden:
\begin{itemize}
\item Ereignis $L$: Surfer folgt einem Link.
\item Ereignis $F$: Surfer benutzt seinen freien Willen,
um auf eine neue Seite zu kommen.
\end{itemize}
Wir nehmen an, dass $\Omega=L\cup F$ und $\emptyset=L\cap F$.
Die Wahrscheinlichkeit,
auf der Seite $j$ zu landen, wird damit
\[
p'_j=P(S'_j|L)P(L)+P(S_j'|F)P(F).
\]
Im Fall $L$ nehmen wir an, dass weiterhin die Matrix $H$ angibt, wie sich die
die Surfer bewegen werden.
Im Falle $F$ müssen wir eine Annahme darüber treffen,
dass der Surfer auf Seite $j$ landen wird.
Ohne weiteres Wissen liegt die Laplace-Annahme
nahe, dass jede Seite genau gleich wahrscheinlich ist.
Wir kürzen $\alpha=P(L)$ ab und bekommen so
\begin{equation}
p'_j=\sum_{i}P(S'_j|S_i)\alpha p_i+\frac{1-\alpha}N.
\label{freewilltransition}
\end{equation}
Wie vorher vermuten wir, dass wir den stationären Zustand $p$ finden können,
indem wir eine Matrix immer wieder auf eine Ausgangsverteilung anwenden können.
Dazu müssen wir aber den Übergang von $p$ zu $p'$ als ein Matrixprodukt
schreiben können.

Betrachten wir die Matrix $A$ bestehend aus lauter Einsen:
\[
A=
\begin{pmatrix}
1&1&1&\dots&1\\
1&1&1&\dots&1\\
1&1&1&\dots&1\\
\vdots&\vdots&\vdots&\ddots&\vdots\\
1&1&1&\dots&1\\
\end{pmatrix}.
\]
Da die Summe der $p_i$ eins ergeben muss, folgt
\[
Ap=\left(\begin{matrix}1\\1\\\vdots\end{matrix}\right).
\]
Der zweite Term auf der rechten Seite von (\ref{freewilltransition})
addiert auf jeder
Zeile den gleichen Wert $\frac{1-\alpha}{N}$, dies entspricht genau dem Vektor
$\frac{1-\alpha}{N}Hp$.
Damit können wir die Gleichung (\ref{freewilltransition}) auch schreiben als
\begin{equation}
p'=\alpha Hp + \frac{1-\alpha}{N}Ap = \left(\alpha H+\frac{1-\alpha}{N}A\right)p
\end{equation}
\begin{definition} Die Matrix
\[
G=
\alpha H+\frac{1-\alpha}{N}A
\]
heisst Google-Matrix.
\index{Google-Matrix}
\end{definition}

Der stationäre Zustand unter Berücksichtigung des freien Willens ist also
so beschaffen, dass gilt
\[
p=Gp,
\]
$p$ ist also ein Eigenvektor der Google-Matrix zum Eigenwert $1$.

\subsubsection{Berechnung des PageRank}
\index{Pagerank!Berechnung}
Das Problem ist jetzt bereits darauf reduziert, einen Eigenvektor der
Google-Matrix $G$ zum Eigenwert $1$ zu finden.
Es wurde auch vermutet, dass man diesen Eigenvektor
aus einem beliebigen Anfangszustand $p_0$ durch wiederholte Anwendung von
$G$ finden kann, also
\[
p=\lim_{n\to\infty}G^np_0.
\]
Für die Link-Matrix $H$ kann das allgemein nicht gelten.
Es lassen sich leicht einfache Modell-Internets konstruieren,
die Gegenbeispiele dazu liefern.
Es gibt Beispiele, in denen
$H$ keinen solchen Grenzwert besitzt, oder einen Grenzwert, von dem
viele Einträge $0$ sind, was bedeuten würde, dass diese Seiten nie besucht
werden.
Die Matrix $G$ hingegen besteht ausschliesslich aus positiven Einträgen, es
gibt also zwischen zwei beliebigen Seiten immer einen Link, auch wenn er nur
durch den ``freien Willen'' zur Verfügung gestellt wird.
Daher ist es unmöglich,
dass durch die Iteration von $G$ je ein Element von $p$ zu $0$ wird.

Die Geschwindigkeit der Konvergenz von $G^np_0$ hängt davon ab, wie gross der
zweitgrösste Eigenwert von $G$ ist.
Man kann zeigen, dass dieser betragsmässsig
nicht grösser als $\alpha$ ist, also
\[
\lambda_2\le \alpha.
\]
Daraus lässt sich die Geschwindigkeit der Konvergenz von $G^np_0$ gegen $p$
abschätzen.
In der tatsächlichen Realisierung bei Google kann natürlich
für $p_0$ jeweils der bei der letzten Berechnung gefundene Vektor $p$ verwendet
werden, der ja die Verteilung der Surfer bereits relativ gut wiedergeben sollte.
Trotzdem kann es bei der Neuberechnung von $p$ zu starken Änderungen kommen,
eine Erscheinung, die manchmal ``Google Dance'' genannt wird.
