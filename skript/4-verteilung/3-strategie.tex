\section{Lösungsstrategie für Probleme über Zufallsvariablen}
\kopfrechts{Lösungsstrategie}
Die für die Berechnung aller wesentlichen wahrscheinlichkeitstheoretischen
Kennzahlen einer Zufallsvariablen nötige Information ist in der
Verteilungsfunktion codiert.
Probleme über Zufallsvariablen können daher mit folgender Lösungsstrategie
angegangen werden.
\begin{enumerate}
\item
Zunächst muss festgestellt werden, welche Verteilung die Zufallsvariablen
haben.
Zu jeder später zu besprechenden Verteilung ist also auch
festzuhalten, für welchen Zweck sie geeignet ist.
\item 
Die Verteilungen haben Parameter, die als nächstes bestimmt werden
müssen.
Dazu können die bekannten Formeln für Kennzahlen einer
Verteilung wie Erwartungswert oder Varianz.
So können zum Beispiel aus Erwartungswert $\mu$ und Varianz $\sigma^2$ 
einer auf einem Intervall gleichverteilten Zufallsvariable  die
Gleichungen
\begin{align*}
a+b&=2\mu\\
b^2-a^2&=12\sigma^2\quad\Rightarrow\quad b-a=\frac{6\sigma^2}{\mu}
\end{align*}
abgeleitet werden, welche $a$ und $b$ zu bestimmen erlauben.
\item
Sobald die Parameter festliegen, können beliebige Wahrscheinlichkeiten
oder Erwartungswerte von $X$ oder Funktionen $f(X)$ ermittelt werden,
womit sich so ziemlich jede Frage beantworten lässt.
\end{enumerate}
Im Folgenden geht es also vor allem darum, einen geeignet grossen
Katalog von Verteilungen bereitzustellen.
Dazu dienen neben einigen
Grundverteilungen auch Rechenregeln, welche aus bereits bekannten
Verteilungen neue Verteilungen abzuleiten gestatten.
Diese Rechenregeln
sind der Inhalt des nächsten Abschnittes.
Im folgenden Kapitel wird
dann eine Auswahl von praktisch nützlichen Verteilungen mit ihren
Eigenschaften vorgestellt.

