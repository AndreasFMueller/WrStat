Sei $X_1,X_2,X_3$ eine Stichprobe von Messungen einer im
Intervall $[0,1]$ gleichverteilten Zufallsvariable $X$,
d.~h.~alle $X_i$ haben die gleiche Verteilung. Betrachten Sie die folgenden
zwei Sch"atzer f"ur den Erwartungswert von der Verteilung:
\begin{align*}
\hat\mu_1&=\frac{X_1+X_2+X_3}3\\
\hat\mu_2&=\operatorname{med}(X_1,X_2,X_3)
\end{align*}
Beide Gr"ossen $\hat\mu_1$ und $\hat\mu_2$ sind Zufallsvariablen.
\begin{teilaufgaben}
\item
Berechnen Sie die Varianz von $\hat\mu_1$.
\item
Simulieren Sie das Experiment auf dem Computer, und berechnen Sie
$\hat\mu_2$ 10000 mal, und ermitteln Sie die Varianz.
\item
Welcher Sch"atzer hat die gr"ossere Varianz?
\item
Bonusaufgabe:
K"onnen Sie die Varianz von $\hat\mu_2$ auch theoretisch
berechnen?
\end{teilaufgaben}

\begin{loesung}
\begin{teilaufgaben}
\item Die Varianz von $\hat\mu_1$ kann mit den Rechenregeln bestimmt werden:
\begin{align*}
\operatorname{var}(\hat\mu_1)&=\operatorname{var}\biggl(\frac{X_1+X_2+X_3}3\biggr)
\\
&=\frac19(
\operatorname{var}(X_1)+
\operatorname{var}(X_2)+
\operatorname{var}(X_3)
)
=\frac13\operatorname{var}(X).
\end{align*}
Die Varianz von $X$ wurde in der Vorlesung berechnet:
$\operatorname{var}(X)=\frac1{12}$. Also ist
$\operatorname{var}(\hat\mu_1)=\frac1{36}=0.027778$.
\item Die Simulation kann mit folgendem Programm durchgef"uhrt werden:
\verbatimainput{med.m}
Es ergibt in f"unf verschiedenen L"aufen:
\[
0.050045,\quad
0.049567,\quad
0.049557,\quad
0.049983,\quad
0.050909.
\]
\item
Aus dem empirischen Resultat kann man vermuten dass der Mittelwert
deutlich weniger Varianz hat als der Median.
\item
Um die Varianz des Medians berechnen zu k"onnen braucht man dessen
Verteilungsfunktion.
Sei $F(x)$ die Verteilungsfunktion der Zufallsvariablen $X$.
Dann gilt f"ur $Z=\operatorname{med}(X_1,X_2,X_3)$:
\begin{align*}
F_Z(x)&=P(Z\le x)\\
&=P(X_1\le x\wedge X_2\le x\wedge X_3 > x)
+P(X_2\le x\wedge X_3\le x\wedge X_1 > x)\\
&\qquad+P(X_3\le x\wedge X_1\le x\wedge X_2 > x)
+ P(X_1\le x\wedge X_2\le x\wedge X_3\le x)\\
&=3\cdot P(X_1\le x\wedge X_2\le x\wedge X_3 > x) 
+ P(X_1\le x\wedge X_2\le x\wedge X_3\le x)\\
&=3\cdot P(X_1\le x)\cdot P(X_2\le x)\cdot P(X_3 > x)\\
&\qquad
+P(X_1\le x)\cdot P(X_2\le x)\cdot P(X_3\le x)\\
&=3\cdot F(x)\cdot F(x)\cdot (1-F(x))+F(x)^3=3F(x)^2-3F(x)^3
\end{align*}
Daraus kann man jetzt die Dichtefunktion finden:
\begin{align*}
\varphi_Z(x)
&=
6F'(x)F(x)-6F'(x)F(x)^2=6\varphi(x)F(x)(1-F(x))
\end{align*}
Die Funktionen $F(x)$ und $\varphi(x)$ sind bekannt:
\begin{align*}
F(x)&=\begin{cases}
0&\quad x \le 0\\
x&\quad 0<x\le1\\
1&\quad 1 < x
\end{cases},
&
\varphi(x)
&=\begin{cases}
0&\quad x \le 0\\
1&\quad 0<x\le1\\
0&\quad 1 < x
\end{cases}
\end{align*}
Damit kann man jetzt $E(Z^2)$ berechnen:
\begin{align*}
E(Z^2)&=\int_{-\infty}^{\infty}x^2 \varphi(x)\,dx=\int_{-\infty}^{\infty}
x^2
6\varphi(x)F(x)(1-F(x))
\,dx
\\
&=6\int_0^1x^2\cdot x(1-x)\,dx=6\int_0^1x^3-x^4\,dx\\
&=
6\biggl[
\frac14x^4-\frac15x^5
\biggr]_0^1=6\biggl(\frac14-\frac15\biggr)=6\frac{5-4}{20}=\frac3{10}=0.3.
\end{align*}
Die Varianz ist daher
\[
\operatorname{var}(Z)=E(Z^2)-E(Z)^2=0.3 - 0.5^2=0.05.
\]
Das Resultat stimmt gut mit dem in den Simulationen in b) gefunden Wert
"uberein. Der Median ist also zwar ein konsistenter und (mindestens f"ur
drei Messwerte) erwartungstreuer Sch"atzer f"ur $\mu$, aber er ist
dem Mittelwert durch seine gr"ossere Varianz unterlegen.
\qedhere
\end{teilaufgaben}
\end{loesung}

