Die Varianz der Augenzahl $X$ eines Würfels wurde in der Vorlesung
berechnet. Wirft man einen Würfel $n=3$ Mal, kann man mit der
``alten'' Schätzformel die Varianz bestimmen:
\[
\hat\sigma^2=\frac1n\sum_{i=1}^3x_i^2-\biggl(\frac1n\sum_{i=1}^3x_i\biggr)^2.
\]
Führen Sie dies mit dem Computer 10000 mal durch, und berechnen Sie den
Mittelwert. Vergleichen Sie mit der theoretischen Varianz und dem
korrigierten Varianz-Schätzer.

\begin{loesung}
Die Rechnung kann man zum Beispiel mit folgendem Octave/Matlab-Programm
durchführen:
\verbatimainput{var.m}
Das Programm berechnet zuerst einen Vektor von 3 Zufallszahlen zwischen 1
und 6. Dann wird die Varianz nach der alten Schätzformel als Element
dem Vektor {\tt a} hinzugefügt. Am Schluss wird die Varianz des Vektors
{\tt a} berechnet.

Die Berechnung dauert einige Sekunden und ergibt immer wieder
ähnliche Werte:
\[
1.9663,\quad 
1.9384,\quad
1.9088,\quad
1.9668.
\]
Der theoretische Wert für die Varianz ist jedoch
\[
\frac{n^2-1}{12}=\frac{6^2-1}{12}=\frac{35}{12}=2.9167.
\]
Der Unterschied rührt vom in der Vorlesung abgeleiteten Korrekturfaktor
$\frac{n}{n-1}=\frac32$ her. Mit dem Korrekturfaktor liefert der Schätzer die
Werte
\[
2.9494,\quad 
2.9076,\quad
2.8632,\quad
2.9602,
\]
was recht gut mit den theoretischen Werten übereinstimmt.
\end{loesung}

