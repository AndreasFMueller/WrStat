Die Zufallsvariable $X$ sei exponentialverteilt mit Parameter $a$.
Finden Sie einen Schätzer, der $\frac1a$ aus Messungen $x_1,\dots,x_n$
von $X$ bestimmt. Ist Ihr Schätzer konsistent und/oder erwartungstreu?

\themaL{Schatzer}{Schätzer}
\thema{konsistent}
\thema{erwartungstreu}

\begin{loesung}
Das Maximum Likelihood-Prinzip gibt eine Methode, mit der man einen
Schätzer finden kann. Aus der Dichtefunktion
\[
\varphi(x,a)=\begin{cases}
ae^{-ax}&\qquad x \ge 0\\
0&\qquad\text{sonst}
\end{cases}
\]
kann man die Likelihood-Funktion
\[
L(x_1,\dots,x_n,a)=\varphi(x_1,a)\dots\varphi(x_n,a)=
\begin{cases}
a^ne^{-a(x_1+x_2+\dots+x_n)}&\qquad \text{alle $x_i \ge 0$}\\
0&\qquad\text{sonst}
\end{cases}
\]
Jetzt muss $a$ so gewählt werden, dass $L(x_1,\dots,x_n,a)$
maximal wird.

Die Funktion $e^{-a(x_1+\dots+x_n)}$ nimmt für grössers $a$ monoton ab,
wird die Likelihood-Funktion für mit zunehmendem $a$ immer kleiner. Für
sehr kleine Werte von $a$ sorgt der Faktor $a^n$ umgekehrt dafür dass
die Likelihood-Funktion klein wird. Das Maximum muss also irgendwo
zwischen diesen beiden Extremen liegen.

Um das Maximum zu finden, leiten wir nach $a$ ab:
\begin{align*}
\frac{d}{da}L(x_1,\dots,x_n,a)&=na^{n-1}e^{-a(x_1+\dots+x_n)}-a^n(x_1+\dots+x_n)e^{-a(x_1+\dots+x_n)}\\
&=a^n
\left(\frac{n}a-(x_1+\dots+x_n)\right)
e^{-a(x_1+\dots+x_n)}=0
\end{align*}
Der Exponentialfaktor ist immer positiv, also muss der Klammerausdruck
verschwinden:
\[
\left(\frac{n}a-(x_1+\dots+x_n)\right)=0
\quad\Rightarrow\quad
\frac1a=\frac{x_1+\dots+x_n}{n}.
\]
Der gefundene Schätzer ist der Schätzer für den Erwartungswert,
welcher natürlich konsistent und erwartungstreu ist.
\end{loesung}

