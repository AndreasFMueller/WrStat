Ein Zufallszahlgenerator liefert gleichverteilte Zufallszahlen im
Intervall $[0,\vartheta]$. Aus $n$ Beobachtungen $x_1,\dots,x_3$ kann
man den Wert von $\vartheta$ schätzen, indem man das Maximum der
drei Werte ermittelt. Dieser Schätzer ist sicher konsistent.
Bestimmen Sie experimentell, ob er auch erwartungstreu sein kann.
Schreiben Sie dazu ein Simulationsprogramm, welches jeweils $n=3$
Zufallszahlen ermittelt, und davon den grössten Wert als
Schätzwerte $\hat\vartheta$ bestimmt.
Dieses Zufallsexperiment wird $N=10000$ durchgeführt, und der Mittelwert
der $\hat\vartheta$ ermittelt. Was ergibt sich und was schliessen Sie
daraus?

\thema{Schätzer}
\thema{erwartungstreu}

\begin{loesung}
Die Simulation ist möglich mit dem Program:
{\small
\verbatimainput{sim.c.expanded}
}
Für $n=3$ und $N=10000$ ergeben sich Werte  um $E(\hat\vartheta)=0.75$.
Der Schätzer ist also höchstwahrscheinlich nicht erwartungstreu.
Dies deckt sich mit der Theorie, welche vorhersagt, dass der Erwartungswert
des Schätzers $\frac{n}{n+1}$ ist.
\end{loesung}

