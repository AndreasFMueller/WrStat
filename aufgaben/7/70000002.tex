Auf einem Autobahnteilstück fanden in den letzten Jahren
\[
4, 7, 3, 7, 5, 4, 8, 6, 6, 4
\]
tödliche Unfälle statt.
Wie gross ist die Wahrscheinlichkeit, dass im nächsten Jahr
weniger tödliche Unfälle passieren wie in jedem Vorjahr?
Wie wahrscheinlich ist, dass so viele Unfälle wie noch nie
passieren?

\thema{Poissonverteilung}

\begin{loesung}
Die Unfallzahlen sind poissonverteilt.
Der Parameter $\lambda$ ist gleichzeitig der Erwartunsgwert, also
ist der Mittelwert ein konsistenter und erwartungstreuer Schätzer
für $\lambda$. der Mittelwert der vorliegenden Zahlen ist $\lambda=5.4$.

Diesen $\lambda$-Wert kann man jetzt verwenden, um mit der Poissonverteilung
die Wahrscheinlichkeit zu schätzen:
\begin{align*}
P(k<3)&=
P_\lambda(0)
+
P_\lambda(1)
+
P_\lambda(2)
\\
&=
e^{-\lambda}\frac{\lambda^0}{0!}
+
e^{-\lambda}\frac{\lambda^1}{1!}
+
e^{-\lambda}\frac{\lambda^2}{2!}
\\
&=
e^{-\lambda}\biggl(1+\lambda+\frac{\lambda^2}{2}\biggr)
\end{align*}
Setzt man darin den Wert $\lambda=5.4$ ein, erhält man
\[
P(k<3)=e^{-5.4}(1+5.4+5.4^2/2)=0.004516581\cdot 20.98= 0.0948.
\]
Für die zweite Frage muss die Verteilungsfunktion für die
Poissonverteilung für $x=8$ ausgewerte werden, also
\[
P(k>8)=
1-e^{-\lambda}\biggl(
1+\lambda+\frac{\lambda^2}{2!}+\dots+\frac{\lambda^8}{8!}
\biggr)
\]
Einsetzen von $\lambda=5.4$ ergibt $P(k>8)=0.0973$.
\end{loesung}

