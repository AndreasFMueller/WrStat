Stellen Sie die Likelihood-Funktion
$L(a; x_1,\dots,x_n)$ zur Schätzung des Parameters $a$ einer 
Exponentialverteilung aus einer Stichprobe $x_1,\dots,x_n$ auf.


\begin{loesung}
Die Wahrscheinlichkeitsdichte der Exponentialverteilung ist
\[
\varphi(x)
=
\begin{cases}
ae^{-ax} &\qquad \text{für $x> 0$}\\
0        &\qquad \text{für $x\le 0$.}
\end{cases}
\]
Die Likelihood-Funktion entsteht, indem man die Werte der Stichprobe
in $\varphi$ einsetzt und das Produkt bildet:
\begin{align*}
L(a;x_1,\dots,x_n)
&=
\prod_{i=1}^n \varphi(x_i)
=
\begin{cases}
\displaystyle
\prod_{i=1}^n ae^{-ax_i} &\qquad \text{falls $x_i>0\,\forall i$}\\[15pt]
0                        &\qquad \text{falls $\exists\, i\,(x_i\le0)$.}
\end{cases}
\intertext{Unter Ausschluss des Falls, in dem eines der $x_i\le0$ ist, gilt}
&=
a^n e^{-a(x_1+\dots+x_n)}.
\end{align*}
Der Parameter $a$ kann jetzt geschätzt werden durch Minimierung der
Likelihood-Funktion.
\end{loesung}
