Aus f"unf Messungen einer normalverteilten Gr"osse wurde der Erwartungswert
$\mu$ und die Varianz $\sigma^2$ ermittelt.
Wegen der kleinen Zahl von Messungen sind beide mit einer gewisssen
Unsicherheit behaftet, $\mu$ und  $\sigma^2$ sind Zufallsvariablen.
Die Gr"osse
\begin{equation}
Z=\frac{X-\mu(X_1,\dots,X_n)}{\sigma(X_1,\dots,X_n)},
\label{80000018:1}
\end{equation}
die in der Standardisierung immer verwendet worden ist, ist daher
nicht mehr standardnormalverteilt, sondern $t$-verteilt, mit $n-1$ 
Freiheitsgraden. Finden Sie ein Interval, in das die Werte von $X$
mit Wahrscheinlichkeit $0.95$ hineinfallen.

\begin{loesung}
Zun"achst m"ussen wir ein Interval finden, in das die Werte von
$Z$ mit Wahrscheinlichkeit $0.95$ hineinfallen. Wir suchen dazu in
der Tabelle der Verteilungsfunktion der $t$-Verteilung Zahlen $t_+$
und $t_-$ so, dass $X$ mit Wahrscheinlichkeit $2.5\%$ gr"osser als $t_+$
ist und mit Wahrscheinlichkeit $2.5\%$ kleiner als $t_-$. Die Tabelle
der $t$-Verteilung liefert f"ur $p=0.975$ und $k=5-1=4$ den Wert
$t_+=2.7764$. Entsprechend ist $t_-=-2.7764$. Aufl"osen von
(\ref{80000018:1}) nach $X$ liefert
\begin{align*}
t_\pm &=\frac{x_{\pm}-\mu}{\sigma}\\
x_\pm&=\mu +t_\pm\sigma = \mu \pm 2.7764\sigma.
\end{align*}
Das gesuchte Interval ist also $[\mu-2.7764\sigma,\mu+2.7764\sigma]$.
\end{loesung}
