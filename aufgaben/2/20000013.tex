Aus einer Website "uber Verkehrsunf"alle und Alkohol: Im Jahr 2001
gab es 23896 Unf"alle mit Personenschaden, bei 2454 davon war
Alkohol im Spiel. Im gleichen Jahr gab es 544 Verkehrstote, bei 104
davon war Alkohol im Spiel. Formulieren Sie geeignete Ereignisse, und
bestimmen sie n"aherungsweise deren Wahrscheinlichkeit. Sind die Ereignisse
unabh"angig?

\begin{hinweis}
Nehmen sie an, dass es bei jedem Unfall mit Todesfolge jeweils
genau einen Toten gegeben hat.
\end{hinweis}

\begin{loesung}
Die Menge $\Omega$ besteht aus den Unf"allen mit Personenschaden,
$N=|\Omega|=23896$. Das Ereignis $A\subset \Omega$ besteht aus denjenigen
Unf"allen, bei denen Alkohol im Spiel war. N"aherungsweise gilt $P(A)=\frac{|A|}{|\Omega|}=\frac{2454}{23896}=0.102695=10.2695\%$.
Das Ereignis $T$ bestehe aus den
Unf"allen mit Todesfolge. Offenbar ist
$|T| = 544$, also ist $P(T) = \frac{544}{23896}=0.022765316$.
W"aren die Ereignisse unabh"angig, m"usste gelten
$ P(A\cap T)=P(A)P(T) $. N"aherungsweise gilt aber
\begin{align*}
P(A\cap T)&\simeq\frac{104}{23896}=0.00435219283562102443\\
\\
P(A)\cdot P(T)&=\frac{2454}{23896}\cdot\frac{544}{23896}=0.00233788443146503949
\end{align*}
Es sieht also eher danach aus, dass die beiden Ereignisse
abh"angig sind. Und tats"achlich sagt der gesunde Menschenverstand ja auch,
dass Alkohol im Blut die Fahrweise und damit die Unfallgefahr beeinflussen
kann.
\end{loesung}

