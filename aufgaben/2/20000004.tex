Ein noch in der Entwicklung befindlicher Feuermelder gibt bei Feuer in 95\%
der Fälle Alarm. Wenn dagegen kein Feuer brennt, gibt der Feuermelder
nur in 0.01\% der Fälle einen (falschen) Alarm. Allerdings sind
Zimmerbrände auch einigermassen selten, die Wahrscheinlichkeit
ist sicher kleiner als $0.0001$.

\begin{teilaufgaben}
\item Wie wahrscheinlich ist ein Alarm?
\item Wie wahrscheinlich ist es, dass es tatsächlich brennt, wenn
der Feuermelder Alarm gibt?
\item Wie wahrscheinlich ist es, dass der Feuermelder zwar keinen
Alarm gibt, es aber trotzdem brennt?
\end{teilaufgaben}

\thema{bedingte Wahrscheinlichkeit}
\thema{totale Wahrscheinlichkeit}
\thema{Satz von Bayes}

\begin{loesung}
In dieser Aufgabe geht es um die zwei Ereignisse
\begin{align*}
A&=\{\text{Alarm ausgelöst}\}
\\
F&=\{\text{Feuer}\}
\end{align*}
Aus der Aufgabenstellung sind die folgenden bedingten Wahrscheinlichkeiten
bekannt:
\begin{align*}
P(A|F)&=0.95
\\
P(A|\overline F)&=0.0001
\\
P(F)&=0.0001
\end{align*}
\begin{teilaufgaben}
\item Gesucht ist $P(A)$, also
\begin{align*}
P(A)&=P(A|F)P(F)+P(A|\overline F)P(\overline F)
\\
&=0.95\cdot 0.0001+0.0001\cdot(1-0.0001)= 0.00019499
\end{align*}
\item
Gesucht ist die bedingte Wahrscheinlichkeit $P(F|A)$
\begin{align*}
P(F|A)
&=
P(A|F) \frac{P(F)}{P(A)}
\\
&=0.95\cdot\frac{0.0001}{P(A)}\simeq0.4872
\end{align*}
Dieser Feuermelder ist also unbrauchbar unzuverlässig, nur bei etwa der
Hälfte der Alarme brennt es auch tatsächlich.
\item Gesucht ist die bedingte Wahrscheinlichkeit $P(F|\overline A)$.
\begin{align*}
P(F|\overline A)
&=
P(\overline A|F)\frac{P(F)}{P(\overline A)}=
\\
&=
(1-P(A|F))\frac{P(F)}{1-P(A)}
\\
&\simeq
0.05\cdot \frac{0.0001}{0.99981}
\simeq
0.000005
\end{align*}
es ist also recht unwahrscheinlich, dass es brennt,
wenn der Feuermelder keinen Alarm gibt, das hat aber vor allem auch
mit der geringen Wahrscheinlichkeiten von Bränden zu tun.
\qedhere
\end{teilaufgaben}
\end{loesung}

