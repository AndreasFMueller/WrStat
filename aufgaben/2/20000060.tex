In einem Dreikönigskuchen mit 7 Teilen befindet sich in genau einem
Teil eine kleine Königsfigur.
$A$ und $B$ möchten unter sich ausmachen, wer König sein soll.
Sie einigen sich darauf, dass sie abwechselnd jeweils ein Teil
des Kuchens wählen und essen dürfen.
Sobald eine der beiden Personen den König hat, endet die Ausmarchung.
$A$ beginnt.
\begin{teilaufgaben}
\item
Wie gross ist die Wahrscheinlichkeit, dass $A$ den König gewinnt?
\item
Ist dies ein faires Spiel?
\item
Wieviele Kuchenteile werden im Mittel für so eine Ausmarchung gegessen?
\item
Wie gross ist die erwartete Anzahl Kuchenteile, die $A$ essen muss?
\end{teilaufgaben}

\begin{loesung}
Wir lösen das Problem für einen $n$-teiligen Kuchen.
Sie $K$ die Anzahl der Teile, die gegessen werden, bis das Resultat
der Ausmarchung festliegt.

\begin{teilaufgaben}
\item
Damit die Ausmarchung genau im $k$-ten Schritt endet, müssen erst
$k-1$ Teile ohne König gewählt werden, dann der König.
Nach $i$ Schritten ohne König ist die Anzahl der zur Auswahl
stehenden Teile $n-i$ und die Wahrscheinlichkeit, im nächsten Schritt
wieder ein Teil ohne König zu wählen ist $(n-i-1)/(n-i)$.
Nach $n-1$ Schritten ist diese Wahrscheinlichkeit
\[
\frac{n-(n-1)-1}{n-(n-1)}
=
\frac{0}{1}=0,
\]
dies drückt aus, dass die Ausmarchung nach dem $n$-ten Schritt nicht
mehr weitergehen kann.
Die Wahrscheinlichkeit, dass der König genau im $k$-ten Schritt gefunden
wird, ist
\begin{align*}
P(K=k)
&=
\underbrace{
\frac{n-1}{n}
\cdot
\frac{n-2}{n-1}
\cdot \ldots \cdot
\frac{n-(k-2)-1}{n-(k-2)}
}_{\text{$k-1$ Teile ohne König}}
\cdot
\underbrace{
\frac{1}{n-(k-1)}
}_{\text{König im Teil $k$}}
=
\frac{1}{n}.
\end{align*}

In obiger Rechnung sind wir davon ausgegangen, dass der König fest
``verankert'' ist, und haben die Wahrscheinlichkeit untersucht, dass
er von der Auswahl ``gefunden'' wird.
Wir könnten uns aber auch vorstellen, dass die Reihenfolge, in der
die Stücke gewählt werden, in den Köpfen der Teilnehmer bereits 
festgelegt ist, und wir jetzt einfach noch einen König zufällig
da hineinsprenkeln.
Dann wird das Problem viel übersichtlicher: für den König gibt es 
genau $n$ mögliche Schritte, in denen er gefunden wird.
Die Wahrscheinlichkeiteit, dass er im $k$-ten Schritt gefunden wird
ist also $\frac1n$, wie vorhin schon gefunden.

\item
$A$ gewinnt, wenn die Ausmarchung in einem ungeraden Schritt endet.
Die Wahrscheinlichkeit dafür ist
\[
P(A)
= 
P(K=1) + P(K=3) + P(K=5) + \dots 
=
\sum_{l=0}^{\lfloor\frac{n-1}{2}\rfloor} P(K=2l+1)
=
\sum_{l=0}^{\lfloor\frac{n-1}{2}\rfloor} \frac1n
=
\biggl(
1+
\biggl\lfloor \frac{n-1}{2}\biggr\rfloor
\biggr)
\cdot
\frac{1}{n}.
\]
Man beachte, dass die Wahrscheinlichkeit für die beiden
Mitspieler verschieden ist, wenn $n$ ungerade ist.
Wenn $n$ gerade ist, bekommen $A$ und $B$ gleichviele
Chancen, ein Kuchenstück zu probieren.
Wenn $n$ ungerade ist, dann kommt $A$ am Ende noch ein
zusätzliches Mal dran und hat damit auch die grössere
Wahrscheinlichkeit zu gewinnen.
Aus diesem Grund sollten Dreikönigskuchen immer eine
durch möglichst viele Zahlen teilbare Anzahl Teile haben,
zum Beispiel 6 oder 12 (teilbar durch 2, 3, 4, 6).

\item
$K$ ist die Anzahl der Kuchenstücke, die in der Ausmarchung
gegessen werden.
Der Erwartungswert von $K$ ist die Summe
\[
E(K)
=
\sum_{k=1}^n k P(K=k)
=
\frac1n \sum_{k=1}^n k
=
\frac1n\cdot\frac{n(n+1)}2
=
\frac{n+1}2.
\]

\item
$E(K)$ ist nicht die erwartete Anzahl Kuchenteile, die ein
Teilnehmer essen muss, dazu müssen wir anders vorgehen.
Sei $K_A$ die Zufallsvariable, die angibt, wie viele Kuchenteile
$A$ essen muss.
Da $A$ beginnt, ist $K_A\ge 1$.
Wenn er den König findet, oder wenig $B$ im nächsten Schritt
den König findet, dann ist $K_A=1$.
Die Wahrscheinlichkeit dafür ist also $\frac{2}n$.
Es ist $K_A=2$, wenn $K=2$ oder $K=3$.
Allgemein ist $K_A=l$ wenn $K=2l$ oder $K=2l+1$.
Da jeder Wert von $K$ gleich wahrscheinlich ist, ist auch jeder
mögliche Wert von $K_A$ gleich wahrscheinlich, mit Ausnahme des
grössten, wenn $n$ ungerade ist.
Für gerades $n$ ist
\begin{align*}
E(K_A)
&=
1\cdot P(K=1\vee K=2) + 2\cdot P(K=3\vee K=4) + \dots
+ \frac{n}2\cdot P(K=n-1\vee K=n)
\\
&=
1\cdot \frac{2}{n} + 2 \cdot \frac{2}{n} + \dots + \frac{n}{2}\cdot \frac{2}{n}
\\
&=
\frac{2}{n}\cdot\biggl(1+2+\dots+\frac{n}2\biggr)
=
\frac{2}{n} \cdot \frac{\frac{n}2\left(\frac{n}{2}+1\right)}{2}
=
\frac{n}{4}+\frac12,
\intertext{für ungerades $n$ kommt noch ein letzter Term hinzu}
E(K_A) &=
1\cdot P(K=1\vee K=2) %+ 2\cdot P(K=3\vee K=4)
+ \dots
+ \frac{n-1}2\cdot P(K=n-2\vee K=n-1)
+ \frac{n+1}2\cdot P(K=n)
\\
&=
1\cdot\frac{2}{n}
+
2\cdot\frac{2}{n}
+
\dots
+
\frac{n-1}2\cdot\frac{2}{n}
{\color{red}\mathstrut+
\frac{n+1}2\cdot\frac{1}{n}}
\\
&=
\frac2n \cdot \biggl(1+2+ \dots +\frac{n-1}2\biggr)
+\frac{n+1}2\cdot\frac{1}{n}
\\
&=
\frac{2}n \cdot \frac{\frac{n-1}{2}\left(\frac{n-1}2+1\right)}{2}
+\frac{n+1}2\cdot\frac{1}{n}
=
\frac{(n-1)(n+1)}{4n} + \frac{n+1}{2n}
\\
&=
\frac{(n-1)(n+1)}{4n} + \frac{2(n+1)}{4n}
=
\frac{n^2+2n+1}{4n}
=
\frac{n}4+\frac12 + \frac{1}{4n}
\end{align*}
Im Fall $n$ ungerade kommt also noch ein zusätzlicher Summand $\frac{1}{4n}$
hinzu, man kann sich vorstellen, dass er von den höhren Gewinnchancen von $A$
in diesem Fall herrührt.
\end{teilaufgaben}

Nach diesen allgemeinen Vorarbeiten können wir jetzt die einzelnen Fragen
beantworten:
\begin{teilaufgaben}
\item
Für $n=7$ ist die Wahrscheinlichkeit 
\[
P(A)
=
\biggl(1+\frac{7-1}{2}\biggr)\cdot \frac17
=
\frac{4}{7}
=
0.\overline{571428}.
\]
\item
Für ungerades $n$ ist das Spiel unfair, da $A$ eine Wahlmöglichkeit mehr
hat, den König zu gewinnen.
\item
Die erwartete Anzahl Teile ist
\[
E(K)
=
\frac{7+1}{2}
=
4.
\]
\item
Die erwartete Anzahl der Kuchenteile, die $a$ ist, ist
\[
E(K_A)=\frac{n}4+\frac12+\frac{1}{4\cdot 7}
=
\frac{49 + 14 + 1}{28}
=
\frac{64}{28}
=
\frac{16}{7}
=
2.\overline{285714}.
\qedhere
\]
\end{teilaufgaben}

\end{loesung}

\begin{bewertung}
\begin{teilaufgaben}
\item
Wahrscheinlichkeit für König im $k$-ten Schritt ({\bf K}) 1 Punkt,
Anzahl Möglichkeiten für $A$ ({\bf M}) 1 Punkt.
\item
Spiel ist unfair, weil $n=7$ ungerade ist ({\bf U}) 1 Punkt.
\item
Erwartungswert für $K$ ({\bf E}) 1 Punkt, Zahlenwert ({\bf Z}) 1 Punkt.
\item
Erwartungswert für $K_A$ ({\bf A}) 1 Punkt.
\end{teilaufgaben}
\end{bewertung}
