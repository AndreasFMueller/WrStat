In einer Stadt geht die Grippe um, ausserdem sind in letzter Zeit
vermehrt Masern-F"alle aufgetreten.
Ein Kinderarzt weiss, dass 90\% der kranken Kinder, die er besucht,
die Grippe haben, aber nur 10\% die Masern.
Ein bekanntes Symptom der Masern ist der charakteristische Ausschlag,
der allerdings nicht immer auftritt, sondern nur bei 95\% der Erkrankten.
Und 8\% der an Grippe erkrankten Kinder haben ebenfalls einen Ausschlag.
\begin{teilaufgaben}
\item
Beim n"achsten Krankenbesuch findet er ein Kind vor mit einem Ausschlag,
und schliesst sofort, dass das Kindern die Masern hat.
Wie wahrscheinlich ist eine Fehldiagnose?
\item
Wie wahrscheinlich ist es, bei einem Patienten einen Ausschlag zu finden.
\end{teilaufgaben}

\begin{loesung}
Wir verwenden die Ereignisse:
\begin{align*}
G&=\{\text{Kind hat Grippe}\}\\
M&=\{\text{Kind hat Masern}\}\\
A&=\{\text{Kind hat Ausschlag}\}
\end{align*}
mit den Wahrscheinlichkeiten
\[
\begin{aligned}
P(G)&=0.9,&&&P(A|G)&=0.08,\\
P(M)&=0.1,&&&P(A|M)&=0.95.
\end{aligned}
\]
\begin{teilaufgaben}
\item
Gesucht ist die Wahrscheinlichkeit $P(G|A)$.
Nach dem Satz von Bayes ist sie
\[
P(G|A)=P(A|G)\frac{P(G)}{P(A)}.
\]
Darin ist $P(A)$ nicht bekannt, kann aber mit dem Satz "uber die totale
Wahrscheinlichkeit berechnet werden:
\[
P(A)=P(A|G)P(G)+P(A|M)P(M)=0.08\cdot 0.9 + 0.95\cdot 0.1=0.167,
\]
Damit wird die gesuchte Wahrscheinlichkeit
\[
P(G|A)=P(A|G)\frac{P(G)}{P(A)}=0.08\cdot\frac{0.9}{0.167}=0.431.
\]
Die Wahrscheinlichkeit f"ur eine Fehldiagnose ist also sehr hoch,
dieser Arzt handelt unverantwortlich.
\item Gesucht ist $P(A)$, was in der letzten Teilaufgabe bereits
berechnet wurde: $P(A)=0.167$.
\end{teilaufgaben}
\end{loesung}

