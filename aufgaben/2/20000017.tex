Eine Familie hat drei Kinder verschiedenen Jahrgangs. Welche der
folgenden Ereignisse sind unabhängig?

\begin{tabular}{rl}
A & Das älteste und das jüngste Kind haben das selbe Geschlecht.\\
B & Die Mädchen sind in der "Uberzahl.\\
C & Jeder Knabe hat eine jüngere Schwester.
\end{tabular}

\begin{loesung}
Elementarereignisse sind Tripel bestehend aus dem Geschlecht
des jüngsten, mittlern und des ältesten Kindes.
Jedes dieser 8 Elementarereignisse ist gleich wahrscheinlich (unter
der Annahme, dass Mädchen und Knaben gleich wahrscheinlich sind). Die
gesuchten Ereignisse sind
\begin{center}
\begin{tabular}{|c|l|l|}
\hline
Ereignis&Elementarereignisse&$P(\,\cdot\,)$\\
\hline
$A$&KMK, KKK, MKM, MMM&0.5\\
$B$&KMM, MKM, MMK, MMM&0.5\\
$C$&MMM, MMK, MKM, MKK&0.5\\
\hline
$A\cap B$&MKM, MMM&0.25\\
$A\cap C$&MMM, MKM&0.25\\
$B\cap C$&MMM, MMK, MKM&0.375\\
\hline
\end{tabular}
\end{center}
Nun ist zu prüfen, ob $P(X\cap Y)=P(X)\cdot P(Y)$ gilt für alle
möglichen Paare von Ereignissen $A$, $B$, $C$.
Die Produktregel trifft zu für
das Paar $(A,B)$ und $(A,C)$, nicht jedoch für $(B,C)$.
Somit sind $A,B$ und $A,C$ unabhängig.
\end{loesung}

\begin{loesung}
Das Ereignis $C$ ist gleichbedeutend damit, dass das jüngste
Kind ein Mädchen ist, also ist $P(C)=\frac12$. Das Ereignis $A$ hat
offensichtlich ebenfalls Wahrscheinlichkeit $\frac12$.
Die Mädchen sind in der "Uberzahl, wenn mindestens zwei Kinder Mädchen,
das geschieht in 4 von 8 möglichen Fällen, also ist auch $P(B)=\frac12$.

$A\cap B$: Wenn die Mädchen in der "Uberzahl sein müssen, dann kann
das Ereignis $A$ nur eintreten, wenn das älteste und das jüngste Kind
Mädchen sind, das Geschlecht des mittlere Kindes spielt also keine
Rolle, es sind somit zwei Fälle möglich, $P(A\cap B)=\frac14$
$\Rightarrow$
$A$ und $B$ sind unabhängig.

$A\cap C$: Da das jüngste Kind ein Mädchen sein muss, müssen wieder
das älteste und das jüngste Kind Mädchen sein. Sowohl ein Knabe
wie auch ein Mädchen als mittleres Kind erfüllt die Bedingung des
Ereignisses $C$, es sind also wieder zwei Fälle möglich, $P(A\cap C)=\frac14$
$\Rightarrow$
$A$ und $C$ sind unabhängig.

$B\cap C$: Wenn das jüngste Kind ein Mädchen ist, dann muss das mittlere
oder das älteste Kind ein Mädchen sein, damit auch die Bedingung von
Ereignis $B$ erfüllt ist. Dies ist nur dann nicht gegeben, wenn
das mittlere und das älteste Kind Knaben Knaben sind. Diese drei
Fälle haben Wahrscheinlichkeit $\frac38\ne P(B)\cdot P(C)$
$\Rightarrow$
$B$ und $C$ sind abhängig.
\end{loesung}

