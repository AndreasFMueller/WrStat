Eine Familie hat drei Kinder verschiedenen Jahrgangs. Welche der
folgenden Ereignisse sind unabh"angig?

\begin{tabular}{rl}
A & Das "alteste und das j"ungste Kind haben das selbe Geschlecht.\\
B & Die M"adchen sind in der "Uberzahl.\\
C & Jeder Knabe hat eine j"ungere Schwester.
\end{tabular}

\begin{loesung}
Elementarereignisse sind Tripel bestehend aus dem Geschlecht
des j"ungsten, mittlern und des "altesten Kindes.
Jedes dieser 8 Elementarereignisse ist gleich wahrscheinlich (unter
der Annahme, dass M"adchen und Knaben gleich wahrscheinlich sind). Die
gesuchten Ereignisse sind
\begin{center}
\begin{tabular}{|c|l|l|}
\hline
Ereignis&Elementarereignisse&$P(\,\cdot\,)$\\
\hline
$A$&KMK, KKK, MKM, MMM&0.5\\
$B$&KMM, MKM, MMK, MMM&0.5\\
$C$&MMM, MMK, MKM, MKK&0.5\\
\hline
$A\cap B$&MKM, MMM&0.25\\
$A\cap C$&MMM, MKM&0.25\\
$B\cap C$&MMM, MMK, MKM&0.375\\
\hline
\end{tabular}
\end{center}
Nun ist zu pr"ufen, ob $P(X\cap Y)=P(X)\cdot P(Y)$ gilt f"ur alle
m"oglichen Paare von Ereignissen $A$, $B$, $C$.
Die Produktregel trifft zu f"ur
das Paar $(A,B)$ und $(A,C)$, nicht jedoch f"ur $(B,C)$.
Somit sind $A,B$ und $A,C$ unabh"angig.
\end{loesung}

\begin{loesung}
Das Ereignis $C$ ist gleichbedeutend damit, dass das j"ungste
Kind ein M"adchen ist, also ist $P(C)=\frac12$. Das Ereignis $A$ hat
offensichtlich ebenfalls Wahrscheinlichkeit $\frac12$.
Die M"adchen sind in der "Uberzahl, wenn mindestens zwei Kinder M"adchen,
das geschieht in 4 von 8 m"oglichen F"allen, also ist auch $P(B)=\frac12$.

$A\cap B$: Wenn die M"adchen in der "Uberzahl sein m"ussen, dann kann
das Ereignis $A$ nur eintreten, wenn das "alteste und das j"ungste Kind
M"adchen sind, das Geschlecht des mittlere Kindes spielt also keine
Rolle, es sind somit zwei F"alle m"oglich, $P(A\cap B)=\frac14$
$\Rightarrow$
$A$ und $B$ sind unabh"angig.

$A\cap C$: Da das j"ungste Kind ein M"adchen sein muss, m"ussen wieder
das "alteste und das j"ungste Kind M"adchen sein. Sowohl ein Knabe
wie auch ein M"adchen als mittleres Kind erf"ullt die Bedingung des
Ereignisses $C$, es sind also wieder zwei F"alle m"oglich, $P(A\cap C)=\frac14$
$\Rightarrow$
$A$ und $C$ sind unabh"angig.

$B\cap C$: Wenn das j"ungste Kind ein M"adchen ist, dann muss das mittlere
oder das "alteste Kind ein M"adchen sein, damit auch die Bedingung von
Ereignis $B$ erf"ullt ist. Dies ist nur dann nicht gegeben, wenn
das mittlere und das "alteste Kind Knaben Knaben sind. Diese drei
F"alle haben Wahrscheinlichkeit $\frac38\ne P(B)\cdot P(C)$
$\Rightarrow$
$B$ und $C$ sind abh"angig.
\end{loesung}

