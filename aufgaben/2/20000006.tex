Innerhalb einer Risiko\-gruppe weiss man folgendes über die Wirksamkeit
der Grippe\-impfung:
\begin{enumerate}
\item 24\% der nicht geimpften erkranken.
\item Die Impfung halbiert die Erkrankungswahrscheinlichkeit.
\item Die Impfkampagne hat die Wahrscheinlichkeit
auf 18\% gesenkt.
\end{enumerate}
\begin{teilaufgaben}
\item Welcher Teil der Gruppe hat sich impfen lassen?
\item Wie gross ist die Wahrscheinlichkeit, dass ein Erkrankter
bereits geimpft war?
\end{teilaufgaben}

\thema{bedingte Wahrscheinlichkeit}
\thema{totale Wahrscheinlichkeit}
\thema{Satz von Bayes}

\begin{loesung}
Wir haben mit folgenden Ereignissen zu tun:
\begin{align*}
I&=\{\text{Patient ist geimpft}\}\\
K&=\{\text{Patient ist erkrankt}\}
\end{align*}
Die Aussagen in der Aufgaben bedeuten
\begin{enumerate}
\item $P(K|\bar I)=0.24$
\item $P(K|I)=0.12$
\item $P(K)=0.18$
\end{enumerate}
\begin{teilaufgaben}
\item
Der Satz über die totale Wahrscheinlichkeit bedeutet
\begin{align*}
P(K)&=P(K|I)P(I)+P(K|\bar I)P(\bar I)
\\
0.18&=0.12\cdot P(I)+0.24\cdot (1-P(I))
\\
0.18-0.24&=(0.12-0.24)\cdot P(I)
\\
P(I)&=\frac{-0.06}{-0.12}=\frac12
\end{align*}
Die Hälfte hat sich impfen lassen.
\item
Der Satz von Bayes liefert:
\[
P(I|K)=P(K|I)\frac{P(I)}{P(K)}=0.12\cdot\frac{0.5}{0.18}=\frac13.
\]
Ein Drittel der Kranken ist also bereits geimpft.
\qedhere
\end{teilaufgaben}
\end{loesung}

