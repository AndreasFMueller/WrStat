In einem Spiel wird statt eines W"urfels eine faire M"unze wie folgt verwendet:
Der Spieler wirft die M"unze so lange, bis sie Kopf zeigt, maximal aber sechs mal.
Sei $k$ die Anzahl der W"urfe bis zum ersten Kopf.
\begin{teilaufgaben}
\item
Beschreiben Sie $\Omega$ f"ur dieses Experiment, und die Ereignisse zu jedem $k$.
\item
Berechnen Sie die Wahrscheinlichkeit f"ur jedes $k$.
\item
Betrachten Sie folgende Variante des Experimentes: man wirft die M"unze genau
sechs mal hintereinander und pr"uft erst nach dem letzten Wurf, in welchem Wurf
zum ersten Mal Kopf aufgetreten ist, dies ist die Zahl $k$.
Wie sehen $\Omega$ und die Ereignisse f"ur jedes $k$ f"ur dieses Experiment aus?
Berechnen Sie auch die Anzahl Elementarereignisse in jedem Ereignis.
\item 
Bestimmen Sie die Wahrscheinlichkeit f"ur $k$ f"ur die Variante des Experimentes
aus Teilaufgabe c).
\item
Ist der sechste Wurf der M"unze n"otig? Wenn nein, wie sehen $\Omega$ und die
Ereignisse f"ur jedes $k$ aus, wenn man die M"unze nur f"unf mal wirft?
\item
Was "andert sich, wenn eine beliebige Anzahl W"urfe erlaubt ist?
Kontrollieren Sie insbesondere, ob $\Omega=\bigcup_{k=1}^\infty A_k$
und ob die Summenformel stimmt.
\end{teilaufgaben}

\begin{loesung}
\begin{teilaufgaben}
\item
Die m"oglichen Versuchsausg"ange k"onnen durch die Folge von Kopf oder Zahl
beschrieben werden, die mit K und Z abgek"urzt werden sollen.
H"ochstens das letzte Zeichen der Folge darf ein K sein, da nach einem
K nicht mehr weiter geworfen werden darf.
Somit ist
\[
\Omega = \{
\text{K},
\text{ZK},
\text{ZZK},
\text{ZZZK},
\text{ZZZZK},
\text{ZZZZZK},
\text{ZZZZZZ}
\}
\]
Das Ereignis 
\[
A_k=\{
\text{$k$ W"urfe bis zum Abbruch}
\}
\]
besteht f"ur $k<6$ genau aus dem Elementarereignis ``Kopf im $k$-ten Wurf'',
f"ur $k=6$ ist $A_6= \{\text{ZZZZZK}, \text{ZZZZZZ}\}$.
\item
Der einzelne M"unzwurf ist zwar ein Laplace-Experiment, doch bedeutet das
nicht, dass die Elementarereignisse in $\Omega$ die gleiche Wahrscheinlichkeit
haben.
In der H"alfte der F"alle wird das Experiment wegen eines K im ersten Wurf abbrechen,
es kommt daher gar nicht mehr zu einem zweiten Wurf.
Der Fortgang des Experimentes h"angt ab vom Resultat im ersten Wurf.

In der H"alfte der F"alle ist das Experiment nach dem ersten Wurf zu Ende,
es ist also
\[
P(A_1) = \frac12.
\]
In der anderen H"alfte der F"alle wird noch einmal geworfen, dabei bricht das 
Experiment wieder in der H"alfte der F"alle ab.
Durch Wiederholen erh"alt man die Wahrscheinlichkeiten
\[
P(A_2)=\frac14,\qquad
P(A_3)=\frac18,\qquad
P(A_4)=\frac1{16},\qquad
P(A_5)=P(A_6)=\frac1{32}.
\]
\item
In dieser Variante des Experiments wird die M"unze immer sechs mal geworfen,
es entstehen also immer Folgen von sechs Zeichen K und Z,
\begin{align*}
\Omega&=\{
\text{KKKKK},
\text{KKKKZ},
\text{KKKZK},
\text{KKKZZ},
\text{KKZKK},\dots
\},&|\Omega|&=2^6=64,
\end{align*}
sie werden nur anders zu Ereignissen zusammengefasst:
\begin{align*}
A_6&=\{\text{ZZZZZZ}, \text{ZZZZZK}\},&|A_6|&=2\\
A_5&=\{
\text{ZZZZKK},
\text{ZZZZKZ}
\},&|A_5|&=2\\
A_4&=\{
\text{ZZZKKK},
\text{ZZZKKZ},
\text{ZZZKZK},
\text{ZZZKZZ}
\},&|A_4|&=4\\
A_3&=\{
&|A_3|&=8\\
&\qquad
\text{ZZKKKK},
\text{ZZKKKZ},
\text{ZZKKZK},
\text{ZZKKZZ},\\
&\qquad\text{ZZKZKK},
\text{ZZKZKZ},
\text{ZZKZZK},
\text{ZZKZZZ}\\
&\phantom{\mathstrut=\mathstrut }\},\\
A_2&=\{\text{Folgen mit K an der zweiten Stelle}\},&|A_2|&=16\\
A_1&=\{\text{mit K beginnende Folgen}\},&|A_1|&=32\\
\end{align*}
Zusammengefasst:
\[
|A_k|=\begin{cases}
2^{6-k}&\qquad k<6\\
2&\qquad k=6.
\end{cases}
\]
\item
Da der Ablauf des Experimentes nicht mehr von Zwischenresultaten abh"angig ist,
sind alle Elementarereignisse gleich wahrscheinlich, es liegt ein Laplace-Experiment
vor.
Daher kann man die Wahrscheinlichkeit aus der Kardinalit"at der Mengen $A_k$
bestimmen:
\[
P(A_k)=\frac{|A_k|}{|\Omega|}
=\begin{cases}
\frac{2^{6-k}}{2^6}=2^{-k}&\qquad k<6\\
\frac{2}{2^6}=2^{-5}&\qquad k=6.
\end{cases}
\]
\item
Wenn im f"unften Wurf zum f"unften Mal ein Z geworfen wurde ist klar,
dass das Ereignis $A_6$ eintreten wird.
Es ist also nicht mehr n"otig, die M"unze ein sechstes Mal zu werfen, 
das Resultat hat keinen Einfluss mehr auf das Eintreten des Ereignisses
$A_6$.
Dieses modifizierte Experiment hat die Elementarereignisse
\[
\Omega = \{
\text{K},
\text{ZK},
\text{ZZK},
\text{ZZZK},
\text{ZZZZK},
\text{ZZZZZ}
\},
\]
und die Ereignisse $A_k$ sind
\[
A_1=\{\text{K}\},\quad
A_2=\{\text{ZK}\},\quad
A_3=\{\text{ZZK}\},\quad
A_4=\{\text{ZZZK}\},\quad
A_5=\{\text{ZZZZK}\},\quad
A_6=\{\text{ZZZZZ}\}.
\]
\item
Wenn die M"unze beliebig oft geworfen werden darf, dann wird $\Omega$ die unendlich
viele Versuchsausg"ange
\[
\Omega = \{
\text{K},
\text{ZK},
\text{ZZK},
\text{ZZZK},
\text{ZZZZK},
\text{ZZZZZK},
\dots
\}
\]
enthalten, und die das Ereignis $A_k$ enth"alt genau das Elementarereignis,
das man $\text{Z}^{k-1}\text{K}$ schreiben k"onnte.
Es hat Wahrscheinlichkeit
\[
P(A_k)=2^{-k}.
\]
Die Summenformel liefert
\[
\bigcup_{k=1}^\infty A_k=\Omega
\qquad\Rightarrow\qquad
1=P(\Omega)=P\biggl(
\bigcup_{k=1}^\infty A_k
\biggr)
=\sum_{k=1}^\infty P(A_k)=\sum_{k=1}^\infty 2^{-k}=
\frac12+\frac14+\frac18+\frac1{16}+\dots
\]
Rechts steht die Summer einer geometrischen Reihe mit Quotient $q=\frac12$ und
Anfangsterm $a=\frac12$, die Summenformel f"ur die geometrische Reihe liefert
\[
\frac12+\frac14+\frac18+\frac1{16}+\dots=\frac{a}{1-q}=\frac12\frac1{1-\frac12}=1,
\]
die Summenformel ist also erf"ullt.
\end{teilaufgaben}
\end{loesung}

