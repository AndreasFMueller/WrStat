In einem Spiel wird statt eines Würfels eine faire Münze wie folgt verwendet:
Der Spieler wirft die Münze so lange, bis sie Kopf zeigt, maximal aber sechs mal.
Sei $k$ die Anzahl der getätigten Würfe.
\begin{teilaufgaben}
\item
Beschreiben Sie $\Omega$ für dieses Experiment, und die Ereignisse zu jedem $k$.
\item
Berechnen Sie die Wahrscheinlichkeit für jedes $k$.
\item
Betrachten Sie folgende Variante des Experimentes: man wirft die Münze genau
sechs mal hintereinander und prüft erst nach dem letzten Wurf, in welchem Wurf
zum ersten Mal Kopf aufgetreten ist, dies ist die Zahl $k$.
Wie sehen $\Omega$ und die Ereignisse für jedes $k$ für dieses Experiment aus?
Berechnen Sie auch die Anzahl Elementarereignisse in jedem Ereignis.
\item 
Bestimmen Sie die Wahrscheinlichkeit für $k$ für die Variante des Experimentes
aus Teilaufgabe c).
\item
Ist der sechste Wurf der Münze nötig? Wenn nein, wie sehen $\Omega$ und die
Ereignisse für jedes $k$ aus, wenn man die Münze nur fünf mal wirft?
\item
Was ändert sich, wenn eine beliebige Anzahl Würfe erlaubt ist?
Kontrollieren Sie insbesondere, ob $\Omega=\bigcup_{k=1}^\infty A_k$
und ob die Summenformel stimmt.
\end{teilaufgaben}

\thema{Ereignis}
\thema{Laplace-Experiment}
\thema{Rechenregeln für Wahrscheinlichkeit}

\begin{loesung}
\begin{teilaufgaben}
\item
Die möglichen Versuchsausgänge können durch die Folge von Kopf oder Zahl
beschrieben werden, die mit K und Z abgekürzt werden sollen.
Höchstens das letzte Zeichen der Folge darf ein K sein, da nach einem
K nicht mehr weiter geworfen werden darf.
Somit ist
\[
\Omega = \{
\text{K},
\text{ZK},
\text{ZZK},
\text{ZZZK},
\text{ZZZZK},
\text{ZZZZZK},
\text{ZZZZZZ}
\}
\]
Das Ereignis 
\[
A_k=\{
\text{$k$ Würfe bis zum Abbruch}
\}
\]
besteht für $k<6$ genau aus dem Elementarereignis ``Kopf im $k$-ten Wurf'',
für $k=6$ ist $A_6= \{\text{ZZZZZK}, \text{ZZZZZZ}\}$.
\item
Der einzelne Münzwurf ist zwar ein Laplace-Experiment, doch bedeutet das
nicht, dass die Elementarereignisse in $\Omega$ die gleiche Wahrscheinlichkeit
haben.
In der Hälfte der Fälle wird das Experiment wegen eines K im ersten Wurf abbrechen,
es kommt daher gar nicht mehr zu einem zweiten Wurf.
Der Fortgang des Experimentes hängt ab vom Resultat im ersten Wurf.

In der Hälfte der Fälle ist das Experiment nach dem ersten Wurf zu Ende,
es ist also
\[
P(A_1) = \frac12.
\]
In der anderen Hälfte der Fälle wird noch einmal geworfen, dabei bricht das 
Experiment wieder in der Hälfte der Fälle ab.
Durch Wiederholen erhält man die Wahrscheinlichkeiten
\[
P(A_2)=\frac14,\qquad
P(A_3)=\frac18,\qquad
P(A_4)=\frac1{16},\qquad
P(A_5)=P(A_6)=\frac1{32}.
\]
\item
In dieser Variante des Experiments wird die Münze immer sechs mal geworfen,
es entstehen also immer Folgen von sechs Zeichen K und Z,
\begin{align*}
\Omega&=\{
\text{KKKKKK},
\text{KKKKKZ},
\text{KKKKZK},
\text{KKKKZZ},
\text{KKKZKK},\dots
\},&|\Omega|&=2^6=64,
\end{align*}
sie werden nur anders zu Ereignissen zusammengefasst:
\begin{align*}
A_6&=\{\text{ZZZZZZ}, \text{ZZZZZK}\},&|A_6|&=2\\
A_5&=\{
\text{ZZZZKK},
\text{ZZZZKZ}
\},&|A_5|&=2\\
A_4&=\{
\text{ZZZKKK},
\text{ZZZKKZ},
\text{ZZZKZK},
\text{ZZZKZZ}
\},&|A_4|&=4\\
A_3&=\{
&|A_3|&=8\\
&\qquad
\text{ZZKKKK},
\text{ZZKKKZ},
\text{ZZKKZK},
\text{ZZKKZZ},\\
&\qquad\text{ZZKZKK},
\text{ZZKZKZ},
\text{ZZKZZK},
\text{ZZKZZZ}\\
&\phantom{\mathstrut=\mathstrut }\},\\
A_2&=\{\text{Folgen mit dem ersten K an der zweiten Stelle}\},&|A_2|&=16\\
A_1&=\{\text{mit K beginnende Folgen}\},&|A_1|&=32\\
\end{align*}
Zusammengefasst:
\[
|A_k|=\begin{cases}
2^{6-k}&\qquad k<6\\
2&\qquad k=6.
\end{cases}
\]
\item
Da der Ablauf des Experimentes nicht mehr von Zwischenresultaten abhängig ist,
sind alle Elementarereignisse gleich wahrscheinlich, es liegt ein Laplace-Experiment
vor.
Daher kann man die Wahrscheinlichkeit aus der Kardinalität der Mengen $A_k$
bestimmen:
\[
P(A_k)=\frac{|A_k|}{|\Omega|}
=\begin{cases}
\frac{2^{6-k}}{2^6}=2^{-k}&\qquad k<6\\
\frac{2}{2^6}=2^{-5}&\qquad k=6.
\end{cases}
\]
\item
Wenn im fünften Wurf zum fünften Mal ein Z geworfen wurde ist klar,
dass das Ereignis $A_6$ eintreten wird.
Es ist also nicht mehr nötig, die Münze ein sechstes Mal zu werfen, 
das Resultat hat keinen Einfluss mehr auf das Eintreten des Ereignisses
$A_6$.
Dieses modifizierte Experiment hat die Elementarereignisse
\[
\Omega = \{
\text{K},
\text{ZK},
\text{ZZK},
\text{ZZZK},
\text{ZZZZK},
\text{ZZZZZ}
\},
\]
und die Ereignisse $A_k$ sind
\[
A_1=\{\text{K}\},\quad
A_2=\{\text{ZK}\},\quad
A_3=\{\text{ZZK}\},\quad
A_4=\{\text{ZZZK}\},\quad
A_5=\{\text{ZZZZK}\},\quad
A_6=\{\text{ZZZZZ}\}.
\]
\item
Wenn die Münze beliebig oft geworfen werden darf, dann wird $\Omega$ die unendlich
viele Versuchsausgänge
\[
\Omega = \{
\text{K},
\text{ZK},
\text{ZZK},
\text{ZZZK},
\text{ZZZZK},
\text{ZZZZZK},
\dots
\}
\]
enthalten, und die das Ereignis $A_k$ enthält genau das Elementarereignis,
das man $\text{Z}^{k-1}\text{K}$ schreiben könnte.
Es hat Wahrscheinlichkeit
\[
P(A_k)=2^{-k}.
\]
Die Summenformel liefert
\[
\bigcup_{k=1}^\infty A_k=\Omega
\qquad\Rightarrow\qquad
1=P(\Omega)=P\biggl(
\bigcup_{k=1}^\infty A_k
\biggr)
=\sum_{k=1}^\infty P(A_k)=\sum_{k=1}^\infty 2^{-k}=
\frac12+\frac14+\frac18+\frac1{16}+\dots
\]
Rechts steht die Summe einer geometrischen Reihe mit Quotient $q=\frac12$ und
Anfangsterm $a=\frac12$, die Summenformel für die geometrische Reihe liefert
\[
\frac12+\frac14+\frac18+\frac1{16}+\dots=\frac{a}{1-q}=\frac12\frac1{1-\frac12}=1,
\]
die Summenformel ist also erfüllt.
\qedhere
\end{teilaufgaben}
\end{loesung}

