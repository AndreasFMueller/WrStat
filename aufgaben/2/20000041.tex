In einer Studie wurden "Arzte dar"uber befragt, wie sie das Burstkrebsrisiko
von gewissen Patientinnen einsch"atzen.
Es ging um Frauen, die mit einer Wahrscheinlichkeit von 1\% an
Brustkrebs erkranken.
Ein Mammogramm erkennt die Krankheit in 80\% der F"alle, zeigt aber auch
bei 10\% der gesunden Frauen Brustkrebs an.
Die "Arzte wurden befragt, mit welcher Wahrscheinlichkeit eine Frau
Brustkrebs hat, wenn ein positiver Mammogramm-Befund vorliegt.
Von hundert "Arzten sch"atzen 95, die Wahrscheinlichkeit sei etwa 75\%.
Was meinen Sie dazu?

\begin{loesung}
Wir verwenden die Ereignisse
\begin{align*}
B&=\{\text{Patientin hat Brustkrebs}\}\\
G&=\{\text{Patientin ist besund}\} =\bar B\\
M&=\{\text{Patientin hat postiven Mammogramm-Befund}\}
\end{align*}
mit den Wahrscheinlichkeiten
\[
\begin{aligned}
P(B)&=0.01,&&\qquad&P(M|B)&=0.8,\\
    &      &&\qquad&P(M|G)&=0.1.
\end{aligned}
\]
Gesucht ist die Wahrscheinlichkeit $P(B|M)$, die mit Hilfe des Satzes
von Bayes berechnet werden kann:
\[
P(B|M) = P(M|B) \frac{P(B)}{P(M)}
\]
Die Wahrscheinlichkeit $P(M)$ kann mit dem Satz "uber die totale
Wahrscheinlichkeit berechnet werden:
\begin{align*}
P(M)
&=
P(M|B)P(B)+P(M|\bar B)P(\bar B)
=
0.8\cdot 0.01 + 0.1\cdot 0.99
=
0.107.
\end{align*}
Damit kann man jetzt die gesuchte Wahrscheinlichkeit $P(B|M)$ bestimmen
\[
P(B|M)
=
P(M|B) \frac{P(B)}{P(M)}
=
0.8 \cdot\frac{0.01}{0.107}
=
0.075.
\]
Die befragten "Arzte lagen um eine ganze Gr"ossenordnung falsch.
Der Grund daf"ur ist die Seltenheit der Krankheit: Ereignisse
mit geringer Wahrscheinlichkeit k"onnen nur mit einem Test zuverl"assig
detektiert werden, der eine noch viel geringere Fehlerwahrscheinlichkeit hat.
\end{loesung}
