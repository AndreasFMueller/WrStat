Dominosteine enthalten zwei Felder, in denen die Augenzahlen
0 bis 6 eingetragen sind.
\begin{teilaufgaben}
\item Wieviele verschiedene Dominosteine gibt es?
\item Wir nehmen an, wir z"ogen aus einem vollst"andigen Satz verschiedener
Dominosteine jeweils einen Stein.
Wie gross ist die erwartete Augensumme?
\item
Wie gross die Wahrscheinlichkeit, einen Stein mit genau der
erwarteten Augensumme zu ziehen?
\end{teilaufgaben}

\begin{loesung}
\begin{figure}
\centering
\includeagraphics[]{graph-1.pdf}
\caption{Histogramm der Augensummen von Dominosteinen in Aufgabe~\ref{20000035}.
\label{20000035:histogram}}
\end{figure}
\begin{teilaufgaben}
\item Dominosteine sind Paare von Zahlen aus $\{0,1,2,3,4,5,6\}$, wobei
wir $(x,y)$ und $(y,x)$ als gleich zu betrachten haben. Wir k"onnen die
Steine daher so anordnen, dass $x\ge y$, es ergibt sich dann folgende
Darstellung aller Elementarereignisse:
\[
\begin{matrix}
(0,0)&(1,0)&(2,0)&(3,0)&(4,0)&(5,0)&(6,0)\\
     &(1,1)&(2,1)&(3,1)&(4,1)&(5,1)&(6,1)\\
     &     &(2,2)&(3,2)&(4,2)&(5,2)&(6,2)\\
     &     &     &(3,3)&(4,3)&(5,3)&(6,3)\\
     &     &     &     &(4,4)&(5,4)&(6,4)\\
     &     &     &     &     &(5,5)&(6,5)\\
     &     &     &     &     &     &(6,6)
\end{matrix}
\]
Die gesucht Anzahl ist also die Summe der Zahlen von 1 bis 7,
\[
\sum_{k=1}^7k=\frac{7(7+1)}2=28.
\]
\item Ersetzt man den Dominostein $(x,y)$ durch den Stein $(6-x,6-y)$,
ensteht ein Stein mit der Augenzahl $12-(x+y)$. Durch die genannte
Ersetzung werden die Dominosteine aufeinander abgebildet, die Dominosteine
mit Augensumme $\ge6$ kommen also genau gleich oft vor wie jene mit
Augensumme $\le6$, und zu ``komplement"aren'' Augensummen (also Augensummen,
die sich zu 12 addieren) jeweils auch in gleicher Zahl. Daher ist
der Erwartungswert bei dieser Operation ebenfalls unver"andert,
\[
E(X+Y)= 12-E(X+Y)\quad\Rightarrow\quad E(X+Y)=6.
\]
Alternativ kann man dies auch aus der obigen Auflistung aller
Elementarereignisse abz"ahlen. Es ergibt sich die folgende Tabelle
der m"oglichen Augensummen:
\begin{center}
\begin{tabular}{|l|rrrrrrrrrrrrr|}
\hline
Augensumme & 0& 1& 2& 3& 4& 5& 6& 7& 8& 9& 10& 11& 12\\
\hline
Anzahl     & 1& 1& 2& 2& 3& 3& 4& 3& 3& 2&  2&  1&  1\\
\hline
\end{tabular}
\end{center}
oder das Histogram in Abbildung~\ref{20000035:histogram}.
Wieder erkennt man die Symmetrie der Anzahlen, woraus man bereits
ablesen k"onnte, dass der Erwartungswert der Augensumme $6$ ist.
Man kann ihn aber auch explizit ausrechnen:
\begin{align*}
E(X+Y)&=
\frac1{28}(
0\cdot 1+
1\cdot 1+
2\cdot 2+
3\cdot 2+
4\cdot 3+
5\cdot 3+
6\cdot 4\\
&\qquad +
7\cdot 3+
8\cdot 3+
9\cdot 2+
10\cdot 2+
11\cdot 1+
12\cdot 1
)
\\
&=
\frac1{28}(1+4+6+12+15+24+21+24+18+20+11+12)\\
&=
\frac{168}{28}=6.
\end{align*}
\item
Die folgenden Dominosteine haben Augensumme 6:
\[
\{\text{Augensumme 6}\}
=\{
(0,6),\;
(1,5),\;
(2,4),\;
(3,3)\}.
\]
Also ist die Wahrscheinlichkeit, die Augensumme 6 zu erreichen
\[
P(\text{Augensumme 6})=\frac{4}{28}=\frac17=0.14285714285714285714.
\]
\end{teilaufgaben}
\end{loesung}

