"Uber die Todesf"alle in der originalen Star Trek Fernsehserie wurden
detaillierte Statistiken gef"uhrt:
\begin{center}
\begin{tabular}{|l|l|r|r|}
\hline
Funktion&Uniformfarbe&Bestand&Todesf"alle\\
\hline
Command&Gold&55&9\\
Science&Blau&136&7\\
Engineering and Support&Rot&239&24\\
&Unbekannt&&15\\
\hline
\end{tabular}
\end{center}
Die Enterprise (NCC-1701) hat einen Bestand von 430 Mann.
Die 15 Toten, die nicht einer Uniformfarbe zugeordnet werden k"onnen, werden
im folgenden einfach ignoriert.
\begin{teilaufgaben}
\item
Wie gross ist die Wahrscheinlichkeit, dass ein wissenschaftlicher Mitarbeiter
ums Leben kommt?
\item
Wie gross ist die Wahrscheinlichkeit, dass ein Toter zum Wissenschaftsteam
geh"ort?
\item
Wie gross ist die Wahrscheinlichkeit, dass jemand ums Leben kommt, der
nicht zum Kommando geh"ort?
\item 
Die Sicherheitsabteilung (Security) innerhalb von Engineering and Support
hat 90 Mitglieder, wovon 18 ums Leben gekommen sind.
Haben Security-Mitarbeiter eine h"ohere Wahrscheinlichkeit umzukommen?
\item
Wie gross ist die Todesfallwahrscheinlichkeit f"ur ein Redshirt, welches
nicht Security-Mitglied ist?
\end{teilaufgaben}

\begin{loesung}
Wir verwenden folgende Ereignisse:
\begin{align*}
G&=\{\text{Goldshirts}\},\\
B&=\{\text{Blueshirts}\},\\
R&=\{\text{Redshirts}\},\\
S&=\{\text{Security}\}\subset R,\\
T&=\{\text{in der originalen Star Trek Serie gestorben}\}.
\end{align*}
\begin{teilaufgaben}
\item
$
\displaystyle
P(T|B)
=
\frac{P(T\cap B)}{P(B)}
=
\frac{7/430}{136/430}
=
\frac{7}{136}
=0.0515.
$
\item
$
\displaystyle
P(B|T)
=
\frac{P(B\cap T)}{P(T)}
=
\frac{7/430}{40/430}
=
\frac{7}{40}
=0.175.
$
\item
$
\displaystyle
P(T|\bar G)
=
\frac{P(T\cap\bar G)}{P(\bar G)}
=
\frac{31/430}{375/430}
=
\frac{31}{375}
=0.0827.
$
\item
In a) haben wir bereits $P(T|B)$ berechnet, f"ur den Vergleich
brauchen wir aber noch alle anderen bedingten Wahrscheinlichkeiten:
\begin{align*}
P(T|S)&
=\frac{P(T\cap S)}{P(S)}
=\frac{18/430}{90/430}
=\frac{18}{90}
=0.2,
\\
P(T|G)&
=\frac{P(T\cap G)}{P(G)}
=\frac{9/430}{55/430}
=\frac{9}{55}
=0.164,
\\
P(T|R)&
=\frac{P(T\cap R)}{P(R)}
=\frac{24/430}{239/430}
=\frac{24}{239}
=0.1.
\end{align*}
Daraus lesen wir ab, dass Security tats"achlich ein gef"ahrlicher
Job ist.
\item
$
\displaystyle
P(T|R\setminus S)
=\frac{P(T\cap(R\setminus S))}{P(R\setminus S)}
=\frac{6/430}{149/430}
=\frac{6}{149}
=0.0403$.
\qedhere
\end{teilaufgaben}
\end{loesung}

\begin{diskussion}
Ein Star Trek Mythos besagt, dass Redshirts besonders gef"ahrlich
leben. Die Aufgabe zeigt, dass dies nicht zutrifft.
Redshirts leben zwar deutlich gef"ahrlicher als Blueshirts, aber
dies r"uhrt vollst"andig von Security her. Die Todesfallwahrscheinlichkeit
f"ur ein Nicht-Security Redshirt ist sogar geringer als f"ur Blueshirts.
\end{diskussion}


