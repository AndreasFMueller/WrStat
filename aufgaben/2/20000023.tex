"Uber die Todesfälle in der originalen Star Trek Fernsehserie wurden
detaillierte Statistiken geführt:
\begin{center}
\begin{tabular}{|l|l|r|r|}
\hline
Funktion&Uniformfarbe&Bestand&Todesfälle\\
\hline
Command&Gold&55&9\\
Science&Blau&136&7\\
Engineering and Support&Rot&239&24\\
&Unbekannt&&15\\
\hline
\end{tabular}
\end{center}
Die Enterprise (NCC-1701) hat einen Bestand von 430 Mann.
Die 15 Toten, die nicht einer Uniformfarbe zugeordnet werden können, werden
im folgenden einfach ignoriert.
\begin{teilaufgaben}
\item
Wie gross ist die Wahrscheinlichkeit, dass ein wissenschaftlicher Mitarbeiter
ums Leben kommt?
\item
Wie gross ist die Wahrscheinlichkeit, dass ein Toter zum Wissenschaftsteam
gehört?
\item
Wie gross ist die Wahrscheinlichkeit, dass jemand ums Leben kommt, der
nicht zum Kommando gehört?
\item 
Die Sicherheitsabteilung (Security) innerhalb von Engineering and Support
hat 90 Mitglieder, wovon 18 ums Leben gekommen sind.
Haben Security-Mitarbeiter eine höhere Wahrscheinlichkeit umzukommen?
\item
Wie gross ist die Todesfallwahrscheinlichkeit für ein Redshirt, welches
nicht Security-Mitglied ist?
\end{teilaufgaben}

\thema{bedingte Wahrscheinlichkeit}
\thema{Ereignis}

\begin{loesung}
Wir verwenden folgende Ereignisse:
\begin{align*}
G&=\{\text{Goldshirts}\},\\
B&=\{\text{Blueshirts}\},\\
R&=\{\text{Redshirts}\},\\
S&=\{\text{Security}\}\subset R,\\
T&=\{\text{in der originalen Star Trek Serie gestorben}\}.
\end{align*}
\begin{teilaufgaben}
\item
$
\displaystyle
P(T|B)
=
\frac{P(T\cap B)}{P(B)}
=
\frac{7/430}{136/430}
=
\frac{7}{136}
=0.0515.
$
\item
$
\displaystyle
P(B|T)
=
\frac{P(B\cap T)}{P(T)}
=
\frac{7/430}{40/430}
=
\frac{7}{40}
=0.175.
$
\item
$
\displaystyle
P(T|\bar G)
=
\frac{P(T\cap\bar G)}{P(\bar G)}
=
\frac{31/430}{375/430}
=
\frac{31}{375}
=0.0827.
$
\item
In a) haben wir bereits $P(T|B)$ berechnet, für den Vergleich
brauchen wir aber noch alle anderen bedingten Wahrscheinlichkeiten:
\begin{align*}
P(T|S)&
=\frac{P(T\cap S)}{P(S)}
=\frac{18/430}{90/430}
=\frac{18}{90}
=0.2,
\\
P(T|G)&
=\frac{P(T\cap G)}{P(G)}
=\frac{9/430}{55/430}
=\frac{9}{55}
=0.164,
\\
P(T|R)&
=\frac{P(T\cap R)}{P(R)}
=\frac{24/430}{239/430}
=\frac{24}{239}
=0.1.
\end{align*}
Daraus lesen wir ab, dass Security tatsächlich ein gefährlicher
Job ist.
\item
$
\displaystyle
P(T|R\setminus S)
=\frac{P(T\cap(R\setminus S))}{P(R\setminus S)}
=\frac{6/430}{149/430}
=\frac{6}{149}
=0.0403$.
\qedhere
\end{teilaufgaben}
\end{loesung}

\begin{diskussion}
Ein Star Trek Mythos besagt, dass Redshirts besonders gefährlich
leben. Die Aufgabe zeigt, dass dies nicht zutrifft.
Redshirts leben zwar deutlich gefährlicher als Blueshirts, aber
dies rührt vollständig von Security her. Die Todesfallwahrscheinlichkeit
für ein Nicht-Security Redshirt ist sogar geringer als für Blueshirts.
\end{diskussion}


