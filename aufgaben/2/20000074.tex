Betrachten Sie wieder das Ziehen einer Karte aus einem Kartenspiel
mit 52 Karten und das Ereignis
\[
X
=
\{
\text{schwarze Bildkarte oder rote Zahlkarte}
\}.
\]

\begin{teilaufgaben}
\item
In Aufgabe~\ref{20000073} wurde $P(K|R)$ und $P(R|K)$.
Verwenden Sie den Satz von Bayes um die gefundenen Resultate zu
verifizieren.
\item
Bestimmen Sie $P(X)$ mit Hilfe des Satzes von der totalen Wahrscheinlichkeit.
\end{teilaufgaben}

\begin{loesung}
\begin{teilaufgaben}
\item
Der Satz von Bayes besagt
\begin{align*}
\frac{2}{26}
=
P(K|R)
&=
P(R|K)\cdot \frac{P(K)}{P(R)}
=
\frac{1}{2}
\cdot
\frac{4/52}{26/52}
=
\frac{2}{26}.
\end{align*}
\item
Nach dem Satz von der totalen Wahrscheinlichkeit ist
\begin{align*}
P(X)
&=
P(X|R) \cdot P(R) + P(X|S) \cdot P(S)
\\
&=
\frac{10/52}{26/52}\cdot \frac{26}{52}
+
\frac{3/52}{26/52}\cdot \frac{26}{52}
\\
&=
\frac{10}{26} \cdot \frac{26}{52}
+
\frac{4}{26} \cdot \frac{26}{52}
\\
&=
\frac{14}{26}
=
\frac{7}{13}
=
0.\overline{538461}.
\qedhere
\end{align*}
\end{teilaufgaben}
\end{loesung}
