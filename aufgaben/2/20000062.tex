In einem perfekten Justizsystem werden Schuldige verurteilt und
Unschuldige freigesprochen.
Reale Justizsysteme sind dagegen eher stochastische Prozesse, in denen
Fehler mit nicht verschwindender Wahrscheinlichkeit auftreten.
Eine statistische Langzeituntersuchung über Angeklagte und ihr Schicksal
hat folgende Zahlen zu Tage gefördert.
Da es der Staatsanwaltschaft nicht immer gelingt, Schuld zu beweisen,
werden nur 82\% der Schuldigen tatsächlich verurteilt.
Normalerweise werden Unschuldige freigesprochen, dies geschieht in
80\% der Fälle.
Von den Angeklagten sind 85\% schuldig.

\begin{teilaufgaben}
\item
Wie gross ist die Wahrscheinlichkeit, dass ein Angeklagter 
verurteilt wird?
\item
Wie gross ist die Wahrscheinlichkeit, dass ein Verurteilter tatsächlich
unschuldig ist?
\end{teilaufgaben}

\begin{loesung}
Wir verwenden die Ereignisse
\begin{align*}
S&=             \{\text{schuldig}\}      \\
U&=\overline{S}=\{\text{unschuldig}\}    \\
V&=             \{\text{verurteilt}\}    \\
F&=\overline{V}=\{\text{freigesprochen}\}.
\end{align*}
Aus dem Aufgabentext können folgende Wahrscheinlichkeiten abgelesen
werden:
\[
\begin{aligned}
P(S)   &= 0.85 &&\Rightarrow& P(\bar{S}) &= 1- P(S) = 0.15 \\
P(V|S) &= 0.82 &&\Rightarrow&     P(F|S) &= P(\bar{V}|S) = 1-P(V|S) = 0.18 \\
P(F|U) &= 0.80 &&\Rightarrow&     P(V|U) &= P(\bar{V}|U) = 1-P(F|U) = 0.20
\end{aligned}
\]
Damit können die Fragen beantwortet werden.
\begin{teilaufgaben}
\item
Mit dem Satz von der totalen Wahrscheinlichkeit findet man
\begin{align*}
P(V)
&=
P(V|S) P(S) + P(V|\bar{S}) P(\bar{S})
\\
&=
P(V|S) P(S) + P(V|U) P(\bar S)
\\
&=
P(V|S) P(S) + P(V|U) (1-P(S))
\\
&=
0.82\cdot 0.85 + 0.20\cdot 0.15
=
0.727.
\end{align*}
72.7\% der Angeklagten werden verurteilt.
\item
Gesucht ist die bedingte Wahrscheinlichkeit $P(U|V)=1-P(S|V)$,
wir wissen jedoch nur $P(V|S)$.
Man kann die gesuchte Wahrscheinlichekit
wie folgt mit dem Satz von Bayes berechnen:
\begin{align*}
P(S|V)
&=
P(V|S)\frac{P(S)}{P(V)}
=
0.82\cdot\frac{0.85}{0.727}
=
0.9587.
\end{align*}
Daraus  findet man
\[
P(U|V) = 1-0.9587 = 0.0413.
\]
4.13\% der Verurteilten sind unschuldig.
\qedhere
\end{teilaufgaben}
\end{loesung}

\begin{bewertung}
Ereignisse ({\bf E}) 1 Punkt,
bedingte Wahrscheinlichkeiten ({\bf C}) 1 Punkt,
Satz von der totalen Wahrscheinlichkeit ({\bf T}) 1 Punkt,
Wahrscheinlichkeit in a) ({\bf A}) 1 Punkt,
Satz von Bayes ({\bf B}) 1 Punkt,
Wahrscheinlichkeit in b) ({\bf W}) 1 Punkt.
\end{bewertung}

