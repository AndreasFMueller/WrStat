Ein Gl"ucksrad ist eingeteilt in vier Sektoren, deren Zentriwinkel
sich wie $1:2:3:4$ verhalten, und die auch mit $1$, $2$, $3$ und $4$
angeschrieben sind.
\begin{teilaufgaben}
\item Wie gross ist die Wahrscheinlichkeit, in vier Spielrunden genau die
Folge $1$, $2$, $3$, $4$ zu spielen?
\item Wie gross ist die Wahrscheinlichkeit, genau die Zahlen $\{1,2,3,4\}$
in einer beliebigen Reihenfolge zu erhalten.
\end{teilaufgaben}

\begin{loesung}
Der Ausgang der verschiedenen Runden ist offenbar unabh"angig.
Wir spielen vier Runden $A$, $B$, $C$ und $D$. Das Ereignis, in
Runde $A$ ein $i$ zu spielen ist $A_i$.

\begin{teilaufgaben}
\item
In dieser Teilaufgabe suchen wir die Wahrscheinlichkeit
$P(A_1\cap B_2\cap C_3\cap D_4)$. Da alle diese Ereignisse
unabh"angig sind, d"urfen wir die Wahrscheinlichkeit der
Schnittereignisse in Produkte umformen:
\begin{align*}
P(A_1\cap B_2\cap C_3\cap D_4)
&=
P(A_1\cap B_2\cap C_3)\cdot P(D_4)
\\
&=
P(A_1\cap B_2)\cdot P(C_3)\cdot P(D_4)
\\
&=
P(A_1)\cdot P(B_2)\cdot P(C_3)\cdot P(D_4)
\end{align*}
Die Wahrscheinlichkeiten der Einzelereignisse folgen aber aus den
Fl"achenverh"altnissen der Sektoren:
\begin{align*}
P(\text{``1''})&=\frac1{10}\\
P(\text{``2''})&=\frac2{10}\\
P(\text{``3''})&=\frac3{10}\\
P(\text{``4''})&=\frac4{10}
\end{align*}
Somit ist die Wahrscheinlicheit der genannten Folge
\[
P(A_1\cap B_2\cap C_3\cap D_4)
=\frac{1\cdot2\cdot 3\cdot 4}{10^4}=\frac{4!}{10^4}=0.0024.
\]
\item Jede andere Reihenfolge der genannten Zahl hat die gleiche
Wahrscheinlichkeit. Es gibt $4!=24$ verschiedene Reihenfolgen, also
ist die Wahrscheinlichkeit $1$ bis $4$ in beliebiger Reihenfolge
zu finden
\[
P(\text{$1$ bis $4$ in beliebiger Reihenfolge})=4!P(1,2,3,4)=
\frac{576}{10000}=0.0576
\qedhere
\]
\end{teilaufgaben}
\end{loesung}

