Ein Entomologe entdeckt ein Insekt. Wegen eines speziellen Musters auf
dem Rücken des Insekts vermutet er, dass es zu einer seltenen Art
gehört.
98\% dieser seltenen Art haben das Muster.
Aber auch bei den weniger seltenen Arten kommt das Muster in 5\% der
Fälle vor.
Nur eines von 1000 Insekten gehört zu dieser seltenen Unterart.

\begin{teilaufgaben}
\item
Wie wahrscheinlich ist es, spezielle Muster zu finden?
\item
Wie wahrscheinlich ist, dass das gefundene Insekt tatsächlich zu der
seltenen Unterart gehört?
\end{teilaufgaben}

\begin{loesung}
Wir verwenden die Ereignisse
\begin{align*}
S&=\{\text{gehört zur seltenen Unterart}\}
\\
M&=\{\text{hat das besondere Muster}\}
\end{align*}
mit den Wahrscheinlichkeiten
\begin{align*}
P(S) &= 0.001
\\
P(M|S) &= 0.98
\\
P(M|\overline{S}) &= 0.05.
\end{align*}
Damit kann man die Fragen jetzt beantworten
\begin{teilaufgaben}
\item
Der Satz von der totalen Wahrscheinlichkeit liefert
\begin{align*}
P(M)
&=
P(M|S)P(S) + P(M|\overline{S})P(\overline{S})
\\
&=
P(M|S)P(S) + P(M|\overline{S})(1-P(S))
\\
&=
0.98\cdot 0.001 + 0.05\cdot 0.999
=
0.05093.
\end{align*}
\item
Mit dem Satz von Bayes kann man jetzt auch $P(S|M)$ berechnen:
\[
P(S|M)
=
\frac{P(S)}{P(M)}P(M|S)
=
\frac{0.001}{0.05093}0.98
=
0.019242.
\]
Aus der Beobachtung des Musters darf man also keinesfalls schliessen,
dass man ein Exemplar der seltene Insektenart gefunden hat.
\qedhere
\end{teilaufgaben}
\end{loesung}

\begin{bewertung}
Ereignisse ({\bf E}) 1 Punkt,
Wahrscheinlichkeiten ({\bf W}) 1 Punkt,
Satz von der Totalen Wahrscheinlichkeit ({\bf T}) 1 Punkt,
Wahrscheinlichkeit $P(M)$ ({\bf P}) 1 Punkt,
Satz von Bayes ({\bf B}) 1 Punkt,
Wahrscheinlichkeit $P(S|M)$ ({\bf Q}) 1 Punkt.
\end{bewertung}




