Masern ist eine hoch ansteckende und gef"ahrliche Krankheit,
allein 2014 starben weltweit "uber 114900 Menschen an Masern.
Im gleichen Jahr erhielten 85\% aller Kinder eine Impfdosis.
Diese f"uhrt aber nicht immer zur Immunit"at, so dass trotz Impfung
ein Erkrankungsrisiko von etwa 15\% bleibt.
In den westlichen L"andern werden daher mehrere Impfdosen verabreicht,
umd das Risiko weiter zu senken.

Mit verschiedenen Modellen wurde untersucht, wie gross die Wahrscheinlichkeit
ist, dass ein Kind sich in einer bestimmten Situation, zum Beispiel
bei einem Krankenhausbesuch oder im Kindergarten mit Masern ansteckt.
Wir nehmen an, dass in einer spezifischen Situation eine Maserninfektion
bei den Kindern die nicht immun sind, in 20\% der F"alle auftritt

\begin{teilaufgaben}
\item
Wie gross ist die Wahrscheinlichkeit, dass ein Kind erkrankt?
\item
Wie gross ist die Wahrscheinlichkeit, dass ein krankes Kind nicht
geimpft war?
\end{teilaufgaben}

\begin{loesung}
Die besondere Schwierigkeit dieser Aufgabe ist, dass ein Unterschied
zwischen Impfung und erreichter Immunit"at besteht.
Man muss also unterscheiden, ob ein Kind geimpft worden ist, wir nennen
dies das Ereignis $I$, und ob das Kind tats"achlich immun gegen Masern
geworden ist, Ereignis $J$. 
Ausserdem verwenden wir das Ereignis $M$, welches eintritt, wenn ein
Kind an Masern erkrankt.

Aus den Angaben im ersten Absatz des Textes kann man ablesen:
\begin{align*}
P(I)&=0.85
\\
P(J|I)&=0.85
\end{align*}
Es ist klar, dass man bei nicht geimpften Kindern auch keine Immunit"at
erwarten kann, also $P(J|\overline{I})=0$.
Aus diesen Informationen kann man mit dem Satz von der totalen
Wahrscheinlichkeit
die Wahrscheinlichkeit
\[
P(J)
=
P(J|I)P(I) + \underbrace{P(J|\overline{I})}_{\displaystyle=0}P(\overline{I})
= 
0.85 \cdot 0.85=0.7225
\]
berechnen, dass ein Kind immun ist.

Im zweiten Teil erf"ahrt man etwas "uber Kinder, die bereits immun sind,
also eine Wahrscheinlichkeit abh"angig vom Ereignis $J$, nicht $I$.
Es steht da, dass $P(M|\overline{J})=0.2$ sei.

\begin{teilaufgaben}
\item
Es wird gefragt nach der Wahrscheinlichkeit, dass ein Kind krank wird.
Aus dem zweiten Absatz weiss man, dass die nicht immunen Kinder mit
Wahrscheinlichkeit 20\% krank werden.
Die Wahrscheinlichkeit, dass ein Kind immun ist, ist $P(J)=0.7225$.
Also ist die Wahrscheinlichkeit, dass ein Kind an Masern erkrankt
$P(M)=0.2\cdot P(J)=0.1445$.
\item
Gesucht ist die Wahrscheinlichkeit $P(\overline{I}|M)$.
Nach dem Satz von Bayes gilt
\[
P(\overline{I}|M)
=
P(M|\overline{I})\frac{P(\overline{I})}{P(M)}.
\]
Darin wurde $P(M)=0.1445$ in a) berechnet, und $P(\overline{I})=0.15$
ist aus der Aufgabenstellung bekannt.
Da Kinder, die nicht geimpft sind, auch nicht immun sein k"onnen,
ist die Wahrscheinlichkeit, dass sie krank werden, gleich gross wie
bei den nicht immunen Kindern: $P(M|\overline{I})=P(M|\overline{J})=0.2$.
Damit sind alle Faktoren bestimmt, es folgt
\[
P(\overline{I}|M)
=
0.2\cdot\frac{0.15}{0.1445}
=
0.2076.
\qedhere
\]
\end{teilaufgaben}
\end{loesung}

\begin{bewertung}
Geeignete Ereignisse $I$ und $M$ ({\bf E}) 1 Punkt,
Wahrscheinlichkeiten der bedingten Wahrscheinlichkeiten $P(M|\overline{I})$
und $P(M|I)$ ({\bf C}) 1 Punkt,
Bayessche Formel ({\bf B}) 1 Punkt,
Totale Wahrscheinlichkeit ({\bf T}) 1 Punkt,
L"osung zu Teilaufgabe a) ({\bf A}) 1 Punkt,
L"osung zu Teilaufgabe b) ({\bf B}) 1 Punkt.
\end{bewertung}

