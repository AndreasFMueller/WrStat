Masern ist eine hoch ansteckende und gef"ahrliche Krankheit,
allein 2014 starben weltweit "uber 114900 Menschen an Masern.
Im gleichen Jahr erhielten 85\% aller Kinder eine Impfdosis.
Diese f"uhrt aber nicht immer zur Immunit"at, so dass trotz Impfung
ein Erkrankungsrisiko von etwa 15\% bleibt.
In den westlichen L"andern werden daher mehrere Impfdosen verabreicht,
umd das Risiko weiter zu senken.

Mit verschiedenen Modellen wurde untersucht, wie gross die Wahrscheinlichkeit
ist, dass ein Kind sich in einer bestimmten Situation, zum Beispiel
bei einem Krankenhausbesuch oder im Kindergarten mit Masern ansteckt.
Wir nehmen an, dass in einer spezifischen Situation ein Maserninfektion
bei den Kindern die immun sind, in 20\% der F"alle auftritt

\begin{teilaufgaben}
\item
Wie gross ist die Wahrscheinlichkeit, dass ein Kind erkrankt?
\item
Wie gross ist die Wahrscheinlichkeit, dass ein krankes Kind nicht
geimpft war?
\end{teilaufgaben}

\begin{loesung}
Wir arbeiten mit folgenden Ereignissen:
\begin{align*}
I&=\{\text{Kind geimpft}\}\\
M&=\{\text{Kind erkrankt an Masern}
\end{align*}
Aus den Angaben in der Aufgabe wissen wir:
\begin{align*}
P(I)&=0.85\\
P(M|\overline I)&=0.2\\
P(M|I)&=0.15 \cdot P(M|\overline I)= 0.03
\end{align*}
\begin{teilaufgaben}
\item
Gesucht ist $P(M)$, diese Wahrscheinlichkeit kann mit dem Satz von der
totalen Wahrscheinlichkeit gefunden werden:
\begin{align*}
P(M) 
&=
P(M|I)P(I) + P(M|\overline I)P(\overline I)
=
P(M|I)P(I) + P(M|\overline I)(1-P(I))
\\
&=
0.03\cdot 0.85 + 0.2\cdot(1-0.85)=0.0555.
\end{align*}
\item
Dies ist die Wahrscheinlichkeit  ist nach dem Satz von Bayes
\begin{align*}
P(\overline I|M)
&=
\frac{P(\overline I)}{P(M)}P(M|\overline I)
=
\frac{1-P(I)}{P(M)}P(M|\overline I)
=
\frac{0.15}{0.0555}\cdot 0.2
=
0.54054.
\qedhere
\end{align*}
\end{teilaufgaben}
\end{loesung}


