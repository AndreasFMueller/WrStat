In einem Bohrloch soll festgestellt werden, in welcher Art von Gestein
sich der Bohrkopf gerade befindet. Dazu wird ein Messger"at verwendet, welches
die nat"urliche Radioaktivit"at der Gesteine misst, insbesondere die
Gammastrahlung.
In Vergleichsmessungen wurde festgestellt, dass sich Schiefer
und Dolomit dadurch unterscheiden, dass in Schiefer das Messger"at
in 95\% der F"alle eine Strahlungsintensit"at von gr"osser als 60 (in den
nicht weiter interessierenden Einheiten des Messger"ates) anzeigt,
in Dolomit aber nur in 7\% der F"alle.
Ausserdem ist bekannt, dass in den zu untersuchenden Bohrl"ochern
Schiefer mit einer H"aufigkeit von 40\% vorkommt, Dolomit aber mit 60\%.
\begin{teilaufgaben}
\item
Wie gross ist die Wahrscheinlichkeit, dass das Messger"at mehr als 60
Einheiten anzeigt?
\item
Mit welcher Wahrscheinlichkeit liegt tats"achlich Schiefer vor, wenn 
das Messger"at mehr als 60 anzeigt?
\item
Wie gross ist die Wahrscheinlichkeit, dass Dolomit vorliegt, wenn
das Messger"at weniger also 60 anzeigt?
\end{teilaufgaben}


\begin{loesung}
Wir verwenden die Ereignisse
\begin{align*}
A&=\{\text{Messger"at zeigt mehr als 60 an}\}\\
S&=\{\text{es liegt Schiefer vor}\}\\
D&=\{\text{es liegt Dolomit vor}\}
\end{align*}
Aus der Aufgabenstellung sind folgende Wahrscheinlichkeiten bekannt:
\begin{align*}
P(A|S)&=0.95&
P(S)&=0.4\\
P(A|D)&=0.07&
P(D)&=0.6
\end{align*}
\begin{teilaufgaben}
\item Mit dem Satz von der totalen Wahrscheinlichkeit findet man
\[
P(A)=P(A|S)P(S)+P(A|D)P(D)=0.95\cdot 0.4+0.07\cdot 0.6=0.422.
\]
\item Gesucht ist offenbar die bedingte Wahrscheinlichkeit $P(S|A)$. Der
Satz von Bayes liefert
\begin{align*}
P(S|A)&=P(A|S)\frac{P(S)}{P(A)}=0.95\cdot\frac{0.4}{0.422}=0.900.
\end{align*}
\item
Gesucht ist in diesem Fall die bedingte Wahrscheinlichkeit $P(D|\bar A)$:
\[
P(D|\bar A)=P(\bar A|D)\frac{P(D)}{P(\bar A)}
=(1-P(A|D))\frac{P(D)}{1-P(A)}=0.93 \cdot \frac{0.6}{0.578}=0.965.
\qedhere
\]
\end{teilaufgaben}
\end{loesung}


\begin{bewertung}
Ereignisse ({\bf E}) 1 Punkt,
"Ubersetzung der bedingten Wahrscheinlichkeiten ({\bf W}) 1 Punkt,
Satz von Bayes ({\bf B}) 1 Punkt,
Satz von der totalen Wahrscheinlichkeit, Teilaufgabe a) ({\bf T}) 1 Punkt,
Wahrscheinlichkeit f"ur Schiefer $P(S|A)$ b) ({\bf S}) 1 Punkt,
Resultat $P(\bar S|\bar A)$ c) ({\bf C}) 1 Punkt.
\end{bewertung}

