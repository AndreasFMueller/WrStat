$A$ und $B$ sind beide fleissige Leser, aber auch ziemlich
kritisch: nur 10\% der Bücher, die $A$ liest, gefallen ihm.
In der Diskussion merkt
$B$, dass er 80\% der Bücher, die $A$ gut findet, auch gut findet,
und 90\% der Bücher, die $A$ nicht gefallen, ihm auch nicht
gefallen.
\begin{teilaufgaben}
\item Welcher Prozentsatz der Bücher gefällt $B$ überhaupt?
\item Wie häufig sind Bücher, die sowohl $A$ als auch $B$ gut finden?
\item Wie wahrscheinlich ist, dass $A$ ein Buch gut findet, welches
$B$ schon für gut befunden hat? Könnte es sich für $A$ zum Beispiel
lohnen, nur noch Bücher zu lesen, die $B$ empfehlen kann?
\end{teilaufgaben}

\begin{loesung}
Die Bücher sind offenbar die Elementarereignisse $\omega$.
Sei $A$ das Ereignis, dass $A$ ein Buch $\omega$ gut findet,
und $B$ entsprechend das Ereignis, dass $B$ ein Buch gut findet.
Wir haben folgende Wahrscheinlichkeiten:
\begin{align*}
P(A)&=0.1\\
P(B|A)&=0.8\\
P(\bar B|\bar A)&=0.9
\end{align*}
\begin{teilaufgaben}
\item
Nach dem Satz über die totale Wahrscheinlichkeit ist die Wahrscheinlichkeit
$P(B)$, dass $B$ ein Buch gut findet:
\begin{align*}
P(B)
&=
P(B|A)P(A) + P(B|\bar A)P(\bar A)
\\
&=
P(B|A)P(A) + P(B|\bar A)(1-P(A))
\\
&=
P(B|A)P(A) + (1-P(\bar B|\bar A))(1-P(A))
\\
&=
0.8\cdot 0.1+0.1\cdot 0.9=0.08 + 0.09=0.17
\end{align*}
\item
Nach der Definition der bedingten Wahrscheinlichkeit
ist 
\[
P(B|A)=\frac{P(A\cap B)}{P(A)}
\quad\Rightarrow\quad
P(A\cap B)=P(B|A)P(A)=0.8\cdot 0.1=0.08.
\]
\item Gesucht ist $P(A|B)$. Nach dem Satz von Bayes ist
\[
P(A|B)=P(B|A)\frac{P(A)}{P(B)}=0.8\cdot\frac{0.1}{0.17}=0.47059
\]
Wenn $A$ sich an die Empfehlungen von $B$ hielte, würde er wesentlich
häufiger mit einem Buch zufrieden sein.
\qedhere
\end{teilaufgaben}
\end{loesung}

