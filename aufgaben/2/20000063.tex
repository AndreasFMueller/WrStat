Eine Internatsschule unterrichtet auch externe Schüler.
60\% der Schüler wohnen im Internat, der Rest lebt extern.
Am Ende des Jahres wird die Leistung der Internatsschüler mit
den Externen verglichen.
30\% der Internatsschüler erreichen die Höchstnote, aber nur
20\% der Externen.
\begin{teilaufgaben}
\item
Wie wahrscheinlich ist es, dass ein Schüler die Höchstnote erreicht?
\item
Wie gross ist die Wahrscheinlichkeit, dass ein Schüler, der die
Höchstnote erreicht, ein externer Schüler ist?
\end{teilaufgaben}

\begin{loesung}
Wir verwenden die Ereignisse:
\begin{align*}
I&=\{\text{Internatsschüler}\}
&
H&=\{\text{Höchstnote}\}
\\
E&=\{\text{externer Schüler}\}=\overline{I}.
\end{align*}
Aus dem Text lesen wir die folgenden Erreignisse
\begin{align*}
P(H|I)            &= 0.3 \\
P(H|\overline{I}) &= 0.2 \\
P(I)              &= 0.6
\end{align*}
\begin{teilaufgaben}
\item
Es muss $P(H)$ bestimmt werden.
Dazu kann der Satz von der totalen Wahrscheinlichkeit verwendet werden:
\begin{align*}
P(H)
&=
P(H|I) P(I) + P(H|\overline{I}) P(\overline{I})
=
P(H|I) P(I) + P(H|\overline{I}) (1-P(I))
=
0.3\cdot 0.6 + 0.2\cdot 0.4
=
0.26.
\end{align*}
\item
Es muss die Wahrscheinlichkeit $P(\overline{I}|H)$ bestimmt werden.
Es gilt
\begin{align*}
P(\overline{I}|H)
&=
1-P(I|H)
\\
P(I|H)
&=
P(H|I)
\frac{P(I)}{P(H)}
=
0.3\cdot\frac{0.6}{0.26}
=
0.6923.
\intertext{Aus den beiden Gleichungen folgt das Resultat}
P(\overline{I}|H)
&= 1-0.6923=0.3077.
\qedhere
\end{align*}
\end{teilaufgaben}
\end{loesung}

\begin{bewertung}
Ereignisse ({\bf E}) 1 Punkt,
bedinge Wahrscheinlichkeiten ({\bf C}) 1 Punkt,
Satz von der totalen Wahrscheinlichkeit ({\bf T}) 1 Punkt,
Satz von Bayes ({\bf B}) 1 Punkt,
Wahrscheinlichkeit $P(H)$ ({\bf H}) 1 Punkt,
Wahrscheinlichkeit $P(\overline{I}|H)$ ({\bf I}) 1 Punkt.
\end{bewertung}

