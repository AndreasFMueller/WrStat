Eine neue Werbekampagne f"ur ein schon l"anger auf dem Markt
befindliches Produkt wird auf ihre Wirksamkeit gepr"uft.
Dazu werden Leute befragt, ob sie den neuen Werbespot gesehen
haben und ob sie das Produkt kennen.
60\% der Leute, die den neuen Werbespot gesehen haben,
k"onnen sich an das Produkt erinnern. Aber auch 40\% der
Leute, die den Werbespot nicht gesehen haben, k"onnen sich
an das Produkt erinnern. 20\% der Leute haben den Spot gesehen.
\begin{teilaufgaben}
\item Wie gross ist die Wahrscheinlichkeit, dass sich jemand
an das Produkt erinnert?
\item Eine befragte Person kennt das Produkt. Wie gross ist
die Wahrscheinlichkeit, dass sie den Werbespot gesehen hat?
\item  Eine befragte Person kennt das Produkt nicht. Wie gross
ist die Wahrscheinlichkeit, dass Sie den Fernsehspot nicht
gesehen hat?
\end{teilaufgaben}

\begin{loesung}
Offenbar haben wir es mit folgenden Ereignissen zu tun:
\begin{align*}
W&=\{\text{hat Werbespot gesehen}\}
\\
E&=\{\text{kann sich an Produkt erinnern}\}
\end{align*}
Die Angaben in der Aufgabe bedeuten
\begin{align*}
P(W)&=0.2\\
P(E|W)&=0.6\\
P(E|\overline W)&=0.4
\end{align*}
\begin{teilaufgaben}
\item Es ist $P(E)$ zu berechnen. Nach dem Satz "uber die totale
Wahrscheinlichkeit ist
\begin{align*}
P(E)&=P(E|W)P(W)+P(E|\overline W)P(\overline W)\\
&=P(E|W)P(W)+P(E|\overline W)(1-P(W))\\
&=0.6\cdot 0.2+0.4\cdot (1-0.2)\\
&=0.12+0.32=0.44
\end{align*}
\item Gesucht ist $P(W|E)$. Nach dem Satz von Bayes ist
\begin{align*}
P(W|E)&=P(E|W)\frac{P(W)}{P(E)}\\
&=0.6\cdot\frac{0.2}{0.44}=\frac{3}{11}
%0.6 * 0.2/0.44
= 0.27272727272727272727
\end{align*}
\item Gesucht ist $P(\overline W|\overline E)$. Nat"urlich ist
$P(\overline W|\overline E)=1-P(W|\overline E)$, es gen"ugt also,
$P(W|\overline E)$ zu berechnen. Wiederum nach dem Satz von Bayes
gilt
\begin{align*}
P(W|\overline E)
&=
P(\overline E|W)\frac{P(W)}{P(\overline E)}\\
&=
(1-P(E|W))\frac{P(W)}{1-P(E)}\\
&=
0.4\cdot\frac{0.2}{1-0.44}=\frac17
=
%0.4*(0.2/(1-0.44))
0.14285714285714285714
\end{align*}
Folglich ist die gesucht Wahrscheinlichkeit
\begin{align*}
P(\overline W|\overline E)=1-\frac17=\frac67=
%6/7
0.85714285714285714285
\end{align*}
\end{teilaufgaben}
\end{loesung}

