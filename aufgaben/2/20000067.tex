Aus einem Kartenspiel mit 52 Karten wird eine Karte gezogen.
\begin{teilaufgaben}
\item Was ist $\Omega$?
\item Wieviele Elementarereignisse gibt es?
\item Wieviele Elementarereignisse hat das Ereignis
$H=\{\text{die Karte zeigt Herz}\}$
\item Wieviele Elementarereignisse hat das Ereignis
$K=\{\text{die Karte zeigt einen König}\}$
\end{teilaufgaben}

\begin{loesung}
\begin{teilaufgaben}
\item
Die Versuchsausgänge sind die Karten, die durch 4 Farben (Kreuz, Schaufel,
Herz, Ecke) und 13 verschiedene Werte (A, 2, \dots, 9, J, Q, K) charakterisiert
sind.
\item
Es gibt 52 Karten also auch 52 Versuchsausgänge: $|\Omega|=52$.
\item
Es gibt $|H|=13$ Herzkarten.
\item
Es gibt $|K|=4$ Königskarten.
\qedhere
\end{teilaufgaben}
\end{loesung}

