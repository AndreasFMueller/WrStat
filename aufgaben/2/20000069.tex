Aus einem Kartenspiel mit 52 Karten werden 5 Karten gezogen, dabei
können die folgenden Ereignisse eintreten.
\begin{teilaufgaben}
\item
$P=\{\text{enthält zwei Karten mit gleichem Wert}\}$, d.~h.~ein Paar
\item
$Z=\{\text{enthält zwei verschiedene Paare}\}$
\item
$T=\{\text{enthält drei Karten mit gleichem Wert}\}$, d.~h.~ein Tripel
\item
$V=\{\text{enthält vier Karten mit gleichem Wert}\}$, d.~h.~Poker
\item
$F=\{\text{full house}\}$, d.~h.~ein Paar und ein Tripel, jeweils mit
verschiedenem Wert
\end{teilaufgaben}
Zeichnen Sie die Ereignisse in einem Venn-Diagramm und geben Sie
ein Beispiel für ein Elementarereignis in
\[
X
=
(P\cap T)\setminus(F\cup V)
.
\]

\begin{loesung}
Die Ereignisse können wie folgt gezeichnet werden:
\begin{center}
\begin{tikzpicture}[>=latex,thick]
\coordinate (ur) at (9,7);
\def\pathall{ (0,0) rectangle (ur) }
\draw (0,0) rectangle (ur);
\node at (ur) [above left] {$\Omega$};

\def\pathp{ (1,1) rectangle (8,6) }
%\def\patht{ (4,2) rectangle (11,7) }
\def\patht{ (4,2) rectangle (8,6) }
\def\pathz{ (2,2.5) rectangle (6,4.5) }
\def\pathf{
	(4,2.5) -- (5.6,2.5) arc(-90:0:0.4) -- (6,4.1) arc(0:90:0.4) -- (4,4.5) -- cycle
}
\def\pathv{ (6,4.5) rectangle (8,6) }

\begin{scope}
	\clip[rounded corners=0.4cm] \pathp;
	\fill[color=red!20,opacity=0.5] \pathall;
\end{scope}

\begin{scope}
	\clip[rounded corners=0.4cm] \patht;
	\fill[color=blue!20,opacity=0.5] \pathall;
\end{scope}

\begin{scope}
	\clip[rounded corners=0.4cm] \pathz;
	\fill[color=orange!20,opacity=0.5] \pathall;
\end{scope}

\begin{scope}
	\clip \pathf;
	\fill[color=darkgreen!20,opacity=0.5] \pathall;
\end{scope}

\begin{scope}
	\clip[rounded corners=0.4cm] \pathv;
	\fill[color=brown!20,opacity=0.5] \pathall;
\end{scope}

\draw[color=darkred,rounded corners=0.4cm] \pathp;
\draw[color=blue,rounded corners=0.4cm] \patht;
\draw[color=orange,rounded corners=0.4cm] \pathz;
\draw[color=darkgreen] \pathf;
\draw[color=brown,rounded corners=0.4cm] \pathv;

\node[color=darkred] at (1,6) [below right,distance=0.7cm] {$P$};
\node[color=orange] at (2,4.5) [below right,distance=0.7cm] {$Z$};
\node[color=brown] at (7,5.25) {$V$};
\node[color=darkgreen] at (5.15,3.5) {$F$};
\node[color=blue] at (8,2) [above left,distance=0.7cm] {$T$};

\coordinate (omega) at (7,3);
\fill (omega) circle[radius=0.08];
\node at (omega) [above] {$\omega$};

\node at (4.9,5.25) {$X$};

\end{tikzpicture}
\end{center}
Ein Elementarereignis $\omega$ in $X$ muss ein Tripel enthalten, darf aber 
weder eine weiter Karte wie im Tripel noch ein vom Tripel
verschiedenes Paar enthalten.
Es ist daher von der Form
\[
\omega
=
\{
a_1, a_2, a_3, b, c
\},
\]
also drei gleiche Karten $a_i$ mit dem gleichen Wert und zwei 
Karten $b$ und $c$ mit von $a_i$ und untereinander verschiedenem Wert.
\end{loesung}

