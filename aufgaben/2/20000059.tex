In einem Computerladen werden Computer von drei Herstellern angeboten:
HAL, Ananas und \reflectbox{ASUS}.
Zwei Drittel der ausgestellten Computer stammen von \reflectbox{ASUS},
die anderen beiden Hersteller teilen sich den Rest zu gleichen Teilen.
Die Wahrscheinlichkeit, dass an einem Tag ein Computer von HAL verkauft
wird, ist trotz der sogar in Dokumentarfilmen\footnote{Gut bekannt ist
der Film von Stanley Kubrik, IMDb-Eintrag
\url{https://www.imdb.com/title/tt0062622/}}
festgehaltenen
Benutzerfeindlichkeit 5\%, bei \reflectbox{ASUS} sind es 2\%.
Die Computer von Ananas mit dem berühmten Logo mit einer angebissenen
Ananas sind sehr beliebt und werden mit einer Wahrscheinlichkeit von 8\%
verkauft.

\begin{teilaufgaben}
\item
Wie gross ist die Wahrscheinlichkeit, dass ein Computer verkauft wird?
\item
Wie gross ist die Wahrscheinlichkeit, dass ein verkaufter Computer ein
Ananas-Computer ist?
\end{teilaufgaben}

\begin{loesung}
Wir verewnden die Ereignisse
\begin{align*}
H&=\{\text{Computer von HAL}\}
\\
A&=\{\text{Computer von Ananas}\}
\\
S&=\{\text{Computer von \reflectbox{ASUS}}\}
\\
V&=\{\text{verkauft}\}
\end{align*}
Die folgenden Wahrscheinlichkeiten sind bekannt:
\begin{align*}
P(V|S)&= 0.02 & P(S) &= \frac23 \\
P(V|H)&= 0.05 & P(H) &= \frac16 \\
P(V|A)&= 0.08 & P(A) &= \frac16 
\end{align*}
\begin{teilaufgaben}
\item
Gesucht ist die Wahrscheinlichkeit $P(V)$, die mit dem Satz von
der totalen Wahrscheinlichkeit ermittel werden kann:
\begin{align*}
P(V)
&=
P(V|S)P(S)
+
P(V|H)P(H)
+
P(V|A)P(A)
\\
&=
0.02\cdot \frac23
+
0.08\cdot \frac16
+
0.05\cdot \frac16
\\
&=
0.035.
\end{align*}
\item
Gesucht ist die bedingte Wahrscheinlichkeit $P(A|V)$, die mit
dem Satz von Bayes verkauft werden kann:
\begin{align*}
P(A|V)
&=
P(V|A)\frac{P(A)}{P(V)}
=
0.08 \frac{1/6}{0.035}
=
\frac{0.08}{0.21}
\approx
38.1\%.
\qedhere
\end{align*}
\end{teilaufgaben}
\end{loesung}

\begin{bewertung}
Ereignisse ({\bf E}) 1 Punkt,
bedingte Wahrscheinlichkeiten ({\bf C}) 1 Punkt,
Satz von der totalen Wahrscheinlichkeit ({\bf T}) 1 Punkt,
Wahrscheinlichkeit in a) ({\bf A}) 1 Punkt,
Satz von Bayes ({\bf B}) 1 Punkt,
Wahrscheinlichkeit in b) ({\bf W}) 1 Punkt.
\end{bewertung}
