Beschreiben Sie $\Omega$ und m"ogliche Ereignisse in folgenden Situationen
\begin{teilaufgaben}
\item
Die Enten im Hafen vor der HSR verhalten sich monogam, d.~h.~sie bleiben einem
einmal gefundenen Partner treu. Allerdings kann es passieren, dass ein
Erpel keine Partnerin findet. Erpel mit Partnerin haben meist mehr
als einen Nachkommen.
\item
Eine Stricknadel f"allt auf einen schwarz-weiss gestreiften Boden, die
Nadel ist gerade so lange wie die Breite der Streifen.
\item
Im Strandbad mit vielen B"aumen breiten die Badeg"aste bei der Ankunft
ihre Badet"ucher auf der Liegewiese aus.
Im Laufe des Tages bewegt sich die Sonne, so dass
einzelne Badeg"aste pl"otzlich einen Schattenplatz haben.
\item
Stellt man am Fernseher einen Kanal ein, kann es passieren, dass
man Nachrichten sieht, irgend einen Film, oder Werbung.
\end{teilaufgaben}

\begin{loesung}
\begin{teilaufgaben}
\item Das Experiment besteht darin, den ``Beziehungsstatus'' eines
einzelnen Erpels zu untersuchen. Dabei ergeben sich folgende m"oglichen
Versuchsausg"ange:
\begin{align*}
\Omega=\{&\text{``single''},\\
&\text{keine direkte Nachkommen},\\
&\text{1 direkter Nachkomme},\\
&\text{2 direkte Nachkomme},\\
&\dots \}
\end{align*}
Darin kann man zum Beispiel folgende nichtelementare Ereignisse finden:
\begin{align*}
K&=\{\text{``kinderlos'', d.~h.~single oder ohne direkte Nachkommen}\}\\
F&=\{\text{mehr als ein direkter Nachkomme}\}
\end{align*}
\item
Das Experiment besteht darin, die Nadel zu Boden zu werfen. Sie nimmt dann
eine beliebige Lage in, die man zum Beispiel durch die Koordinaten
des Schwerpunktes der Nadel und den Winkel zwischen Nadel und Streifenrichtung
festlegen k"onnte. Die Menge $\Omega$ ist daher
\[
\Omega=\{ (x,y,\alpha) \,|\, \text{$(x,y)$ Schwerpunkt der Nadel, $\alpha$ Richtung}\}.
\]
Die Nadel kann ganz innerhalb eines Streifens zu liegen kommen, oder sie
kann die Grenzlinie zwischen zwei Streifen schneiden. Sehr unwahrscheinlich,
aber grunds"atzlich m"oglich ist auch, dass die Nadel genau senkrecht
zu den Grenzlinien zu liegen kommt, und ihre Enden genau auf zwei Grenzlinien
liegen. Man hat also folgende Ereignisse
\begin{align*}
S&=\{ \text{Nadel ganz in einem schwarzen Streifen}\}\\
W&=\{ \text{Nadel ganz in einem weissen Streifen}\}\\
G&=\{ \text{Nadel schneidet Grenzlinie zwischen Streifen}\}\\
V&=\{ \text{Nadel verbindet zwei Grenzlinien}\}
\end{align*}
\item
Das Experiment besteht darin, einen Punkt auf der Liegewiese auszuw"ahlen.
Die Menge $\Omega$ ist also die Menge aller Punkte der Liegewiese.
Im Laufe des Tages werden einige Punkte der Liegewiese beschattet.
Das Ereignis $S=\{\text{Schattenplatz}\}\subset\Omega$
tritt also genau f"ur jene Punkte
der Liegewiese ein, die im Laufe des Tages irgend wann einmal beschattet 
werden.
\item 
Das Experiment besteht darin, zu einem bestimmten Zeitpunkt
einen Kanal am Fernseher auszuw"ahlen.
Die Menge $\Omega$ besteht also aus Paaren von Zeitpunkten und
empfangbaren Kan"alen.
Darin kann man folgende Ereignisse identifizieren
\begin{align*}
W&=\{\text{Sender strahlt gerade Werbung aus}\}\\
N&=\{\text{Sender strahlt gerade Nachrichten aus}\}\\
D&=\{\text{Sender strahlt gerade eine Dokumentation aus}\}\\
S&=\{\text{Sender strahlt gerade einen Spielfilm aus}\}\\
T&=\{\text{Sender ist ausgefallen, nur ein Testbild ist sichtbar}\}
\end{align*}
\end{teilaufgaben}
\end{loesung}


