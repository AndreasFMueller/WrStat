Ein Paläontolge weiss, dass er Fossilien einer bestimmte Dinosaurierart $B$
mit Wahrscheinlichkeit $0.8$ in den Schichten der Zeitspanne $A$, 
aber nur mit Wahrscheinlichkeit $0.6$ in Schichten aller anderen in
Frage kommenden Zeiträume finden wird.
Die Analyse der Gesteine an einer Grabungsstelle ergab, dass die Gesteine
mit Wahrscheinlichkeit $0.2$ zur Zeitspanne $A$ gehören.
\begin{teilaufgaben}
\item
Wie gross ist die Wahrscheinlichkeit, an der Grabungsstelle ein Fossil
der Art $B$ zu finden?
\item
Wie gross ist die Wahrscheinlichkeit, dass die Gesteine tatsächlich zur
Zeitspanne $A$ gehören, wenn ein Fossil der genannten Art gefunden wird?
\end{teilaufgaben}

\thema{bedingte Wahrscheinlichkeit}
\thema{Satz von Bayes}
\thema{totale Wahrscheinlichkeit}

\begin{loesung}
Wir betrachten die Ereignisse
\begin{align*}
F&=\{\text{Fossil der Art $B$ gefunden}\},
\\
A&=\{\text{Gesteine gehören zum Zeitraum $A$}\}.
\end{align*}
Gegeben sind die folgenden Wahrscheinlichkeiten:
\begin{align*}
P(A)&=0.2,
\\
P(F|A)&=0.8,
\\
P(F|\bar A)&=0.6.
\end{align*}
\begin{teilaufgaben}
\item
Gesucht ist die Wahrscheinlichkeit $P(F)$, dies ist mit dem Satz von der
totalen Wahrscheinlichkeit möglich:
\begin{align*}
P(F)
&=
P(F|A)\,P(A) + P(F|\bar A)\,P(\bar A)
=
P(F|A)\,P(A) + P(F|\bar A)(1-P(A))
\\
&=
0.8\cdot 0.2 + 0.6\cdot (1-0.2)
=
0.8\cdot 0.2 + 0.6\cdot 0.8
=0.64.
\end{align*}
\item
Gesucht ist die Wahrscheinlichkeit $P(A|F)$, die mit dem Satz von Bayes 
gefunden werden kann:
\begin{align*}
P(A|F)
&=
\frac{P(A)}{P(F)}P(F|A)
=
\frac{0.2}{0.64}\cdot 0.8
=
0.25.
\qedhere
\end{align*}
\end{teilaufgaben}
\end{loesung}

\begin{bewertung}
Ereignisse ({\bf E}) 1 Punkt,
bedingte Wahrscheinlichkeiten ({\bf C}) 1 Punkt,
Satz von Bayes ({\bf B}) 1 Punkt,
Satz von der totalen Wahrscheinlichkeit ({\bf T}) 1 Punkt,
Wahrscheinlichkeit $P(F)$ ({\bf F}) 1 Punkt,
Wahrscheinlichkeit $P(A|F)$ ({\bf A}) 1 Punkt.
\end{bewertung}


