In einem Restaurant wird ein Frühstücksmüsli angeboten.
Zum Standard-Müsli gibt es noch zwei optionale Zutaten, die einzeln
wählbar sind, nämlich Cornflakes und Datteln.
Aus Erfahrung weiss mann, dass $4/9$ der Gäste die
Datteln wählen.
25\% der Gäste, die Datteln gewählt haben, wählen auch Cornflakes.
Von den Gästen, die keine Dateln wählen, wählen $3/5$ auch keine Cornflakes.

\begin{teilaufgaben}
\item Wie gross ist die Wahrschleinlichkeit, dass ein Gast Cornflakes wählt?
\item Wie gross ist die Wahrscheinlichkeit, dass jemand, der Cornflakes
gewählt hat, auch Datteln wählt?
\end{teilaufgaben}

\begin{loesung}
Wir haben zwei Ereignisse zu modellieren:
\begin{align*}
C
&=
\{\text{Gast wählt Cornflakes}\}
\\
D
&=
\{\text{Gast wählt Datteln}\}.
\end{align*}
Folgende Wahrscheinlichkeiten sind bekannt:
\begin{align*}
P(D)
&=
\frac49
\\
P(C|D)
&=
\frac14
\\
P(\bar C|\bar D)
&=
\frac35.
\end{align*}
\begin{teilaufgaben}
\item Wir müssen $P(C)$ bestimmen.
Dies ist möglich mit dem Satz von der totalen Wahrscheinlichkeit
\begin{align*}
P(C)
&=
P(C|D)P(D) + P(C|\bar D)P(\bar D)
\\
&=
\frac14\cdot \frac49 + (1-P(\bar C|\bar D))(1-P(D))
\\
&=
\frac14\cdot \frac49 + \frac25 \cdot\frac59
\\
&=
\frac19 +\frac29=\frac13.
\end{align*}
\item
Es ist die Wahrscheinlichkeit $P(D|C)$ gefragt, die mit dem Satz von Bayes
bestimmt werden kann.
Man findet
\begin{align*}
P(D|C)
&=
P(C|D)\frac{P(D)}{P(C)}
=
\frac14\cdot\frac{4/9}{1/3}=\frac13.
\qedhere
\end{align*}
\end{teilaufgaben}
\end{loesung}

\begin{diskussion}
Diese Aufgabe stammt aus der BM-Prüfung der BMS Pfäffikon SZ im Sommer 2017.
\end{diskussion}

