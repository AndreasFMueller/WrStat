Beim Spiel {\em Glückswurf}\/%
\footnote{\url{https://www.spielregeln-spielanleitungen.de/wuerfelspiel/glueckswurf/}}
würfelt der Spieler mit fünf Würfeln.
Die Augenzahlen werden addiert, aber erst nach folgenden Modifikationen:
\begin{enumerate}
\item Jede \epsdice{1} wird auf \epsdice{6} gedreht
\item Die \epsdice{5} gilt als Jocker: Falls eine \epsdice{5} vorkommt,
werden alle Würfel auf \epsdice{6} gedreht.
\end{enumerate}
\begin{teilaufgaben}
\item
Wie gross ist die Wahrscheinlichkeit $P(J)$ für einen Joker?
\item
Mit welcher Augenzahl kann man rechnen, wenn man nur die Spiele
ohne Joker zählt?
\item
Welche Augenzahl kann man erwarten, wenn man sowohl die Spiele mit
als auch ohne Joker zählt?
\end{teilaufgaben}

\begin{hinweis}
Berechnen Sie zuerst die Erwartung für die einzelnen Fälle $J$ und
$\overline{J}$ und setzen Sie diese dann zusammen.
\end{hinweis}

\begin{loesung}
\begin{teilaufgaben}
\item
Wir bezeichnen das Ereignis, dass ein Joker aufgetreten ist mit $J$.
Die Wahrscheinlichkeit, dass einer der Würfel keine \epsdice{5} zeigt,
ist $\frac{5}{6}$. 
Die Wahrscheinlichkeit, dass alle 5 Würfel keine \epsdice{5} zeigen, ist
$(\frac{5}{6})^5$.
Die Wahrscheinlichkeit, dass mindesstens einer der Würfel eine \epsdice{5}
zeigt, ist
\[
P(J)
=
1-\biggl(\frac{5}{6}\biggr)^5\approx 0.598122.
\]
\item
Sei $X$ die Punktzahl eines Würfels unter Ausschluss der Joker.
In einem Spiel ohne Joker hat ein Würfel die möglichen Resultate
\epsdice{2}--\epsdice{4} und \epsdice{6}.
Die Wahrscheinlichkeit von \epsdice{2}--\epsdice{4} ist $\frac{1}{5}$,
da die \epsdice{5} ausgeschlossen worden ist.
Die Wahrscheinlichkeit für \epsdice{6} ist doppelt so gross, nämlich
$\frac{2}{5}$.
Der Erwartungswert eines einzelnen Würfels ist daher
\[
E(X)
=
\frac15\cdot 2
+
\frac15\cdot 3
+
\frac15\cdot 4
+
\frac25\cdot 6
=
\frac{2+3+4+2\cdot 6}5
=
\frac{21}{5} = 4.2.
\]
Nun hat man aber fünf unabhängige solche Würfel $X_1,\dots,X_5$,
die die gleiche Verteilung haben wir $X$.
Der Erwartungswert der Summe ist
\[
E(X_1+\dots+X_5)
=
E(X_1)+\dots+E(X_5)
=
5\cdot \frac{21}{5} = 21.
\]
\item
Bei einem Joker ist die erreiche Punktzahl $5\cdot\epsdice{6}=30$,
die Wahrscheinlichkeit dafür ist in Teilaufgabe a)
als $1-(\frac{5}{6})^5$ berechnet worden.
Die erwartete Punktzahl ist daher
\begin{align*}
E(X)
&=
P(J)
\cdot 30
+
P(\overline{J})
\cdot E(X|\overline{J})
=
\biggl(
1-
\biggl( \frac{5}{6}\biggr)^5
\biggr)
\cdot 30
+
\biggl( \frac{5}{6}\biggr)^5
\cdot 21
\\
&=
\frac{4651}{7776}\cdot 30 + \frac{3125}{7776}\cdot 21
=
\frac{4651\cdot 30 + 3125\cdot 21}{7776}
=
\frac{205155}{7776}
=
26.3831.
\qedhere
\end{align*}
\end{teilaufgaben}
\end{loesung}

\begin{bewertung}
\begin{teilaufgaben}
\item
Wahrscheinlichkeit für den Joker von einem Würfel ({\bf J}) 1 Punkt,
Wahrscheinlichkeit für den Joker von allen Würfeln ({\bf A}) 1 Punkt.
\item
Erwartungswert von einem Würfel {(\bf E)} 1 Punkt,
Erwartungswert von allen Würfeln {(\bf B)} 1 Punkt.
\item
Fallunterscheidung mit und ohne Joker ({\bf F}) 1 Punkt,
kombinierter Erwartungswert ({\bf G}) 1 Punkt.
\end{teilaufgaben}
\end{bewertung}

