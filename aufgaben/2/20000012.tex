Ein oft von Evolutionszweiflern vorgebrachtes Argument besagt, dass
die Abfolge von Ereignissen, die zu unserer heutigen Artenvielfalt
gef"uhrt hat, so unwahrscheinlich ist, dass es vermessen w"are zu
glaube, dass sie zuf"allig stattgefunden haben k"onnte. Dieses Argument
ist nat"urlich falsch. Es nimmt stillschweigend an, dass es nur eine
einzige m"ogliche Biologie gibt. In Wahrheit w"urde eine Wiederholung des
Experimentes wohl v"ollig andere Lebensformen hervorbringen. Nur schon
die heute Vormachtstellung der S"augetiere ist mit dem Unstand zu
verdanken, dass ein Asteroid im richtigen Moment die Dinosaurier
ausgerottet hat.

Zur Vereinfachung der Analyse des Argumentes nehmen wir an, es g"abe nur
zwei Arten von j"ahrlichen Schritten: $+1$ und $-1$, Fortschritt und
R"uckschritt,
und beide haben die gleiche Wahrscheinlichkeit $\frac12$.
\begin{teilaufgaben}
\item
Wie gross ist
die Wahrscheinlichkeit, dass eine bestimmte Evolutionsgeschichte erneut
geschieht?
\item Wie gross ist die Wahrscheinlichkeit, dass in einer Million
Jahren insgesamt ein Fortschritt erzielt wird, als mindestens so viele $+1$-Schritte stattfanden wie $-1$-Schritte?
%\item Die Realit"at unterscheidet sich von diesem vereinfachten 
%Modell dadurch, dass die nat"urliche Selektion Fortschritte
%beg"unstigt. Beantworten Sie b) nochmals unter der Annahme, dass 
%$p=0.5000001$.
\end{teilaufgaben}

Offenbar ist es ein grosser Unterschied, ob man erwartet, dass genau
das gleiche nochmals passiert. Treffen sich Leute zum Kartenspiel,
w"aren sie ausserordentlich "uberrascht, wenn die Karten nochmals so
gemischt w"aren wie bei einem fr"uheren Spiel. Dies ist exterem
unwahrscheinlich, sie w"urden den Kartenmischer sofort des Betruges
verd"achtigen. Das heisst aber nicht, dass das deswegen kein Spiel mehr
stattfinden kann.

\begin{loesung}
\begin{teilaufgaben}
\item
In einer Million Jahren finden $10^6$ Schritte mit Wahrscheinlichkeit
$p=\frac12$ statt, diese vorgegebene Abfolge von Schritten hat
Wahrscheinlichkeit $2^{-10^{6}}\simeq 1.0100\cdot 10^{-301030}\simeq 0$.
\item
Die Wahrscheinlichkeit, dass ein Fortschritt stattfindet, ist aber
\begin{align*}
P(\text{Fortschritt})
&=
P\biggl(\text{$\frac{n}2$ mal $+1$}\biggr)
+
P\biggl(\text{$\frac{n}2+1$ mal $+1$}\biggr)
+
P(\text{$n$ mal $+1$})
\\
&=
\sum_{k=\frac{n}2}^n\binom{n}{k}\frac1{2^n}
\simeq
\frac12
\sum_{k=0}^n\binom{n}{k}\frac1{2^n}
= 0.5.
\end{align*}
Ein Fortschritt ist also ziemlich wahrscheinlich.
\end{teilaufgaben}
Mit $p=0.501$ ergibt sich f"ur a) wieder ein praktisch unbedeutender
Wert. F"ur b) kann man die Wahrscheinlichkeit nicht mehr direkt
berechnen, die meisten Computer haben zu wenig weit gehende
Datentypen. Man kann aber die Wahrscheinlichkeit approximativ
bestimmen (Normalapproximation), wobei mabn f"ur b) die Wahrscheinlichkeit
0.9772 findet. Nichtstattfinden von Fortschritt ist also eher unwahrscheinlich.

Es ist also extrem unwahrscheinlich, dass sich eine bestimmte
Evolutionsgeschichte je wiederholen wird, aber es ist sehr wahrscheinlich,
dass eine Entwicklung stattfindet, die zu einem Fortschritt f"uhrt,
selbst ohne das f"ur die tats"achliche Entwicklung entscheidende
nat"urliche Selektion, welche f"ur $p>\frac12$ sorgt.
\end{loesung}

