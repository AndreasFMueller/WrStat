Ein Mathematiker lebt in einem Quartier, in dem die Wahrscheinlichkeit
für einen Einbruch in sein Haus 10\% ist.
Um die Situation zu verbessern, schafft er einen Hund an.
Dieser bellt in 95\% der Fälle, wenn ein Einbrecher in der Gegend ist.
Andernfalls ist der Hund ziemlich ruhig, er bellt nur mit Wahrscheinlichkeit
1\%, wenn kein Einbrecher in der Nähe ist.
\begin{teilaufgaben}
\item
Wie gross ist die Wahrscheinlichkeit, dass der Hund bellt?
\item
Wie gross ist die Wahrscheinlichkeit, dass tatsächlich ein Einbrecher
zugegen ist, wenn der Hund bellt?
\end{teilaufgaben}

\begin{loesung}
Wir verwenden die Ereignisse
\begin{align*}
B&=\{\text{Hund bellt}\}
&
E&=\{\text{Einbrecher}\}.
\end{align*}
Aus dem Text entnehmen wir die folgenden Wahrscheinlichkeiten
\begin{align*}
P(E)&=0.1 &
P(B|E) &= 0.95
\\
&&
P(B|\overline{E}) &= 0.01.
\end{align*}
Damit können die Fragen der Teilaufgaben beantwortet werden.
\begin{teilaufgaben}
\item
Die Wahrscheinlichkeit $P(B)$ kann mit dem Satz von der totalen
Wahrscheinlichkeit berechnet werden:
\begin{align*}
P(B)
&=
P(B|E)P(E) + P(B|\overline{E}) P(\overline{E})
\\
&=
P(B|E)P(E) + P(B|\overline{E}) (1-P(E))
=
0.95\cdot 0.1 + 0.01\cdot 0.9
=
0.104.
\end{align*}
\item
Mit dem Satz von Bayes kann die gesuchte Wahrscheinlichkeit $P(E|B)$
berechnet werden:
\begin{align*}
P(E|B)
&=
P(B|E)\frac{P(E)}{P(B)}
=
0.95\frac{0.1}{0.104}
=
0.913462.
\qedhere
\end{align*}
\end{teilaufgaben}
\end{loesung}

\begin{bewertung}
Ereignisse ({\bf E}) 1 Punkt,
bedinge Wahrscheinlichkeiten ({\bf C}) 1 Punkt,
Satz von der totalen Wahrscheinlichkeit ({\bf T}) 1 Punkt,
Satz von Bayes ({\bf B}) 1 Punkt,
Wahrscheinlichkeit $P(B)$ ({\bf W}) 1 Punkt,
Wahrscheinlichkeit $P(E|B)$ ({\bf E}) 1 Punkt.
\end{bewertung}



