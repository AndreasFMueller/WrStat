Drei unabhängige Ereignisse $A$, $B$ und $C$ haben je die Wahrscheinlichkeit
$P(A)=P(B)=P(C)=p$. Ausserdem weiss man, dass $P(A\cap B\cap C)=q$.
\begin{teilaufgaben}
\item Wie gross $P(A\setminus(B\cup C))$?
\item Wie wahrscheinlich ist, dass keines der Ereignisse eintritt?
\item Angenommen, eines der Ereignisse tritt sicher ein, und $q=0$,
wie gross ist dann $p$?
\item Zeigen Sie: drei Ereignisse mit Wahrscheinlichkeit $\frac12$,
die nicht gleichzeitig eintreten können, sind abhängig.
\end{teilaufgaben}

\thema{Ereignis}
\themaL{Rechenregeln fur Wahrscheinlichkeit}{Rechenregeln für Wahrscheinlichkeit}

\begin{loesung}
Unter den Voraussetzungen der Aufgaben ist
\begin{align*}
P(A\cap B)&=P(A)\cdot P(B) = p^2\\
P((A\cap B)\setminus C)&=p^2-q
\end{align*}
\begin{teilaufgaben}
\item Es ist
\begin{align*}
P(A\setminus (B\cup C))
&=
P(A) - P(A\cap B\setminus C) - P(A\cap C\setminus B)-P(A\cap B\cap C)
\\
&=p-p^2+q-p^2+q-q
\\
&=p-2p^2+q
\end{align*}
\item
Es ist die Wahrscheinlichkeit $P(\overline{A\cup B\cup C}) = 1-P(A\cup B\cup C)$ zu berechnen.
Es ist
\begin{align*}
P(\overline{A\cup B\cup C})
&=
1-P(A\cup B \cup C)\\
P(A\cup B \cup C)
&=
P(A\setminus(B\cup C))
+P(B\setminus(C\cup A))
+P(C\setminus(A\cup B))
\\
&\quad
+P((A\cap B)\setminus C)
+P((B\cap C)\setminus A)
+P((C\cap A)\setminus B)
\\
&\quad
+P(A\cap B\cap C)
\\
&=3(p-2p^2+q)+3(p^2-q)+q
\\
&=3p-3p^2+q
\\
P(\overline{A\cup B\cup C})
&=
1-3p+3p^2-q
\end{align*}
\item
Wenn eines der Ereignisse sicher eintritt, dann ist $P(A)=p=1$, dann muss
aber auch $q=1$ sein. Dieser Fall kann also eigentlich gar nicht
eintreten.

Man könnte die Frage aber auch so verstehen: Angenommen, mindestens
eines der Ereignisse tritt ein, und $q=0$. Wie gross ist dann $p$?
Wenn mindestens eines der Ereignisse eintritt, dann ist
$P(A\cup B\cup C)=1$, also
\[
1=3p-3p^2+q.
\]
Wegen $q=0$ folgt
\[
3p^2-3p+1=0
\]
Diese quadratische Gleichung kann man nach $p$ auflösen
\[
p=\frac{3\pm\sqrt{9-12}}{6}
\]
Auch wenn man die Frage so versteht, kann dieser Fall nicht eintreten.
\item
Wenn Ereignisse nicht gleichzeitig eintreten können, dann sind
sie bereits abhängig, die Kenntnis der Wahrscheinlichkeit $p$
ist dazu nicht erforderlich.
\qedhere
\end{teilaufgaben}
\end{loesung}

