Zwei Standardfamilien (2 Eltern, 2 nicht erwachsene Kinder) und eine
Einelternfamilie mit 2 (nicht erwachsenen) Kinden gehen zusammen Skifahren,
insgesamt 11 Personen.
Sie stehen bei einem vierplätzigen Sessellift an und bemühen sich,
möglichst zusammen Sessel zu besteigen. Tatsächlich gelingt es
ihnen, zwei Sessel mit 4 Familienmitgliedern und einen dritten
Sessel mit drei Mitgliedern zu besetzen.
\begin{teilaufgaben}
\item Auf wieviele Arten können sich die Familienmitglieder auf
die drei Sessel des Sessellifts verteilen?
\item
Wie gross ist die Wahrscheinlichkeit, dass sich auf dem ersten
Sessel nur Erwachsene befinden?
\item
Wie gross ist die Wahrscheinlichkeit, dass sich auf dem letzten Sessel
genau ein Kind befindet?
\end{teilaufgaben}

\thema{Kombinatorik}
\thema{Wahrscheinlichkeit}
\thema{hypergeometrische Verteilung}

\begin{loesung}
\begin{teilaufgaben}
\item Für den ersten Sessel müssen 4 Personen aus der Gesellschaft
von 11 Personen ausgewählt werden, was auf $\binom{11}{4}=330$
Arten möglich ist. Für jede solche Wahl müssen jetzt 4 Personen
aus den verbleibenden 7 für den zweiten Sessel gewählt werden,
was auf $\binom{7}{4}=35$ Arten möglich ist. Insgesamt gibt es
also 
\[
\binom{11}{4}\binom{7}{4}=330\cdot 35=11550
\]
Möglichkeiten.

Man könnte natürlich auch beim letzten Sessel beginnen, für
den 3 von 11 Personen ausgewählt werden müssen. Für den 
zweitletzten Sessel müssen dann 4 Personen aus den verbleibenden
8 ausgewählt werden, also 
\[
\binom{11}{3}\binom{8}{4}=165\cdot 70=11550
\]
Möglichkeiten, wie vorhin.
\item
Um die Wahrscheinlichkeit zu berechnen muss man zählen, auf
wieviele Arten die Mitglieder verteilt werden können, so dass
auf dem ersten Sessel genau 4 Erwachsene platziert werden.
Dazu müssen für den ersten Sessel 4 aus 5 Erwachsenen gewählt werden,
für den zweiten Sessel 4 aus den verbleibenden 7 Personen. Insgesamt
gibt es also
\[
\binom{5}{4}\binom{7}{4}=5\cdot 35=175
\]
Möglichkeiten. Die gesuchte Wahrscheinlichkeit ist daher
\[
P(\text{nur Erwachsene auf dem ersten Sessel})=
\frac{175}{11550}=0.01515152.
\]
\item
Wieder muss gezählt werden, auf wieviele Arten eines der 6 Kinder
und 2 der 5 Erwachsenen 
für
den letzten Sessel ausgewählt werden können, es gibt also
\[
\binom{6}{1}\binom{5}{2}\binom{8}{4}=6\cdot 10\cdot70=4200
\]
Möglichkeiten, die Personen so zu verteilen, dass auf dem dritten
Sessel genau ein Kinde Platz findet. Die Wahrscheinlichkeit dafür ist
also
\[
P(\text{genau ein Kind auf dem dritten Sessel})=\frac{4200}{11550}=0.3636364.
\]
\end{teilaufgaben}
Die Wahrscheinlichkeiten in b) und c) könnte man auch mit der
hypergeometrischen Verteilung berechnen. Für b) muss man von
den 11 Personen 4 für den ersten Sessel auswählen.
Die Wahrscheinlichkeit,
dass alle vier Erwachsene sind, ist
die gleiche, wie bei einer Lotterie, in der fünf aus elf Zahlen gezogen wurde,
mit 4 Markierungen vier Richtige zu erreichen:
\[
\frac{
\binom{5}{4}\binom{6}{0}
}{
\binom{11}{4}
}
=\frac{5}{330}\simeq 0.01515
\]
Oder analog in c):
\[
\frac{
\binom{6}{1}\binom{5}{2}
}{
\binom{11}{3}
}
=\frac{6\cdot 10}{165}\simeq 0.3636.
\qedhere
\]
\end{loesung}
