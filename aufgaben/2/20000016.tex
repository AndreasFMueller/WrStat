Die Elementarereignisse $\Omega=\{1,2,3,4,5,6,7,8\}$ haben je die gleiche
Wahrscheinlichkeit. Ausserdem seien die folgenden Ereignisse gegeben:
\begin{align*}
A&=\{1,2,3,4\}\\
B&=\{2,3,4,5\}\\
C&=\{4,6,7,8\}
\end{align*}
\begin{teilaufgaben}
\item Berechnen Sie $P(A\cap B\cap C)$
\item Berechnen Sie $P(A)P(B)P(C)$
\item Sind $A$, $B$ und $C$ unabhängig?
\end{teilaufgaben}

\thema{Rechenregeln für Wahrscheinlichkeit}
\thema{Unabhängigkeit}

\begin{loesung}
\begin{teilaufgaben}
\item Die Schnittmenge ist $A\cap B\cap C=\{4\}$, also ist
\[
P(A\cap B\cap C)=\frac18|A\cap B\cap C|=\frac18
\]
\item Da alle drei Mengen $A$, $B$ und $C$ vier Elemente haben,
ist $P(A)=P(B)=P(C)=\frac{4}{8}=\frac12$, also ist
$P(A)P(B)P(C)=\frac18$, also auch $P(A)P(B)P(C)=\frac18$.
\item Es ist zu untersuchen, ob $P(A\cap B)=P(A)P(B)$ und analog für
die anderen Paare.
\begin{align*}
A\cap B&=\{2,3,4\}&P(A\cap B)=\frac38&\ne P(A)P(B)=\frac14\\
A\cap C&=\{4\}&P(A\cap C)=\frac18&\ne P(A)P(C)=\frac14\\
B\cap C&=\{4\}&P(B\cap C)=\frac18&\ne P(B)P(C)=\frac14
\end{align*}
Somit sind die drei Ereignisse nicht unabhängig.
\qedhere
\end{teilaufgaben}
\end{loesung}

