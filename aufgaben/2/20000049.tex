Brustkrebs tritt bei etwa 1\% der Frauen auf.
Im Rahmen der Diskussion um die Krankenkassenprämien wird allenthalben
die Frage nach der Wirksamkeit von Vorsorgeuntersuchungen ins Feld 
geführt.
Ein Mammogramm kann bei einer an Brustkrebs erkrankten Frau in 80\% der
Fälle den Krebs korrekt erkennen.
Bei einer gesunden Frau zeigt das Mammogramm aber trotzdem in 9.6\% der Fälle
eine Erkrankung an.

\thema{Satz von Bayes}
\thema{bedingte Wahrscheinlichkeit}
\thema{totale Wahrscheinlichkeit}

\begin{teilaufgaben}
\item
Wie gross ist die Wahrscheinlichkeit, dass ein Mammogramm Krebs anzeigt?
\item
Wie gross ist die Wahrscheinlichkeit, dass eine Frau, deren Mammogramm
Krebs anzeigt, tatsächlich Brustkrebs hat?
\end{teilaufgaben}

\begin{loesung}
Wir verwenden die folgenden Ereignisse
\begin{align*}
B&=\{\text{an Brustkrebs erkrankt}\}
\\
M&=\{\text{Mammogramm zeigt Krebs}\}.
\end{align*}
Bekannt sind die folgenden Wahrscheinlichkeiten
\begin{align*}
P(B)&=0.01\\
P(M|B)&=0.8\\
P(M|\bar{B})&=0.096.
\end{align*}
\begin{teilaufgaben}
\item
Die Wahrscheinlichkeit kann mit dem Satz über die totale Wahrscheinlichkeit
ermittelt werden:
\begin{align*}
P(M)
&=
P(M|B) P(B) + P(M|\bar{B}) P(\bar{B})
=
P(M|B) P(B) + P(M|\bar{B}) (1-P(B))
\\
&=
0.8\cdot 0.01 + 0.096\cdot 0.99
=
0.10304.
\end{align*}
\item
Gesucht ist die Wahrscheinlichkeit $P(B|M)$, sie kann mit dem Satz von
Bayes berechnet werden:
\begin{align*}
P(B|M)
&=
\frac{P(B)}{P(M)} P(M|B)
=
\frac{0.01}{0.10304}\cdot 0.8
=
0.07764.
\end{align*}
Nur etwa in jedem 13-ten Fall ist die Diagnose des Mammogramms also richtig.
\qedhere
\end{teilaufgaben}
\end{loesung}

\begin{diskussion}
Dieses Problem wird auf der Website
\url{https://betterexplained.com/articles/an-intuitive-and-short-explanation-of-bayes-theorem/}
in einem Video erklärt.
\end{diskussion}

\begin{bewertung}
Ereignisse ({\bf E}) 1 Punkt,
Wahrscheinlichkeiten ({\bf W}) 1 Punkt,
Satz von der totalen Wahrscheinlichkeit ({\bf T}) 1 Punkt,
Wert für $P(M)$ ({\bf M}) 1 Punkt,
Satz von Bayes ({\bf B}) 1 Punkt,
Wert für $P(B|M)$ ({\bf Z}) 1 Punkt.
\end{bewertung}
