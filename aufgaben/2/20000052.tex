Masern ist eine hoch ansteckende und gefährliche Krankheit,
allein 2014 starben weltweit über 114900 Menschen an Masern.
Im gleichen Jahr erhielten 85\% aller Kinder eine Impfdosis.
Diese führt aber nicht immer zur Immunität, so dass trotz Impfung
ein Erkrankungsrisiko von etwa 15\% bleibt.
In den westlichen Ländern werden daher mehrere Impfdosen verabreicht,
umd das Risiko weiter zu senken.

Mit verschiedenen Modellen wurde untersucht, wie gross die Wahrscheinlichkeit
ist, dass ein Kind sich in einer bestimmten Situation, zum Beispiel
bei einem Krankenhausbesuch oder im Kindergarten mit Masern ansteckt.
Wir nehmen an, dass in einer spezifischen Situation eine Maserninfektion
bei den Kindern die nicht immun sind, in 20\% der Fälle auftritt

Was für Ereignisse bieten sich an, wenn man untersuchen will, mit welcher
Wahrscheinlichkeit Kinder unter verschiedenen Voraussetzungen
an Masern erkranken.

\begin{loesung}
Das Experiment besteht darin, ein Kind aus der Weltbevölkerung auszuwählen
und daraufhin zu untersuchen, ob es gegen Masern geimpft ist und/oder
dagegen immun ist.
\begin{align*}
\Omega &= \{\text{alle Kinder}\}
\\
I&=\{\text{gegen Masern geimpft}\}
\\
J&=\{\text{immun gegen Masern}\}
\\
M&=\{\text{Kind ist an Masern erkrankt}\}
\qedhere
\end{align*}
\end{loesung}
