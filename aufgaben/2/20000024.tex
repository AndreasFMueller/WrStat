$6\%$ aller Kinder zwischen 6 und 12 Jahre haben schlechte Z"ahne.
Von den Kindern mit schlechten Z"ahnen essen zwei Drittel mindestens eine
Tafel Schokolade pro Woche oder trinken w"ochentlich mindestens
dreimal ein S"ussgetr"ank. Von den Kindern mit guten Z"ahnen tun dies
nur 17\%.
\begin{teilaufgaben}
\item Wie gross ist die Wahrscheinlichkeit, dass ein Kind, das "uberm"assig
Schokolade isst, schlechte Z"ahne hat?
\item Wie gross ist dei Wahrscheinlichkeit, dass ein Kind, das nicht
"ubrerm"assig Schokolade isst, gute Z"ahne hat?
\end{teilaufgaben}

\begin{loesung}
Wir verwenden die Ereignisse
\begin{align*}
Z&=\{\text{Kind hat gute Z"ahne}\}
\\
S&=\{\text{Kind konsumiert viel Schokolade}\}
\end{align*}
Wir wissen aus der Aufgabenstellung folgende Wahrscheinlichkeiten:
\begin{align*}
P(\bar Z)&=0.06\\
P(S|\bar Z)&=\frac23\\
P(S|Z)&=0.17.
\end{align*}
\begin{teilaufgaben}
\item Gesucht ist $P(\bar Z|S)$. Der Satz von Bayes liefert
\[
P(\bar Z|S)= P(S|\bar Z) \frac{P(\bar Z)}{P(S)}.
\]
Darin kennen wir $P(S)$ nicht, k"onnen es aber aus dem Satz von der
totalen Wahrscheinlichkeit bekommen:
\[
P(S)=P(S|Z)P(Z)+P(S|\bar Z)P(\bar Z)
=0.17\cdot (1-0.06)+\frac23\cdot 0.06=0.1998.
\]
Damit finden wir das Resultat
\[
P(\bar Z|S)=\frac23\cdot \frac{0.06}{0.1998}=0.2002
\]
Ein F"unftel der Kinder, die "uberm"assig Schokolade essen, haben
schlechte Z"ahne.
\item Gesucht ist $P(Z|\bar S)$. Aus dem Satz von Bayes folgt wieder
\[
P(Z|\bar S)=P(\bar S|Z)\frac{P(Z)}{P(\bar S)}=(1-P(S|Z))\frac{P(Z)}{P(\bar S)}
=(1-0.17)\frac{1-0.06}{1-0.2002}=0.975.
\]
97.5\% der Kinder, die nicht "uberm"assig Schokolade essen, haben gute 
Z"ahne.
\end{teilaufgaben}
\end{loesung}

\begin{bewertung}
Ereignisse ({\bf E}) 1 Punkt,
"Ubersetzung der bedingten Wahrscheinlichkeiten ({\bf W}) 1 Punkt,
Satz von Bayes ({\bf B}) 1 Punkt,
Satz von der Totalen Wahrscheinlichkeit ({\bf T}) 1 Punkt,
Resultat a) ({\bf A}) 1 Punkt,
Resultat b) ({\bf B}) 1 Punkt.
\end{bewertung}



