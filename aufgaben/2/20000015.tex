Gemäss Zahlen des deutschen statistischen Bundesamster
befanden sich 2002 in deutschen Strafanstalten 58000 Männer
und 2700 Frauen. Sind Geschlecht und Kriminalität unabhängig?

\thema{Unabhängigkeit}

\begin{loesung}
Die Ereignisse in diesem Fall sind $S$, bestehend aus allen in Gefägnissen
einsitzenden Straftätern, und $M$, allen Männern. Zu testen ist, ob
$P(S\cap M)=P(S)\cdot P(M)$. Wir dürfen annehmen, dass $P(M)=0.5$.
Dann gilt
\begin{align*}
P(S\cap M)&=\frac{58000}{N}\\
P(S)\cdot P(M)&=\frac{60700}{N}\cdot 0.5 = \frac{30350}{N}
\end{align*}
Auch in diesem Fall dürfte Unabhängigkeit nicht gegeben sein.
\end{loesung}

