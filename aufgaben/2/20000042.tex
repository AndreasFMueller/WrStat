In der Gratiszeitung 20minuten vom 8.~Februar 2016 erschien die folgende
Notiz:
\begin{center}
\includeagraphics[width=0.5\hsize]{raeppler.jpg}
\end{center}
\begin{teilaufgaben}
\item Wie gross ist der erwartete Gewinn, wenn man eine Münze zufällig
aus dem Raum auswählen kann?
\item
Jeder konnte während höchstens zwei Minuten versuchen, die 1-Millionen-Rappen
Münze zu finden. 
Unter der Annahme, dass man pro Sekunde eine Münze inspizieren kann, 
berechnen Sie die Gewinnerwartung dieses Spiels.
\end{teilaufgaben}
Wir gehen davon aus, dass man in beiden Fällen die ausgewählten Münzen
behalten darf.

\begin{loesung}
\begin{teilaufgaben}
\item
In dem Raum befinden sich 3999999 Münzen mit Wert 1 und 1 Münze mit
Wert 1000000.
Wählt man eine Münze zufällig, erhält man mit Wahrscheinlichkeit
$3999999/4000000$ eine Münze mit Wert 1, und mit Wahrscheinlichkeit
$1/4000000$ Eine Münze mit Wert 1000000.
Der erwartete Gewinn ist daher
\[
E(X) = \frac{3999999}{4000000}\cdot 5 + \frac{1}{4000000}\cdot 1000000
=4.99999975 + 0.25 = 5.24999975\,\text{Rappen}.
\]
\item
Da es nur eine einzige Münze mit Wert 1000000 gibt, gewinnt man
in zwei Minuten entweder 120 Rappen oder 1000119 Rappen.
Die Wahrscheinlichkeit, dass die 1000000 Rappen Münze unter den
120 Münzen ist, ist $120/4000000$, so dass der erwartete Gewinn
\[
\frac{3999880}{4000000}\cdot 120\cdot 5
+
\frac{120}{4000000}\cdot 1000119
=
529.981 + 30.01785=629.99985\,\text{Rappen}
\]
ist.
\end{teilaufgaben}
\end{loesung}

\begin{diskussion}
Man kann für dieses Resultat natürlich auch die hypergeometrische
Verteilung verwenden, denn es geht ja darum, in einem ``Lotto'', bei dem
man aus $n=4000000$ Zahlen $k=120$ auswählen darf, einen ``Einer''
zu erzielen.
Die gesuchte Wahrscheinlichkeit ist
\begin{align*}
\frac{\displaystyle\binom{n-1}{k-1}\binom{1}{1}}{\displaystyle\binom{n}{k}}
&=
\frac{\displaystyle\frac{(n-1)(n-2)\dots(n-k+1)}{1\cdot 2\cdot\dots\cdot (k-1)}}%
{\displaystyle\frac{n(n-1)(n-2)\dots(n-k+1)}{1\cdot 2\cdot\dots\cdots k}}
=\frac{k}{n},
\end{align*}
wie in der Lösung verwendet.
\end{diskussion}

