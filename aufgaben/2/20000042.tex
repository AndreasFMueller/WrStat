In der Gratiszeitung 20minuten vom 8.~Februar 2016 erschien die folgende
Notiz:
\begin{center}
\includeagraphics[width=0.5\hsize]{raeppler.jpg}
\end{center}
\begin{teilaufgaben}
\item Wie gross ist der erwartete Gewinn, wenn man eine M"unze zuf"allig
aus dem Raum ausw"ahlen kann?
\item
Jeder konnte w"ahrend h"ochstens zwei Minuten versuchen, die 1-Millionen-Rappen
M"unze zu finden. 
Unter der Annahme, dass man pro Sekunde eine M"unze inspizieren kann, 
berechnen Sie die Gewinnerwartung dieses Spiels.
\end{teilaufgaben}
Wir gehen davon aus, dass man in beiden F"allen die ausgew"ahlten M"unzen
behalten darf.

\begin{loesung}
\begin{teilaufgaben}
\item
In dem Raum befinden sich 3999999 M"unzen mit Wert 1 und 1 M"unze mit
Wert 1000000.
W"ahlt man eine M"unze zuf"allig, erh"alt man mit Wahrscheinlichkeit
$3999999/4000000$ eine M"unze mit Wert 1, und mit Wahrscheinlichkeit
$1/4000000$ Eine M"unze mit Wert 1000000.
Der erwartete Gewinn ist daher
\[
E(X) = \frac{3999999}{4000000}\cdot 5 + \frac{1}{4000000}\cdot 1000000
=4.99999975 + 0.25 = 5.24999975\,\text{Rappen}.
\]
\item
Da es nur eine einzige M"unze mit Wert 1000000 gibt, gewinnt man
in zwei Minuten entweder 120 Rappen oder 1000119 Rappen.
Die Wahrscheinlichkeit, dass die 1000000 Rappen M"unze unter den
120 M"unzen ist, ist $120/4000000$, so dass der erwartete Gewinn
\[
\frac{3999880}{4000000}\cdot 120\cdot 5
+
\frac{120}{4000000}\cdot 1000119
=
529.982 + 30.01785=629.99985\,\text{Rappen}
\]
ist.
\end{teilaufgaben}
\end{loesung}

\begin{diskussion}
Man kann f"ur dieses Resultat nat"urlich auch die hypergeometrische
Verteilung verwenden, denn es geht ja darum, in einem ``Lotto'', bei dem
man aus $n=4000000$ Zahlen $k=120$ ausw"ahlen darf, einen ``Einer''
zu erzielen.
Die gesuchte Wahrscheinlichkeit ist
\begin{align*}
\frac{\displaystyle\binom{n-1}{k-1}\binom{1}{1}}{\displaystyle\binom{n}{k}}
&=
\frac{\displaystyle\frac{(n-1)(n-2)\dots(n-k+1)}{1\cdot 2\cdot\dots\cdot (k-1)}}%
{\displaystyle\frac{n(n-1)(n-2)\dots(n-k+1)}{1\cdot 2\cdot\dots\cdots k}}
=\frac{k}{n},
\end{align*}
wie in der L"osung verwendet.
\end{diskussion}

