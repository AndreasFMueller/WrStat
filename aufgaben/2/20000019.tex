Ein Alzheimer-Test erkennt Alzheimer in 95\% der Fälle, in denen tatsächlich
Alzheimer vorliegt. Leider zeigt er auch bei 10\% der gesunden Patienten
fälschlicherweise Alzheimer an. Nehmen Sie an, dass 4\% der über 65-jährigen
Alzheimer haben.
Wie gross ist die Wahrscheinlichkeit, dass der Test bei einer über
65-jährigen Person Alzheimer anzeigt?

\thema{bedingte Wahrscheinlichkeit}
\thema{totale Wahrscheinlichkeit}

\begin{loesung}
Wir unterscheiden die folgenden Ereignisse
\begin{align*}
A&=\{\text{Patient hat Alzheimer}\}
\\
T&=\{\text{Test zeigt Alzheimer an}\}
\\
N&=\{\text{Test zeigt kein Alzheimer an}\}
\end{align*}
Darüber sind folgende Wahrscheinlichkeiten bekannt:
\begin{align*}
P(T|A)&=0.95\\
P(T|\bar A)&=0.1\\
P(A)&=0.04
\end{align*}
Es ist $P(T)$ zu berechnen.
\begin{align*}
P(T)&=P(T|A)P(A)+P(T|\bar A)P(\bar A)=0.95\cdot 0.04 + 0.1\cdot (1-0.04)
=0.134.
\qedhere
\end{align*}
\end{loesung}

