An einer gewissen Hochschule sind 4\% der Studenten grösser als 1.8m
aber nur 1\% der Frauen.
Die Hochschule ist offenbar nicht die HSR, denn das Verhältnis von
Männern zu Frauen ist $2:3$.
Sie begegnen einer Person, die an dieser Hochschule studiert und über
1.8m gross ist.
Wie gross ist die Wahrscheinlichkeit, dass es sich um eine Frau handelt?

\thema{bedingte Wahrscheinlichkeit}
\thema{Satz von Bayes}
\thema{totale Wahrscheinlichkeit}

\begin{loesung}
Ereignisse und bedingte Wahrscheinlichkeiten:
\begin{align*}
F&=\{\text{Frau}\}&P(F)&=\frac{3}{5}\\
M&=\{\text{Mann}\}&P(M)&=\frac{2}{5}\\
G&=\{\text{über 1.8m gross}\}&&
\end{align*}
Ausserdem sind die bedingten Wahrscheinlichkeiten für $G$ bekannt:
\begin{align*}
P(G|M)&= 0.04\\
P(G|F)&= 0.01
\end{align*}
Gesucht ist $P(F|G)$.

Nach dem Satz von Bayes ist zunächst
\begin{equation}
P(F|G)
=
P(G|F)\frac{P(F)}{P(G)},
\label{20000046:1}
\end{equation}
darin ist aber $P(G)$ zunächst noch nicht bekannt.
Wir können $P(G)$ aber mit Hilfe des Satzes von der totalen Wahrscheinlichkeit
ermitteln, es gilt nämlich
\begin{align*}
P(G)
&=
P(G|M)P(M) + P(G|F)P(F)
\\
&=
0.04\cdot 0.4+0.01\cdot0.6=0.022.
\end{align*}
Eingesetzt in \eqref{20000046:1} erhalten wir
\begin{align*}
P(F|G)
&=0.01\cdot\frac{0.6}{0.022}
=
\frac{0.006}{0.022}
=
\frac{3}{11}
=
0.272727272.
\qedhere
\end{align*}
\end{loesung}

\begin{diskussion}
Man kann die Aufgabe auch missverstehen indem man annimmt, dass 
das Wort ``Studenten'' im ersten Satz sowohl Männer als auch Frauen 
meint.
Dies entspricht jedoch nicht dem aktuell akzeptierten politisch
korrekten Sprachgebrauch, in dem sich das Wort ``Studenten'' nur
auf Männer beziehen kann, während für Männer und Frauen das Wort
``Studierende'' verwendet werden muss.

Trotz dieses Missverständnisses kann man zu einer Schlussfolgerung kommen.
Gegenüber der Lösung oben braucht man in diesem Fall
$P(G)=0.04$, während wir nichts über $P(G|M)$ wissen.
Gefragt ist
\[
P(F|G)
=
P(G|F)\frac{P(F)}{P(G)}
=
0.01\cdot
\frac{0.6}{0.04}
=
0.15.
\]
\end{diskussion}

\begin{bewertung}
Ereignisse ({\bf E}) 1 Punkt,
bedingte Wahrscheinlichkeiten ({\bf C}) 1 Punkt,
Satz von Bayes ({\bf B}) 1 Punkt,
Satz von der totalen Wahrscheinlichkeit ({\bf T}) 1 Punkt,
Wahrscheinlichkeit $P(G)$ ({\bf G}) 1 Punkt,
Wahrscheinlichkeit $P(F|G)$ ({\bf P}) 1 Punkt.
\end{bewertung}
