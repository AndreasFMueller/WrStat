Modellraketen aus dem Spielwarenhandel verwenden eine Fallschirm aus
Plastikfolie. Eine Schwarzpulverladung sollte diesen Fallschirm am
Scheitelpunkt der Flugbahn auswerfen. Doch manchmal, in etwa 20\%
der Fl"uge, l"asst die heisse
Flamme der Ausstossladung den Fallschirm zusammenkleben. Einige
Rocketeers pudern den Fallschirm mit Talkpuder ein, dadurch wird
die Wahrscheinlichkeit des Verklebens halbiert.
Zwei von f"unf Rocketeers verwenden diesen Trick.
\begin{teilaufgaben}
\item Wie gross ist die Wahrscheinlichkeit, dass ein Fallschirm
verklebt?
\item Wie gross ist die Wahrscheinlichkeit, dass eine Rakete
ohne verklebten Fallschirm einen gepuderten Fallschirm hatte?
\end{teilaufgaben}

\ifthenelse{\boolean{loesungen}}{
\begin{loesung}
Wir haben folgende Ereignisse:
\begin{align*}
P&=\{\text{Fallschirm wurde gepudert}\}\\
V&=\{\text{Fallschirm verklebt}\}
\end{align*}
Und folgende Informationen "uber die Wahrscheinlichkeiten:
\begin{align*}
P(V|\bar P)&=0.2\\
P(V|P)&=0.1\\
P(P)&=0.4
\end{align*}
\begin{teilaufgaben}
\item Gesucht ist $P(V)$, was man mit dem Satz "uber die totale
Wahrscheinlichkeit berechnen kann:
\begin{align*}
P(V)&=P(V|P)P(P)+P(V|\bar P)P(\bar P)\\
&=P(V|P)P(P)+P(V|\bar P)(1-P(P))\\
&=0.1\cdot 0.4+0.2\cdot 0.6=0.16
\end{align*}
\item Gesucht wird $P(P|\bar V)$, dazu kann die Formel von Bayes verwendet
werden:
\begin{align*}
P(P|\bar V)
&=
P(\bar V |P)\frac{P(P)}{P(\bar V)}
\\
&=(1-P(V|P))\frac{P(P)}{1-P(V)}\\
&=
0.9 \cdot\frac{0.4}{0.84}=0.42857
\end{align*}
\end{teilaufgaben}
\end{loesung}
}{}
