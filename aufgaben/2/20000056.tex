In den Kalenderwochen 40-43 des Jahres 2021
waren waren in deutschen Intensivstationen
34\% der über 60-jährigen Corona-Patienten geimpft.
85\% aller über 60-jähriger ist bereits geimpft.
Der Anteil der über 60-jährigen, die auf der Intensivstation zu landen,
ist trotzdem klein und es ist schwierig, darüber genaue Daten zu finden.
Arbeiten Sie mit dem Parameter $p$.
\begin{teilaufgaben}
\item 
Wie gross ist die Wahrscheinlichkeit, dass ein Geimpfter auf der
Instenstation landet?
\item
Wieviel mal grösser ist die Wahrscheinlichkeit, dass ein Geimpfter
auf der Intensivstation landet als ein Ungeimpfter?
\end{teilaufgaben}

%https://www.youtube.com/watch?v=mVdKwSY6FLg

\begin{hinweis}
Ihre Resultate werden zum Teil den Parameter $p$ enthalten.
\end{hinweis}

\begin{loesung}
Wir verwenden die folgenden Ereignisse:
\begin{align*}
\Omega&=\{\text{alle über 60-jährigen}\}
\\
I&=\{\text{Covid-19-Patienten, die auf der Intensivstation landen}\}
\\
G&=\{\text{geimpfte}\}
\\
U&=\overline{G}=\{\text{ungeimpfte}\}
\end{align*}
Bekannt sind aus der Aufgabenstellung die folgenden Wahrscheinlichkeiten:
\begin{align*}
P(G)       &= 0.85 &
P(G\mid I) &= 0.34 \\
P(U)       &= 0.15 &
P(U\mid I) &= 0.66 \\
P(I)       &= p
\end{align*}
\begin{teilaufgaben}
\item
Gesucht ist
\[
P(I|G)
=
P(G|I) \frac{P(I)}{P(G)}
=
\frac{P(G|I)p}{P(G)}.
\]
\item
Wir müssen $P(I\mid G)$ und $P(I\mid U)$ bestimmen.
Dazu kann der Satz von Bayes verwendet werden:
\[
\left.
\begin{aligned}
P(I|G) &= \frac{P(G|I)p}{P(G)}
\\
P(I|U) &= \frac{P(U|I)p}{P(U)}
\end{aligned}
\right\}
\Rightarrow
\frac{P(I|U)}{P(I|G)}
=
\frac{P(U|I)P(I)}{P(U)} \frac{P(G)}{P(G|I)P(I)}
=
\frac{P(U|I)P(G)}{P(G|I)P(U)}
=
\frac{0.66\cdot 0.85}{0.34\cdot 0.15}
\approx
10.68
\]
\end{teilaufgaben}
\end{loesung}

