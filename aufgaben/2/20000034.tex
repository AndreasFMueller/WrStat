Regelm"assig gibt die Departementsverteilung im Bundesrat zu reden.
Nehmen wir an, alle $n$ Bundesr"ate sind wiedergew"ahlt worden,
und bestimmen jetzt eine neue Departementsverteilung.
\begin{teilaufgaben}
\item Wieviele m"ogliche Departementsverteilungen gibt es insgesamt?
\item Sei $A_i$ die Menge aller Departementsverteilungen, bei denen
Bundesrat $i$ sein Departement beh"alt. Wie gross ist $|A_i|$?
\item Was bedeuten $A_i\cap A_j$ und $A_i\cap A_j\cap A_k$,
wobei $i$, $j$ und $k$ alle verschieden sind. Wieviele Elemente
enthalten $A_i\cap A_j$ und $A_i\cap A_j\cap A_k$?
\item Zeichnen Sie das Ereignis ``Bundesr"ate 1 und 2 behalten ihr Departement,
Bundesrat 5 wechselt''.
\item Zeichnen Sie das Ereignis $A=\{\text{mindestens ein Bundesrat beh"alt sein Departement}\}$.
\item Bestimmen Sie $P(A)$ f"ur den Fall $n=3$.
\item Wie gross ist die Wahrscheinlichkeit, dass bei zuf"alliger
Departementsverteilung kein Bundesrat sein altes Departement weiterf"uhrt?
\end{teilaufgaben}

\begin{loesung}
\begin{teilaufgaben}
\item Anzahl Permutationen: $n!$.
\item Departement eines Bundesrates ist festgelegt, die verbleibenden
$(n-1)$ Bundesr"ate k"onnen beliebig auf $n-1$ Departemente verteilt
werden: Permutationen von $n-1$ Elementen, also $|A_i|=(n-1)!$.
\item Die Eregnisse und ihre Kardinalit"aten sind
\begin{align*}
A_i\cap A_j&=\{\text{Bundesr"ate $i$ und $j$ behalten ihr Departement}\}
\\
|A_i\cap A_j|&=(n-2)!
\\
A_i\cap A_j\cap A_k&=\{\text{Bundesr"ate $i$, $j$ und $k$ behalten ihr Departement}\}
\\
|A_i\cap A_j\cap A_k|&=(n-3)!
\end{align*}
Behalten $k$ Bundesr"ate ihr Departement, m"ussen die "ubrigen
$n-k$ Departemente auf  die verbleibenden $n-k$ Bundesr"ate verteilt
werden, was auf $(n-k)!$ Arten m"oglich ist.
\item
\item $A$ ist die Vereinigung aller Ereignisse $A_i$:
\[
A=\bigcup_{i=1}^nA_i.
\]
\item
Dazu muss man zun"achst $|A|$ bestimmen.
Obwohl in der Aufgabe nur der Fall $n=4$ verlangt war, und sich dieser
Fall direkt aufschreiben l"asst, f"uhren wir hier zun"achst die allgemeine
Rechnung f"ur beliebiges $n$ durch.
Es gilt
\begin{align*}
|A|&=\sum_{i}|A_i| - \sum_{i\ne j}|A_i\cap A_j| + \sum_{i,j,k}|A_i\cap A_j\cap A_k|-\dots-(-1)^n|A_1\cap\dots\cap A_n|
\\
&=
n\cdot (n-1)!
-\binom{n}{2}\cdot (n-2)!
+\binom{n}{3}\cdot (n-3)!
-\dots
-(-1)^n
\binom{n}{n}\cdot (n-n)!
\\
&=\sum_{k=1}^n(-1)^{k-1}\binom{n}{k}(n-k)!
=\sum_{k=1}^n(-1)^{k-1}\frac{n!}{k!(n-k)!}(n-k)!
\\
&=n!\sum_{k=1}^n(-1)^{k-1}\frac{1}{k!}
\end{align*}
denn die Summe aus Schnitten von $k$ verschiedenen Mengen hat
$\binom{n}{k}$ Summanden (es m"ussen $k$ verschiedenen Indizes
aus den $n$ m"oglichen Indizes ausgew"ahlt werden), die alle
den gleichen Betrag $(n-k)!$ haben.

F"ur den Fall $n=3$ erh"alt man
\begin{align*}
|A|
&=
3\cdot 2! - \binom{3}{2}\cdot 1! + \binom{3}{3}\cdot 0! 
\\
&=
3! - 3\cdot 1! + 1\cdot 0!
\\
&=
6-3+1=4,
\end{align*}
Was man nat"urlich auch durch die direkte Anwendung der Ein-/Ausschalt-Formeln
oder durch Ausz"ahlen der m"oglichen Departementsverteilungen finden kann\footnote{Verteilungen ohne Fixpunkte sind nur die beiden zyklischen Vertauschungen,
also gibt es genau $4$ Departementsverteilungen in $A$}.
Die Wahrscheinlichkeit von $A$ ist also $P(A)=\frac{4}{6}\simeq0.666666$.
\item
$P(\bar A)=1-P(A)\simeq 0.333333$.
\end{teilaufgaben}
Da die Kardinalit"at von $\Omega\setminus A$
\[
|\Omega\setminus A|
= n!\biggl(1-\frac1{1!}+\frac1{2!}-\frac1{3!}+\dots+(-1)^n\frac1{n!}\biggr)
\]
ist, ist
\[
P(\bar A)
= 1-\frac1{1!}+\frac1{2!}-\frac1{3!}+\dots+(-1)^n\frac1{n!}
\]
Die Summe auf der rechten Seite ist aber die beim $(n+1)$-ten Term
abgebrochene Reihenentwicklung von $e^{-1}$. Daher kann man auch
den Grenzwert f"ur $n\to\infty$ berechnen:
\[
\lim_{n\to\infty}P(\bar A)
= 1-\frac1{1!}+\frac1{2!}-\frac1{3!}+\dots=e^{-1}
\simeq 0.367879
\]
Ebenso erkennt man, dass die L"osung von Teilaufgabe g) relativ
nahe an $e^{-1}$ ist.
\end{loesung}

