%
% based on http://www.askamathematician.com/2011/10/q-what-is-bayes-rule-and-how-do-i-use-it-in-daily-life/
%
Jemand klagt über einen Husten, der einfach nicht weggehen will.
Er könnte verursacht sein durch eine allergische Reaktion oder durch
eine Bronchitis. Er nimmt ein anti-allergisches Medikament und
nach einer Stunde ist der Husten weg. Er vermutet daher, das es tatsächlich
eine allergische Reaktion war.

Solche Medikamente sind recht wirkungsvoll, in einem von drei Fällen
verschwindet ein allergischer Husten nach der Einnahme des Medikamentes.
Wäre der Husten durch eine Bronchitis bedingt, würde das (wahrscheinlich
vor allem Dank des Placebo-Effektes) nur in einem von 10 Fällen passieren.
Andererseits ist Bronchitis die etwa fünf mal häufigere Ursache für
Husten.

Was ist wahrscheinlicher: dass der Husten durch eine Allergie verursacht
war oder dass eine Bronchitis im Spiel war?

\begin{loesung}
Wir verwenden folgende Ereignisse
\begin{align*}
A&=\{\text{Husten ist durch eine Allergie verursacht}\}
\\
B&=\{\text{Husten ist durch eine Bronchitis verursacht}\}
\\
M&=\{\text{Husten verschwindet nach Medikamenteinnahme}\}
\end{align*}
Wir kennen folgende Wahrscheinlichkeiten:
\begin{align*}
P(M|A)&=0.33333\\
P(M|B)&=0.1\\
P(B)&=5P(A)&P(A)+P(B)&=1\\
\Rightarrow\quad P(A)&=\frac16&P(B)=&\frac56
\end{align*}
Gefragt wird jetzt nach den Wahrscheinlichkeiten $P(A|M)$ und $P(B|M)$.
Nach dem Satz von Bayes gilt
\[
P(A|M)=\frac{P(A)}{P(M)}P(M|A)
\]
Wir brauchen also $P(M)$, dies können wir aber mit dem Satz
über die totale Wahrscheinlichkeit ausrechnen:
\[
P(M)
=
P(M|A)P(A)+P(M|B)P(B)
=
\frac13\cdot\frac16+\frac1{10}\cdot\frac56
=
\frac1{18}+\frac1{12}=\frac{5}{36}
=
0.1388888
\]
Damit können wir jetzt die gesuchte Wahrscheinlichkeit ausrechnen
\[
P(A|M)
=
\frac{P(A)}{P(M)}P(M|A)
=
\frac{1/6}{0.138888}\frac13
=
0.4
\]
Andererseits ist natürlich
\[
P(B|M)
=
\frac{P(B)}{P(M)}P(M|B)
=
\frac{5/6}{0.138888}\frac1{10}
=
0.6,
\]
d.~h.~es ist wahrscheinlicher, dass der Husten durch eine Bronchitis
verursacht war.
\end{loesung}


\begin{bewertung}
Wahl zweckmässiger Ereignisse ({\bf E}) 1 Punkt,
Identifikation der gegeben Wahrscheinlichkeiten ({\bf I}) 1 Punkt,
Berechnung von $P(A)$ ({\bf A}) 1 Punkt,
Satz von Bayes ({\bf B}) 1 Punkt,
Satz von der totalen Wahrscheinlichkeit ({\bf T}) 1 Punkt,
Berechnung von $P(A|M)$ ({\bf W}) 1 Punkt
\end{bewertung}


