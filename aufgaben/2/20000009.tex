Gegeben ist das Modell-Internet mit folgender Link-Struktur
\begin{teilaufgaben}
\item Berechnen Sie die Google-Matrix {\bf ohne} freien Willen und bestimmen
Sie die Besuchswahrscheinlichkeiten $P(S_i)$.
\item Berechnen Sie die Google-Matrix {\bf mit} freiem Willen mit
$\alpha=0.9$ und bestimmen
Sie die Besuchswahrscheinlichkeiten $P(S_i)$.
\end{teilaufgaben}
\[\UseTips
\entrymodifiers={++[o][F]}
\xymatrix @-1mm {
1 \ar[r] \ar@(dr,dr)[rd] & 2 \ar[d] \ar@(dl,dl)[dl]\\
4 \ar[u] & 3 \ar[l]
}
\]

\begin{loesung}
\begin{teilaufgaben}
\item
Die Verlinkungsmatrix dieses Modell-Internets ist
\[
H=\left(\begin{matrix}
0&0&0&1\\
\frac12&0&0&0\\
\frac12&\frac12&0&0\\
0&\frac12&1&0
\end{matrix}\right)
\]
Die Eigenvektoren können zum Beispiel mit Mathematica gefunden
werden:
\[
\texttt{v = First[Eigenvectors[H]]; v/Apply[Plus, v]}.
\]
Der zweite Befehl sorgt dafür, dass die Summe der Elemente $1$ wird,
wie das von einem Vektor verlangt werden muss, der Wahrscheinlichkeiten
darstellen soll.

Da man aber den Eigenwert weiss, kann man auch $(H-E)p=0$ lösen, was
mit dem Gaussalgorithmus effizient möglich ist:
\begin{align*}
\begin{tabular}{|>{$}c<{$}>{$}c<{$}>{$}c<{$}>{$}c<{$}|}
\hline
     -1&      0& 0& 1\\
\frac12&     -1& 0& 0\\
\frac12&\frac12&-1& 0\\
      0&\frac12& 1&-1\\
\hline
\end{tabular}
&\rightarrow
\begin{tabular}{|>{$}c<{$}>{$}c<{$}>{$}c<{$}>{$}c<{$}|}
\hline
      1&      0& 0&     -1\\
      0&     -1& 0&\frac12\\
      0&\frac12&-1&\frac12\\
      0&\frac12& 1&     -1\\
\hline
\end{tabular}
\\
&\rightarrow
\begin{tabular}{|>{$}c<{$}>{$}c<{$}>{$}c<{$}>{$}c<{$}|}
\hline
      1&      0& 0&      -1\\
      0&      1& 0&-\frac12\\
      0&      0&-1& \frac34\\
      0&      0& 1&-\frac34\\
\hline
\end{tabular}
\\
&\rightarrow
\begin{tabular}{|>{$}c<{$}>{$}c<{$}>{$}c<{$}>{$}c<{$}|}
\hline
      1&      0& 0&      -1\\
      0&      1& 0&-\frac12\\
      0&      0& 1&-\frac34\\
      0&      0& 0&       0\\
\hline
\end{tabular}
\end{align*}
Die letzte Variable ist wie erwartet frei wählbar, setzen wir sie auf $1$,
erhalten wir als Lösungsvektor
\[
\begin{pmatrix}1\\\frac12\\\frac34\\1\end{pmatrix}.
\]
Dieser Vektor hat nicht die gewünschte Normierung: Für einen Vektor
von Wahrscheinlichkeiten müssen die Komponenten sich zu $1$ aufaddieren.
Dies kann man aber immer erreichen, indem man durch die Summe der Komponenten
teilt:
\[
p=\frac{4}{13}\begin{pmatrix}1\\\frac12\\\frac34\\1\end{pmatrix}=
\begin{pmatrix}
0.307692\\
0.153846\\
0.230769\\
0.307692
\end{pmatrix}.
\]

Auch mit Hilfe des in der Vorlesung vorgeführten Iterationsalgorithmus
kann man folgenden Eigenvektor zum Eigenwert $1$ finden:
\[
\begin{pmatrix}
0.307692\\
0.153846\\
0.230769\\
0.307692
\end{pmatrix}
\]
Man kann dies auch mit Hilfe des beigefügten kleinen Programms
{\tt google.cpp} herausfinden.
Dass die Seiten $1$ und $4$ gleich bewertet sind ist darauf zurückzuführen,
dass alle Besucher von Seite $4$ unbedingt auf Seite $1$ weitergeleitet
werden, somit müssen die beiden Seiten gleich viele Besucher haben.
\item
Bei der Berücksichtigung des Freien Willens wird die Matrix zu
\[
\alpha H+\frac{1-\alpha}N A
=
\left(\begin{matrix}
0.025&0.025&0.025&0.925\\
0.475&0.025&0.025&0.025\\
0.475&0.475&0.025&0.025\\
0.025&0.475&0.925&0.025
\end{matrix}\right),
\]
wobei die Matrix $A$ aus lauter Einsen besteht.
\[
\left(\begin{matrix}
0.300761\\
0.160342\\
0.232496\\
0.306401\\
\end{matrix}\right)
\]
Man kann sehen, dass die Einführung des ``freien Willens'' die
Unterbewertung der Seite $2$ etwas anhebt auf Kosten der
überbewerteten Seiten $1$ und $4$.

Interessant ist auch, dass selbst die Verwendung sehr kleine Werte von
$\alpha$ nahe $0$ nicht zu einer "Anderung der Rangfolge der vier
Seiten führt.
\qedhere
\end{teilaufgaben}
\end{loesung}

