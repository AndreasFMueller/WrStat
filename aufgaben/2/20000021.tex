Im Prozess im Jahre 1894 gegen den französische Artillerie-Hauptmann
Alfred Dreyfus wegen
angeblichen Landesverrates spielte ein graphologisches Gutachten eine
Rolle, welches mit an Sicherheit grenzender Wahrscheinlichkeit bewiesen
zu haben behauptete, dass nur Alfred Dreyfus als Autor des einzigen
Beweisstücks in Frage kam.
Bei dem Dokument handelte es sich um ein Begleitschreiben, in dem der
Autor die dem deutschen Kaisserreich übergebenen geheimen Dokumente
beschrieb.
Auf dieser Grundlage wurde Dreyfus verurteilt.
Es stellt sich erst viel später als Fälschung heraus.

Die Affäre stürzte den französischen Staat in eine tiefe Krise
voll Antisemitismus, gewaltsamer Ausschreitungen und Attentaten.
In dem Revisionsverfahren, welches schliesslich 1906 in die Rehabilitation
von Dreyfus mündete, wurden auch die graphologischen ``Beweise''
genauer unter die Lupe genommen.
Sein Autor Alphonse Bertillon genoss als einer der Begründer
der wissenschaftlichen
Kriminalistik grosses Ansehen. In der Dreyfus-Affäre nahm sein Ruf
allerdings Schaden, denn seine graphologische Analyse enthielt
ausser vielen kriminalistischen Fehlern auch einen Fehler in der
Berechnung von Wahrscheinlichkeiten.
Als Experte im Revisionsprozess wies der berühmte Mathematiker
Henri Poincar\'e diesen Fehler nach.
Er wies darauf hin, dass es zwei Arten Wahrscheinlichkeiten gäbe,
die er {\it probabilit\'e des causes} und {\it probabilit\'e des effets}
nennt.
Zur Erläuterung beschrieb er das folgende Experiment, mit
dem er den Unterschied illustrieren wollte. 

Gegeben sind eine Urne $A$ mit 9 weissen und einer schwarzen Kugel
und zwei Urnen $B$ und $C$ mit je 10 weissen und 90 schwarzen Kugeln.
Unter {\it probabilit\'e des effets} versteht er die Frage, mit welcher
Wahrscheinlichkeit man beim Ziehen einer Kugel aus Urne $A$ eine
weisse Kugel erhalten würde.
Wenn man jetzt aber zunächst zufällig eine der Urnen auswählt, und dann
aus der gewählten Urne eine Kugel zieht, hat man ein anderes
Experiment. 
Wenn bei diesem Experiment eine weisse Kugel gezogen wird, wie gross
ist dann die Wahrscheinlichkeit, dass Urne $A$ ausgewählt worden war?
Dies ist, was Poincar\'e als {\it probabilit\'e des causes} bezeichnet.

\begin{teilaufgaben}
\item Wie gross ist die Wahrscheinlichkeit, aus Urne $A$ eine weisse Kugel
zu ziehen?
\item Wie gross ist die Wahrscheinlichkeit, Urne $A$ gewählt zu haben,
wenn eine weisse Kugel gezogen wurde?
\end{teilaufgaben}


\begin{loesung}
Hierbei handelt es sich um ein Problem bedingter Wahrscheinlichkeiten.
Folgende Ereignisse können eintreten:
\begin{align*}
A&=\{\text{Kugel wird aus Urne $A$ gezogen}\}\\
B&=\{\text{Kugel wird aus Urne $B$ gezogen}\}\\
C&=\{\text{Kugel wird aus Urne $C$ gezogen}\}\\
W&=\{\text{Weise Kugel gezogen}\}\\
S&=\{\text{Schwarze Kugel gezogen}\}
\end{align*}
\begin{teilaufgaben}
\item $P(W|A)=\frac{9}{10}= 0.9$
\item Hier ist nach der bedingten Wahrscheinlichkeit $P(A|W)$ gefragt.
Diese kann mit dem Satz von Bayes berechnet werden:
\begin{equation}
P(A|W)=P(W|A)\frac{P(A)}{P(W)}.
\label{20000021:bayes}
\end{equation}
Damit diese Formel genutzt werden kann, müssen $P(W)$ und $P(A)$
bestimmt werden.
Nach Voraussetzung ist $P(A)=P(B)=0.5$, beide Urnen sind gleich
wahrscheinlich. $P(W)$ kann mit dem Satz von der totalen
Wahrscheinlichkeit bestimmt werden\footnote{Es ist nicht zulässig,
einfach die Zahl der weissen Kugel durch die Gesamtzahl der Kugeln
zu teilen, das beschreibt nicht das zweistufige Experiment, welches hier
durchgeführt wird.}:
\begin{align*}
P(W)&=P(W|A)P(A)+P(W|B)P(B)+P(W|C)P(C)
     =0.9\cdot \frac13+0.1\cdot \frac23
\\
    &=0.366666
\end{align*}
Damit bekommen wird aus
(\ref{20000021:bayes})
\[
P(A|W)=0.9\cdot\frac{0.33333}{0.36666}=0.818181.
\qedhere
\]
\end{teilaufgaben}
\end{loesung}

\begin{bewertung}
a) 1 Punkt.
b) Modellierung mit zweckmässigen Ereignissen ({\bf E}) 1 Punkt,
Berechnung einzelner Wahrscheinlichkeiten ({\bf W}) 1 Punkt,
Satz von Bayes ({\bf B}) 1 Punkt,
Satz von der totalen Wahrscheinlichkeit ({\bf T}) 1 Punkt,
Resultat ({\bf R}) 1 Punkt.
\end{bewertung}

\begin{diskussion}
Weitergehende Information zum Beitrag von Henri Poincar\'e im Revisionsprozess
der Dreyfus-Affäre kann in folgendem PDF gefunden werden:
\url{http://poincare.univ-nancy2.fr/digitalAssets/12593_dreyfus_poincare.pdf}
\end{diskussion}


