Von zwei Ereignissen $A$ und $B$ weiss man, dass $P(A)=0.8$ und $P(B)=0.6$.
\begin{teilaufgaben}
\item
Was können Sie daraus über $P(A\cap B)$ schliessen?
\item
Was können Sie über $P(A\cap B)$ sagen, wenn Sie ausserdem wissen, dass 
dass Ereignis $\overline{A\cup B}$ mit Wahrscheinlichkeit $0$ eintritt?
\end{teilaufgaben}

\thema{Ereignis}
\thema{Rechenregeln für Wahrscheinlichkeit}

\begin{hinweis}
Eine mögliche Antwort für a) ist eine Ungleichung für $P(A\cap B)$.
\end{hinweis}

\begin{loesung}
\begin{teilaufgaben}
\item 
Es gilt
\[
1\ge P(A\cup B)=P(A)+P(B)-P(A\cap B),
\]
Daraus kann man ableiten
\[
1-P(A)-P(B)=1-0.8-0.6=-0.4 \ge -P(A\cap B)
\]
oder
\[
0.4\le P(A\cap B).
\]
\item 
Wenn das Ereignis $\overline{A\cup B}$ mit Wahrscheinlichkeit $0$ eintritt,
dann tritt $A\cup B$ mit Wahrscheinlichkeit $1$ ein.
Dann folgt aus der Ein-/Ausschaltformel
\begin{align*}
P(A\cup B)&=P(A)+P(B)-P(A\cap B)\\
1&=0.8+0.6-P(A\cap B)\\
-0.4&=-P(A\cap B)\\
P(A\cap B)&=0.4
\qedhere
\end{align*}
\end{teilaufgaben}
\end{loesung}

