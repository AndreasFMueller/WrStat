Ein Test, der Allergie auf Katzenhaare diagnostizieren kann, 
zeigt bei 80\% der Personen, die tatsächlich eine Katzenallergie haben,
ein positives Resultat.
Bei denjenigen Personen, die keine Allergie haben, zeigt er trotzdem
in 10\% der Fälle eine Allergie an.
Die Allergie selbst ist nicht so verbreitet, nur etwa 1\% der Bevölkerung
haben sie.
\begin{teilaufgaben}
\item
Wie häufig zeigt der Test ein positives Resultat an?
\item
Wie gross ist die Wahrscheinlichkeit, dass eine Person mit positivem
Test tatsächlich Katzenhaarallergie haben?
\end{teilaufgaben}

\thema{bedingte Wahrscheinlichkeit}
\thema{Satz von Bayes}
\thema{totale Wahrscheinlichkeit}

\begin{loesung}
Wir verwenden die folgenden Ereignisse:
\begin{align*}
A&=\{\text{Person hat Allergie auf Katzenhaare}\}
\\
T&=\{\text{Test zeigt Katzenhaarallergie an}\}
\end{align*}
und Wahrscheinlichkeiten
\begin{align*}
P(T|A)&=0.8\\
P(T|\overline{A})&=0.1\\
P(A)&=0.01
\end{align*}
Damit lassen sich die Fragen jetzt beantworten.
\begin{teilaufgaben}
\item
Mit dem Satz von der totalen Wahrscheinlichkeit findet man
\begin{align*}
P(T) 
&=
P(T|A)P(A) + P(T|\overline{A}) P(\overline{A})
\\
&=
P(T|A)P(A) + P(T|\overline{A}) (1-P(A))
\\
&=
0.8\cdot 0.01 + 0.1 \cdot 0.99
= 0.107.
\end{align*}
\item
Gefragt ist $P(A|T)$, was man mit dem 
Satz von Bayes bestimmen kann:
\begin{align*}
P(A|T)
&=
\frac{P(A)}{P(T)}P(T|A)
=
\frac{0.01}{0.107} 0.8
=
0.074766.
\end{align*}
Der Test ist also eigentlich wertlos.
\qedhere
\end{teilaufgaben}
\end{loesung}

\begin{bewertung}
Ereignisse ({\bf E}) 1 Punkt,
Wahrscheinlichkeiten ({\bf W}) 1 Punkt,
Satz von der Totalen Wahrscheinlichkeit ({\bf T}) 1 Punkt,
Wahrscheinlichkeit $P(T)$ ({\bf P}) 1 Punkt,
Satz von Bayes ({\bf B}) 1 Punkt,
Wahrscheinlichkeit $P(A|T)$ ({\bf Q}) 1 Punkt.
\end{bewertung}

