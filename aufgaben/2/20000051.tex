In einer Stadt bieten zwei Taxiunternehmen Taxidienste an, die einen mit
grünen, die anderen mit blauen Taxis. 85\% der Taxis sind grün.
Ein Taxi ist in einen Unfall mit Fahrerflucht verwickelt.
Ein Zeuge sagt, das Taxi sei blau gewesen.
Welche Ereignisse bieten sich für die Analyse dieses Problems an?

\thema{Ereignis}

\begin{loesung}
Das Experiment besteht darin, ein Taxi auszuwählen und in einen Unfall
mit Fahrerflucht zu verwicklen.
\begin{align*}
\Omega&=\{\text{alle Taxis}\}\\
G&=\{\text{Taxi ist grün}\}\\
B&=\{\text{Taxi ist blau}\}\\
Z&=\{\text{Zeuge erkennt Farbe grün}\}\\
Y&=\{\text{Zeuge erkennt Farbe blau}\}
\qedhere
\end{align*}
\end{loesung}
