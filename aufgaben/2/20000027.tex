In einer Stadt bieten zwei Taxiunternehmen Taxidienste an, die einen mit
grünen, die anderen mit blauen Taxis. 85\% der Taxis sind grün.
Ein Taxi ist in einen Unfall mit Fahrerflucht verwickelt.
Ein Zeuge sagt, das Taxi sei blau gewesen.
In einem Test fand man heraus, dass der Zeuge in 80\% der Versuche
die Farbe richtig erkannte.
Wie gross ist die Wahrscheinlichkeit, dass das Taxi grün war?

\thema{bedingte Wahrscheinlichkeit}
\thema{totale Wahrscheinlichkeit}
\thema{Satz von Bayes}

\begin{loesung}
Wir verwenden folgende Ereignisse:
\begin{align*}
G&=\{\text{Taxi ist grün}\}\\
B&=\{\text{Taxi ist blau}\}\\
Z&=\{\text{Zeuge erkennt Farbe grün}\}\\
Y&=\{\text{Zeuge erkennt Farbe blau}\}
\end{align*}
Die Wahrscheinlichkeiten sind
\begin{align*}
P(G)&=0.85\\
P(B)&=P(\bar G)=0.15\\
P(Z|G)&=0.8 &&\Rightarrow& P(Y|G)&=1-P(Z|G)=0.2\\
P(Y|B)&=0.8 &&\Rightarrow& P(Z|B)&=1-P(Y|B)=0.2
\end{align*}
Aussserdem gilt natürlich $Z=\bar Y$, und daher $P(Z)=1-P(Y)$.

Die Aufgabe fragt nach der Wahrscheinlichkeit, dass das Taxi grün war,
obwohl der Zeuge es als blau identifiziert hat.
Dies ist die bedingte Wahrscheinlichkeit $P(G|Y)$. Nach dem Satz von
Bayes gilt
\[
P(G|Y)=P(Y|G)\frac{P(G)}{P(Y)}.
\]
Wir brauchen also zuerst $P(Y)$, was wir mit dem Satz über die totale
Wahrscheinlichkeit bestimmen können:
\begin{align*}
P(Y)
&=
P(Y|B)P(B) + P(Y|G)P(G)\\
&=
0.8 \cdot 0.15 + 0.2 \cdot 0.85\\
&=0.29
\end{align*}
Damit können wir jetzt die bedingte Wahrscheinlichkeit berechnen
\begin{align*}
P(G|Y)
&=
P(Y|G)\frac{P(G)}{P(Y)}
=
0.2\frac{0.85}{0.29}
=
0.586
\end{align*}
Die Wahrscheinlichkeit ist also fast 59\%, dass das Taxi trotz der
Zeugenaussage grün war.
\end{loesung}

\begin{bewertung}
Wahl zweckmässiger Ereignisse ({\bf E}) 1 Punkt,
Identifikation der gegeben Wahrscheinlichkeiten ({\bf I}) 1 Punkt,
Satz von der totalen Wahrscheinlichkeit ({\bf T}) 1 Punkt,
Berechnung von $P(Y)$ ({\bf Y}) 1 Punkt,
Satz von Bayes ({\bf B}) 1 Punkt,
Berechnung von $P(G|Y)$ ({\bf W}) 1 Punkt.
\end{bewertung}

\begin{diskussion}
Diese Aufgabe hat George Smoot in einem TEDx-Talk beschrieben
(\url{http://tedxtalks.ted.com/video/You-are-a-simulation-physics-ca}).
\end{diskussion}

