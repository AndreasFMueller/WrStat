In einem Behälter befinden sich $N=105$ Stecknadeln, wovon $M=23$ einen
schwarzen Kopf haben (Abbildung links).
Wie gross ist die Wahrscheinlichkeit dass beim zufälligen Ziehen von
$n=4$ Nadeln alle einen schwarzen Kopf haben (Abbildung rechts)?
\begin{center}
\begin{tikzpicture}[>=latex,thick]
\begin{scope}[xshift=-0.25\textwidth]
\node at (0,0) {\includeagraphics[width=0.48\textwidth]{nadeln.jpg}};
\end{scope}
\begin{scope}[xshift=0.25\textwidth]
\node at (0,0) {\includeagraphics[width=0.48\textwidth]{nadel4.jpg}};
\end{scope}
\end{tikzpicture}
\end{center}

% weiss    23
% schwarz  23
% gelb     14
% grün     22
% blau      8
% rot      15
% N     = 105


\begin{loesung}
Sei $X$ die Anzahl der Nadeln mit schwarzem Kopf in der Auswahl.
Die Wahrscheinlichkeit für das Ereignis $X=4$ wird durch die
hypergeometrische Verteilung gegeben:
\begin{align*}
P(X=m)
&=
\frac{ \binom{M}{m} \binom{N-M}{n-m} }{ \binom{N}{n} }
\\
P(X=4)
&=
\frac{ \binom{M}{4} \binom{N-M}{4-4} }{ \binom{N}{4} }
\\
&=
\frac{
\frac{M(M-1)(M-2)(M-3)}{1\cdot 2 \cdot 3\cdot 4}\cdot 1
}{
\frac{N(N-1)(N-2)(N-3)}{1\cdot 2 \cdot 3\cdot 4}
}
\\
&=
\frac{
M(M-1)(M-2)(M-3)
}{
N(N-1)(N-2)(N-3)
}
\\
&=
\frac{
23\cdot 22\cdot 21\cdot 20
}{
105\cdot 104\cdot 103\cdot 102
}
=
\frac{
212520
}{
114725520
}
=
0.00185242.
\end{align*}
Die Wahrscheinlichkeit dieses Ereignisses ist also nur etwa $0.18\%$.
\end{loesung}
