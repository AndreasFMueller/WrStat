Dem Statistischen Jahrbuch 2006 des deutschen statistischen Bundesamtes
entnimmt man folgende Zahlen "uber Ausl"anderkriminalit"at:
\begin{center}
\begin{tabular}{|l|r|r|}
\hline
Deliktgruppe&Anzahl Straftaten&davon Ausl"ander\\
\hline
Mord und Totschlag           &  5889&  1457\\
K"orperverletzung            &159512& 38128\\
Vergewaltigung               &  6868&  2037\\
Diebstahl                    &536198&111807\\
Raub, r"auberische Erpressung& 33988&  9786\\
Verbrechen gegen die Umwelt  & 11859&  1507\\
\hline
\end{tabular}
\end{center}
Die Ausl"ander stellen dabei einen Bev"olkerungsanteil von 8.8\%.
Sind die Eigenschaften, kriminell zu sein und Ausl"ander zu sein
unabh"angig?

\begin{loesung}
F"ur $\Omega$ nimmt man die Menge aller Einzelf"alle, also alle
Menschen in Deutschland. Die Ausl"ander bilden eine Teilmenge $A$,
8.8\%. F"ur jede Deliktgruppe bilden wir jetzt ein weiteres Ereignis,
zum Beispiel soll $V$ die Vergewaltiger enthalten. Die Wahrscheinlichkeit,
Vergewaltiger und Ausl"ander zu sein, ist dann $P(V\cap A)$. W"aren
$A$ und $V$ unabh"angig, m"usste $P(V\cap A)=P(V)\cdot P(A)$ gelten.
Aber:
\begin{align*}
P(A\cap V)&=\frac{2037}{N}\\
P(A)\cdot P(V)&=\frac{|A|}{N}\cdot\frac{6868}{N}=0.088\cdot \frac{6868}{N}=\frac{604.384}{N}
\end{align*}
Daraus liest man ab, dass die beiden Seiten wohl eher nicht "ubereinstimmen,
so dass die Ereignisse $V$ und $A$ wohl voneinander abh"angig sind.
\end{loesung}


