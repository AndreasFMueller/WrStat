Eine Karte wir aus einem Kartenspiel mit 52 Karten gezogen.
Dabei können folgende Ereignisse eintreten
\begin{align*}
R&=\{\text{rote Karte}\}
&
S&=\{\text{schwarze Karte}\}
&
H&=\{\text{Herz}\}
&
E&=\{\text{Ecke}\}
\\
K&=\{\text{König}\}
&
Q&=\{\text{Dame}\}
&
J&=\{\text{Bube}\}
&
A&=\{\text{Ass}\}
\end{align*}
\begin{teilaufgaben}
\item
Berechnen Sie $P(S|A)$
\item
Berechnen Sie
$P(Q|E)$
und
$P(Q|R)$.
\item
Wie gross ist die Wahrscheinlichkeit, dass eine rote Karte ein
König ist?
\item
Wie gross ist die Wahrscheinlichkeit, dass eine König-Karte rot ist?
\end{teilaufgaben}

\begin{loesung}
\begin{teilaufgaben}
\item
$
P(Q|E)
=
\frac{P(Q\cap E)}{P(E)}
=
\frac{1/52}{13/52}
=
\frac{1}{13}
=
0.\overline{076923}
$
\item
Die beiden bedingten Wahrscheinlichkeiten sind gleich:
\begin{align*}
P(Q|E)
&=
\frac{P(Q\cap E)}{P(E)} = \frac{1/52}{13/52} = \frac{1}{13}
\\
P(Q|R)
&=
\frac{P(Q\cap R)}{P(R)} = \frac{2/52}{26/52} = \frac{2}{26}
= \frac{1}{13}.
\end{align*}
\item
Gesucht ist die Wahrscheinlichkeit
\[
P(K|R)
=
\frac{P(K\cap R)}{P(R)}
=
\frac{2/52}{13/52}
=
\frac{2}{13}
=
0.\overline{153846}.
\]
\item
Gesucht ist die Wahrscheinlichkeit
\[
P(R|K)
=
\frac{P(R\cap K)}{P(K)}
=
\frac{2/52}{4/52}
=
\frac{2}{4}
=
\frac{1}{2}
=
0.5.
\qedhere
\]
\end{teilaufgaben}
\end{loesung}

