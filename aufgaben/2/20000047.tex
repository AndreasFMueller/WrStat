In der Diskussion um Drogen-Schnelltests wurde argumentiert, dass
Polizisten ein gutes Gespür dafür hätten, ob eine Person Drogen
konsumiert.
Nehmen Sie an, dass ein Polizist 90\% der Drogenkonsumenten erkennt,
und bei 5\% der Nichtkonsumenten meint, einen Drogenkonsumenten vor sich
zu haben.
Nehmen Sie weiter an, dass etwa 22\% tatsächlich konsumieren.
\begin{teilaufgaben}
\item
Wie wahrscheinlich ist, dass der Polizist eine Person des Drogenkonsums
verdächtigt?
\item
Wie gross ist die Wahrscheinlichkeit, dass eine Person tatsächlich Drogen
konsumiert, wenn der Polizist dies glaubt.
\end{teilaufgaben}

\thema{Satz von Bayes}
\thema{bedingte Wahrscheinlichkeit}
\thema{totale Wahrscheinlichkeit}

\begin{loesung}
Wir betrachten die Ereignisse 
\begin{align*}
D&=\{\text{konsumiert Drogen}\}
\\
V&=\{\text{vom Polizisten des Drogenkonsums verdächtigt}\}
\end{align*}
Aus dem Text lesen wir die folgenden Wahrscheinlichkeiten ab:
\begin{align*}
P(V|D)
&=
0.9
\\
P(V|\overline{D})
&=
0.05
\\
P(D)
&=
0.22
\end{align*}
\begin{teilaufgaben}
\item
Die Wahrscheinlichkeit, dass der Polizist eine Person des Drogenkonsums
verdächtigt, ist nach dem Satz von der totalen Wahrscheinlichkeit
\begin{align*}
P(V)
&=
P(V|D)\,P(D) + P(V|\overline{D})\,P(\overline{D})
=
P(V|D)\,P(D) + P(V|\overline{D})\,(1-P(D))
\\
&=
0.9\cdot 0.22 + 0.05\cdot 0.78
=
0.237.
\end{align*}
\item
Gesucht ist $P(D|V)$.
Nach dem Satz von Bayes gilt
\begin{align*}
P(D|V)
&=
P(V|D)\frac{P(D)}{P(V)}
=
0.9\cdot\frac{0.22}{0.237}
=
0.835.
\end{align*}
In einem von sechs Fällen ist der Verdacht des Polizisten also falsch.
\qedhere
\end{teilaufgaben}
\end{loesung}

\begin{bewertung}
Ereignisse ({\bf E}) 1 Punkt,
bedingte Wahrscheinlichkeiten ({\bf C}) 1 Punkt,
Satz von Bayes ({\bf B}) 1 Punkt,
Satz von der totalen Wahrscheinlichkeit ({\bf T}) 1 Punkt,
Wahrscheinlichkeit $P(V)$ ({\bf V}) 1 Punkt,
Wahrscheinlichkeit $P(D|V)$ ({\bf D}) 1 Punkt.
\end{bewertung}

