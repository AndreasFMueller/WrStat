An den National- und Ständeratswahlen 2019 haben sich 46\% der 
Wahlberechtigten beteiligt.
Die Gratiszeitung 20minuten hat eine Leserumfrage gemacht, nach der sich
84\% der an Wahlresultaten interessierten Leser an der Wahl beteiligt haben.
Offenbar sind Wahlteilnahme und Interesse an Wahlresultaten nicht unabhängig.
Nehmen Sie an, dass alle Wahlteilnehmer auch an den Resultaten interessiert
sind.
\begin{teilaufgaben}
\item
Welcher Anteil der Bevölkerung interessiert sich für Wahlresultate?
\item 
Welcher Anteil der Personen, die nicht an der Wahl teilgenommen haben,
interessiert sich für die Resultate?
\end{teilaufgaben}

\begin{loesung}
Wir verwenden die Ereignisse
\begin{align*}
W&=\{\text{hat sich an den Wahlen beteiligt}\}
\\
I&=\{\text{interessiert sich für die Wahlresultate}\}.
\end{align*}
Aus dem Text wissen wir, dass
\begin{align*}
P(W)&=0.46&P(W|I)&=0.84\\
    &     &P(I|W)&=1.
\end{align*}
\begin{teilaufgaben}
\item
Wir verwenden den Satz von Bayes und lösen nach $P(I)$ auf:
\[
P(W|I)=P(I|W)\frac{P(W)}{P(I)}
\qquad\Rightarrow\qquad
P(I)=P(W) \frac{P(I|W)}{P(W|I)}
=
0.5467.
\]
\item
Wir wenden den Satz von Bayes erneut an, diesmal auf $P(I|\overline{W})$ und
verwenden die Rechenregeln und das Resultat von a):
\[
P(I|\overline{W})
=
P(\overline{W}|I)\frac{P(I)}{P(\overline{W})}
=
(1-P(W|I))\frac{P(I)}{1-P(W)} = 0.16226.
\qedhere
\]
\end{teilaufgaben}
\end{loesung}

\begin{bewertung}
Wahl geeigneter Ereignisse ({\bf E}) 1 Punkt,
Bestimmung der Bedingungen Wahrscheinlichkeiten ({\bf C}) 1 Punkt,
\begin{teilaufgaben}
\item
Satz von Bayes ({\bf B}) 1 Punkt,
Wahrscheinlichkeit ({\bf B}) 1 Punkt,
\item
Negation ({\bf N}) 1 Punkt,
Wahrscheinlichkeit ({\bf W}) 1 Punkt.
\end{teilaufgaben}
\end{bewertung}
