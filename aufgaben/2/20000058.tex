In einer Kunstausstellung bieten die Künstler Salvador D., Vincent G.~und
Pablo P.~Bilder zum Verkauf an.
Dabei hat Salvador die Hälfte aller in der Ausstellung gezeigten
Bilder eingereicht und Pablo einen Drittel.
Über die Wahrscheinlichkeit, dass ein Bild verkauft wird, weiss
man folgendes: bei den Werken von Salvador ist die Wahrscheinlichkeit
4\%, bei den Werken von Vincent ist sie 6\%, bei Pablos Werken ist sie 3\%.
\begin{teilaufgaben}
\item Wie gross ist die Wahrscheinlichkeit, dass ein Bild verkauft wird?
\item Wie gross ist die Wahrscheinlichkeit, dass ein verkauftes
Bild von Salvador ist?
\end{teilaufgaben}

\begin{loesung}
Das Experiment besteht darin, ein Bild zufällig auszuwählen und dann
festzustellen, von welchem Künstler es stammt und ob es verkauft wurde.
Wir haben die folgenden Ereignisse:
\begin{align*}
V&=\{\text{Werke von Vincent~G.}\}\\
S&=\{\text{Werke von Salvador~D.}\}\\
P&=\{\text{Werke von Pablo~P.}\}\\
A&=\{\text{verkauft}\}
\end{align*}
Die bekannten Wahrscheinlichkeiten sind
\begin{align*}
P(A|S)&=0.04 & P(S)&=\frac12\\
P(A|P)&=0.03 & P(P)&=\frac13\\
P(A|V)&=0.06 & P(V)&=1-P(S)-P(V) = \frac16
\end{align*}
\begin{teilaufgaben}
\item
Zu bestimmen ist $P(A)$.
Nach dem Satz von der totalen Wahrscheinlichkeit ist
\begin{align*}
P(A)
&=
P(A|S)P(S)
+
P(A|V)P(V)
+
P(A|P)P(P)
\\
&=
0.04\cdot 0.5
+
0.06\cdot 0.1666
+
0.03\cdot 0.3333
\\
&=0.039995.
\end{align*}
\item
Gesucht ist die Wahrscheinlichkdit $P(S|A)$, dies kann mit dem 
Satz von Bayes geschehen:
\begin{align*}
P(S|A)
&=
P(A|S)\frac{P(S)}{P(A)}
=
0.04\frac{0.5}{0.039995}
=
\frac{0.020}{0.039995}
=
0.5001
\approx
50.01\%.
\qedhere
\end{align*}
\end{teilaufgaben}
\end{loesung}

\begin{bewertung}
\begin{teilaufgaben}
\item
Ereignisse ({\bf E})  1 Punkt,
Bedingte Wahrscheinlichkeiten ({\bf C}) 1 Punkt,
Satz von der totalen Wahrscheinlichkeit ({\bf T}) 1 Punkt,
Wahrscheinlichkeit von $A$ ({\bf A}) 1 Punkt.
\item
Satz von Bayes ({\bf B}) 1 Punkt,
Wahrscheinlichkeit ({\bf W}) 1 Punkt.
\end{teilaufgaben}
\end{bewertung}
