Die Fluggesellschaften empfehlen uns, die Sicherheitsvorschriften
genau zu lesen, denn $80\%$ derjenigen, die einen Fluzeugcrash
"uberlebt haben, haben die Sicherheitsvorschriften gelesen.
Das Argument klingt im ersten Moment "uberzeugend, bis man
sich fragt, was denn diejenigen getan haben, die nicht
"uberlebt haben. Oder wieviele denn "uberlebt haben.
Nehmen wir an, die Wahrscheinlichkeit, dass ein
beim Crash gestorbener Fluggast die Vorschriften gelesen hat, sei
$q$.
Aus der Statistik\footnote{Zum Beispiel von \url{http://planecrashinfo.com}}
ist ausserdem bekannt dass ein Passagier bei einem Crash
mit Wahrscheinlichkeit $0.75$ umkommt.
\begin{teilaufgaben}
\item
Wie gross ist die Wahrscheinlichkeit, dass ein Fluggast "uberlebt, der die
Sicherheitsvorschriften gelesen hat?
\item
Ein Kritiker behauptet, das Lesen der Sicherheitsvorschriften
habe "uberhaupt keinen Einfluss auf die "Uberlebenschancen.
Falls dies stimmt, welcher Anteil der Flugg"aste hat dann die
Vorschriften gelesen?
\item
Ein Sprecher der Fluggesellschaften behauptet, es werde immer sichergestellt,
dass alle Passagiere die Sicherheitsvorschriften studieren w"urden.
Kann das stimmen?
\item
Gibt es Werte von $q$, die den Schluss nahelegen,
dass das Lesen der Sicherheitsvorschriften
die "Uberlebenschancen verringert?
\end{teilaufgaben}

\begin{loesung}
Offenbar haben wir es hier mit den zwei Ereignissen
\begin{align*}
V&=\{\text{Fluggast hat Vorschriften gelesen}\}
\\
U&=\{\text{Fluggast "uberlebt Crash}\}
\end{align*}
zu tun. Es gilt $P(V)=p$ nach Annahme, in der Aufgabe geben sind
\begin{align*}
P(V|U)&=0.8\\
P(V|\bar U)&=q\\
P(U)&=0.25\\
\end{align*}
Mit dem Satz "uber die totale Wahrscheinlichkeit bekommt man daraus auch
\begin{align*}
P(V)
&=
P(V|U)P(U)+P(V|\bar U)P(\bar U)
=
P(V|U)P(U)+P(V|\bar U)(1-P(U))
\\
&=
0.8\cdot0.25 +q\cdot 0.75
=0.2+ 0.75q
\end{align*}
\begin{teilaufgaben}
\item
Gesucht ist $P(U|V)$.
Nach dem Satz von Bayes ist
\begin{equation}
P(U|V)=P(V|U)\frac{P(U)}{P(V)}
=0.8 \frac{0.25}{0.2+0.75q}
=\frac{0.2}{0.2+0.75q}
\label{pq}
\end{equation}
\item
Der Kritiker sagt, dass $U$ und $V$ unabh"angig sind. Wenn dem so
ist, dann ist sowohl
$P(U)=P(U|V)=P(U|\bar V)$
als auch
$P(V)=P(V|U)=P(V|\bar U)$.
Daraus leiten wir ab $P(V)=P(V|U)=0.8=P(V|\bar U)=q$, 80\% der
Flugg"aste haben die Vorschriften gelesen.
\item
Dazu m"usste $P(V)=1$ sein, also
$0.2+0.75q=1$.
Aufl"osen nach $q$ ergibt $q=0.8/0.75=1.06666$, dies kann aber nicht
sein, da Wahrscheinlichkeiten nicht gr"osser als 1 werden k"onnen.
Der Sprecher der Fluggesellschaft hat die Wahrheit etwas gesch"ont.
Ist aber auch intuitiv ganz klar: wenn alle Passagiere die Vorschriften
gelesen h"atten, dann h"atten auch 100\% der Passagiere, die
"uberlebt haben, die Vorschriften gelesen, $P(V|U)$ m"usste also $1$
sein, nicht $0.8$.
\item
Die "Uberlebenschancen bei einem Crash sind 0.25, gefragt wird
also danach, ob die "Uberlebenschancen f"ur die ``Vorschriftenleser''
sogar kleiner werden k"onnen, ob also $P(U|V)<0.25$ sein kann. Wir
wissen bereits, wie gross
$P(U|V)$ ist:
\[
P(U|V)=\frac{1}{1+3.75q}<0.25
\quad
\Rightarrow
\quad
4<1+3.75q
\quad
\Rightarrow
\quad
0.8<q.
\]
Wenn also mehr als 80\% verstorbenen Flugg"aste die Vorschriften
gelesen haben, dann n"utzt Lesen der Vorschriften offenbar
nichts.
\qedhere
\end{teilaufgaben}
\end{loesung}

