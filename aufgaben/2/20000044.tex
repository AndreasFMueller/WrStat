Marie heiratet morgen.
Als Location hat sie einen vermeintlich wettersicheren Ort in 
der W"uste gew"ahlt, wo es nur an f"unf Tagen pro Jahr regnet.
Der Wetterbericht sagt f"ur morgen leider Regen voraus.
Wenn es regnet, dann sagt der Wetterbericht dies in 90\% der F"alle
korrekt voraus.
Wenn es nicht regnet, dann sagt der Wetterbericht in 10\% der F"alle
f"alschlicherweise Regen voraus.
Wie wahrscheinlich ist es, dass es an der Hochzeit von Marie regnet?


\begin{loesung}
Wir verwenden die folgenden Ereignisse:
\begin{align*}
V&=\{\text{Wettebericht sagt regen voraus}\}\\
R&=\{\text{Es regnet}\}
\end{align*}
Die folgenden Wahrscheinlichkeiten sind bekannt:
\begin{align*}
P(R)
&=
\frac{5}{365}
\\
P(V|R)
&=
0.9
\\
P(V|\overline{R})
&=
0.1
\end{align*}
Da bereits bekannt ist, dass der Wetterbericht Regen vorhersagt, ist
$P(R|V)$ gesucht.
Nach dem Satz von Bayes ist
\begin{equation}
P(R|V)
=
\frac{P(R)}{P(V)}P(V|R).
\label{20000044:bayes} 
\end{equation}
Darin sind $P(R)$ und $P(V|R)$ bereits bekannt, es muss noch $P(V)$ bestimmt
werden, was mit dem Satz von der totalen Wahrscheinlichkeit geschehen kann:
\begin{align*}
P(V)
&=
P(V|R)P(R)+P(V|\overline{R})P(\overline{R})
=
P(V|R)P(R)+P(V|\overline{R})(1-P(R))
=
0.9\cdot\frac{5}{365}+0.1\cdot\frac{360}{365}
\\
&=
0.1109589.
\end{align*}
Dies kann man jetzt in \eqref{20000044:bayes} einsetzen und erh"alt
\[
P(R|V)
=
\frac{5/365}{0.1109589}\cdot 0.9
=
0.111111.
\qedhere
\]
\end{loesung}

\begin{bewertung}
Ereignisse ({\bf E}) 1 Punkt,
Wahrscheinlichkeit $P(R)$ ({\bf R}) 1 Punkt,
Identifikation der bedingten Wahrscheinlichkeiten ({\bf B}) 1 Punkt,
Satz von Bayes ({\bf B}) 1 Punkt,
Satz von der totalen Wahrscheinlichkeit ({\bf T}) 1 Punkt,
Wahrscheinlichkeit ({\bf W}) 1 Punkt.
\end{bewertung}

