Ein Arzt wird zu einem kranken Kind gerufen.
Er weiss, dass eine Grippe ``umgeht''.
90\% der Kinder in dieser Gegend haben die Grippe.
Aber auch Masern sind im Moment recht verbreitet, 10\% der Kinder,
die er behandelt, sind betroffen.
Die Masern unterscheiden sich aber dadurch, dass sie oft mit
einem Ausschlag zusammen auftreten, in etwa 95\% der Fälle.
Aber auch bei der Grippe ist manchmal ein Ausschlag zu beobachten,
jedoch in nur 8\% der Grippefälle.
\begin{teilaufgaben}
\item
Wie gross ist die Wahrscheinlichkeit, dass ein Kind keinen Ausschlag
hat?
\item
Wie gross ist die Wahrscheinlichkeit, dass ein Kind mit Ausschlag doch
nur eine Grippe hat?
\end{teilaufgaben}

\begin{hinweis}
Sie dürfen annehmen, dass das Kind entweder Masern oder Grippe hat,
also nicht beides.
\end{hinweis}

\begin{loesung}
Wir verwenden die folgenden Ereignisse:
\begin{align*}
A&=\{\text{Ausschlag}\}
\\
G&=\{\text{Grippe}\}
\\[-2pt]
M=\overline{G}&=\{\text{Masern}\}.
\end{align*}
Aus der Aufgabenstellung lesen wir die folgenden Wahrscheinlichkeiten
ab:
\begin{align*}
P(G) &= 0.9
&
P(A|M) &= 0.95
&
P(\overline{A}|M) &= 0.05
\\
P(M) &= 1-P(G) = 0.1
&
P(A|G) &= 0.08
&
P(\overline{A}|G) &= 0.92
\end{align*}
\begin{teilaufgaben}
\item
Die Wahrscheinlichkeit $P(\overline{A})$ kann mit dem Satz von der totalen
Wahrscheinlichkeit berechnet werden:
\begin{align*}
P(\overline{A})
&=
P(\overline{A}|G) P(G) + P(\overline{A}|M) P(M)
=
0.92\cdot 0.9 + 0.05\cdot 0.1
=
0.833.
\end{align*}
\item
Es ist die bedingte Wahrscheinlichkeit $P(G|A)$ zu berechnen, dazu
verwendet man den Satz von Bayes:
\begin{align*}
P(G|A)
&=
P(A|G)\frac{P(G)}{P(A)}
=
P(A|G)\frac{P(G)}{1-P(\overline{A})}
=
0.08\frac{0.9}{0.166}
=
0.4337.
\end{align*}
Wegen der überwiegenden Zahl der Grippefälle ist die Diagnose von
Masern nur aufgrund des Ausschlages eher unsicher, man muss in etwa
43.4\% der Fälle mit einer Fehldiagnose rechnen.
\qedhere
\end{teilaufgaben}
\end{loesung}

\begin{bewertung}
Ereignisse ({\bf E}) 1 Punkt,
bedinge Wahrscheinlichkeiten ({\bf C}) 1 Punkt,
Satz von der totalen Wahrscheinlichkeit ({\bf T}) 1 Punkt,
Satz von Bayes ({\bf B}) 1 Punkt,
Wahrscheinlichkeit $P(A)$ ({\bf A}) 1 Punkt,
Wahrscheinlichkeit $P(G|A)$ ({\bf G}) 1 Punkt.
\end{bewertung}



