Keith Devlin hat dem Monty Hall Problem eine Kolumne auf www.maa.org
gewidmet (leider nicht mehr zug"anglich). Am Ende
fordert er den Leser heraus, sein Verst"andnis des Problems an
Hand der folgenden Variante zu beweisen:
\medskip

Finally, just to see how well you have done on this teaser, suppose
you are playing a seven door version of the game. You choose three
doors. Monty now opens three of the remaining doors to show you
that there is no prize behind it. He then says, ``Would you like to
stick with the three doors you have chosen, or would you prefer to
swap them for the one other door I have not opened?'' What do you
do? Do you stick with your three doors or do you make the 3 for 1
swap he is offering?

\begin{loesung}
Nat"urlich m"ussen wir wieder die Ereignisse $E$ und $F=\bar E$ unterscheiden,
wobei $E$ der Fall ist, dass hinter einer der drei T"uren der ersten Wahl
bereits der Preis verborgen ist.  Dieser Fall tritt mit Wahrscheinlichkeit
$P(E)=\frac37$ ein.
Die Gewinnwahrscheinlichkeit mit der Wechselstrategie ist
\begin{align*}
P(W)&=P(W|E)P(E)+P(W|F)P(F)\\
&=0\cdot \frac37+1\cdot\frac47=\frac47
\end{align*}
F"ur die sture Strategie gilt stattdessen
\begin{align*}
P(B)&=P(B|E)P(E)+P(B|F)P(F)\\
&=1\cdot \frac37+0\cdot\frac47=\frac37
\end{align*}
Auch hier ist also die Wechselstrategie mit $P(W)=\frac47$
erfolgreicher als $B$ mit $P(B)=\frac37$, obwohl man
nach dem Wechsel nur noch eine einzige T"ur zur Auswahl hat, f"ur
die die tr"ugerische Intuition nur eine Wahrscheinlichkeit von
$\frac17$ "ubrig hat.
\end{loesung}

