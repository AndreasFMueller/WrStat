Drei Maschinen $M_i$, $1\le i\le 3$, produzieren Bauteile, wobei $P(M_i)$
der Anteil der von der Maschine $M_i$ produzierten Bauteile ist.
Durch eine Kontrollmessung werden die Bauteile anschliessen in
drei Klassen eingeteilt: $A$, $B$ und Ausschuss $U$. Folgende statistischen
Informationen sind bekannt:
\begin{enumerate}
\item Insgesamt werden 10\% Ausschuss produziert.
\item Der Ausstoss an Klasse $B$ Teilen ist doppelt so gross wie
der Ausstoss an Klasse A Teilen.
\item Die Maschine $M_3$ ist zu wenig präzise, um überhaupt
Klasse $A$ Teile zu produzieren.
\item Die Maschine $M_1$ produziert nur 2\% Ausschuss und gleichviele
Klasse $A$ wie Klasse $B$ Teile.
\item Maschine $M_1$ ist doppelt so produktiv wie Maschine $M_2$, welche
wiederum doppelt so produktiv ist wie Maschine $M_3$.
\item Maschine $M_2$ produziert 4\% Ausschuss.
\end{enumerate}
Beantworten Sie damit die folgenden Fragen:
\begin{teilaufgaben}
\item
Wie gross ist der Gesamtausstoss von Kategorie $A$ Teilen?
\item
Wieviel Ausschuss produziert die Maschine $M_3$?
\item
Welchen Anteil haben die drei Maschinen an den Kategorie-$B$-Teilen?
\end{teilaufgaben}

\begin{loesung}
Wir übersetzen zunächst die Aussagen der Aufgabenstellung
in Wahrscheinlichkeitsaussagen. Dabei bezeichnen wir mit $M_k$ das
Ereignis, dass ein Teil auf der Maschine Nummer $k$ gefertigt
worden ist, und mit $A$, $B$ und $U$ die Ereignisse, dass das Teil
in die entsprechenden Klassen fällt.
\begin{enumerate}
\item $P(U)=0.1$                                %
\item $P(B)=2P(A)$                              %
\item $P(A|M_3)=0$
\item $P(A|M_1)=P(B|M_1)$, $P(U|M_1)=0.02$
\item $P(M_1)=2P(M_2)$, $P(M_2)=2P(M_3)$
\item $P(U|M_2)=0.04$                           % 
\end{enumerate}
In den Teilaufgaben werden die folgenden Grössen gesucht:
\begin{teilaufgaben}
\item $P(A)$
\item $P(U|M_3)$
\item $P(M_1|B)$, $P(M_2|B)$, $P(M_3|B)$
\end{teilaufgaben}

Aus den ersten zwei Gleichungen ermittelt man
\[
P(B)=0.6\qquad P(A)=0.3
\]
Damit lässt sich Teilaufgabe a) bereits lösen.

Aus der vierten Aussage folgt
\begin{align*}
1&=P(A|M_1)+P(B|M_1)+P(U|M_1)=2P(A|M_1)+0.02
\\
\Rightarrow\quad
P(A|M_1)&=P(B|M_1)=0.49.
\end{align*}
Aus der fünften Aussage folgt
\[
1=P(M_1)+P(M_2)+P(M_3)=7P(M_3)
\quad\Rightarrow\quad
P(M_3)=\frac17.
\]

Die Wahrscheinlichkeit von $U$ ist einerseits in der ersten Aussage
gegeben, andererseits kann man sie mit dem Satz über die totale
Wahrscheinlichkeit auch schreiben als
\begin{align*}
0.1=P(U)
&=P(U|M_1)P(M_1)+P(U|M_2)P(M_2)+P(U|M_3)P(M_3)
\\
&=0.02\cdot 4P(M_3)+0.04\cdot 2P(M_3)+P(U|M_3)P(M_3)
\\
7\cdot 0.1&=4\cdot 0.02+2\cdot 0.04+P(U|M_3)
\\
0.54&=P(U|M_3)
\end{align*}
Damit ist die Teilaufgabe b) gelöst.
Mit dem Satz über die totale Wahrscheinlichkeit für $A$ findet man
\begin{align*}
0.3=P(A)&=
P(A|M_1)P(M_1)
+
P(A|M_2)P(M_2)
+
P(A|M_3)P(M_3)
\\
&=
0.49\cdot 4P(M_3)
+
P(A|M_2)\cdot 2P(M_3)
\\
&=
(4\cdot 0.49
+
2P(A|M_2))P(M_3)
\\
7\cdot 0.3
&=4\cdot 0.49+2P(A|M_2)
\\
P(A|M_2)&=0.07
\end{align*}
Da nun $P(U|M_2)$ bereits bekannt ist, und zusammen mit $P(A|M_2)$
und $P(B|M_2)$ $1$ ergeben muss, folgt
\[
P(B|M_2)=1-P(A|M_2)-P(U|M_2)=1-0.07-0.04=0.89
\]
Ebenso kann man den Satz über die totale Wahrscheinlichkeit auch
für $B$ anwenden und findet
\begin{align*}
0.6=P(B)
&=
P(B|M_1)P(M_1)
+
P(B|M_2)P(M_2)
+
P(B|M_3)P(M_3)
\\
&=
0.49\cdot 4P(M_3)
+
P(B|M_2)\cdot 2P(M_3)
+
P(B|M_3)P(M_3)
\\
&=
(4\cdot 0.49
+
2P(B|M_2)
+
P(B|M_3))P(M_3)
\\
7\cdot 0.6
&=
4\cdot 0.49+2\cdot 0.89 +P(B|M_3)
\\
P(B|M_3)&=0.46
\end{align*}
Damit sind jetzt alle Wahrscheinlichkeiten $P(B|M_i)$, $P(B)$ und
$P(M_i)$ bekannt, und mit dem Satz von Bayes kann man auch $P(M_i|B)$
berechnen:
\begin{align*}
P(M_1|B)&=
P(B|M_1)\frac{P(M_1)}{P(B)}
\\
&=0.49\cdot\frac{4\cdot\frac17}{0.6}
=0.466666
\\
P(M_2|B)&=
P(B|M_2)\frac{P(M_2)}{P(B)}
\\
&=0.89\cdot\frac{2\cdot\frac17}{0.6}
=0.423810
\\
P(M_3|B)&=
P(B|M_3)\frac{P(M_3)}{P(B)}
\\
&=0.46\cdot\frac{\frac17}{0.6}
=0.109523
\end{align*}
Zur Kontrolle:
\[
0.466666+0.423810+0.109523=0.999999.
\qedhere
\]
\end{loesung}

