In einem Gerät sollen zwei Beschleunigungssensoren verwendet
werden. Ein AD22279, welcher einen etwas grösseren Messbereich von
$\pm37g$ hat, aber eine etwas beschränkte Grenzfrequenz von nur 400Hz.
Ergänzt werden soll er durch einen Dreiachssensor ADIS16240,
der weniger Strom braucht und ausserdem eine deutlich höhere
Grenzfrequenz von 1600Hz hat, dafür aber nur bis $\pm19g$ messen kann.
Das Datenblatt gibt für den ADIS16240 eine Standardabweichung von
$24\text{m}g$ an, beim AD22279 findet man $22\text{m}g$.
Um wieviele Prozent kann man den Fehler, gemessen mit
der Standardabweichung, maximal korrigieren, wenn man beide
Messwerte geeignet mittelt.

\thema{Varianz}

\begin{loesung}
Die beiden Sensoren liefern Werte $X_1$ und $X_2$, die
beide normalverteilt sind mit gleichem $\mu$, aber verschiedenen
Varianzen. Gesucht ist der gewichtete Mittelwert
\[
X=tX_1+(1-t)X_2
\]
so, dass die Varianz $\operatorname{var}(X)$ minimal wird.

Nach den Rechenregeln ist die Varianz von $X$:
\[
\operatorname{var}(X)=t^2\operatorname{var}(X_1)+(1-t)^2\operatorname{var}(X_2).
\]
Das Minimum finden wir durch Nullsetzen der Ableitung:
\begin{align*}
0=\frac{d}{dt}\operatorname{var}(X)
&=
2t\operatorname{var}(X_1)-2(1-t)\operatorname{var}(X_2)
\\
&=
2t(\operatorname{var}(X_1)+\operatorname{var}(X_2)) -2\operatorname{var}(X_2)
\\
\Rightarrow\qquad
t&=\frac{\operatorname{var}(X_2)}{\operatorname{var}(X_1)+\operatorname{var}(X_2)}
=\frac{\sigma_2^2}{\sigma_1^2+\sigma_2^2}
\\
1-t&=
\frac{\sigma_1^2}{\sigma_1^2+\sigma_2^2}
\end{align*}
Setzt man dies ein, erhält man die verbleibende Varianz
\begin{align*}
\operatorname{var}(X)
&=
\biggl(
\frac{\sigma_2^2}{\sigma_1^2+\sigma_2^2}
\biggr)^2
\sigma_1^2
+
\biggl(
\frac{\sigma_1^2}{\sigma_1^2+\sigma_2^2}
\biggr)^2
\sigma_2^2
\\
&=\frac{\sigma_1^2\sigma_2^4+\sigma_1^4\sigma_2^2}{(\sigma_1^2+\sigma_2^2)^2}
=\frac{\sigma_1^2\sigma_2^2(\sigma_2^2+\sigma_1^2)}{(\sigma_1^2+\sigma_2^2)^2}
\\
&=
\frac{\sigma_1^2\sigma_2^2}{\sigma_1^2+\sigma_2^2}
\end{align*}
Unter Verwendung der Zahlen aus der Aufgabenstellung
$\sigma_1=0.024$ und $\sigma_2=0.022$ findet man für
den minimalen Fehler
\[
\frac{\sigma_1^2\sigma_2^2}{\sigma_1^2+\sigma_2^2}=\frac{0.022^2\cdot 0.024^2}{0.022^2+0.024^2}=0.000263
\quad\Rightarrow\quad
\sigma
=0.016
\]
Man kann also den Fehler von $0.022\text{m}g$ auf
$0.016\text{m}g$ reduzieren, eine Reduktion um 27\%.
\end{loesung}

