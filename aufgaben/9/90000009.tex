Schon seit langer Zeit wird das Meeresniveau millimetergenau gemessen.
Natürlich ist ein grosser Teil der Schwankungen durch die Gezeiten
verursacht, die aber vollständig bekannt sind.
Es verbleiben einerseits zufällige Schwankunge von der
Grössenordnung von etwa 10\,mm sowie ein gleichmässiger Anstieg, verursacht
durch das Abschmelzen des Polareises und der Gletscher wegen des Klimawandels.

In erster Näherung können wir annehmen, dass der jährliche Anstieg $d$
der Höhe $h$ konstant ist.
Wegen der geringen Zunahme von Jahr zu Jahr kann man $d$ nicht direkt
messen sondern muss einen möglichst genauen Wert aus den Messdaten
für die Höhe herausfiltern.
Stellen Sie die Matrizen zusammen, die für den Kalman-Filter für diese
Filteraufgabe benötigt werden.

\thema{Kalman-Filter}

\begin{loesung}
Wir verwenden $h$ und $d$ als Systemvariablen.
Das System kann dann mit 
\[
x_{k+1}
=
\begin{pmatrix}
h_{k+1}\\
d_{k+1}
\end{pmatrix}
=
\begin{pmatrix}
h_k + d_k \Delta t\\
d_k
\end{pmatrix}
=
\underbrace{
\begin{pmatrix}
1&\Delta t\\
0&1
\end{pmatrix}}_{\displaystyle\varphi}
\begin{pmatrix}
h_k\\d_k
\end{pmatrix}
\]
modelliert werden.
Gemessen wird nur die Höhe, d.~h.~Messmatrix ist
\[
H=\begin{pmatrix}
1&0
\end{pmatrix}.
\]
Bekannt ist ausserdem die Messgenauigkeit, der Messfehler ist $1\,\text{mm}$,
also ist die Messfehlermatrix
\[
R=\begin{pmatrix}
1\text{mm}^2
\end{pmatrix}.
\]
Über die Systemfehler wissen wir aus der Aufgabenstellung nur wenig.
Die Systemfehlerkovarianzmatrix muss die Form
\[
Q=\begin{pmatrix}
\sigma_h^2&0\\
0&\sigma_d^2
\end{pmatrix}
\]
haben.
Die Aufgabe gibt Auskunft über $\sigma_h=10\,\text{mm}$, allerdings
finden wir nichts über $\sigma_d^2$.
\end{loesung}

\begin{bewertung}
Wahl der Zustandsvariablen ({\bf Z}) 1 Punkt,
Zeitentwicklungsmatrix ($\bm{\varphi}$) 1 Punkt,
Messmatrix ({\bf H}) 1 Punkt,
Messfehlermatrix ({\bf R}) 1 Punkt,
Systemkovarianz-Matrix Struktur ({\bf S}) 1 Punkt,
Varianzwert $\sigma_h^2$ ({\bf V}) 1 Punkt.
\end{bewertung}



