Über Jahrhunderte ist die Temperatur der Erde gemessen worden.
Eine Messgenauigkeit von $0.1^\circ \text{C}$ ist üblich.
Für unsere Betrachtungen gehen wir von der Tagesmitteltemperatur aus,
die die Meteorologen seit langer Zeit erfassen.
Diese schwankt über das Jahr.
Wir gehen von einer Schwankung von $\pm 10^\circ\text{C}$ aus,
obwohl Standorte mit kontinentalem Klima grössere
Schwankungen haben können.

Bis zum Beginn der industriellen Revolution waren die Sonneneinstrahlung
und die Ausstrahlung im Gleichgewicht, die Temperatur der Erde hat sich
nicht verändert.
Seit die Menschheit im grossen Stil $\text{CO}_2$ in die Atmosphäre
pumpt, hat sich dieses Gleichgewicht verschoben.
Die Ausstrahlung ist durch den Treibhauseffekt reduziert worden,
so dass sich die Atmosphäre langsam aufheizt.
Wir können den Zusammenhang zwischen Strahlung und Temperatur mit einer
einzigen Grösse $p$ beschreiben.
In einem Zeitintervall $\Delta t$ nimmt die Tempertur der Erde um
$p\cdot\Delta t$ zu.
$p=0$ bedeutet, dass sich die Temperatur der Atmosphäre nicht verändert,
dies ist der vorindustrielle Zustand.

Wir können $p$ nicht direkt messen.
Wir möchten daher einen Kalman-Filter designen, der $p$ aus den
Temperaturmessungen ausfiltert.
Stellen Sie die Matrizen zusammen, die für den Kalman-Filter benötigt werden,
soweit dies mit den Informationen aus der Aufgabenstellung möglich ist.

\thema{Kalman-Filter}

\begin{loesung}
Wir verwenden $T$ und $p$ als Systemvariablen.
Das System kann dann mit 
\[
x_{k+1}
=
\begin{pmatrix}
T_{k+1}\\p_{k+1}
\end{pmatrix}
=
\begin{pmatrix}
T_k + p_k\Delta t\\
p_k
\end{pmatrix}
=
\underbrace{
\begin{pmatrix}
1&\Delta t\\
0&1
\end{pmatrix}
}_{\displaystyle=\varphi}
\begin{pmatrix}T_k\\p_k\end{pmatrix}
\]
modelliert werden.
Gemessen wird nur die Temperatur, d.~h.~die Messmatrix ist
\[
H=\begin{pmatrix}1&0\end{pmatrix}.
\]
Bekannt ist ausserdem die Messgenauigkeit, der Messfehler ist $0.1\text{K}$,
also ist die Messfehlermatrix
\[
R=\begin{pmatrix}0.01\text{K}^2\end{pmatrix}.
\]
Über die Systemfehler wissen wir aus der Aufgabenstellung nur wenig.
Die Systemfehlerkovarianzmatrix muss die Form
\[
Q=\begin{pmatrix}
\sigma_T^2&0\\
0&\sigma_p^2
\end{pmatrix}
\]
haben.
Für $\sigma_T^2$ könnten wir schätzen, dass wir in unserem Modell die
Wetterentwicklung nicht modelliert haben, die Temperaturschwankungen
duch das Wetter müssen also im Systemfehler untergebracht werden.
Gemäss Aufgabenstellung ist $\sigma_T=10\text{K}$.
In der Aufgabenstellung haben wir keine Information über $\sigma_p^2$.
\end{loesung}

\begin{bewertung}
Wahl der Zustandsvariablen ({\bf Z}) 1 Punkt,
Zeitentwicklungsmatrix ($\bm{\varphi}$) 1 Punkt,
Messmatrix ({\bf H}) 1 Punkt,
Messfehlermatrix ({\bf R}) 1 Punkt,
Systemkovarianz-Matrix Struktur ({\bf S}) 1 Punkt,
Varianzwert $\sigma_T^2$ ({\bf T}) 1 Punkt.
\end{bewertung}

