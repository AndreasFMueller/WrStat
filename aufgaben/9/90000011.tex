Ein Drehteller dreht sich mit Variabler Winkelgeschwindigkeit $\omega$,
die langsam verändert werden kann, d.~h.~die Winkelbeschleunigung $\alpha$
ist klein.
Die aktuelle Position des Drehtellers wird durch den Winkel $\vartheta$ 
gegeben, der mit Hilfe eines Encoders ermittelt werden kann.
Stellen Sie die Systemmatrix $\varphi$ und die Messmatrix $H$ auf.

\begin{loesung}
Die kinematischen Gleichungen für die Drehung sind
\begin{align*}
\vartheta(t+\Delta t)
&=
\vartheta(t)
+
\omega(t)\Delta t
+
{\textstyle \frac12}
\alpha(t)\Delta t^2
\\
\omega(t+\Delta t)
&=
\phantom{\vartheta(t)+\mathstrut}%
\omega(t)\phantom{\Delta t}+\phantom{\textstyle\frac12}\alpha(t)\Delta t
\\
\alpha(t+\Delta t)
&=
\phantom{\vartheta(t)+\omega(t)\Delta t + \textstyle\frac12}\alpha(t).
\end{align*}
Für einen Zustandsvektor, der $\vartheta(t)$, $\omega(t)$ und $\alpha(t)$ enthält,
lauten diese in Matrixform
\[
\begin{pmatrix}
\vartheta(t+\Delta t)\\
\omega(t+\Delta t)\\
\alpha(t+\Delta t)
\end{pmatrix}
=
\begin{pmatrix}
1 & \Delta t & \frac12 \Delta t^2 \\
0 &    1     & \Delta t \\
0 &    0     & 1
\end{pmatrix}
\begin{pmatrix}
\vartheta(t)\\
\omega(t)\\
\alpha(t)
\end{pmatrix}.
\]
Diese Gleichungen drücken die Tatsache, dass sich $\omega$ langsam ändert dadurch
aus, dass $\alpha(t)$ konstant ist.
Veränderungen von $\alpha(t)$ werden also durch Modellfehler abgebildet.

Die Messmatrix ermittelt $\vartheta(t)$ aus dem Zustandsvektor, somit ist
\[
\vartheta(t)
=
\begin{pmatrix}1&0&0\end{pmatrix}
\begin{pmatrix}
\vartheta(t)\\
\omega(t)\\
\alpha(t)
\end{pmatrix}
\qquad\Rightarrow\qquad
H
=
\begin{pmatrix}1&0&0\end{pmatrix}.
\qedhere
\]
\end{loesung}


