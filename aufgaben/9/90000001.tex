Ein weisser Zylinder wird auf der Aussenseite markiert, und zwar
am oberen und unteren Rand mit einem schwarzen Kreis um den Zylinder
herum und dazwischen mit einer Spirale, die sich genau einmal
um den Zylinder windet, ähnlich wie bei der Redstone Trägerrakete von
Explorer 1, dem ersten amerikanischen Satelliten:
\begin{center}
\includeagraphics[width=2cm]{explorer.jpg}
\end{center}
Dann wird der Zylinder in Drehung um seine Längsachse versetzt.
Auch aus grosser Distanz kann jetzt durch Ausmessen der Position
der schwarzen Spirale zwischen den beiden Kreisen der aktuelle Drehwinkel
ausgemessen werden. Leider erlauben die unscharfen Bilder nur
eine fehlerbehaftete Bestimmung des Winkels.
\begin{teilaufgaben}
\item
Für ein Experiment
möchte man die aktuelle Winkelgeschwindigkeit wissen.
Stellen sie die Matrizen eines Kalmanfilters zusammen, der
diesen zu Schätzen erlaubt.
\item
Die Kamera, mit der der Zylinder verfolgt wird, hat natürlich eine
begrenzte Auflösung. Bei grosser Entfernung ist der Abstand der
beiden Kreise am oberen und unteren Rand des Zylinders nur 60 Pixel.
Bestimmen Sie Näherungsweise die Matrix der Messfehlerkovarianzen.
\end{teilaufgaben}

\begin{loesung}
\begin{teilaufgaben}
\item
Es geht um ein System, bei dem man konstante Drehgeschwindigkeit annehmen
kann. Der Azimutwinkel $\varphi$ wird also aus der Winkelgeschwindigkeit
$\omega$ und der Zeit $t$ berechnet werden können:
\[
\varphi=\omega t
\]
Als Systemvariablen muss man also $\varphi$ und $\omega$ verwenden,
die man in einem Vektor
\[
\begin{pmatrix}\varphi\\\omega\end{pmatrix}
\]
zusammenfassen kann. Der Winkel zu einem um $\Delta t$ späteren Zeitpunkt
$\varphi+\omega\Delta t$, die Entwicklungsgleichungen sind also
\[
\begin{pmatrix}\varphi_{k+1}\\\omega_{k+1}\end{pmatrix}
=\begin{pmatrix}1&\Delta t\\0&1\end{pmatrix}
\begin{pmatrix}\varphi_k\\\omega_k\end{pmatrix}
\]
Bei der Messung kann nur der Winkel ermittel werden, die Messmatrix ist
also
\[
H=\begin{pmatrix}1&0\end{pmatrix}
\]
\item
Es stehen nur 60 Pixel zur Verfügung, mit denen Winkel zwischen $0$ und
und $2\pi$ unterschieden werden können müssen.
Der Abstand zwischen den Winkeln, die aus der Pixelposition abgeleitet
werden können, ist also $\frac{2\pi}{60}$. Zur Pixelposition $x$ gehören
alle Winkel im Interval $[2\pi x/60-\delta, 2\pi x/60+\delta]$ mit
\[
\delta=\frac{\pi}{60}.
\]
Die Winkel sind also in einem Interval der Länge $\frac{2\pi}{60}$ gleichverteilt,
somit haben Sie die Varianz
\[
\varrho^2=\frac1{12}\left(\frac{2\pi}{60}\right)^2.
\]
Die Messfehlerkovarianzmatrix besteht genau aus diesem einen Element.
\qedhere
\end{teilaufgaben}
\end{loesung}

