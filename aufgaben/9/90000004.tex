Während eines Spazierganges wurden von einem GPS-Empfänger die
$(x,y)$-Positionsdaten im File \texttt{gps.csv} aufgezeichnet.
In einer frühren Messung wurde ermittelt, dass der mittlere Positionsfehler
des GPS-Empfängers bei ca 3m liegt.
Ermitteln Sie zu jedem Zeitpunkt die bestmögliche Schätzung 
der Geschwindigkeit.

Die Daten für diese Aufgabe wurden von Martin Stypinski beigetragen.

\thema{Kalman-Filter}

\begin{loesung}
Da die mittlere Geschwindigkeit eines Spaziergängers bei etwa 1m/s liegt,
würde eine numerische Ableitung der Rohdaten von den Messfehlern dominiert,
man könnte keine sinnvollen Geschwindigkeitswerte ermitteln.
Andererseits kann man auch nicht einfach nur den ersten und letzten
Datenpunkt nehmen und die mittlere Geschwindigkeit berechnen, weil
sich die Richtung ja ändern könnte.

Ein Kalman-Filter ist in der Lage, Position und Geschwindigkeit aus
den Positionsdaten zu filtern. Dazu muss zunächst ein Modell formuliert
werden. Ein Spaziergang spielt sich mit mehr oder weniger konstanter
Geschwindigkeit ab, das Modell darf also annehmen, dass die Geschwindigkeit
konstant ist. Als Zustandsvariablen für das Modell kann man also die
zweidimensionalen Positionen und die zweidimensionalen Geschwindigkeit
in einem vierdimensionalen Vektor kombinieren.
Die Zeitentwicklung kann durch die Matrix
\begin{equation}
x_{k+1}=\varphi x_k=
\begin{pmatrix}
1&\Delta t & 0 & 0       \\
0&    1    & 0 & 0       \\
0&    0    & 1 & \Delta t\\
0&    0    & 0 & 1
\end{pmatrix}
\end{equation}
beschrieben werden. Die Systemfehler beschreiben das Ausmass, in
welchem die tatsächliche Zeitentwicklung von der modellierten
Zeitwentwicklung abweicht.
Für die Geschwindigkeit ist das die zu erwartende Schwankung der
Geschwindigkeit während des Spaziergangs. Wenn der Spaziergänger
alle paar Schritte stehen bleibt, dann ist die halbe Geschwindigkeit
die mittlere Geschwindigkeitsabweichung. Wenn stehenbleiben selten
ist, wird der Systemfehler der Geschwindigkeit entsprechend kleiner.
Der Positionsfehler setzt sich zusammen aus einem Fehler, der vom
Fehler der Geschwindigkeit herrührt, und einem Fehler, der damit
zusammenhängt, dass während eines Zeitschrittes die Geschwindigkeit
sich ändern kann. Eine solche Positionsänderung ist von
der Grössenordnung $\frac12a\Delta t^2$. Die Beschleunigung kann nicht
mehr als $v/\Delta t$, sein, also ist der Positionsfehler von der
Grössenordnung $\frac12v \Delta t$. Damit können wir die Systemfehlermatrix
als
\[
Q=
\begin{pmatrix}
\sigma_1^2&0&0&0\\
0&\sigma_2^2&0&0\\
0&0&\sigma_1^2&0\\
0&0&0&\sigma_2^2
\end{pmatrix}
=
\begin{pmatrix}
1\text{m}^2&0&0&0\\
0&1\left(\frac{\text{m}}{\text{s}}\right)^2&0&0\\
0&0&1\text{m}^2&0\\
0&0&0&1\left(\frac{\text{m}}{\text{s}}\right)^2
\end{pmatrix}
\]
Die Werte in der Matrix $Q$ sind naturgemäss etwas unsicher, sie können
später dazu benutzt werden, den Filter zu tunen.

Der GPS-Empfänger liefert nur die Position, also ist die Messmatrix
\[
H=
\begin{pmatrix}
1&0&0&0\\
0&0&1&0
\end{pmatrix}.
\]
Die Messfehler sind ebenfalls bekannt, die $R$-Matrix ist
\[
R=\begin{pmatrix}
\varrho^2&0\\
0&\varrho^2
\end{pmatrix}.
\]
Damit ist der Messprozess vollständig beschrieben.

Mit diesen Daten kann man jetzt den Kalman-Filter konstruieren, die
Formeln aus dem Skript lassen sich ohne "Anderung übernehmen.
Der Vorhersage-Schritt ist
\begin{align*}
\hat x_{k+1|k}&=\varphi \hat x_k
\\
P_{k+1|k}&=\varphi P_k\varphi^t+Q.
\end{align*}
Für den Korrektur-Schritt muss man zunächst die Kalman-Matrix 
berechnen, daraus kann man dann die korrigierte Schätzung sowie
ihren Fehler ermitteln:
\begin{align*}
K_{k+1}&=P_{k+1|k}H^t(HP_{k+1|k}H^t+R)^{-1}
\\
\hat x_{k+1}&=
\hat x_{k+1|k} - K_{k+1}(H\hat x_{k+1|k}-z_{k+1})
\\
P_{k+1}&=(E-K_{k+1}H)P_{k+1|k}
\end{align*}
Die Berechnung der inversen Matrix in der Formel für $K_{k+1}$ ist
unproblematisch, da es sich bei $HP_{k+1|k}H^t$ um eine $1\times 1$-Matrix
handelt.
Diesen Kalman-Filter kann man sehr einfach in Octave implementieren:
{\small
\verbatimainput{kalman.m.expand}
}
\begin{figure}
\includeagraphics[width=\hsize]{positions.pdf}
\caption{Gefilterte Positionen}
\end{figure}
\begin{figure}
\includeagraphics[width=\hsize]{velocities.pdf}
\caption{Gefilterte Geschwindigkeiten}
\end{figure}
\end{loesung}


