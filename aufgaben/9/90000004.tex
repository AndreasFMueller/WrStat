W"ahrend eines Spazierganges wurden von einem GPS-Empf"anger die
$(x,y)$-Positionsdaten im File \texttt{gps.csv} aufgezeichnet.
In einer fr"uhren Messung wurde ermittelt, dass der mittlere Positionsfehler
des GPS-Empf"angers bei ca 3m liegt.
Ermitteln Sie zu jedem Zeitpunkt die bestm"ogliche Sch"atzung 
der Geschwindigkeit.

\begin{loesung}
Da die mittlere Geschwindigkeit eines Spazierg"angers bei etwa 1m/s liegt,
w"urde eine numerische Ableitung der Rohdaten von den Messfehlern dominiert,
man k"onnte keine sinnvollen Geschwindigkeitswerte ermitteln.
Andererseits kann man auch nicht einfach nur den ersten und letzten
Datenpunkt nehmen und die mittlere Geschwindigkeit berechnen, weil
sich die Richtung ja "andern k"onnte.

Ein Kalman-Filter ist in der Lage, Position und Geschwindigkeit aus
den Positionsdaten zu filtern. Dazu muss zun"achst ein Modell formuliert
werden. Ein Spaziergang spielt sich mit mehr oder weniger konstanter
Geschwindigkeit ab, das Modell darf also annehmen, dass die Geschwindigkeit
konstant ist. Als Zustandsvariablen f"ur das Modell kann man also die
zweidimensionalen Positionen und die zweidimensionalen Geschwindigkeit
in einem vierdimensionalen Vektor kombinieren.
Die Zeitentwicklung kann durch die Matrix
\begin{equation}
x_{k+1}=\varphi x_k=
\begin{pmatrix}
1&\Delta t & 0 & 0       \\
0&    1    & 0 & 0       \\
0&    0    & 1 & \Delta t\\
0&    0    & 0 & 1
\end{pmatrix}
\end{equation}
beschrieben werden. Die Systemfehler beschreiben das Ausmass, in
welchem die tats"achliche Zeitentwicklung von der modellierten
Zeitwentwicklung abweicht.
F"ur die Geschwindigkeit ist das die zu erwartende Schwankung der
Geschwindigkeit w"ahrend des Spaziergangs. Wenn der Spazierg"anger
alle paar Schritte stehen bleibt, dann ist die halbe Geschwindigkeit
die mittlere Geschwindigkeitsabweichung. Wenn stehenbleiben selten
ist, wird der Systemfehler der Geschwindigkeit entsprechend kleiner.
Der Positionsfehler setzt sich zusammen aus einem Fehler, der vom
Fehler der Geschwindigkeit herr"uhrt, und einem Fehler, der damit
zusammenh"angt, dass w"ahrend eines Zeitschrittes die Geschwindigkeit
sich "andern kann. Eine solche Positions"anderung ist von
der Gr"ossenordnung $\frac12a\Delta t^2$. Die Beschleunigung kann nicht
mehr als $v/\Delta t$, sein, also ist der Positionsfehler von der
Gr"ossenordnung $\frac12v \Delta t$. Damit k"onnen wir die Systemfehlermatrix
als
\[
Q=
\begin{pmatrix}
\sigma_1^2&0&0&0\\
0&\sigma_2^2&0&0\\
0&0&\sigma_1^2&0\\
0&0&0&\sigma_2^2
\end{pmatrix}
=
\begin{pmatrix}
1\text{m}^2&0&0&0\\
0&1\left(\frac{\text{m}}{\text{s}}\right)^2&0&0\\
0&0&1\text{m}^2&0\\
0&0&0&1\left(\frac{\text{m}}{\text{s}}\right)^2
\end{pmatrix}
\]
Die Werte in der Matrix $Q$ sind naturgem"ass etwas unsicher, sie k"onnen
sp"ater dazu benutzt werden, den Filter zu tunen.

Der GPS-Empf"anger liefert nur die Position, also ist die Messmatrix
\[
H=
\begin{pmatrix}
1&0&0&0\\
0&0&1&0
\end{pmatrix}.
\]
Die Messfehler sind ebenfalls bekannt, die $R$-Matrix ist
\[
R=\begin{pmatrix}
\varrho^2&0\\
0&\varrho^2
\end{pmatrix}.
\]
Damit ist der Messprozess vollst"andig beschrieben.

Mit diesen Daten kann man jetzt den Kalman-Filter konstruieren, die
Formeln aus dem Skript lassen sich ohne "Anderung "ubernehmen.
Der Vorhersage-Schritt ist
\begin{align*}
\hat x_{k+1|k}&=\varphi \hat x_k
\\
P_{k+1|k}&=\varphi P_k\varphi^t+Q.
\end{align*}
F"ur den Korrektur-Schritt muss man zun"achst die Kalman-Matrix 
berechnen, daraus kann man dann die korrigierte Sch"atzung sowie
ihren Fehler ermitteln:
\begin{align*}
K_{k+1}&=P_{k+1|k}H^t(HP_{k+1|k}H^t+R)^{-1}
\\
\hat x_{k+1}&=
\hat x_{k+1|k} - K_{k+1}(H\hat x_{k+1|k}-z_{k+1})
\\
P_{k+1}&=(E-K_{k+1}H)P_{k+1|k}
\end{align*}
Die Berechnung der inversen Matrix in der Formel f"ur $K_{k+1}$ ist
unproblematisch, da es sich bei $HP_{k+1|k}H^t$ um eine $1\times 1$-Matrix
handelt.
Diesen Kalman-Filter kann man sehr einfach in Octave implementieren:
{\small
\verbatimainput{kalman.m.expand}
}
\begin{figure}
\includeagraphics[width=\hsize]{positions.pdf}
\caption{Gefilterte Positionen}
\end{figure}
\begin{figure}
\includeagraphics[width=\hsize]{velocities.pdf}
\caption{Gefilterte Geschwindigkeiten}
\end{figure}
\end{loesung}


