In der Vorlesung wurde gezeigt, dass die Geschwindigkeit der "Uberschallrakete
``Chicago'' bestimmt werden konnte. Dazu wurde ein Messger"at gebaut,
welches mit Hilfe eines Kalman-Filters aus der Beschleunigung die
Geschwindigkeit bestimmt hat. Dabei wurde ein Zeitschritt von $\Delta t=0.01s$
verwendet.
\begin{teilaufgaben}
\item Modellieren Sie das System.
"Uberlegen Sie sich plausible Werte f"ur die Systemfehler $Q$.
Gehen Sie von einer Beschleunigungsunsicherheit von $1g$
aus.
\item Modellieren Sie den Messprozess. Verwenden Sie den Wert $1g$ f"ur
den Messfehler.
\item Stellen Sie die Kalman-Filter-Formeln f"ur Pr"adiktor- und
Korrektor-Schritt zusammen.
\item Verwenden Sie die Daten im File {\tt data.csv} als Input zu einem
Kalman-Filter, den Sie zum Beispiel in Octave implementieren k"onnen.
\end{teilaufgaben}

\begin{loesung}
\begin{teilaufgaben}
\item Die Modellierung wurde schon in der Vorlesung kurz gezeigt:
\[
x_k=\begin{pmatrix}v_k\\a_k\end{pmatrix},\qquad
\varphi_k=\begin{pmatrix}1&\Delta t\\0&1\end{pmatrix}.
\]
Die Matrix der Systemfehler enth"alt ungef"ahre Gr"ossen f"ur die Dinge,
die vernachl"assigt worden sind. Das Modell nimmt an, dass die
Beschleunigung konstant ist.
Das ist zwar weitgehend richtig, aber es gibt Schwankungen durch
nicht konstanten Abbrand und durch die "Anderung der Masse der
Rakete (die Beschleunigung nimmt zu). Erfahrungsgem"ass ist die
Massenabnahme der dominante Beitrag, die Beschleunigung "andert sich
dabei um ca $1g$ im Laufe des Abbrandes des Motors (innert 1 Sekunde).
Die Beschleunigungsfehler bewegen sich also im Bereich von $1g$, entsprechend
ist die Geschwindigkeit pro Zeitschritt mit einer Unsicherheit von
$0.01\text{m/s}$ behaftet.
\[
Q=\begin{pmatrix}
(0.01g)^2&0\\
0&(1g)^2
\end{pmatrix}
\]
\item
Der Messprozess extrahiert aus dem Zustandsvektor die Beschleunigung
\[
H=\begin{pmatrix}0&1\end{pmatrix}.
\]
Da nur ein Messfehler vorhanden ist, n"amlich derjenige des
Beschleunigungssensors, ist die Matrix
\[
R=\begin{pmatrix}
(1g)^2
\end{pmatrix}.
\]
\item Pr"adiktorschritt:
\begin{align*}
\hat x_{k|k-1}&=
\begin{pmatrix}1&\Delta t\\0&1\end{pmatrix}
\hat x_{k-1}
\\
P_{k|k-1}&=
\begin{pmatrix}1&\Delta t\\0&1\end{pmatrix}P_k
\begin{pmatrix}1&0\\\Delta t&1\end{pmatrix}
+
Q_k
\end{align*}
Im Korrektorschritt wir die Kalman-Matrix gebraucht, eine
$2\times 1$-Matrix, also ein Spaltenvektor
\[
K=
\begin{pmatrix}k_1\\k_2\end{pmatrix}
\]
Korrektorschritt:
\begin{align*}
\hat x_k
&=
\begin{pmatrix}k_1\\k_2\end{pmatrix}z_k
+
\begin{pmatrix}
1&-k_1\\
0&1-k_2
\end{pmatrix}
\hat x_{k|k-1}
\\
P_k
&=
(I-KH)P_{k|k-1}(I-KH)^t+KRK^t
\\
&=
\begin{pmatrix}
1&-k_1\\
0&1-k_2
\end{pmatrix}
P_{k|k-1}
\begin{pmatrix}
1&0\\
-k_1&1-k_2
\end{pmatrix}
+
R
\begin{pmatrix}
k_1^2&k_1k_2\\
k_1k_2&k_2^2
\end{pmatrix}
\end{align*}
\item
Das folgende Octave-Programm f"uhrt die Berechnung durch.
\verbatimainput{kalman-expanded.m}
\qedhere
\end{teilaufgaben}
\end{loesung}

