In der Vorlesung wurde gezeigt, dass die Geschwindigkeit der "Uberschallrakete
``Chicago'' bestimmt werden konnte. Dazu wurde ein Messgerät gebaut,
welches mit Hilfe eines Kalman-Filters aus der Beschleunigung die
Geschwindigkeit bestimmt hat. Dabei wurde ein Zeitschritt von $\Delta t=0.01s$
verwendet.
\begin{teilaufgaben}
\item Modellieren Sie das System.
"Uberlegen Sie sich plausible Werte für die Systemfehler $Q$.
Gehen Sie von einer Beschleunigungsunsicherheit von $1g$
aus.
\item Modellieren Sie den Messprozess. Verwenden Sie den Wert $1g$ für
den Messfehler.
\item Stellen Sie die Kalman-Filter-Formeln für Prädiktor- und
Korrektor-Schritt zusammen.
\item Verwenden Sie die Daten im File {\tt data.csv} als Input zu einem
Kalman-Filter, den Sie zum Beispiel in Octave implementieren können.
\end{teilaufgaben}

\thema{Kalman-Filter}

\begin{loesung}
\begin{teilaufgaben}
\item Die Modellierung wurde schon in der Vorlesung kurz gezeigt:
\[
x_k=\begin{pmatrix}v_k\\a_k\end{pmatrix},\qquad
\varphi_k=\begin{pmatrix}1&\Delta t\\0&1\end{pmatrix}.
\]
Die Matrix der Systemfehler enthält ungefähre Grössen für die Dinge,
die vernachlässigt worden sind. Das Modell nimmt an, dass die
Beschleunigung konstant ist.
Das ist zwar weitgehend richtig, aber es gibt Schwankungen durch
nicht konstanten Abbrand und durch die "Anderung der Masse der
Rakete (die Beschleunigung nimmt zu). Erfahrungsgemäss ist die
Massenabnahme der dominante Beitrag, die Beschleunigung ändert sich
dabei um ca $1g$ im Laufe des Abbrandes des Motors (innert 1 Sekunde).
Die Beschleunigungsfehler bewegen sich also im Bereich von $1g$, entsprechend
ist die Geschwindigkeit pro Zeitschritt mit einer Unsicherheit von
$0.01\text{m/s}$ behaftet.
\[
Q=\begin{pmatrix}
(0.01g)^2&0\\
0&(1g)^2
\end{pmatrix}
\]
\item
Der Messprozess extrahiert aus dem Zustandsvektor die Beschleunigung
\[
H=\begin{pmatrix}0&1\end{pmatrix}.
\]
Da nur ein Messfehler vorhanden ist, nämlich derjenige des
Beschleunigungssensors, ist die Matrix
\[
R=\begin{pmatrix}
(1g)^2
\end{pmatrix}.
\]
\item Prädiktorschritt:
\begin{align*}
\hat x_{k+1|k}&=
\begin{pmatrix}1&\Delta t\\0&1\end{pmatrix}
\hat x_{k}
\\
P_{k+1|k}&=
\begin{pmatrix}1&\Delta t\\0&1\end{pmatrix}P_k
\begin{pmatrix}1&0\\\Delta t&1\end{pmatrix}
+
Q_k
\end{align*}
Im Korrektorschritt wir die Kalman-Matrix gebraucht, eine
$2\times 1$-Matrix, also ein Spaltenvektor
\[
K=
\begin{pmatrix}k_1\\k_2\end{pmatrix}
\]
Korrektorschritt:
\begin{align*}
\hat x_k
&=
\begin{pmatrix}k_1\\k_2\end{pmatrix}z_k
+
\begin{pmatrix}
1&-k_1\\
0&1-k_2
\end{pmatrix}
\hat x_{k|k-1}
\\
P_k
&=
(I-KH)P_{k|k-1}(I-KH)^t+KRK^t
\\
&=
\begin{pmatrix}
1&-k_1\\
0&1-k_2
\end{pmatrix}
P_{k|k-1}
\begin{pmatrix}
1&0\\
-k_1&1-k_2
\end{pmatrix}
+
R
\begin{pmatrix}
k_1^2&k_1k_2\\
k_1k_2&k_2^2
\end{pmatrix}
\end{align*}
\item
Das folgende Octave-Programm führt die Berechnung durch.
\verbatimainput{kalman-expanded.m}
\qedhere
\end{teilaufgaben}
\end{loesung}

