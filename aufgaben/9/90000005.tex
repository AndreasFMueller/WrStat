Mit Hilfe eines Kreiselsensors, der die Winkelgeschwindigkeit $\omega$
mit der Genauigkeit $\varrho_{\omega}^2$ angeben kann, soll ein System gebaut
werden, mit dem man die aktuelle Winkelbeschleunigung $\alpha$ wie auch
das aktuelle Azimut $\varphi$ bestimmen kann.
Modellieren Sie das System und den Messprozess.

\thema{Kalman-Filter}

\begin{loesung}
Wir nehmen als zusätzliche Systemannahme an, dass die Winkelbeschleunigung
konstant ist.
Dann können wir als Zustandsvektor
\[
x_k=\begin{pmatrix}\varphi_k\\\omega_k\\\alpha_k\end{pmatrix}
\]
wählen.
Die Zeitentwicklung ist
\[
x_{k+1}=\underbrace{\begin{pmatrix}
1&\Delta t&\frac12\Delta t^2\\
0&1       &\Delta t\\
0&0       &1
\end{pmatrix}}_{=\varphi_k}x_k.
\]
Die Systemefehler werden durch eine Diagonalmatrix
\[
Q=\operatorname{diag}(\sigma_{\varphi}^2, \sigma_{\omega}^2, \sigma_{\alpha}^2)
\]
gegeben.

Der Messprozess isoliert die Winkelgeschwindigkeit aus dem Zustandsvektor,
die Messmatrix ist daher
\[
H=\begin{pmatrix}
0&1&0
\end{pmatrix}.
\]
Die Messfehlerkovarianz ist die $1\times 1$-Matrix
$\begin{pmatrix}\varrho_\omega^2\end{pmatrix}$.
\end{loesung}


