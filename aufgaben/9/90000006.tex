Im Rahmen eines Experimentes wurde ein Grösse mit einem alten Messgerät
100 mal gemessen und der Mittelwert bestimmt.
Zur Vervollständigung der Daten wurde dann ein moderneres Messgerät
gemietet, welches halb so grossen Messfehler hat.
Damit konnten jedoch aus Zeitgründen nur 20 Messungen gemacht werden,
von denen ebenfalls der Mittelwert bestimmt wurde.
\begin{teilaufgaben}
\item
Wie muss man die beiden Mittelwert gewichten, um das genauestmögliche
Resultat zu erhalten.
\item
Wieviel mal kleiner ist der Fehler dieses optimalen Mittelwertes als 
der Messfehler des alten Messgerätes?
\item
Wieviel mal kleiner ist der Fehler dieses optimalen Mittelwertes als
der Fehler des mit dem alten Messgerät ermittelten Mittelwertes.
\end{teilaufgaben}

\begin{loesung}
Wir bezeichnen die Einzelmessungen des alten Messgerätes mit $X_i$, sie
haben alle die gleiche Varianz $\operatorname{var}(X_i)=\sigma_{\text{alt}}^2$.
Die Varianz des Mittelwertes der $100$ Messungen ist
\begin{align*}
\overline X
&=
\frac1n\sum_i X_i
\\
\sigma_{\overline X}^2
&=
\operatorname{var}(\overline X)
=
\operatorname{var}\biggl(\frac1n\sum X_i\biggr)
=
\frac1{n^2}\cdot n\operatorname{var}(X_i)=\frac1n\sigma_{\text{alt}}^2.
\qquad\text{mit $n=100$}
\\
&=\frac1{100}\sigma_{\text{alt}}^2.
\end{align*}
Das neue Messgerät hat halb so grossen Messfehler, also
$\sigma_{\text{neu}}=\frac12\sigma_{\text{alt}}$.
Damit werden $20$ Messungen $Y_j$ gemacht, ihr Mittelwert hat die Varianz
\begin{align*}
\overline Y
&=
\frac1{20}\sum_j Y_j
\\
\sigma_{\overline Y}^2
&=
\operatorname{var}(\overline Y)
=
\operatorname{var}\biggl(\frac1{20}\sum_j Y_j\biggr)
=
\frac1{20}\sigma_{\text{neu}}^2
=
\frac1{80}\sigma_{\text{alt}}^2.
\end{align*}
\begin{teilaufgaben}
\item
Aus der Vorlesung ist bekannt, dass die optimalen Gewichte in dieser
Situation
\[
t
=
\frac{\sigma_{\overline Y}^2}{\sigma_{\overline X}^2 + \sigma_{\overline Y}^2}
\qquad\text{und}\qquad
1-t
=
\frac{\sigma_{\overline X}^2}{\sigma_{\overline X}^2 + \sigma_{\overline Y}^2}
\]
sind.
Setzt man die bekannten Werte für $\sigma_{\overline X}^2$ und
$\sigma_{\overline Y}^2$ ein, erhält man
\begin{align*}
t
&=
\frac{\frac1{80}}{\frac1{80}+\frac1{100}}
=
\frac{80\cdot 100}{80\cdot(100+80)}
=
\frac{100}{180}=0.5555
\\
1-t
&=
\frac{\frac1{100}}{\frac1{80}+\frac1{100}}
=
\frac{80\cdot 100}{100\cdot(80+100)}
=
\frac{80}{180}
=
0.4444
\end{align*}
\item
Auch ist die Varianz des optimierten Mittelwertes bekannt:
\begin{align*}
\sigma_{\text{opt}}^2
&=
\frac{\sigma_{\overline X}^2\sigma_{\overline Y}^2}{\sigma_{\overline X}^2 +\sigma_{\overline Y}^2}
=
\frac{\frac1{80}\frac1{100}}{\frac1{80}+\frac1{100}}\sigma_{\text{alt}}^2
=
\frac1{180}\sigma_{\text{alt}}^2
\\
\sigma_{\text{opt}}
&=
\frac1{\sqrt{180}}\sigma_{\text{alt}}
=
0.0745\cdot\sigma_{\text{alt}}.
\qedhere
\end{align*}
\item
Der Messfehler der Messungen mit dem alten Messgerät ist
$0.1\sigma_{\text{alt}}$.
Der Fehler des optimierten Mittelwertes ist also $0.745$ mal kleiner,
eine Verbesserung von $25\%$.
\end{teilaufgaben}
\end{loesung}

\begin{bewertung}
Varianz der Mittelwerte aus den Rechenregeln ($\textbf{V}_{\text{alt}}$ und
$\textbf{V}_{\text{neu}}$ je 1 Punkt,
Berechnung der Gewichte $t$ und $1-t$ gemäss Formel ({\bf G}) 1 Punkt,
Varianz des optimierten Mittelwertes ({\bf V}) 2 Punkte,
Verbesserung ({\bf I}) 1 Punkt.
\end{bewertung}


