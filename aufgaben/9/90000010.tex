In einer Wohnung soll die relative Luftfeuchtigkeit $h$ gemessen werden.
Feuchtigkeitssensoren sind nicht sehr genau.
\begin{teilaufgaben}
\item
Mit welchen zusätzlichen Zustandsvariablen könnte man $h$ zu einem Systemzustand
erweitern?
\item
Wie beeinflussen sich die Zustandsvariablen gegenseitig?
\end{teilaufgaben}

\begin{loesung}
\begin{teilaufgaben}
\item
Innentemperatur $T_i$, Aussentemperatur $T_a$, Heizleistung $P$
\item
Wenn die Innentemperatur sinkt, wird die Luftfeuchtigkeit ansteigen.
Je grösser die Temperaturdifferenz zwischen Innen- und Aussentemperatur ist, desto
schneller verändert sich die Innentemperatur in Richtung zur Aussentemperatur.
Der Zusammenhang ist allerdings nichtlinear.
Je höher die Heizleistung, desto schneller steigt die Innentemperatur.
Es gilt die Differentialgleichung
\[
\frac{dT_i}{dt}
=
\kappa
(T_a-T_i)
+
cP.
\]
Ein Heizungssteuerung würde zusätzlich die Heizleistung reduzieren, wenn die Temperatur
dem Zielwert nahe kommt.
\qedhere
\end{teilaufgaben}
\end{loesung}
