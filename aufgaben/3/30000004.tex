Zwei Standardfamilien (2 Eltern, 2 nicht erwachsene Kinder) und eine
Einelternfamilie mit 2 (nicht erwachsenen) Kinden gehen zusammen Skifahren,
insgesamt 11 Personen.
Sie stehen bei einem vierpl"atzigen Sessellift an und bem"uhen sich,
m"oglichst zusammen Sessel zu besteigen. Tats"achlich gelingt es
ihnen, zwei Sessel mit 4 Familienmitgliedern und einen dritten
Sessel mit drei Mitgliedern zu besetzen.
\begin{teilaufgaben}
\item Auf wieviele Arten k"onnen sich die Familienmitglieder auf
die drei Sessel des Sessellifts verteilen?
\item
Wie gross ist die Wahrscheinlichkeit, dass sich auf dem ersten
Sessel nur Erwachsene befinden?
\item
Wie gross ist die Wahrscheinlichkeit, dass sich auf dem letzten Sessel
genau ein Kind befindet?
\end{teilaufgaben}

\begin{loesung}
\begin{teilaufgaben}
\item F"ur den ersten Sessel m"ussen 4 Personen aus der Gesellschaft
von 11 Personen ausgew"ahlt werden, was auf $\binom{11}{4}=330$
Arten m"oglich ist. F"ur jede solche Wahl m"ussen jetzt 4 Personen
aus den verbleibenden 7 f"ur den zweiten Sessel gew"ahlt werden,
was auf $\binom{7}{4}=35$ Arten m"oglich ist. Insgesamt gibt es
also 
\[
\binom{11}{4}\binom{7}{4}=330\cdot 35=11550
\]
M"oglichkeiten.

Man k"onnte nat"urlich auch beim letzten Sessel beginnen, f"ur
den 3 von 11 Personen ausgew"ahlt werden m"ussen. F"ur den 
zweitletzten Sessel m"ussen dann 4 Personen aus den verbleibenden
8 ausgew"ahlt werden, also 
\[
\binom{11}{3}\binom{8}{4}=165\cdot 70=11550
\]
M"oglichkeiten, wie vorhin.
\item
Um die Wahrscheinlichkeit zu berechnen muss man z"ahlen, auf
wieviele Arten die Mitglieder verteilt werden k"onnen, so dass
auf dem ersten Sessel genau 4 Erwachsene platziert werden.
Dazu m"ussen f"ur den ersten Sessel 4 aus 5 Erwachsenen gew"ahlt werden,
f"ur den zweiten Sessel 4 aus den verbleibenden 7 Personen. Insgesamt
gibt es also
\[
\binom{5}{4}\binom{7}{4}=5\cdot 35=175
\]
M"oglichkeiten. Die gesuchte Wahrscheinlichkeit ist daher
\[
P(\text{nur Erwachsene auf dem ersten Sessel})=
\frac{175}{11550}=0.01515152.
\]
\item
Wieder muss gez"ahlt werden, auf wieviele Arten eines der 6 Kinder
und 2 der 5 Erwachsenen 
f"ur
den letzten Sessel ausgew"ahlt werden k"onnen, es gibt also
\[
\binom{6}{1}\binom{5}{2}\binom{8}{4}=6\cdot 10\cdot70=4200
\]
M"oglichkeiten, die Personen so zu verteilen, dass auf dem dritten
Sessel genau ein Kinde Platz findet. Die Wahrscheinlichkeit daf"ur ist
also
\[
P(\text{genau ein Kind auf dem dritten Sessel})=\frac{4200}{11550}=0.3636364.
\]
\end{teilaufgaben}
Die Wahrscheinlichkeiten in b) und c) k"onnte man auch mit der
hypergeometrischen Verteilung berechnen. F"ur b) muss man von
den 11 Personen 4 f"ur den ersten Sessel ausw"ahlen.
Die Wahrscheinlichkeit,
dass alle vier Erwachsene sind, ist
die gleiche, wie bei einer Lotterie, in der f"unf aus elf Zahlen gezogen wurde,
mit 4 Markierungen vier Richtige zu erreichen:
\[
\frac{
\binom{5}{4}\binom{6}{0}
}{
\binom{11}{4}
}
=\frac{5}{330}\simeq 0.01515
\]
Oder analog in c):
\[
\frac{
\binom{6}{1}\binom{5}{2}
}{
\binom{11}{3}
}
=\frac{6\cdot 10}{165}\simeq 0.3636.
\]
\end{loesung}
