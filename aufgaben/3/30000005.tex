In einem ``einarmigen Banditen'' zeigen drei Walzen jeweils f"unf
verschiedene Symbole an, n"amlich Caro, Pik, Herz, Kreuz und Glocke.
Sehen kann man von jeder Walze immer nur ein Symbol.
Bet"atigt man den Hebel (den ``Arm'' des Banditen),
werden die Walzen in Drehung versetzt. Kommen sie wieder zum Stillstand,
wird je nach Anzeige ein Gewinn ausgezahlt.
\begin{teilaufgaben}
\item
Wie gross ist die Wahrscheinlichkeit, dass drei Glocken sichtbar sind?
\item
Wie gross ist die Wahrscheinlichkeit, dass drei gleiche Symbole
sichtabar sind?
\end{teilaufgaben}

\begin{loesung}
Es gibt $5^3=125$ verschiedene Versuchsausg"ange, die wir als gleich
wahrscheinlich annehmen.
\begin{teilaufgaben}
\item Nur auf eine Art ist es m"oglich, f"unf Glocken anzuzeigen, die
Wahrscheinlichkeit dieses Ereignisses ist also $\frac1{125}=0.008=0.8\%$.
\item Es gibt f"unf m"ogliche Symbole, also ist die Wahrscheinlichkeit
$\frac{5}{125}=\frac1{25})=0.04=4\%$.
\end{teilaufgaben}
\end{loesung}

