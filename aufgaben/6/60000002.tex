Der CCD-Chip einer Digitalkamera besteht aus einer grossen Zahl einzelner
Pixel, die w"ahrend der Blichtung vom Licht erzeugte Elektronen speichern.
Die Zahl $X$ dieser Photoelektronen ist proportional zur Belichtungszeit.
Sie wird jedoch verf"alscht durch zwei Effekte:
\begin{enumerate}
\item
Thermisches Rauschen: Durch die thermische Bewegung der Atome
im Kristall des CCD-Chips gehen Elektronen verloren oder kommen hinzu,
die nichts mit dem Lichteinfall zu tun haben. Dieser zus"atzliche
Elektronenstrom hat Erwartungswert $0$ und bekannte Varianz $\sigma_T$,
man findet $\sigma_T$ im Datenblatt des CCD-Chip.
\item
Auslesefehler: Beim Auslesen der einzelnen Pixel entsteht ein Messfehler
$\sigma_A$, der ebenfalls dem Datenblatt entnommen werden kann.
\end{enumerate}
Der 16 Megapixel-CCD-Chip KAF-16803 von Kodak hat Pixel, die $C = 100000$
Elektronen fassen k"onnen (sogenannte {\it full well capacity}).
Das thermische Rauschen verursacht in jeder
Sekunde $\sigma_T = 3$ zus"atzliche Elektronen, der Auslesefehler
betr"agt $\sigma_A=9$ Elektronen.
\begin{teilaufgaben}
\item Wie gross ist der durch thermisches Rauschen und Auslesefehler
verursachte Fehler bei einer Belichtungszeit von $n$ Sekunden?
\item Nach welcher Belichtungszeit "ubersteigt der Beitrag des
thermischen Rauschens den Auslesefehler?
\item Wie lange darf man maximal belichten, wenn 95\% der Pixel einen Fehler
von weniger als 1\% der Full well capacity aufweisen sollen.
\end{teilaufgaben}

\begin{hinweis}
Sie d"urfen annehmen, dass eventuell vorhandene systematische
Fehler durch Eichung des Chips bereits entfernt worden sind.
\end{hinweis}

\begin{loesung}
Bei allen Zufallsvariablen d"urfen wir annehmen, dass sie normalverteilt
sind. Der Beitrag des thermischen Rauschens ist also in jeder Sekunde
eine normalverteilte Zufallsvariable $T_t$ mit Erwartungswert 0
und Varianz $\sigma_T^2$.
Der Auslesefehler ist normalverteilt mit Erwartungswert 0 und
Varianz $\sigma_A^2$.
\begin{teilaufgaben}
\item In $n$ Sekunden kommen die Beitr"age des thermischen Rauschens
f"ur jede Sekunde zusammen. Das totale thermische Rauschen ist
also
\[
T=\sum_{t=1}^n T_t
\]
Da die einzelnen Beitr"age unabh"angig sind, d"urfen wir die Varianzen
addieren:
\[
\operatorname{var}(T)
=\sum_{t=1}^n\operatorname{var}(T_t)=n\sigma_T^2.
\]
Dazu kommt jetzt noch der Auslesefehler, der ebenfalls unabh"angig ist:
\begin{align*}
\sigma_X^2&=\sigma_A^2+n\sigma_T^2\\
\sigma_X&=\sqrt{\sigma_A^2+n\sigma_t^2}=\sqrt{81+9n}
\end{align*}
\item Wir m"ussen herausfinden, f"ur welches $n$ das thermische
Rauschen $n\sigma_T^2$ den Auslesefehler "ubersteigt:
\[
n\sigma_T^2>\sigma_A^2\quad\Rightarrow\quad n>\frac{\sigma_A^2}{\sigma_T^2}=\frac{81}{9}=9.
\]
Nach $9$ Sekunden beginnt also das thermische Rauschen zu dominieren.
F"ur l"angere Belichtunsgzeiten ist es daher unbedingt notwendig,
das thermische Rauschen zu reduzieren, professionelle Kameras
schaffen dies, indem sie den CCD-Chip stark abk"uhlen.
\item
Die Elektronenzahl in einem Pixel nach $n$ Sekunden Belichtungszeit ist
nach a) eine normalverteilte Zufallsvariable mit Erwartungswert
$\mu$ (nicht bekannt, wir brauchen ihn aber auch nicht) und Varianz
$\sigma^2=\sigma_A^1+n\sigma_T^2$. Die Wahrscheinlichkeit, dass ein Pixel
um weniger als 1\% von $C=100000$ von $\mu$ abweicht ist
$
P(|X-\mu| < 1000),
$
sie soll mindestens $0.95$ sein. Wir suchen also $n$ so, dass
\[
P(|X-\mu| > 1000) = 0.95
\]
Nach Standardisierung bekommen wir daraus
\[
P\biggl(\frac{X-\mu}{\sigma}> \frac{1000}{\sigma}\biggr)=0.95.
\]
Die Gr"osse $(X-\mu)/\sigma$ ist eine standardnormalverteilte
Zufallsvariable. den Wert f"ur $1000/\sigma$ k"onnen wir aus
der Normalverteilungstabelle ablesen:
\[
P\biggl(\frac{X-\mu}{\sigma}> \frac{1000}{\sigma}\biggr)=0.95.
\quad
\Rightarrow
\quad
F\biggl(\frac{1000}{\sigma}\biggr)=0.975
\quad
\Rightarrow
\quad
\frac{1000}{\sigma}=1.96.
\]
Die letzte Gleichung ist aber eine Gleichung f"ur $n$:
\begin{align*}
1000&=1.96\sqrt{\sigma_A^2 + n\sigma_T^2}\\
n&=\frac1{\sigma_T^2}\biggl(\biggl(\frac{1000}{1.96}\biggr)^2-\sigma_A^2\biggr)
=28914.
\end{align*}
Man kann also gut acht Stunden belichten, bis das Rauschen bei 95\% der
Pixel die Schranke von 1\% der maximalen Amplitude "uberschritten hat.
\qedhere
\end{teilaufgaben}
\end{loesung}

