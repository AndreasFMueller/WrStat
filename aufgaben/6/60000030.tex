Wer beim Fernsehen durch die Kanäle zappt, möchte sich gerne einen
"Uberblick über das Unterhaltungsangebot verschaffen.  Dies wird allerdings dadurch erschwert, dass die Sender
auch Werbung ausstrahlen. Wir nehmen an, dass wir 50 Sender empfangen
können, die alle den gleichen Prozentsatz von 20\%
ihrer Sendezeit für Werbung verkaufen.
\begin{teilaufgaben}
\item
Wie gross ist die Wahrscheinlichkeit, dass man beim Durchzappen
keinen Sender findet, der gerade Werbung ausstrahlt?
\item
Wie gross ist die Wahrscheinlichkeit, dass man beim Zappen auf
genau 10 Sendern Werbung sieht?
\item
Wie gross ist die Wahrscheinlichkeit, dass man beim Zappen durch
50 Sender auf mehr als einem Drittel der Kanäle Werbung sieht?
\end{teilaufgaben}

\thema{Binomialverteilung}
\thema{Normalapproximation}
\thema{Standardisierung}

\begin{loesung}
Sei $X$ die Zahl der Sender, die gerade Werbung ausstrahlen.
\begin{teilaufgaben}
\item
Die Wahrscheinlichkeit, dass ein Sender keine Werbung ausstrahlt, ist
$0.8$. Die Wahrscheinlichkeit, dass alle 50 Sender keine Werbung
ausstrahlen, ist $0.8^{50}=0.000014272$.
Dabei nehmen wir an, dass die Sender Werbung unabhängig voneinander 
ausstrahlen. Dies ist wahrscheinlich nicht richtig, denn viele Sender
strahlen die Werbung zum Beispiel vor oder nach üblichen Sendezeiten
für Nachrichten aus. Oder vor dem ``Hauptfilm'' des Abends, also
typischerweise kurz vor 2000 Uhr.
\item
Die Wahrscheinlichkeit, auf genau 10 Sendern Werbung zu sehen, wird durch
die Binomialverteilung geben: 
\begin{align*}
P(X=10)&=\binom{50}{10}0.2^{10}(1-0.2)^{40}=0.13982.
\end{align*}
\item
Dies ist ein Bernoulli-Experiment mit Wahrscheinlichkeit $p=0.2$
und $n=50$ Wiederholungen. Die gesuchte Wahrscheinlichkeit kann mit
der Normalverteilung approximiert werden, dazu muss man $\mu$ und
$\sigma$ dieser Normalverteilung schätzen.

Die erwartete Anzahl $E(X)$ ist $\mu=E(X)=np=10$. Die Varianz ist
$\operatorname{var}(X)=np(1-p)=8$, $\sigma=\sqrt{8}=2\sqrt{2}$.
Gesucht ist die Wahrscheinlichkeit $P(X>50/3)$:
\begin{align*}
P\biggl(
X>\frac{50}3
\biggr)
&=
1-P\biggl(
\frac{X-\mu}{\sigma}\le \frac{\frac{50}3-\mu}{\sigma}
\biggr)
=
1-P\biggl(
Z\le \frac{\frac{50}3-10}{\sqrt{8}}
\biggr)
\\
&=
1-P\biggl(
Z\le \frac{10}{3\sqrt{2}}
\biggr)
=
1-P(Z\le 2.357) = P(-2.357) = 0.009211623
\end{align*}
Natürlich kann man aus diesem Resultat nicht schliessen, dass 
die ``Verseuchung'' mit Werbung nicht allzu schlimm sei.
Die Zahl bedeutet nur, dass die Zahl der gerade Werbung ausstrahlenden
Sender nicht viel grösser als die zu erwartende Zahl von Sendern
sein wird.

Mit einem System wir R kann man nicht nur eine Approximation berechnen,
sondern auch exakt die Wahrscheinlichkeitsfunktion der Binomialverteilung
berechnen. Man findet
\[
\sum_{k>\frac{50}3}^{k\le 50}\binom{50}{k}0.2^k0.8^{50-k}
=
\text{\tt 1 - pbinom(16.666, 50, 0.2)}
=0.01444166,
\]
also ein mit dem approximativen Resultat vergleichbarer Wert.
\qedhere
\end{teilaufgaben}
\end{loesung}

\begin{bewertung}
a) Wahrscheinlichkeit für Werbung ({\bf W}) 1 Punkt,
Wahrscheinlichkeit für Werbung auf allen Kanälen ({\bf A}) 1 Punkt.
b) Binomialverteilung ({\bf B}) 1 Punkt, $P(X=10)$ ({\bf P}) 1 Punkt.
c) Normalapproximation ({\bf N}) 1 Punkt, Standardisierung ({\bf S}) 1 Punkt.
\end{bewertung}
