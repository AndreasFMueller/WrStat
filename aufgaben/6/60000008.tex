Ein seltene Krankheit ist nach in Europa üblicher Definition eine
Krankheit, die weniger als 5 von 10000 Einwohnern haben. Wie gross
ist die Wahrscheinlichkeit, dass in Rapperswil mit seinen 26177
Einwohnern zwischen 12 und 14 Einwohner an einer bestimmen
seltenen Krankheit leiden, die mit einer Häufigkeit von 5 Erkrankten
pro 10000 Einwohnern auftritt?

\begin{loesung}
Die Poissonverteilung berechnet die Wahrscheinlichkeit, dass genau
$k$ Fälle eintreten, wenn die Fälle mit einer Häufigkeit
von $\lambda$ eintreten. In Rapperswil tritt die Krankheit mit
einer Häufigkeit von $\lambda=0.00005 \cdot 26177=13.0885$ auf.
Die Wahrscheinlichkeit, genau 12, 13, oder 14 Fälle zu finden, sind
daher
\begin{center}
\begin{tabular}{|c|c|}
\hline
$k$&$P_\lambda(k)$\\
\hline
12&0.1091637\\
13&0.1099069\\
14&0.1027511\\
\hline
  &0.3218217\\
\hline
\end{tabular}
\end{center}
Die gesuchte Wahrscheinlichkeit ist also $0.32$.

Die Wahrscheinlichkeit kann natürlich mit der Binomialverteilung
gelöst werden, was aber die Berechnung sehr grosser Binomialkoeffizienten
erfordert. Es ist
\[
p=\sum_{k=12}^{14}\binom{26177}{k}(1-p)^{26177-k}p^k=0.321897.
\qedhere
\]
\end{loesung}

