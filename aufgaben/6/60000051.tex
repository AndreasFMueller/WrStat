In einer Galaxie wie unserer Milchstrasse werden etwa 20 Supernova-Explosionen
per Jahrtausend erwartet.
Im letzten Jahrtausend wurden aber nur sechs beobachtet:
\begin{center}
\begin{tabular}{|>{$}c<{$}|l|>{$}c<{$}|}
\hline
\text{Jahr}&Sternbild&\text{scheinbare Helligkeit}\\
\hline
1006&Wolf            &-7.5\\
1054&Stier           &-6\\
1181&Kassiopeia      &-2\\
1572&Kassiopeia      &-4\\
1604&Schlangenträger&-2\\
1680&Kassiopeia      &+6\\
\hline
\end{tabular}
\end{center}
Wie wahrscheinlich ist es, dass in einem Jahrtausend in einer Galaxie
wie unserer Milchstrasse mehr als sechs Supernovae beobachtet werden?

\thema{Poisson-Verteilung}

\begin{loesung}
Supernovae sind offenbar sehr seltene Ereignisse.
Ihr Anzahl kann daher mit der Poissonverteilung modelliert werden.
Die erwartete Anzahl von Supernovae in einem Jahrtausend ist $\lambda=20$.
Die gesucht Wahrscheinlichkeit ist
\begin{align*}
P(X>6)
&=1-P(X\le 6)
=
1-e^{-\lambda}\sum_{k=0}^6 \frac{\lambda^k}{k!}
=
1-0.0002551225
=
0.9997449
\end{align*}
Es hätten also mit sehr hoher Wahrscheinlichkeit mehr als $6$ Supernovae
beobachtet werden müssen.
\end{loesung}

\begin{bewertung}
Poisson-Verteilung ({\bf P}) 1 Punkt,
$\lambda$ ({\bf L}) 1 Punkt,
Negation ({\bf N}) 1 Punkt,
Berechnung der Summe ({\bf S}) 2 Punkte,
Zahlenwert der Wahrscheinlichkeit ({\bf W}) 1 Punkt.
\end{bewertung}

