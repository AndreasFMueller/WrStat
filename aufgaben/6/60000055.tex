Die Schildzecke (Ixodes ricinus, in der Schweiz häufig auch Holzbock genannt) 
ist für die Über\-tragung der Borreliose und anderer Krankheiten verantwortlich.
Borreliose ist relativ häufig, in der Schweiz sind bis zu 50\% der Zecken mit
dem Überträgerbakterium Borrelia burgdorferi infiziert.
Borrelien werden aber erst nach einer gewissen Zeit übertragen,
entfernt man die Zecke sofort, unterbleibt eine Infektion.
So wird nur bei 5\% der von Zecken Gestochenen mit dem Auftreten
der Krankheit gerechnet.

Die von einem Virus ebenfalls durch Zeckenstich übertragene
Frühsommermeningoenzephalitis (FSME) ist dagegen sehr viel seltener.
Ausserhalb der Endemiegebiete besteht nur bei jedem tausendsten Zeckenstich
ein Infektionsrisiko.

\begin{teilaufgaben}
\item
Wie gross ist die Wahrscheinlichkeit dass bei 2000 Zeckenstichen mehr als 110
Borreliose-Erkrankungen auftreten?
\item
Wie gross ist die Wahrscheinlichkeit, dass bei 2000 Zeckenstichen mehr als 2
FSME-Er\-kran\-kun\-gen auftreten?
\end{teilaufgaben}

\begin{loesung}
\begin{teilaufgaben}
\item
Die Infektion mit Borrelien ist ein Bernoulli-Experiment mit $p=0.05$.
Es wird $n=2000$ mal wiederholt.
Da Borreliose relativ häufig ist, kann man die Wahrscheinlichkeit mit
Hilfe der Normalapproximation berechnen, wir verwenden
$\mu=np=100$ und $\sigma=\sqrt{np(1-p)}=9.7468$.
Sei $X$ die Anzahl der Borreliose-Fälle, dann müssen wir die
Wahrscheinlichkeit $P(X>110)$ bestimmen.
Es ist
\begin{align*}
P(X>110)
=
1-P(X\le 110)
&=
1-P\biggl(\frac{X-\mu}{\sigma}\le \frac{110-\mu}{\sigma}\biggr)
\\
&=
1-P(Z\le 1.026)
\end{align*}
Da die Zufallsvariable $Z$ (angenähert) standardnormalverteilt ist, finden wir
aus der Tabelle der Standardnormalverteilung den Wert der Wahrscheinlichkeit
als 
\[ 
P(Z\le 1.026) = 0.8575
\qquad\Rightarrow\qquad
P(X>110)\simeq 0.1475.
\]
\item
Da FSME selten ist, können wir die Poisson-Verteilung zur Approximation der
gesuchten Wahrscheinlichkeit verwenden.
Die erwartete Anzahl von FSME-Fällen bei $2000$ Zeckenstichen ist
$\lambda = 2000\cdot 0.001=2$.
Die gesuchte Wahrscheinlichkeit ist daher
\begin{align*}
P(X>2)
&=
\sum_{k=0}^2 P_\lambda(k)
=
\sum_{k=0}^2 \frac{\lambda^k}{k!} e^{-\lambda}
\\
&=
\biggl(1+\lambda+\frac{\lambda}2\biggr) e^{-\lambda}
=
(1+2+1) e^{-2}=4e^{-2}
=
0.5413.
\qedhere
\end{align*}
\end{teilaufgaben}
\end{loesung}




