Um die Qualität des Chips einer CCD-Kamera zu untersuchen, werden mehrere
Bilder mit geschlossenem Verschluss aufgenommen.
Die Bilder werden dann miteinander verglichen, von jeder Pixelposition
wird Mittelwert und Standardabweichung ermittelt.
Pixel, deren Werte zwischen den einzelnen Bildern deutlich stärker streuen
als der Durchschnitt, werden als beschädigt eingestuft.
Die Resultate werden in einem Bild im FITS-Format festgehalten, die
gesuchten Informationen kann man dem Header entnehmen:
\begin{verbatim}
SIMPLE  =                    T / file does conform to FITS standard
BITPIX  =                  -32 / number of bits per data pixel
NAXIS   =                    3 / number of data axes
NAXIS1  =                 3900 / length of data axis 1
NAXIS2  =                 2616 / length of data axis 2
NAXIS3  =                    1 / length of data axis 3
BADPIXEL=                 2897 / Number of bad pixels  
\end{verbatim}
Man kann daraus ablesen, dass auf dem Chip im Format
$3900\times 2616$ insgesamt 2897 beschädigte Pixel gefunden wurden.
Der Hersteller spezifiziert, dass die Wahrscheinlichkeit, dass ein Pixel
schlecht ist, $0.0003$ sei.
Wie gross ist die Wahrscheinlichkeit nicht mehr als 2897 beschädigte Pixel
zu finden?

\begin{loesung}
Die Untersuchung aller Pixel des CCD-Chip ist ein $n=3900\cdot2616=10202400$
mal wiederholtes Bernoulli-Experiment mit $p=0.0003$.
Sei die Zufallsvariable $X$ die Anzahl der beschädigten Pixel.
Sie ist binomialverteilt mit den genannten Parametern $n$ und $p$.
Sie kann durch eine Normalverteilung mit $\mu = np = 3060.72$
und $\sigma = \sqrt{np(1-p)})=55.31$ approximiert werden.
Die gesuchte Wahrscheinlichkeit ist
\begin{align*}
P(X\le 2897)
&=
P\biggl(
\frac{X-\mu}{\sigma}\le\frac{2897-\mu}{\sigma}
\biggr)
=
P(Z\le -2.96)
=
1-P(Z\le 2.96)
\end{align*}
Den Wert für die Normalverteilung entnimmt man der Tabelle für die
Verteilungsfunktion der Standardnormalverteilung und findet für die
gesuchte Wahrscheinlichkeit
$P(X\le 2897) = 1-0.9985=0.0015$.
\end{loesung}

\begin{bewertung}
Binomialverteilung ({\bf B}) 1 Punkt,
Normalapproximation ({\bf N}) 1 Punkt,
Berechnung von $\mu$ ({\bf M}) und $\sigma$ ($\mathbf{\Sigma}$) 1 Punkt,
Standardisierung ({\bf S}) 1 Punkt,
Wahrscheinlichkeit ({\bf W}) 1 Punkt.
\end{bewertung}


