Ernesto Flach, davon überzeugt, dass die Erde eine Scheibe ist,
möchte dies der ganzen Welt endgültig beweisen.
Zu diesem Zweck kauft er auf Ebay einen alten Vermessungstheodoliten
aus Armeebeständen.
Auf dem Internet findet er die Information, dass dieser Theodolit Winkel
mit einer Genauigkeit von $5''$ messen könne.
Entsprechend einer auf Youtube gefundenen Anleitung misst er jetzt an
den Endpunkten $A$ und $B$ einer Strecke die Winkel zwischen Zenith
(der Richtung vertikal nach oben) und der Ziellinie zum anderen Endpunkt
(Abbildung).
Ist die Erde eine Scheibe, sollten sich die Winkel $\alpha$
und $\beta$ zu $180^\circ$
addieren.
Entsprechend der Anleitung macht er drei Messungen, die er am Schluss 
mittelt.
Auf einer gekrümmten Erde dagegen erwartet man einen Winkelüberschuss,
der umso grösser ist, je länger die Messstrecke ist.
Aus Bequemlichkeit ist Ernesto Flachs Messstrecke beim Schiessstand
seiner Wohngemeinde nur 300m lang.
\begin{center}
\includeagraphics[]{earth-1.pdf}
\end{center}

\begin{teilaufgaben}
\item
Wie gross ist der Winkelüberschuss, den man erwarten dürfte%
\footnote{Natürlich nur, wenn man nicht selbst Flat Earther ist.}?
\item
Von welchem Messfehler (nach Mittelung) muss man bei dieser Messung
des Winkelüberschusses ausgehen?
\item
Wie gross ist die Wahrscheinlichkeit, dass er einen Winkelüberschuss $<5''$
misst, und damit, wenigstens aus seiner Sicht, bewiesen hat, dass
die Erde eine Scheibe ist?
\item 
Wie kann die Zuverlässigkeit des Experimentes verbessert werden?
\end{teilaufgaben}

\thema{Normalverteilung} 
\thema{Standardisierung}
\thema{Rechenregeln für Zufallsvariablen}

\begin{hinweis}
Erdradius: 6371km
\end{hinweis}

\begin{loesung}
\begin{teilaufgaben}
\item
Der zu erwartende Winkelüberschuss ist $l/R$, wobei $l$ die Länge der
Messstrecke und $R$ der Erdradius ist.
In Winkelsekunden ausgedrückt: 
\[
\alpha = \frac{180^\circ}{\pi}\cdot\frac{l}{R}=9.71''.
\]
\item
Bei der Berechnung des "Uberschusses werden zwei normalverteilte
Zufallsvariablen mit $\sigma=5''$ addiert, die Summe hat daher
die Standardabweichung $\sqrt{2}\cdot 5''$.
Mittelt man $n$ normalverteilte Zufallsvariablen mit Standardabweichung
$\sigma$, dann hat der Mittelwert die Standardabweichung $\sigma/\sqrt{n}$.
Insgesamt muss man also mit einer Standardabweichung von 
\[
\frac{\sqrt{2}}{\sqrt{3}}5''= 4.08''
\]
rechnen.
\item
Sei $X$ der gemessene Winkelüberschuss, er ist eine normalverteilte
Zufallsvariable mit $\mu=9.71''$ und $\sigma=4.08''$.
Wir müssen die Wahrscheinlichkeit $P(X < 5'')$ bestimmen.
Dazu standardisieren wir:
\begin{align*}
P(X<5'')
&=
P\biggl(
\frac{X-\mu}{\sigma}<\frac{5''-\mu}{\sigma}\biggr)
\\
&=P(S<-1.1544)
=1-P(S<1.1544)
=1-0.8758 = 12.42\%
\end{align*}
Mit einer Wahrscheinlichkeit 12.4\% (also in einem von 8 Fällen)
wird Ernesto Flach also die Erdkrümmung nicht messen können.
\item
Eine längere Messstrecke würde zu einem grösseren Winkelüberschuss
führen, der sich zuverlässiger von $0$ unterscheiden liesse.
Wäre die Länge der Messstrecke 1km, wäre nach der gleichen Rechnung
wie in c) die Wahrscheinlichkeit für einen Fehler nur noch
$5.9\cdot10^{-5}=0.006\%$.

Alternativ könnte man die Zuverlässigkeit auch mit einem genaueren
Theodoliten steigern.
Moderne Präzisionstheodoliten können eine Genauigkeit von $0.2''$
erreichen, damit wäre ein Fehlschluss ebenfalls praktisch ausgeschlossen.
\qedhere
\end{teilaufgaben}
\end{loesung}

\begin{bewertung}
Teilaufgabe a) ({\bf W}) 1 Punkt,
Addition von Standardabweichungen, Faktor $\sqrt{2}$ ({\bf A}) 1 Punkt,
Mittelung, Nenner $\sqrt{3}$ ({\bf M}) 1 Punkt,
Normalverteilung und Standardisierung ({\bf N}) 1 Punkt,
Wahrscheinlichkeit in c) ({\bf C}) 1 Punkt,
Mindestens ein Verbesserungsvorschlag ({\bf V}) 1 Punkt.
\end{bewertung}



