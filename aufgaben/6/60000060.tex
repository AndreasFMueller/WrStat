Bei Verbindungsexperimenten mit Long Range Wide Area Network (LoRaWAN, 
drahtloses Netzwerk für das Internet of Things)
wurde bei verschiedenen Distanzen gezählt, welcher Prozentsatz der
gesendeten Pakete ankommt.
\begin{teilaufgaben}
\item
Bei einer Distanz von 1\,km kommen recht viele Pakete an,
typischerweise etwa 60\%.
Wie gross ist die Wahrscheinlichkeit, dass mehr als 20 von 50 Paketen
ankommen?
\item
Bei einer Distanz von 10\,km kommen nur noch ganz selten Pakete an,
im Mittel etwa 1\%.
Wie gross ist die Wahrscheinlichkeit, dass mindestens zwei von 50 Pakete
ankommen?
\end{teilaufgaben}

\thema{Poisson-Verteilung}
\thema{Binomial-Verteilung}
\thema{Normalapproximation}

\begin{loesung}
\begin{teilaufgaben}
\item
Die Anzahl $X$ der ankommenden Pakete ist binomialverteilt mit $p=0.6$ und
$n=50$.
Gesucht ist $P(X > 20)$, aber dies ist nicht einfach direkt zu berechnen.
Daher verwenden wir die Normalapproximation mit $\mu=np$ und $\sigma^2=np(1-p)$
und erhalten
\begin{align*}
P(X>20)
&=
1-P(X\le 20)
=
1- P\biggl( \frac{X-\mu}{\sigma}\le \frac{20-\mu}{\sigma}\biggr)
=
1-P(Z\le -2.88675)
\\
&=
P(Z\le 2.887)
=0.9981
\end{align*}
\item
Die Wahrscheinlichkeit kann mit der Poissonverteilung berechnet werden.
Die Anzahl $X$ der angekommenen Pakete ist Poisson-verteilt mit $\lambda=0.5$.
Die gesuchte Wahrscheinlichkeit ist
\begin{align*}
P(X\ge 2)
&=1-P(X < 2) = 1-\sum_{k=0}^1 P(X=k)
= 1-\sum_{k=0}^1 e^{-\lambda}\frac{\lambda^k}{k!}
\\
&= 1- e^{-\lambda}\biggl(1 + \frac{\lambda}{1!}\biggr)
= 1- e^{-0.5}(1+0.5)=1-0.909796=0.090204.
\qedhere
\end{align*}
\end{teilaufgaben}
\end{loesung}

\begin{bewertung}
\begin{teilaufgaben}
\item
Binomialverteilung ({\bf B}) 1 Punkt,
Normalapproximation mit Parametern $\mu$ und $\sigma$ ({\bf N}) 1 Punkt,
Wahrscheinlichtkeit ({\bf P}) 1 Punkt.
\item
Poissonverteilung und Parameter $\lambda$ ({\bf L}) 1 Punkt,
Summenformel ({\bf S}) 1 Punkt,
Wahrscheinlichkeit ({\bf W}) 1 Punkt.
\end{teilaufgaben}
\end{bewertung}


