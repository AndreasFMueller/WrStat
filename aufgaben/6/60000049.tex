Ein neuer Produktionsprozess von Teilen hat eine Ausschussrate von 15\%.
Wie viele Teile m"ussen in Auftrag gegeben werden, damit darunter
mit Wahrschkeit 99\% mindestens 40 gute Teile sind?



\begin{loesung}
Ob ein Teil gut ist, ist ein Bernoulli-Experiment mit Wahrscheinlichkeit
$p=0.85$.
Sie $X$ die Zahl der guten Teile bei $n$ in Auftrag gegebenen Teilen.
$X$ ist binomialverteilt mit Parametern $p$ und $n$.
Da $n$ gross ist, k"onnen wir die Binomialverteilung mit einer
Normalverteilung mit $\mu=np$ und $\sigma^2=np(1-p)$ approximieren.
Wir suchen den $x$-Wert, f"ur den die Wahrscheinlichkeit f"ur $X\ge 0$
0.99 ist:
\begin{align*}
0.99
&=
P(X\ge x)\\
&=
P\biggl(\frac{X-\mu}{\sigma}\ge \frac{x-\mu}{\sigma}\biggr)
=
P\biggl(S\ge \frac{x-np}{\sqrt{np(1-p)}}\biggr)
\end{align*}
mit $x=40$.
Da $S$ (im Rahmen der N"aherung) standardnormalverteilt ist, muss der
gelten
\begin{align*}
\frac{x-np}{\sqrt{np(1-p)}}
&=
x_{0.99}=2.3263
\\
(x-np)^2
&=
x_{0.99}np(1-p)
\\
x^2-2npx+n^2p^2
&=
x_{0.99}np(1-p)
\\
p^2{\color{red}n}^2
-
(2xp+x_{0.99}p(1-p)) {\color{red}n}
+
x^2
&=0
\\
0.7225 {\color{red}n}^2 -68.2966 {\color{red}n} + 1600
&= 
0.
\end{align*}
Diese quadratische Gleichung f"ur $n$ kann man mit der "ublichen
L"osungsformel l"osen und erh"alt
\[
n=51.66.
\]
Es m"ussen also 52 Teile in Auftrag gegeben werden, um 99\% sicher zu sein,
mindestens 40 gute Teile zu erhalten.
\end{loesung}

\begin{bewertung}
\end{bewertung}
