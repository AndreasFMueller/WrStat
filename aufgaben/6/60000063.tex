Nach den US-Wahlen 2020 hat die Trump-Administration in einer Klage vor
dem obersten Gerichtshof behauptet, die Wahrscheinlichkeit für das 
Wahlresultat im Staat Michigan sei kleiner als $10^{-15}$.
Das Argument eines Ökonomen namens Charles J.~ Cicchetti ging etwa so:
Am Wahltag haben von 4662328 Wählern 51.09\% für Trump abgegeben.
Später kamen dann noch 268204 Stimmen von Briefwählern hinzu, von denen 
71.6\% für Biden waren.

\begin{teilaufgaben}
\item
Unter der Annahme, dass die Wahrscheinlichkeit, für Trump zu stimmen, unter
den Wählern, die am Wahltag ihre Stimme abgegeben haben gleich ist wie
die entsprechende Wahrscheinlichkeit unter den Briefwählern, wie gross
ist die Wahrscheinlichkeit dass so viel mehr Briefwähler für Biden stimmen?
\item
Warum ist dieses Argument völlig falsch?
\item
Woher kommt wohl die Zahl $10^{-15}$?
\end{teilaufgaben}

Matt Parker vom Youtube-Kanal Stand-up Maths hat die Lächerlichkeit
dieser ziemlich naiven Argumentation von Herr Cicchetti in einem
unterhaltsamen Video dargestellt: \url{https://youtu.be/ua5aOFi-DKs}.

\begin{teilaufgaben}
\item
Unter der genannten Annahme ist die Anzahl $X$ der Briefstimmen für
Trump binomialverteilt mit $p=0.5109$ und $n=268204$.
Gefragt wird die Wahrscheinlichkeit
\begin{align*}
P(X < 0.284\cdot n) 
&=
P\biggl(\frac{X-np}{\sqrt{np(1-p)}} < \frac{0.284 n -np}{\sqrt{np(1-p)}}\biggr)
\\
&=
\Phi\biggl(
\frac{n(0.284-p)}{\sqrt{np(1-p)}}
\biggr)
=
1
-
\Phi\biggl(
\frac{(p-0.284)n}{\sqrt{np(1-p)}}
\biggr)
=
1-\Phi(235.7)
\intertext{Da $\Phi(235.7)$ sehr nahe bei $1$ sind, ist die Differenz
nicht wirklich numerisch berechenbar, siehe auch Teilaufgabe c).
Man kann aber numerisch wie folgt rechnen:}
P(X < 0.284\cdot n) 
=
\Phi(-235.7).
\end{align*}
Das Argument von $\Phi$ ist derart klein, dass man auch mit den 
breitesten Datentypen auf einem Computer den Wert nicht von $0$
unterscheiden kann.
\item
Die Annahme, dass die Briefwähler mit der gleichen Wahrscheinlichkeit
für Trump stimmen wie die Wähler, die am Wahltag gewählt haben, ist
völlig unhaltbar.
Das Gegenteil ist der Fall und wurde von den Parteien sogar aktiv
herbeitgeführt.
Die Repulikaner haben Briefwählen dämonisiert und ihren Anhängern empfohlen,
am Wahltag zu wählen.
Die Demokraten haben dagegen empfohlen, wegen der Pandemie brieflich
zu wählen.
\item
Der Wert von $\Phi(z)$ ist sehr nahe bei $1$.
Da aber nur die Differenz $1-\Phi(z)$ interessiert, ist die kleinste
Differenz, die mit einer \texttt{double}-Variablen im Computer dargestellt
werden kann, ungefähr $10^{-15}$.
Es handelte sich also gar nicht um eine sinnvolle Wahrscheinlichkeitsangabe,
sondern nur um eine Angabe über die Maschinengenauigkeit des verwendeten
Computers, was die ganze Gesichte natürlich noch viel lächerlicher macht.
\end{teilaufgaben}

\begin{diskussion}
Man kann versuchen $\Phi(-235.7)$ abzuschätzen.
Zum Beispiel ist klar, dass
\[
\Phi(z)
=
\frac{1}{\sqrt{2\pi}}
\int_{-\infty}^{z} e^{-x^2/2}\,dx
\ge
\frac{1}{\sqrt{2\pi}}
\int_{z-1}^{z} e^{-x^2/2}\,dx
\ge
\frac{1}{\sqrt{2\pi}}
e^{-(z-1)^2/2},
\]
indem man vom ganzen uneigentlichen Integral nur ein Interval der Länge $1$
mitnimmt.
Für $z=-235.7$ folgt dann
\[
\Phi(-235.7)
\ge
\frac{e^{-236.7^2/2}}{\sqrt{2\pi}}.
\]
Auch diesen Wert kann man mit dem Computer nicht berechnen.
Aber man kann den Logarithmus davon berechnen:
\[
\log_{10}
\frac{e^{-236.7^2/2}}{\sqrt{2\pi}}
=
-\frac{236.7^2}{2} \log_{10}e - \frac12 \log_{10}(2\pi)
=
-12166.48.
\]
Man kann daraus schliessen, dass die gesuchte Wahrscheinlichkeit
zwar kleiner ist als mit jedem beliebigen Maschinentypen darstellbar,
aber mit Sicherheit grösser ist also $10^{-12166}$.

Man kann den Wert $\Phi(-235.7)$ mit Hilfe einer geeigneten
Bibliothek für hochgenaue Gleitkomma-Typen wie die GNU MPFR
berechnen.
Die MPFR beinhaltet eine Implementation der komplementären Fehlerfunktion
$\operatorname{erfc}(x) = 1-\operatorname{erfc}(x)$.
Mit einem kleinen Programm kann man jetzt die Wahrscheinlichkeit
berechnen und erhält:
\[
\Phi(-235.7)
=
\frac12\operatorname{erfc}\biggl(\frac{-235.7}{\sqrt{2}}\biggr)
=
5.298\cdot10^{-12067},
\]
was natürlich mit obiger Abschätzung konsistent ist.
Die Trumpisten lagen mit ihrer Berechnung der Wahrscheinlichkeit
nur etwa 12052 Grössenordnungen daneben.

Die grosse Entfernung vom Erwartungswert bedeutet, dass die Normalapproximation
wohl kaum adäquat ist.
Man kann aber mit MPFR die Berechnung auch für die Binomialverteilung exakt
durchführen.
Dabei kommt man auf $1.35999\cdot 10^{-12363}$, also nochmals etwa
296 Grössenordnungen kleiner.
Betrachtet man aber nur die Logarithmen der Wahrscheinlichkeiten,
was für derart kleine Zahlen sinnvoll sein kann, dann ist die
Normalapproximation nur 2\% daneben.

So eine Zahl mit über 12000 Stellen wäre in der genannten Klage sicher
noch viel eindrucksvoller gewesen.
Zum Vergleich, zwischen der Planck Länge (etwa $10^{-20}$ mal der Durchmesser
eines Protons) und dem Durchmesser des beobachtbaren Universums sind nur
etwa 60 Grössenordnungen.
Aber es ist auch nachvollziehbar, dass die oben durchgeführte Berechnung
von $\Phi(-235.7)$ weit über das mathematische Niveau der Klagenden
hinausgeht.
\end{diskussion}




