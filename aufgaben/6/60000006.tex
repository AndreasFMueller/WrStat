James Randi bietet jedem eine Million US Dollar an, der beweisen kann,
dass er über aussersinnliche Fähigkeiten verfügt. Ein Wünschelrutengänger
behauptet, er könne mit seiner Wünschelrute Erz von Blindgestein
unterscheiden. Folgender Test wird vorgeschlagen: der Wünschelrutengänger
muss von 10 Kisten entscheiden, ob sich darin Erz oder Blindgestein
befindet, und darf sich höchstens 2 mal irren.
\begin{teilaufgaben}
\item
Wie gross ist die
Wahrscheinlichkeit, dass der Wünschelrutengänger nur durch Zufall
den Test besteht?
\item
Als Alternative wird vorgeschlagen, dass der Wünschelrutengänger
in 10 Versuchen
8 mal hintereinander richtig detektieren muss, insbesondere darf
er die gleiche Anzahl Fehler machen.
Ist dieser Test
besser oder schlechter als der ursprünglich vorgeschlagene?
\end{teilaufgaben}

\begin{loesung}
\begin{teilaufgaben}
\item Wenn der Wünschelrutengänger nur Zufallsresultate erziehlt,
dann ist die Anzahl der richtigen (oder der falschen) binomialverteilt
mit $p=0.5$ und $n=10$. Die Wahrscheinlichkeit für 2 falsche ist also
\begin{align*}
P(\text{höchstens 2 falsche})&=\sum_{k=0}^2 \binom{10}{k}p^{10}
=
\frac1{1024}\biggl(
\binom{10}{0}
+
\binom{10}{1}
+
\binom{10}{2}
\biggr)
\\
&=
\frac1{1024}( 1 + 10 + 45)
=\frac{56}{1024}=0.0546875.
\end{align*}
\item Es sind 8 Versuchsausgänge möglich, wie die folgende Tabelle
zeigt. Die erste Gruppe von Versuchen umfasst diejenigen, in denen
genau zwei Fehler gemacht wurden. In der zweiten Gruppe wurde jeweils
ein Fehler gemacht, dieser muss auf den ersten beiden oder den letzten
beiden vorkommen, da man sonst nicht 8 richtige hintereinander hat.
Ausserdem könnte der Wünschelrutengänger auch in allen Fällen richtig
raten.
\begin{center}
\begin{tabular}{cccccccccc}
R&R&R&R&R&R&R&R&F&F\\
F&R&R&R&R&R&R&R&R&F\\
F&F&R&R&R&R&R&R&R&R\\
\hline
F&R&R&R&R&R&R&R&R&R\\
R&R&R&R&R&R&R&R&R&F\\
R&R&R&R&R&R&R&R&F&R\\
R&F&R&R&R&R&R&R&R&R\\
\hline
R&R&R&R&R&R&R&R&R&R
\end{tabular}
\end{center}
Jeder Versuchsausgang hat unter Annahme von Zufallsresultaten die gleiche
Wahrscheinlichkeit
$p^{10}$, insgesamt ist die Wahrscheinlichkeit, diesen Test ohne
besondere Fähigkeiten zu bestehen also $8\cdot \frac12^{10}=0.0078125$
also deutlich kleiner.
\qedhere
\end{teilaufgaben}
\end{loesung}

