James Randi bietet jedem eine Million US Dollar an, der beweisen kann,
dass er "uber aussersinnliche F"ahigkeiten verf"ugt. Ein W"unschelruteng"anger
behauptet, er k"onne mit seiner W"unschelrute Erz von Blindgestein
unterscheiden. Folgender Test wird vorgeschlagen: der W"unschelruteng"anger
muss von 10 Kisten entscheiden, ob sich darin Erz oder Blindgestein
befindet, und darf sich h"ochstens 2 mal irren.
\begin{teilaufgaben}
\item
Wie gross ist die
Wahrscheinlichkeit, dass der W"unschelruteng"anger nur durch Zufall
den Test besteht?
\item
Als Alternative wird vergeschlagen, dass der W"unschelruteng"anger
in 10 Versuchen
8 mal hintereinander richtig detektieren muss, insbesondere darf
er die gleiche Anzahl Fehler machen.
Ist dieser Test
besser oder schlechter als der urspr"unglich vorgeschlagene?
\end{teilaufgaben}

\begin{loesung}
\begin{teilaufgaben}
\item Wenn der W"unschelruteng"anger nur Zufallsresultate erziehlt,
dann ist die Anzahl der richtigen (oder der falschen) binomialverteilt
mit $p=0.5$ und $n=10$. Die Wahrscheinlichkeit f"ur 2 falsche ist also
\begin{align*}
P(\text{h"ochstens 2 falsche})&=\sum_{k=0}^2 \binom{10}{k}p^{10}
=
\frac1{1024}\biggl(
\binom{10}{0}
+
\binom{10}{1}
+
\binom{10}{2}
\biggr)
\\
&=
\frac1{1024}( 1 + 10 + 45)
=\frac{56}{1024}=0.0546875.
\end{align*}
\item Es sind 8 Versuchsausg"ange m"oglich, wie die folgende Tabelle
zeigt. Die erste Gruppe von Versuchen umfasst diejenigen, in denen
genau zwei Fehler gemacht wurden. In der zweiten Gruppe wurde jeweils
ein Fehler gemacht, dieser muss auf den ersten beiden oder den letzten
beiden vorkommen, da man sonst nicht 8 richtige hintereinander hat.
Ausserdem k"onnte der W"unschelruteng"anger auch in allen F"allen richtig
raten.
\begin{center}
\begin{tabular}{cccccccccc}
R&R&R&R&R&R&R&R&F&F\\
F&R&R&R&R&R&R&R&R&F\\
F&F&R&R&R&R&R&R&R&R\\
\hline
F&R&R&R&R&R&R&R&R&R\\
R&R&R&R&R&R&R&R&R&F\\
R&R&R&R&R&R&R&R&F&R\\
R&F&R&R&R&R&R&R&R&R\\
\hline
R&R&R&R&R&R&R&R&R&R
\end{tabular}
\end{center}
Jeder Versuchsausgang hat unter Annahme von Zufallsresultaten die gleiche
Wahrscheinlichkeit
$p^{10}$, insgesamt ist die Wahrscheinlichkeit, diesen Test ohne
besondere F"ahikeiten zu bestehen also $8\cdot \frac12^{10}=0.0078125$
also deutlich kleiner.
\end{teilaufgaben}
\end{loesung}

