Eine Maschine soll $1\text{g}$ schwere Kleinteile abfüllen, 
jeweils 100 Stück in einen Beutel.
Die Masse der Teile hat eine Standardabweichung von $20\text{mg}$.
Da wiederholt Fehler vorgekommen sind,
wird folgende Erweiterung realisiert: Jeder Beutel wird gewogen und
Beutel, die ein gewisse Gewichtsschranke $m_{\text{min}}$ unterschreiten,
werden ausgeschieden.
\begin{teilaufgaben}
\item Wie gross muss $m_{\text{min}}$ gewählt werden, damit höchstens
$0.1\%$ aller Beutel aussortiert werden müssen?
\item Wie gross ist die Wahrscheinlichkeit, dass bei der Wahl von
$m_{\text{min}}$ gemäss Teilaufgabe a) ein Beutel mit $99$ Teilen den
Test passiert?
\end{teilaufgaben}

\begin{loesung}
Wir schreiben $m=1\text{g}$ für die mittlere Masse eines Teiles und
$\sigma=0.02\text{g}$ für die Standardabweichung.
Die Masse von $n$ Teilen ist eine Zufallsvariable $X_n$, die normalverteilt
ist mit Erwartungswert $E(X_n)=n\cdot m$ und Varianz
$\operatorname{var}(X_n)=n\sigma^2$, oder Standardabweichung
$\sigma_n=\sqrt{n}\cdot\sigma$.
\begin{teilaufgaben}
\item Die Wahrscheinlichkeit, dass $X_{100}$ kleiner ist als $m_{\text{min}}$,
ist
\(
P(X_{100}\le m_{\text{min}})
\). Wir standardisieren:
\begin{align*}
0.001 &= P\biggl(
\frac{X_{100}-100m}{\sigma_{100}}\le \frac{m_{\text{min}}-100m}{\sigma_{100}}
\biggr)
\end{align*}
Da $X_{100}$ normalverteilt ist, ist $(X_{100}-100m)/\sigma_{100}$
standardnormalverteilt, und es folgt mit Hilfe der Quantilentabelle:
\begin{align*}
\frac{m_{\text{min}}-100m}{\sigma_{100}}&=-3.0902\\
m_{\text{min}}&=100m-3.0902\cdot 10\cdot \sigma
\\
&=99.381960\text{g}.
\end{align*}
\item Jetzt ist die Wahrscheinlichkeit $P(X_{99} \ge m_{\text{min}})$ gesucht.
Diese kann man wieder mit Hilfe von Standardisierung finden:
\begin{align*}
P(X_{99} \ge m_{\text{min}})
=
1-P(X_{99} \le m_{\text{min}})
&=
1-P\biggl(
\frac{X_{99}-99m}{\sigma_{99}}\le \frac{m_{\text{min}} - 99m}{\sigma_{99}}
\biggr)
\\
&=1-F\biggl(
\frac{m_{\text{min}} - 99m}{\sigma_{99}}
\biggr)
\end{align*}
Das Argument der Verteilungsfunktion $F$ ist mit dem oben gefundenen
$m_{\text{min}}$
\begin{align*}
\frac{m_{\text{min}} - 99m}{\sigma_{99}}
&=
\frac1{\sigma\sqrt{99}}(
100m-3.0902\cdot 10\cdot \sigma-99m)\\
&=
\frac1{\sigma\sqrt{99}}(m-3.0902\cdot 10\cdot \sigma)
=1.9194.
\end{align*}
Aus der Tabelle der Verteilungsfunktion der Normalverteilung liest man
ab, dass $F(1.919)=0.9726$ ist. Die gesuchte Wahrscheinlichkeit ist
also $1-0.9726=2.74\%$.
\qedhere
\end{teilaufgaben}
\end{loesung}

\begin{bewertung}
In beiden Teilaufgaben benötigt:
Standardabweichung eines Beutels  mit 100 oder 99 Teilen
($\bf \Sigma$) 1 Punkt,
Standardisierung (\textbf{S}) 1 Punkt.
\begin{teilaufgaben}
\item
Standardisierter $x$-Wert für Wahrscheinlichkeit 0.1\% (\textbf{X}) 1 Punkt,
Berechnung $x_{\text{min}}$ (\textbf{M}) 1 Punkt.
\item
Standardisierter $x$-Wert für einen Beutel mit 99 Teilen (\textbf{Y}) 1 Punkt,
Wahrscheinlichkeit (\textbf{W}) 1 Punkt.
\end{teilaufgaben}
\end{bewertung}

