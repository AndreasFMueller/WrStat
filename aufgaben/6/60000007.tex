Die Ladung eines Raketenmotors vom Typ J570W besteht aus 9 zylindrischen
Treibstoffbl"ocken, den sogenannten Grains,
die je 99g Treibstoff enthalten. Zur Vereinfachung der Qualit"atssicherung
m"ochte man jetzt statt jedes einzelne Grain nur noch das Gesamtpaket
aus 9 Grains
w"agen. Die Messungen zeigen, dass die Mittelwerte recht gut eingehalten
werden, im Mittel enth"alt ein Grain 98g Treibstoff (die gesetzlich erlaubte
Abweichung ist viel gr"osser), und die Standardabweichung eines Pakets
ist 10g. Welcher Prozentsatz aller Grains enth"alt mehr als 100g Treibstoff?

\begin{loesung}
Die Treibstoffmenge ist eine normalverteilte Zufallsvariable $X_i$
($i$ ist die Nummer des Grains)
mit Mittelwert $\mu$ und
Standardabweichung $\sigma$. Die Treibstoffmenge in einem
Paket ist daher auch normalverteilt mit Mittelwert $9\mu$ und
Varianz
\[
\operatorname{var}(X_1+\dots+X_9)=9\operatorname(X_i)=9\sigma^2
\]
Die Standardabweichung des Paketes ist also $3\sigma$. Die Messungen
zeigen, dass $3\sigma=10g$, also $\sigma=3.3333g$. Der Mittelwert
eines Paketes ist
\[
E(X_1+\dots+X_9)=9E(X)=9\mu=9\cdot 98\text{g} = 882\text{g}.
\]
Die Wahscheinlichkeit daf"ur, dass ein Grain mehr als 100g Treibstoff
enth"alt, kann jetzt mit der Normalverteilung berechnet werden:
\begin{align*}
P(X>100\text{g})
&=
P\biggl(\frac{X-\mu}{\sigma}>\frac{100-\mu}{\sigma}\biggr)
=
P\biggl(Z>\frac{100-98}{\frac{10}{3}}\biggr)
=
P\biggl(Z>\frac{3}{5}\biggr)
=
P(Z > 0.6)
\end{align*}
wobei $Z$ die standardnormalverteilte Zufallsvariable $(X-\mu)/\sigma$ ist.
Die
Verteilungsfunktion der Standardnormalverteilung hat gem"ass
Tabelle bei $0.6$
den Wert $0.7257$, die Wahrscheinlichkeit f"ur einen Wert gr"osser
als $0.6$ ist also $0.2743$, mehr als 27\% der Grains haben mehr
als 100g Treibstoff.
\end{loesung}

