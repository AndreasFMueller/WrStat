2.9 Mio von 317 Mio Amerikanern glauben, an sich Symptome einer 
Entführung durch Ausserirdische festgestellt zu haben.
Wie gross ist die Wahrscheinlichkeit, in einem Dorf mit 1000
Einwohnern weniger als 6 Leute zu finden, die glauben, von
Ausserirdischen entführt worden zu sein.

\begin{loesung}
Entführung durch Ausserirdische ist ein seltenes Phänomen,
die Wahrscheinlichkeit dafür ist $p=2.9/317=0.00914826498$, also
weniger als 1\%.

Die Wahrscheinlichkeit für die Anzahl Ereignisse kann mit der
Poissonverteilung approximiert werden. Entführungen durch
Ausserirdische treten mit einer Rate von $\lambda = 9.14$ pro 1000 Einwohner
auf. Die Wahrscheinlichkeit, dass genau $k$ Entführungen in 1000
Einwohnern auftreten, ist daher
\[
P_\lambda(k)=e^{-\lambda}\frac{\lambda^k}{k!}.
\]
Für die gesuchte Wahrscheinlichkeit von höchstens 5 Einwohnern, die glauben,
von Ausserirdischen entführt worden zu sein, gilt
\begin{align*}
P(\text{höchstens 5 Entführungen})
&=
\sum_{k=0}^5P_\lambda(k)=\sum_{k=0}^5 e^{-\lambda}\frac{\lambda^k}{k!}
=e^{-\lambda}\sum_{k=0}^5\frac{\lambda^k}{k!}
\\
&=e^{-\lambda}\biggl(
1+\lambda+\frac{\lambda^2}{2}+\frac{\lambda^3}{6}+\frac{\lambda^4}{24}+\frac{\lambda^5}{120}
\biggr)
=0.106979.
\end{align*}

Alternativ kann man die Wahrscheinlichkeit für höchstens 5 Einwohner, die
glauben, von Ausserirdischen entführt worden zu sein auch mit der
Binomialverteilung ausrechnen, allerdings wegen der grossen
Binomialkoeffizienten nur mit dem Computer
\begin{align*}
P(\text{höchstens 5 Entführungen})
&=
\sum_{k=0}^5\binom{1000}{k}p^k(1-p)^{1000-k}
=0.1058988
\end{align*}
(mit R gerechnet).

Die Normalverteilung als Approximation der Binomialverteilung funktioniert
dagegen nicht. Die vorliegende Binomialverteilung hat Erwartungswert
$\mu = np=9.14$ und Varianz $np(1-p)=9.14 * (1-0.00914)=9.0564604$.
Die Standardabweichung ist also $\sigma = 3.00939$. Damit bekäme man
\begin{align*}
P(\text{höchstens 5 Entführungen})
&=
P(X\le 5)=P\biggl(\frac{X-\mu}{\sigma}\le \frac{5-\mu}{\sigma}\biggr)\\
&=F(-1.37569)=0.08445881,
\\
P(\text{höchstens 6 Entführungen})
&=P(-1.0434)=0.1483815,
\\
P(\text{höchstens 5.5 Entführungen})
&=P(-1.20954)=0.1132277.
\end{align*}
Alle drei Werte sind relativ weit weg vom korrekten Wert.
\end{loesung}

\begin{bewertung}
Ansatz mit Poisson-Verteilung ({\bf P}) 1 Punkt,
Wahrscheinlichkeit für Einzelfall ({\bf W}) 1 Punkt,
Rate $\lambda$ ({\bf L}) 1 Punkt,
Summe ({\bf S}) 1 Punkt,
Resultat ({\bf R}) 2 Punkt.
\end{bewertung}

