Auf der Website \url{http://blutspende.ch} findet man Informationen
über die Verteilung der verschiedenen Blutgruppen in der
Bevölkerung.
Zum Beispiel steht da, dass die Blutgruppe A+ die häufigste ist, 38\%
der der Menschen in der Schweiz haben diese Blutgruppe.
Relativ selten ist aber die Blutgruppe B- mit nur 1\%.
Bei einer Blutspendeaktion, wie sie jedes Jahr auch in den Räumlichkeiten
der OST RJ stattfinden, wurden 200 Spenden abgegeben.
\begin{teilaufgaben}
\item
Wie gross ist die Wahrscheinlichkeit, dass weniger als 70 Proben die
Blutgruppe A+ haben?
\item
Wie gross ist die Wahrscheinlichkeit, mehr als 3 Proben der Blutgruppe B-
zu finden?
\end{teilaufgaben}

\thema{Binomialverteilung}
\thema{Normalapproximation}
\thema{Standardisierung}

\begin{loesung}
\begin{teilaufgaben}
\item
Die Anzahl $X$ der Blutproben mit Blutgruppe A+ ist binomialverteilt mit
$n=200$ und $p=0.38$.
Wir approximieren die Verteilung mit der Normalverteilung, dazu müssen
wir mit Erwartungswert $\mu=np=76$ und Varianz $\sigma^2=np(1-p)=47.12$
standardisieren:
\begin{align*}
P(X<70)
&=
P(X\le 69.5)
=
P\biggl(
\frac{X-\mu}{\sigma} \le \frac{69.5-\mu}{\sigma}
\biggr)
\\
&=
\Phi\biggl(
\frac{69.5-\mu}{\sigma}
\biggr)
=
1-\Phi\biggl(
\frac{\mu-69.5}{\sigma}
\biggr)
=
1-\Phi\biggl(
\frac{76-69.5}{6.864}
\bigr)
=
1-\Phi\biggl(
\frac{6.5}{6.864}
\biggr)
\\
&=
1-\Phi(0.947)
=
1-0.8282
=
0.1718.
\end{align*}
Die Wahl der Grenze $69.5$ liefert die beste Approximation, $69$ und $70$
ebenfalls akzeptabel Schranken.
Definitiv falsch ist jedoch die Schranke $70.5$, da damit die Anzahl $70$,
die in der Aufgabenstellung ausgeschlossen ist, explizit mit eingerechnet
wird.
\item
Die Anzahl $Y$ der Blutproben der Blutgruppe B- ist ebenfalls binomialverteilt,
aber da diese Blutgruppe selten ist, wird sie besser mit einer
Poisson-Vorteilung approximiert.
Die erwartete Anzahl von Blutproben ist $\lambda=2$.
Die Wahrscheinlichkeit mehr als zwei Blutproben mit Blutgruppe B-
zu finden ist
\begin{align*}
P(Y>3)
=
1-P(Y\le 3)
&=
1-(
P_{\lambda}(0)
+
P_{\lambda}(1)
+
P_{\lambda}(2)
+
P_{\lambda}(3)
)
)
\\
&=
1
-
\frac{\lambda^0}{0!}e^{-\lambda}
-
\frac{\lambda^1}{1!}e^{-\lambda}
-
\frac{\lambda^2}{2!}e^{-\lambda}
-
\frac{\lambda^3}{3!}e^{-\lambda}
\\
&=
1-e^{-2}\biggl((1+\lambda+\frac12\lambda^2+\frac16\lambda^3\biggr)
=
1-e^{-2}\biggl(1+2+\frac{2^2}{2}+\frac{8}{6}\biggr)
\\
&=
1-e^{-2}\frac{3+6+6+4}{3}
=
1-e^{-2}\frac{19}{3}
=
0.1429.
\qedhere
\end{align*}
\end{teilaufgaben}
\end{loesung}

\begin{bewertung}
Erwartungswert $\mu$ ({\bf M}) 1 Punkt,
Varianz $\sigma^2$ ({\bf V}) 1 Punkt,
Standardisierung ({\bf S}) 1 Punkt,
Normalapproximation ({\bf N}) 1 Punkt,
Wahrscheinlichkeit der Normalapproximation ({\bf W}) 1 Punkt,
Poissonverteilung und Wahrscheinlichkeit ({\bf P}) 1 Punkt.
\end{bewertung}



