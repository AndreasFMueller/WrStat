Bio-Produkte haben es an sich, dass häufiger mit Schädlingsbefall gerechnet
werden muss als bei Produkten mit massivem Pestizid-Einsatz.
Nehmen wir an, dass von einem bestimmten Produkt 15\% der Früchte wegen von
aussen nicht erkennbarem Schädlingsbefall unbrauchbar sind.
Wieviele Früchte muss man bestellen, damit man mit Wahrscheinlichkeit 99\%
mindestens 40 gute Früchte hat.

\thema{Binomalverteilung}
\thema{Normalapproximation}
\thema{Standardisierung}

\begin{loesung}
Ob eine Frucht gut ist, ist ein Bernoulli-Experiment mit Wahrscheinlichkeit
$p=0.85$.
Sei $X$ die Zahl der guten Früchte bei $n$ in bestellten Früchten.
$X$ ist binomialverteilt mit Parametern $p$ und $n$.
Da $n$ gross ist, können wir die Binomialverteilung mit einer
Normalverteilung mit $\mu=np$ und $\sigma^2=np(1-p)$ approximieren.
Wir suchen diejenige Zahl $n$ von zu bestellenden Früchten, für
die die Wahrscheinlichkeit $0.99$ ist, dass $X>40 =x$:
\begin{align*}
0.99
&=
P(X\ge x)\\
&=
P\biggl(\underbrace{\frac{X-\mu}{\sigma}}_{\displaystyle=S}\ge \frac{x-\mu}{\sigma}\biggr)
=
P\biggl(S\ge \frac{x-{\color{red}n}p}{\sqrt{{\color{red}n}p(1-p)}}\biggr)
\end{align*}
mit $x=40$.
Die Unbekannte $\color{red}n$ ist rot hervorgehoben.
Da $S$ (im Rahmen der Näherung) standardnormalverteilt ist, muss
gelten
\begin{align*}
\frac{x-np}{\sqrt{np(1-p)}}
&=
x_{0.99}=2.3263
\\
(x-np)^2
&=
x_{0.99}^2np(1-p)
\\
x^2-2npx+n^2p^2
&=
x_{0.99}^2np(1-p)
\\
p^2{\color{red}n}^2
-
(2xp+x_{0.99}^2p(1-p)) {\color{red}n}
+
x^2
&=0
\\
0.7225 {\color{red}n}^2 -68.69 {\color{red}n} + 1600
&= 
0.
\end{align*}
Im letzten Schritt haben wir wieder $x=40$ eingesetzt.
Diese quadratische Gleichung für $n$ kann man mit der üblichen
Lösungsformel lösen und erhält die beiden Lösungen
\[
n_{1,2}
=
\frac{68.69 \pm \sqrt{68.69^2-4\cdot 0.7225\cdot 1600}}{2\cdot 0.7225}
=
\frac{68.69 \pm 9.7116}{2\cdot 0.7225}
=
\begin{cases}
54.2571\\
40.8155
\end{cases}
\]
Davon ist die zweite viel zu kleine, es bleibt daher als einzig relevante
Lösung $n=54.26$.
Es müssen also 55 Früchte bestellt werden, um 99\% sicher zu sein,
mindestens 40 brauchbare Früchte zu erhalten.
\end{loesung}

\begin{bewertung}
Normalapproximation der Binomialverteilung ({\bf B}) 1 Punkt,
Erwartungswert ({\bf E}) 1 Punkt,
Varianz ({\bf V}) 1 Punkt
Standardisierung ({\bf S}) 1 Punkt,
Gleichung für $n$ ({\bf G}) 1 Punkt,
Lösung für $n$ ({\bf N}) 1 Punkt.
\end{bewertung}


