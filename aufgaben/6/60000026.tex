Wie gross muss die Lebensdauer eines Bauteils sein, damit nach 500
Tagen mit Wahrscheinlichkeit mindestens 0.9 fünf von fünf Bauteilen
immer noch ganz sind?

\begin{loesung}
Sei $T_i$ die Lebensdauer des Bauteils $i$. Wir verlangen, dass
die Wahrscheinlichkeit, dass alle Bauteile nach 500 Tagen noch
ganz sind, mindestens 0.9 ist, d.~h.
\begin{align*}
0.9
&\le
P(
T_1 > 500
\wedge
T_2 > 500
\wedge
T_3 > 500
\wedge
T_4 > 500
\wedge
T_5 > 500)
\\
&=
P(T_1 > 500)
\cdot
P(T_2 > 500)
\cdot
P(T_3 > 500)
\cdot
P(T_4 > 500)
\cdot
P(T_5 > 500)
\\
&=(1-F(500))^5=(e^{-a\cdot 500})^5=e^{-2500a}
\end{align*}
Nimmt man den natürlichen Logarithmus (bezeichnet mit $\log$),
erhält man:
\[
\log 0.9\le -2500 a\quad\Rightarrow\quad \frac1a \ge -\frac{2500}{\log 0.9}
\,\text{Tage}
=23728.1
\,
\text{Tage}
\]
Da $\frac1a$ die mittlere Lebensdauer ist, muss man also Teile mit
einer Lebensdauer von mindestens 23728.1 Tagen verwenden.
\end{loesung}

