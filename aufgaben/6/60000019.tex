Ein Signal mit normalverteilten Werten $X$ mit Erwartungswert
$1$ und Standardabweichung $2$ wird über einen Eingangsverstärker
geleitet, der Werte zwischen $\pm3.3$ verarbeiten kann, bevor
er übersteuert wird. Wie häufig wird der Verstärker übersteuert?

\thema{Normalverteilung}

\begin{loesung}
Die Signalwerte $X$ bilden eine normalverteilte Zufallsvariable mit $\mu = 1$
und $\sigma=2$. Gefragt ist die Wahrscheinlichkeit, dass
$X$ ausserhalb des Intervals $[-3.3, 3.3]$ liegt. Wir lösen dieses
Problem mit Hilfe der Standardisierung $Z=(X-\mu)/\sigma$.
\begin{align*}
P(-3.3\le X\le 3.3)
&=
P\left(
\frac{-3.3-\mu }{\sigma}
\le
\frac{X-\mu}{\sigma}
\le
\frac{3.3-\mu }{\sigma}
\right)
\\
&=
P\left(
\frac{-3.3-1}{2}
\le
Z
\le
\frac{3.3-1}{2}
\right)
\\
&=
P\left(
\frac{-3.3-1}{2}
\le
Z
\le
\frac{3.3-1}{2}
\right)
\\
&=P(-2.15 \le Z \le 1.15)
=F(1.15)-F(-2.15)\\
&=F(1.15)-1+F(2.15)
=0.8749+0.9842 - 1=0.8591.
\end{align*}
Die Wahrscheinlichkeit einer "Ubersteuerung ist also 14.09\%.
\end{loesung}

