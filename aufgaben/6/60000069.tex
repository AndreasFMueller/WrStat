Sei $X$ die binomialverteilte Zufallsvariable von Aufgabe \ref{60000068}
($n=10$ und $p=\frac16$).
Wie kann die Wahrscheinlichkeit für $X\le 2$ approximativ berechnet
werden?

\begin{loesung}
Aus der Lösung von Aufgabe~\ref{60000068} ist $\mu=np=\frac53$ und
$\sigma=\!\sqrt{np(1-p)}=\!\sqrt{\frac{25}{18}}=\frac{5}{6}\sqrt{2}$.
Die Approximationsformel sagt dann
\begin{align*}
P(X\le 2)
&\approx
P\biggl(
\frac{X-\mu}{\sigma}
\le
\frac{2-\mu}{\sigma}
\biggr)
\\
&=
P\biggl(Z \le \frac{2-\frac53}{\frac56\sqrt{2}}\biggr)
=
P\biggl(Z \le \frac{\frac13}{\frac56\sqrt{2}}\biggr)
=
P\biggl(Z \le \frac{1}{\frac52\sqrt{2}}\biggr)
=
P\biggl(Z \le \frac{\!\sqrt{2}}{5}\biggr)
\\
&=
\Phi\biggl( \frac{\!\sqrt{2}}{5}\biggr)
=
\Phi(0.283)
=
0.6114.
\end{align*}
Die exakte Berechnung ergibt
\[
P(X\le 2)
=
\sum_{k=0}^2
\binom{n}{k}p^k(1-p)^{n-k}
=
(1-p)^10
+
\binom{10}{1}
p(1-p)^9
+
\binom{10}{2}
p^2(1-p)^8
=
0.7752.
\]
Verwendet man die Korrektur von de Moivre, setzt man also 2.5 in die
Standardisierungsformel ein statt 2, erhält man
\begin{align*}
P(X\le 2)
&=
\Phi\biggl( \frac{2.5-\mu}{\sigma}\biggr)
=
\Phi(0.707)
=
0.7602.
\end{align*}
Die korrigierte Formel liefert also ein wesentlich genaueres Resultat.
\end{loesung}

