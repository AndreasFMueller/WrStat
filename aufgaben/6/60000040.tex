Flash-Speicherzellen verlieren mit der Zeit die darin gespeicherte Ladung.
Im Datenblatt des Microcontrollers Atmel ATtiny25 steht:
\begin{center}
\includeagraphics[width=\hsize]{retention.pdf}
\end{center}
Das bedeutet, dass die
Wahrscheinlichkeit $10^{-6}$ sei\footnote{Genauer: die Wahrscheinlichkeit
ist kleinar als $10^{-6}$, f"ur die Zwecke dieser Aufgabe gehen wir vom
worst case aus, dass also die Wahrscheinlichkeit genau $10^{-6}$ ist.}
f"ur den Verlust eines Datenbits innerhalb
von 100 Jahren bei $25^\circ$C.
\begin{teilaufgaben}
\item Bestimmen Sie die mittlere Lebensdauer eines Bits.
\item Nach welcher Zeit ist die H"alfte der Bits verloren gegangen?
\item Wie gross ist die Wahrscheinlichkeit, dass ein Bit nach 20000
Jahren verloren gegangen ist?
\item Wieviele Bits gehen im Mittel in dem 2kByte umfassenden Flash-Speicher
des ATtiny25 in 20000 Jahren verloren?
\item Wie gross ist die Wahrscheinlichkeit, dass nach 20000 Jahren
in dem 2kByte umfassenden Flash\-speicher des Chips
weniger als 3 Bits verloren gegangen sind?
\end{teilaufgaben}

\begin{hinweis}
Da die Wahrscheinlichkeiten in dieser Aufgabe sehr klein sind, k"onnen
in den Formeln f"ur die Verteilungsfunktionen lineare N"aherungen
verwendet werden.
\end{hinweis}

\begin{loesung}
Wir gehen davon aus, dass die Zeit bis zum Zerfall eines Bits 
exponentialverteilt ist, und m"ussen zun"achst den Parameter $a$
aus den Angaben der Aufgabe bestimmen.

Die Wahrscheinlichkeit, dass ein Bit nach 100 Jahren verloren gegangen
ist, ist
\begin{equation}
P(T < t) = 10^{-1}=
1-F(t)=1-e^{-at}
=at-\frac{a^2t^2}{2!}+\frac{a^3t^3}{3!}-\frac{a^4t^4}{4!}+\dots
\label{60000034:potenzreihe}
\end{equation}
Da die Lebensdauer $1/a$ eines Bits offenbar sehr lang ist, ist $at$
eine so kleine Zahl, dass man alle auf $at$ folgenden Terme
in (\ref{60000034:potenzreihe}) vernachl"assigen kann. Es folgt, dass
\[
a=\frac{10^{-6}}{100\text{ Jahre}}=10^{-8}\frac{1}{\text{Jahr}}
\]
\begin{teilaufgaben}
\item
Die mittlere Lebensdauer eines Bits ist daher $1/a=10^8$~Jahre.
\item
Die Halbwertszeit ist auch der Median, der in der Vorlesung als
\[
\operatorname{med} T=\frac1a\log 2=6.9314\cdot10^{7} \text{ Jahre}
\]
bestimmt worden ist (Siehe auch ``Datenblatt'' der Exponentialverteilung).
\item
Es ist $P(T\le t)$ mit $t=20000\text{ Jahre}$ zu bestimmen:
\[
P(T\le t)
=
F(t)
=
1-e^{-at}
=
at-\frac{a^2t^2}{2!}+\frac{a^3t^3}{3!}-\dots
\]
Da $at=10^{-8}\cdot 2\cdot 10^4=2\cdot 10^{-4}$ sehr klein ist, kann man die 
Terme h"oherer Ordnung vernachl"assigen, und bekommt
\[
P(T\le t) \simeq 2\cdot 10^{-4}.
\]
\item 
Das Experiment ``Geht das Bit in 20000 Jahren verloren'' wird
f"ur alle $2048\cdot 8$ Bits des Flash-Speichers wiederholt, und es
wird gez"ahlt, wie oft der Fall eingetreten ist.
Dies ist ein Binomial-Experiment mit $p=2\cdot 10^{-4}$ und $n=16384$.
Die erwartete Anzahl Ausf"alle ist $np=3.2768$.
\item
Ausf"alle von einzelnen Zellen in 20000 Jahren sind offenbar seltene Ereignisse,
die mit der Poissonverteilung modelliert werden k"onnen.
In der vorangegangenen Teilaufgabe wurde der Erwartungswert der Anzahl
der Ausf"alle zu $\lambda = 3.2768$ ermittelt, dies ist auch der Parameter
der Poisson-Verteilung.
Die Wahrscheinlichkeit f"ur h"ochstens drei ausgefallene Bits ist
\begin{align*}
P(k\le 3)&=e^{-\lambda}\sum_{k=0}^3\frac{\lambda^k}{k!}
\\
&=e^{-\lambda}\biggl(
1+\frac{\lambda}{1!}
+
\frac{\lambda^2}{2!}
+
\frac{\lambda^3}{3!}
\biggr)
=0.5854
\qedhere
\end{align*}
\end{teilaufgaben}
\end{loesung}

\begin{bewertung}
Exponentialverteilung ({\bf E}) 1 Punkt,
Wert f"ur $a$ ({\bf A}) 1 Punkt,
Halbwertszeit ({\bf H}) 1 Punkt,
Ausfallwahrscheinlichkeit f"ur 20000 Jahre ({\bf L}) 1 Punkt,
Ausfallwahrscheinlichkeit in $8*2048$ bits ({\bf F}) 1 Punkt,
Ausfallwahrscheinlichkeit f"ur drei bits ({\bf D}) 1 Punkt.
\end{bewertung}

