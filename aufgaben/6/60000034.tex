Marrons glac\'es sind kandierte Kastanien, sie sind in Südfrankreich
erfunden worden, kurz nachdem die Kreuzfahrer den Zucker nach Westeuropa
gebracht haben. Ein Hersteller von Marrons glac\'es hat bis anhin
jeweils 10 Stück in eine rechteckige Schachtel in einer $2\times 5$
Anordnung verkauft. Ein neues Marketing soll dem Produkt zu mehr
Umsatz verhelfen. Eine Werbeagentur hat daher eine neue quadratische Schachtel 
für $3\times 3$ Kastanien kreiert, die einen grösseren Eindruck
macht als die alten rechteckigen Schachteln. Auf den alten Schachteln
war das Gewicht 220g aufgedruckt, tatsächlich war das mittlere
Gewicht des Inhalts aber 230g, das aufgedruckte Gewicht wurde
nur in 5\% der Fälle unterschritten. Die neuen Schachteln werden mit
200g angeschrieben. Wie wahrscheinlich ist, dass dieses Gewicht unterschritten
wird?

\thema{Normalverteilung}
\themaL{Rechenregeln fur Zufallsvariablen}{Rechenregeln für Zufallsvariablen}
\thema{Standardisierung}

\begin{loesung}
Die Zufallsvariable $X$ ist die Masse eines Marron glac\'e. $X$ ist
normalverteilt mit $\mu = 23\text{g}$. Die Varianz $\sigma^2$ ist vorerst
noch nicht bekannt.

Sei $Y$ die Masse von 10 Marrons glac\'es, sie haben ein mittleres Gewicht von
$E(Y)=10\mu$, die Varianz ist $\operatorname{var}(Y)=10\sigma^2$.
Die Wahrscheinlichkeit dafür, dass $Y$ kleiner als $220$g ist, ist
\[
P(Y<220)=P\biggl(\frac{Y-10\mu}{\sqrt{10}\sigma}<\frac{220-230}{\sqrt{10}\sigma}\biggr)=0.05.
\]
Da $(Y-10\mu)/\sqrt{10}\sigma$ eine standardnormalverteilte Zufallsvariable
ist, kann man den Wert der rechten Seite der Ungleichung aus der
Normalverteilungstabelle ablesen:
\begin{align*}
-1.644854&= \frac{220-230}{\sqrt{10}\sigma}\\
-1.644854&= -\frac{10}{\sqrt{10}\sigma}\\
\sigma &= \frac{\sqrt{10}}{1.644854}=1.922528\\
\end{align*}
Da jetzt auch $\sigma$ bekannt ist, kann man auch die Wahrscheinlichkeit
berechnen, dass die Masse $Z$ von 9 Marrons glac\'es 200g unterschreitet:
\begin{align*}
P(Z<200)
&=
P\biggl(\frac{Z-9\mu}{\sqrt{9}\sigma}<\frac{200-9\mu}{\sqrt{9}\sigma} \biggr)
\\
&=
P\biggl(\frac{Z-207}{3\sigma}<\frac{-7}{3\sigma} \biggr)
\end{align*}
Die Zufallsvariable $(Z-207)/3\sigma$ ist standardnormalverteilt, somit
kann die Wahrscheinlichkeit aus der Normalverteilungstabelle abgelesen
werden für
\[
x=-\frac{7}{3\sigma}=-\frac{7\cdot1.644854}{3\sqrt{10}}=-1.213680
\]
Die Tabelle oder R lieferen
\[
P(Z<200)=1-F(1.213680)=1-0.887565=0.112435.
\qedhere
\]
\end{loesung}


\begin{bewertung}
Anwendung der Normalverteilung ({\bf N}) 1 Punkt,
Bestimmung von $\mu$ ({\bf M}) 1 Punkt,
Standardisierung ({\bf S}) 1 Punkt,
Bestimmung der Standardabweichung des Gewichts einer Anzahl Kastanien ({\bf K}) 1 Punkt,
Verteilung für 9 Kastanien ({\bf 9}) 1 Punkt,
Wahrscheinlichkeit ({\bf W}) 1 Punkt.
\end{bewertung}



