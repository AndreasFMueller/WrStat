Eine Sorte Körnermais lieferte in den letzten zehn Jahren
einen Ertrag von durchschnittlich
8.5t/ha, mit einer empirischen Standardabweichung von 1.5t/ha.
Nun wird ein neues Düngemittel ausprobiert.
Verwenden Sie die Normalverteilungsnäherung um die minimale
Ertragssteigerung zu bestimmen, für die mit 95\% Sicherheit geschlossen
werden kann, dass das Düngemittel eine Ertragssteigerung gebracht hat.

\begin{loesung}
Gesucht ist derjenige Ertrag, der rein zufällig nur mit einer
Wahrscheinlichkeit
von 5\% übertroffen werden würde.
Die 95\%-Quantile der
Normalverteilung ist 1.6449, d.h.~auf eine Ertragssteigerung kann
geschlossen werden, wenn der Ertrag um
$1.5\cdot 1.6449=2.46735$t/ha gesteigert wurde.
\end{loesung}

