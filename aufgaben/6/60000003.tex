Stellen Sie sich vor, Sie leben auf in einer hochentwickelten Zivilisation,
deren Raumfahrt soweit entwickelt ist, dass sie mit Leichtigkeit ihre
unmittelbare galaktische Umgebung von einigen hundert Lichtjahren
bereisen kann. Ihr K"onig (erstaunlich, dass es in einer hochentwickelten
Zivilisation so etwas noch gibt) ist eine dumme Wette eingegangen (auch
das gibt es noch), er hat behauptet, mehr als die H"alfte aller Sterne der
Galaxie h"atten bewohnbare Planeten. Sie werden vom K"onig losgeschickt,
die Planeten der
100 n"achsten Sterne zu untersuchen, und erst zur"uckzukommen, wenn
Sie 99\% sicher sind, dass Ihr K"onig seine Wette gewonnen hat.
Sollten Sie jedoch nicht zu einem eindeutigen Schluss kommen, sollen Sie
"uberhaupt nicht zur"uckkommen (``zur H"olle fahren'' war seine Wortwahl).
\begin{teilaufgaben}
\item
Wie viel bewohnbare Planeten m"ussen Sie finden, damit sie zur"uckkehren
d"urfen?
\item
Angenommen die Wahrscheinlichkeit, dass ein Stern bewohnbare Planeten
hat, ist genau $p=0.49$.
Wie gross ist die Wahrscheinlichkeit, dass Sie nach Erkundung von 100
Planeten unter Einhaltung des in a) gefundenen Kriteriums zur"uckkehren
d"urfen?
\end{teilaufgaben}

\begin{loesung}
\begin{teilaufgaben}
\item
Es interessiert die Anzahl $X$ Sterne mit bewohnbaren Planeten
unter den $n$ n"achsten Sternen.
Ist $p$ die  Wahrscheinlichkeit daf"ur, dass ein Stern einen bewohnbaren
Planeten hat, dann ist die Zahl $X$ binomialverteilt mit Erwartungswert
$\mu=np$ und Varianz $\sigma^2=np(1-p)$. Da $n$ ziemlich gross ist, d"urfen wir
Wahrscheinlichkeiten durch die Normalverteilung approximieren.

Der K"onig h"atte seine Wette verloren, wenn $p<0.5$ ist. Wir k"onnen
also genau dann 99\% sicher sein, dass er gewonnen hat, wenn
die Wahrscheinlichkeit nur durch Zufall eine gewisse Anzahl $x$
bewohnbare Planeten zu finden, kleiner als 1\% ist.
Wir suchen also $x$ so, dass $P(X>x)<0.01$ gilt, oder $P(X\le x)=0.99$.

Zur Berechnung von $x$ standardisieren wir
\[
P\biggl(
\frac{X-\mu}{\sigma}\le \frac{x-\mu}{\sigma}
\biggr)
=0.99
\]
Da $(X-\mu)/\sigma$ standardnormalverteilt ist, finden wir aus
der Tabelle der Quantilen der Normalverteilung, dass
\[
\frac{x-\mu}{\sigma}=2.3262
\quad\Rightarrow\quad
x=\mu+2.3263\sigma=np+2.3263\sqrt{np(1-p)}
\]
gilt. Setzen wir $n=100$ und $p=0.5$ ein finden wir
\[
x=50 + 2.3263\sqrt{25}=50 + 2.3263\cdot 5=61.6315.
\]
Sie m"ussen also 62 bewohnbare Planeten gefunden haben, bevor sie heimkehren
d"urfen.

\item Es ist die Wahrscheinlichkeit
$P(X\ge 62)$ gesucht, wenn $X$ ein binomialverteilte Zufallsvariable
ist mit $\mu=np$ und $\sigma^2=np(1-p)$. Wir verwenden wieder die
Normalverteilungsapproximation:
\begin{align*}
P(X\ge 62)
&=
P\biggl(
\frac{X-\mu}{\sigma}\ge \frac{62-\mu}{\sigma}
\biggr)\\
&
=1-F\biggl(
\frac{62-np}{\sqrt{np(1-p)}}
\biggr)
\\
&
=1-\biggl(F(\frac{62-49}{\sqrt{49\cdot 0.51}}\biggr)
\\
&
=
1-F\biggl(\frac{13}{4.99899989997999}\biggr)
\\
&=1-F(2.601)=1-0.9953=0.0047
\qedhere
\end{align*}
\end{teilaufgaben}
\end{loesung}
