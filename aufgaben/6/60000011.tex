In seinem Buch ``Why people believe weird things'' beschreibt Michael
Shermer den folgenden Test für aussersinnliche Wahrnehmung
(ESP, extra sensory perception), den er bei der Association for Research and Enlightenment in Virginia Beach, Virginia, selbst miterlebt hat:
die ``sendende'' Person konzentriert sich
auf eines von fünf Symbolen, die ``empfangende'' Person konzentriert sich 
auf die Stirn der sendenden Person und versucht das Symbol zu erkennen.
Mit Hilfe einer ESP-Machine wird dann gezählt, wie oft der ``Empfänger''
das Symbol richtig erkannt hat.
\begin{center}
\includeagraphics[width=0.6\hsize]{esp-machine.png}
\end{center}
Das Experiment wird 25 mal wiederholt. Der Versuchsleiter erklärte:
``5 richtige ist Durchschnitt, 3 bis 7 ist Zufall,
alles über 7 ist ein Hinweis auf aussersinnliche Wahrnehmung''.
\begin{teilaufgaben}
\item Wie wahrscheinlich ist, dass ein völlig unbegabter ``Empfänger''
mehr als 7 Symbole richtig empfängt?
\item Beschreiben Sie einen Test, mit dem man mit grosser Zuverlässigkeit
ESP-begabte ``Empfänger'' identifizieren könnte.
%\item Welchen Prozentsatz der Bevölkerung wird von Ihrem Test als
%ESP-begabt erkannt, selbst wenn es ESP nicht gibt?
\end{teilaufgaben}

\thema{Binomialverteilung}
\thema{Normalapproximation}
\thema{Standardisierung}

\begin{loesung}
\begin{teilaufgaben}
\item Ein Empfänger, der nur zufällige Antworten gibt, wählt
die richtige Antwort mit Wahrscheinlichkeit $p=\frac15$.
Seine Anzahl $X$ richtiger Antworten in 25 Versuchen ist binomialverteilt mit
$p=0.2$ und $n=25$. Gesucht ist die Wahrscheinlichkeit
\begin{align*}
P(X > 7)&= 1 - P(X\le 7)\\
P(X\le 7)&=\sum_{k=0}^7\binom{25}{k}
\biggl(\frac15\biggr)^k
\biggl(1-\frac15\biggr)^{25-k}
 = 0.1091228
\end{align*}
(gerechnet mit R).
Da die Binomialverteilung etwas unhandlich zu berechnen ist, kann man sie
dank des relativ grossen $n=25$ gut durch die Normalverteilung
mit $\mu=np=5$ und $\sigma^2=np(1-p)=25\cdot\frac{4}{25}=4$
approximieren.
\[
P(X\le 7)=P\biggl(\frac{X-\mu}{\sigma}\le \frac{7-\mu}{\sigma}\biggr)
=
0.8413447
\]
Die Grösse $\frac{X-\mu}{\sigma}$ ist angenährt standardnormalverteilt.
Noch etwas besser wird das Result allerdings, wenn man statt $7$ den
Wert $7.5$ nimmt:
\[
P\biggl(Z\le\frac{7.5-\mu}{\sigma}\biggr)
=
P\biggl(Z\le \frac{7.5-5}{2}\biggr)
=
P(Z\le 1.25)=0.8943502\\
\]
Die Wahrscheinlichkeit, mehr als 7 Symbole richtig zu Raten ist
also $1-0.8943502= 0.1056498$.
\item
Wir verwenden einen Test mit $\alpha=0.01$.
Der Empfänger muss eine so grosse Zahl Symbole richtig erkennen,
dass dies nur mit Wahrscheinlichkeit kleiner als $\alpha$ zufällig
passieren kann.
Die Zahl $k$ der richtigen Antworten muss also so gewählt sein,
dass $P(X\ge k) \le \alpha$ ist für binomialverteiltes $X$ mit $p=\frac15$
und $n=25$. Wie in a) approximieren wir dies durch die Normalverteilung
suchen also $k$ so, dass $P(X\ge k)\le\alpha$ unter der Annahme, dass
$X$ normalverteilt ist mit $\mu=5$ und $\sigma=2$.
\begin{align*}
P\biggl(Z\le \frac{k - \mu}{\sigma}\biggr)&\ge 1-\alpha=0.99\\
\frac{k-5}2&\ge 2.326348\\
k&\ge 2\cdot 2.326348 + 5=9.652696
\end{align*}
Von ESP kann man also mit grosser Sicherheit erst dann sprechen, wenn
ein Kandidat mindestens 10 Symbole richtig empfängt.
Bei etwas sorgfältiger Approximation, d.~h.~unter Berücksichtigung
der Klassengrenzen bei halbzahligen $x$-Werten, findet man, dass
$k\ge 9.652696$ sein muss, man also erst bei mindestens 10 richtig
empfangenen Symbolen von den ESP-Fähigkeiten der Testperson
überzeugt sein kann.
%\item
%Die Wahrscheinlichkeit, 10 (11) oder mehr Symbole richtig zu empfangen,
%selbst wenn man nicht über ESP-Fähigkeiten verfügt, kann wie in
%a) berechnet werden. Man findet die folgenden Resultate:
%\begin{center}
%\begin{tabular}{|r|r|r|r|}
%\hline
%$k$&$1-P(X\le k)$&$1-P(Z\le k)$&$1-P(Z\le k+\frac12)$\\
%\hline
% 9&0.01733&0.02275&0.01222\\
%10&0.00555&0.00621&0.00298\\
%11&0.00154&0.00135&0.00058\\
%\hline
%\end{tabular}
%\end{center}
%Darin ist $X$ die binomialverteilte Zufallsvariable mit $n=25$ und $p=\frac12$,
%und $Z$ ist die normalverteilte Approximation mit $\mu=5$ und $\sigma=2$.
\qedhere
\end{teilaufgaben}
\end{loesung}
