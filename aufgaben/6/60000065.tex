Eine Zufallsvariable $X$ hat die Wahrscheinlichkeitsdichte
\[
\varphi_X(x)
=
\frac{1}{\!\sqrt{8\pi}}
e^{-(x-2)^2/8}.
\]
\begin{teilaufgaben}
\item
Bestimmen Sie $\sigma$.
\item
Finden Sie ein Intervall, in das 95\%{} der Beobachtungen von $X$
fallen.
\end{teilaufgaben}

\begin{loesung}
\begin{teilaufgaben}
\item
Im Nenner der Exponentialfunktion einer Normalverteilungsdichte
steht $2\sigma^2$.
Im vorliegenden Fall ist also $2\sigma^2=8$, woraus $\sigma^2=4$ oder
$\sigma=2$ folgt.
\item
95\% der Werte findet man im Intervall $[\mu-2\sigma,\mu+2\sigma]=
[2-2\cdot 2,2+2\cdot 2]=[-2,6]$.
\qedhere
\end{teilaufgaben}
\end{loesung}

