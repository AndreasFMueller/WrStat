In einer Numb3rs Folge bemerkt Charlie, dass trockene Spaghetti beim
Biegen jeweils
in mindestens drei Teile brechen. Nachmessen der dazu nötigen Kräfte
ergibt, dass bei 1N 95\% der Spaghetti gebrochen sind, dass aber 90\%
der Spaghetti eine Kraft von 0.5N tragen können. Wie gross dürfte
die Kraft sein, bei der der Spaghetto bricht? Wie gross ist ihre Varianz?

\thema{Normalverteilung}

\begin{loesung}
Die Kraft, bei der der Spaghetto bricht, ist eine normalverteilte
Zufallsvariable $X$, wir müssen $\mu$ und $\sigma$ dieser Verteilung
bestimmen. Aus der Aufgabenstellung wissen wir
\begin{align*}
P(X \le 1)&=0.95\\
P(X < 0.5)&=0.1
\end{align*}
Standardisierung ergibt
\begin{align*}
P\biggl(\frac{X-\mu}{\sigma} \le \frac{1-\mu}{\sigma}\biggr)&=0.95\\
P\biggl(\frac{X-\mu}{\sigma} < \frac{0.5-\mu}{\sigma}\biggr)&=0.1
\end{align*}
Aus der Tabelle der Standardnormalverteilung weiss man
\begin{align*}
\Phi(1.6449)&=0.95\\
\Phi(-1.2816)&=0.1
\end{align*}
also folgt das Gleichungssystem
\begin{align*}
\frac{1-\mu}{\sigma}&=1.6449\\
\frac{0.5-\mu}{\sigma}&=-1.2816
\end{align*}
welches gleichbedeutend ist mit
\begin{align*}
1&=\mu+1.6449\sigma\\
0.5&=\mu-1.2816\sigma
\end{align*}
Die Differenz ergibt
\[
0.5 = 2.9265\sigma\quad\Rightarrow\quad \sigma = 0.17085
\]
Daraus kann man jetzt auch $\mu$ bestimmen:
\[
\mu = 1-1.6449\sigma =  0.71896.
\qedhere
\]
\end{loesung}

