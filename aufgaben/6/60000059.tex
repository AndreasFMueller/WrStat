Die Wahrscheinlichkeit, sich während einer Grippe-Epidemie anzustecken,
ist sehr hoch.
Auf dem Internet kann man verschiedene Werte für die Wahrscheinlichkeit
einer Ansteckung finden.
Eine Website spricht von 50\%.
Für die Prüfung WrStat sind 96 Studierende angemeldet.
Während einer solchen Grippe-Epidemie müsste man also mit 48 Studierenden
rechnen, die krank an die Prüfung müssen.

Die Grippeimpfung reduziert die Infektionsgefahr etwas.
Nehmen Sie an, dass nur 30\% der Geimpften erkranken.
Nehmen Sie ausserdem an, dass die Hälfte der Studierenden geimpft ist.
Sei $X_u$ die Zahl der ungeimpft erkrankten und $X_i$ die Zahl der geimpft
erkrankten.
Wie gross ist die Wahrscheinlichkeit, dass immer noch
mehr als die Hälfte der Studierenden krank an die Prüfung kommen, also
dass $X=X_u+X_i>48$?

\thema{Binomialverteilung}
\thema{Normalapproximation}
\thema{Standardisierung}

\begin{loesung}
Beide Zufallsvariablen sind binomialverteilt, $X_u$ mit $p_u=0.5$ und $n_u=48$,
$X_i$ mit $p_i=0.3$ und $n_i=48$.
Uns interessiert aber $X=X_u+X_i$.
Wir können $X_u$ und $X_i$ approximieren mit Hilfe einer Normalverteilung
\[
\begin{aligned}
\mu_u&=n_up_u=24,  &&&\sigma_u^2&=n_up_u(1-p_u)=12 \\
\mu_i&=n_ip_i=14.4, &&&\sigma_i^2&=n_ip_i(1-p_i)=10.08
\end{aligned}
\]
Die Summe $X_u+X_i$ ist dann ebenfalls normalverteilt mit 
\[
\mu=\mu_u + \mu_i = 38.4
\qquad\text{und}\qquad
\sigma^2 = \sigma_u^2 + \sigma_i^2 = 22.08.
\]
Die Wahrscheinlichkeit für $X>48$ können wir daher mit Hilfe von
Standardisierung ausrechnen:
\begin{align*}
P(X\le 48)
&=
P\biggl(
\frac{X-\mu}{\sigma}
\le
\frac{48 - \mu}{\sigma}
\biggr)
=
F\biggl(
\frac{48-38.4}{4.69894}
\biggr)
=
F(2.043)
=
0.9793
\\
\Rightarrow
\qquad
P(X>48)
&=
1-P(X\le 48)
=
1-0.9793=0.0207.
\end{align*}

Alternativ kann man auch wie folgt argumentieren.
Sei $E$ das Ereigniss, dass ein Studierender erkrankt ist, und $I$
das Ereignis, dass ein Studierender geimpft ist.
Dann ist die Wahrscheinlichkeit, dass ein Studierender Erkrankt nach dem
Satz von der totalen Wahrscheinlichkeit
\[
P(E)
=
P(E|I)\,P(I)
+
P(E|\bar I)\,P(\bar I)
=
0.3\cdot0.5 + 0.5\cdot 0.5
=
0.4.
\]
Erkrankung ist jetzt ein Bernoulli-Experiment mit $p=0.4$, welches $n=96$ mal
wiederholt wird.
Die Anzahl Eintreten des Ereignisses $E$ kann mit der Normalapproximation
mit $\mu=38.4$ und $\sigma=4.8$ berechnet werden.
Standardisierung ergibt:
\begin{align*}
P(X\le 48)
&=
P\biggl(\frac{X-\mu}{\sigma}\le \frac{48-\mu}{\sigma}\biggr)
=
F\biggl(\frac{48-38.4}{4.8}\biggr)
=
F(2.000)
=
0.9772
\\
\Rightarrow
\qquad
P(X>48)
&=1-P(X\le 48)=1-0.9772 = 0.0228.
\qedhere
\end{align*}
\end{loesung}

\begin{bewertung}
Normalapproximation von $X_u$ ($\textbf{N}_u$) und $X_i$ ($\textbf{N}_i$)
je 1 Punkt,
$X=X_u+X_i$ ist normalverteilt ({\bf N}) 1 Punkt,
Erwartungswert und Varianz von $X$ ({\bf E}) 1 Punkt,
Standardisierung für $P(X\le 48)$ ({\bf S}) 1 Punkt,
gesuchte Wahrscheinlichkeit ({\bf W}) 1 Punkt.

Alternativer Lösungsweg: Ereignisse ({\bf V}) 1 Punkt,
Wahrscheinlichkeiten ({\bf C}) 1 Punkt,
Totale Wahrscheinlichkeit ({\bf T}) 1 Punkt,
Erwartungswert und Varianz ({\bf E}) 1 Punkt,
Standardisierung ({\bf S}) 1 Punkt,
gesuchte Wahrscheinlichkeit ({\bf W}) 1 Punkt.
\end{bewertung}

