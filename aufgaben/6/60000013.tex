In einer Firma werden in den Servern Fibrechannel-Karten verwendet,
die sich als etwas unzuverl"assig herausgestellt haben.
Im Mittel gehen in einer Serverpopulation von 1000 Servern jede Woche
drei Karten kaputt, aus vorerst noch unbekannten Gr"unden.
Mit dem Support des Herstellers wird folgender Deal
ausgehandelt: Bis das Problem vollst"andig gel"ost werden kann, 
ersetzt der Hersteller jede Woche bis zu 5 Karten kostenfrei.
Sollten weitere Karten ersetzt werden m"ussen, muss die Firma
die Kosten von 4000 Franken selber tragen.
\begin{teilaufgaben}
\item
Wie gross ist die Wahrscheinlichkeit, dass dies in einer Woche nicht
reicht?
\item
Mit welchen w"ochentlichen Kosten muss die Firma rechnen?
\end{teilaufgaben}

\begin{loesung}
Kaputte Fibrechannel-Karten sind selten, also kann
die Wahrscheinlichkeitsverteilung f"ur die Anzahl $N$ der kaputten Karten
mit einer Poissonverteilung approximiert werden. Der Parameter $\lambda$
ist die Rate, mit der die Karten kaputt gehen, als $\lambda=3$ (3 Karten
pro Woche).
\begin{teilaufgaben}
\item
Die Wahrscheinlichkeit $p$, dass in einer Woche mehr als 5 Karten kaputt
gehen, ist also
\[
p=1-P(\text{h"ochstens 5 Karten kaputt})=1-F_N(5)
=1-\sum_{k=0}^5\frac{\lambda^k}{k!}e^{-\lambda}
=1-e^{-\lambda}\sum_{k=0}^5\frac{\lambda^k}{k!}
\]
Dabei ist $F_N(x)$ die Verteilungsfunktion der Poissonverteilung mit
Parameter $\lambda=3$. Mit R oder durch direkte Berechnung dieser
Summe findet man daf"ur 0.08391794.
\item
Bei Kosten $K_0$ f"ur eine einzelne Karte sind die zu erwartenden Kosten 
\begin{align*}
K
&=
\sum_{k=6}^\infty K_0(k-5)P_\lambda(k)
=
K_0
\sum_{k=6}^\infty (k-5)\frac{\lambda^k}{k!}e^{-\lambda}
\\
&=
K_0\biggl(
\sum_{k=6}^\infty k\frac{\lambda^k}{k!}e^{-\lambda}
-5\sum_{k=6}^\infty k\frac{\lambda^k}{k!}e^{-\lambda}
\biggr)
\\
&=
K_0\biggl(
\lambda-\sum_{k=1}^5k\frac{\lambda^k}{k!}e^{-\lambda}
-5p
\biggr)
\\
&=
K_0\biggl(\lambda- e^{-\lambda}\sum_{k=1}^5\frac{\lambda^k}{(k-1)!} -5p\biggr)
\end{align*}
Dabei wurde ausgenutzt, dass der Erwartungswert der Poissonverteilung
$\lambda$ ist.
Numerische Auswertung ergibt f"ur die Terme mit $\lambda$
\begin{align*}
\lambda-e^{-\lambda}\sum_{k=1}^5\frac{\lambda^k}{(k-1)!}
=
3-e^{-3}\sum_{k=1}^5\frac{3^k}{(k-1)!}
=0.55421.
\end{align*}
Zusammen mit dem in a) bereits berechneten $p$ findet man f"ur die
zu erwartenden Kosten:
\[
K_0(0.55421 - 5\cdot 0.083918)=0.13462\cdot K_0 = 538.48.
\]
\end{teilaufgaben}
\end{loesung}
