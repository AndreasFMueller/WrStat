In Europa sind 9\% der Männer farbenfehlsichtig (``farbenblind'').
Bei Frauen ist diese erblich bedingte Sehschwäche sehr viel seltener,
nur 0.8\% der Frauen sind farbenfehlsichtig.
\begin{teilaufgaben}
\item
Für die Prüfung WrStat sind 91 Männer angemeldet.
Wie gross ist die Wahrscheinlichkeit, dass kein Prüfungsteilnehmer
farbenfehlsichtig ist?
\item
An der HSR studierten im Jahre 2013 1291
Männer\footnote{Zahlen aus dem Jahresbericht 2013 der HSR},
Wie gross ist die Wahrscheinlichkeit, dass höchstens 100 davon
farbenfehlsichtig sind?
\item 
An der HSR studierten im Jahre 2013 217 Frauen. Wie gross ist die
Wahrscheinlichkeit, unter ihnen mehr als drei farbenfehlsichtige
zu finden?
\end{teilaufgaben}

\begin{loesung}
\begin{teilaufgaben}
\item
Die Wahrscheinlichkeit, nicht farbenfehlsichtig zu sein ist $0.91$, die
Wahrscheinlichkeit, dass 91 Kandidaten alle nicht farbenfehlsichtig sind
ist $0.91^{91}=0.000187399$.
\item
Die Anzahl $X$ der farbenfehlsichtigen ist binomialverteilt mit $n=1291$ und
$p=0.09$. Wir erwarten also etwa $np=116.19$ farbenfehlsichtige Männer,
Farbenfehlsichigkeit ist also nicht selten. Um $P(X\le 100)$ zu
berechnen, approximieren wir die Binomialverteilung mit einer Normalverteilung
mit $\mu=np=116.19$ und $\sigma=\sqrt{np(1-p)}=10.28265$.
\begin{align*}
P(X\le 100)
&=
P\biggl(
\frac{X-\mu}{\sigma}\le \frac{100-\mu}{\sigma}
\biggr)
=
P\biggl(
\frac{X-\mu}{\sigma}\le -1.5745
\biggr)
=
1-F(1.5745)
=
0.0576744
\end{align*}
Die Wahrscheinlichkeit, höchstens 100 farbenfehlsichtige Männer unter
der männlichen Studentenschaft zu finden, ist also nur 5.8\%, ist sind
also fast sicher mehr als 100 männliche Studenten an der HSR
farbenfehlsichtig.
\item
Die Anzahl $X$ der farbenfehlsichtigen Frauen ist ebenfalls binomialverteilt
mit $n=217$ und $p=0.008$. Wir erwarten also $np=1.73$ farbenfehlsichtige
Frauen. Offenbar ist Farbenfehlsichtigkeit unter Frauen selten, und wir
können als Approximation der Anzahlverteilung eine Poissonverteilung mit
$\lambda = np$ verwenden.
Die Wahrscheinlichkeit, höchstens drei farbfehlsichtige Frauen zu finden
ist
\begin{align*}
P(X\le 3)&=\sum_{k=0}^3 e^{-\lambda}\frac{\lambda^k}{k!}=
e^{-\lambda}\biggl(
1+\lambda +\frac{\lambda^2}{2}+\frac{\lambda^3}{6}
\biggr)
=
0.9013518
\\
P(X>3)&=1-P(X\le 3)=
0.0986481
\end{align*}
Mit 90\%-iger Wahrscheinlichkeit gibt es also nicht mehr als drei
farbenfehlsichtige Frauen unter der weblichen Studentenschaft
der HSR im Jahre 2013.

Alternativ könnte man die Wahrscheinlichkeit auch direkt mit der 
Binomialverteilung ausrechnen:
\begin{align*}
P(X\le 3)
&=
\sum_{k=0}3\binom{217}{k}p^k(1-p)^{217-k}=0.9021338
\\
P(X>3)&=1-P(X\le 3)=0.0978662.
\qedhere
\end{align*}
\end{teilaufgaben}
\end{loesung}

\begin{bewertung}
Teilaufgabe a) ({\bf A}) 1 Punkt,
Teilaufgabe b) Normalapproximation:
Parameter $\mu$ und $\sigma$ ({\bf P}) 1 Punkt,
Standardisierung ({\bf S}) 1 Punkt,
Berechnung der Wahrscheinlichkeit ({\bf W}) 1 Punkt,
Teilaufgabe c) Approximation mit Poisson: Parameter $\lambda$ ({\bf L}) 1 Punkt,
Berechnung der Wahrscheinlichkeit (Summe) ({\bf P}) 1 Punkt.
\end{bewertung}

\begin{diskussion}
In Teilaufgabe a) wird nur ein einzelner Term der Summe der
Binomialverteilung ausgewertet, dafür ist die Normalapproximation viel
zu ungenau.
In Teilaufgabe b) kann man nicht die Poisson-Verteilung benutzen, denn
man interessiert sich für eine Approximatino mit vielen Termen in der
Nähe des Erwartungswertes, die Poisson-Reihe hat genau gleich viele
Terme wie die Binomial-Summe.
In Teilaufgabe c) kann man nicht die Normalapproximation benutzen, weil
man sich für Wahrscheinlichkeiten ganz am Rand des Definitionsbereiches
interessiert.
\end{diskussion}

