Internationale Sportwettk"ampfe suggerieren immer wieder, dass grosse
L"ander es offenbar leichter haben, Sporttalente zu rekrutieren.
Nehmen wir vereinfachend an, dass alle Menschen in einer bestimmten Sportart
unabh"angig von der Herkunft vergleichbare, aber nat"urlich 
normalverteilte Resultate erzielen k"onnen.

Ein Ausnahmetalent soll jemand sein, der 4 Standardabweichungen
besser ist als der Durchschnitt, dies tritt mit Wahrscheinlichkeit
$p=0.0000317$ ein.
Wir nehmen, ebenfalls vereinfachen, an, dass 10\% der Bev"olkerung
eines Landes in einem Alter sind, in der solche Spitzenleistungen m"oglich
sind.
\begin{teilaufgaben}
\item
Mit welcher Anzahl von Spitzentalenten kann ein Land wie die Schweiz mit 
8.3 Mio Einwohnern rechnen?
\item 
Das Grossherzogtum Luxemburg hat 576249 Einwohner (1.~Januar 2016)
Wie gross ist Wahrscheinlichkeit, dass Luxemburg eine Olympiamannschaft
rekrutieren kann, die mindestens 4 Ausnahmetalente enth"alt?
\end{teilaufgaben}

\begin{loesung}
Ausnahmetalente sind selten, wir k"onnen daher die Poissonverteilung
als Approximation der Binomailverteilung verwenden.
\begin{teilaufgaben}
\item
Ein Ausnahmetalent zu sein ist ein Binomialexperiment mit $p=0.0000317$,
angewendet auf die $n=830000$ Einwohner im Spitzensportf"ahigen Alter
erwarten wir $np=26.287$ Ausnahmetalente in der Schweiz.
\item
Luxemburg kann nach der gleichen Rechnung nur mit
$\lambda = np = 1.8267$ Ausnahmetalenten rechnen.
Sei $X$ die Anzahl der Ausnahmetalente, sie ist poissonverteilt mit Parameter
$\lambda$.
Die Wahrscheinlichkeit, genau $k$ Ausnahmetalente zu finden, ist
\[
P_{\lambda}(k) = e^{-\lambda}\frac{\lambda^k}{k!}.
\]
Gesucht ist die Wahrscheinlichkeit, mindestens 4 Ausnahmetalente
zu finden:
\begin{align*}
P(X\ge 4)
&=
1-P(X<4)
=
1-\sum_{k=0}^3 e^{-\lambda}\frac{\lambda^k}{k!}
=
1-e^{-\lambda}\biggl(
1+\lambda+\frac{\lambda^2}{2} + \frac{\lambda^3}6
\biggr)
\\
&=
1-0.16094\cdot (1.0000 + 1.8267 + 1.6684 + 1.0159)
=
1-0.16094\cdot 5.5111
\\
&= 0.11304.
\end{align*}
Die Wahrscheinlichkeit, dass Luxemburg eine Mannschaft mit mindestens
vier Ausnahmetalenten aufstellen kann, ist also nur $11.3\%$.
\end{teilaufgaben}
\end{loesung}

\begin{bewertung}
\end{bewertung}


