In einer Fernsehdokumentation wurde über eine Strasse in einem Dorf berichtet,
in der in einem Jahr gleich fünf Personen an Leukämie erkrankt sind.
Pro Jahr treten etwa 160 neue Leukämie-Fälle pro 100000 Personen auf,
so dass gleich fünf neue Fälle für die Anwohner etwas beunruhigend
waren.
\begin{teilaufgaben}
\item
Wie gross ist die Wahrscheinlichkeit, dass in einer Gruppe von
100 Personen (``Strasse'') innert eines Jahres mindestens fünf
Leukämie-Fälle auftreten?
\item
Wir teilen die Weltbevölkerung von 7 Milliarden Menschen
in Gruppen von jeweils 100 Personen auf.
In wievielen dieser Gruppen ist mit mindestens fünf neuen
Leukämie-Erkrankungen in einem Jahr zu rechnen?
\item
Wie wahrscheinlich ist, dass fünf neue Leukämie-Erkrankungen in einem 
Jahr in der gleichen ``Strasse'' nirgends auf der Welt auftreten?
\end{teilaufgaben}

\begin{loesung}
\begin{teilaufgaben}
\item
Leukämie ist zweifellos eine seltene Krankheit, die in Gruppen von 
100 Leuten mit einer Rate von 
\[
\lambda = \frac{160}{100000}\cdot 100=0.16
\]
auftritt.
Seit $X$ die Anzahl neuer Leukämiefälle in einer Gruppe von $100$
Personen, $X$ ist Poisson-verteilt mit $\lambda=0.16$.
Die Wahrscheinlichkeit, häufiger als $k=4$ mal aufzutreten, ist
\begin{align*}
P(X>4)
&=
1-P(X\le 4)
=
1-\sum_{k=0}^4 \frac{\lambda^k}{k!}e^{-\lambda}
=
7.649\cdot10^{-7}=p
\end{align*}
\item
Da es $n=70000000=7\cdot 10^7$ solche ``Strassen'' gibt, handelt es sich hier um
ein $n$ fach wiederholtes Bernoulli-Experiment mit Wahrscheinlichkeit $p$.
Die erwartet Anzahl Fälle, in denen dies eintritt (also mindestens 5
neue Leukämie-Fälle in einem Jahr) ist
\[
np=7\cdot 10^7\cdot 7.649\cdot 10^{-7}=53.543.
\]
\item
Die Wahrscheinlichkeit, dass das Ereignis in einer ``Strasse'' nicht eintritt,
ist $1-p$.
Nehmen wir an, dass die einzelnen ``Strassen'' unabhängig sind, dann
folgt, dass die Wahrscheinlichkeit, dass in keiner einzigen ``Strasse''
mehr als 4 neue Leukämie-Fälle in einem Jahr auftreten werden, 
etwa $(1-p)^{70000000}$ ist, was numerische einen Wert von etwa
$5.6\cdot 10^{-24}$ ergibt.
\qedhere
\end{teilaufgaben}
\end{loesung}

\begin{bewertung}
Poisson-Verteilung ({\bf P}) 1 Punkt,
Bestimmung von $\lambda$ ({\bf L}) 1 Punkt,
$P(X>4)$ ({\bf W}) 1 Punkt,
Bernoulli-Experiment in b) ({\bf B}) 1 Punkt,
Erwartungswert für die Anzahl: ({\bf E}) 1 Punkt,
Nichteintretenswahrscheinlichkeit ({\bf N}) 1 Punkt.
\end{bewertung}



