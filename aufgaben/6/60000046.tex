In einer Fernsehdokumentation wurde "uber eine Strasse in einem Dorf berichtet,
in der in einem Jahr gleich f"unf Personen an Leuk"amie erkrankt sind.
Pro Jahr treten etwa 160 neue Leuk"amie-F"alle pro 100000 Personen auf,
so dass gleich f"unf neue F"alle f"ur die Anwohner etwas beunruhigend
waren.
\begin{teilaufgaben}
\item
Wie gross ist die Wahrscheinlichkeit, dass in einer Gruppe von
100 Personen (``Strasse'') innert eines Jahres mindestens f"unf
Leuk"amie-F"alle auftreten.
\item
Wir teilen die Weltbev"olkerung von 7 Millionen in Gruppen von jeweils 100
Personen auf.
In wievielen dieser Gruppen ist mit mindestens f"unf neuen
Leuk"amie-Erkrankungen in einem Jahr zu rechnen?
\item
Wie wahrscheinlich ist, dass f"unf neue Leuk"amie-Erkrankungen in einem 
Jahr in der gleichen ``Strasse'' nirgends auf der Welt auftritt?
\end{teilaufgaben}

\begin{loesung}
\begin{teilaufgaben}
\item
Leuk"amie ist zweifellos eine seltene Krankheit, die in Gruppen von 
100 Leuten mit einer Rate von 
\[
\lambda = \frac{160}{100000}\cdot 100=0.16
\]
auftritt.
Seit $X$ die Anzahl neuer Leuk"amief"alle in einer Gruppe von $100$
Personen.
Die Wahrscheinlichkeit, h"aufiger als $k=4$ mal aufzutreten, ist
\begin{align*}
P(X>4)
&=
1-P(X\le 4)
=
1-\sum_{k=0}^4 \frac{\lambda^k}{k!}e^{-\lambda}
=
7.649\cdot10^{-7}=p
\end{align*}
\item
Da es $n=70000000=7\cdot 10^7$ solche ``Strassen'' gibt, handelt es sich hier um
ein $n$ fach wiederholtes Bernoulli-Experiment mit Wahrscheinlichkeit $p$.
Die erwartet Anzahl F"alle, in denen dies eintritt (also mindestens 5
neue Leuk"amie-F"alle in einem Jahr) ist
\[
np=7\cdot 10^7\cdot 7.649\cdot 10^{-7}=53.543.
\]
\item
Die Wahrscheinlichkeit, dass das Ereignis in einer ``Strasse'' nicht eintritt,
ist $1-p$.
Nehmen wir an, dass die einzelnen ``Strassen'' unabh"angig sind, dann
folgt, dass die Wahrscheinlichkeit, dass in keiner einzigen ``Strasse''
mehr als 4 neue Leuk"amie-F"alle in einem Jahr auftreten werden, 
etwa $(1-p)^70000000$ ist, was numerische einen Wert von etwa
$5.6\cdot 10^{-24}$ ergibt.
\end{teilaufgaben}
\end{loesung}

\begin{bewertung}
\end{bewertung}



