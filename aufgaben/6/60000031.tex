Ein Bibliotheksfunktion liefert zwischen $0$ und $1$ gleichverteilte
Zufallszahlen. Wie gross ist die Wahrscheinlichkeit, dass die Summe
zweier solcher Zufallszahlen den Wert $0.5$ nicht "uberschreitet?

\begin{loesung}
Wir haben zwei gleichverteilte Zufallsvariablen $X_1$ und $X_2$. 
In der Vorlesung wurde besprochen, wie man die Dichtefunktion
von $X=X_1+X_2$ durch Faltung ermitteln kann.
Im Falle der Gleichverteilung haben wir dies durchgef"uhrt,
wir wissen, dass die Dichtefunktion von $X$
eine ``Dreiecksfunktion'' mit Basis $[0,1]$ und H"ohe $1$ ist.
Die Verteilungsfunktion kann man daher durch Integrieren bestimmen:
\[
F_X(x)=\begin{cases}
0&\qquad x\le 0\\
\frac12x^2&\qquad 0\le x\le 1\\
1-\frac{(x-2)^2}2&\qquad1\le x\le 2\\
1&\qquad x\ge 2
\end{cases}
\]
Man kann sich allerdings die Arbeit f"ur den Teil $1\le x\le 2$ sparen,
wir brauchen nur den Teil $0\le x\le 1$.
Daraus kann man die gesuchte Wahrscheinlichkeit ablesen:
\[
P(X\le 0.5)=F_X(0.5)=\frac120.5^2=0.125.
\]
\end{loesung}

