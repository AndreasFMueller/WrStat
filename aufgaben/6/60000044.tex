In einem Produktionsprozess stellt sich heraus, dass ein Teil mit Wahrscheinlichkeit
$p=5\cdot10^{-4}$ fehlerhaft herauskommt.
\begin{teilaufgaben}
\item Mit wievielen fehlerhaften Teilen muss man in einem Batch von 10000 Teilen
rechnen?
\item Berechnen Sie die Wahrscheinlichkeit, genau $k$ fehlerhafte Teile in einem
Batch von 10000 Teilen zu finden sowohl mit der Poisson-Verteilung als auch
mit der Binomialverteilung, und stellen Sie die Resultate in einer Tabelle
zusammen.
\end{teilaufgaben}

\begin{loesung}
\begin{teilaufgaben}
\item Die erwartete Anzahl von fehlerhaften Teilen ist
$E(X)=np=10000\cdot 5\cdot10^{-4}=5$.
\item F"ur die Poissonverteilung brauchen wir $\lambda=E(X)=5$.
Die Berechnung der Werte der Binomialverteilung ist nicht ganz trivial, weil
die Zahl $n=10000$ sehr gross ist.
Mit R ist die Berechnung allerdings einfach:
\verbatimainput{code}
Gegen"ubergestellt:
\begin{center}
\begin{tabular}{c|rr}
$k$&$\binom{n}{k}p^k(1-p)^{n-k}$&$P_\lambda(k)$\\
\hline
 0&0.006729527& 0.006737947\\
 1&0.033664467& 0.033689735\\
 2&0.084194850& 0.084224337\\
 3&0.140366868& 0.140373896\\
 4&0.175493694& 0.175467370\\
 5&0.175511252& 0.175467370\\
 6&0.146259377& 0.146222808\\
 7&0.104460531& 0.104444863\\
 8&0.065274768& 0.065278039\\
 9&0.036252875& 0.036265577\\
10&0.018119183& 0.018132789\\
\hline
\end{tabular}
\end{center}
\end{teilaufgaben}
\end{loesung}

