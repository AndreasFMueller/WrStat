Im Youtube-Video \url{https://www.youtube.com/watch?v=QPKKQnijnsM} wird
gesagt, dass die 1\%-Topverdiener in Amerika 40\% des Vermögens
besitzen. Nehmen Sie an, dass die Vermögen potenzverteilt sind.
\begin{teilaufgaben}
\item
Berechnen Sie den zugehörigen Gini-Koeffizienten $\alpha$.
\item
Welchen Anteil am Volksvermögens haben diejenigen, welche
weniger als das Median-Vermögen haben?
\item
Im Video erfährt man auch, dass die unteren 80\% der Amerikaner nur
7\% des Volksvermögens haben. 
Auf welchen Gini-Koeffizienten lässt dies schliessen?
\end{teilaufgaben}

\thema{Potenzverteilung}

\begin{loesung}
\begin{teilaufgaben}
\item
In der Diskussion der 80/20-Regeln in der Vorlesung wurde die Formel
\[
\alpha=\frac{\lambda-2}{\lambda-1},\qquad \lambda=\frac{\log 0.8}{\log 0.2}
\]
hergeleitet. Angewendet auf die vorliegende Situation ergibt sich
\[
\lambda=\frac{\log 0.4}{\log 0.01}=0.19897
\qquad
\Rightarrow
\qquad
\alpha=\frac{\lambda - 2}{\lambda - 1}=2.2484
\]
\item
Die Beziehung zwischen $P(x)$ und $W(x)$ für eine Potenzverteilung erlaubt
uns nun auch, den Anteil der 50\%-Topverdiener ($P(x)=0.5$) zu berechnen.
In der Vorlesung wurde hergeleitet, dass 
\[
W(x)=P(x)^\frac{\alpha - 2}{\alpha - 1}=P(x)^\lambda
\]
gilt. Mit $P(x)=0.5$ und dem oben gefundenen Wert von $\lambda$
folgt
\[
W(x)=0.5^{0.19897}=0.87117
\]
Die Leute, die weniger als das Medianvermögens haben, haben zusammen
nicht einmal $13\%$ des Volksvermögens.
\item
Wenn die unteren 80\% nur 7\% verdienen, dann verdienen die 20\%-Topverdiener
93\%, es liegt also eine 93/20-Regel vor. Mit den bekannten Formeln finden
wir den Gini-Koeffizienten
\[
\lambda=\frac{\log 0.93}{\log 0.2}=0.045091
\qquad\Rightarrow\qquad
\alpha=\frac{\lambda - 2}{\lambda - 1}=2.0472
\]
Die Ungleichheit scheint also noch viel grösser zu sein, als
man nach Teilaufgabe a) hätte vermuten können.
\qedhere
\end{teilaufgaben}
\end{loesung}

