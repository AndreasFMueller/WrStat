In der Fachzeitschrift {\em Science} erschien 2004 ein Artikel über den
Border-Collie Rico, der 200 verschiedene Objekte, meistens Spielsachen
für Kinder, beim Namen kannte.
Rief ihm seine Halterin den Namen des Objektes zu, konnte er das verlangte
Objekt in einem Nebenraum in 37 von 40 Fällen holen.

Rico konnte sogar ein Spielzeug erkennen, dessen Namen er noch nicht
kannte.
Fügte man ein solches Spielzeug hinzu und verlangte von Rico, es zu bringen,
erkannte er in 7 von 10 Fällen das ihm unbekannte Spielzeug.

Nehmen Sie an, dass Rico einfach zufällig gewählte Spielzeuge aus
dem Nebenraum bringt, also die Namen gar nicht versteht.
Dann würde es wohl nur ganz selten eine Übereinstimmung geben.
\begin{teilaufgaben}
\item
Wie gross wäre dann die Wahrscheinlichkeit, genau 37 mal das richtige
Spielzeug zu bringen?
\item
Wie wahrscheinlich wäre, dass Rico mindestens ein Spielzeug richtig
bringt?
\item
Wie wahrscheinlich wäre, dass Rico das unbekannte Spielzeug mindestens
einmal finden würde?
\end{teilaufgaben}

\thema{Poisson-Verteilung}

\begin{loesung}
Rein zufällig könnte Rico das richtige Spielzeug nur mit Wahrscheinlichkeit
$p=1/200$ bringen.
In $n=40$ Versuchen könnte man so nur mit $np=0.2$ richtigen rechnen.
Sei $K$ die Anzahl der richtigen Spielzeuge, die Rico gefunden hat.
Wir können die Wahrscheinlichkeit $P(K=k)$ in diesem Experiment mit der
Poisson-Verteilung
\[
P(K=k)
=
P_\lambda(k)
=
e^{-\lambda}\frac{\lambda^k}{k!}
\]
mit $\lambda=0.2$ approximieren.
\begin{teilaufgaben}
\item
Die Wahrscheinlichkeit ist verschwindend klein:
\[
P_\lambda(37)
=
e^{-0.2}\frac{0.2^{37}}{37!}=8.1755\cdot10^{-70}.
\]
Mit der Binomialverteilung:
\[
\binom{40}{37}p^{37}(1-p)^{40-37}=7.0814\cdot10^{-82}.
\]
\item
Die gesuchte Wahrscheinlichkeit ist
\begin{align*}
P(K>0)
&=
\sum_{k=1}^\infty P_\lambda(k)
=
1-P_\lambda(0)
=
1-e^{-\lambda}\frac{\lambda^0}{0!}
=
0.181269.
\end{align*}
Mit Binomialverteilung:
\[
1-
\binom{40}{0}p^0(1-p)^{40-0}=1- 0.995^40 = 0.18168.
\]
\item
In diesem Experiment ist $p=1/201$ und $n=10$, also $\lambda=0.049751$.
Die gleiche Rechnung wie in Teilaufgabe b) liefert jetzt
\[
1-P_\lambda(0)
=
1-e^{-\lambda}
=0.048534.
\qedhere
\]
\end{teilaufgaben}
\end{loesung}

\begin{bewertung}
Verwendung der Poisson-Verteilung ({\bf P}) 1 Punkt,
Bestimmung von $\lambda$ ($\bm{\Lambda_1}$) 1 Punkt,
\begin{teilaufgaben}
\item Berechnung von $P_\lambda(37)$ ({\bf A}) 1 Punkt,
\item Summenformel und Berechnung von $P(K>0)$ ({\bf 0}) 1 Punkt,
\item Neues $p$ und neues Lambda ($\bm{\Lambda_2}$) 1 Punkt,
Wahrscheinlichkeit $1-P_\lambda(0)$ ({\bf W}) 1 Punkt.
\end{teilaufgaben}
\end{bewertung}

