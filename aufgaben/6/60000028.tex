Die ESO (European Southern Observatory, Europäische Südsternwarte)
hat im Frühjahr 2013 beschlossen,
das E-ELT (European Extremely Large Telescope) auf dem
Cerro Armazones in Chile zu bauen.
Dieser Berggipfel 3064m über dem Meerspiegel soll das neue Teleskop
mit 39m Spiegeldurchmesser beheimaten.
Ein Faktor für diese Wahl war die grosse Zahl klarer, für astronomische
Beobachtungen geeigneter Nächte von durchschnittlich 350 pro Jahr.
\begin{teilaufgaben}
\item Wie gross ist die Wahrscheinlichekeit,
dass in einem Monat mehr als zwei Nächte
nicht für Beobachtungen geeignet sind?
\item Wieviel Zeit verstreicht im Mittel zwischen zwei wolkigen Nächten?
\item Ein Astronom braucht für ein Forschungsprojekt Messungen aus
14 aufeinanderfolgenden Nächten. Wie gross ist die Wahrscheinlichkeit,
dass sich dieses Beobachtungsprogramm auf dem Cerro Armazones
im ersten Anlauf realisieren lässt?
\end{teilaufgaben}

\thema{Poisson-Verteilung}
\thema{Rechenregeln für Wahrscheinlichkeiten}

\begin{loesung}
Wolkige Nächte auf dem Cerro Armazones sind offenbar seltene
Ereignisse, können also mit der Poisson-Verteilung modelliert
werden.
\begin{teilaufgaben}
\item Wolkige Nächte treten auf mit einer Rate 15 pro Jahr, oder
$\lambda=\frac{15}{12}=1.25$ pro Monat. Mit Hilfe der Poisson-Verteilung
kann man jetzt die Wahrscheinlichkeit für $k=0,1,2$ wolkige Nächte
ausrechnen:
\begin{align*}
P(\text{mehr als 2 wolkige Nächte})
&=1-P(X\le 2)
=
1-\sum_{k=0}^2\frac{\lambda^k}{k!}e^{-\lambda}
\\
&=1-e^{-\lambda}\biggl(1+\lambda+\frac{\lambda^2}2\biggr)
=0.13153
\end{align*}
\item 
Wenn in einem Jahr auf 365 Tage mit 15 Schlechtwetternächten gerechnet
werden muss, dann ist naheliegend, dass die  mittlere Zeit zwischen
den Schlechtwetternächten $\frac{365}{15}=24.3333$ sein könnte.
Doch was könnte der Einfluss extrem unterschiedlicher Verteilung
der vorkommenden Intervalle sein? Könnte es sein, dass die mittlere
Zeit zwischen zwei Schlechtwetternächten kleiner wird, wenn die
ungeeigneten Nächte zum Beispiel jeweils immer in der gleichen Jahreszeit
auftreten?

Die Poissonverteilung kann auch dazu verwendet werden, die Wahrscheinlichkeit
dafür zu berechnen, dass in einem bestimmten Interval eine bestimmte
Anzahl Ereignisse eintreten, deren Zeitintervalle exponentialverteilt ist.
Dieses Modell ist dann verwendbar, wenn der Prozess, der wolkige Nächte
verursacht, kein Erinnerungsvermögen hat. Dies dürfe wegen der extremen
Seltenheit von wolkigen Nächten auf dem Cerro Armazones der Fall sein.

Die Zeit zwischen wolkigen Nächten ist $1/a=\frac{365}{15}=24.33333$.
\item
Wir überlegen uns, wie
die Zeit $X$ zwischen zwei wolkigen Nächten verteilt ist. Nach den Bemerkungen
in b) ist $X$ eine exponentialverteilte Zufallsvariable mit $1/a=24.3333$
oder $a=0.04109589$.
Sie hat die Verteilungsfunktion
$
P(X\le x)=
F(x)=1-e^{-ax}.
$
Das Programm lässt sich durchführen, wenn die Zeit zwischen zwei
wolkigen Nächten $X> 14$ ist:
\begin{align*}
P(X> 14)&=1-P(X\le 14)=1-F(14)=1-(1-e^{-14a})=e^{-14a}
\\
&= 0.56251219022075086284
\end{align*}
Die Chancen stehen also besser als $50:50$, dass dieses Projekt
erfolgreich durchgeführt werden kann.

Alternativ kann man diese Wahrscheinlichkeit auch so bestimmen:
die Wahrscheinlichkeit, dass ein Nacht klar ist, ist $p=\frac{350}{365}=0.9589$.
Entsprechend ist die Wahrscheinlichkeit, dass 14 aufeinanderfolgende
Nächte klar sind: $p^{14}=0.5557$, in guter "Ubereinstimmung mit der
ersten Lösung.
\qedhere
\end{teilaufgaben}
\end{loesung}

\begin{diskussion}
Mehr Information über das Observatorium auf dem Cerro Armazones
findet man auf \url{http://www.astronomictourism.com/cerro-armazones-observatory.html} oder auf der Website der ESO unter \url{http://www.eso.org}.
\end{diskussion}

\begin{bewertung}
a) Poissonverteilung ({\bf P}) 1 Punkt, Wahrscheinlichkeit ({\bf W}) 1 Punkt.
b) Exponentialverteilung ({\bf E}) 1 Punkt, Parameter $a$ ({\bf A}) 1 Punkt.
c) Lösungsweg ({\bf L}) 1 Punkt, Resultate ({\bf R}) 1 Punkt.
\end{bewertung}
