F"ur einen Multiple Choice Test mit 50 Fragen mit drei m"oglichen Antworten
pro Frage, wovon nur eine richtig ist, soll die f"ur
Bestehen des Testes n"otige Anzahl $k$ korrekt beantworteter Fragen
festgelegt werden.
\begin{teilaufgaben}
\item Wie gross muss $k$ sein, damit ein Kandidat,
der nichts kann, und die Fragen nur nach dem Zufallsprinzip beantwortet,
nur mit einer Wahrscheinlichkeit von h"ochstens 0.1\% den Test besteht?
\item Wie gross ist die Wahrscheinlichkeit, dass ein Kandidat, der jede
einzelne Frage mit Wahrscheinlichkeit $0.7$ richtig beantwortet, den
Test nicht besteht?
\end{teilaufgaben}

\begin{loesung}
\begin{teilaufgaben}
\item Die Anzahl der richtig beantworteten Fragen ist binomialverteilt
mit $n=50$ und $p=\frac13$. Die Wahrscheinlichkeit, dass die Zahl $X$
der korrekten Antworten die Schranke $k$ "uberschreitet, ist $P(X\ge k)$.
Zur einfacheren Rechnung kann dies mit der Normalverteilung f"ur
$\mu = np$ und $\sigma^2=np(1-p)$ approximiert werden. Durch Standardisierung
findet man dann
\begin{align*}
0.001=P(X>k)
&=
P\biggl(
\frac{X-np}{\sqrt{np(1-p)}}>\frac{k-np}{\sqrt{np(1-p)}}
\biggr)
\\
\frac{k-np}{\sqrt{np(1-p)}}&=3.0902
\\
k&= np+3.0902\cdot \sqrt{np(1-p)}
=26.967333333.
\end{align*}
Man muss also verlangen, dass 27 Fragen richtig beantwortet werden.

Mit Hilfe der Binomialverteilung und einem Programm wie R kann man den
Wert auch direkt bekommen:
\texttt{qbinom(0.999, 50, 1/3)}
liefert den Wert $27$.
\item
Auch hier kann man die Normalapproximation verwenden, diesmal aber mit
$p=0.7$:
\begin{align*}
P(X\le k)&=
P\biggl(
\frac{X-np}{\sqrt{np(1-p)}}\le\frac{k-np}{\sqrt{np(1-p)}}
\biggr)
\\
&=F\biggl(
\frac{k-np}{\sqrt{np(1-p)}}
\biggr)
=F(-2.469)=1-F(2.469)=1-0.9932254=0.006774561
\end{align*}
Ein Kandidat, der nur eine Treffsicherheit von $70\%$ hat, hat also
immer noch sehr gute Chancen, den Test zu bestehen.
\qedhere
\end{teilaufgaben}
\end{loesung}

\begin{bewertung}
Binomialverteilung (\textbf{B}) 1 Punkt,
Normalapproximation, d.~h.~die Formeln $\mu=np$ und $\sigma=\sqrt{np(1-p)}$
(\textbf{N}) 1 Punkt,
\begin{teilaufgaben}
\item
Standardisierter $x$-Wert ($3.0902$, \textbf{X}) 1 Punkt,
Wert f"ur $k$ (\textbf{K}) 1 Punkt,
\item
Standardsierter $x$-Wert (\textbf{Y}) 1 Punkt,
Wahrscheinlichkeit (\textbf{W}) 1 Punkt,
\end{teilaufgaben}
\end{bewertung}

