Folgendes Experiment soll herauszufinden erlauben, ob aussersinnliche
Wahrnehmung existiert.
Zwanzig mal wird einer Versuchsperson in einem Web-Browser die Wahl zwischen
zwei Fenstern angeboten. Die Versuchsperson soll vorhersagen,
in welchem Fenster ein Bild erscheinen wird. Es wird gez"ahlt,
wie oft die Versuchsperson das Fenster richtig vorhersagt.
\begin{teilaufgaben}
\item Wieviele Male muss die Versuchsperson richtig vorhersagen,
damit man behaupten kann, es g"abe aussersinnliche Wahrnehmung?
\item Dieses Experiment wurde mit verschiedenen Personen
und mit verschiedenen Kategorien von Bildern durchgef"uhrt,
insgesamt "uber 1000 mal. Wie oft erwarten Sie, dass das
Experiment aussersinnliche Wahrnehmung beweisen wird?
\end{teilaufgaben}

\begin{loesung}
\begin{teilaufgaben}
\item Angenommen es gibt keine mit diesem Experiment nachweisbare
aussersinnliche Wahrnehmung, dann ist die Wahrscheinlichkeit, das
richtige Fenster vorherzusagen, $p=\frac12$. Die Zahl der richtigen
Vorhersagen ist eine binomialverteilte Zufallsvariable $X$ mit
Erwartungswert $np=20\cdot\frac12=10$ und Varianz
$np(1-p)=20\frac12\bigl(1-\frac12\bigr)=5$. Gesucht ist jetzt
der kritische Wert $x_{\text{krit}}$ derart, dass
$P(|X-10| > x_{\text{krit}})=0.95$. Dann wird $X$ nur in 5\% aller
Experimente um mehr als $x_{\text{krit}}$ vom Erwartungswert 10 abweichen.

Da die Verteilungsfunktion einer binomialverteilten Zufallsvariable
m"uhsam zu berechnen ist, approximieren wir sie durch eine Normalverteilung,
wir nehmen also an, dass $X$ normalverteilt ist mit $\mu = 10$
und $\sigma^2=5$. Dann ist $(X-10)/\sqrt{5}$ standardnormalverteilt.
Die Quantilentabelle der Normalverteilung liefert, dass der Betrag
von $(X-10)/\sqrt{5}$
nur in 5\% aller Experimente 1.6449 "ubersteigt. Also ist der kritische
Wert f"ur $X$
\[
x_{\text{krit}} =  \mu + 1.6449 \sigma =  13.67801.
\]
\item
Immer noch unter der Annahme, dass es keine Aussersinnliche Wahrnehmung
gibt, wird das eben beschriebene Experiment in 5\% aller Experimente
einen Wert von $X$ ergeben, der weiter als $3.67801$ von $10$ entfernt
ist. Bei 1000 Experimenten erwartet man also ungef"ahr 50 Durchf"uhrungen,
in denen man glaubt, aussersinnliche Wahrnehmung zu sehen.
\end{teilaufgaben}
Rechnet man mit $\alpha = 0.01$ an Stelle von $\alpha = 0.05$,
ergibt sich als kritischer Wert $x_{\text{krit}}=15.20188$.
\end{loesung}

