Ein Dreik"onigskuchen besteht aus mehreren Teilen, von denen genau einer 
eine kleine
K"onigsfigur enth"alt. Ein B"acker will sich die Arbeit vereinfachen, und
f"ugt die K"onigsfiguren bereits beim Mischen des Teigs hinzu,
wobei er nat"urlich das Risiko eingeht, dass einzelne Kuchen mehrere oder
gar keine K"onigsfiguren enthalten. Nehmen Sie an, dass der B"acker Teig
f"ur $n$ K"onigskuchen mit je 10 Teilen mischt.
\begin{teilaufgaben}
\item Wie gross ist die Wahrscheinlichkeit, dass jeder Dreik"onigskuchen genau
eine K"onigsfigur enth"alt?
\item Wie gross ist die Wahrscheinlichkeit, dass ein Dreik"onigskuchen gar
keine K"onigsfigur enth"alt?
\item Wie gross ist die Wahrscheinlichkeit, dass der erste von
10 Dreik"onigskuchen zwei oder mehr K"onigsfiguren enth"alt?
\item Wie gross ist die Wahrscheinlichkeit, dass in einem Dreik"onigskuchen,
welcher zwei K"onigsfiguren enth"alt, ein Teil
beide K"onigsfiguren enth"alt?
\end{teilaufgaben}

\begin{loesung}
\begin{teilaufgaben}
\item Es gibt $n^n$ Zuteilungen von K"onigsfiguren zu K"onigskuchen.
Davon sind $n!$ Zuteilungen, welche jedem Dreik"onigskuchen genau eine
K"onigsfigur zuteilt, die K"onige also nur ihre ``K"onigreiche''
vertauscht haben. Die Wahrscheinlichkeit, dass jeder Dreik"onigskuchen
genau eine K"onigsfigur enth"alt, ist also
$\frac{n!}{n^n}$.
\item Wenn ein Dreik"onigskuchen keine K"onigsfigur enth"alt, dann muss
ein anderer mehr als eine K"onigsfigur enthalten.
Jeder der Dreik"onigskuchen enth"alt genau dann genau eine K"onigsfigur, wenn
kein Dreik"onigskuchen keine K"onigsfigur enth"alt, also $1-\frac{n!}{n^n}$.
\item Die Wahrscheinlichkeit, dass der erste von $n$ Kuchen genau $k$
enth"alt ist binomialverteilt mit $p=\frac1{n}$. Man kann sich n"amlich
vorstellen, dass das ``Einstreuen'' der K"onigsfiguren ein Experiment
mit zwei m"oglichen Ausg"angen ist: ``Figur ist im ersten Kuchen drin''
bzw.~``Figur ist in einem anderen Kuchen drin''. Somit ist zu bestimmen
\begin{align*}
P(K\ge 2)
&=
1 -P(K=0) - P(K=1)
\\
&=
1 - \binom{n}{0}p^0(1-p)^n
- \binom{n}{1}p(1-p)^n
=
1 - (1-p)^n - np(1-p)^{n-1}
\\
&=
1 - \biggl(\frac{n-1}{n}\biggr)^n
- n\frac1{n}\biggl(\frac{n-1}{n}\biggr)^{n-1}.
\\
&=
1 - \biggl(\frac{n-1}{n}\biggr)^n
- \biggl(\frac{n-1}{n}\biggr)^{n-1}.
\\
&=
1 - \biggl(\frac{n-1}{n}+1\biggr) \biggl(\frac{n-1}{n}\biggr)^{n-1}.
\\
&=
1 - \biggl(\frac{2n-1}{n}\biggr) \biggl(\frac{n-1}{n}\biggr)^{n-1}.
\end{align*}
F"ur $n=10$ findet man:
\[
1-\frac{19}{10}\biggl(\frac{9}{10}\biggr)^9
=
0.2639010709.
\]
\item Es gibt $10^2$ M"oglichkeiten, zwei K"onigsfiguren auf 10 Teile
zu verteilen, bei $10$ davon sind die beiden K"onigsfiguren im gleichen
Teil. Die Wahrscheinlichkeit ist also $\frac1{10}$.

Oder: der ``erste'' K"onig markiert ein Teil, und wir m"ussen jetzt
die Wahrscheinlichkeit bestimmen, dass der ``zweite'' K"onig in
das gleiche Teil kommt. Diese ist nat"urlich $\frac1{10}$.
\qedhere
\end{teilaufgaben}
\end{loesung}

\begin{diskussion}
Mit Hilfe der Stirling-Formel kann man die Wahrscheinlichkeit absch"atzen:
\[
\frac{n!}{n^n}
=
\sqrt{2\pi n}\frac1{e^n}\to0\quad\text{f"ur}\quad n\to\infty.
\]
Die Wahrscheinlichkeit, dass jeder Dreik"onigskuchen genau eine K"onigsfigur
enth"alt, wird also immer kleiner, je gr"osser die Anzahl der Kuchen wird.
\end{diskussion}

