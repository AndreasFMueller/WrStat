Ein Dreikönigskuchen besteht aus mehreren Teilen, von denen genau einer 
eine kleine
Königsfigur enthält. Ein Bäcker will sich die Arbeit vereinfachen, und
fügt die Königsfiguren bereits beim Mischen des Teigs hinzu,
wobei er natürlich das Risiko eingeht, dass einzelne Kuchen mehrere oder
gar keine Königsfiguren enthalten. Nehmen Sie an, dass der Bäcker Teig
für $n$ Königskuchen mit je 10 Teilen mischt.
\begin{teilaufgaben}
\item Wie gross ist die Wahrscheinlichkeit, dass jeder Dreikönigskuchen genau
eine Königsfigur enthält?
\item Wie gross ist die Wahrscheinlichkeit, dass ein Dreikönigskuchen gar
keine Königsfigur enthält?
\item Wie gross ist die Wahrscheinlichkeit, dass der erste von
10 Dreikönigskuchen zwei oder mehr Königsfiguren enthält?
\item Wie gross ist die Wahrscheinlichkeit, dass in einem Dreikönigskuchen,
welcher zwei Königsfiguren enthält, ein Teil
beide Königsfiguren enthält?
\end{teilaufgaben}

\begin{loesung}
\begin{teilaufgaben}
\item Es gibt $n^n$ Zuteilungen von Königsfiguren zu Königskuchen.
Davon sind $n!$ Zuteilungen, welche jedem Dreikönigskuchen genau eine
Königsfigur zuteilt, die Könige also nur ihre ``Königreiche''
vertauscht haben. Die Wahrscheinlichkeit, dass jeder Dreikönigskuchen
genau eine Königsfigur enthält, ist also
$\frac{n!}{n^n}$.
\item Wenn ein Dreikönigskuchen keine Königsfigur enthält, dann muss
ein anderer mehr als eine Königsfigur enthalten.
Jeder der Dreikönigskuchen enthält genau dann genau eine Königsfigur, wenn
kein Dreikönigskuchen keine Königsfigur enthält, also $1-\frac{n!}{n^n}$.
\item Die Wahrscheinlichkeit, dass der erste von $n$ Kuchen genau $k$
enthält ist binomialverteilt mit $p=\frac1{n}$. Man kann sich nämlich
vorstellen, dass das ``Einstreuen'' der Königsfiguren ein Experiment
mit zwei möglichen Ausgängen ist: ``Figur ist im ersten Kuchen drin''
bzw.~``Figur ist in einem anderen Kuchen drin''. Somit ist zu bestimmen
\begin{align*}
P(K\ge 2)
&=
1 -P(K=0) - P(K=1)
\\
&=
1 - \binom{n}{0}p^0(1-p)^n
- \binom{n}{1}p(1-p)^n
=
1 - (1-p)^n - np(1-p)^{n-1}
\\
&=
1 - \biggl(\frac{n-1}{n}\biggr)^n
- n\frac1{n}\biggl(\frac{n-1}{n}\biggr)^{n-1}.
\\
&=
1 - \biggl(\frac{n-1}{n}\biggr)^n
- \biggl(\frac{n-1}{n}\biggr)^{n-1}.
\\
&=
1 - \biggl(\frac{n-1}{n}+1\biggr) \biggl(\frac{n-1}{n}\biggr)^{n-1}.
\\
&=
1 - \biggl(\frac{2n-1}{n}\biggr) \biggl(\frac{n-1}{n}\biggr)^{n-1}.
\end{align*}
Für $n=10$ findet man:
\[
1-\frac{19}{10}\biggl(\frac{9}{10}\biggr)^9
=
0.2639010709.
\]
\item Es gibt $10^2$ Möglichkeiten, zwei Königsfiguren auf 10 Teile
zu verteilen, bei $10$ davon sind die beiden Königsfiguren im gleichen
Teil. Die Wahrscheinlichkeit ist also $\frac1{10}$.

Oder: der ``erste'' König markiert ein Teil, und wir müssen jetzt
die Wahrscheinlichkeit bestimmen, dass der ``zweite'' König in
das gleiche Teil kommt. Diese ist natürlich $\frac1{10}$.
\qedhere
\end{teilaufgaben}
\end{loesung}

\begin{diskussion}
Mit Hilfe der Stirling-Formel kann man die Wahrscheinlichkeit abschätzen:
\[
\frac{n!}{n^n}
=
\sqrt{2\pi n}\frac1{e^n}\to0\quad\text{für}\quad n\to\infty.
\]
Die Wahrscheinlichkeit, dass jeder Dreikönigskuchen genau eine Königsfigur
enthält, wird also immer kleiner, je grösser die Anzahl der Kuchen wird.
\end{diskussion}

