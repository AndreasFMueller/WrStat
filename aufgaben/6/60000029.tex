Von einer Messgr"osse $X$ mit Erwartungswert $1$ und Varianz $1$
kann ein Sensor jeweils nur die positiven Werte messen.
Da man die negativen Werte nicht ``sieht'', wird der Mittelwert 
den Erwartungswert "ubersch"atzen.
Wir m"ochten gerne wissen, um wieviel der Mittelwert den Erwartungswert
von $X$ "ubersch"atzt.
Wie gross wird der Mittelwert einer sehr grossen Zahl von Sensorwerten sein?

\begin{hinweis}
$\int xe^{-\frac{x^2}2}\,dx = -e^{-\frac{x^2}2}.$
\end{hinweis}

\begin{loesung}
Wir m"ussen den Erwartungswert der Werte von $X$ berechnen, die gr"osser
als $0$ sind. Es ist klar dass die Dichtefunktion etwa so aussieht:
\[
\varphi(x)=\begin{cases}
0&\qquad x<0\\
\frac1{\sqrt{2\pi}}e^{-\frac{(x-1)^2}2}&\qquad x\ge 0
\end{cases}
\]
Allerdings stimmt die Normierung nicht, da wir die Werte $x < 0$ weggelassen
haben, ist das Integral "uber die ganze reelle Achse jetzt nicht mehr $1$,
sondern geringer. Durch diesen Wert mu"ssen wir teilen. Das Integral ist
\[
\frac1{\sqrt{2\pi}}
\int_{0}^\infty
e^{-\frac{(x-1)^2}2}\,dx=
P(X > 0).
\]
Der Korrekturfaktor, durch den zu teilen ist, ist also $P(X>0)$.

Die Wahrscheinlichkeit, dass $X>0$ ist, ist f"ur $\mu=E(X)$
\begin{align*}
P(X>0)&=P(X_s > -\mu)=1-P(X_s\le-\mu)=1-F(-\mu)=F(\mu),
\end{align*}
wobei $X_s$ eine standardnormalverteilte Zufallsvariable und $F$ die
Verteilungsfunktion der Standardnormalverteilung ist.

Die Wahrscheinlichkeitsdichte der Sensorwerte ist
\[
\varphi(x)=\begin{cases}
\displaystyle
\frac1{F(\mu)\sqrt{2\pi}}e^{-\frac{(x-\mu)^2}2}&\qquad x > 0\\
0&\qquad\text{sonst}
\end{cases}
\]
Damit kann man jetzt
den Erwartungswert der Werte $X>\mu$ berechnen:
\begin{align*}
E(X| X > 0)
&=
\frac1{F(\mu)}\cdot\frac1{\sqrt{2\pi}}\int_0^\infty x\,e^{-\frac{(x-\mu)^2}2}\,dx
\\
&=
\mu + \frac1{F(\mu)}\cdot\frac1{\sqrt{2\pi}}\int_{-\mu}^\infty x\,e^{-\frac{x^2}2}\,dx
\\
&=
\mu + \frac1{F(\mu)}
\cdot\frac1{\sqrt{2\pi}}
\left[
-e^{-\frac{x^2}2}
\right]_{-\mu}^\infty
=\mu + \frac{e^{-\frac{\mu^2}2}}{F(\mu)\sqrt{2\pi}}
\end{align*}
Setzen wir $\mu=1$ ein, bekommen wir
\begin{align*}
E(X|X>0)
&
=
1+\frac{e^{-\frac12}}{F(1)\sqrt{2\pi}}
=
1+\frac{e^{-\frac12}}{0.8413\cdot \sqrt{2\pi}}=1.28761526746599708760
\end{align*}
Wir "ubersch"atzen also den Erwartungswert von $X$ um mehr als 28\%.
\end{loesung}

\begin{bewertung}
Normalverteilung ({\bf N}) 1 Punkt,
Problemanalyse ({\bf A}) 1 Punkt,
Korrektur der Normierung ({\bf K}) 1 Punkt,
Formel f"ur Erwartungswert ({\bf E}) 1 Punkt,
\end{bewertung}
