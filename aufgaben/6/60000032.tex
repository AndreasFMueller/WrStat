Inspektor Columbo möchte herausfinden, ob in einem Spielkasino
am Roulette-Tisch betrogen wird.
Da dort nicht gern gesehen wird, wenn sich die Leute umfangreiche Notizen
machen, zählt er im Kopf die Anzahl $X$ der ``roten'' Zahlen,
die im Laufe eines Abends auftreten.
Wir gehen davon aus, dass an einem Abend 100 Spiele gespielt werden.
\begin{teilaufgaben}
\item Geben Sie ein Intervall für die Werte von $X$ an, die ``unverdächtig''
sind. Wählen Sie als Kriterium für ``unverdächtig'' ein Intervall so, dass
die Anzahl $X$ bei einem ehrlichen Casino nur in 1\% der Fälle nicht
in dem Intervall liegen würde.
\item Um den vermuteten Betrügern das Handwerk zu legen, muss Columbo
etwas Hartnäckigkeit beweisen, und geht daher jeden Abend erneut
ins Kasino um ``Rot'' zu zählen. Durch Zufall kann es vorkommen, dass
das Kriterium, welches Sie in Teilaufgabe a) für ``verdächtige''
Werte von $X$ festgelegt haben, erfüllt wird. Wie wahrscheinlich ist,
dass das Kriterium an 30 Abenden mindestens einmal erfüllt wird, selbst
wenn alle im Kasino ehrlich arbeiten?
\end{teilaufgaben}

\thema{Binomialverteilung}
\thema{Normalapproximation}
\thema{Standardisierung}

\begin{loesung}
\begin{teilaufgaben}
\item
``Rot'' und ``Nicht-Rot'' sind die Ausgänge eines Bernoulli-Experiments mit
Wahrscheinlichkeit $p=\frac{18}{37}$. Die Anzahl $X$ von ``Rot'' ist daher
eine binomialverteilte Zufallsvariable. Wegen der grossen Zahl $n=100$
von Spielen, verwenden wir die Normalapproximation, nach der 
\[
\frac{X-E(X)}{\sqrt{\operatorname{var}(X)}}=\frac{X-np}{\sqrt{np(1-p)}}
\]
standardnormalverteilt ist. Verdächtig sind Werte von $X$, die weit
weg vom Erwartungswert $np$ sind. Wir wählen die Schranken des Intervalls
so, dass bei einem ehrlichen Betrieb die Zahl $X$ nur mit Wahrscheinlichkeit
1\% ausserhalb des Intervalls liegt. In anbetracht der Konsequenzen
für das Kasino, wenn sich Columbo irrt, ist das soger eher noch ein
grosser Wert. Nach der Quantilentabelle für die
Standardnormalverteilung muss dafür die Zahl $X$ um mehr als $2.5758\sigma$
von $np$ abweichen, also
\[
\left.
\begin{aligned}
\mu&=48.65\\
\sigma&=4.998
\end{aligned}
\quad
\right\}
\qquad
\Rightarrow
\qquad
\left\{\quad
\begin{aligned}
\mu - 2.5758\sigma&=35.77\\
\mu + 2.5758\sigma&=61.52
\end{aligned}
\right.
\]
Anzahlen im Intervall $[35,61]$ sind also unverdächtig.

Geht man statt von einer grünen Null von zwei grünen Nullen aus, verschiebt
sich die Wahrscheinlichkeit leicht, es ist jetzt $p=\frac{18}{38}$, dadurch
ergeben sich jetzt folgende Zahlenwerte:
\[
\left.
\begin{aligned}
\mu&=47.65\\
\sigma&=4.993
\end{aligned}
\quad
\right\}
\qquad
\Rightarrow
\qquad
\left\{\quad
\begin{aligned}
\mu - 2.5758\sigma&=34.51\\
\mu + 2.5758\sigma&=60.23
\end{aligned}
\right.
\]

Und schliesslich kann man Wahrscheinlichkeit 10\% statt 1\% arbeiten, 
es ist dann $2.5758$ durch $1.6449$ zu ersetzen, es ergeben sich
die Intervalle $[40.43, 56.87]$ bzw.~$[39.16, 55.58]$.
\item
Das Kriterium war so gewählt, dass es nur in 1\% der Fälle zufällig
eintreten kann. Sei $A_i$ das Ereignis, dass das Kriterium am Abend $i$
eingetritt, $P(A_i)=0.01$.
Die Wahrscheinlichkeit, dass es an keinem Abend eintritt,
ist daher 
\begin{align*}
P(\text{nie an 30 Abenden})&=P(\overline{A_1\cup A_2\cup\dots\cup A_{30}})
=P(
\overline{A}_1
\cap
\overline{A}_2
\cap
\dots
\cap
\overline{A}_{30}
)
\\
&=
P(\overline{A}_1)
\cdot
P(\overline{A}_2)
\cdot
\dots
\cdot
P(\overline{A}_{30})
=0.99^{30}=0.7397
\end{align*}
Mit Wahrscheinlichkeit $1-0.7397=0.2603$ wird also das Kriterium an mindestens
einem der 30 Abende erfüllt sein.

Wenn man in a) mit 10\% gearbeitet hat, dann wird die Wahrscheinlichkeit in
$b$ natürlich auch grösser: $0.9^{30}=0.0424$, die Wahrscheinlichkeit ist
dann $0.9576$. Daran kann man sehen, dass $10\%$ ein zu grobes Kriterium
für diese Art Untersuchung ist.
\qedhere
\end{teilaufgaben}
\end{loesung}

\begin{bewertung}
\begin{teilaufgaben}
\item Binomialverteilung mit Normalapproximation ({\bf B}) 1 Punkt,
Erwartungswert und Varianz ({\bf E}) 1 Punkt,
Wahl einer geeigneten Wahrscheinlichkeit für ein Intervall ({\bf A}) 1 Punkt,
Berechnung der Intervallgrenzen ({\bf G}) 1 Punkt.
\item Iteriertes Bernoulli-Experiment mit Wahrscheinlichkeit 1\% ({\bf I})
1 Punkt,
Produktformel und Resultat ({\bf P}) 1 Punkt.
\end{teilaufgaben}
\end{bewertung}


