Inspektor Columbo m"ochte herausfinden, ob in einem Spielkasino
am Roulette-Tisch betrogen wird.
Da dort nicht gern gesehen wird, wenn sich die Leute umfangreiche Notizen
machen, z"ahlt er im Kopf die Anzahl $X$ der ``roten'' Zahlen,
die im Laufe eines Abends auftreten.
Wir gehen davon aus, dass an einem Abend 100 Spiele gespielt werden.
\begin{teilaufgaben}
\item Geben Sie ein Interval f"ur die Werte von $X$ an, die ``unverd"achtig''
sind. W"ahlen Sie als Kriterium f"ur ``unverd"achtig'' ein Interval so, dass
die Anzahl $X$ bei einem ehrlichen Casino nur in 1\% der F"alle nicht
in dem Interval liegen w"urde.
\item Um den vermuteten Betr"ugern das Handwerk zu legen, muss Columbo
etwas Hartn"ackigkeit beweisen, und geht daher jeden Abend erneut
ins Kasino um ``Rot'' zu z"ahlen. Durch Zufall kann es vorkommen, dass
das Kriterium, welches Sie in Teilaufgabe a) f"ur ``verd"achtige''
Werte von $X$ festgelegt haben, erf"ullt wird. Wie wahrscheinlich ist,
dass das Kriterium an 30 Abenden mindestens einmal erf"ullt wird, selbst
wenn alle im Kasino ehrlich arbeiten?
\end{teilaufgaben}

\begin{loesung}
\begin{teilaufgaben}
\item
``Rot'' und ``Nicht-Rot'' sind die Ausg"ange eines Bernoulli-Experiments mit
Wahrscheinlichkeit $p=\frac{18}{37}$. Die Anzahl $X$ von ``Rot'' ist daher
eine binomialverteilte Zufallsvariable. Wegen der grossen Zahl $n=100$
von Spielen, verwenden wir die Normalapproximation, nach der 
\[
\frac{X-E(X)}{\sqrt{\operatorname{var}(X)}}=\frac{X-np}{\sqrt{np(1-p)}}
\]
standardnormalverteilt ist. Verd"achtig sind Werte von $X$, die weit
weg vom Erwartungswert $np$ sind. Wir w"ahlen die Schranken des Intervals
so, dass bei einem ehrlichen Betrieb die Zahl $X$ nur mit Wahrscheinlichkeit
1\% ausserhalb des Intervals liegt. In anbetracht der Konsequenzen
f"ur das Kasino, wenn sich Columbo irrt, ist das soger eher noch ein
grosser Wert. Nach der Quantilentabelle f"ur die
Standardnormalverteilung muss daf"ur die Zahl $X$ um mehr als $2.5758\sigma$
von $np$ abweichen, also
\[
\left.
\begin{aligned}
\mu&=48.65\\
\sigma&=4.998
\end{aligned}
\quad
\right\}
\qquad
\Rightarrow
\qquad
\left\{\quad
\begin{aligned}
\mu - 2.5758\sigma&=35.77\\
\mu + 2.5758\sigma&=61.52
\end{aligned}
\right.
\]
Anzahlen im Interval $[35,61]$ sind also unverd"achtig.

Geht man statt von einer gr"unen Null von zwei gr"unen Nullen aus, verschiebt
sich die Wahrscheinlichkeit leicht, es ist jetzt $p=\frac{18}{38}$, dadurch
ergeben sich jetzt folgende Zahlenwerte:
\[
\left.
\begin{aligned}
\mu&=47.65\\
\sigma&=4.993
\end{aligned}
\quad
\right\}
\qquad
\Rightarrow
\qquad
\left\{\quad
\begin{aligned}
\mu - 2.5758\sigma&=34.51\\
\mu + 2.5758\sigma&=60.23
\end{aligned}
\right.
\]

Und schliesslich kann man Wahrscheinlichkeit 10\% statt 1\% arbeiten, 
es ist dann $2.5758$ durch $1.6449$ zu ersetzen, es ergeben sich
die Intervalle $[40.43, 56.87]$ bzw.~$[39.16, 55.58]$.
\item
Das Kriterium war so gew"ahlt, dass es nur in 1\% der F"alle zuf"allig
eintreten kann. Sei $A_i$ das Ereignis, dass das Kriterium am Abend $i$
eingetritt, $P(A_i)=0.01$.
Die Wahrscheinlichkeit, dass es an keinem Abend eintritt,
ist daher 
\begin{align*}
P(\text{nie an 30 Abenden})&=P(\overline{A_1\cup A_2\cup\dots\cup A_{30}})
=P(
\overline{A}_1
\cap
\overline{A}_2
\cap
\dots
\cap
\overline{A}_{30}
)
\\
&=
P(\overline{A}_1)
\cdot
P(\overline{A}_2)
\cdot
\dots
\cdot
P(\overline{A}_{30})
=0.99^{30}=0.7397
\end{align*}
Mit Wahrscheinlichkeit $1-0.7397=0.2603$ wird also das Kriterium an mindestens
einem der 30 Abende erf"ullt sein.

Wenn man in a) mit 10\% gearbeitet hat, dann wird die Wahrscheinlichkeit in
$b$ nat"urlich auch gr"osser: $0.9^{30}=0.0424$, die Wahrscheinlichkeit ist
dann $0.9576$. Daran kann man sehen, dass $10\%$ ein zu grobes Kriterium
f"ur diese Art Untersuchung ist.
\end{teilaufgaben}
\end{loesung}

\begin{bewertung}
\begin{teilaufgaben}
\item Binomialverteilung mit Normalapproximation ({\bf B}) 1 Punkt,
Erwartungswert und Varianz ({\bf E}) 1 Punkt,
Wahl einer geeigneten Wahrscheinlichkeit f"ur ein Interval ({\bf A}) 1 Punkt,
Berechnung der Intervallgrenzen ({\bf G}) 1 Punkt.
\item Iteriertes Bernoulli-Experiment mit Wahrscheinlichkeit 1\% ({\bf I})
1 Punkt,
Produktformel und Resultat ({\bf P}) 1 Punkt.
\end{teilaufgaben}
\end{bewertung}


