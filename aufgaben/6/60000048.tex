Internationale Sportwettkämpfe suggerieren immer wieder, dass grosse
Länder es offenbar leichter haben, Sporttalente zu rekrutieren.
Nehmen wir vereinfachend an, dass alle Menschen in einer
bestimmten Sportart unabhängig von der Herkunft vergleichbare,
aber natürlich normalverteilte Resultate erzielen können.

Ein Ausnahmetalent soll jemand sein, der 4 Standardabweichungen besser ist
als der Durchschnitt, dies tritt mit Wahrscheinlichkeit $p = 0.0000317$ ein.
Wir nehmen, ebenfalls vereinfachend, an, dass $10\%$ der Bevölkerung
eines Landes in einem Alter sind, in dem solche Spitzenleistungen
möglich sind.
\begin{teilaufgaben}
\item
Mit welcher Anzahl von Ausnahmetalenten kann ein Land wie die Schweiz mit 
8.3 Mio Einwohnern rechnen?
\item
Das Grossherzogtum Luxemburg hat 576249 Einwohner (1.~Januar 2016).
Wie gross ist die Wahrscheinlichkeit, dass Luxemburg eine Olympiamannschaft
rekurtieren kann, die mindestens 4 Ausnahmetalente enthält?
\end{teilaufgaben}

\begin{loesung}
Ausnahmetalente sind selten, wir können daher die Poissonverteilung als
Approximation der Binomialverteilung verwenden.
\begin{teilaufgaben}
\item
Ein Ausnahmetalent zu sein ist ein Binomialexperiment mit $p=0.0000317$,
angewendet auf die $n=830000$ Einwohner im spitzensportfähigen Alter
erwarten wir $np=26.287$ Ausnahmetalente in der Schweiz.
\item
Luxemburg kann nach der gleichen Rechnung nur mit $\lambda=np=1.8267$
Ausnahmetalenten rechnen.
Sei $X$ die Anzahl der Ausnahmetalente, sie ist poissonverteilt mit
Parameter $\lambda$.
Die Wahrscheinlichkeit, genau $k$ Ausnahmetaleten zu finden, ist
\[
P_\lambda(k)=e^{-\lambda}\frac{\lambda^k}{k!}.
\]
Gesucht ist die Wahrscheinlichkeit, mindestens 4 Ausnahmetalente zu finden:
\begin{align*}
P(X \ge 4)
&=
1-P(X<4)=1-\sum_{k=0}^3 e^{-\lambda}\biggl(
1+\lambda+ \frac{\lambda^2}{2} + \frac{\lambda^3}{6}
\biggr)
\\
&=
1-0.16094\cdot(1+1.8267+1.6684+1.0159)
=
1-0.16094\cdot 5.5111\\
&=
0.11304.
\end{align*}
Die Wahrscheinlichkeit, dass Luxemburg eine Mannschaft mit mindestens
vier Ausnahmetalenten aufstellen kann, ist also nur $11.3\%$.

Man kann in diesem Fall die Wahrscheinlichkeit auch direkt mit der
Binomialverteilung ausrechnen, also als
\[
1-\sum_{k=0}^3\binom{n}{k}p^k(1-p)^{n-k}
=
1-0.887 = 11.3\%
\]
mit $n=57625$.
Die Poisson-Approximation ist also sehr genau.
\qedhere
\end{teilaufgaben}
\end{loesung}

\begin{bewertung}
\begin{teilaufgaben}
\item
Binomialverteilung ({\bf B}) 1 Punkt, Erwartungswert ({\bf E}) 1 Punkt.
\item
Poissonverteilung ({\bf P}) 1 Punkt, 
Parameter $\lambda$ ({\bf L}) 1 Punkt,
Formel für Wahrscheinlichkeit ({\bf F}) 1 Punkt,
Wert ({\bf W}) 1 Punkt.
\end{teilaufgaben}
\end{bewertung}

