Internationale Sportwettk"ampfe suggerieren immer wieder, dass grosse
L"ander es offenbar leichter haben, Sporttalente zu rekrutieren.
Nehmen wir vereinfachend an, dass alle Menschen in einer
bestimmten Sportart unabh"angig von der Herkunft vergleichbare,
aber nat"urlich normalverteilte Resultate erzielen k"onnen.

Ein Ausnahmetalent soll jenad sein, der 4 Standardabweichungen besser ist
als der Durchschnitt, dies tritt mit Wahrscheinlichkeit $p = 0.0000317$ ein.
Wir nehmen, ebenfalls vereinfachend, an, dass $10\%$ der Bev"olkerung
eines Landes in einem Alter sind, in dem solche Spitzenleistungen
m"oglich sind.
\begin{teilaufgaben}
\item
Mit welcher Anzahl von Ausnahmetalenten kann ein Land wie die Schweiz mit 
8.3 Mio Einwohnern rechnen!
\item
Das Grossherzogtum Luxenburg hat 576249 Einwohner (1.~Januar 2016).
Wie gross ist die Wahrscheinlichkeit, dass Luxenburg eine Olympiamannschaft
rekurtieren kann, die mindestens 4 Ausnahmetalente enth"alt?
\end{teilaufgaben}

\begin{loesung}
Ausnahmetalente sind selten, wir k"onnen daher die Poissonverteilung als
Approximation der Binomialverteilung verwenden.
\begin{teilaufgaben}
\item
Ein Ausnahmetalent zu sein ist ein Binomialexperiment mit $p=0.0000317$,
angewendet auf die $n=830000$ Einwohner im spitzensportf"ahigen Alter
erwarten wir $np=26.287$ Ausnahmetalente in der Schweiz.
\item
Luxenburk kann nach der gleichen Rechnung nur mit $\lambda=np=1.8267$
Ausnahetalenten rechnen.
Sei $X$ die Anzahl der Ausnahmetalente, sie ist poissonverteilt mit
Parameter $\lambda$.
Die Wahrscheinlichkeit, genau $k$ Ausnahmetaleten zu finden, ist
\[
P_\lambda(k)=e^{-\lambda}\frac{\lambda^k}{k!}.
\]
Gesucht ist die Wahrscheinlichkeit, mindestens 4 Ausnahmetalente zu finden:
\begin{align*}
P(X \ge 4)
&=
1-P(X<4)=1-\sum_{k=0}^3 e^{-\lambda}\biggl(
1+\lambda+ \frac{\lambda^2}{2} + \frac{\lambda^3}{6}
\biggr)
\\
&=
1-0.16094\cdot(1+1.8267+1.6684+1.0159)1-0.16094\cdot 5.5111\\
&=
0.11304.
\end{align*}
Die Wahrscheinlichkeit, dass Luxenburg eine Mannschaft mit mindestens
vier Ausnahmetalenten aufstellen kan, ist also nur $11.3\%$.
\end{teilaufgaben}
\end{loesung}

\begin{bewertung}
\end{bewertung}

