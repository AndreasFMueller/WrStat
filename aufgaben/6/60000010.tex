Ein Schiessbudenbetreiber möchte seine Gäste dazu animieren,
längere Serien zu schiessen. Dazu verspricht er jedem, der eine
gewisse Punktzahl überschreitet, eine Sonderprämie.
\begin{teilaufgaben}
\item
Die Kunden schiessen 20 Schuss auf 10er Scheiben. Um sein Risiko
zu kontrollieren schreibt er sich die Resultat seiner Kunden auf,
und findet heraus, dass sie im Mittel 70 Punkte erzielen, mit einer
Varianz von 36. Damit die Aktion genügend Wirkung entfaltet,
muss die Sonderprämie auch tatsächlich von Zeit zu Zeit
gewonnen werden, er glaubt, dass ein Gewinn der Sonderprämie
bei jedem zwanzigsten Kunden diesen Effekt haben sollte. Wie
muss er die Punkte-Limite für die Prämie ansetzen, damit dies
passiert?
\item
Die Kunden schiessen 30 Schuss auf vorbeifahrende Figürchen, die
sie erfahrungsgemäss in zwei Dritteln der Fälle treffen. Wie
gross muss der Schiessbudenbetreiber die für die Sonderprämie
zu erziehlende Anzahl Treffer ansetzen, damit die Sonderprämie
etwa von jedem hundertsten Kunden gewonnen wird?
\end{teilaufgaben}

\thema{Zentraler Grenzwertsatz}
\thema{Normalverteilung}
\thema{Standardisierung}

\begin{loesung}
\begin{teilaufgaben}
\item
Die erziehlte Punktzahl eines Gastes ist eine Zufallsvariable $X$.
$X$ ist angenähert normalverteilt mit $\mu=70$ und
$\sigma=6$. Gesucht ist die Schranke $x_0$ so dass
$P(X\ge x_0)=0.05$ oder $P(X\le x_0)=F_X(x_0)=0.95$.
Da $X$ nicht standardnormalverteilt ist, können wir $x_0$ nicht
aus der Normalverteilungstabelle erhalten, wir können dies aber
durch Standardisierung korrigieren:
\begin{align*}
P(X\le x_0)
&=
P\left(\frac{X-\mu}{\sigma}\le\frac{x_0-\mu}{\sigma}\right)
=
P\left(Z\le \frac{x_0-\mu}{\sigma}\right)
\end{align*}
$Z=(X-\mu)/\sigma$, den Wert für $(x_0-\mu)/\sigma=1.6449$ erhalten
wir aus der Quantilen-Tabelle der Normalverteilung. Also ist
$x_0=\mu +1.6449\sigma=70 + 1.6449\cdot 6=79.8694$ anzusetzen.
\item
Die Zahl $X$ der Treffer ist binomialverteilt mit $E(X) = np=20$ und
$\operatorname{var}(X)=np(1-p)=30\cdot\frac23\cdot\frac13=\frac{60}{9}$,
$\sigma = \frac23\sqrt{15}=2.58199$.
Gesucht ist der Wert $x_0$ so, dass $P(X>x_0)=0.01$, oder
$P(X\le x_0)=F_X(x_0)=0.99$. Wieder durch Standardisierung bekommt
man
\begin{align*}
P(X\le x_0)
&=
P\left(
\frac{X-\mu}{\sigma}\le\frac{x_0-\mu}{\sigma}
\right)
=
P\left(
Z\le \frac{x_0-\mu}{\sigma}
\right)
\end{align*}
$Z$ ist angenähert Standardnormalverteilt, also bekommt man aus
der Quantilen-Tabelle:
\[
\frac{x_0-\mu}{\sigma}=2.3263
\quad
\Rightarrow
\quad
x_0=\mu+2.3263\cdot \sigma=20 + 2.3263\cdot 2.58199 = 26.0065.
\qedhere
\]
\end{teilaufgaben}
\end{loesung}

