Der Median $\operatorname{med}(X)$ einer Zufallsvariable $X$
ist derjenige $x$-Wert, f"ur den die
Verteilungsfunktion den Wert $\frac12$ "uberschreitet. In einer
Stichprobe wird die H"alfte der Werte gr"osser als der Median
sein, die andere H"alfte kleiner als der Median. Asymmetrische
Verteilungen haben oft vom Mittelwert abweichenden Median.

\begin{teilaufgaben}
\item Bestimmen Sie den Median der Gleichverteilung.
\item Zeigen Sie: bei einer symmetrischen Verteilung einer
stetigen Zufallsvariable sind Median
und Erwartungswert gleich gross.
\item Bestimmen Sie den Median der Exponentialverteilung mit Parameter $a$.
\end{teilaufgaben}
Der Median der Exponentialverteilung heisst auch Halbwertszeit.

\begin{loesung}
\begin{teilaufgaben}
\item Die Verteilungsfunktion der Gleichverteilung ist
\[
F(x)=\frac{x-a}{b-a}
\]
f"ur $x$-Werte zwischen $a$ und $b$. Diese Funktion nimmt den
Wert $\frac12$ an, wenn
\[
\frac{x-a}{b-a}=\frac12
\quad\Rightarrow\quad
x-a=\frac12(b-a)
\quad\Rightarrow\quad
x=\frac{a+b}2=E(X).
\]
\item Ist die Verteilung symmetrisch um den Wert $\mu=E(X)$, dann
ist $P(x<\mu)=P(x>\mu)$. Zusammen geben diese Teile aber $1$, also
muss $P(x<\mu)=P(x>\mu)=\frac12$ sein, $\mu$ ist also auch der Median.
\item
Wir kennen die Verteilungsfunktion, und k"onnen daher den Median
leicht ausrechnen. Ist $m=\operatorname{med}(X)$, dann muss
gelten $F(m)=\frac12$. Also auch
\begin{align*}
\frac12&=1-e^{-am}\\
e^{-am}&=\frac12\\
m&=-\frac1a\log\frac12=\frac1a\log 2\simeq\frac{0.6931}{a}
\end{align*}
Dabei ist mit $\log$ der nat"urliche Logarithmus gemeint.
\end{teilaufgaben}
\end{loesung}

