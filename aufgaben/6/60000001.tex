Ein intelligenter Sensor sendet im Zeitabstand von $T=\text{5 min}$
Messdaten an einen zentralen Server. Um "Uberlastung des Servers
vorzubeugen, wird jedes
Zeitinterval um einen zuf"alligen Betrag vergr"ossert oder verkleinert, so
dass die "Ubertragung Nummer $i$ ein Zeitinterval $T_i$ nach "Ubertragung
Nummer $i-1$ stattfindet.
Die $T_i$ sind im Interval $[T-\delta,T+\delta]$
gleichverteilte Zufallsvariable, $\delta = \text{1min}$.
So ist sichergestellt, dass sich die "Ubermittlungszeitpunkt etwas verteilen.
\begin{teilaufgaben}
\item Finden Sie Erwartungswert und Varianz von $T_i$
\item Pro Tag sollten im Schnitt 288 "Ubertragungen stattfinden. Wie gross
ist die Wahrscheinlichkeit, dass 24 Stunden nach "Ubertragung Nummer $0$
nicht mehr als 280 "Ubetragungen stattgefunden haben.
\end{teilaufgaben}

\begin{loesung}
\begin{teilaufgaben}
\item F"ur eine gleichverteilte Zufallsvariable wurden Erwartungswert
und Varianz in der Vorlesung berechnet.
Der Erwartungswert ist der Mittelpunkt des Intervals und die
Varianz ist $\frac1{12}l^2$, wenn $l$ die Intervall"ange ist.
Also gilt
\begin{align*}
E(T_i)&=T\\
\operatorname{var}(T_i)&=\frac13\delta^2.
\end{align*}
\item
Die Frage, ob nach 24 Stunden nicht mehr als 280 "Ubertragungen
stattgefunden haben ist gleichbedeutend damit, ob nach 280 "Ubertragungen
die 24 Stunden schon um sind. Ist $t(N)$ also die Zeit f"ur $N$
"ubertragungen, dann gilt
\begin{equation}
P(\text{nicht mehr als $N$ "Ubertragungen in 24 Stunden})
=
P(t(N) \ge 24\text{h}).
\label{pt280}
\end{equation}
Die Zeit f"ur $N$ "Ubertragungen ist
\[
t(N)=\sum_{i=1}^NT_i,
\]
dies ist eine Zufallsvariable, die als Summe ziemlich vieler
Zufallsvariablen mit gleicher Varianz ann"ahernd normalverteilt ist
(zentraler Grenzwertsatz). Wir m"ussen also nur Erwartungswert und
Varianz von $t(N)$ bestimmen, dann k"onnen wir mit Hilfe der Normalverteilung
Wahrscheinlichkeiten wie~(\ref{pt280}) bestimmen.

Es gilt
\begin{align*}
E(t(N))
&=
E\biggl(\sum_{i=1}^{N}T_i\biggr)
= \sum_{i=1}^{280}E(T_i)=NT\\
\operatorname{var}(t(N))
&=
\operatorname{var}\biggl(\sum_{i=1}^{N}T_i\biggr)
= \sum_{i=1}^{280}\operatorname{var}(T_i)=\frac{N\delta^2}3.
\end{align*}
Jetzt k"onnen wir die Berechnung der Wahrscheinlichkeit~(\ref{pt280})
durchf"uhren
\[
P(t(N)\ge \text{24h})=
P\biggl(
\frac{t(N)-E(t(N))}{\sqrt{\operatorname{var}(t(N))}}
\ge
\frac{\text{24h}-E(t(N))}{\sqrt{\operatorname{var}(t(N))}}
\biggr)
\]
In der rechten Seite steht links eine standardnormalverteilte
Zufallsvariable $X$, rechts eine Konstante. Also gilt
\begin{align*}
P(t(N)\ge \text{24h})
&
=
P\biggl(
X\ge
\frac{\text{24h}-NT}{\sqrt{\frac{N\delta^2}3}}
\biggr)
=1-F_X\biggl(
\frac{3\cdot(\text{24h}-NT)}{\sqrt{N}\delta}
\biggr)\\
&
=1-F_X\biggl(
\sqrt{\frac{3}{N}}
\frac{\text{24h}-NT)}{\delta}
\biggr)
=1-F_X\biggl(
\frac{\sqrt{3N}}{\delta}
\left(\frac{\text{24h}}{N}-T\right)
\biggr).
\end{align*}
Darin ist $F_X$ die Verteilungsfunktion einer standardnormalverteilten
Zufallsvariable.

Im vorliegenden Fall ist $T=\text{5min}$ und $N=280$ und $\delta=\text{1min}$.
Daraus ergibt sich $\frac{\text{24h}}{280}=5.142857\text{min}$ und
\begin{align*}
P(t(280)\ge \text{24h})
&=
1-F_X(
\sqrt{3\cdot 280}\frac{0.142857}{5}
)
=1-F_X(0.82807867)
\\
&
=0.203813.
\qedhere
\end{align*}
\end{teilaufgaben}
\end{loesung}

