Mit der Verkleinerung der Strukturen von Halbleiterchips wurde auch deren
Anfälligkeit für durch natürliche Strahlung verursachte Einzelbit-Fehler
(soft errors)
höher.
Die {\em Failuers in Time} (FIT) Zahl bezeichnet die zu erwartende Anzahl
Fehler eines Chips pro Milliarde Betriebsstunden.
In der Wikipedia liest man von einem Experiment, welches 5950 FIT
gemessen hat.

In einem Computer werden 16 dieser Chips eingesetzt.
Wie gross ist die Wahrscheinlichkeit, dass innerhalb eines
Jahres mehr als ein Fehler auftritt?

\thema{Poisson-Verteilung}

\begin{loesung}
In einer Stunde akkumulieren die 16 Chips 16 Betriebsstunden, die erwartete
Anzahl Fehler ist also
\[
\lambda = 16\cdot365\cdot 24 \cdot \frac{5950}{10^{9}}=0.833952
\]
Mit Hilfe der Poisson-Verteilung findet man die Wahrscheinlichkeit für
genau $k$ Fehler:
\begin{align*}
P_{\lambda}(0)&=0.434294\\
P_{\lambda}(1)&=0.362099\\
P_{\lambda}(2)&=0.151033\\
P_{\lambda}(3)&=0.041985\\
P_{\lambda}(4)&=0.008754
\end{align*}
Die Wahrscheinlichkeit dafür, dass mehr als ein Fehler auftritt, ist
\[
P(k > 1)=1-P_\lambda(0)-P_\lambda(1)=0.2034607
\qedhere
\]
\end{loesung}


