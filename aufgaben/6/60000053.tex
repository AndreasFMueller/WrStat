Ein Läufer läuft am Zürcher Silvesterlauf die Strecke von 10km in
einer Zeit von 0:44:30 und landet dabei auf Rang 1000 von 4000 Läufern.
Die durchschnittliche Laufzeit aller Läufer beträgt 0:48:55.
\begin{teilaufgaben}
\item 
Bestimmen Sie die Standardabweichung der Laufzeiten.
\item
Wie schnell muss man laufen, um unter die ersten 100 Läufer zu gelangen?
\item
Welcher Prozentsatz der Läufer braucht mehr als eine Stunde für die 10km?
\end{teilaufgaben}

\begin{loesung}
Die Laufzeit eines Läufers ist eine Zufallsvariable $X$.
Wir dürfen davon ausgehen, dass $X$ normalverteilt sind mit $\mu=48.917$.
\begin{teilaufgaben}
\item 
Ein Viertel aller Läufer sind schneller, in Formeln
\begin{align*}
0.25
=
P(X \le \text{0:44:30})
&=
P\biggl(
\frac{X-\mu}{\sigma} \le \frac{48.917 - \mu}{\sigma}
\biggr)
\\
\frac{48.917 - 44.5}{\sigma}
&=
-0.6745
\\
\sigma
&=
\frac{44.5- 48.917}{-0.6745}
=
6.54855
\end{align*}
\item 
Um unter die ersten Hundert Läufer zu gelangen, die Zeit $\color{red}t$
erreicht werden,
für die $P(X<{\color{red}t})=100/4000$ gilt.
Es folgt
\begin{align*}
P(X<{\color{red}t})
&=
0.975
\\
P\biggl(\frac{X-\mu}{\sigma}\le\frac{{\color{red}t}-\mu}{\sigma}\biggr)
&=
0.975
\\
\Rightarrow\qquad
\frac{{\color{red}t}-\mu}{\sigma}
&=
1.96
&
\Rightarrow\qquad
{\color{red}t}&=\mu - 1.96\sigma
=
48.917 - 1.96\cdot 6.5485
\\
&&&=
36.0817
=
\text{36:05}.
\end{align*}
\item
Gefragt ist $P(X>\text{1:00:00})$.
Es ist
\begin{align*}
P(X>\text{1:00:00})
=1-P(X\le 60)
&=
1- P\biggl(\frac{X-\mu}{\sigma} \le \frac{60-\mu}{\sigma}\biggr)
\\
&=
1 - F(1.69)
=
1-0.9545=4.55\%.
\end{align*}
Weniger als 5\% der Läufer brauchen mehr als eine Stunde.
\qedhere
\end{teilaufgaben}
\end{loesung}

\begin{bewertung}
Normalverteilung ({\bf N}) 1 Punkt,
Standardisierung ({\bf S}) 1 Punkt,
Bestimmung von $\sigma$ ($\mathbf{\Sigma}$) 1 Punkt,
Bedingung für die erste 100 Läufer ({\bf H}) 1 Punkt,
Zeit für die ersten 100 Läufer ({\bf Z}) 1 Punkt,
Wahrscheinlichkeit für mehr als eine Stunde ({\bf L}) 1 Punkt.
\end{bewertung}



