Dave Jones vom EEVblog hat 400 1k$\Omega$ Widerstände aus dem gleichen
Produktionslos mit einem Präzisionsmessgerät (mit 5\nicefrac{1}{2} Stellen
Genauigkeit) ausgemessen. 
Ein Viertel der Widerstände hatten einen Wert von über $1000.72\Omega$,
10\% Prozent hatten nicht mehr als $997.90\Omega$.
Schätzen Sie den mittleren Widerstand und Standardabweichung dieser
Messreihe ab.

\thema{Normalverteilung}
\thema{Standardisierung}

\begin{loesung}
Wir nehmen an, dass die Widerstände normalverteilte Werte haben\footnote{
Dies ist nicht richtig, wie man mit einem Hypothesen-Test nachweisen kann.
Auch Dave Jones hat dies angenommen, und ein Histogramm suggeriert
dies auch. Die Widerstände sind jedoch nicht normalverteilt, weil zu
stark abweichende Widerstände eliminiert werden.}, sei $X$ die Zufallsvariable
des Widerstands.
Wir nehmen an, dass $X$ normalverteilt ist mit Parametern $\mu$ und $\sigma$.
Wir wissen aus der Aufgabe:
\begin{equation}
\begin{aligned}
P(X\le 1000.72)&=0.75,
\\
P(X\le \phantom{0}997.90)&=0.10.
\end{aligned}
\label{60000042:prob}
\end{equation}
Standardisierung liefert
\begin{equation*}
\begin{aligned}
P\biggl(\frac{X-\mu}{\sigma}\le \frac{1000.72-\mu}{\sigma}\biggr)&=0.75,
\\
P\biggl(\frac{X-\mu}{\sigma}\le \phantom{0}\frac{997.90-\mu}{\sigma}\biggr)&=0.10.
\end{aligned}
\end{equation*}
Aus der Quantilentabelle für die Standardnormalverteilung liest man ab:
\[
\begin{aligned}
F(x_+)&=0.75&&\Rightarrow&&x_+&=\phantom{-}0.6745,\\
F(x_-)&=0.10&&\Rightarrow&&x_-&=-1.2816.
\end{aligned}
\]
Daraus folgt das Gleichungssystem
\[
\begin{linsys}{3}
1000.72&-&\mu&=&\phantom{-}0.6745\sigma,\\
 997.90&-&\mu&=&-1.2816\sigma.
\end{linsys}
\]
Die Differenz dieser beiden Gleichungen liefert
\[
1000.72-997.90 = (0.6745 + 1.2816)\sigma
\qquad
\Rightarrow
\qquad
\sigma = 1.44,
\]
und daraus kann man jetzt durch einsetzen auch $\mu$ bestimmen:
\begin{align*}
\mu&=1000.72-0.6745\sigma=999.75,\\
\mu&=\phantom{0}997.90+1.2816\sigma=999.75.
\end{align*}
Der Mittelwert dürfte also $\mu=999.79\Omega$ sein, die Standardabweichung
$\sigma=1.44\Omega$.
\end{loesung}

\begin{diskussion}
Rechnet man Mittelwert und Standardabweichung direkt aus den aus, findet man
$\mu=999.8746$ und $\sigma=1.695783$.
Die Abweichung rührt unter anderem daher, dass die Hypothese der
Normalverteilung nicht zutrifft.
\end{diskussion}

\begin{bewertung}
Normalverteilung ({\bf N}) 1 Punkt,
Ansatz, Gleichungen (\ref{60000042:prob}) ({\bf A}) 1 Punkt,
Standarddisierung ({\bf S}) 1 Punkt,
Quantilien ({\bf Q}) 1 Punkt,
lineares Gleichungssytem ({\bf L}) 1 Punkt,
Werte für $\mu$ und $\sigma$ ({\bf W}) 1 Punkt.
\end{bewertung}

