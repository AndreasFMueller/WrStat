Eine faire M"unze wird 12 mal geworfen und gez"ahlt, wie h"aufig die
M"unze Kopf zeigt.
\begin{teilaufgaben}
\item Berechnen Sie die Wahrscheinlichkeit exakt, dass mindestens vier mal
und h"ochstens sieben mal Kopf geworfen wird.
\item Approximieren Sie dieselbe Gr"osse mit Hilfe einer Normalverteilung.
\item Erkl"aren Sie gegebenenfalls die Abweichung zwischen den Resultaten
von a) und b)
\end{teilaufgaben}

\begin{loesung}
\begin{teilaufgaben}
\item Exakte Berechnung mit der Binomialverteilung:
\begin{align*}
P(4\le X\le 7)&=\sum_{k=4}^7\binom{12}{k}
\biggl(\frac12\biggr)^k
\biggl(\frac12\biggr)^{(12-k)}
\\
&=\frac1{2^{12}}\sum_{k=4}^{7}\binom{12}{k}
\\
&=\frac1{4096}\biggl(
\binom{12}{4}+
\binom{12}{5}+
\binom{12}{6}+
\binom{12}{7}
\biggr)
\\
&=\frac{495+792+924+792}{4096}
=
\frac{3003}{4096}=0.73315
\end{align*}
\item
F"ur eine Approximation mit der Normalverteilung brauchen wir Erwartungswert
und Varianz, f"ur die vorliegende Binomialverteilung sind dies
\begin{align*}
E(K) &=np=6
&\Rightarrow\quad\mu&=6
\\
\operatorname{var}(K)&=np(1-p)=\frac{12}{4}=3
&\Rightarrow\quad\sigma&=\sqrt{3}
\end{align*}
Die Approximation ist dann
\begin{align*}
P(4\le X\le 7)
&\simeq
P\biggl(
\frac{-2}{\sqrt{3}}\le \frac{X-6}{\sqrt{3}}\le \frac1{\sqrt{3}}
\biggr)
\\
&=F(\frac1{\sqrt{3}})-F(-\frac2{\sqrt{3}})
=F(\frac1{\sqrt{3}})-(1-F(\frac2{\sqrt{3}}))
\end{align*}
wobei $F$ die Verteilungsfunktion der Standardnormalverteilung ist. Aus
der Tabelle erh"alt man
\begin{align*}
P(4\le X\le 7)
&\simeq
F(0.57735) - (1-F(1.1547))
\\
&=0.7182 + 0.8765 - 1=0.5947
\end{align*}
\item
F"ur die Anwendbarkeit der Normalapproximation ist die Anzahl $n=12$
eher etwas gering, so dass man nicht unbedingt mit einer guten Approximation
rechnen darf.

Die Approximation ist aber auch noch aus einem anderen Grund schlecht.
Die Normalverteilung beschreibt eine Verteilung, bei der auch Zwischenwerte
zwischen den ganzen Zahlen m"oglich sind. Die Binomialverteilung l"asst
dagegen nur ganze Zahlen als Werte zu. Die Aussage, dass die Normalverteilung
die Binomialverteilung approximiere, bezieht sich aber nur auf die
Verteilungsfunktion, welche im Falle der Normalverteilung auch die
Zwischenwerte ber"ucksichtigt. Eine etwas genauere "Ubersetzung
ist wohl, dass die Zwischenwerte der normalverteilten Zufallsvariablen,
die auf $k$ gerundet werden m"ussten, der Wahrscheinlichkeit der Zahl
$k$ bei der Binomialverteilung entsprechen. Das bedeutet, dass die
Wahrscheinlichkeit der Zahl $k$ etwa $F(k+\frac12)-F(k-\frac12)$
entsprechen m"usste.
Entsprechen kann man eine
bessere Approximation erwarten, indem man als Grenzen
$3.5$ und $7.5$ verwendet, in diesem Fall ergibt sich
\[
P(3.5\le X\le 7.5)=F(0.86603)-F(-1.44338)=0.73230,
\]
was ausserordentlich gut stimmt.
\qedhere
\end{teilaufgaben}
\end{loesung}

