Die Zufallsvariable $X$ von Aufgabe~\ref{60000065} hat Erwartungswert
$\mu$ und Varianz $\sigma^2=4$.
\begin{teilaufgaben}
\item
Welchen Wert muss man in einer Tabelle der Standardnormalverteilung
nachschlagen, wenn man die Wahrscheinlichkeit $P(X\le \pi)$ bestimmen
will?
\item
Finden Sie $P(X\le \pi)$.
\end{teilaufgaben}

\begin{loesung}
\begin{teilaufgaben}
\item
Die Grösse $Z=(X-\mu)/\sigma$ ist standardnormalverteilt, d.~h.
\[
P(X\le \pi)
=
P\biggl(
\frac{X-\mu}{\sigma}
\le
\frac{\pi-\mu}{\sigma}
\biggr)
=
P\biggl(
Z
\le
\frac{\pi-2}{2}
\biggr).
\]
Es muss also der Wert $(\pi-2)/2 = \frac{\pi}2-1$ nachgeschlagen werden.
\item
Die Verteilungsfunktion der Standardnormalverteilung liefert:
\[
\Phi\biggl(
\frac{\pi-2}{2}
\biggr)
=
\Phi(0.571)
=
0.7160.
\]
\qedhere
\end{teilaufgaben}
\end{loesung}
