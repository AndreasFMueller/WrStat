Betrachten Sie den Fall, dass der Parameter $\lambda$ einer Poisson-Verteilung
zuf"allig eine ganze Zahl ist.
Berechnen Sie $P_\lambda(\lambda-1)$ und $P_\lambda(\lambda)$.

\begin{loesung}
Wir setzen $k=\lambda-1$ und $k=\lambda$ in die Formel f"ur $P_\lambda(k)$ ein
und erhalten:
\begin{align*}
P_\lambda(\lambda-1)
&=
e^{-\lambda}\frac{\lambda^{\lambda -1 }}{(\lambda)!}
\\
P_\lambda(\lambda)
&=
e^{-\lambda}\frac{\lambda^{\lambda}}{\lambda!}
=
e^{-\lambda}\frac{\lambda^{\lambda-1}\lambda}{(\lambda-1)!\cdot\lambda}
=
e^{-\lambda}\frac{\lambda^{\lambda -1 }}{(\lambda)!}
\end{align*}
Es gilt also
\[
P_\lambda(\lambda -1)=P_\lambda(\lambda).
\qedhere
\]
\end{loesung}

