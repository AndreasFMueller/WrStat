Im Sommer verschenkte Coop in einer Aktion Rubbelkarten, mit denen
man Zutaten f"ur Grilliaden gewinnen konnte.
Jede Karte enthielt vier Rubbelfelder, drei mit einer Grillflamme,
eine mit einem traurigen Smiley (Bild mitte).
Der Teilnehmer muss drei Felder freirubbeln, wenn dabei nur die Grillflammen
erscheinen, gewinnt er (Bild rechts).
\begin{center}
\includeagraphics[width=\hsize]{gratta.jpg}
\end{center}
In einem Coop-Laden werden die Karten freiz"ugig verschenkt, und ein Kunde
nimmt einen ganzen Stapel von 100 Karten mit, in der Hoffnung so die Zutaten
f"ur ein ganze Grillparty zu gewinnen.

\begin{teilaufgaben}
\item Wie gross ist die Wahrscheinlichkeit, in einem einzelnen Rubbelspiel zu
gewinnen?
\item Wie gross ist die Wahrscheinlichkeit, dass der Kunde mit seinem 
Kartenstapel mindestens  30 Gewinne erzielt?
\end{teilaufgaben}

\begin{loesung}
\begin{teilaufgaben}
\item
Der Spieler muss sich entscheiden, welches Feld er {\em nicht} freirubbeln will.
Mit Wahrscheinlichkeit $\frac14$ w"ahlt er dabei den traurigen Smiley,
er gewinnt also mit Wahrscheinlichkeit $p=\frac14$.
\item
Die Anzahl der Gewinne ist binomialverteilt mit Parametern $n=100$ und
$p=\frac14$.
Wir m"ussen die Wahrscheinlichkeit ausrechnen, dass die Anzahl $K$ der
Gewinne $\ge 30$ ist. 
Wir verwenden dazu die Normalapproximation der Binomialverteilung.
Wir verwenden die Parameter
\[
\begin{aligned}
\mu&=np = 100\cdot \frac14=25\\
\sigma^2&=np(1-p)=100\cdot\frac14\cdot\frac34=18.75&&\Rightarrow&\sigma&=4.33
\end{aligned}
\]
Die gesuchte Wahrscheinlichkeit ist
\begin{align*}
P(K\ge 30)
&=1-P(K \le 29)
\simeq
1-F\biggl(\frac{29.5-\mu}{\sigma}\biggr)
=
1-F\biggl(\frac{29.5-25}{4.33}\biggr)
\\
&=
1-F(1.03923)=1-0.85065=0.1493.
\end{align*}

Etwas weniger genau wird es, wenn man statt 29.5 den Wert 30 verwendet:
\begin{align*}
P(K\ge 30)
&=1-P(K \le 29)
\simeq
1-F\biggl(\frac{30-\mu}{\sigma}\biggr)
=
1-F\biggl(\frac{30-25}{4.33}\biggr)
\\
&=
1-F(1.1547)=1-0.8759=0.1241
\end{align*}


Mit R kann man die Wahrscheinlichkeit f"ur die Binomialverteilung
auch direkt ausrechnen:
\[
P(K\ge 30)
=1-P(K\le 29)=0.1495,
\]
in sehr guter "Ubereinstimmung mit der approximativen Berechnung.
\end{teilaufgaben}
\end{loesung}

\begin{bewertung}
\begin{teilaufgaben}
\item
Wahrscheinlichkeit ({\bf W}) 1 Punkt,
\item
Binomialverteilung ({\bf B}) 1 Punkt,
Normalapproximation ({\bf N}) 1 Punkt,
Parameter ({\bf P}) 1 Punkt,
Verteilungsfunktion und Standardisierung ({\bf S}) 1 Punkt,
Kumulative Wahrscheinlichkeit ({\bf K}) 1 Punkt.
\end{teilaufgaben}
\end{bewertung}

