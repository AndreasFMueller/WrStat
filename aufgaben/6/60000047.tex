Die 3D-Drucktechnik f"ur Consumer steckt immer noch etwas in den 
Kinderschuhen, vor allem die Drucke gr"osserer oder sonstwie komplexer
Teile schlagen "ofters fehl.
F"ur ein Projekt werden 30 identische Teile ben"otigt,
doch schl"agt der Druck eines Teils in einem von f"unf F"allen fehl.
Wieviele Drucke sollen geplant werden, damit mit Wahrscheinlichkeit 99\%
gen"ugend brauchbare Teile darunter sind?

\begin{loesung}
Gesucht ist die Zahl $n$ der n"otigen Drucke.
3D-Druck eines Teils ist ein Bernoulli-Experiment mit
Erfolgswahrscheinlichkeit $p=0.8$.
Ist $X$ die Zahl der erfolgreich gedruckten Teile, dann ist $X$
binomialverteilt mit Parametern $p$ und $n$.
Die Zahl $n$ soll so bestimmt werden, dass 
\[
0.99 \ge P(X\ge 30).
\]
Es ist klar, dass $n\ge 30$ sein muss, $X$ kann daher durch eine
Normalverteilung approximiert werden mit $\mu=np$ und $\sigma=\sqrt{np(1-p)}$,
die Wahrscheinlichkeit ist dann
\[
P(X\ge 30)=P\biggl(
\underbrace{\frac{X-\mu}{\sigma}}_{=Z}\ge\frac{30-\mu}{\sigma}
\biggr)
=P\biggl(Z\ge \frac{30-np}{\sqrt{np(1-p)}}\biggr)
\]
Die Zufallsvariable $Z$ ist angen"ahert standardnormalverteilt, die
Wahrscheinlichkeit $0.99$ wird also erreicht, wenn 
\[
\frac{30-np}{\sqrt{np(1-p)}} = 2.3263=x
\]
ist.
Quadrieren und multiplizieren mit dem Nenner gibt
\begin{align*}
900-60np+n^2p^2
&=
x^2 np(1-p)
\\
p^2{\color{red}n}^2-(60p+x^2p(1-p)){\color{red}n}+900&=0,
\end{align*}
eine quadratische Gleichung f"ur $n$ mit den L"osungen
\[
n=\frac{60p+x^2p(1-p)\pm\sqrt{(60p+x^2p(1-p))^2 - 3600p^2}}{2p^2}.
\]
Einsetzen der bekannten Zahlenwerte $31.02$ und $45.33$.
Es m"ussen also $46$ Drucke geplant werden.

Zur Kontrolle berechnen wir die Wahrscheinlichkeit, dass von 46 Drucken
weniger als 30 brauchbar sind mit Hilfe von R:
\verbatimainput{r1}
Macht man dagegen nur 45 Drucke, ist die Wahrscheinlichkeit, weniger als
30 brauchbare zu erhalten nur
\verbatimainput{r2}
Die Anzahl 46 ist also genau die gesuchte Anzahl Druckversuche.
\end{loesung}



