Die 3D-Drucktechnik für Consumer steckt immer noch etwas in den 
Kinderschuhen, vor allem die Drucke grösserer oder sonstwie komplexer
Teile schlagen öfters fehl.
Für ein Projekt werden 30 identische Teile benötigt,
doch schlägt der Druck eines Teils in einem von fünf Fällen fehl.
Wieviele Drucke sollen geplant werden, damit mit Wahrscheinlichkeit 99\%
genügend brauchbare Teile darunter sind?

\begin{loesung}
Gesucht ist die Zahl $n$ der nötigen Drucke.
3D-Druck eines Teils ist ein Bernoulli-Experiment mit
Erfolgswahrscheinlichkeit $p=0.8$.
Ist $X$ die Zahl der erfolgreich gedruckten Teile, dann ist $X$
binomialverteilt mit Parametern $p$ und $n$.
Die Zahl $n$ soll so bestimmt werden, dass 
\[
0.99 \ge P(X\ge 30).
\]
Es ist klar, dass $n\ge 30$ sein muss, $X$ kann daher durch eine
Normalverteilung approximiert werden mit $\mu=np$ und $\sigma=\sqrt{np(1-p)}$,
die Wahrscheinlichkeit ist dann
\[
P(X\ge 30)=P\biggl(
\underbrace{\frac{X-\mu}{\sigma}}_{=Z}\ge\frac{30-\mu}{\sigma}
\biggr)
=P\biggl(Z\ge \frac{30-np}{\sqrt{np(1-p)}}\biggr)
\]
Die Zufallsvariable $Z$ ist angenähert standardnormalverteilt, die
Wahrscheinlichkeit $0.99$ wird also erreicht, wenn 
\[
\frac{30-np}{\sqrt{np(1-p)}} = 2.3263=x
\]
ist.
Quadrieren und multiplizieren mit dem Nenner gibt
\begin{align*}
900-60np+n^2p^2
&=
x^2 np(1-p)
\\
p^2{\color{red}n}^2-(60p+x^2p(1-p)){\color{red}n}+900&=0,
\end{align*}
eine quadratische Gleichung für $n$ mit den Lösungen
\[
n=\frac{60p+x^2p(1-p)\pm\sqrt{(60p+x^2p(1-p))^2 - 3600p^2}}{2p^2}.
\]
Einsetzen der bekannten Zahlenwerte liefert die Nullstellen $31.02$ und $45.33$.
Davon ist die erste sicher zu klein, denn bei $31$ Drucken kann man nur mit $24.8$ erfolgreichen Drucken rechnen.
Es müssen also $46$ Drucke geplant werden.

Zur Kontrolle berechnen wir die Wahrscheinlichkeit, dass von 46 Drucken
weniger als 30 brauchbar sind mit Hilfe von R:
\verbatimainput{r1}
Macht man dagegen nur 45 Drucke, ist die Wahrscheinlichkeit, weniger als
30 brauchbare zu erhalten nur
\verbatimainput{r2}
Die Anzahl 46 ist also genau die gesuchte Anzahl Druckversuche.
\end{loesung}


\begin{bewertung}
Binomialverteilung ({\bf B}) 1 Punkt,
Bedingung $0.99 \ge P(x\le 30)$ ({\bf C}) 1 Punkt,
Normalapproximation, insbesondere die Formeln $\mu=np$ und $\sigma^2=np(1-p)$
({\bf N}) 1 Punkt,
Standardisierung ({\bf S}) 1 Punkt,
Gleichung für $n$ ({\bf G}) 1 Punkt,
Bestimmung von $n$ ({\bf N}) 1 Punkt.
\end{bewertung}

