Zur Performance-"Uberwachung l"asst der Systemadministrator auf einem Server
ein Script laufen, welches wie folgt funktioniert. Es sammelt einige
kritische Performance-Zahlen und schreibt diese in eine Datenbank.
Daf"ur braucht es im Mittel die Zeit $\delta$ Sekunden. Nat"urlich
ist eine gewisse Streuung $\sigma=3$ unvermeidlich.
Anschliessend wartet das Script f"ur $60-\delta$ Sekunden.
Wie gross ist die Wahrscheinlichkeit, dass in 24 Stunden statt
der vorgesehenen 1440 Messungen h"ochstens 1435 aufgezeichnet werden?

\begin{loesung}
Die Laufzeit der Messfunktion ist eine Zufallsvariable $\Delta$.
Wir d"urfen annehmen, dass $\Delta$ normalverteilt mit Erwartungswert
$\delta$ und Varianz $\sigma^2=\sigma_0^2$ ist, $\sigma=\sqrt{n}\sigma_0$.
Die tats"achliche Laufzeit $X_i$ eines
Durchgangs durch die Messschleife ist daher ebenfalls normalverteilt
mit Erwartungswert $\mu_0=60\,[s]$ und Varianz $\sigma_0^2=9\,[s^2]$.
Die Gesamtzeit $X$ f"ur
$n$ Schleifendurchl"aufe ist daher
\[
X=\sum_{i=1}^nX_i,
\]
ebenfalls eine normalverteilte Zufallsvariable
mit Erwartungswert $\mu = n$ Minuten,
mit einer Varianz von $\sigma^2=n\sigma_0^2\,[s^2]$.
Es werden in 24 Stunden h"ochstens 1435 Messungen aufgezeichnet,
wenn die Laufzeit f"ur
1436 Durchl"aufe gr"osser als 24 Stunden ist. Die Wahrscheinlichkeit daf"ur,
dass f"ur $n=1436$ Durchl"aufe mehr als 24 Stunden ben"otigt werden, 
ist $P(X > 86400) = 1-P(X\le 86400)$.
Durch Standardisierung finden wir
\[
P(X\le 86400)=P\biggl(
\frac{X-\mu}{\sigma}
\le
\frac{86400-\mu}{\sigma}
\biggr)
=
P\biggl(
\frac{X-n\mu_0}{\sqrt{n}\sigma_0}
\le
\frac{86400-n\mu_0}{\sqrt{n}\sigma_0}
\biggr)
=
P\biggl(
Z\le\frac{(1440-n)\cdot 60}{\sqrt{n}\sigma_0}
\biggr),
\]
wobei $Z$ eine standardnormalverteilte Zufallsvariable ist. F"ur $n=1436$
finden wir
\[
P(X\le 86400)=F\biggl(
\frac{4\cdot 60}{3\sqrt{1436}}
\biggr)=0.982619.
\]
Die Wahrscheinlichkeit daf"ur, dass h"ochstens 1435 Messungen gespeichert
werden, ist also 0.01738103.
\end{loesung}
