In Lourdes sollen angebliche Kranke durch Wunder geheilt werden.
Die katholische Kirche hat in der 156-jährigen Geschichte des Wallfahrtsortes
in den Pyrenäen bisher 69 solche Wunder proklamiert, das letzte im Jahre 2013.
Es fällt auch auf, dass die Anzahl der Wunder abgenommen hat, obwohl
die Zahl der jährlichen Besucher auf bis zu 5 Mio angestiegen ist:
\begin{center}
\begin{tabular}{|l|r|}
\hline
Zeitperiode&Wunderheilungen\\
\hline
1858--1907 &35\\
1908--1957 &25\\
1958--heute& 9\\
\hline
\end{tabular}
\end{center}
Ausgehend von der ``Wunderrate'' der Jahre 1908-1957, wie gross ist die
Wahrscheinlichkeit, dass in der Zeit zwischen 1958 und heute nicht mehr
als 9 Wunder stattfinden?

\begin{loesung}
Wunder sind offenbar seltene Ereignisse, man sollte deren Anzahl daher
mit einer Poisson-Verteilung modellieren können. Die Wunderrate in den
Jahren 1908--1957 war $0.5$ pro Jahr.
Die Anzahl $X$ der Wunder in der Zeitperiode
seit 1958 sollte daher Poisson-verteilt sein mit einer Rate von $0.5$ 
Wundern pro Jahr, oder $\lambda=57\cdot0.5 =28.5$ Wundern in der
ganzen Zeitperiode.
Die Wahrscheinlichkeit, nicht mehr als $9$ Wunder zu beobachten,
ist daher:
\begin{align*}
P(X\le 9)
&=
\sum_{k=0}^9 P_\lambda(k)=e^{-\lambda}\sum_{k=0}^9 \frac{\lambda^k}{k!}
\end{align*}
Zur Berechnung von $P(X\le 9)$ stellen wir folgende Tabelle auf:
\begin{center}
\begin{tabular}{|l|l|}
\hline
$k$&$P_\lambda(k)=e^{-\lambda}\frac{\lambda^k}{k!}$\\
\hline
  0&$4.1938\cdot 10^{-13}$\\
  1&$1.1952\cdot 10^{-11}$\\
  2&$1.7032\cdot 10^{-10}$\\
  3&$1.6180\cdot 10^{-9}$\\
  4&$1.1529\cdot 10^{-8}$\\
  5&$6.5713\cdot 10^{-8}$\\
  6&$3.1214\cdot 10^{-7}$\\
  7&$1.2708\cdot 10^{-6}$\\
  8&$4.5274\cdot 10^{-6}$\\
  9&$1.4337\cdot 10^{-5}$\\
%  0&$1.4032\cdot 10^{-15}$\\
%  1&$4.7990\cdot 10^{-14}$\\
%  2&$8.2064\cdot 10^{-13}$\\
%  3&$9.3553\cdot 10^{-12}$\\
%  4&$7.9987\cdot 10^{-11}$\\
%  5&$5.4711\cdot 10^{-10}$\\
%  6&$3.1186\cdot 10^{-9}$\\
%  7&$1.5236\cdot 10^{-8}$\\
%  8&$6.5135\cdot 10^{-8}$\\
%  9&$2.4751\cdot 10^{-7}$\\
\hline
   &$2.0526\cdot 10^{-5}$\\
%   &$3.3164\cdot 10^{-7}$\\
\hline
\end{tabular}
\end{center}
Die Wahrscheinlichkeit, nicht mehr als 9 Wunder seit 1958 zu beobachten,
ist also nur $P(X\le 9)=0.000020526$.
Die tatsächlich beobachtet Zahl der Wunder ist also viel zu klein,
oder in den Jahren 1908--1957 war die katholische Kirche viel wundergläubiger.
\end{loesung}

\begin{diskussion}
Informationen über die angeblichen Wunder von Lourdes findet man auf
der Website \url{http://de.lourdes-france.org/vertiefen/heilungen-und-wunder/die-geheilten-von-lourdes}.
\end{diskussion}

\begin{bewertung}
Verwendung der Poisson-Verteilung (\textbf{P}) 1 Punkt,
Bestimmung der Wunderrate, d.~h.~des Parameters $\lambda$ (\textbf{L}) 1 Punkt,
Summe von Poisson-Wahrscheinlichkeiten (\textbf{S}) 1 Punkt,
Berechnung der Terme (\textbf{R}) 2 Punkte,
Berechnung der Gesamtwahrscheinlichkeit (\textbf{W}) 1 Punkt.
\end{bewertung}

