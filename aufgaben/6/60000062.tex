Am 12.~Dezember 2020 hatte die Schweiz 76371 aktive Covid-19 Fälle,
d.~h.~bestätigte Fälle, in denen der Patient noch nicht genesen oder
gestorben ist.
Es wird geschätzt, dass es in etwa 3\% der Fälle zu einer Hospitalisierung
kommt und in etwa 1\% der Fälle zum Tod.
\begin{teilaufgaben}
\item
Mit wievielen Covid-19-bedingten Hospitalisierungen muss man rechnen, und
wie gross ist die Varianz dieser Anzahl?
\item
Die Besorgnis ob der Ausbreitung von SARS-CoV-2 betrifft vor allem die
Überlastung des Gesundheitswesens.
Wir verwenden ein sehr simplistisches Modell, welches von 2350 
verfügbaren Spitalplätzen ausgeht.
Dies ist mehr, als man erwartet.
Wie gross ist die Wahrscheinlichkeit, dass es trotzdem zu einem Engpass kommt?
\item
Die aktiven Fälle machen 0.9\% der Bevölkerung aus.
Nimmt man Gleichverteilung der Fälle an, wie gross ist dann die
Wahrscheinlichkeit, in einem Dorf mit 200 Einwohnern mehr als 2 
aktive Fälle zu finden?
\end{teilaufgaben}

\begin{loesung}
\begin{teilaufgaben}
\item
Die Anzahl $X$ Hospitalisierungen ist binomialverteilt mit $p=0.03$ und
$n=76371$.
Der Erwartungswert ist $E(X)=np = 2291.13$ und
$\operatorname{var}(X)=np(1-p)=2222.3961$
\item
Für die Anzahl $X$ kann man die Normalapproximation verwenden und damit
die gesuchte Wahrscheinlichkeit $P(X>2300)$ mit Hilfe von Standardisierung
berechnen:
\begin{align*}
P(X>2300)
&=
1-P(X<2350)
=
1-P(X\le 2350)
=
1-P\biggl(\frac{X-\mu}{\sigma}\le \frac{2350-\mu}{\sigma}\biggr)
\\
&=
1-\Phi\biggl(\frac{2350-\mu}{\sigma}\biggr)
=
1-\Phi(1.249)
=
1-0.8942
=
0.1058.
\end{align*}
\item
Die Anzahl der aktiven Fälle $Y$ ist ebenfalls binomialverteilt,
wir approximieren sie mit Hilfe der Poissonverteilung mit
$\lambda = 0.009\cdot 200 = 1.8$.
Es ist
\begin{align*}
P(X > 2)
=
1-P(X\le 2)
&=
1-P_{\lambda}(0)-P_{\lambda}(1)-P_{\lambda}(2)
\\
&=
1-
\frac{\lambda^0}{0!}e^{-\lambda}
-
\frac{\lambda^1}{1!}e^{-\lambda}
-
\frac{\lambda^2}{2!}e^{-\lambda}
\\
&=
1-(1+\lambda+{\textstyle\frac12}\lambda^2)e^{-\lambda}
=
0.2694.
\qedhere
\end{align*}
\end{teilaufgaben}
\end{loesung}

\begin{bewertung}
Erwartungswert $\mu$ ({\bf M}) 1 Punkt,
Varianz $\sigma^2$ ({\bf V}) 1 Punkt,
Standardisierung ({\bf S}) 1 Punkt,
Normalapproximation ({\bf N}) 1 Punkt,
Wahrscheinlichkeit der Normalapproximation ({\bf W}) 1 Punkt,
Poissonverteilung un Wahrscheinlichkeit ({\bf P}) 1 Punkt.
\end{bewertung}


