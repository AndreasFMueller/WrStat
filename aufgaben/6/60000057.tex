Das File \texttt{data.csv} enthält 100 potenzverteilte Zufallszahlen mit
$x_{\text{min}}=1$ und $\alpha=2.5$.
Ziel dieser Aufgabe ist, zu verstehen, wie man sich davon überzeugen
kann, dass die Zahlen tatsächlich potenzverteilt sind.
Dazu muss die Verteilungsfunktion der Potenzverteilung mit einer
empirischen Verteilungsfunktion verglichen weren.
Diese kann wie folgt gewonnen werden:
\[
F_{\text{empirisch}}(x)
=
\frac{\text{Anzahl Werte $\le x$}}{\text{Gesamtzahl der Werte}}.
\]
Sortiert man die Werte $x_0,x_1,\dots,x_{n-1}$, dann ist 
\begin{equation}
F_{\text{empirisch}}(x_j) = \frac{j}{n}
\label{60000057}
\end{equation}
\begin{teilaufgaben}
\item 
Erstellen Sie einen $\log$-$\log$-Plot der Funktion $1-F(x)$.
\item
Welche Steigung hat die in a) gefundene Gerade?
\item
Zeichnen Sie die Punkte $(x_j,1-F_{\text{empirisch}}(x_j))$
im selben Graphen.
\item
Verwenden Sie lineare Regression, um $\alpha$ aus den Daten
zu bestimmen.
\end{teilaufgaben}

\thema{Potenzverteilung}

\begin{loesung}
\begin{figure}
\centering
\includeagraphics[width=\hsize]{d.pdf}
\caption{Empirische Verteilung potenzverteilter Zufallszahlen
\label{60000057:graph}
}
\end{figure}
\begin{teilaufgaben}
\item
Aus der Verteilungsfunktion der Potenzverteilung wissen wir
\[
1-F(x)
=
\biggl(\frac{x}{x_{\text{min}}} \biggr)^{1-\alpha}.
\]
Der Logarithmus davon ist
\begin{equation}
\log(1-F(x))=(1-\alpha)(\log(x) - \log(x_{\text{min}})).
\label{60000057:log}
\end{equation}
\item
Die Gleichung \eqref{60000057:log} beschreibt eine Gerade mit
Steigung $1-\alpha$.
\item
Ein solcher Graph kann mit dem R-Program
\verbatimainput{p.R}
ermittelt werden.
\item
Die Berechnung mit dem eben gezeigten R-Programm liefert
\verbatimainput{computation.txt}
woraus man die Steigung $1-\alpha=-1.3987$ oder $\alpha=2.3987$
ablesen kann.
Man kann auch erkennen, dass $\alpha$ etwas zu gering ausfällt,
was vor allem an den beiden extrem grossen letzten Werten zu liegen
scheint.
Dieses Schätzverfahren für $\alpha$ ist also sehr empfindlich auf
grosse Werte.
\qedhere
\end{teilaufgaben}
\end{loesung}


