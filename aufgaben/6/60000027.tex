Die Funktionsfähigkeit eines Gerätes hängt kritisch von einem etwas
heiklen Bauteil ab, welches zur Verbesserung der Verfügbarkeit redundant
(d.~h.~doppelt)
ausgeführt ist. Das Gerät kann also selbst beim Ausfall einer dieser
kritischen
Komponenten noch weiterlaufen. Die einzelne Komponente hat eine mittlere
Lebensdauer von einem Jahr.
\begin{teilaufgaben}
\item Wie wahrscheinlich ist der Ausfall des Bauteils innerhalb der
ersten Betriebswoche?
\item Wie wahrscheinlich ist ein Ausfall des Gerätes innerhalb der
ersten Woche?
\item Die Autoren der Wartungsvorschriften können sich im folgenden Punkt
nicht einigen. Der eine will verlangen, dass das Gerät bei der jährlichen
Revision immer ausgetauscht werden muss, der andere will dies nur tun, wenn
eine der kritischen Komponenten kaputt ist.
Der erste behauptet, die Zuverlässigkeit werde so erhöht, der zweite
stellt sich auf den Standpunkt, sein Vorschlag sei genauso zuverlässig,
aber viel billiger. Wer hat recht?
\item
Wie gross ist die Wahrscheinlichkeit, dass nach einem Jahr genau
eine der kritischen Komponenten kaputt ist?
\end{teilaufgaben}

\thema{Exponentialverteilung}

\begin{loesung}
\begin{teilaufgaben}
\item Da es sich um einen Prozess ohne Gedächtnis handelt, können wir
annehmen, dass der erste Ausfall nach einem Zeitintervall auftreten
wird, dessen Länge $X$ exponentialverteilt ist mit $a=\frac1{1\text{Jahr}}$.
Damit ist
\begin{align*}
P\left(X\le \frac{7}{365}\right)
&=
F_X\left(\frac{7}{365}\right)\\
&=
1-e^{-\frac{7}{365}}
\simeq
0.01899535277127424748
\end{align*}
\item
Das Gerät fällt aus, wenn beide Bauteile innerhalb der ersten
Woche ausfallen. Die Wahrscheinlichkeit dafür ist
\[
P\biggl(X_1\le \frac{7}{365}\wedge X_2\le \frac{7}{365}\biggr)
=F_X\biggl(\frac{7}{365}\biggr)^2=0.000361.
\]
\item
Da es sich um einen Prozess ohne Gedächtnis handelt, bringt es
nichts, das Gerät auszutauschen, wenn die beiden redundanten Komponenten
noch funktionsfähig sind. Da sie nicht altern, sind sie, sofern sie
noch ganz sind, neuwertig.
\item
Der Fall, dass eine Komponente kaputt ist, die andere aber noch ganz,
ist
\begin{align*}
&\quad P(X_1 < 1)\cdot P(X_2>1)
+P(X_1 > 1)\cdot P(X_2<1)
\\
&=
2F(1)\,(1-F(1))
\\
&=
2(1-e^{-1})(1-1+e^{-1})=2e^{-1}(1-e^{-1})=0.4650883
\qedhere
\end{align*}
\end{teilaufgaben}
\end{loesung}

