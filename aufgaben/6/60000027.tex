Die Funktionsf"ahigkeit eines Ger"ates h"angt kritisch von einem etwas
heiklen Bauteil ab, welches zur Verbesserung der Verf"ugbarkeit redundant
ausgef"uhrt ist. Das Ger"at kann also selbst beim Ausfall der einen
Komponente noch weiterlaufen. Die einzelne Komponente hat eine mittlere
Lebensdauer von einem Jahr.
\begin{teilaufgaben}
\item Wie wahrscheinlich ist der Ausfall des Bauteils innerhalb der
ersten Betriebswoche?
\item Wie wahrscheinlich ist ein Ausfall des Ger"ates innerhalb der
ersten Woche?
\item Die Autoren der Wartungsvorschriften k"onnen sich im folgenden Punkt
nicht einigen. Der eine will verlangen, dass das Ger"at bei der j"ahrlichen
Revision ausgetauscht werden muss, der andere will dies nur tun, wenn
die kritische Komponente kaputt ist.
Der erste behauptet, die Zuverl"assigkeit werde so erh"oht, der zweite
stellt sich auf den Standpunkt, sein Vorschlag sei genauso zuverl"assig,
aber viel billiger. Wer hat recht? Wie wahrscheinlich ist, dass nach der
zweiten Variante das Ger"at ausgetauscht werden muss?
\end{teilaufgaben}

\begin{loesung}
\begin{teilaufgaben}
\item Da es sich um einen Prozess ohne Ged"achtnis handelt, k"onnen wir
annehmen, dass der erste Ausfall nach einem Zeitinterval auftreten
wird, dessen L"ange $X$ exponentialverteilt ist mit $a=\frac1{1\text{Jahr}}$.
Damit ist
\begin{align*}
P\left(X\le \frac{7}{365}\right)
&=
F_X\left(\frac{7}{365}\right)\\
&=
1-e^{-\frac{7}{365}}
\simeq
0.01899535277127424748
\end{align*}
\item
Das Ger"at f"allt aus, wenn beide Bauteile innerhalb der ersten
Woche ausfallen. Die Wahrscheinlichkeit daf"ur ist
\[
P\biggl(X_1\le \frac{7}{365}\wedge X_2\le \frac{7}{365}\biggr)
=F_X\biggl(\frac{7}{365}\biggr)^2=0.000361.
\]
\item
Da es sich um einen Prozess ohne Ged"achtnis handelt, bringt es
nichts, das Ger"at auszutauschen, wenn die beiden redundanten Komponenten
noch funktionsf"ahig sind. Da sie nicht altern, sind sie, sofern sie
noch ganz sind, neuwertig.
Der Fall, dass eine Komponente kaputt ist, die andere aber noch ganz,
ist
\begin{align*}
&\quad P(X_1 < 365)\cdot P(X_2>365)
+P(X_1 > 365)\cdot P(X_2<365)
\\
&=
2F(365)\,(1-F(365))
\\
&=
2(1-e^{-1})(1-1+e^{-1})=2e^{-1}(1-e^{-1})=0.4650883
\qedhere
\end{align*}
\end{teilaufgaben}
\end{loesung}

