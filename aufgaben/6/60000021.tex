Eine Firma vertreibt Zements"acke \`a 100kg. Die Abf"ullmaschinen weisen
eine Standardabweichung von einem Kilogramm auf.

Zur Kontrolle der Ehrlichkeit des Betriebes werden durch einen Kontrolleur
von Zeit zu Zeit 25 S"acke zuf"allig herausgegriffen und gewogen. Liegt
das durchschnittliche Gewicht der 25 S"acke unter 99.4kg, werden weitere
Untersuchungen eingeleitet.

Nun beschliesst die Leitung des Betriebs, in Zukunft nur noch durchschnittlich
99.8kg abzuf"ullen, bei gleicher Standardabweichung. Wie gross ist die
Wahrscheinlichkeit das der Betrieb erwischt wird und weitere, wahrscheinlich
verh"angnisvolle Untersuchungen "uber sich ergehen lassen muss.

\begin{loesung}
Man darf annehmen, dass die Abf"ullmenge pro Sack nach der neuen Regelung
eine Normalverteilte Zufallsvariable mit Mittelwert $\mu=99.8\text{kg}$ und
$\sigma=1\text{kg}$ ist. Bei der Kontrolle werden 25 S"acke $X_1,\dots,X_{25}$
gewogen und der Mittelwert gebildet:
\[
X=\frac1{25}(X_1+\cdots +X_{25}).
\]
Der Mittelwert von $X$ ist $99.8\text{kg}$, f"ur die Varianz gilt
\[
\operatorname{var}(X)=\frac1{25}\operatorname{var}{X_1}=\frac1{25},
\]
also f"ur die Standardabweichung
\[
\sigma(X)=\frac15=0.2\text{kg}.
\]
Folglich ist $(X-99.8)/0.2$ eine standardnormalverteilte Zufallsvariable.

Gefragt ist nun die Wahrscheinlichkeit, dass $X<99.4$ wird, denn dies ist ja
das bei der Kontrolle angewandte Kriterium. Damit ist
\begin{align*}
P(X<99.4)
&=P\left(\frac{X-99.8}{0.2} < \frac{99.4-99.8}{0.2}\right)\\
&=P\left(\frac{X-99.8}{0.2} < -2\right)
\end{align*}
Es soll also ermittelt werden, wie gross die Wahrscheinlichkeit $p$ ist, dass
eine standardnormalverteilte Zufallsvariable einen Wert $<-2$ annimmt.
Da in der Tabelle der Normalverteilung nur positive $x$ dargestellt sind,
muss man $1-p$ berechnen, also die Wahrscheinlichkeit, dass $X>100.2$,
oder
\[
P(X<99.4)=1-P\left(\frac{X-99.8}{0.2}>2\right)
=1-0.9772=0.0228,
\]
wobei man den Zahlenwerte in der Tabelle der Normalverteilung bei $x=2$ abliest.
\end{loesung}

