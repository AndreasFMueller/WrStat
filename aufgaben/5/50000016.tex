Von einer Zufallsvariablen $X$ ist bekannt, dass sie eine Dichtefunktion
der Form
\[
\varphi(x)=\begin{cases}
a\cos x&\qquad x\in[-\frac{\pi}2,\frac{\pi}2]\\
0&\qquad\text{sonst}
\end{cases}
\]
hat.
\begin{teilaufgaben}
\item Wie muss $a$ gew"ahlt werden, damit tats"achlich eine
Wahrscheinlichkeitsdichte entsteht?
\item Berechnen Sie die Verteilungsfunktion $F$
\item Berechnen Sie den Erwartungswert von $X$.
\end{teilaufgaben}

\begin{loesung}
\begin{teilaufgaben}
\item Damit eine Wahrscheinlichkeitsdichte entsteht, muss das Integral
"uber die ganze reelle Achse $1$ werden, also
\begin{eqnarray*}
1&=&\int_{-\infty}^\infty\varphi(x)\,dx
=a\int_{-\frac{\pi}2}^{\frac{\pi}2}\cos x\,dx\\
&=&a\left[\sin x\right]_{-\frac{\pi}2}^{\frac{\pi}2}
=a\left(\sin\frac{\pi}2 -\sin\left(-\frac{\pi}2\right)\right)
=2a
\end{eqnarray*}

Somit muss $a=\frac12$ gesetzt werden.
\item Die Verteilungsfunktion ist $0$ f"ur $x$-Werte links von $-\frac{\pi}2$
und $1$ f"ur $x$-Werte rechts von $\frac{\pi}2$. Dazwischen gilt:
\[
F(x)=\int_{-\infty}^x\varphi(x)\,dx
=\frac12\int_{-\frac{\pi}2}^x\cos x\,dx
=\frac12\left[\sin x\right]_{-\frac{\pi}2}^x
=\frac12(\sin x + 1)
\]
Insgesamt also das Resultat
\[
F(x)=\begin{cases}
0&\qquad x<-\frac{\pi}2\\
\frac12(1+\sin x)&\qquad -\frac{\pi}2\le x\le\frac{\pi}2\\
1&\qquad x>\frac{\pi}2
\end{cases}
\]
\item Der Erwartungswert verschwindet, da die Dichtefunktion symmetrisch ist.
\qedhere
\end{teilaufgaben}
\end{loesung}

