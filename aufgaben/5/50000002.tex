Eine Zufallsvariable $X$ hat die Wahrscheinlichkeitsdichte
\[
\varphi(x)=\begin{cases}
0&\qquad |x|>1\\
|x|&\qquad |x|\le 1
\end{cases}
\]
\begin{teilaufgaben}
\item Bestimmen Sie die Verteilungsfunktion.
\item Bestimmen Sie Erwartungswert und Varianz von $X$.
\end{teilaufgaben}

\begin{loesung}
\begin{teilaufgaben}
\item Die Verteilungsfunktion ist
\[
F(x)=\int_{-\infty}^x\varphi(\xi)\,d\xi
\]
Wir behandeln die F"alle $x<-1$, $-1<x<0$, $0<x<1$ und $1<x$
separat. F"ur $x<-1$ und $x>1$ ist $\varphi(x)=0$, d.~h.~die
Verteilungsfunktion "andert sich in diesem Bereich nicht mehr,
$F(x)=0$ f"ur $x<-1$ und $F(x)=1$ f"ur $x>1$.

F"ur $x\in[-1,0]$ ist
\begin{align*}
F(x)&=\int_{-\infty}^x\varphi(\xi) \,d\xi
=
\int_{-1}^x|\xi|\,d\xi
=
\int_{-1}^x-\xi\,d\xi
\\
&=\left[-\frac12\xi^2\right]_{-1}^x
=-\frac12x^2+\frac12=\frac12(1-x^2)
\end{align*}
F"ur $x\in[0,1]$ ist
\begin{align*}
F(x)
&=
\int_{-\infty}^x \varphi(\xi)\,d\xi
=
F(0)+\int_0^x|\xi|\,d\xi
=
\frac12+\int_0^x\xi\,d\xi
\\
&=
\frac12+\left[
\frac12\xi^2
\right]_0^x
=\frac12(1+x^2)
\end{align*}
Insgesamt ist also die Verteilungsfunktion
\[
F(x)
=
\begin{cases}
0&\qquad x<-1\\
\frac12(1-x^2)&\qquad -1\le x < 0\\
\frac12(1+x^2)&\qquad 0\le x <1\\
1&\qquad 1 \le x
\end{cases}
\]
\item
Da $\varphi$ eine gerade Funktion ist, ist $E(X)=0$.
Wegen $E(X)=0$ ist die Varianz
\begin{align*}
\operatorname{var}(X)&=E(X^2)
=
\int_{-\infty}^\infty \xi^2\varphi(\xi)\,d\xi
=
\int_{-1}^1\xi^2|\xi|\,d\xi
=2\int_0^1\xi^3\,d\xi=2\left[\frac14\xi^4\right]_0^1=\frac12
\qedhere
\end{align*}
\end{teilaufgaben}
\end{loesung}

