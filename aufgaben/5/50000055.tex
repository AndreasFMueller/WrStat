Die QAnon-Verschwörungstheorie hat inzwischen die Popularität
der schon länger etablierten Religionen erreicht.
14\% der Amerikaner glauben, dass eine einflussreiche, weltweit agierende,
satanische Elite vom Keller einer Pizzeria in Washington
aus einen Kinderhändlerring betreibt (das Gebäude hat keinen Keller).
Sie sind überzeugt, dass diese Elite die Wiederwahl Donald Trumps
verhindert hat.
Innerhalb der Republikaner sind es sogar 29\%.

In einer Umfrage 2017 sind die Flacherdler dagegen vergleichsweise
selten, nicht einmal 1\% glauben, dass die Erde flach ist.

Das Dorf Hyder in Alaska hat 87 Einwohner, die sich politisch ungefähr
so wie in den übrigen USA verteilen.
\begin{teilaufgaben}
\item
Wie gross ist die Wahrscheinlichkeit mehr als 10 QAnon-Anhänger zu finden?
\item
Wie wahrscheinlich ist es, dass Hyder mehr als zwei Flacherdler hat?
\end{teilaufgaben}

\begin{loesung}
\begin{teilaufgaben}
\item
Sei $X$ die Zufallsvariable der Anzahl der QAnon-Anhänger in Hyder.
Sie ist binomialverteilt mit $n=87$ und $p=0.14$.
Der Erwartungswert ist $\mu=np=12.180$ und die Standardabweichung
ist $\sigma=\sqrt{np(1-p)}\approx 3.2365$.
Mit der Normalverteilungsapproximation der Binomialverteilung
kann man jetzt die Wahrscheinlichkeit ausrechnen.
Dazu bestimmt man die Standardisierung $Z=(X-\mu)/\sigma$ und
bekommt
\begin{align*}
P(X\le 10)
&\approx
P\biggl(Z < \frac{10.5-\mu}{\sigma}\biggr)
=
P(Z < -0.5191) = 0.3018,
%\\
%P(X>10)
%&\approx 30.2\%,
\\
P(X > 10)
&\approx
1-0.302 = 0.698
\end{align*}
die gesuchte Wahrscheinlichkeit ist $69.8\%$.
\item
Da Flacherdler so selten sind, kann man ihre Anzahl mit der
Poisson-Verteilung mit dem Parameter $\lambda = n\cdot 0.01=0.87$
beschreiben.
Sei $X$ die Anzahl der Flacherdler, dann ist die Wahrscheinlichkeit,
mehr als 2 Flacherdler zu finden
\begin{align*}
P(X>2)
&\approx
1-\sum_{k=0}^2 P_\lambda(k)
=
1-e^{-\lambda} \sum_{k=0}^2 \frac{\lambda^k}{k!}
\\
&=
1-e^{-\lambda}\biggl(1+\lambda+\frac{\lambda^2}{2}\biggr)
=
1- 0.9420
=
0.0580.
\end{align*}
also 5.08\%.
\qedhere
\end{teilaufgaben}
\end{loesung}

\begin{bewertung}
\begin{teilaufgaben}
\item
Binomialverteilung ({\bf B}) 1~Punkt,
Erwartungswert und Varianz ({\bf E}) 1~Punkt,
Standardisierung ({\bf S}) 1~Punkt,
Wahrscheinlichkeit ({\bf W}) 1~Punkt,
\item
Poisson-Verteilung und Wert des Parameters ({\bf P}) 1~Punkt,
Resultierende Wahrscheinlichkeit ({\bf R}) 1~Punkt.
\end{teilaufgaben}
\end{bewertung}

