In einer Z"undholzfabrik wird die Qualit"atssicherung wie folgt gehandhabt.
Die Maschine, die die H"olzer abl"angt ist auf die L"ange $l=5\text{cm}$ eingestellt.
Abweichungen nach oben oder unten bis 5\% von dieser L"ange werden toleriert,
l"angere H"olzer passen nicht mehr in die Schachteln. Es stellt sich
heraus, dass 10\% der abgel"angten H"olzer als zu kurz und 5\% als zu
lange ausgeschieden werden. Wie gross ist die mittlere L"ange der abgel"angten
H"olzer, und welche Standardabweichung hat die L"ange?

\begin{loesung}
Wir d"urfen Annehmen, dass die L"ange $X$ eines Streichholzes eine
normalverteilte Zufallsvariable mit Erwartungswert $\mu$ und Varianz
$\sigma^2$ ist. Wir wissen, dass
\begin{align*}
P(X<0.95 l)&=0.1&P\left(\frac{X-\mu}{\sigma}<\frac{0.95l-\mu}{\sigma}\right)&=0.1\\
P(X>1.05l)&=0.05&P\left(\frac{X-\mu}{\sigma}>\frac{1.05l-\mu}{\sigma}\right)&=0.05
\end{align*}
Daraus ergibt sich das Gleichungssystem
\begin{align*}
0.95l-\mu&=-1.281552\sigma\\
1.05l-\mu&=1.644854\sigma
\end{align*}
Die Differenz der beiden Gleichungen liefert
\[
0.1l=2.926405\sigma\qquad\Rightarrow\qquad
\sigma=0.03417162l.
\]
Aus der halben Summe der beiden Gleichungen folgt
\[
l-\mu=0.1816510\sigma\qquad\Rightarrow\qquad\mu=l-0.00620731l
=0.9937927l
\]
F"ur die eingestellte L"ange von $l=5\text{cm}$ folgt
\begin{align*}
\mu&=
4.9689635
\text{cm}
\\
\sigma&=
1.7085810
\text{mm}
\end{align*}
\end{loesung}

