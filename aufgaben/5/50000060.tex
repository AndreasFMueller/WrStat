Betrachten Sie die Funktion
\[
\varphi(x)
=
\begin{cases}
a\!\sqrt{1-x^2}&\qquad -1<x<1\\
0&\qquad\text{sonst.}
\end{cases}
\]
\begin{figure}[h]
\centering
\pgfmathparse{2/3.1415926}
\xdef\a{\pgfmathresult}
\begin{tikzpicture}[>=latex,thick,scale=4]
\fill[color=red,color=red!20] (-1,0) 
	-- plot[domain=0:180,samples=100] ({-cos(\x)},{\a*sin(\x)})
	-- cycle;
\draw[->] (-2.0,0) -- (2.0,0) coordinate[label={$x$}];
\draw[->] (0,-0.02) -- (0,0.9) coordinate[label={right:$\varphi(x)$}];
\draw[color=red,line width=1.4pt] (-1.9,0)
	-- plot[domain=0:180,samples=100] ({-cos(\x)},{\a*sin(\x)})
	-- (1.9,0);
\node at (0,{\a}) [above left] {$a\mathstrut$};
\node at (0,0) [below] {$0\mathstrut$};
\draw (-1,-0.02) -- (-1,0.02);
\node at (-1,0) [below] {$-1$};
\draw (1,-0.02) -- (1,0.02);
\node at (1,0) [below] {$1$};
\draw (-0.02,{\a}) -- (0.02,{\a});
\end{tikzpicture}
\end{figure}
\begin{teilaufgaben}
\item
Für welchen Wert von $a$ ist die Funktion
die Dichtefunktion einer Wahrscheinlichkeitsverteilung?
\item
Bestimmen Sie den Erwartungswert.
\item
Bestimmen Sie die Varianz.
\end{teilaufgaben}


\begin{hinweis}
Verwenden Sie für die Berechnung der Integrale die folgenden Stammfunktionen:
\begin{align}
\int\sqrt{1-x^2}\,dx
&=
\frac12\arcsin x + C,
\label{50000060:kreisformel}
\\
\int x^2\sqrt{1-x^2}\,dx
&=
\frac{1}{8}\arcsin x - \frac{x}{4}(1-x^2)^{\frac32} + \frac{x}{8}\sqrt{1-x^2}
+C.
\label{50000060:quadrat}
\end{align}
\end{hinweis}

\begin{loesung}
\begin{teilaufgaben}
\item
Die Konstante $a$ muss so bestimmt werden, dass das Integral über $\mathbb{R}$
den Wert $1$ hat.
Es gilt also
\begin{align*}
1
&=
\int_{-\infty}^{\infty}
\varphi(x)
\,dx
=
\int_{-1}^1
a\sqrt{1-x^2}
\,dx
=
a\frac{\pi}{2}
\qquad\Rightarrow\qquad
a=\frac{2}{\pi}.
\end{align*}
Die im Hinweis angegebene Integralformel~\eqref{50000060:kreisformel}
ist nicht nötig, da das Integral den Flächeninhalt eines halben
Einheitskreises berechnet, der $\frac{\pi}2$ ist.
\item
Wegen der Symmetrie der Funktion $\varphi(x)$ ist der Erwartungswert
$E(X)=0$.
\item 
Für die Varianz muss das Integral
\[
E(X^2)
=
\int_{-\infty}^\infty x^2\varphi(x)\,dx
=
\frac{2}{\pi}\int_{-1}^1 x^2\sqrt{1-x^2}\,dx
\]
bestimmt werden.
Mit der Formel~\eqref{50000060:quadrat} aus dem Hinweis folgt:
\begin{align*}
E(X^2)
&=
\frac{2}{\pi}
\biggl[
\frac18\arcsin x - \frac{x}4(1-x^2)^{\frac32}+\frac{x}8\sqrt{1-x^2}
\bigg]_{-1}^1
=
\frac{2}{\pi}\biggl(
\frac18\biggl(\frac{\pi}{2}-\biggl(-\frac{\pi}2\biggr)\biggr)
\biggr)
=
\frac{1}{4}
\\
\operatorname{var}(X)
&=
E(X^2)-E(X)^2
=
\frac{1}{4}-0
=
\frac{1}{4}.
\qedhere
\end{align*}
\end{teilaufgaben}
\end{loesung}

\begin{bewertung}
Normierungsbedingung ({\bf N}) 1 Punkt,
Konstante $a$ ({\bf A}) 1 Punkt,
Erwartungswert ({\bf E}) 1 Punkt,
Erwartungswert $E(X^2)$ 1 Punkt,
Varianzformel ({\bf V}) 1 Punkt,
Wert der Varianz ({\bf W}) 1 Punkt.
\end{bewertung}
