Bei einer Untersuchung in Äthiopien im Jahr 2014 wurde der Blutdruck
von 1669 Erwachsenen gemessen.
Dabei wurde festgestellt, dass 18.1\% der Bevölkerung einen 
systolischen Blutdruck von über 130\,\text{mmHg} haben, was die
American Heart Association (AHA) als {\em Bluthochdruck} definiert.
{\em Erhöhter Blutdruck} ist dagegen definiert als ein systolischer
Blutdruck zwischen 120 und 130\,\text{mmHg}.
Diesen Zustand findet man bei 37.2\% der Studienteilnehmer.
Wie gross ist die Wahrscheinlichkeit, dass ein Studienteilnehmer
einen systolischen Blutdruck unter 110\,\text{mmHg} hat?

\begin{loesung}
\begin{figure}
\centering
\def\punkt#1#2{({(#1-100)/4},{(#2)*4})}
\begin{tikzpicture}[>=latex,thick,
declare function = {
	p(\t) = exp(-(\t-121.275)*(\t-121.275)/(2*9.5712*9.5712));
}]

\fill[color=darkred!20]
	plot[domain=130:145,samples=20]
		({(\x-100)/4},{4*exp(-(\x-121.275)*(\x-121.275)/(2*9.5712*9.5712))})
		--
		\punkt{145}{0} -- \punkt{130}{0} -- cycle;
\fill[color=blue!20]
	plot[domain=120:130,samples=20]
		({(\x-100)/4},{4*exp(-(\x-121.275)*(\x-121.275)/(2*9.5712*9.5712))})
		--
		\punkt{130}{0} -- \punkt{120}{0} -- cycle;
\fill[color=darkgreen!20]
	plot[domain=99:110,samples=20]
		({(\x-100)/4},{4*exp(-(\x-121.275)*(\x-121.275)/(2*9.5712*9.5712))})
		--
		\punkt{110}{0} -- \punkt{99}{0} -- cycle;

\node[color=darkred] at \punkt{135}{0.17} {$18.1\%$};
\node[color=blue] at \punkt{125}{0.3} {$37.2\%$};
\node[color=darkgreen] at \punkt{105}{0.1} {gesucht};

\draw[->] \punkt{99}{0} -- \punkt{150}{0} coordinate[label={$x$}];
\draw[->] \punkt{100}{-0.01} -- \punkt{100}{1.01} coordinate[label={$\varphi$}];;
\draw plot[domain=99:145,samples=100]
	({(\x-100)/4},{4*exp(-(\x-121.275)*(\x-121.275)/(2*9.5712*9.5712))});

\foreach \x in {110,120,130}{
	\draw \punkt{\x}{-0.01} -- \punkt{\x}{0.01};
}
\foreach \x in {110,120,130}{
	\node at \punkt{\x}{-0.01} [below] {$\x$};
}

\end{tikzpicture}
\caption{Normalverteilung der Blutdruckmessungen für Aufgabe~\ref{50000069}.
\label{50000069:blutdruck}}
\end{figure}
Der Blutdruck $X$ ist eine normalverteilte Zufallsvariable
mit Erwartungswert $\mu$ und $\sigma$.
Aus den Informationen der Aufgabe leitet man ab (siehe auch
Abbildung~\ref{50000069}), dass
\begin{equation*}
\left.
\begin{aligned}
P(X\ge 130) &= 0.181 \\
P(120\le X\le 130) &= 0.372
\end{aligned}
\right\}
\qquad\Rightarrow\qquad
P(X\ge 120) = 0.553
\end{equation*}
Durch Standardisierung
\begin{align*}
P\biggl(
\frac{X-\mu}{\sigma} \le \frac{130-\mu}{\sigma}
\biggr)
&=
0.819
&
P\biggl(
\frac{X-\mu}{\sigma} \le \frac{120-\mu}{\sigma}
\biggr)
&=
0.447
\intertext{$Z=(X-\mu)/\sigma$ ist standardnormalverteilt.
Die Umkehrfunktion der Verteilungsfunktion $\Phi$ der Standardnormalverteilung
führt auf die Gleichungen}
\frac{130-\mu}{\sigma}
&=
0.9116
&
\frac{120-\mu}{\sigma}
&=
-0.1332
\end{align*}
Das Gleichungssystem
\[
\renewcommand{\arraycolsep}{3pt}
\begin{array}{rcrcr}
\mu &+& 0.9116 \sigma &=& 130\phantom{.}\\
\mu &-& 0.1332 \sigma &=& 120.
\end{array}
\]
Die Differenz der beiden Gleichungen ist
\[
(0.9116+0.1332)\sigma
=
1.0448\sigma
=
10
\qquad\Rightarrow\qquad
\sigma
=
9.5712.
\]
Eingesetzt in die erste Gleichung kann man daraus auch 
\[
\mu
=
130-0.9116\sigma
= 
121.275
\]
bekommen.
Daraus kann man jetzt die Wahrscheinlichkeit für einen systolischen
Blutdruck unter 110\,\text{mmHg} ableiten:
\[
P(X\le 110)
=
P\biggl(Z\le \frac{X-\mu}{\sigma}\biggr)
=
\Phi\biggl(
\frac{X-\mu}{\sigma}
\biggr)
=
\Phi(-1.178)
=
0.1193
=
11.93\%.
\qedhere
\]
\end{loesung}

\begin{bewertung}
Normalverteilung ({\bf N}) 1 Punkt,
Standardisierung ({\bf S}) 1 Punkt,
Gleichungssystem für ({\bf G}) 1 Punkt,
Werte für $\mu$ ({\bf M}) und $\sigma$ ({\bf D}) je 1 Punkt,
Wahrscheinlichkeit ({\bf W}) 1 Punkt.
\end{bewertung}

