Rüben werden nach der Ernte von einer Maschine nach Durchmesser
in drei Kategorien sortiert. Dünne Rüben haben einen Durchmesser
kleiner als 1cm, dicke einen solchen grösser als 2.5cm. In einem
Fall fallen 5\% der Rüben in die Kategorie der dünnen Rüben,
aber 25\% in die Kategorie der dicken Rüben. Wie dick sind
die Rüben im Mittel?

\thema{Normalverteilung}
\thema{Mittelwert}
\thema{Standardabweichung}

\begin{loesung}
Wir können annehmen, dass der Durchmesser eine normalverteilte
Zufallsvariable $X$ ist
mit Mittelwert $\mu$ und Standardabweichung $\sigma$. Aus den
Angaben in der Aufgabenstellung folgt
\begin{align*}
P(X>2.5)&= 0.25\\
P(X<1)&= 0.05
\end{align*}
Wir standardisieren:
\begin{align*}
P(X>2.5)&=P\left(
\frac{X-\mu}{\sigma}>\frac{2.5-\mu}{\sigma}
\right)=0.25
\\
P(X<1)&=P\left(
\frac{X-\mu}{\sigma}<\frac{1-\mu}{\sigma}
\right)=0.05
\end{align*}
Mit Hilfe der Tabelle der Quantilen der Normalverteilung finden
wir jetzt die Gleichungen
\begin{align*}
\frac{2.5-\mu}{\sigma}&=0.6745
\\
\frac{1-\mu}{\sigma}&=-1.6449
\end{align*}
oder
\begin{align*}
2.5&=\mu+0.6745\sigma\\
1&=\mu-1.6449\sigma
\end{align*}
Dieses lineare Gleichungssystem kann nach $\mu$ und $\sigma$ aufgelöst
werden, man findet
\[
\mu=2.06379,\qquad
\sigma=0.64672.
\qedhere
\]
\end{loesung}

