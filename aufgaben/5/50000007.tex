R"uben werden nach der Ernte von einer Maschine nach Durchmesser
in drei Kategorien sortiert. D"unne R"uben haben einen Durchmesser
kleiner als 1cm, dicke einen solchen gr"osser als 2.5cm. In einem
Fall fallen 5\% der R"uben in die Kategorie der d"unnen R"uben,
aber 25\% in die Kategorie der dicken R"uben. Wie dick sind
die R"uben im Mittel?

\begin{loesung}
Wir k"onnen annehmen, dass der Durchmesser eine normalverteilte
Zufallsvariable $X$ ist
mit Mittelwert $\mu$ und Standardabweichung $\sigma$. Aus den
Angaben in der Aufgabenstellung folgt
\begin{align*}
P(X>2.5)&= 0.25\\
P(X<1)&= 0.05
\end{align*}
Wir standardisieren:
\begin{align*}
P(X>2.5)&=P\left(
\frac{X-\mu}{\sigma}>\frac{2.5-\mu}{\sigma}
\right)=0.25
\\
P(X<1)&=P\left(
\frac{X-\mu}{\sigma}<\frac{1-\mu}{\sigma}
\right)0.05
\end{align*}
Mit Hilfe der Tabelle der Quantilen der Normalverteilung finden
wir jetzt die Gleichungen
\begin{align*}
\frac{2.5-\mu}{\sigma}&=0.6745
\\
\frac{1-\mu}{\sigma}&=-1.6449
\end{align*}
oder
\begin{align*}
2.5&=\mu+0.6745\sigma\\
1&=\mu-1.6449\sigma
\end{align*}
Dieses lineare Gleichungssystem kann nach $\mu$ und $\sigma$ aufgel"ost
werden, man findet
\[
\mu=2.06379,\qquad
\sigma=0.64672.
\]
\end{loesung}

