%
% shot noise 
%
% (c) 2019 Prof Dr Andreas Müller, Hochschule Rapperswil
%
%
% Vendor Sensor    Dimensionen    Pixelgrösse      Full well    AD Auflösung
% Sony   IMX586    8000 x 6000    0.8              4500
%
Der CMOS-Bildsensor IMX586 von Sony mit seinen 48 Megapixeln ist sehr beliebt
in Mobiltelefonen. 
Er hat sehr kleine Pixel von nur $0.8\,\mu\text{m}$ Kantenlänge, die im
Maximum etwa 4500 Elektronen fassen können.
Ein einfallendes Photon wird mit Wahrscheinlichkeit 56\% in ein Elektron
umgewandelt.
Am Ende der Belichtungszeit werden die Elektronen aus dem Pixel ausgelesen
und von einem A/D-Wandler gezählt.
\begin{teilaufgaben}
\item
In einem hellen Bildteil fallen während der Belichtungszeit $7000$\,Photonen
auf den Pixel.
Welchen Wert (Signal) erwarten Sie am Ausgang des A/D-Wandlers und welche
Varianz hat er?
\item
In einem dunklen Bildteil fallen während der Belichungszeit nur $200$\,Photonen
auf einen Pixel.
Welchen Wert erwartet Sie am Ausgang des A/D-Wandlers und welche Varianz
hat er?
\item
Wenn während der Belichtungszeit $n$ Photonen auf den Pixel fallen,
wie gross ist das erwartet Signal und die Varianz?
\item
Das Rauschen ist die Standardabweichung von $X$.
Wie gross ist das Verhältnis von Rauschen zu Signal und wie hängt
es von $n$ ab?
\end{teilaufgaben}
Das in dieser Aufgabe beschriebene Rauschphänomen heisst {\em Schrotrauschen}
(engl.~{\em shot noise}, auch bekannt als Schottky-Rauschen).
Es manifestiert sich in unregelmässigen Helligkeitsschwankungen, die
vor allem bei geringer Bildhelligkeit auffallen
(Abbildung~\ref{50000039:shotnoise}).

\begin{diskussion}
Schrotrauschen ist unvermeidlich, wenn sich ein Signal aus Quanten
zusammensetzt, die mit einer gewissen Wahrscheinlichkeit detektiert werden.
Es fällt vor allem bei geringer Intensität auf.
Im Gegensatz zum thermischen Rauschen, welches unabhängig von der
Intensität ist, lässt es sich durch Kühlung nicht reduzieren.

Eine weitere Quelle von Rauschen bei Bildsensoren ist die Umwandlung
der Ladung in einem Pixel in eine elektrische Spannung. 
Bei modernen CMOS-Sensoren ist dieses Rauschen von der Grössenordnung
eines einzelnen Elektrons, also vernachlässigbar im Vergleich zum
Schrotrauschen.
Das thermische Rauschen wird bei Chip-Temperaturen um $-20^\circ\text{C}$
klein im Vergleich zum Schrotrauschen.

Da der Rauschanteil mit $1/\sqrt{n}$ geht, helfen nur grössere Pixel mit
einer grösseren Zahl von Elektronen gegen das Schrotrauschen.
Der grossformatige CCD-Sensoren wie der KAF-16803 von Onsemi hat
Pixel mit $9\,\mu\text{m}$ Kantenlänge, die etwa 100000 Elektronen
fassen können. 
Die Quanteneffizienz ist $p=0.6$.
Daraus ergibt sich für den Rauschanteil nahe der Sättigung der
Wert $0.0026$, mehr als zehnmal besser als der beschriebene
CMOS-Sensor.

In der Astrophotographie und anderen Gebieten (z.~B.~Fluoreszenzmikroskopie
in der Biologie), wo man sehr geringe Signalintensitäten aufzeichnen muss,
erhöht man die Belichtungszeit, weil sich dadurch die Zahl $n$ erhöht. 
Wenn man statt einer einzelnen Aufnahmen 100 Aufnahmen mit der gleichen
Belichtungszeit addiert, verkleinert sich der Rauschanteil um den
Faktor 10.
\end{diskussion}

\begin{figure}[h]
\centering
\includeagraphics[width=1.0\hsize]{shotnoise.png}
\caption{Schrottrauschen eines Bildsensors
\label{50000039:shotnoise}}
\end{figure}

\thema{Binomialverteilung}
\thema{Normalapproximation}

\begin{loesung}
Die Zahl $X$ der Elektronen, die vom A/D-Wandler gezählt wird, entsteht
als Resultat eines wiederholten Bernoulli-Experiments mit Wahrscheinlichkeit
$p=0.56$.
Für $n$ einfallende Elektronen erwartet man $E(X)=np$ gezählte Elektronen
mit einer Varianz $\operatorname{var}(X)=np(1-p)$.
\begin{teilaufgaben}
\item
Für $n=7000$ ist das erwartete Signal $np=3920$ mit einer Varianz
von $np(1-p)=1724.8$.
\item
Für $n=200$ ist das erwartete Signal $np=112$ mit einer Varianz von
$np(1-p)=49.28$.
\item
$E(X)=np$, $\operatorname{var}(X)=np(1-p)$
\item
Die Standardabweichung ist $\sqrt{\mathstrut np(1-p)}$ und das Verhältnis
zwischen Rauschen und Signal ist
\[
\frac{\sqrt{\mathstrut\operatorname{var}(X)}}{E(X)}
=
\frac{\sqrt{\mathstrut np(1-p)}}{np}
=
\frac{1}{\sqrt{\mathstrut n}}
\cdot
\sqrt{\frac{1-p}{p}}
=
0.8864\cdot\frac{1}{\sqrt{n\mathstrut}}.
\]
Der Rauschanteil wird also mit zunehmender Signalintensität kleiner.
Für die beiden Werte $n=7000$ und $n=800$ sind die Verhältnisse
\begin{align*}
\frac{0.8864}{\sqrt{800}} &= 0.0313
&
&\text{and}&
\frac{0.8864}{\sqrt{7000}} &= 0.0106,
\end{align*}
trotz des grösseren Absolutwertes des Rauschens bei grösserer Intensität
ist das Verhältnis zwischen Signal und Rauschen bei hoher Intensität
dreimal besser als bei niedriger Intensität.
\qedhere
\end{teilaufgaben}
\end{loesung}

\begin{bewertung}
\begin{teilaufgaben}
\item 2 Punkte für Erwartungswert und Varianz
\item 2 Punkte für Erwartungswert und Varianz
\item 1 Punkt für Formeln
\item Verhältnis 1 Punkt.
\end{teilaufgaben}
\end{bewertung}


