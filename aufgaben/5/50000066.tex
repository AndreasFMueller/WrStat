Die Arbeitsgruppe \emph{Digitale Migration} hat im Rahmen des Swiss Radio Day
am 31.~August 2023 mitgeteilt, dass 81\% aller gehörten Radiominuten
in der Schweiz digital sind (Quelle: BAKOM).
Die Hälfte davon erfolgt über DAB+, vor allem bei den über 55-jährigen,
der Rest über das Internet.
Sogar im Auto erfolgt 70\% der Radionutzung digital, 54\% der
Autos in der Schweiz sind DAB+-fähig.
8 von 100 Radiohörern in der Schweiz hören Radio immer noch nur analog
über UKW.
\begin{teilaufgaben}
\item
Wie gross ist die Wahrscheinlichkeit, in einer Stichprobe von
100 Autos bei weniger als der Hälfte DAB+-fähige Radios zu finden?
\item
UKW-Hörer scheinen ziemlich selten geworden zu sein.
Wie gross ist die Wahrscheinlichkeit, dass von 10 zufällig gewählten
Radiohörern in der Schweiz mehr als 2 ausschliesslich über UKW Radio hören?
\end{teilaufgaben}

\begin{loesung}
\begin{teilaufgaben}
\item
Die Anzahl $X$ der DAB+-fähigen Autos ist eine binomialverteilte
Zufallsvariable mit $p=0.54$ und $n=100$.
Gesucht ist die Wahrscheinlichkeit
\begin{align*}
P(X < 50)
&=
P\biggl(
\frac{X-\mu}{\sigma}\le \frac{50-\mu}{\sigma}
\biggr)
\intertext{mit $\mu=np=54$ und $\sigma=\sqrt{np(1-p)}=4.984$.
Da $(X-\mu)/\sigma$ ungefähr standardnormalverteilt ist, kann
man die Wahrscheinlichkeit administrieren}
P(X<50)
&\approx
P\biggl(Z\le \frac{49.5-\mu}{\sigma}\biggr)
=
P(Z\le -0.9029)
\\
&=
\Phi(-0.9029)
=
1-\Phi(0.9029)
=
0.1833.
\end{align*}
Unterlässt man die Korrektur nach deMoivre-Laplace, findet man 
$P(X<50)\approx 0.2044$.
Die direkte Berechnung mit der Binomialverteilung für $n=100$, $p=0.54$ ergibt
die Wahrscheinlichkeit $0.1832$, die Approximation durch die Normalverteilung
ist also sehr genau.
\item
Da es sich bei Nur-UKW-Hörern um seltene Ereignisse handelt, kann deren
Zahl $Y$ mit der Poisson-Verteilung modelliert werden kann.
Die erwartete Anzahl von Nur-UKW-Hörern pro 10 Radiohörern ist
$\lambda=0.08\cdot 10=0.8$.
Mit der Poisson-Verteilung ergibt sich daraus
\begin{align*}
P(Z > 2)
&=
1-P(Z\le 2)
\\
P(Z\le 2)
&\approx
e^{-\lambda} \sum_{k=0}^2 \frac{\lambda^2}{2!}
=
e^{-\lambda} \biggl(1+\lambda+\frac{\lambda^2}{2!}\biggr)
=
0.9525774
\\
P(Z>2)
&\approx
0.0474226
\approx
4.74\%
\end{align*}
Die direkte Berechnung mit der Binomialverteilung für $n=10$ und
$p=0.08$ ergibt 4.01\%.

Die Approximation ist in diesem Fall nicht so gut, weil $n$ relativ
klein ist.
Für $n=1000$ und $p=0.0008$ liefert die Binomialverteilung eine
Wahrscheinlichkeit von 4.735354\%, der Fehler der Approximation ist
nur noch $0.000069$.
\qedhere
\end{teilaufgaben}
\end{loesung}

\begin{bewertung}
Binomialverteilung ({\bf B}) 1 Punkt,
Berechnung von $\mu$ und $\sigma$ ({\bf M}) 1 Punkt,
Standardisierung ({\bf S}) 1 Punkt,
Normalapproximation und Wahrscheinlichkeit ({\bf N}) 1 Punkt,
Parameter $\lambda$ ({\bf L}) 1 Punkt,
Poisson-Verteilung und Wahrscheinlichkeit ({\bf P}) 1 Punkt.
\end{bewertung}

\begin{diskussion}
Die ursprüngliche Formulierung der Teilaufgabe b) war leider etwas
unglücklich und hat zu einer Interpretation verleitet, nach der ein
Ereignis mit Wahrscheinlichkeit 0.19 als selten hätte betrachtet werden
müssen, was sicher nicht sinnvoll ist.
Die obige Formulierung korrigiert dies und stellt sicher, dass ein
Ereignis mit Wahrscheinlichkeit 0.08 betrachtet wird, für das die
Approximation mit der Poisson-Verteilung, um die es in dieser Teilaufgabe
ging, angemessener ist.
\end{diskussion}
