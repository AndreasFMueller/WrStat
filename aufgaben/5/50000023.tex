Es wird behauptet, die Zufallsvariable $X$ hätte die Wahrscheinlichkeitsdichte
\[
\varphi(x)=\begin{cases}
0&\qquad x\le 0\\
\frac12\sin x&\qquad 0\le x\le \pi\\
0&\qquad x > \pi
\end{cases}
\]
\begin{teilaufgaben}
\item Ist $\varphi$ überhaupt eine Wahrscheinlichkeitsdichte?
\item Bestimmen Sie die Verteilungsfunktion $F_X(x)$.
\item An welcher Stelle erreicht die Verteilungsfunktion den Wert $0.5$?
\item Bestimmen Sie den Erwartungswert $E(X)$.
\item Bestimmen Sie die Varianz $\operatorname{var}(X)$.
\end{teilaufgaben}

\thema{Wahrscheinlichkeitsdichte}
\thema{Verteilungsfunktion}
\thema{Erwartungswert}
\thema{Varianz}

\begin{loesung}
\begin{teilaufgaben}
\item
$\varphi(x)\ge 0$ und es gilt
\[
\int_{-\infty}^\infty\varphi(x)=\int_0^\pi\frac12\sin x\,dx
=
\frac12\left[-\cos x\right]_0^\pi=\frac12(1+1)=1,
\]
also ist $\varphi$ eine Wahrscheinlichkeitsdichte.
\item
Die Verteilungsfunktion ist eine Stammfunktion, interessant ist nur
der Teil für $0\le x\le \pi$:
\[
\int_{-\infty}^x\varphi(\xi)\,d\xi
=
\int_0^x\frac12\sin\xi\,d\xi
=
\frac12\left[-\cos\xi\right]_0^\xi=\frac12(1-\cos x).
\]
Die Verteilungsfunktion ist daher:
\[
F_X(x)=\begin{cases}
0&\qquad x\le 0\\
\frac12(1-\cos x)&\qquad 0\le x\le \pi\\
1&\qquad x > \pi.
\end{cases}
\]
\item Die Verteilungsfunktion erreicht den Wert an der Stelle $x$, für die
\begin{align*}
F_X(x)=\frac12(1-\cos x)&=\frac12\\
1-\cos x&=1\\
\cos x&=0\\
x&=\frac{\pi}2
\end{align*}
gilt.
\item
Der Erwartungswert ist
\begin{align*}
E(X)
&=
\int_{-\infty}^\infty x\varphi(x)\,dx
=
\int_0^\pi x\cdot\frac12\sin x\,dx
=
\frac12
\left[\sin x-x\cos x\right]_0^\pi=\frac{\pi}2.
\end{align*}
\item
Für die Varianz brauchen wir nur noch den Erwartungswert von $X^2$:
\begin{align*}
E(X^2)
&=
\int_{-\infty}^\infty x^2\varphi(x)\,dx
=
\frac12\int_{-\infty}^\infty x^2\sin x\,dx
=
\frac12\int_0^\pi x^2\sin x\,dx 
\\
&=
\frac12\left[
2x\sin x-(2-x^2)\cos x
\right]_0^\pi
=
\frac12\pi^2-2
\\
\operatorname{var}(X)&=E(X^2)-E(X)^2=
\frac12\pi^2-2-\biggl(\frac{\pi}2\biggr)^2
=\frac{\pi^2}4-2
\qedhere
\end{align*}
\end{teilaufgaben}
\end{loesung}

\begin{bewertung}
\begin{teilaufgaben}
\item 1 Punkt
\item 1 Punkt
\item 1 Punkt
\item 1 Punkt
\item 1 Punkt
\item Erwartungswert von $X^2$ ($\textbf{X}^2$) 1 Punkt,
Varianzberechung (\textbf{V}) 1 Punkt.
\end{teilaufgaben}
\end{bewertung}

