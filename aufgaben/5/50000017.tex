Sei $X$ eine im Interval $[a,b]$ gleichverteilten Zufallsvariable.
\begin{teilaufgaben}
\item Berechnen Sie das sogenannte dritte Moment $E(X^3)$ von $X$
\item Die ``Schiefe'' einer Verteilung ist der Ausdruck
\[
\frac{E((X-E(X))^3)}{\operatorname{var}(X)^{\frac32}}
=
\frac{E(X^3)-3E(X)\operatorname{var}(X)-E(X)^3}{\operatorname{var}(X)^{\frac32}}.
\]
Berechnen Sie die Schiefe.
\end{teilaufgaben}

\begin{loesung}
\begin{teilaufgaben}
\item
Für $E(X^3)$ können wir einfach drauflos rechnen:
\begin{align*}
E(X^3)
&=
\int_{-\infty}^\infty x^3\varphi(x)\,dx
=
\int_a^bx^3\frac1{b-a}\,dx
=
\frac1{b-a}\left[\frac14x^4\right]_a^b
\\
&=
\frac14\frac{b^4-a^4}{b-a}
=
\frac{b^3+b^2a+ba^2+a^3}{4}
\end{align*}
\item Aus der Berechnung des dritten Momentes kann man jetzt
die Schiefe ableiten, wobei wir zur Vereinfachung zunächst
nur den Zähler ausrechnen:
\begin{align*}
\text{Schiefe}
&=
E(X^3)-3E(X)\operatorname{var}(X)-E(X)^3
\\
&=
\frac{b^3+b^2a+ba^2+a^3}{4}
-3\frac{a+b}2\frac{(b-a)^2}{12}
-\frac{(a+b)^3}{8}
\\
&=
\frac18\bigl(
2(b^3+b^2a+ba^2+a^3)
-(a^3-a^2b-ab^2+b^3)
\\
&\qquad
-(a^3+3a^2b+3ab^2+b^3)
\bigr)
=0
\qedhere
\end{align*}
\end{teilaufgaben}
\end{loesung}

