In der Silvesternacht wird jeweils aus unerfindlichen Gründen eine grosse
Menge Feuerwerk durch Privatpersonen abgebrannt.
Angeblich soll es im neuen Jahr Glück bringen, doch Länder, in denen
diese Form von Aberglauben nicht üblich ist, scheinen nicht weniger
von Glück gesegnet zu sein.
Natürlich ist dieses Geknalle meistens völlig desorganisiert, scheint
sich aber ungefähr um Mitternacht zu konzentrieren.
Etwa 5\%  scheinen aber nicht warten zu können und brennen ihr
Feuerwerk schon vor 22 Uhr abends ab.
Aber auch nach Mitternacht hört das Geballer nicht auf, 25\% werden erst
nach 1~Uhr verfeuert.
\begin{teilaufgaben}
\item Zu welcher Zeit erreicht die Explosionsdichte ihr Maximum?
\item Wie gross ist die Wahrscheinlichkeit, dass auch nach 2 Uhr
immer noch Feuerwerk explodiert?
\end{teilaufgaben}

\begin{loesung}
Wir nehmen an, dass Zeitpunkte, zu denen ein Feuerwerkskörper 
abgebrannt wird, eine normalverteilte Zufallsvariable $X$
ist mit Erwartungswert $\mu$ und Standardabweichung $\sigma$.
Die beiden Bedingungen lauten
\[
\begin{aligned}
P(X < 22) &= 0.05 &&           &         &       \\
P(X > 25) &= 0.25 &&\Rightarrow& P(X<25) &= 0.75
\end{aligned}
\]
oder nach Standardisierung
\begin{align*}
P\biggl( \frac{X-\mu}{\sigma} < \frac{21-\mu}{\sigma} \biggr) &= 0.05 \\
P\biggl( \frac{X-\mu}{\sigma} < \frac{25-\mu}{\sigma} \biggr) &= 0.75
\end{align*}
Da $Z=(X-\mu)/\sigma$ standardnormalverteilt ist, folgt
\[
\left.
\begin{aligned}
\frac{22-\mu}{\sigma} &=          - 1.6449 \\
\frac{25-\mu}{\sigma} &= \phantom{-}0.6745
\end{aligned}
\quad
\right\}
\qquad\Rightarrow\qquad
\left\{
\quad
\begin{linsys}{3}
22 &=& \mu &-& 1.6449\sigma\phantom{.}\\
25 &=& \mu &+& 0.6745\sigma.
\end{linsys}
\right.
\]
Dieses lineare Gleichungssystem kann zum Beispiel mit der Cramerschen
Regel gelöst werden, welche
\[
\mu
=
\frac{
\left|\begin{matrix*}[r]22&-1.6449\\25&0.6745\end{matrix*}\right|
}{
\left|\begin{matrix*}[r]1&-1.6449\\1&0.6745\end{matrix*}\right|
}
=
\frac{55.962}{2.3194}
=
24.1276
\qquad\text{und}\qquad
\sigma
=
\frac{
\left|\begin{matrix*}[r]1&22\\1&25\end{matrix*}\right|
}{
\left|\begin{matrix*}[r]1&-1.6449\\1&0.6745\end{matrix*}\right|
}
=
\frac{3}{2.3194}
=
1.2934
\]
ergibt.
\begin{teilaufgaben}
\item
Das Maximum wird zur Zeit $\mu$ erreicht, also um 00:08 Uhr.
\item
Die Wahrscheinlichkeit, auch nach 2 Uhr noch Explosionen zu hören ist
\[
P(X>26) = P\biggl(Z> \frac{26-\mu}{\sigma}\biggr)
=
1-
P(Z <  1.4477) = 1 - 0.9261 = 0.0739
\qedhere
\]
\end{teilaufgaben}
\end{loesung}

\begin{bewertung}
Normalverteilung ({\bf N}) 1 Punkt,
Standardisierung ({\bf S}) 1 Punkt,
Quantilen ({\bf Q}) 1 Punkt,
Gleichungssystem für $\mu$ und $\sigma$ ({\bf G}), 1 Punkt,
Teilaufgaben a) ({\bf A}) und b) ({\bf B}) je 1 Punkt.

\end{bewertung}
