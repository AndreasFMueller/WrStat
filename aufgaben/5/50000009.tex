Betrachten Sie die Funktion
\[
\varphi(x)=\begin{cases}
0&\qquad x < 0 \\
a(1 - x)&\qquad 0 \le x < 1\\
0&\qquad x>1
\end{cases}
\]
\begin{teilaufgaben}
\item Wie muss $a$ gew"ahlt werden, damit $\varphi$ eine Dichtefunktion
ist?
\item Wie lautet die zugeh"orige Verteilungsfunktion $F$?
\item Sei $X$ eine Zufallsvariable mit Verteilungsfunktion $F$, berechnen
Sie $E(X)$ und $\operatorname{var}(X)$.
\item Wie wahrscheinlich ist das Ereignis $\{ X > \frac12\}$?
\end{teilaufgaben}

\begin{loesung}
\begin{teilaufgaben}
\item
Damit $\varphi$ eine Dichtefunktion ist, muss $a$ so gew"ahlt sein,
dass das Integral von $\varphi$ "uber $\mathbb R$ den Wert 1 hat.
\begin{align*}
\int_{\mathbb R}\varphi(x)
&=
\int_{0}^{1}a(1-x)\,dx=a\left[x-\frac12x^2\right]_0^1
=
\frac{a}2
\end{align*}
Die richtige Wahl f"ur $a$ ist also $2$.
\item
Die Teilfunktion f"ur das Interval $[0,1]$ ist
\[
F(x)=\int_{0}^x 2(1-\xi)\,d\xi
=
2\left[\xi - \frac12\xi^2\right]_0^x
=
2(x-\frac12x^2)
=
2x-x^2.
\]
Ausserhalb dieses Intervals ist die Verteilungsfunktion konstant, also:
\[
F(x)=\begin{cases}
0&\qquad x < 0\\
2x-x^2
&\qquad 0 \le x < 1\\
1&\qquad x\ge1
\end{cases}
\]
\item
Erwartungswert:
\begin{align*}
E(X)&=\int_{\mathbb R} x\varphi(x)\,dx
=
\int_0^1x\cdot 2(1-x)\,dx
=
2\int_0^1 x-x^2\,dx
\\
&=
2\left[\frac12x^2-\frac13x^3\right]_0^1
=2\biggl(\frac12-\frac13\biggr)
=2\biggl(\frac3{6}-\frac{2}{6}\biggr)
=\frac2{6}=\frac13.
\end{align*}
F"ur die Varianz wird zus"atzlich $E(X^2)$ ben"otig:
\begin{align*}
E(X^2)
&=
\int_{\mathbb R} x^2\varphi(x)\,dx
=
\int_0^1x^2\cdot 2(1-x)\,dx
=
2\int_0^1 x^2-x^3\,dx
\\
&=\left[\frac23x^3-\frac24x^4\right]_0^1
=\frac23-\frac12=\frac4{6}-\frac{3}{6}=\frac{1}{6}.
\end{align*}
Damit kann jetzt die Varianz berechnet werden
\[
\operatorname{var}(X)=E(X^2)-E(X)^2=\frac{1}{6}-\left(\frac{1}{3}\right)^2
=\frac16-\frac19=\frac3{18}-\frac2{18}
=\frac{1}{18}=0.0555555.
\]
\item
Die Wahrscheinlichkeit kann direkt mit der Verteilungsfunktion
berechnet werden, indem man $x=\frac12$ einsetzt:
\[
P(X>\frac12)
=
1-F(\frac12)=1-\biggl(2\cdot \frac12-\frac14\biggr)=\frac14.
\]
\end{teilaufgaben}
\end{loesung}

