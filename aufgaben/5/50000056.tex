41\% der Amerikaner glauben, dass vor langer Zeit Aliens auf der
Erde gelandet sind und sich Spuren davon immer noch nachweisen lassen.
Dieser Unsinn geht auf Bücher von Erich von Däniken zurück und wurde
in den USA durch die Fernsehserie {\em Ancient Aliens}
mit ähnlich tiefen intellektuellen Standards popularisiert.

Im Vergleich dazu sind die Leute, die an die Thesen von Masaru Emoto
(\begin{CJK}{UTF8}{min}江本 勝\end{CJK}, 1943--2014) glauben,
vergleichsweise selten.
Dieser glaubt, dass die Bildung von Schneeflocken dadurch positiv
beeinflusst werden kann, indem man den Behälter des Wassers mit
positiven Botschaften beschriftet.
Wenn man die Zahl der Google-Treffer als Masszahl für die Verbreitung
solcher Pseudowissenschaften annimmt, dann sind Emotos Jünger etwa
hundert mal seltener.

Das Dorf Union im Staat Connecticut in den USA hat 693 Einwohner.
\begin{teilaufgaben}
\item
Wie wahrscheinlich ist es, mehr als 300 Leute in Union zu finden,
die an Ancient Aliens glauben?
\item
Wie wahrscheinlich ist es, mehr als drei Leute in Union zu finden,
die an Masaru Emotos belebtes Wasser glauben?
\end{teilaufgaben}

\begin{loesung}
\begin{teilaufgaben}
\item
Sei $X$ die Zufallsvariable der Anzahl der Ancient-Aliens-Anhänger in Union.
Sie ist binomialverteilt mit $n=693$ und $p=0.41$.
Der Erwartungswert ist $\mu=np=284.13$ und die Standardabweichung
ist $\sigma=\sqrt{np(1-p)}\approx 12.9475$.
Mit der Normalverteilungsapproximation der Binomialverteilung
kann man jetzt die Wahrscheinlichkeit ausrechnen.
Dazu bestimmt man die Standardisierung $Z=(X-\mu)/\sigma$ und
bekommt
\begin{align*}
P(X\le 300)
&\approx
P\biggl(Z < \frac{300.5-\mu}{\sigma}\biggr)
=
P(Z < 1.26434) = 0.8969461,
\\
P(X>300)
&\approx 0.10305,
\end{align*}
die gesuchte Wahrscheinlichkeit ist $10.3\%$.
\item
Da der Glaube an Masaru Emotos Thesen so selten ist, kann man ihn
mit einer
Poisson-Verteilung mit dem Parameter $\lambda = n\cdot 0.0041=2.8413$
beschreiben.
Sei $X$ die Anzahl der Emoto-Jünger, dann ist die Wahrscheinlichkeit,
mehr als 3 Emoto-Jünger zu finden
\begin{align*}
P(X>3)
&\approx
1-\sum_{k=0}^3 P_\lambda(k)
=
1-e^{-\lambda} \sum_{k=0}^2 \frac{\lambda^k}{k!}
\\
&=
1-e^{-\lambda}\biggl(1+\lambda+\frac{\lambda^2}{2}+\frac{\lambda^3}{3!}\biggr)
=
1- 0.68274
=
0.31726.
\end{align*}
Die Wahrscheinlichkeit, mehr als 3 Emoto-Jünger zu finden, ist daher 
immer noch 31.73\%.
\qedhere
\end{teilaufgaben}
\end{loesung}

\begin{bewertung}
\begin{teilaufgaben}
\item
Binomialverteilung ({\bf B}) 1~Punkt,
Erwartungswert und Varianz ({\bf E}) 1~Punkt,
Standardisierung ({\bf S}) 1~Punkt,
Wahrscheinlichkeit ({\bf W}) 1~Punkt,
\item
Poisson-Verteilung und Wert des Parameters ({\bf P}) 1~Punkt,
Resultierende Wahrscheinlichkeit ({\bf R}) 1~Punkt.
\end{teilaufgaben}
\end{bewertung}



