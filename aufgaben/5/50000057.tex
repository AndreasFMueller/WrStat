Das Institut für Raum Quanten Physik von Hans Lehner ist eine
pseudowissenschafltliche Randerscheinung.
Es ging hervor aus dem Konkurs der Rapperswiler Raum-Quanten-Motoren AG,
die seit den frühen 90er Jahren mit abstrusen pseudophysikalischen
Theorien verspricht, Energie aus dem Nichts zu gewinnen.
Hunderte von Investoren glauben zum Teil immer noch,
dass die Ablehnung Lehners durch Leute, die etwas von Physik verstehen,
eine Verschwörung rückständiger Forscher sei.
Über 11~Mio Franken Schaden sind durch den Konkurs entstanden.
das Institut führte die agressive Akquisition von leichtgläubigen
Investoren fort, seit 2015 gab es auf der Website allerdings
keine Updates mehr.
In diesem Jahr ist Lehner in Thailand gestorben.

Die Corona-Leugner dagegen sind eine vergleichsweise verbreitete
Erscheinung.
Waren es im Jahre 2020 noch etwa 14\%, ist die Zahl 2021 auf 9\%
gesunken.
Will man die Anzahl Google-Suchresultate als Mass für die Akzeptanz
einer Verschwörungstheorie ansehen, dann hat der Raum-Quanten-Motor 
mindstens 100mal weniger Anhänger\footnote{Anzahl Suchresultate zu
\texttt{raum quanten physik}: 528, Anzahl resultate zu
\texttt{``mass-voll'' corona}: 56100.}.

\begin{teilaufgaben}
\item 
Wie gross ist die Wahrscheinlichkeit, dass in einer Gemeinde mit
1000 Einwohnern weniger als 70 Schwurbler leben?
\item 
Wie gross ist die Wahrscheinlichkeit, in der gleichen Gemeinde
mehr als zwei Einwohner zu finden, die an den Raum-Quanten-Motor
glauben?
\end{teilaufgaben}


\begin{loesung}
\begin{teilaufgaben}
\item
Sei $X$ die Zufallsvariable der Anzahl der Schwurbler.
Sie ist binomialverteilt mit $n=1000$ und $p=0.09$.
Der Erwartungswert ist $\mu=np=90$ und die Standardabweichung
ist $\sigma=\sqrt{np(1-p)}\approx 9.04986$.
Mit der Normalverteilungsapproximation der Binomialverteilung
kann man jetzt die Wahrscheinlichkeit ausrechnen.
Dazu bestimmt man die Standardisierung $Z=(X-\mu)/\sigma$ und
bekommt
\begin{align*}
P(X< 90)
&\approx
P\biggl(Z < \frac{69.5 -\mu}{\sigma}\biggr)
=
P(Z < -2.2652)
=
1-P(Z>2.2652)
=
1-0.98825
\approx
0.01174935,
\end{align*}
die gesuchte Wahrscheinlichkeit ist $1.17\%$.
\item
Da der Glaube an Raum Quanten Motor so selten ist, kann man ihn
mit einer
Poisson-Verteilung mit dem Parameter $\lambda = n\cdot 0.0009=0.9$
beschreiben.
Sei $X$ die Anzahl der RQM-Anhänger, dann ist die Wahrscheinlichkeit,
mehr als 2 RQM-Anhänger zu finden
\begin{align*}
P(X>2)
&\approx
1-\sum_{k=0}^2 P_\lambda(k)
=
1-e^{-\lambda} \sum_{k=0}^2 \frac{\lambda^k}{k!}
\\
&=
1-e^{-\lambda}\biggl(1+\lambda+\frac{\lambda^2}{2}\biggr)
=
1- 0.9371
=
0.062857.
\end{align*}
Die Wahrscheinlichkeit, mehr als 2 RQM-Anhänger zu finden, ist daher 
höchstens 6.28\%.
\qedhere
\end{teilaufgaben}
\end{loesung}

\begin{bewertung}
\begin{teilaufgaben}
\item
Binomialverteilung ({\bf B}) 1~Punkt,
Erwartungswert und Varianz ({\bf E}) 1~Punkt,
Standardisierung ({\bf S}) 1~Punkt,
Wahrscheinlichkeit ({\bf W}) 1~Punkt,
\item
Poisson-Verteilung und Wert des Parameters ({\bf P}) 1~Punkt,
Resultierende Wahrscheinlichkeit ({\bf R}) 1~Punkt.
\end{teilaufgaben}
\end{bewertung}



