Eine Zufallsvariable $X$ hat die Wahrscheinlichkeitsdichte
\[
\varphi_X(x)=\begin{cases}
0&\qquad x <-1.5\\
a&\qquad -1.5\le x<-0.5\\
0&\qquad -0.5\le x<0.5\\
a&\qquad 0.5\le x<1.5\\
0&\qquad 1.5<x
\end{cases}
\]
\begin{teilaufgaben}
\item Wie gross muss $a$ sein?
\item Bestimmen Sie die Verteilungsfunktion.
\item Bestimmen Sie Erwartungswert und Varianz.
\end{teilaufgaben}

\thema{Wahrscheinlichkeitsdichte}
\thema{Verteilungsfunktion}
\thema{Erwartungswert}
\thema{Varianz}

\begin{loesung}
\begin{teilaufgaben}
\item $a$ muss so gross ein, dass
\[
1=\int_{-\infty}^\infty \varphi_X(x)\,dx=
\int_{-1.5}^{-0.5}a\,dx+\int_{0.5}^{1.5}a\,dx=2a
\]
gilt, also $a=\frac12$.
\item Die Verteilungsfunktion ist eine Stammfunktion von $\varphi_X(x)$.
Da die $\varphi_X(x)$ stückweise konstant ist, ist $\varphi_X(x)$
stückweise linear:
\[
F_X(x)=\begin{cases}
0&\qquad x <-1.5\\
\frac12(x+1.5)=\frac12x+\frac34&\qquad -1.5\le x<-0.5\\
\frac12&\qquad -0.5\le x<0.5\\
\frac12+\frac12(x-0.5)=\frac12x+\frac14&\qquad 0.5\le x<1.5\\
1&\qquad 1.5<x
\end{cases}.
\]
\item Da die Verteilungsfunktion symmetrisch ist, ist $E(X)=0$, und
für die Varianz gilt die vereinfachte Formel $\operatorname{var}(X)=E(X^2)$.
Dies kann man mit Hilfe eines Integrals ausrechnen:
\begin{align*}
\operatorname{var}(X)
&=\int_{-\infty}^\infty \varphi_X(x)\,dx\\
&=\int_{-1.5}^{-0.5}x^2\cdot\frac12\,dx+\int_{0.5}^{1.5} x^2\cdot \frac12\,dx\\
&=2\int_{0.5}^{1.5}\frac{x^2}2\,dx=\int_{0.5}^{1.5}x^2\,dx\\
&=\left[\frac13x^3\right]_{0.5}^{1.5}
 =\frac13\left(\frac32\right)^3-\frac13\left(\frac12\right)^3
=\frac13\left(\frac{27}8-\frac18\right)=
\\
&=\frac{26}{24}=\frac{13}{12}
\simeq 1.08333333333333333333
\qedhere
\end{align*}
\end{teilaufgaben}
\end{loesung}

