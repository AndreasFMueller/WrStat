Schreiben Sie ein Programm, mit welchem Sie Zufallszahlen mit der
Verteilungsfunktion
\[
F(x)=\begin{cases}
0&\qquad x< 0\\
1-e^{-x}&\qquad x\ge 0
\end{cases}
\]
erzeugen k"onnen. Wir nennen diese Zufallswerte die Zufallsvariable $X$.
\begin{teilaufgaben}
\item Bestimmen Sie durch Simulation den Erwartungswert und die Varianz
von $X$.
\item Wieviele Zufallszahlen m"ussen Sie ber"ucksichtigen, damit Ihre
Simulationsrechnung mit Wahrscheinlichkeit 1\% einen auf drei Stellen
nach dem Komma genauen Wert liefert?
\end{teilaufgaben}

\begin{loesung}
\begin{teilaufgaben}
\item Die Zufallsvariable $F(X)$ ist gleichverteilt, d.~h.~man kann aus
einer gleichverteilten Zufallsvariablen $Y$ durch Anwendung der Inversen
Funktion $F^{-1}$ eine Zufallsvariable mit Verteilungsfunktion $F$ bekommen.
Die inverse Funktion ist f"ur positive $y$:
\begin{align*}
y&=F(x)=1-e^{-x}\\
e^{-x}&=1-y\\
x&=-\log(1-y)
\end{align*}
Die verlangten Simulationen lassen sich mit dem Program
\verbatimainput{sim.expanded.c} 
durchf"uhren. Man findet $E(X)\simeq 1$ und $\operatorname{var}(X)\simeq 1$.
\item
Das Gesetz der grossen Zahlen von Bernoulli besagt, dass 
eine Abweichung des Mittelwertes $M_n$ von mehr als $\varepsilon= 0.0005$
vom Erwartungswert die Wahrscheinlichkeit
\[
P(|M_n-\mu| \le \varepsilon)\le \frac{\operatorname{var}(X)}{n\varepsilon^2}
\]
hat. Die Simulation hat die Varianz $1$ geliefert.
Um die Schranke von 1\% zu unterschreiten braucht es also
\[
\frac1{n\varepsilon^2}\le 0.01
\qquad\Rightarrow\qquad
\frac1{0.01\cdot 0.0005^2}=400000000\le n,
\]
400Mio Zufallswerte m"ussen gemittelt werden. Experimentell findet man
jedoch, dass viel weniger Werte auch ausreichen, dies hat wieder mit der
Tatsache zu tun, dass die Tschebyscheff-Ungleichung sehr grob ist, und
daher f"ur ``gute'' Zufallsvariablen zu weite Schranken findet.
\end{teilaufgaben}
\end{loesung}

