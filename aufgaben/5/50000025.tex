Sei $X$ eine Zufallsvariable mit Verteilungsfunktion $F(x)=F_X(x)$, und
sei\[
Z=\frac{X-\mu}{\sigma},
\]
wobei $\mu$ und $\sigma$ beliebige Zahlen sind mit $\sigma > 0$.

\begin{teilaufgaben}
\item Bestimmen Sie die Verteilungsfunktion der Zufallsvariablen $Z$.
\item Bestimmen Sie die Dichtefunktion der Zufallsvariablen $Z$.
\end{teilaufgaben}
\begin{hinweis}
``Bestimmen'' heisst in diesem Fall $F_Z(x)$ bzw.~$\varphi_Z(x)$ ausdrücken durch
$F_X(x)$ bzw.~$\varphi_X(x)$.
\end{hinweis}

\thema{Rechenregeln}
\thema{Wahrscheinlichkeitsdichte}

\begin{loesung}
\begin{teilaufgaben}
\item Wir verwenden den Standardalgorithmus:
\begin{align*}
F_Z(z)&=P(Z\le z)
=P\biggl(\frac{X-\mu}{\sigma}\le z\biggr)
=P(X\le \sigma z+\mu)=F_X(\sigma z+\mu).
\end{align*}
Die Verteilungsfunktion von $Z$ ist also $F_Z(z)=F_X(\sigma z+\mu)$.
\item Die Dichtefunktion ist die Ableitung der Verteilungsfunktion:
\begin{align*}
\varphi_Z(z)&=\frac{d}{dz}F_Z(z)
=\frac{d}{dz}F_X(\sigma z+ \mu)
=F_X'(\sigma z+ \mu)\cdot \sigma=\sigma\varphi_X(\sigma z+\mu).
\qedhere
\end{align*}
\end{teilaufgaben}
\end{loesung}

