Die Funktion
\[
\varphi(x)
=
\begin{cases}
a/\!\sqrt{x(1-x)}&\qquad 0 < x < 1\\
0                &\qquad\text{sonst}
\end{cases}
\]
ist bei korrekter Wahl von $a$ die Wahrscheinlichkeitsdichte der sogenannten
{\em Arkussinus-Verteilung}, die zum Beispiel in der Theorie des
Wiener-Prozesses eine grosse Bedeutung hat.
\begin{figure}[h]
\centering
\begin{tikzpicture}[>=latex,thick,scale=4]
\begin{scope}
\clip (-0.5,0) rectangle (1.5,1.5);
\fill[color=red!20] (0,0) -- (0,3)
	-- plot[domain=0.01:0.99,samples=100]
		({\x},{1/(3.1415*sqrt(\x*(1-\x)))})
	-- (1,3) -- (1,0) -- cycle;
\draw[color=red,line width=1.4pt] (-0.5,0) -- (0,0);
\draw[color=red,line width=1.4pt] (1.0,0) -- (1.5,0);
\draw[color=red,line width=1.4pt] plot[domain=0.01:0.99,samples=100]
	({\x},{1/(3.1415*sqrt(\x*(1-\x)))});
\end{scope}
\draw[->] (-0.5,0) -- (1.5,0) coordinate[label={$x$}];
\draw[->] (0,-0.02) -- (0,1.5) coordinate[label={left:$\varphi(x)$}];
\draw (1,-0.02) -- (1,0.02);
\node at (1,0) [below] {$1$};
\node at (0,0) [below] {$0$};
\draw (-0.02,1) -- (0.02,1);
\node at (0,1) [left] {$1$};

\ifthenelse{\boolean{loesungen}}{
\begin{scope}[xshift=2.1cm]
\draw[line width=0.3pt] (-0.5,1) -- (1.5,1);
\draw[->] (-0.5,0) -- (1.5,0) coordinate[label={$x$}];
\draw[->] (0,-0.02) -- (0,1.1) coordinate[label={right:$F_X(x)$}];
\draw[color=darkgreen,line width=1.4pt] (-0.5,0) -- (0,0)
	-- plot[domain=0:1,samples=100] ({\x},{2*asin(2*\x-1)/360+0.5}) 
	-- (1,1) -- (1.5,1);
\draw (1,-0.02) -- (1,0.02);
\node at (1,0) [below] {$1$};
\node at (0,0) [below] {$0$};
\draw (-0.02,1) -- (0.02,1);
\node at (0,1) [below left] {$1$};
\end{scope}}{}
\end{tikzpicture}
\ifthenelse{\boolean{loesungen}}{
\caption{Graph der Funktion $\varphi(x)$ von Aufgabe~\ref{50000061}.
Die Verteilungsfunktion ist im rechten Graphen dargestellt.
\label{50000061:fig}}
}{
\caption{Graph der Funktion $\varphi(x)$ von Aufgabe~\ref{50000061}.
\label{50000061:fig}}
}
\end{figure}

\begin{teilaufgaben}
\item
Wie muss $a$ gewählt werden, damit $\varphi$ tatsächlich eine
Wahrscheinlichkeitsdichte einer Zufallsvariable $X$ ist?
\item
Bestimmen Sie die Verteilungsfunktion $F_X(x)$.
\item
Betimmen Sie den Erwartungswert $E(X)$.
\item
Bestimmen Sie die Varianz $\operatorname{var}(X)$.
\end{teilaufgaben}

\begin{hinweis}
Verwenden Sie die Stammfunktionen
\begin{align}
\int\frac{dx}{\sqrt{x(1-x)}}
&=
\arcsin(2x-1) + C,
\label{50000061:int0}
\\
\int\frac{x\,dx}{\sqrt{x(1-x)}}
&=
\frac12\arcsin(2x-1) - \sqrt{x(1-x)} + C,
\label{50000061:int1}
\\
\int\frac{x^2\,dx}{\sqrt{x(1-x)}}
&=
\frac38\arcsin(2x-1) - \frac{x}2\sqrt{x(1-x)} - \frac34\sqrt{x(1-x)} + C.
\label{50000061:int2}
\end{align}
\end{hinweis}

\begin{loesung}
\begin{teilaufgaben}
\item
Damit $\varphi(x)$ eine Wahrscheinlichkeitsdichte ist, muss die
Normierungsbedingung
\[
1
=
\int_{-\infty}^\infty \varphi(x)\,dx
\]
erfüllt sein.
Aus der Formel~\eqref{50000061:int0} folgt
\begin{align*}
1
&=
\int_{-\infty}^\infty \frac{a\,dx}{\sqrt{x(1-x)}}
=
a\int_0^1 \frac{dx}{\!\sqrt{x(1-x)}}
\\
&=
a
\biggl[
\arcsin(2x-1)
\biggr]_0^1
=
a\bigl(\arcsin(1) - \arcsin(-1)\bigr)
=
a\pi
\qquad\Rightarrow\qquad
a=\frac{1}{\pi}\approx 0.318309886.
\end{align*}
\item
Als Verteilungsfunktion im Intervall $[0,1]$ kann die Funktion
\eqref{50000061:int0} verwendet werden, es muss allerdings die
richtige Integrationskonstante gewählt werden.
Sie muss so sein, dass $0$ der Funktionswert bei $x=0$ wird,
also $C=$
\[
F_X(x)
=
\begin{cases}
0&\qquad x\le 0\\
\frac{1}{\pi}\bigl(\arcsin(2x-1)+\frac{\pi}2\bigr) &\qquad 0<x\le 1\\
1&\qquad x>1
\end{cases}
\]
\item
Die Funktion $\varphi(x)$ ist symmetrisch bezüglich $x=\frac12$, daher ist
$E(X)=\frac12$.
\item
Zur Bestimmung der Varianz muss auch $E(X^2)$ bestimmt werden, dies ist
mit der Formel \eqref{50000061:int2} möglich:
\begin{align*}
E(X^2)
&=
\int_{-\infty}^\infty x^2 \varphi(x)\,dx
=
\int_0^1 \frac{x^2\,dx}{\pi x(1-x)}
=
\frac{1}{\pi}
\biggl[
\frac38 \arcsin(2x-1)
\biggr]_0^1
=
\frac{1}{\pi}\cdot
\frac{3\pi}{8}
=
\frac{3}{8}
\\
\operatorname{var}(X)
&=
E(X^2)-E(X)^2
=
\frac{3}{8} -\frac14
=
\frac38-\frac28
=
\frac18.
\qedhere
\end{align*}
\end{teilaufgaben}
\end{loesung}

\begin{bewertung}
Normierungsbedingung ({\bf N}) 1 Punkt,
Konstante $a$ ({\bf A}) 1 Punkt,
Verteilungsfunktion ({\bf F}) 1 Punkt,
Erwartungswert ({\bf E}) 1 Punkt,
Varianzformel ({\bf V}) 1 Punkt,
Erwartungswert $E(X^2)$ und Varianz ({\bf E2}) 1 Punkt.
\end{bewertung}
