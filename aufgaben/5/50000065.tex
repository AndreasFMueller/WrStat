Ein auf Internet-Sicherheit spezialisiertes Unternehmen hat 2019 geschätzt,
dass 45\% aller Anrufe auf Mobiltelefone Werbeanrufe seien.
\begin{teilaufgaben}
\item
Wie gross ist die Wahrscheinlichkeit, dass mehr als die Hälfte von
50 zufällig ausgewählten Anrufen Werbeanrufe sind?
\item
Da es nur elf nigerianische Prinzen gibt%
\footnote{Quelle: \url{https://en.wikipedia.org/wiki/Category:Nigerian_princes}},
sind Anrufe von einem nigerianischen Prinzen, der einem Geld schicken 
möchte, eher selten.
Für die Zwecke dieser Aufgabe nehmen wir gross\-zügig an, dass die
nigerianischen Prinzen mit ihrem Stab und Computerhilfe erreichen können,
dass 1\% aller Anrufe von ihnen stammen.
Wie gross ist die Wahrscheinlichkeit, mehr als zwei solche Anrufe in
der oben genannten Stichprobe von 50 Anrufen zu finden?
\end{teilaufgaben}

\begin{loesung}
\begin{teilaufgaben}
\item
Die Unterschiedung zwischen Werbeanruf und normalem Anruf ist ein
Bernoulli-Experiment.
Die Anzahl $X$ der Werbeanrufe innerhalb der Stichprobe ist binomialverteilt
mit $p=0.45$ und $n=50$.
Die gesuchte Wahrscheinlichkeit ist
\begin{align}
P(X > 25) &= 1 -P(X\le 25)
\\
P(X\le 25)
&=
P\biggl(\frac{X-\mu}{\sigma} \le \frac{25-\mu}{\sigma}\biggr)
\notag
\intertext{wobei $\mu=np=22.5$ und $\sigma=\sqrt{np(1-p)}=3.517812$.
Die Normalapproximation approximiert dies durch}
&=
P\biggl(Z \le \frac{25.5-\mu}{\sigma}\biggr)
=
P(Z\le 0.8528029)
=
0.8031157
\label{50000065:korrektur}
\intertext{für eine standardnormalverteilte Zufallsvariable $Z$.
Die gesuchte Wahrscheinlichkeit ist daher}
P(X>25)
&\cong
0.197.
\notag
\end{align}
Ohne die Korrektur $+0.5$ in \eqref{50000065:korrektur}
ergibt sich $P(0.7107)=0.7613649$ und ein Schlussresultat von
$P(X>25)\cong 0.239$.
\item
Mit einer Wahrscheinlichkeit von 1\% erwartet man $\lambda=0.5$ Anrufe
von nigerianischen Prinzen auf 50 Anrufe.
Mit der Poisson-Verteilung lässt sich die Wahrscheinlichkeit von
mehr als 2 Anrufen abschätzen:
\begin{align*}
P(Y>2)
&=
1-P(Y\le 2)
\\
P(Y\le 2)
&\cong
e^{-\lambda}\sum_{k=0}^3 \frac{\lambda^k}{k!}
=
e^{-\lambda}
\biggl(
1+\lambda+\frac{\lambda^2}{2!}
\biggr)
=
0.98561
\\
P(Y>2)
&\cong
0.014388
=
1.4388\%
\qedhere
\end{align*}
\end{teilaufgaben}
\end{loesung}

\begin{bewertung}
Binomialverteilung ({\bf B}) 1 Punkt,
Berechnung von $\mu$ und $\sigma$ ({\bf M}) 1 Punkt,
Standardisierung ({\bf S}) 1 Punkt,
Normalapproximation und Wahrscheinlichkeit ({\bf N}) 1 Punkt,
Parameter $\lambda$ ({\bf L}) 1 Punkt,
Poisson-Verteilung und Wahrscheinlichkeit ({\bf P}) 1 Punkt.
\end{bewertung}

