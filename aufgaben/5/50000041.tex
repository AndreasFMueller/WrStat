Gibt es den Durchschnittsmenschen? 
Diese Frage hat sich die amerikanische Air Force in den 40er Jahren gestellt
als es darum ging, für die Piloten der neuen und schnelleren Kampfjets
genügend enge Anzüge zu beschaffen, mit denen den erhöhten G-Kräften
begegnet werden konnte.
In einer grossangelegten Untersuchung wurden von Tausenden von
Air Force-Angehörigen über 150 Masse genommen und statistisch ausgewertet.
Es stellte sich heraus, dass es niemanden gab, der in all diesen Massen
dem Mittel entsprach.

Dies kann man am folgenden einfachen Modell verstehen.
Nehmen wir an, dass es $n$ unabhängige Messgrössen $X_i$, $1\le i\le n$  gibt,
die Mittelwerte $E(X) = \mu_i$ und Varianzen
$\sigma_i^2 = \operatorname{var}(X_i)$ haben.

\thema{Normalverteilung}

\begin{teilaufgaben}
\item
Wie gross ist die Wahrscheinlichkeit, dass alle Werte innerhalb einer
Standardabweichung vom Mittelwert liegen?
\item
Wenn man die Toleranz verkleinert, wird die Wahrscheinlichkeit noch kleiner.
Sei $r\in\mathbb R$ und wir fragen nach der Wahrscheinlichkeit, dass alle
Masse innerhalb von $r\sigma_i$ vom jeweiligen Mittelwert $\mu_i$ liegen.
Nehmen Sie an, dass $n=20$.
Wir muss man $r$ wählen, dass die Wahrscheinlichkeit kleiner als $0.1\%$
wird?
\item
Nehmen Sie an, dass $n=18$.
Wie gross ist die erwartete Anzahl von Individuen in einer Stichprobe
vom Umfang $N=4000$, deren Masse alle innerhalb einer Standardabweichung
des Mittelwertes liegen?
\item
Unter den Voraussetzungen von c), wie gross ist die Wahrscheinlichkeit,
in der Stichprobe mehr als 2 Individuen zu finden, deren Masse alle
innerhalb einer Standardabweichung vom Mittelwert liegen?
\end{teilaufgaben}

\begin{loesung}
Wir schreiben $p(r)= P(|X_i-\mu_i|\le r\sigma_i)$.
In den Teilaufgaben a), c) und d) geht es um $p(1)$, in b) muss $r$ bestimmt
werden.
Die Tabelle der Normalverteilung besagt, dass
\begin{align*}
p(r)
&=
P(\mu_i-r\sigma_i < X_i \le \mu_i+r\sigma_i)
\\
&=
\Phi(r) - \Phi(-r)
=
\Phi(r) - (1-\Phi(r))
=
2\Phi(r)-1
\\
p(1)
&=
2\Phi(1)-1
=
2\cdot 0.8413 -1
=
0.6826.
\end{align*}
Natürlich kann man auch $\Phi(r)$ aus $p(r)$ bestimmen:
$\Phi(r)=\frac12(p(r)+1)$.
\begin{teilaufgaben}
\item
Da die Zufallsvariablen $X_i$ unabhängig sind, kann man die Produktregel
verwenden und findet
\begin{align*}
P(|X_i-\mu_i|\le \sigma_i\forall i)
&=
\prod_{i=1}^n P(|X_i-\mu_i|\le \sigma_i)
=
p(1)^n
=
0.6826^n.
\end{align*}
\item
Mit einer Rechnung analog zu der von a) findet man
\[
P(|X_i-\mu_i|\le r\sigma_i) =p(r)^n,
\]
es muss also die Gleichung $p(r)^n = 0.001$ gelöst werden:
\begin{align*}
p(r)^{20} &= 0.001
\\
p(r)&=\sqrt[20]{0.001} = 0.70795
\\
\Phi(r)&=\frac12(0.70795+1)=0.85938
\\
r&=1.070795
\end{align*}
\item
Wie in a) findet man für die Wahrscheinlichkeit den Wert $p(1)^{18}=0.001027$.
Die Anzahl Individuen ist binomialverteilt mit $N=4000$ und $p=0.001027$,
daher ist der erwartete Wert $Np=4.1079$.
\item
Die Wahrscheinlichkeit von c) ist ziemlich klein, wir approximieren daher
die Binomialverteilung der Anzahl $X$ von Individuen, die die Bedingung
erfüllen, mit Hilfe der Poissonverteilung mit $\lambda=Np=4.1079$.
Gefragt ist nach
\begin{align*}
P(X>2)
&=
1- P_\lambda(0)-P_\lambda(1)-P_\lambda(2)
\\
&=
1-e^{-\lambda}\biggl(1+\lambda + \frac{\lambda^2}2\biggr)
=
0.77728.
\qedhere
\end{align*}
\end{teilaufgaben}
\end{loesung}

\begin{diskussion}
Die Untersuchung der Air Force zum ``Durchschnittspiloten'' wird im Detail
erklärte von Youtuber Matt Parker vom Kanal Standupmaths in seinem Buch
{\em Humble Pi}, das 2019 erschienen ist.
\end{diskussion}

\begin{bewertung}
\begin{teilaufgaben}
\item
Wahrscheinlichkeit $p(1)$ ({\bf P}) 1 Punkt,
Wahrscheinlichkeit $p(1)^n$ ({\bf N}) 1 Punkt,
\item
Bestimmung von $r$ ({\bf R}) 1 Punkt,
\item
Binomialverteilung mit $N=4000$ und $p=p(1)^18$ ({\bf B}) 1 Punkt,
Erwartungswert ({\bf E}) 1 Punkt,
\item
Poisson ({\bf P}) 1 Punkt:
\end{teilaufgaben}
\end{bewertung}





