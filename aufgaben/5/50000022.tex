Es wird behauptet, die Zufallsvariable $X$ hätte die
Wahrscheinlichkeitsdichte
\[
\varphi(x)=\begin{cases}
     0&\qquad x < 0,\\
2(1-x)&\qquad 0\le x < 1,\\
     0&\qquad x \ge 1.
\end{cases}
\]
\begin{teilaufgaben}
\item Ist $\varphi$ überhaupt eine Wahrscheinlichkeitsdichte?
\item Bestimmen Sie die Verteilungsfunktion $F_X(x)$.
\item An welcher Stelle erreicht die Verteilungsfunktion den Wert 0.5?
\item Bestimmen Sie den Erwartungswert $E(X)$.
\item Bestimmen Sie die Varianz $\operatorname{var}(X)$.
\end{teilaufgaben}

\begin{loesung}
\begin{teilaufgaben}
\item Damit $\varphi(x)$ eine Wahrscheinlichkeitsdicht ist, muss man 
kontrollieren, ob $\int_{-\infty}^\infty\varphi(x)\,dx=1$ gilt. Das
wird sich in Teilaufgabe $b$ automatisch ergeben.
\item Die Verteilungsfunktion ist das Integral der Dichtefunktion.
Nur der ``mittlere'' Teil für $0\le x\le 1$
der Verteilungsfunktion ist interessant:
\[
F_X(x)=\int_{-\infty}^x\varphi(\xi)\,d\xi =\int_0^x2(1-\xi)\,d\xi=
\left[2(\xi-\frac12\xi^2)\right]_0^x=2x-x^2.
\]
Also ist die Verteilungsfunktion
\[
F_X(x)=\begin{cases}
     0&\qquad x < 0\\
2x-x^2&\qquad 0\le x < 1\\
     1&\qquad x \ge 1
\end{cases}
\]
Damit ist auch klar, dass $\varphi(x)$ tatsächlich ein Dichtefunktion ist.
\item $F_X(x)=0.5$ impliziert
\begin{align*}
2x-x^2&=\frac12\\
x^2-2x+\frac12&=0\\
x_\pm=1\pm\sqrt{1-\frac12}=1\pm\sqrt{\frac12}
\end{align*}
Nur der Wert für das negative Vorzeichen ist im Interval $[0,1]$, das gesuchte
$x$ ist also
\[
x=1-\sqrt{\frac12}=0.29289322.
\]
\item
Der Erwartungswert ist
\begin{align*}
E(X)&=\int_{-\infty}^{\infty}x\varphi(x)\,dx
=\int_0^12x-2x^2\,dx=\left[ x^2-\frac23x^3\right]_0^1=1-\frac23=\frac13.
\end{align*}
\item Die Varianz kann mit der Formel
$\operatorname{var}(X)=E(X^2)-E(X)^2$ bestimmt werden:
\begin{align*}
E(X^2)&=\int_{-\infty}^{\infty}x^2\varphi(x)\,dx
\int_0^1 2x^2-2x^3\,dx
=\left[\frac23x^3-\frac12x^4\right]_0^1
=\frac23-\frac12=\frac16,
\\
\operatorname{var}(X)&=E(X^2)-E(X)^2=\frac16 - \frac19=\frac1{18}=0.055555.
\qedhere
\end{align*}
\end{teilaufgaben}
\end{loesung}

\begin{bewertung}
\begin{teilaufgaben}
\item Normalisierungsbedingung ({\bf N}) 1 Punkt.
\item Verteilungsfunktion ({\bf V}) 1 Punkt.
\item Median ({\bf M}) 1 Punkt.
\item Erwartungswert ({\bf E}) 1 Punkt.
\item Varianzformel ({\bf F}) 1 Punkt, Berechnung der Varianz ({\bf R}) 1 Punkt.
\end{teilaufgaben}
\end{bewertung}

