Im Jahr 2019 wurde in der Zeitschrift Nature eine Studie publiziert,
die 1.2 Millionen Fussabdrücke analysiert hat.
Ziel war, Grundlagen für die Schuhgrössenraster der Schuhindustrie zu
schaffen.
Zum Beispiel ergab sich bei der Untersuchung der Füsse der
270\,mm-Längenklasse, dass 5\% schmaler als 94.4\,mm sind und 5\% breiter
als 110.2\,mm.
Wie wahrscheinlich ist, dass ein Fuss in dieser Klasse schmaler als
100\,mm ist?

\begin{loesung}
Die Breite eines Fusses ist eine normalverteilte Zufallsvariable $X$
mit Erwartungswert $\mu$ und Standardabweichung $\sigma$.
Aus den Informationen in der Aufgabenstellung
\begin{align*}
P(X\le 94.4)
&= 0.05
&
P(X\le 110.2)
&=
0.95.
\intertext{Durch Standardisierung wird daraus}
P\biggl(
\frac{X-\mu}{\sigma} \le \frac{94.4-\mu}{\sigma}
\biggr)
&=
0.05
&
P\biggl(
\frac{X-\mu}{\sigma} \le \frac{110.2-\mu}{\sigma}
\biggr)
&=
0.95.
\intertext{Da $Z=(X-\mu)/\sigma$ eine standardnormalverteilte
Zufallsvariable ist, folgt}
\frac{94.4-\mu}{\sigma}
&=
-1.6449
&
\frac{110.2-\mu}{\sigma}
&=
1.6449.
\end{align*}
Daraus ergibt sich das lineare Gleichungssystem
\begin{equation*}
\renewcommand{\arraycolsep}{3pt}
\begin{array}{rcrcr}
\mu &-& 1.6449\sigma &=&  94.4\phantom{.}\\
\mu &+& 1.6449\sigma &=& 110.2.
\end{array}
\end{equation*}
Die Differenz der beiden Gleichungen ist
\[
-2\cdot 1.6449\sigma = -15.8
\qquad\Rightarrow\qquad
\sigma=4.803.
\]
Durch Einsetzen in die erste Gleichung kann man auch
\[
\mu
=
94.4 + 1.6449\sigma
=
102.3
\]
bestimmen.
Dasselbe Resultat kann man natürlich auch aus der Summe der beiden
Gleichungen erhalten.
Die gesuchte Wahrscheinlichkeit wird damit
\[
P(X\le 100)
=
P\biggl(
\frac{X-\mu}{\sigma}
\le
\frac{100-\mu}{\sigma}
\biggr)
=
\Phi\biggl(
\frac{100-\mu}{\sigma}
\biggr)
=
\Phi(
-0.4789
)
=
0.316.
\qedhere
\]
\end{loesung}

\begin{bewertung}
Normalverteilung ({\bf N}) 1 Punkt,
Standardisierung ({\bf S}) 1 Punkt,
Gleichungssystem für $\mu$ und $\sigma$ ({\bf G}) 1 Punkt,
Bestimmung von $\mu$ ({\bf M}) und $\sigma$ ({\bf D}) je 1 Punkt,
Berechnung der Wahrscheinlichkeit ({\bf W}) 1 Punkt.
\end{bewertung}

