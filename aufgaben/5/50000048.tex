Bei der Produktion von 1k$\Omega$-Widerständen werden in der
Qualitätskontrolle alle Widerstände, die um mehr als 5\% vom Sollwert
abweichen, aussortiert.
Es stellt sich heraus, dass 5\% einen zu grossen Wert haben und
1\% einen zu kleinen.
\begin{teilaufgaben}
\item
Bestimmen sie den Erwartungswert und die Varianz der produzierten
Widerstände.
\item
Wie gross ist die Wahrscheinlichkeit, dass ein Widerstand um weniger
als $1\%$ vom Sollwert abweicht.
\end{teilaufgaben}

\thema{Normalverteilung}
\thema{Standardisierung}

\begin{loesung}
\begin{teilaufgaben}
\item
Der Wert $X$ eines Widerstandes ist eine normalverteilte Zufallsvariable.
Wir können die Wahrscheinlichkeit dafür, dass $X$ ausserhalb des
Toleranzintervalls liegt, mit Hilfe von Standardisierung berechnen:
\begin{align*}
P(X<0.95)
&=
P\biggl(\frac{X-\mu}{\sigma} < \frac{0.95-\mu}{\sigma}\biggr)
=
\phantom{1-\mathstrut}
\Phi\biggl(\frac{0.95-\mu}{\sigma}\biggr)
=
0.01
&&\Rightarrow&
\frac{0.95-\mu}{\sigma}
&=
-2.3263,
\\
P(X>1.05)
&=
P\biggl(\frac{X-\mu}{\sigma} > \frac{1.05-\mu}{\sigma}\biggr)
=
1-\Phi\biggl(\frac{1.05-\mu}{\sigma}\biggr)
=0.05
&&\Rightarrow&
\frac{1.05-\mu}{\sigma}
&=
\phantom{-}1.6449.
\end{align*}
Dies führt auf das lineare Gleichungssystem 
\[
\left.
\begin{aligned}
\mu-2.3263\sigma&=0.95\\
\mu+1.6449\sigma&=1.05
\end{aligned}
\quad
\right\}
\qquad\text{mit der Lösung}\qquad
\left\{
\quad
\begin{aligned}
\mu&=1.008579\\
\sigma&=0.025181.
\end{aligned}
\right.
\]
\item
Die Wahrscheinlichkeit für einen 1\% Widerstand ist
\begin{align*}
P(0.99<X<1.01)
&=
P\biggl(
\frac{0.99-\mu}{\sigma} < \frac{X-\mu}{\sigma} < \frac{1.01-\mu}{\sigma}
\biggr)
\\
&=
P(-0.7378 < X < 0.0564)
\\
&=
\Phi(0.0564) - (1-\Phi(0.7378))
=
\Phi(0.0564) + \Phi(0.7378) - 1
\\
&=
0.5225 + 0.7697 - 1
=
0.2922.
\end{align*}
Fast ein Drittel der produzierten Widerstände könnte also auch als
1\%-Widerstände verkauft werden.
\qedhere
\end{teilaufgaben}
\end{loesung}

\begin{bewertung}
Standardisierung ({\bf S}) 1 Punkt,
Gleichungssystem ({\bf G}) 1 Punkt,
Bestimmung von $\mu$ ({\bf M}) 1 Punkt,
Bestimmung von $\sigma$ ({\bf V}) 1 Punkt,
Wahrscheinlichkeit für 1\%-Widerstand ({\bf W}) 2 Punkt,
\end{bewertung}




