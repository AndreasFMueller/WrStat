In einer Umfrage wurde festgestellt, dass 18\% der 18--34-jährigen
der Meinung sind, die Mondlandung sei eine Verschwörung.
Im Vergleich dazu sind Flat Earther relativ selten, nur 1.28\%
der Menschen derselben Altersgruppe sind davon überzeugt, dass
die Erde flach ist.

\begin{teilaufgaben}
\item
Wie gross ist die Wahrscheinlichkeit sich in einer Stichprobe von
100 Menschen dieser Altersgruppe mehr als 20 finden lassen, die die
Mondlandung für eine Verschwörung halten?
\item
Wie gross ist die Wahrscheinlichkeit, dass in der gleichen Stichprobe
mehr als zwei Personen davon überzeugt sind, dass die Erde flach ist?
\end{teilaufgaben}

\thema{Binomialverteilung}
\thema{Normalapproximation}

\begin{loesung}
\begin{teilaufgaben}
\item
Sei $X$ die Anzahl der Mondlandungsverweigerer in der Stichprobe.
$X$ ist binomialverteilt mit $p=0.18$ und $n=100$.
Gesucht ist die Wahrscheinlichkeit $P(X>20)$.
Wir approximieren die Wahrscheinlichkeit mit Hilfe der Normalverteilung.
Der Erwartungswert der Binomialverteilung ist $\mu=np=18$ und die
Standardabweichung ist $\sigma=\sqrt{np(1-p)}=3.84187$.
Die gesuchte Wahrscheinlichkeit ist
\begin{align*}
P(X>20)
&=
1-P(X\le 20)
\\
P(X\le 20)
&=
P\biggl( \frac{X-\mu}{\sigma} \le \frac{20.5-\mu}{\sigma}\biggr)
=
\Phi\biggl(\frac{20.5-\mu}{\sigma}\biggr)
=
\Phi(0.6507)
=
0.7424
\\
P(X>20)&\simeq 1-0.7424 = 0.2576.
\end{align*}
Die Wahrscheinlichkeit, mehr als 20 Mondlandungsverschwörer in der
Stichprobe zu finden ist also $25.76\%$.
\item
Da Flat Earther so selten sind, approximieren wir die Binomialverteilung
mit der Poisson-Verteilung.
Die erwartete Anzahl Flat Earther ist $\lambda = np = 100\cdot 0.0128=1.28$.
Die Wahrscheinlichkeit, mehr als 2 Flat Earther zu finden ist daher
\begin{align*}
P(X > 2)&=1-P(X\le 2) = 1-P(X=0) - P(X=1) - P(X=2)
\\
&=
1-P\lambda(0) -P_\lambda(1) -P_\lambda(2)
\\
&=
1-e^{-\lambda}\biggl(1 + \lambda + \frac{\lambda^2}{2}\biggr)
\\
&=
1- 0.86169 = 0.138307,
\end{align*}
die Wahrscheinlichkeit, mehr als zwei Flat Earther zu finden, ist also
13.8\%.
\qedhere
\end{teilaufgaben}
\end{loesung}

\begin{bewertung}
Binomalverteilung ({\bf B}) 1 Punkt,
Formel für Erwartungswert und Varianz ({\bf F}) 1 Punkt.
\begin{teilaufgaben}
\item
Normalapproximation ({\bf N}) 1 Punkt,
Wahrscheinlichkeit ({\bf W}) 1 Punkt,
\item
Approximation mit der Poisson-Verteilung ({\bf P}) 1 Punkt,
Wahrscheinlichkeit ({\bf W}) 1 Punkt.
\end{teilaufgaben}
\end{bewertung}


