Die Wahrscheinlichkeitsdichte (Dichtefunktion) einer Zufallsvariablen $X$
sei
\[
\varphi(x)=\begin{cases}
0&\qquad x<-3\\
\frac16x+\frac12&\qquad -3\le x<0\\
-\frac12x+\frac12&\qquad 0\le x< 1\\
0&\qquad 1\le x
\end{cases}
\]
\begin{teilaufgaben}
\item Zeichnen Sie den Graphen von $\varphi$.
\item Bestimmen Sie die Verteilungsfunktion von $X$ und zeichnen Sie
auch davon einen Graphen.
\item Berechnen Sie den Erwartungswert von $X$.
\end{teilaufgaben}

\thema{Wahrscheinlichkeitsdichte}
\thema{Erwartungswert}
\thema{Verteilungsfunktion}

\begin{loesung}
\begin{teilaufgaben}
\item
Der Graph diese Wahrscheinlichkeitsverteilung ist
\begin{center}
\includeagraphics[width=0.9\hsize]{graph-1}
\end{center}
\item Die Verteilungsfunktion wird bestimmt durch Integration:
\[
F(x)=\int_{-\infty}^x\varphi(\xi)\,d\xi.
\]
Offenbar müssen wir das Integral auf die verschiedenen Teilintervalle
aufteilen. Im Interval $(-\infty,-1]$ ist sicher $F(x)=0$. Für $x$ zwischen
$-3$ und $0$ gilt
\begin{align*}
F(x)&=\int_{-\infty}^x\varphi(\xi)\,d\xi =\int_{-3}^x\frac16\xi+\frac12\,d\xi
\\
&=
\left[\frac1{12}\xi^2+\frac12\xi\right]_{-3}^x=\frac{x^2-9}{12}+\frac{x+3}2
\\
&
=\frac{x^2+6x+9}{12}=\frac{(x+3)^2}{12}
\end{align*}
An der Stelle $x=0$ nimmt diese Funktion den Wert $\frac34$ an.
Der Graph dieses Teils ist eine Parabel mit Scheitelpunkt $(-3,0)$.

Für den Abschnitt zwischen $0$ und $1$ finden wir
\begin{align*}
F(x)
&=
\int_{-\infty}^x\varphi(\xi)\,d\xi
=F(0)+\int_0^x\varphi(\xi)\,d\xi
=\frac34+\int_0^x-\frac12\xi+\frac12\,d\xi\\
&=\frac34+\left[-\frac14\xi^2+\frac12\xi\right]_0^x
=\frac34-\frac14x^2+\frac12x
=\frac{-x^2+2x+3}4\\
&=1-\frac{x^2-2x+1}{4}
=1-\frac{(x-1)^2}4
\end{align*}
Dieser Teil des Graphen ist eine nach unten geöffnete Parabel mit
Scheitelpunkt $(1,1)$.
Der Graph diese Verteilungsfunktion ist
\begin{center}
\includeagraphics[width=0.9\hsize]{graph-2}
\end{center}
\item
Der Erwartungswert ist das Integral
\begin{align*}
E(X)&=\int_{-\infty}^\infty\xi\varphi(\xi)\,d\xi
=
\int_{-3}^0\xi\biggl(\frac16\xi+\frac12\biggr)\,d\xi
+\int_0^1\xi\biggl(-\frac12\xi+\frac12\biggr)\,d\xi
\\
&=
\int_{-3}^0\frac16\xi^2+\frac12\xi\,d\xi
+\int_0^1-\frac12\xi^2+\frac12\xi\,d\xi
\\
&=\left[\frac1{18}\xi^3+\frac14\xi^2\right]_{-3}^0
+\left[-\frac16\xi^3+\frac14\xi^2\right]_0^1\\
&=\frac{27}{18}-\frac94-\frac16+\frac14
=\frac{3}{2}-\frac94-\frac16+\frac14
=\frac{18-27-2+3}{12}
=-\frac{8}{12}
\\
&=-\frac23.
\qedhere
\end{align*}
\end{teilaufgaben}
\end{loesung}
