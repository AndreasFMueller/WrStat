In einem nicht gerade sorgfältig konstruierten Gerät gibt es eine
Komponente, die sehr häufig kaputt geht und ersetzt werden muss.
Die mittlere Lebensdauer ist eine Woche.
Nach einem Jahr (52 Wochen) wird man also diese Komponenten etwa 52
mal ausgewechselt haben, was natürlich mit hohen Kosten verbunden
ist.
Es können aber auch mehr oder weniger Auswechselaktionen nötig werden.
Für Budgetzwecke wird angenommen, dass man mit 10 zusätzlichen
Austauschaktionen pro Jahr durchkommt.
\def\teilaufgabea{Verwenden Sie eine geeignete, einfache Approximation
um die Wahrscheinlichkeit zu berechnen, dass das Budget nicht reicht.}
\ifthenelse{\boolean{pruefung}}{%
\teilaufgabea
}{%
\begin{teilaufgaben}
\item
\teilaufgabea
\item
Wie gross muss die Anzahl der zusätzlichen Austauschaktionen sein,
damit das Budget mit Wahrscheinlichkeit 99\% reicht.
(Diese Teilaufgabe war nicht Teil der Prüfung.)
\end{teilaufgaben}
}

\begin{hinweis}
Verwenden Sie eine Woche als Masseinheit für die Zeit, damit die Formeln
nicht zu kompliziert werden.
\end{hinweis}

\begin{loesung}
\begin{teilaufgaben}
\item
Jede der Komponenten hat eine exponentialverteilte Lebensdauer
$T_1,\dots,T_n$ mit $a=1\,[\text{Woche}^{-1}]$.
Gefragt ist die Wahrscheinlichkeit, dass die gesamte Lebensdauer der $n$
Komponenten kleiner ist als $52$.
Wir schreiben daher $X=T_1+\dots+T_n$ und versuchen, $P(X<52)$ zu berechnen.

Für eine gute Näherung können wir die Eigenschaft verwenden, dass die
Summe vieler kleiner Einflüsse auf eine Normalverteilung führt.
Die Summe $X$ ist also approximativ normalverteilt mit Erwartungswert
$\mu=E(X)=n E(T) = n\cdot 1\,[\text{Woche}]$
und Varianz
$\sigma^2=\operatorname{var}(X) = n\operatorname{var}(T)
=
n\cdot 1\,[\text{Woche}^2]$.
Damit können wir die Wahrscheinlichkeit durch Standardisierung ermitteln:
\begin{align*}
P(X< 52)
=
P(X\le 52)
&=
P\biggl(\frac{X-\mu}{\sigma}\le \frac{52-\mu}{\sigma}\biggr)
=
P\biggl(Z\le \frac{52-n}{\sqrt{n}}\biggr)
\end{align*}
Da $52-n<0$ finden wir den zugehörigen Wert nicht in der Tabelle der
Verteilungsfunktion der Normalverteilung, wir gehen daher über zum 
Komplement:
\begin{align*}
P\biggl(Z\le \frac{52-n}{\sqrt{n}}\biggr)
=
1-
P\biggl(Z\le \frac{n-52}{\sqrt{n}}\biggr)
\end{align*}
Für $n=62$ ist $P(Z\le 10/\sqrt{62})=P(Z\le 1.27) \approx 0.898$
und damit ist die Wahrscheinlichkeit $P(X<52)\approx 0.112$.

Wir können können aber auch eine exakte Antwort finden.
Da es um die Anzahl $K$ der Ereignisse geht, die eintreten, ist die
Poisson-Verteilung die richtige Wahl.
Der Erwartungswert für die Anzahl der Austauschaktionen in einem
Jahr ist $\lambda=52$.
$P_\lambda(k)$ gibt dann die Wahrscheinlichkeit für genau $k$
Austauschaktionen in einem Jahr an.
Die Wahrscheinlichkeit für höchstens $n$ Austauschaktionen ist
\[
P(K\le n)
=
\sum_{k=0}^n  P_{\lambda}(k).
\]
Die Wahrscheinlichkeit dafür, dass es mehr als $n$ Austauschaktionen 
braucht, ist daher
\[
P(K>n)
=
1-P(K\le n)
=
1-\sum_{k=0}^n P_{\lambda}(n).
\]
für $n=62$ finden wir
\[
P(K>62)
=
1-\sum_{k=0}^n P_{\lambda}(n)
=
1-0.924105
=
0.075895.
\]
\item
In diesem Fall muss man $n$ so gross wählen, dass
\[
\sum_{k=0}^nP_\lambda(k) >0.99.
\]
Mit etwas probieren findet man, dass $n\ge 69$ sein muss.

Alternativ kann man wieder die Approximation verwenden.
Es muss dann das $n$ gefunden werden, für das $P(Z\le (n-52)/\sqrt{n})=0.99$
ist.
Dies tritt ein für
\begin{align*}
\frac{n-52}{\sqrt{n}}&=2.3263\\
n-2.3263\sqrt{n}-52&=0
\end{align*}
Dies ist eine quadratische Gleichung mit der Unbekannten $\sqrt{n}$, sie hat
die Lösung
\begin{align*}
\sqrt{n}
&=
\frac{2.3263}{2}\pm\sqrt{\biggl(\frac{2.3263}{2}\biggr)^2+52}
=
1.16315 + \sqrt{52+1.16315^2}
=
8.467
\\
n&=71.6978.
\end{align*}
Um sicher zu sein, dass das Budget in 99\% der Fälle reicht, muss man
also mindestens $72-52=20$ zusätzlichen Austauschaktionen budgetieren.
\qedhere
\end{teilaufgaben}
\end{loesung}

\begin{bewertung}
Exakte Lösung: Poisson-Verteilung ({\bf P}) 1 Punkt,
Bestimmung von $\lambda$ ({\bf L}) 1 Punkt,
Poisson-Terme ({\bf T}) 1 Punkt,
Summenformel ({\bf S}) 1 Punkt,
Komplement ({\bf K}) 1 Punkt,
Zahlenwert für die Wahrscheinlichkeit ({\bf W}) 1 Punkt.

Approximation: Wahl der Exponentialverteilung ({\bf X}) 1 Punkt,
Verwendung der Approximation mit der Normalverteilung ({\bf N}) 1 Punkt,
Erwartungswert von $X$ ({\bf E}) 1 Punkt,
Varianz von $X$ ({\bf V}) 1 Punkt,
Standardisierung ({\bf S}) 1 Punkt,
Komplement ({\bf K}) 1 Punkt,
Berechnung der Wahrscheinlichkeit ({\bf W}) 1 Punkt.
\end{bewertung}



