Im Mandarinenexperiment im HS 2022 wurden 94 Mandarinen ausgemessen.
\begin{teilaufgaben}
\item
Es stellte sich heraus, dass 17\% der Mandarinen einen Durchmesser
grösser als 60\,mm hatten, während 63.8\% einen Durchmesser 
grösser als 55\,mm hatten.
Bestimmen Sie den Mittelwert $\mu$ und die Varianz $\sigma^2$
des Mandarinendurchmessers.
\item
Ausgehend von $\mu$ und $\sigma$ aus Teilaufgabe b),
wie gross ist die Varianz des Mittelwertes $M_n = (X_1+\dots+X_n)/n$
von $n=94$ Messungen $X_1,\dots,X_n$ des Mandarinendurchmessers?
\ifthenelse{\boolean{pruefung}}{}{
\item
Im Mandarinenexperiment wurde als Mittelwert von $n=94$ Mandarinen
der Wert $56.65\,\text{mm}$ bestimmt.
Wie gross ist die Wahrscheinlichkeit, eine Abweichung des Mittelwertes
von dieser Grössenordnung zu finden?
(Diese Teilaufgabe war nicht Teil der Prüfung.)
}
\end{teilaufgaben}

\begin{loesung}
\begin{teilaufgaben}
\item
Es darf angenommen werden dass die Mandarinendurchmesser $X$ eine
normalverteilte Zufallsvariable ist sind mit Erwartungswert $\mu$
und Varianz $\sigma^2$.
Die Wahrscheinlichkeit, dass der Mandarinendurchmesser $\le x$ ist,
ist
\[
P(X\le x)
=
P\biggl(
\underbrace{\frac{X-\mu}{\sigma}}_{\displaystyle=Z}
\le
\frac{x-\mu}{\sigma}
\biggr)
=
\Phi\biggl(
\frac{x-\mu}{\sigma}
\biggr).
\]
Da $Z$ standardnormalverteilt ist, kann die rechte Seite mit der
Verteilungsfunktion $\Phi$ der Standardnormalverteilung berechnet werden.
Indem man die Funktion $\Phi$ umkehrt, kann man nach $(x-\mu)/\sigma$
auflösen:
\[
p
=
\Phi\biggl(\frac{x-\mu}{\sigma}\biggr)
\qquad\Rightarrow\qquad
\frac{x-\mu}{\sigma}
=
\Phi^{-1}(p)
\qquad\Rightarrow\qquad
\mu + \Phi^{-1}(p)\sigma = x.
\]
Dies ist eine lineare Gleichung für $\mu$ und $\sigma$.
Für die Datenpunkte in der Aufgabe ist
\begin{align*}
P(X > 60) &= 0.17  &&\Rightarrow& P(X\le 60) &= 0.83 \\
          &        &&\Rightarrow& \Phi^{-1}(0.83) &= 0.954
&&\Rightarrow& \mu+0.954 \sigma &= 60\\[4pt]
P(X > 55) &= 0.638 &&\Rightarrow& P(X\le 55) &= 0.362 \\
          &        &&\Rightarrow& \Phi^{-1}(0.362) &= -0.353
&&\Rightarrow& \mu-0.353 \sigma &= 55
\end{align*}
Dies ist ein lineares Gleichungssystem für die beiden Unbekannten $\mu$
und $\sigma$, es hat die Lösungen
\[
\left.\begin{aligned}
\mu+0.954 \sigma &= 60\\
\mu-0.353 \sigma &= 55
\end{aligned}\;\right\}
\qquad\Rightarrow\qquad
\left\{\;\begin{aligned}
\sigma&= \frac{60-55}{0.954+0.353} = 3.826\\
\mu   &= 55+0.353\cdot 3.811 = 56.35
\end{aligned}\right.
\]
\item
Die Varianz ist
\[
\operatorname{var}(M_n)
=
\frac{\operatorname{var}(X)}{n}
=
\frac{\sigma^2}{94}
=
\frac{3.826^2}{94}
=
0.1557,
\qquad
\sigma_1
=
\sqrt{\operatorname{var}(M_n)} = 0.3946.
\]
\item
Gesucht ist die Wahrscheinlichikeit dafür, dass die normalverteilte 
Zufallsvariable $M_n$ mit Varianz $\sigma_1$ um mehr als
$|56.35-56.65|=0.30$ vom Erwartungswert $\mu$ abweicht.
Sie ist
\begin{align*}
P(|M_n-\mu|>0.30)
&=
2 \cdot P((M_n-\mu) >0.30)
\\
&=
2(1-P(M_n\le 0.30))
\\
&=
2\biggl(1-P\biggl(\frac{M_n-\mu}{\sigma_1}\le \frac{0.30}{\sigma}\biggr)\biggr)
\\
&=
2\biggl(1-\Phi\biggl(\frac{0.30}{\sigma_1}\biggr)\biggr)
\\
&=
2(1-\Phi(0.7377))
\\
&=
2(1-0.7697)
\\
&=
0.4606.
\qedhere
\end{align*}
\end{teilaufgaben}
\end{loesung}

\begin{bewertung}
Normalverteilung ({\bf N}) 1 Punkt,
Standardisierung ({\bf S}) 1 Punkt,
Gleichungen für $\mu$ und $\sigma$ ({\bf G}) 1 Punkt,
Werte für $\Phi^{-1}(x)$ ({\bf P}) 1 Punkt,
Werte von $\mu$ und $\sigma$ ({\bf L}) 1 Punkt,
Varianz von $M_n$ in Teilaufgabe b) ({\bf B}) 1 Punkt.
\end{bewertung}
