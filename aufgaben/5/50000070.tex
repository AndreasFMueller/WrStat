Zeichnen Sie die Verteilungsfunktion für die Zufallsvariable
\[
X
=
\text{Augenzahl eines Würfels}
\]

\begin{loesung}
Die Verteilungsfunktion springt um $\frac16$ bei den möglichen
Augenzahlen:
\begin{center}
\def\r{0.08}
\begin{tikzpicture}[>=latex,thick]
\draw[->] (0,-0.1) -- (0,4) coordinate[label={$F(x)$}];
\draw[->] (-1,0) -- (7.5,0) coordinate[label={$x$}];
\foreach \x in {1,...,6}{
	\node at (\x,0) [below] {$\x$};
	%\draw[line width=0.2pt] (\x,0) -- ++(0,{0.5*\x});
}
\foreach \x in {2,...,6}{
	\draw (\x,-0.05) -- ++(0,0.1);
}
\draw[line width=0.2pt] (0,3) -- (6,3);
\draw (-0.05,3) -- ++(0.1,0);
\node at (-0.05,3) [left] {$1$};
\draw[color=darkgreen,line width=1.4pt] (-1,0) -- ({1-\r},0);
\draw[color=darkgreen,line width=1.4pt] (1,0.5) -- ++({1-\r},0);
\draw[color=darkgreen,line width=1.4pt] (2,1.0) -- ++({1-\r},0);
\draw[color=darkgreen,line width=1.4pt] (3,1.5) -- ++({1-\r},0);
\draw[color=darkgreen,line width=1.4pt] (4,2.0) -- ++({1-\r},0);
\draw[color=darkgreen,line width=1.4pt] (5,2.5) -- ++({1-\r},0);
\draw[color=darkgreen,line width=1.4pt] (6,3) -- (7,3);
\fill[color=darkgreen] (1,0.5) circle[radius=0.08];
\fill[color=darkgreen] (2,1.0) circle[radius=0.08];
\fill[color=darkgreen] (3,1.5) circle[radius=0.08];
\fill[color=darkgreen] (4,2.0) circle[radius=0.08];
\fill[color=darkgreen] (5,2.5) circle[radius=0.08];
\fill[color=darkgreen] (6,3.0) circle[radius=0.08];
\draw[color=darkgreen,line width=1.4pt] (1,{0.0+\r-0.02}) arc(90:270:{\r-0.02});
\draw[color=darkgreen,line width=1.4pt] (2,{0.5+\r-0.02}) arc(90:270:{\r-0.02});
\draw[color=darkgreen,line width=1.4pt] (3,{1.0+\r-0.02}) arc(90:270:{\r-0.02});
\draw[color=darkgreen,line width=1.4pt] (4,{1.5+\r-0.02}) arc(90:270:{\r-0.02});
\draw[color=darkgreen,line width=1.4pt] (5,{2.0+\r-0.02}) arc(90:270:{\r-0.02});
\draw[color=darkgreen,line width=1.4pt] (6,{2.5+\r-0.02}) arc(90:270:{\r-0.02});
\end{tikzpicture}
\end{center}
\end{loesung}

