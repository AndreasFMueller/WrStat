Das radioaktive Kohlenstoffisotop $\prescript{14}{}C$ wird in der
oberen Atmosphäre laufend neu gebildet und kommt daher mit nahezu
konstanter Konzentration in der Atmosphäre vor, obwohl er mit einer
Halbwertszeit von $5700\pm 30$ Jahren.
Die Zeit bis zum Zerfall eines einzelnen Atomkerns von $\prescript{14}{}C$ 
ist exponentialverteilt. 
Bestimmen Sie den Parameter $a$ dieser Exponentialverteilung in
Anzahl Zerfällen pro Sekunde.

\begin{loesung}
Für die Berechnung von $a$ wird die Halbwertszeit in Sekunden benötigt,
sie ist
\[
t_{\frac12}
=
5700 \cdot 365 \cdot 24 \cdot 60
=
179755200000\,\text{s}.
\]
Damit wird
\[
a
=
\frac{\log 2}{t_{\frac12}}
=
3.856 \cdot 10^{-12}\,\text{s}^{-1}
\qedhere
\]
\end{loesung}


