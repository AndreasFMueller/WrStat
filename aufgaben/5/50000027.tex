Eine Zufallsvariable $X$ ist auf dem Intervall $[0,1]$ so verteilt, dass die
Wahrscheinlichkeitsdichte
\[
\varphi_X(x)=\begin{cases}
0&\qquad x < 0\\
ax^{18}&\qquad 0\le x \le 1\\
0&\qquad x > 1\\
\end{cases}
\]
ist.
\begin{teilaufgaben}
\item Wie gross ist $a$?
\item Welchen Wert hat der Median von $X$?
\item Bestimmen Sie $E(X)$ und $\operatorname{var}(X)$.
\end{teilaufgaben}

\begin{loesung}
\begin{teilaufgaben}
\item
Zun"achst muss $a$ so gew"ahlt werden, dass tats"achlich eine
Wahrscheinlichkeitsdichte vorliegt. Dazu muss gelten
\begin{align*}
1
&=
\int_{-\infty}^{\infty} \varphi(x)\,dx
=
\int_0^1 ax^{18}\,dx
=
\left[
\frac{a}{19}x^{19}
\right]_0^1
=\frac{a}{19}
\\
\Rightarrow \qquad a&= 19.
\end{align*}
\item
Zur Bestimmung des Medians m"ussen wir dasjenige $x$ finden, f"ur das 
$F(x)=0.5$ ist. Dazu berechnen wir
\begin{align*}
\frac12
&=
F(x)
=
\int_{-\infty}^x \varphi(\xi)\,d\xi
=
\int_{-\infty}^x 19\xi^{18}\,d\xi
=
\left[
\frac{19}{19}\xi^{19}
\right]_0^x
=x^{19}
\\
\frac12&=x^{19}\\
x&=0.96418
\end{align*}
\item
F"ur den Erwartungswert und die Varianz berechnen wir
\begin{align*}
E(X)
&=
\int_{-\infty}^\infty x\varphi(x)\,dx
=
\int_0^1x\cdot 19x^{18}\,dx
=
\left[
\frac{19}{20}x^{20}
\right]_0^1
=\frac{19}{20}=0.95
\\
E(X^2)
&=
\int_{-\infty}^\infty x^2\varphi(x)\,dx
=
\int_0^1x^2\cdot 19x^{18}\,dx
=
\left[
\frac{19}{21}x^{21}
\right]_0^1
=\frac{19}{21}
\simeq 0.904761
\\
\operatorname{var}(X)
&=
E(X^2)-E(X)^2
=
\frac{19}{21}-\biggl(\frac{19}{20}\biggr)^2
=
\frac{19}{8400}=0.0022619.
\qedhere
\end{align*}
\end{teilaufgaben}
\end{loesung}

\begin{bewertung}
Bestimmung von $a$ ({\bf A}) 1 Punkt,
L"osungsweg zur Berechnung des Medians ({\bf D}) 1 Punkt,
Medianwert ({\bf M}) 1 Punkt,
Werte f"ur $E(X)$ ({\bf E}),
$E(X^2)$  ({\bf 2}) und $\operatorname{var}(X)$ ({\bf V}) je ein Punkt.
\end{bewertung}

