Eine Zufallsvariable $X$ ist auf dem Interval $[0,\pi]$ so verteilt,
dass die Wahrscheinlichkeitsdichte
\[
\varphi_X(x)
=
\begin{cases}
0&\qquad x < 0\\
a(1-\sin x)&\qquad 0\le x \le \pi\\
0&\qquad x > \pi
\end{cases}
\]
ist wie in dem folgenden Graphen:
\begin{center}
\includeagraphics[]{graph-1.pdf}
\end{center}
\begin{teilaufgaben}
\item
Wie muss man $a$ w"ahlen, damit diese Funktion tats"achlich 
eine Wahrscheinlichkeitsdichte ist?
\item
Berechnen Sie $E(X)$ und $\operatorname{var}(X)$.
\end{teilaufgaben}

\begin{loesung}
\begin{teilaufgaben}
\item Die Konstante $a$ muss so gew"ahlt werden, dass das Integral von
$\varphi_X$ "uber $\mathbb R$ den Wert 1 hat:
\[
\int_{-\infty}^{\infty}\varphi_X(x)\,dx
=
\int_0^\pi a(1-\sin x)\,dx
=
a[x+\cos x]_0^\pi
=
a(\pi - 1 - 0 - 1) = a(\pi-2)
\]
es folgt
\[
a=\frac1{\pi -2}.
\]
\item
F"ur den Erwartungswert $E(X)$ m"ussen wir $x\varphi_X(x)$ integrieren:
\begin{align*}
E(X)
&=
\int_{-\infty}^\infty x\varphi_X(x)\,dx
=
\frac1{\pi-2}\int_0^\pi x(1-\sin x)\,dx
=
\frac1{\pi - 2} \biggl[\frac{x^2}{2}-\sin x + x\cos x\biggr]_0^\pi
\\
&=
\frac1{\pi - 2}\biggl(\frac{\pi^2}{2} +\pi\cos\pi\biggr)
=
\frac{\pi^2-2\pi}{2(\pi-2)}=\frac{\pi}{2} = 1.57079,
\end{align*}
dies ist nat"urlich nicht "uberraschend, da die Verteilung um diesen
Punkt symmetrisch ist.

F"ur die Varianz brauchen wir $E(X^2)$:
\begin{align*}
E(X^2)
&=
\int_{-\infty}^\infty x^2\varphi_X(x)\,dx
=
\frac1{\pi-2}\int_0^\pi x^2(1-\sin x)\,dx
=
\frac1{\pi-2}\biggl[
\frac{x^3}{3}-2x\sin x+(x^2-2)\cos x
\biggr]_0^\pi
\\
&=
\frac{\pi^3-3\pi^2+12}{3(\pi-2)}
\\
\operatorname{var}(X)
&=
E(X^2)-E(X)^2
=
\frac{\pi^3-3\pi^2+12}{3\pi - 6}
-
\frac{\pi^2}{4}
=
\frac {\pi^3-6\pi^2+48}{12\pi-24}
=
1.44452.
\end{align*}
Die Resultate sind in Abbildung~\ref{50000030:graphplus} visualisiert
\begin{figure}
\centering
\includeagraphics[]{graph-2.pdf}
\caption{Wahrscheinlichkeitsdichte $\varphi_X(x)$ aus Aufgabe~\ref{50000030}
mit $E(X)$ und Varianz.
\label{50000030:graphplus}}
\end{figure}
\end{teilaufgaben}
\end{loesung}

\begin{bewertung}
\end{bewertung}

