Eine Maschine soll Bonbons in den vier Farben rot, gelb, grün
und orange so in
Plastiktüten einfüllen, dass im Mittel von jeder Farbe gleich
viele Bonbons vorkommen. Jede Tüte enthält $n=100$ Bonbons.
Zur Qualitätssicherung werden 100 Tüten ausgezählt und
es wird festgestellt, dass bei 10 Tüten die Zahl der roten
Bonbons um mindestens 8 vom Sollwert abweicht. Beweist das,
dass die Abfüllmaschine ein Problem hat?

\begin{loesung}
Die Anzahl der roten Bonbons ist binomialverteilt mit $p=\frac{n}4$,
die Binomialverteilung
kann aber durch die Normalverteilung approximiert werden mit
Varianz $np(1-p)$. Die Wahrscheinlichkeit, dass die Abweichung vom
Sollwert grösser als 5 ist, kann also mit der Normalverteilung
berechnet werden:
\[
P = P\left(\left|X-np\right|\ge 8\right)
=P\left(
\frac{X-np}{\sqrt{np(1-p)}}\ge \frac{8}{\sqrt{np(1-p)}}
\right)
\]
Der Ausdruck
\[
\frac{X-np}{\sqrt{np(1-p)}}
\]
ist eine standardnormalverteilte Zufallsvariable. Auf der
rechten Seite der Ungleichung steht der Wert
\[
\frac{8}{\sqrt{np(1-p)}}
=\frac{8}{\sqrt{100\cdot\frac14\cdot\frac34}}\simeq 1.8475
\]
Die Verteilungsfunktion der Standardnormalverteilung hat den
Wert $F(1.8475)\simeq0.9678$, daraus ergibt sich für die
Wahrscheinlichkeit einer Abweichung $>1.8475$
\[
P=2(1-F(1.8475))=0.644
\]
Eine so grosse Abweichung wird also nur in weniger als 7\%
der Fälle auftreten. Beim Test sind es jedoch 10 Tüten,
also deutlich mehr. Daher liegt die Vermutung nahe, dass
die Abfüllmaschine ein Problem hat.
\end{loesung}

