Berechnen Sie die Wahrscheinlichkeitsverteilung der Anzahl
richtiger Zahlen beim Schweizer Zahlenlotto.

\begin{hinweis}
Die Werte der Wahrscheinlichkeitsverteilung
der hypergeometrischen Verteilung können
in $R$ mit {\tt dhyper} berechnet werden.
\end{hinweis}

\thema{hypergeometrische Verteilung}

\begin{loesung}
Beim Schweizer Zahlenlotten werden 6 aus 45 Zahlen
gezogen. Die zu untersuchenden Verteilungsfunktion ist die
Zahl $K$ der Zahlen der Ziehung, die mit dem Lottozettel
übereinstimmen.
Die Wahrscheinlichkeit, dass $K=k$,
wird durch die hypergeometrische Verteilung berechnet:
\[
P(K=k)=\frac{\binom{6}{k}\binom{39}{6-k}}{\binom{45}{6}}
\]
\begin{center}
\begin{tabular}{|c|l|}
\hline
Anzahl richtige&Wahrscheinlichkeit\\
\hline
0& 0.4005646\\
1& 0.4241273\\
2& 0.1514740\\
3& 0.02244060\\
4& 0.001364631\\
5& 0.00002872907\\
6& 0.0000001227738\\
\hline
\end{tabular}
\end{center}
\end{loesung}

