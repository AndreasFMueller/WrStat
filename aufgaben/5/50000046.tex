Man erwartet, dass die Masse von Kartoffeln einer Sorte normalverteilt ist.
Dies hat zur Folge, dass manchmal auch sehr grosse oder sehr kleine 
Kartoffeln vorkommen.
In der Praxis werden solche Produkte daher meistens nach Grösse
selektiert.
Wenn man ein Kartoffeln aus einer breiten Normalverteilung selektiert,
dann bleiben nur noch Kartoffeln, deren Massen besser mit einer
Gleichverteilung in einem Interval beschrieben werden als mit einer
Normalverteilung.

Jetzt sollen Kartoffeln mit einer Masse zwischen 80g und 100g bestellt
werden, so dass man mindestens 10kg Kartoffeln bekommt.
Normalerweise würde man natürlich nach Gewicht bestellen, aber im neuen
Webshop muss man aus unerfindlichen Gründen die Kartoffeln
nach Anzahl beziehen
(der Manager, der das entschieden hat, arbeitet nicht mehr für die Firma\dots).
Man möchte also $n$ Kartoffeln bestellen.

\begin{teilaufgaben}
\item
Wie gross ist die Varianz der Masse einer selektierten Kartoffel?
\item
Wie gross muss $n$ sein, damit man mit mindestens 10kg Kartoffeln rechnen kann?
\item
Da die Masse der Kartoffeln immer noch streuen kann, wird
man manchmal zu viel erhalten, 
in einigen Fällen aber auch nicht die verlangten 10kg.
Wie gross ist die Wahrscheinlichkeit, dass das passiert?
\end{teilaufgaben}

\thema{Gleichverteilung}
\thema{Normalverteilung}
\thema{Standardisierung}

\begin{loesung}
Sei $K$ die Masse einer selektierten Kartoffel.
Wir gehen davon aus, dass $K$ gleichverteilt ist im Intervall zwischen
$a=80\text{g}$  und $b=100\text{g}$.
Insbesondere ist $E(K) = 90\text{g}=\mu_0$.
\begin{teilaufgaben}
\item
Die Varianz einer Gleichverteilung zwischen $a$ und $b$ ist
\[
\sigma_0^2
=
\operatorname{var}(K)
=
\frac{(b-a)^2}{12}
=
33.333\text{g}^2.
\]
\item
Die Masse $K_1,\dots,K_n$ der $n$ Kartoffeln sind gleichverteilte
Zufallsvariablen mit Erwartungswert 90g, daher hat die Summe
$X=K_1+\dots+K_n$
den Erwartungswert
$E(X)=n \mu_0$.
Damit der Erwartungswert 10kg erreicht wird, muss $n\ge 10kg/\mu_0$ sein,
also $n=112$.
\item
Nach dem zentralen Grenzwertsatz ist $X$ annähernd normalverteilt
mit Erwartungswert $\mu=n\mu_0$ und Varianz
$\sigma^2=n\sigma_0^2$.
Die Wahrscheinlichkeit, dass $X< 10\text{kg}$ ist, kann jetzt mit
Standardisierung berechnet werden:
\begin{align}
P(X< 10\text{kg})
=
P(X\le 10\text{kg})
&=
P\biggl(
\frac{X-\mu}{\sigma} \le \frac{10\text{kg}-\mu}{\sigma}
\biggr)
\notag
\\
&=
P\biggl(
Z \le \frac{10\text{kg}-\mu}{\sigma}
\biggr)
=
P\biggl(
Z \le
\frac{10\text{kg}-n\mu_0}{\sqrt{n}\sigma_0}
\biggr)
=
\Phi
\biggl(
\frac{10\text{kg}-n\mu_0}{\sqrt{n}\sigma_0}
\biggr)
\notag
\intertext{Da das Argument $\Phi$ negativ ist, gehen wir zum Komplement über}
&=
1-
\Phi
\biggl(
\frac{n\mu_0 - 10\text{kg}}{\sqrt{n}\sigma_0}
\biggr).
\label{50000046:formel}
\end{align}
Für $n=112$ ergibt sich
\[
P(X<10\text{kg})
=
1-\Phi(1.3093)
=
1-0.9047
=
0.0953
\qedhere
\]
\end{teilaufgaben}
\end{loesung}

\begin{bewertung}
\begin{teilaufgaben}
\item Varianz ({\bf V}) 1 Punkt
\item Wert für $n$ ({\bf N}) 1 Punkt
\item Normalverteilungsapproximation ({\bf A}) 1 Punkt,
Standardisierung mit $\mu$ ({\bf M}) und $\sigma$ ({\bf S}) je ein Punkt,
Wert für die Wahrscheinlichkeit ({\bf P}) 1 Punkt.
\end{teilaufgaben}
\end{bewertung}

