In dieser Aufgabe geht es um Würfe eines fairen Sechserwürfels.
\begin{teilaufgaben}
\item
Wie gross ist die Wahrscheinlichkeit, in hundert Würfen 
eine um mehr als 5 von der erwarteten Anzahl abweichende
Anzahl gerader Augenzahlen zu erhalten?
\item
Viel seltener passiert es, dass beim Werfen von 10 solchen Würfeln
genau fünf Würfel eine Fünf zeigen.
Wie wahrscheinlich ist, dass dies in 100 Versuchen mehr als 2 mal passiert?
\end{teilaufgaben}

\begin{loesung}
\begin{teilaufgaben}
\item
Die Anzahl $X$ gerader Augenzahlen ist binomialverteilt mit $n=100$
und $p=\frac12$.
Die erwartete Anzahl ist daher $\mu=np=50$ mit $\sigma=\sqrt{np(1-p)}=5$.
Mit der Normalapproximation der Binomialverteilung findet man
\begin{align*}
P(X<45\wedge X> 55)
&=
1
-
P(45\le X\le 55)
\\
P(45\le X\le 55)
&
P\biggl(
\frac{45-\mu}{\sigma} \le \frac{X-\mu}{\sigma} \le \frac{55-\mu}{\sigma}
\biggr)
\intertext{Die Normalapproximation der Binomialverteilung liefert}
&\cong
P\biggl(
\frac{-5.5}{\sigma} \le Z \le \frac{5.5}{\sigma}
\biggr)
=
P(-1.1\le Z \le 1.1),
\intertext{für eine standardnormalverteilte Zufallsvariable $Z$.
Die Wahrscheinlichkeit ist damit}
&=
\Phi(1.1) - \Phi(-1.1)
=
\Phi(1.1) - (1 - \Phi(1.1))
=
2\Phi(1.1)-1
\\
&=
2\cdot 0.8643-1
=
0.7286.
\end{align*}
Die gesuchte Wahrscheinlichkeit ist aber das Komplement, d.~h.~$0.2714$.
\item
Genau fünf Fünfer beim Wurf von 10 Würfeln haben die Wahrscheinlichkeit
\[
p
=
\binom{10}{5}\frac{1^5\cdot 5^5} {6^{10}}
=
0.013024.
\]
Man erwartet also, dass dies in $n=100$ Versuchen $\lambda=np=1.3024$ mal
passiert.
Die Poisson-Verteilung erlaubt, die Wahrscheinlichkeit zu berechnen:
\begin{align*}
P(Y>2)&=1-P(X \le 2)
\\
P(X\le 2)
&=
e^{-\lambda}
\sum_{k=0}^2 \frac{\lambda^k}{k!}
=
e^{-\lambda}\biggl(
1 + \lambda + \frac{\lambda^2}2
\biggr)
=
0.8566
\\
\Rightarrow\qquad
P(Y>2) &= 0.1434.
\qedhere
\end{align*}
\end{teilaufgaben}
\end{loesung}

\begin{bewertung}
Binomialverteilung ({\bf B}) 1 Punkt,
Berechnung von $\mu$ und $\sigma$ ({\bf M}) 1 Punkt,
Standardisierung ({\bf S}) 1 Punkt,
Normalapproximation und Wahrscheinlichkeit ({\bf N}) 1 Punkt,
Parameter $\lambda$ ({\bf L}) 1 Punkt,
Poisson-Verteilung und Wahrscheinlichkeit ({\bf P}) 1 Punkt.
\end{bewertung}

