Die Zufallsvariable $X$ sei die Zahl der Köpfe, die beim
wiederholten Werfen einer fairen Münze auftreten, bis zum
ersten Mal eine Zahl erscheint. Bestimmen Sie Wahrscheinlichkeitsverteilung
und Verteilungsfunktion von $X$.

\thema{Laplace-Experiment}
\thema{Wahrscheinlichkeitsverteilung}
\thema{Verteilungsfunktion}

\begin{loesung}
Die Wahrscheinlichkeit, dass genau $k$ mal hintereinander Kopf geworfen
wird, und dann einmal Zahl, ist $p^{k+1}$, $p=\frac12$. Also ist die
Wahrscheinlichkeitsverteilung
\[
P(K=k)=2^{-k-1},\qquad k\ge 0.
\]
Die Verteilungsfunktion ist für ganzzahlige Werte $x$ die Summe
\[
F(x)=\sum_{k=0}^x2^{-k-1}=\frac12\sum_{k=0}^x p^k=\frac12 \frac{1-p^{x+1}}{1-p}
=1-p^{x+1},
\]
wie man mit der Summenformel für die geometrische Reihe ausrechnen kann.
\end{loesung}

