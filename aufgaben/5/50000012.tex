Wie gross muss $c$ sein, damit
\[
\varphi(x)=\begin{cases}
c(1-x^4)\qquad&|x|\le 1\\
0&\text{sonst}
\end{cases}
\]
die Wahrscheinlichkeitsdichte einer Zufallsvariable $X$ ist? Berechnen 
Sie auch $E(X)$ und $\operatorname{var}(X)$.

\thema{Wahrscheinlichkeitsdichte}
\thema{Verteilungsfunktion}
\thema{Erwartungswert}
\thema{Varianz}

\begin{loesung}
Wir lösen die Aufgabe für die allgemeinere Funktion
\[
\varphi(x)=\begin{cases}
c(1-x^{2k})\qquad&|x|\le 1\\
0&\text{sonst}
\end{cases}
\]
Sie ist eine Wahrscheinlichkeitsdichte, wenn das Integral über $\mathbb R$
den Wert 1 ergibt:
\begin{align*}
1&=\int_{-\infty}^{\infty}\varphi(x)\,dx
\\
&=
\int_{-1}^11-x^{2k}\,dx=c\biggl[x-\frac{x^{2k+1}}{2k+1}\biggr]_{-1}^1
=c\biggl(2-\frac{2}{2k+1}\biggr)=c\frac{4k}{2k+1}
\\
c&=\frac{2k+1}{4k}
\end{align*}
Im vorliegenden Fall $k=2$ ist also $c=\frac{5}{8}$.

Für Erwartungswert und Varianz müssen die Integrale von 
$x\varphi(x)$ und $x^2\varphi(x)$ bestimmt werden:
\begin{align*}
E(X)
&=
\frac{2k+1}{4k}
\int_{-1}^1x(1-x^{2k})\,dx
=
\frac{2k+1}{4k}
\biggl[
\frac12x^2-\frac{x^{2k+2}}{2k+2}
\biggr]_{-1}^1=0,
\\
E(X^2)
&=
\frac{2k+1}{4k}\int_{-1}^1x^2(1-x^{2k})\,dx
=
\frac{2k+1}{4k}
\biggl[
\frac13x^3-\frac{x^{2k+3}}{2k+3}
\biggr]_{-1}^1
\\
&=
\frac{2k+1}{4k}
\biggl(
\frac23-\frac{2}{2k+3}
\biggr)
=
\frac{2k+1}{4k}
\cdot
\frac{4k}{3(2k+3)}
=
\frac13\cdot
\frac{2k+1}{2k+3}.
\end{align*}
Da $E(X)=0$ ist (was man auch aus Symmetrieüberlegungen ohne
Rechnen hätte bekommen können), ist $\operatorname{var}(X)=E(X^2)$,
im vorliegenden Fall $k=2$ also 
$\operatorname{var}(X)=\frac{5}{3\cdot 7}=\frac{5}{21}=0.2380952$.

Man beachte, dass diese verallgemeinerte Verteilung für $k\to\infty$
in eine Gleichverteilung auf dem Interval $[-1,1]$ übergeht, deren
Varianz, wie in der Vorlesung gezeigt, $\frac{l^2}{12}=\frac13$ ist, wobei
$l$ die Intervallänge $l=2$ ist. Tatsächlich strebt der obige 
Ausdruck für $E(X^2)$ für $k\to\infty$ gegen $\frac13$.
\end{loesung}
