In einem Abpackungsbetrieb werden Mandarinen vor der Verpackung
individuell gemessen.
Zu kleine und zu grosse Mandarinen werden aussortiert.
Es stellt sich heraus, dass 10\% der Mandarinen einen Durchmesser
kleiner als 50mm Durchmesser haben und 5\% einen Durchmesser grösser als 60mm.
\begin{teilaufgaben}
\item
Wie gross ist der mittlere Durchmesser einer Mandarine?
\item
Wie gross ist die Wahrscheinlichkeit, unter den aussortierten 
Mandarinen eine zu finden, deren Durchmesser grösser als 62mm ist?
\end{teilaufgaben}

\thema{Normalverteilung}
\thema{Standardisierung}

\begin{loesung}
Der Durchmesser einer Mandarine ist normalverteilt mit Erwartungswert
$\mu$ und Standardabweichung $\sigma$.
Die Angaben über die Sortierresultate zeigen
\begin{align*}
P(X<50) &= 0.10 \qquad\text{und}\\
P(X>60) &= 0.05.
\end{align*}
Mit Standardisierung erhalten wir
\begin{align*}
P\biggl(\frac{X-\mu}{\sigma} < \frac{50-\mu}{\sigma}\biggr)
=
\Phi\biggl(\frac{50-\mu}{\sigma}\biggr)
&= 0.1
&&\Rightarrow&
\frac{50-\mu}{\sigma} 
&=
-1.2816
&&\Rightarrow&
\mu
-1.2816\sigma&=
50,
\\
P\biggl(\frac{X-\mu}{\sigma} < \frac{60-\mu}{\sigma}\biggr)
=
\Phi\biggl(\frac{60-\mu}{\sigma}\biggr)
&= 0.95
&&\Rightarrow&
\frac{60-\mu}{\sigma}
&=
1.6449
&&\Rightarrow&
\mu+1.6449\sigma
&=
60.
\end{align*}
Die Lösung des linearen Gleichungssystems ist
\[
\mu = 54.3793
\qquad\text{und}\qquad
\sigma = 3.4171.
\]
\begin{teilaufgaben}
\item
Der mittlere Durchmesser ist $\mu=54.3793\text{mm}$.
\item
Die Wahrscheinlichkeit, eine Mandarine mit Durchmesser $>62$ 
zu finden, ist
\begin{align*}
P(X>62)
&=
1-P(X<62)
=
1-P\biggl(\frac{X-\mu}{\sigma} < \frac{62-\mu}{\sigma}\biggr)
=
1-\Phi\biggl(\frac{65-\mu}{\sigma}\biggr)
\\
&=
1-\Phi(2.2302)
=
1-0.9871
=
0.0129.
\end{align*}
Dies ist aber die Wahrscheinlichkeit, unter {\em allen} Mandarinen
eine mit Durchmesser grösser als 62mm zu finden.
Gesucht war die Wahrscheinlichkeit, unter den Mandarinen mit
Durchmesser grösser als 60mm eine mit 62mm zu finden.
Diese ist die bedingte Wahrscheinlichkeit
\begin{align*}
P(X > 62 | X > 60)
&=
\frac{P(\{X>62\}\cap\{X>60\})}{P(X>60)}
=
\frac{P(X>62\wedge X>60)}{P(X>60)}
=
\frac{P(X>62)}{P(X>60)}
\\
&=
\frac{0.0129}{0.05}
=
0.258.
\qedhere
\end{align*}
\end{teilaufgaben}
\end{loesung}

\begin{bewertung}
Standardisierung ({\bf S}) 1 Punkt,
Gleichungssystem ({\bf G}) 1 Punkt,
Bestimmung von $\mu$ ({\bf M}) 1 Punkt,
Bestimmung von $\sigma$ ({\bf V}) 1 Punkt,
Wahrscheinlichkeit für $X>62$ ({\bf W}) 1 Punkt,
bedingte Wahrscheinlichkeit ({\bf B}) 1 Punkt.
\end{bewertung}



