Bestimmen Sie Median, Spannweite, Quartilabstand und Mittelwert
des folgenden Datensatzes:
\begin{center}
\begin{tabular}{>{$}c<{$}|rrrrrrrrr}
 i &  1 & 2 & 3 & 4 & 5 & 6 & 7 & 8 & 9 \\
\hline
x_i&
   0.762 & % 1
   0.693 & % 2
   0.708 & % 3
   0.466 & % 4
   0.248 & % 5
   0.242 & % 6
   0.680 & % 7
   0.211 & % 8
   0.285   % 9
\end{tabular}
\end{center}

\begin{loesung}
Zunächst müssen die Datenpunkte sortiert werden:
\begin{center}
\begin{tabular}{>{$}c<{$}|rrrrrrrrr}
 i &
 8 &
 6 &
 5 &
 9 &
 4 &
 7 &
 2 &
 3 &
 1 
\\
\hline
 x_i  &
0.211 & % 8
0.242 & % 6
0.248 & % 5
0.285 & % 9
0.466 & % 4
0.680 & % 7
0.693 & % 2
0.708 & % 3
0.762   % 1
\end{tabular}
\end{center}
Daraus kann man jetzt ablesen, dass der Median
$x_{\text{med}} = 0.466$ ist.
Für Spannweite findet man $R=0.762-0.211=0.551$.
Aus der Graphik
\begin{center}
\begin{tikzpicture}[>=latex,thick]
\xdef\s{0.5555}
\xdef\u{5}

\fill[color=darkgreen!10] (0,0) -- (5.8,0) -- ++(0,5) -- (5,5) -- cycle;

\draw[->] (-1,0) -- (6.1,0) coordinate[label={$x$}];
\draw[->] (0,-0.1) -- (0,5.5) coordinate[label={left:$F(x)$}];

\foreach \y in {0.25,0.5,0.75,1}{
	\draw[line width=0.2] (-1,{\y*\u}) -- (5.8,{\y*\u});
}

\draw[color=blue,line width=0.2pt] ({0.248*\u},0) -- ++(0,{0.25*5});
\draw[color=blue,line width=0.2pt] ({0.466*\u},0) -- ++(0,{0.50*5});
\draw[color=blue,line width=0.2pt] ({0.693*\u},0) -- ++(0,{0.75*5});

\draw[color=darkgreen,line width=2pt] (-1,0) -- ({0.211*\u},0);

\draw[color=darkgreen,line width=0.2pt] ({0.211*\u},0) -- ++(0,\s);

\draw[color=darkgreen,line width=2pt]
	({0.211*\u},{1*\s}) -- ({0.242*\u},{1*\s});

\draw[color=darkgreen,line width=0.2pt] ({0.242*\u},{1*\s}) -- ++(0,\s);

\draw[color=darkgreen,line width=2pt]
	({0.242*\u},{2*\s}) -- ({0.248*\u},{2*\s});

\draw[color=darkgreen,line width=0.2pt] ({0.248*\u},{2*\s}) -- ++(0,\s);

\draw[color=darkgreen,line width=2pt]
	({0.248*\u},{3*\s}) -- ({0.285*\u},{3*\s});

\draw[color=darkgreen,line width=0.2pt] ({0.285*\u},{3*\s}) -- ++(0,\s);

\draw[color=darkgreen,line width=2pt]
	({0.285*\u},{4*\s}) -- ({0.466*\u},{4*\s});

\draw[color=darkgreen,line width=0.2pt] ({0.466*\u},{4*\s}) -- ++(0,\s);

\draw[color=darkgreen,line width=2pt]
	({0.466*\u},{5*\s}) -- ({0.680*\u},{5*\s});

\draw[color=darkgreen,line width=0.2pt] ({0.680*\u},{5*\s}) -- ++(0,\s);

\draw[color=darkgreen,line width=2pt]
	({0.680*\u},{6*\s}) -- ({0.693*\u},{6*\s});

\draw[color=darkgreen,line width=0.2pt] ({0.693*\u},{6*\s}) -- ++(0,\s);

\draw[color=darkgreen,line width=2pt]
	({0.693*\u},{7*\s}) -- ({0.708*\u},{7*\s});

\draw[color=darkgreen,line width=0.2pt] ({0.708*\u},{7*\s}) -- ++(0,\s);

\draw[color=darkgreen,line width=2pt]
	({0.708*\u},{8*\s}) -- ({0.762*\u},{8*\s});

\draw[color=darkgreen,line width=0.2pt] ({0.762*\u},{8*\s}) -- ++(0,\s);

\draw[color=darkgreen,line width=2pt] ({0.762*\u},5) -- (5.8,5);

\draw ({0.211*\u},-0.05) -- ++(0,0.1);
\draw ({0.242*\u},-0.05) -- ++(0,0.1);
\draw ({0.248*\u},-0.05) -- ++(0,0.1);
\draw ({0.285*\u},-0.05) -- ++(0,0.1);
\draw ({0.466*\u},-0.05) -- ++(0,0.1);
\draw ({0.680*\u},-0.05) -- ++(0,0.1);
\draw ({0.693*\u},-0.05) -- ++(0,0.1);
\draw ({0.708*\u},-0.05) -- ++(0,0.1);
\draw ({0.762*\u},-0.05) -- ++(0,0.1);

\end{tikzpicture}
\end{center}
kann man den Quartilabstand $Q = 0.693 - 0.248=0.445$ ablesen.

Der Mittelwert ist
\begin{align*}
\bar{x}
&=
\frac{1}{9}
\sum_{i=1}^9
x_i
=
\frac{
   0.762 + 0.693 + 0.708 + 0.466 + 0.248 + 0.242 + 0.680 + 0.211 + 0.285
}{9}
\\
&=
\frac{4.295}{9}
=
0.477\overline{2}.
\qedhere
\end{align*}
\end{loesung}
