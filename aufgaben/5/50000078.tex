Stellen Sie das Faltungsintegral für die Faltung der beiden
Wahrscheinlichkeitsdichten
\[
\varphi_1(x)
=
\frac{1}{\sqrt{2\pi}}e^{-x^2/2}
\qquad\text{und}\qquad
\varphi_2(x)
=
\frac{1}{\sqrt{8\pi}}e^{-x^2/8}
\]
auf.

\begin{loesung}
Das Integral ist
\begin{align}
\varphi_1*\varphi_2(z)
&=
\int_{-\infty}^\infty
\varphi_1(x) \varphi_2(z-x)\,dx
\notag
\\
&=
\int_{-\infty}^\infty
\frac{1}{\sqrt{2\pi}}e^{-x^2/2}
\frac{1}{\sqrt{8\pi}}e^{(z-x)^2/8}
\,dx
\notag
\\
&=
\frac{1}{4\pi}
\int_{-\infty}^\infty
\exp\biggl(
-\frac{x^2}2-\frac{z^2-2xz+x^2}8
\biggr)
\,dx
\notag
\\
&=
\frac{1}{4\pi}
\int_{-\infty}^\infty
\exp\biggl(
-\frac{4x^2}8-\frac{z^2-2xz+x^2}8
\biggr)
\,dx
\notag
\\
&=
\frac{1}{4\pi}
\int_{-\infty}^\infty
\exp\biggl(
-\frac{5x^2-2xz+z^2}8
\biggr)
\,dx.
\notag
\intertext{Mit quadratischem Ergänzen kann man es weiter zu}
&=
\frac{1}{4\pi}
\int_{-\infty}^\infty
\exp\biggl(
-\frac{5(x-\frac15z)^2+\frac45z^2}8
\biggr)
\,dx
\notag
\intertext{vereinfachen.
Der letzte Summand im Exponenten hängt nicht von der
Integrationsvariable ab und kann daher aus dem Integral gezogen werden.
Der Summand $-\frac15z$ ist nur eine Verschiebung, die auf das
uneigentliche Integral keinen Einfluss hat.
Es bleibt daher}
&=
\frac{1}{4\pi}
e^{-\frac{1}{2\cdot 5}z^2}
\int_{-\infty}^\infty
e^{-\frac58x^2}\,dx.
\label{50000078:eqn:integral}
\end{align}
Das letzte Integral ist eine Konstante, die zusammen mit
den Faktoren vor der Exponentialfunktion eine Wahrscheinlichkeitsdichte
ergeben muss.
Es folgt, dass die Faltung der Funktionen $\varphi_1$ und $\varphi_2$
wieder eine Funktion der selben Form
\[
\varphi_\sigma=\frac{1}{\sqrt{2\pi}\sigma}e^{-x^2/2\sigma^2}
\]
mit $\sigma^2=5$ oder $\sigma=\!\sqrt{5}$ ist, die wir später als die
Normalverteilung bezeichnen werden.

Das Integral auf der rechten Seite von \eqref{50000078:eqn:integral}
kann durch die Substition $\xi = \frac{\sqrt{5}}2x$
bwz.~$dx=\frac{2}{\sqrt{5}}\xi$ in die Form
\begin{align*}
\int_{-\infty}^\infty
e^{-\frac58x^2}\,dx
&=
\int_{-\infty}^\infty
e^{-\frac{\xi^2}{2}}
\,
\frac{2}{\!\sqrt{5}}
\,d\xi
=
\frac{2}{\!\sqrt{5}}
\cdot
\sqrt{2\pi}
\cdot
\underbrace{
\frac{1}{\sqrt{2\pi}}
\int_{-\infty}^\infty
e^{-\frac{\xi^2}{2}}
\,d\xi
}_{\displaystyle=1}
=
\sqrt{\frac{8\pi}{5}}.
\end{align*}
Einsetzen in \eqref{50000078:eqn:integral} ergibt
\begin{align*}
\varphi_1*\varphi_2(z)
&=
\frac{1}{4\pi}
\sqrt{\frac{8\pi}{5}}
e^{-\frac{z^2}{2(\!\sqrt{5})^2}}
=
\frac{1}{\sqrt{2\pi}\sqrt{5}}
e^{-\frac{z^2}{2(\!\sqrt{5})^2}}
=
\frac{1}{\!\sqrt{2\pi}\sigma}
e^{-\frac{x^2}{2\sigma^2}}.
\qedhere
\end{align*}
\end{loesung}
