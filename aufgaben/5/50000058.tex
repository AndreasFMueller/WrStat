Man betrachte die Funktion
\[
\varphi(x)
=
\begin{cases}
a&\qquad -\frac32\le x\le -\frac12 \text{ oder } \frac12\le x\le \frac32
\\
0&\qquad\text{sonst}
\end{cases}
\]
\begin{teilaufgaben}
\item
Wie muss $a$ gewählt werden, damit $\varphi(x)$ die
Wahrscheinlichkeitsdichte einer Zufallsvariable $X$ wird?
\item
Zeichnen Sie einen Graphen der Verteilungsfunktion $F_X(x)$.
\item
Bestimmen Sie $E(X)$.
\item
Bestimmen Sie die Varianz $\operatorname{var}(X)$.
\end{teilaufgaben}


\begin{loesung}
\begin{figure}
\centering
\begin{tikzpicture}[>=latex,thick]
\def\sx{2}
\def\sy{3}
\def\a{0.5}
\draw[->] ({-2.1*\sx},0) -- ({2.1*\sx},0) coordinate[label={$x$}];
\draw[->] (0,-0.1) -- (0,{2*\sx}) coordinate[label={left:$\color{darkgreen}F_X(x)$}];
\fill[color=red!20] ({-1.5*\sx},0) rectangle ({-0.5*\sx},{\a*\sy});
\fill[color=red!20] ({1.5*\sx},0) rectangle ({0.5*\sx},{\a*\sy});
\draw[color=red,line width=1pt] ({-1.5*\sx},{\a*\sy}) -- ({-0.5*\sx},{\a*\sy});
\draw[color=red,line width=1pt] ({1.5*\sx},{\a*\sy}) -- ({0.5*\sx},{\a*\sy});
\draw[color=red,line width=1pt] ({-2*\sx},0) -- ({-1.5*\sx},0);
\draw[color=red,line width=1pt] ({2*\sx},0) -- ({1.5*\sx},0);
\draw[color=red,line width=1pt] ({-0.5*\sx},0) -- ({0.5*\sx},0);
\draw[color=red,line width=0.1pt] ({-1.5*\sx},0) -- ({-1.5*\sx},{\a*\sy});
\draw[color=red,line width=0.1pt] ({-0.5*\sx},0) -- ({-0.5*\sx},{\a*\sy});
\draw[color=red,line width=0.1pt] ({1.5*\sx},0) -- ({1.5*\sx},{\a*\sy});
\draw[color=red,line width=0.1pt] ({0.5*\sx},0) -- ({0.5*\sx},{\a*\sy});
\draw[color=darkgreen,line width=1.4pt]
	   ({-2*\sx},0)
	-- ({-1.5*\sx},0)
	-- ({-0.5*\sx},{0.5*\sy})
	-- ({0.5*\sx},{0.5*\sy})
	-- ({1.5*\sx},{\sy})
	-- ({2*\sx},\sy);
\draw ({-1*\sx},-0.05) -- ({-1*\sx},0.05);
\node at ({-1*\sx},-0.05) [below] {$-1$};
\draw ({1*\sx},-0.05) -- ({1*\sx},0.05);
\node at ({1*\sx},-0.05) [below] {$1$};
\node[color=red] at ({1.4*\sx},{\a*\sy}) [above right] {$\varphi(x)$};
\draw (-0.05,{\sy}) -- (0.05,\sy);
\node at (-0.05,\sy) [left] {$1$};
\node[color=blue] at (0.05,{\a*\sy}) [above left] {$a=\frac12$};
\draw[color=blue,line width=1.5pt] (-0.1,{\a*\sy}) -- (0.1,{\a*\sy});
\draw[color=orange,line width=1.4pt]
	({sqrt(13/12)*\sx},-0.2) -- ({sqrt(13/12)*\sx},{1.2*\sy});
\draw[color=orange,line width=1.4pt]
	({-sqrt(13/12)*\sx},-0.2) -- ({-sqrt(13/12)*\sx},{1.2*\sy});
\draw[<->,color=orange,line width=0.2pt] 
	({-sqrt(13/12)*\sx},{0.85*\sy}) -- (0,{0.85*\sy});
\draw[<->,color=orange,line width=0.2pt] 
	({sqrt(13/12)*\sx},{0.85*\sy}) -- (0,{0.85*\sy});
\node[color=orange] at ({0.5*sqrt(13/12)*\sx},{0.85*\sy})
	[above] {$\sqrt{\operatorname{var}(X)}$};
%\node[color=orange] at ({-0.5*sqrt(13/12)*\sx},{0.85*\sy})
%	[above] {$\sqrt{\operatorname{var}(X)}$};
\end{tikzpicture}
\caption{Verteilungsfunktion der Zufallsvariable $X$ von
Aufgabe~\ref{50000058}.
\label{50000058:verteilungsfunktion}}
\end{figure}
\begin{teilaufgaben}
\item
Das Integral von $\varphi(x)$ muss $1$ werden.
\begin{align*}
1&=
\int_{-\infty}^\infty \varphi(x)\,dx
=
\int_{-\frac32}^{-\frac12} a\,dx
+
\int_{\frac12}^{\frac32} a\,dx
=
a+a
\qquad\Rightarrow\qquad a=\frac12.
\end{align*}
\item
$F_X$ ist konstant in den Teilen des Definitionsbereiches, wo
$\varphi(x)=0$ ist.
Ausserhalb dieser Bereich hat der Graph die Steigung $a=\frac12$.
Der Graph von $F_X(x)$ ist in Abbildung~\ref{50000058:verteilungsfunktion}
dargestellt.
\item
Da $\varphi(x)$ eine gerade Funktion ist, ist $E(X)=0$.
\item
Die Varianz kann mit Hilfe von
\[
\]
berechnet werden.
Der Erwartungswert von $X^2$ ist
\begin{align*}
E(X^2)
&=
\int_{-\infty}^\infty x^2\,\varphi(x)\,dx
=
\int_{-\frac32}^{-\frac12}\frac12x^2\,dx
+
\int_{\frac12}^{\frac32}\frac12x^2\,dx
=
2\int_{\frac12}^{\frac32}\frac12x^2\,dx
=
\biggl[\frac13x^3\biggr]_{\frac12}^{\frac32}
\\
&=
\frac13\biggl(\frac{27}{8} - \frac18\biggr)
=
\frac13\cdot\frac{26}{8}
=
\frac{13}{12}.
\qedhere
\end{align*}
\end{teilaufgaben}
\end{loesung}

\begin{bewertung}
Bestimmung von $a$ ({\bf A}) 1 Punkt,
Graph von $F_X(x)$ ({\bf G}) 1 Punkt,
Erwartungswert ({\bf E}) 1 Punkt,
Integralformel für $E(X^2)$ ({\bf I}) Punkt,
Varianzformel $\operatorname{var}(X)=E(X^2)-E(X)^2$ ({\bf V}) 1 Punkt,
Wert der Varianz ({\bf Z}) 1 Punkt.
\end{bewertung}
