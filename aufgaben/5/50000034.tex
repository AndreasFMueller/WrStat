Ein Sprachschüler möchte im standardisierten Abschlusstest seines
Sprachaufenthalts
ein besonders gutes Resultat erreichen, da es mit darüber entscheidet,
ob er ein Visum für einen weiteren Aufenthalt erhält.
Er weiss, dass 5\% der Schüler im Abschlusstest weniger als 420 Punkte
erreichen, und dass ein Drittel mehr als als 540 Punkte erreichen.
Er möchte zu den besten 10\% gehören, welche Punktzahl muss
er erreichen?

\thema{Normalverteilung}
\thema{Standardisierung}

\begin{loesung}
Man darf annehmen, dass die Punktzahl $X$ eine normalverteilte
Zufallsvariable mit Erwartungswert $\mu$ und Varianz $\sigma^2$ ist.
Wir müssen $\mu$ und $\sigma$ aus den gegebenen Informationen bestimmen.
Die beiden Schranken erfüllen
\begin{align*}
P(X\le 420)&= 0.05
&
&%\Rightarrow
&
&\hspace{-10pt}&
P\biggl(\frac{X-\mu}{\sigma}\le \frac{420-\mu}{\sigma}\biggr)
&=
0.05
&&\Rightarrow
&
\frac{420-\mu}{\sigma}
&=
-1.6449
\\
P(X> 540)&=0.33
&
&\Rightarrow
&
P(X\le 540)=\mathstrut
&\hspace{-10pt}&
P\biggl(\frac{X-\mu}{\sigma}\le \frac{540-\mu}{\sigma}\biggr)
&=
0.66
&&\Rightarrow&
\frac{540-\mu}{\sigma}
&=
\phantom{-}
0.43.
\end{align*}
Multipliziert man mit $\sigma$, findet man das Gleichungssystem
\begin{align}
420-\mu&=-1.6449\sigma
\label{50000034:gl1}
\\
540-\mu&=\phantom{-}0.43\phantom{00}\sigma.
\label{50000034:gl2}
\end{align}
Die Differenz
$
\eqref{50000034:gl2}
-
\eqref{50000034:gl1}
$
ist
\[
120=2.0749\sigma
\qquad\Rightarrow\qquad
\sigma=57.83.
\]
Einsetzen von $\sigma$ in der Gleichung
\eqref{50000034:gl2}
liefert
\[
\mu=540-0.43\sigma=540-0.43\cdot 57.83=515.13.
\]
Der Sprachschüler möchte jetzt aber unter den besten 10\% sein, d.h. er
muss die Punktzahl $x$ erreichen, so dass 90\% der Schüler schlechter sind,
also
\begin{align*}
P(X\le x)=0.9
&=
P\biggl(\frac{X-\mu}{\sigma}\le \frac{x-\mu}{\sigma}\biggr)=0.9
&&\Rightarrow&
\frac{x-\mu}{\sigma} &= 1.2816
\\
&
&&\Rightarrow&
x &= \mu+1.2816\sigma
\\
&&&&
  &=515.13 + 1.2816\cdot 57.83=589.24.
\end{align*}
Er muss also eine Punktzahl von 590 Punkten erreichen, um zu den
obersten 10\% zu gehören.
\end{loesung}

\begin{bewertung}
Normalverteilung ({\bf N}) 1 Punkt,
Standardisierung ({\bf S}) 1 Punkt,
Gleichungen für $\mu$ und $\sigma$ ({\bf G}) 1 Punkt,
Wert von $\mu$ ({\bf M}) 1 Punkt,
Wert von $\sigma$ ({\bf V}) 1 Punkt,
Punktzahl ({\bf P}) 1 Punkt.
\end{bewertung}


