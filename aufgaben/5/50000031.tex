Eine Zufallsvariable $X$ hat die Wahrscheinlichkeitsdichte
\[
\varphi(x)=\begin{cases}
0&\qquad x\le -1\\
ae^x&\qquad -1<x\le 1\\
0&\qquad x > 1
\end{cases}
\]
\begin{teilaufgaben}
\item Wie muss $a$ gewählt werden, damit $\varphi(x)$ tatsächlich eine
Wahrscheinlichkeitsdichte ist?
\item Berechnen Sie den Erwartungswert von $X$.
\item Berechnen Sie die Varianz von $X$.
\end{teilaufgaben}

\begin{loesung}
\begin{teilaufgaben}
\item
Das Integral von $\varphi$ über $\mathbb R$ muss $1$ ergeben:
\begin{align*}
1
&=
\int_{-\infty}^{\infty} \varphi(x)\,dx
=
\int_{-1}^1ae^x\,dx
=
\left[ae^x\right]_{-1}^1
=
ae-ae^{-1}
\\
\Rightarrow\qquad
a&=\frac1{e-e^{-1}}=\frac1{\sinh(-1)}
=
0.4254590641196608.
\end{align*}
\item
Der Erwartungswert ist
\begin{align*}
E(X)
&=
\int_{-\infty}^\infty x\,\varphi(x)\,dx
=
a\int_{-1}^1xe^x\,dx
=
a\left[xe^x\right]_{-1}^1-a\int_{-1}^1e^x\,dx
=
\frac{e+e^{-1}}{e-e^{-1}}-1
\\
&=
\frac{2e^{-1}}{e-e^{-1}}
=
0.3130352854993313.
\end{align*}
\item
Für die Varianz brauchen wir zunächst den quadratischen Mittelwert
\begin{align*}
E(X^2)
&=
\int_{-\infty}^\infty x^2\varphi(x)\, dx
=
a\int_{-1}^1 x^2e^x\,dx
=
a\left[x^2e^x\right]_{-1}^1
-
2a\int_{-1}^1xe^x\,dx
\\
&=
\frac{e-e^{-1}}{e-e^{-1}}
-2E(X)
=
1-2E(X)
=
\frac{e-5e^{-1}}{e-e^{-1}}
\\
\operatorname{var}(X)
&=
E(X^2)-E(X)^2
=
\frac{e-5e^{-1}}{e-e^{-1}}
-
\biggl(
\frac{e-5e^{-1}}{e-e^{-1}}
\biggr)^2
=
0.2759383390336894.
\qedhere
\end{align*}
\end{teilaufgaben}
\end{loesung}

\begin{bewertung}
Normierungsbedingung ({\bf N}) 1 Punkt,
Wert für $a$ ({\bf A}) 1 Punkt,
Erwartungswert Formel ($\textbf{E}_F$) 1 Punkt,
Erwartungswert Resultat ($\textbf{E}_R$) 1 Punkt,
Varianz Formel ($\textbf{V}_F$) 1 Punkt,
Varianz Resultat ($\textbf{V}_R$) 1 Punkt.
\end{bewertung}




