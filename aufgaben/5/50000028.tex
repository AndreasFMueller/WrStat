Betrachten Sie die Wahrscheinlichkeitsdichte mit dem Graphen
\begin{center}
\includeagraphics[]{graph-1.pdf}
\end{center}
\begin{teilaufgaben}
\item Finden Sie die Verteilungsfunktion.
\item Bestimmen Sie Erwartungswert und Varianz einer Zufallsvariable
mit dieser Verteilung.
\end{teilaufgaben}

\thema{Wahrscheinlichkeitsdichte}
\thema{Verteilungsfunktion}
\thema{Erwartungswert}
\thema{Varianz}

\begin{loesung}
Die Wahrscheinlichkeitsdichte ist 
\[
\varphi(x)=\begin{cases}
\frac12+\frac14x&\qquad -1\le x\le 1\\
0&\qquad\text{sonst.}
\end{cases}
\]
\begin{teilaufgaben}
\item
\begin{figure}
\centering
\includeagraphics[]{graph-2.pdf}
\caption{Graph der Verteilungsfunktion der Verteilung von
Aufgabe~\ref{50000028}
\label{50000028:verteilungsfunktion}}
\end{figure}
Die Verteilungsfunktion ist die Stammfunktion von $\varphi$.
Für das Intervall zwischen $-1$ und $1$ gilt
\begin{align*}
\int_{-1}^x\varphi(\xi)\,d\xi
&=
\biggl[\frac12 \xi+\frac18\xi^2\biggr]_{-1}^x
=
\frac18x^2+\frac12x+\frac{3}{8}.
\end{align*}
Die Verteilungsfunktion ist daher
\[
F(x)=\begin{cases}
0&\qquad x \le -1\\
\frac18x^2+\frac12x+\frac{3}{8}
&\qquad -1 < x\le 1\\
1&\qquad x >1
\end{cases}
\]
Der Graph der Verteilungsfunktion ist in
Abbildung~\ref{50000028:verteilungsfunktion} dargestellt.
\item
\begin{figure}
\centering
\includeagraphics[]{graph-3.pdf}
\caption{Graph der Wahrscheinlichkeitsdichte von Aufgabe~\ref{50000028}
mit eingezeichnetem Erwartungswert und Standardabweichung
\label{50000028:varianz}}
\end{figure}
Um Erwarungswert und Varianz zu bestimmen, müssen wir die Erwartungswerte
$E(X)$ und $E(X^2)$ berechnen:
\begin{align*}
E(X)&=
\int_{-\infty}^\infty x\varphi(x)\,dx
=
\int_{-1}^{1}
\frac12x+\frac14x^2
\,dx
=
\biggl[
\frac14x^2+\frac1{12}x^3
\biggr]_{-1}^{1}
=\frac1{6},
\\
E(X^2)
&=
\int_{-\infty}^\infty x^2\varphi(x)\,dx
=
\int_{-1}^{1}
\frac12x^2+\frac14x^3
\,dx
=
\biggl[
\frac16x^3+\frac1{16}x^4
\biggr]_{-1}^{1}
=\frac1{3}.
\end{align*}
Daraus kann man jetzt die Varianz ablesen:
\[
\operatorname{var}(X) = E(X^2)-E(X)^2=\frac1{3}-\biggl(\frac1{6}\biggr)^2
=\frac{11}{36}=0.305555.
\]
Diese Resultate sind in Abbildung~\ref{50000028:varianz} in den Graphen
der Wahrscheinlichkeitsdichte eingezeichnet.
\qedhere
\end{teilaufgaben}
\end{loesung}

\begin{bewertung}
\begin{teilaufgaben}
\item Dichtefunktion ({\bf D}) 1 Punkt,
Integration ({\bf V}) 1 Punkt,
\item
Erwartungswert $E(X)$ ({\bf E}) 1 Punkt,
Erwartungswert $E(X^2)$ ({\bf E2)} 1 Punkt
Varianzformel $\operatorname{var}(X)=E(X^2)-E(X)^2$ ({\bf F}) 1 Punkt,
Varianz $\operatorname{var}(X)$ ({\bf V}) 1 Punkt.
\end{teilaufgaben}
\end{bewertung}


