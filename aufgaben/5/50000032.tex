Betrachten Sie die Funktion
\begin{center}
\begin{minipage}{0.45\hsize}
\[
\varphi(x)
=
\begin{cases}
0     &\qquad x \le -1 \\
a(1+x)&\qquad -1<x\le 0\\
a     &\qquad 0<x\le 1\\
0     &\qquad 1< x
\end{cases}
\]
\end{minipage}
\qquad
\begin{minipage}{0.45\hsize}
\includeagraphics[]{verteilung.pdf}
\end{minipage}
\end{center}
mit dem Parameter $a$.
\begin{teilaufgaben}
\item
Wie muss die Konstante $a$ gewählt werden, dass $\varphi(x)$ eine
Wahrscheinlichkeitsdichte ist?
\item
Berechnen Sie den Erwartungswert einer Zufallsvariablen mit dieser
Wahr\-schein\-lichkeits\-dichte.
\item
Berechnen Sie die Varianz einer Zufallsvariablen mit dieser
Wahrscheinlichkeitsdichte.
\end{teilaufgaben}

\begin{loesung}
\begin{teilaufgaben}
\item
Wir müssen $a$ so wählen, dass das Integral von $\varphi$ über die reelle
Achse $=1$ wird.
Dazu berechnen wir
\begin{align*}
\int_{-\infty}^\infty\varphi(x)\,dx
&=
\int_{-1}^0a(1+x)\,dx +\int_0^1 a\,dx
=
\biggl[ax+\frac12ax^2\biggr]_{-1}^0 + a
=
a-\frac12a+a=\frac32a.
\end{align*}
Damit das Integral $=1$ wird, muss also $a=\frac23$ gewählt werden.
\item
Der Erwartungswert kann mit Hilfe des Integrals
\begin{align*}
E(X)
&=
\int_{-\infty}^\infty x\varphi(x)\,dx
=
\int_{-1}^0 xa(1+x)\,dx + \int_0^1ax\,dx
=
\biggl[
\frac12ax^2+\frac13ax^3
\biggr]_{-1}^0
+
\biggl[
\frac12ax^2
\biggr]_0^1
\\
&=
-\frac12a +\frac13a +\frac12a
=
\frac13a
=
\frac13\cdot\frac23
=
\frac29
\end{align*}
berechnet werden.
\item
Für die Varianz brauchen wir zusätzlich
\begin{align*}
E(X^2)
&=
\int_{-\infty}^\infty x^2\varphi(x)\,dx
=
\int_{-1}^0 ax^2(1+x)\,dx +\int_0^1 ax^2\,dx
=
\biggl[
\frac13ax^3 + \frac14ax^4
\biggr]_{-1}^0
+
\biggl[
\frac13ax^3
\biggr]_0^1
\\
&=
\frac13a - \frac14a + \frac13a
=
\biggl(\frac23-\frac14\biggr)a
=
\frac{8-3}{12}a
=
\frac5{12}a
=
\frac5{12}\cdot\frac{2}{3}
=
\frac{5}{18}.
\end{align*}
Daraus kann jetzt die Varianz bestimmt werden:
\begin{align*}
\operatorname{var}(X)
&=
E(X^2)-E(X)^2
=
\frac{5}{18}-\frac{4}{81}
=
\frac{37}{162}
=
0.228395.
\qedhere
\end{align*}
\end{teilaufgaben}
\end{loesung}

\begin{bewertung}
Normierungsbedingung ({\bf N}) 1 Punkt,
Parameter $a$ ({\bf A}) 1 Punkt,
Formel für Erwartungswert ({\bf F}) 1 Punkt,
Wert des Erwartungswerts ({\bf E}) 1 Punkt,
Varianzformel ({\bf V}) 1 Punkt,
Wert der Varianz ({\bf W}) 1 Punkt.
\end{bewertung}

\begin{diskussion}
Der Erwartungswert in Teilaufgabe b) kann auch mit einer geometrischen 
Überlegung gefunden werden.
Die Fläche unter der Kurve links der $\varphi$-Achse  ist halb so gross
wie die Fläche unter der Kurve rechts der $\varphi$-Achse. 
Wir können daher den Erwartungswert jedes Teils separat berechnen und
dann mit den Gewichten $\frac13\colon \frac23$ mitteln.
Der Erwartungswert der Gleichverteilung rechts der $\varphi$-Achse ist
$\frac12$.
Der Erwartungswert der Verteilung links der $\varphi$-Achse ist
$-\frac13$.
Wir erhalten daher für den Erwartungswert:
\[
E(X)
=
\frac13\cdot\biggl(-\frac13\biggr) + \frac23\cdot\frac 12
=
-\frac19 + \frac13=\frac29.
\]
\end{diskussion}
