Die Zufallsvariable $X$ hat eine symmetrische Verteilung, im
Interval $[0,1]$ ist ihre Verteilungsfunktion $F(x)=\frac12(1+x^n)$.
\begin{teilaufgaben}
\item
Berechnen Sie den Erwartungswert von $X$.
\item
Berechnen Sie ihre Varianz von $X$.
\item
Was passiert, wenn man $n$ beliebig gross werden lässt?
\end{teilaufgaben}

\thema{Wahrscheinlichkeitsdichte}
\thema{Verteilungsfunktion}
\thema{Erwartungswert}
\thema{Varianz}

\begin{loesung}
\begin{teilaufgaben}
\item Da $X$ symmetrisch verteilt ist, ist $E(X)=0$.
\item
Für die Varianz braucht man ausser dem Erwartungswert auch noch
den quadratischen Mittelwert.  Für seine Berechnung braucht man die
Dichtefunktion $\varphi(x)=F'(x)$. Ausserdem kann man die Symmetrie
von $\varphi(x)$ ausnutzen.
\begin{align*}
E(X^2)
&=
\int_{-\infty}^{\infty}x^2\varphi(x)\,dx
=
2\int_{0}^{\infty}x^2\varphi(x)\,dx
=2\int_0^1 x^2 F'(x)\,dx\\
&=
2[x^2F(x)|_0^1-2\int_0^12xF(x)\,dx
\\
&=
2F(1)-20\cdot F(0)-2\int_0^1x(1+x^n)\,dx
\\
&=
2-2\left[
\frac12x^2+\frac1{n+2}x^{n+2}
\right]_0^1
\\
&=
2-2\biggl(\frac12+\frac1{n+2}\biggr)
=
1-\frac{2}{n+2}
=
\frac{n}{n+2}
\end{align*}
Man kann das Integral natürlich auch konventioneller ausrechnen:
\begin{align*}
E(X^2)&=2\int_0^1x^2\cdot\frac{n}2x^{n-1}\,dx
=
n\int_0^1x^{n+1}\,dx=n\left[\frac{x^{n+2}}{n+2}\right]_0^1
=\frac{n}{n+2}
\end{align*}
Daraus lesen wir für die Varianz ab
\[
\operatorname{var}(X)=E(X^2)-E(X)^2=E(X^2)=\frac{n}{n+2}.
\]
\item
Für $n\to\infty$ strebt die Varianz gegen $1$, der Erwartungswert
bleibt natürlich $0$. Man kann das auch der Verteilungsfunktion
ansehen: Sie wird zu einer Stufenfunktion, die an den Stellen
$\pm 1$ je eine Stufe der Höhe $\frac12$ hat. Dies bedeutet
natürlich, dass im Limes $n\to\infty$ nur noch die Werte $\pm1$ mit
Wahrscheinlichkeit auftreten werden.
\qedhere
\end{teilaufgaben}
\end{loesung}

