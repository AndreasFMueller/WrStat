Betrachten Sie die Funktion
\[
\varphi(x)
=
\begin{cases}
a(1-x^4) &\qquad -1\le x \le 1\\
       0 &\qquad \text{sonst.}
\end{cases}
\]
\begin{teilaufgaben}
\item
Wie muss $a$ gewählt werden, damit $\varphi(x)$ die Wahrscheinlichkeitsdichte
einer Zufallsvariable $X$ ist?
\item
Bestimmen Sie die Verteilungsfunktion von $X$.
\item
Bestimmen Sie den Erwartungswert $E(X)$.
\item
Bestimmen Sie die Varianz $\operatorname{var}(X)$.
\end{teilaufgaben}

\begin{loesung}
\begin{figure}
\centering
\begin{tikzpicture}[>=latex,thick,scale=4]
\fill[color=red!20] plot[domain=-1:1,samples=100]
	({\x},{(5/8)*(1-\x*\x*\x*\x)}) -- cycle;
\draw[->] (-1.5,0) -- (1.5,0) coordinate[label={$x$}];
\draw[->] (0,-0.02) -- (0,0.8) coordinate[label={right:$\varphi(x)$}];
\draw[color=red,line width=1.4pt] (-1.45,0)
	-- plot[domain=-1:1,samples=100] ({\x},{(5/8)*(1-\x*\x*\x*\x)})
	-- (1.45,0);
\node at (0,{5/8}) [above left] {$a\mathstrut$};
\draw (-1,-0.02) -- (-1,0.02);
\draw (1,-0.02) -- (1,0.02);
\node at (-1,0) [below] {$-1\mathstrut$};
\node at (0,0) [below] {$0\mathstrut$};
\node at (1,0) [below] {$1\mathstrut$};

\begin{scope}[yshift=-1.3cm]

\draw[->] (-1.5,0) -- (1.5,0) coordinate[label={$x$}];
\draw[->] (0,-0.02) -- (0,1.1) coordinate[label={right:$F_X(x)$}];
\draw[line width=0.3pt] (-1.5,1) -- (1.5,1);
\draw[color=darkgreen,line width=1.4pt]
	(-1.45,0)
	-- plot[domain=-1:1,samples=100]
		({\x},{(5/8)*\x*(1-\x*\x*\x*\x/5)+0.5})
	-- (1.45,1);
\draw (-1,-0.02) -- (-1,0.02);
\draw (1,-0.02) -- (1,0.02);
\node at (-1,0) [below] {$-1\mathstrut$};
\node at (1,0) [below] {$1\mathstrut$};
\node at (0,0) [below] {$0\mathstrut$};
\draw (-0.02,1) -- (0.02,1);
\node at (0,1) [below left] {$1$};
\end{scope}
\end{tikzpicture}
\caption{Wahrscheinlichkeitsdichte und Verteilungsfunktion in 
Aufgabe~\ref{50000062}.
\label{50000062:fig}}
\end{figure}
\begin{teilaufgaben}
\item
Der Wert von $a$ muss so gewählt werden, dass die Normierungsbedingung
\[
1
=
\int_{-\infty}^\infty \varphi(x)\,dx
\]
erfüllt ist.
Die Rechnung ergibt
\begin{align*}
1
&=
\int_{-\infty}^\infty \varphi(x) \,dx
=
a
\int_{-1}^{1} 1-x^4 \,dx
=
a
\biggl[
x-\frac15x^5
\biggr]_{-1}^{1}
\\
&=
a\biggl(
1-\frac15-(-1)+\frac15(-1)^5
\biggr)
=
a\frac{8}{5}
\\
\Rightarrow
\qquad
a
&=
\frac{5}{8} = 0.625.
\end{align*}
\item
Für das Intervall $[-1,1]$ muss die Verteilungsfunktion eine Stammfunktion
von $\varphi(x)$ sein, die an der Stelle $x=-1$ verschwindet:
\begin{align*}
F_X(x)
&=
\int_{-\infty}^x \varphi(\xi)\,d\xi
=
a
\int_{-1}^x 1-\xi^4\,d\xi
=
\frac{5}{8}
\biggl[
\xi-\frac15\xi^5
\biggr]_{-1}^x
=
\frac{5x-x^5}{8} + \frac{1}{2}.
\end{align*}
Ausserhalb des Intervalls ist $F_X$ konstant und zwar $F_X(x)=0$ für $x<-1$
und $F_X(x) = 1$ für $x > 1$.
\item
Da die Funktion $\varphi(x)$ gerade ist, folgt $E(X)=0$.
\item
Zur Berechnung der Varianz muss $E(X^2)$ bestimmt werden:
\begin{align*}
E(X^2)
&=
\int_{-\infty}^\infty x^2\varphi(x)\,dx
=
\int_{-1}^{1} ax^2(1-x^4)\,dx
=
a\biggl[
\frac{x^3}{3}
-
\frac{x^7}{7}
\biggr]_{-1}^1
=
\frac58
\biggl(
\frac13-\frac17+\frac13-\frac17
\biggr)
\\
&=
\frac54\cdot \frac{7-3}{21}
=
\frac{5}{21},
\\
\Rightarrow\qquad
\operatorname{var}(X)
&=
E(X^2) - E(X)^2
=
\frac{5}{21}
\approx
0.238095.
\qedhere
\end{align*}
\end{teilaufgaben}
\end{loesung}

\begin{bewertung}
Normierungsbedingung ({\bf N}) 1 Punkt,
Konstante $a$ ({\bf A}) 1 Punkt,
Verteilungsfunktion ({\bf F}) 1 Punkt,
Erwartungswert ({\bf E}) 1 Punkt,
Varianzformel ({\bf V}) 1 Punkt,
Erwartungswert $E(X^2)$ und Varianz ({\bf E2}) 1 Punkt.
\end{bewertung}
