Jedes Jahr werden 3\% der Schweizer Wohnbevölkerung für eine Befragung
im Rahmen der Volks\-zählung ausgewählt.
Wie gross ist die Wahrscheinlichkeit, dass die Anzahl der ausgewählten
aus Rapperswil (Bevölkerung 26034) um mehr als 30 von der erwarteten
Anzahl abweicht?

\thema{Normalverteilung}
\thema{Standardisierung}

\begin{loesung}
Sie $X$ die Anzahl der ausgewählten Bewohner von Rapperswil.
$X$ ist binomialverteilt mit $p=0.03$ und $n=26034$.
Die erwartete Anzahl ausgewählter Personen ist $\mu X=np=781$.
Gesucht ist die Wahrscheinlichkeit
\begin{equation}
P(X < \mu - 30 \vee X > \mu + 30)
=
1-P(\mu - 30 \le X \le \mu + 30).
\label{50000036:gesucht}
\end{equation}
Wir verwenden die Normalapproximation für $X$ und wenden daher
Standardisierung mit $\sigma=\sqrt{np(1-p)}=27.52$ auf
die Wahrscheinlichkeit in der rechten Seite von
\eqref{50000036:gesucht}
an:
\begin{align*}
P(\mu - 30 \le X \le \mu + 30)
&=
P\biggl(
\frac{-30}{\sigma} \le \frac{X-\mu}{\sigma} \le \frac{30}{\sigma})
\biggr)
\\
&=
P(-1.0899 \le Z \le 1.0899)
=
2F(1.0899)-1
=
2\cdot 0.8621 - 1
=
0.7242.
\end{align*}
Die gesuchte Wahrscheinlichkeit ist daher $1-0.7242= 0.2758$.
\end{loesung}

\begin{bewertung}
Normalverteilung ({\bf N}) 1 Punkt,
Standardisierung ({\bf S}) 1 Punkt,
Werte für $\mu$ ({\bf M}) und $\sigma$ ({\bf G}) je 1 Punkt,
Wahrscheinlichkeit $P(\mu-30\le X\le \mu+30)$ ({\bf W}) 1 Punkt,
Wahrscheinlichkeit für grössere Abweichung ({\bf R}) 1 Punkt.
\end{bewertung}
