Jeden Tag finden weltweit über 1 Milliarde Kreditkartentransaktionen
statt.
2020 wurden mit Karten der grossen Anbieter 468 Milliarden 
Transaktionen durchgeführt, davon entfielen 40\% auf Visa-Karten.
Im Vergleich zu diesen riesigen Transaktionszahlen sind betrügerische
Transaktionen sehr selten, richten aber dennoch beträchtlichen
finanziellen Schaden an, er erreicht etwa 1\% des gesamten
Zahlungsvolumens.
2020 wurden 459297 Fälle von Kreditkartenbetrug gemeldet.
Mit modernen Authentisierungsverfahren kann die Häufigkeit
betrügerischer Transaktionen stark gesenkt werden.
\begin{teilaufgaben}
\item 
Ein international tätiger Online-Händler will die Plausibilität
seiner Kreditkartentransaktionen etwas überwachen und zählt daher die 
Zahl der Transaktionen, die eine Visa-Karte verwenden.
Wenn anteilsmässig zu viele Transaktionen eine Visa-Karte verwenden,
dann liegt der Verdacht nahe, dass ein Betrüger versucht, mit einem
Batch gestohlener Visa-Karten in vielen kleinen Transaktionen Produkte
zu ergaunern.
Wie wahrscheinlich ist es, in einer Stichprobe von 1000 Transaktionen
mehr als 430 Visa-Transaktion zu finden?
\item
Unter der Annahme, dass die Wahrscheinlichkeit einer betrügerischen
Transaktion 0.1\% ist, wie wahrscheinlich ist es, dass der Händler
in einer Stichprobe von 1000 Transaktionen mehr als 3 betrügerische
findet?
\end{teilaufgaben}

\begin{loesung}
\begin{teilaufgaben}
\item
Sei $X$ die Anzahl der Visa-Transaktionen in einer Stichprobe.
$X$ ist Binomialverteilt mit $p=0.4$ und $n=1000$.
Die gesuchte Wahrscheinlichkeit ist $P(X> 430)=1-P(X \le 430)$, sie kann mit
Hilfe der Normalapproximation bestimmt werden.
Sei $Y$ eine normalverteilte Zufallsvariable mit $\mu=np=400$ und 
Varianz $\sigma^2=np(1-p)=240$.
\begin{align*}
P(X\le  430)
&\approx
P\biggl(Y\le 430+\frac12\biggr)
\intertext{Standardisierung führt auf die standardnormalverteilte
Zufallsvariable $Z=(Y-\mu)/\sigma$ mit der Wahrscheinlichkeit}
P\biggl(\frac{Y-\mu}{\sigma}\le \frac{430+\frac12}{\sigma}\biggr)
&=
P\biggl(Z\le \frac{430.5 - 400}{\sqrt{240}}\biggr)
=
P(Z\le 1.9688)
=
0.97551.
\end{align*}
Die gesuchte Wahrscheinlichkeit ist daher
$P(X\le 431)\approx 0.02449$.
\item
Sei $B$ die Zahl der Betrugsfälle.
Da diese selten sind, darf man annehmen, dass $B$ Poisson-verteilt
ist mit $\lambda = 0.001 \cdot 1000 = 1$.
Die Wahrscheinlichkeit, mehr als 3 betrügerische Transaktionen
zu haben, ist daher
\begin{align*}
P(B>3)
&=
1-P(B\le 3)
\\
&=
1-\sum_{k=0}^3 P_\lambda(k)
\\
&=
1- e^{-\lambda} \sum_{k=0}^3 \frac{\lambda^k}{k!}
\\
&=
a-e^{-1}\biggl(
1 + \frac{\lambda}{1} + \frac{\lambda^2}{2!} + \frac{\lambda^3}{3!}
\biggr)
\\
&=
1-e^{-1}\biggl(1+1+\frac12+\frac16\biggr)
=
1-e^{-1}\cdot \frac{16}{6} \approx 0.018988
\qedhere
\end{align*}
\end{teilaufgaben}
\end{loesung}

\begin{bewertung}
\begin{teilaufgaben}
\item
Binomialverteilung ({\bf B}) 1~Punkt,
Erwartungswert und Varianz ({\bf E}) 1~Punkt,
Standardisierung ({\bf S}) 1~Punkt,
Wahrscheinlichkeit ({\bf W}) 1~Punkt,
\item
Poisson-Verteilung und Wert des Parameters ({\bf P}) 1~Punkt,
Resultierende Wahrscheinlichkeit ({\bf R}) 1~Punkt.
\end{teilaufgaben}
\end{bewertung}
