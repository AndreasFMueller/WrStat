Die Exponentialverteilung hat die Verteilungsfunktion
\[
F(x)
=
\begin{cases}
0&\qquad\text{für $x\le 0$}\\
1-e^{-ax}&\qquad\text{für $x>0$.}
\end{cases}
\qquad\qquad
\raisebox{-1cm}{\begin{tikzpicture}
\node at (0,0) {\qrcode[height=2cm]{https://www.random.org/decimal-fractions/?num=10&dec=10&col=1&format=html&rnd=new}};
\end{tikzpicture}}
\]
\begin{teilaufgaben}
\item
Bestimmen Sie die Umkehrfunktion $F^{-1}(y)$ für $y>0$.
\item
Erzeugen Sie einige gleichverteilte Zufallszahlen $y$ mit dem
Zufallszahlgenerator im QR-Code und zeichnen Sie die zugehörigen
$F^{-1}(y)$-Werte für $a=1$ auf.
Es sollte ziemlich grosse Werte dabei haben, aber die meisten Werte
sind nahe bei $0$.
Die $F^{-1}(y)$ sind nicht mehr gleichverteilt.
\end{teilaufgaben}

%\begin{center}
%\qrcode[height=2cm]{https://www.random.org/decimal-fractions/?num=10&dec=10&col=1&format=html&rnd=new}
%\end{center}

\begin{loesung}
\begin{teilaufgaben}
\item
Aus $y=1-e^{-ax}$ erhalten wir durch Auflösen nach der Exponentialfunktion
\begin{align*}
e^{-ax}&=1-y
\intertext{und durch Anwendung des natürlichen Logarithmus}
-ax &= \log(1-y)\\
\Rightarrow\qquad
x &=F^{-1}(y) = -\frac{1}{a}\log(1-y).
\end{align*}
\item
Der QR-Code liefert die $y$-Werte mit den zugehörigen $F^{-1}(y)$-Werten
\begin{center}
\begin{tabular}{|>{$}r<{$}|>{$}r<{$}|>{$}r<{$}|}
\hline
y&1-y&-\log(1-y) \\
\hline
0.5746481160 & 0.4253518840 & 0.854838490276530 \\
0.1418307476 & 0.8581692524 & \color{darkred}0.152953935078669 \\
0.9087866743 & 0.0912133257 & \color{darkgreen}2.394554277444048 \\
0.3912439839 & 0.6087560161 & \color{darkred}0.496337721923698 \\
0.7665202854 & 0.2334797146 & 1.454660081021801 \\
0.2433549105 & 0.7566450895 & \color{darkred}0.278860973660684 \\
0.3535949332 & 0.6464050668 & \color{darkred}0.436328933315873 \\
0.1437008172 & 0.8562991828 & \color{darkred}0.155135451288878 \\
0.1139949198 & 0.8860050802 & \color{darkred}0.121032594533449 \\
0.3845952556 & 0.6154047444 & \color{darkred}0.485475106666649 \\
\hline
\end{tabular}
\end{center}
Rot: Zahlenwerte $-\log(1-y)<0.5$, Grün: Zahlenwerte $-\log(1-y)>2$.
\qedhere
\end{teilaufgaben}
\end{loesung}
