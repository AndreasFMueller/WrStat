Betrachten Sie die Funktion
\[
\varphi(x)=\begin{cases}
0&\qquad x < 1\\
\frac1x&\qquad 1\le x\le a\\
0&\qquad x > a
\end{cases}
\]
\begin{teilaufgaben}
\item Wie muss $a$ gew"ahlt werden, damit $\varphi$ eine Dichtefunktion
ist?
\item Wie lautet die zugeh"orige Verteilungsfunktion $F$?
\item Sei $X$ eine Zufallsvariable mit Verteilungsfunktion $F$, berechnen Sie
$E(X)$ und $\operatorname{var}(X)$.
\item Wie wahrscheinlich ist das Ereignis $\{X>2\}$?
\end{teilaufgaben}

\begin{loesung}
\begin{teilaufgaben}
\item
$a$ muss so gew"ahlt werden, dass
$\int_{-\infty}^\infty  \varphi(x)\,dx=1$.
\begin{align*}
1&=
\int_{-\infty}^{\infty}\varphi(x)\,dx
=
\int_1^a \frac1x\,dx
=
\left[
\log x
\right]_1^a
\\
\log a&=1\qquad \Rightarrow \qquad a=e
\end{align*}
\item
Die Verteilungsfunktion ist die Stammfunktion von $\varphi$. Interessant ist
nur der Teil zwischen $1$ und $a$
\begin{align*}
F(x)&=\int_1^x \varphi(\xi)\,d\xi=\int_1^x\frac1{\xi}\,d\xi=\left[\log\xi\right]_1^x=\log x
\end{align*}
Die Verteilungsfunktion ist also
\begin{align*}
F(x)=\begin{cases}
0&\qquad x < 1\\
\log x&\qquad 1\le x\le e\\
1&\qquad x \ge e
\end{cases}
\end{align*}
\item
Mit der Dichtefunktion kann auch Erwartungswert und Varianz berechnet werden:
\begin{align*}
E(X)
&=
\int_{-\infty}^\infty \xi\varphi(\xi)\,d\xi
=
\int_1^e\xi\frac1\xi\,d\xi
=
\int_1^ed\xi\\
&=e-1\simeq 1.71828
\\
E(X^2)
&=
\int_{-\infty}^\infty \xi^2\varphi(\xi)\,d\xi
=
\int_1^e\xi^2\frac1\xi\,d\xi
=
\int_1^e\xi\,d\xi=\left[\frac12\xi^2\right]_1^e\\
&=\frac{e^2 -1}2
\simeq 3.194528
\\
\operatorname{var}(X)
&=
E(X^2)-E(X)^2=\frac{e^2-1}2-(e-1)^2
=
(e-1)\left(
\frac{e+1}2-(e-1)
\right)
\\
&=
\frac{(e-1)(3-e)}2
\simeq
0.2420356
\end{align*}
\item
Die Wahrscheinlichkeit ist
\begin{align*}
P(X>2)
&=
\int_2^\infty\varphi(\xi)\,d\xi =\int_2^e\frac1\xi\,d\xi=\left[\log\xi\right]_2^e
=1-\log 2=0.3068528
\qedhere
\end{align*}
\end{teilaufgaben}
\end{loesung}

