Betrachten Sie die Funktion
\[
\varphi(x)=\begin{cases}
0&\qquad x < 0\\
\frac23x&\qquad 0\le x< 1\\
\frac23&\qquad 1\le x< 2\\
0&\qquad 2\le x
\end{cases}
\]
\begin{teilaufgaben}
\item Kann $\varphi$ die Wahrscheinlichkeitsdichte einer Zufallsvariablen
$X$ sein?
\item Wenn ja: Welchen Erwartungswert hat $X$?
\item Welche Varianz hat $X$?
\end{teilaufgaben}

\begin{loesung}
\begin{teilaufgaben}
\item Die Funktion ist $\varphi(x)\ge 0$, sie k"onnte also durchaus
eine Wahrscheinlichkeitsdichte sein, vorausgesetzt ihr Integral "uber
$\mathbb R$ ist $1$.
\begin{align*}
\int_{-\infty}^{\infty}\varphi(x)\,dx
&=
\int_0^1\frac23x\,dx
+\int_1^2\frac23\,dx
=\frac23\left[
\frac12x^2
\right]_0^1
+\frac23
=\frac13+\frac23=1.
\end{align*}
Also ist $\varphi(x)$ eine Wahrscheinlichkeitsdichte.
\item F"ur den Erwartungswert gilt:
\begin{align*}
E(X)
&=\int_{-\infty}^{\infty}x\,\varphi(x)\,dx
=
\int_0^1 x\cdot \frac23x\,dx + \int_1^2x\cdot \frac23\,dx
\\
&=
\frac23\left[ \frac13x^3 \right]_0^1 + \frac23\left[ \frac12x^2 \right]_1^2
=\frac29 + \frac23\cdot\frac{2^2-1}2
=\frac29 + 1=\frac{11}9.
\end{align*}
\item F"ur die Varianz muss man auch noch den quadratischen Mittelwert
bestimmen:
\begin{align*}
E(X^2)
&=
\int_{-\infty}^{\infty}x^2\,\varphi(x)\,dx
=
\int_0^1 x^2\cdot \frac23x\,dx + \int_1^2x^2\cdot \frac23\,dx
\\
&=
\frac23 \left[ \frac14 x^4 \right]_0^1
+\frac23 \left[ \frac13 x^3 \right]_1^2
=\frac23\cdot\frac14+\frac23\cdot\frac{8-1}3
=\frac16+\frac{14}9=\frac3{18}+\frac{28}{18}
\\
&=\frac{31}{18}
\\
\operatorname{var}(X)
&=
E(X^2)-E(X)^2
=\frac{31}{18}-\frac{121}{81}
=\frac{37}{162}
=0.228395.
\qedhere
\end{align*}
\end{teilaufgaben}
\end{loesung}

\begin{bewertung}
a) Normierungskriterium ({\bf N}) 1 Punkt,
b) Formel f"ur Erwartungswert ({\bf F1}) 1 Punkt, Berechung ({\bf E}) 1 Punkt.
c) Formel f"ur Varianz ({\bf F2}) 1 Punkt, Berechnung der Integrale ({\bf I})
1 Punkt, Berechnung der Varianz ({\bf V}) 1 Punkt.
\end{bewertung}
