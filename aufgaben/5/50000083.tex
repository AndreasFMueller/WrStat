Ein Atom des radioaktiven Caesium-Isotops $\prescript{137}{}{\text{Cs}}$ 
zerfällt mit einer durchschnittlichen Lebensdauer von 43.34 Jahren.
\begin{teilaufgaben}
\item
Bestimmen Sie den Parameter $a$ der zugehörigen Exponentialverteilung.
\item
Verwenden Sie die Verteilungsfunktion der Exponentialverteilung zum die
Wahrsscheinlichkeit zu berechnen, dass das Atom nach 30.04 Jahren
zerfallen ist.
\end{teilaufgaben}

\begin{loesung}
\begin{teilaufgaben}
\item
$a$ ist der Kehrwert der mittleren Lebensdauer, also
$a=1/43.34\,\text{yr}^{-1}$.
\item
Die Verteilungsfunktion ist
\[
F(30.04\,\text{yr})
=
1-e^{-a\cdot 30.04\,\text{yr}}
=
1-e^{-30.04/43.34}
=
1-e^{-0.69312}
=
0.499986
\approx
0.5.
\qedhere
\]
\end{teilaufgaben}
\end{loesung}
