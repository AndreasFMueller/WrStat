Finden Sie die die Wahrscheinlichkeitsdichte der Gleichverteilung
auf dem Intervall $[0,1]$, deren Verteilungsfunktion durch
\[
F(x)
=
\begin{cases}
0&\quad \text{für}\;\phantom{0<\mathstrut}x \le 0\\
x&\quad \text{für}\;0         < x \le 1\\
1&\quad \text{für}\;1         < x
\end{cases}
\qquad
\qquad
\raisebox{-1.5cm}{
\begin{tikzpicture}[>=latex,thick]
\draw[->] (-1.1,0) -- (4.5,0) coordinate[label={$x$}];
\draw[->] (0,-0.1) -- (0,2.3) coordinate[label={right:$F(x)$}];
\draw[line width=0.2pt] (-1.1,2) -- (3.3,2);
\draw[line width=0.2pt] (2,0) -- (2,2);
\draw (2,-0.05) -- (2,0.05);
\draw (-0.05,2) -- (0.05,2);
\node at (2,0) [below] {$1$};
\node at (0,0) [below] {$0$};
\node at (0,2) [below left] {$1$};
\draw[color=darkgreen,line width=1.4pt] (-1,0) -- (0,0) -- (2,2) -- (4,2);
\end{tikzpicture}}
\]
gegeben ist.
Zeichnen sie die zugehörige Wahrscheinlichkeitsdichte.

\begin{loesung}
Die Dichtefunktion ist die Ableitung, also die stückweise
konstante Funktion
\[
\varphi(x)
=
\begin{cases}
0&\qquad \phantom{0<\mathstrut}x \le 0\\
1&\qquad 0         < x \le 1\\
0&\qquad 1         < x
\end{cases}
\]
Die folgende Abbildung zeigt den Graphen, sie hat das Intervall
$[0,1]$ als Träger:
\[
\definecolor{darkred}{rgb}{0.8,0,0}
\raisebox{-0.5cm}{
\begin{tikzpicture}[>=latex,thick]
\draw[->] (-1.1,0) -- (4.5,0) coordinate[label={$x$}];
\draw[->] (0,-0.1) -- (0,2.3) coordinate[label={left:$F(x)$}];
\draw (2,-0.05) -- (2,0.05);
\draw (-0.05,2) -- (0.05,2);
\node at (2,0) [below] {$1$};
\node at (0,0) [below] {$0$};
\fill[color=darkred!10] (0,0) rectangle (2,2);
\draw[color=darkred,line width=1.4pt] (-1,0) -- (0,0);
\draw[color=darkred,line width=0.1pt] (0,0) -- (0,2);
\draw[color=darkred,line width=1.4pt] (0,2) -- (2,2);
\draw[color=darkred,line width=0.1pt] (2,2) -- (2,0);
\draw[color=darkred,line width=1.4pt] (2,0) -- (4,0);
\end{tikzpicture}}
\qedhere
\]
\end{loesung}
