Betrachten Sie die Funktion
\[
\varphi(x)
=
\begin{cases}
(x+1)^2&\qquad -1\le x\le 0\\
1-x^2&\qquad 0\le x\le 1\\
0&\qquad\text{sonst.}
\end{cases}
\]

\begin{figure}[ht]
\centering
\begin{tikzpicture}[>=latex,thick,scale=2.5]
\pgfmathparse{sqrt(5/36)}
\xdef\varianz{\pgfmathresult}
\ifthenelse{\boolean{loesungen}}{
\fill[color=darkgreen!20] (0,0) rectangle (1,1);
\fill[color=red!20] 
	plot[domain=-1:0,samples=20] (\x,{(\x+1)*(\x+1)})
	--
	plot[domain=0:1,samples=20] (\x,{1-\x*\x});
\draw[color=blue] ({1/6},-0.05) -- ({1/6},1.05);
\node[color=blue] at ({1/6+0.05},1.00) [above] {$E(X)$};
\draw[color=blue] ({1/6-\varianz},-0.05) -- ({1/6-\varianz},1.05);
\draw[color=blue] ({1/6+\varianz},-0.05) -- ({1/6+\varianz},1.05);
}{}

\draw[->] (0,-0.05) -- (0,1.4) coordinate[label={left:$\varphi(x)$}];
\draw[->] (-2.05,0) -- (2.2,0) coordinate[label={$x$}];
\foreach \x in {-2,-1,1,2}{
	\draw (\x,-0.05) -- (\x,0.05);
	\node at (\x,-0.05) [below] {$\x\mathstrut$};
}
\draw[color=red,line width=1.4pt] 
	(-2,0)
	--
	plot[domain=-1:0,samples=20] (\x,{(\x+1)*(\x+1)})
	--
	plot[domain=0:1,samples=20] (\x,{1-\x*\x})
	--
	(2,0);
\end{tikzpicture}
\caption{Graph der Funktion $\varphi(x)$ in Aufgabe~\ref{50000051}
\label{50000051:fig}}
\end{figure}

\begin{teilaufgaben}
\item
Warum ist dies eine Wahrscheinlichkeitsdichte?
\item
Bestimmen Sie den Erwartungswert $E(X)$ einer Zufallsvariable $X$ mit
der Wahrscheinlichkeitsdichte $\varphi(x)$.
\item
Bestimmen Sie $\operatorname{var}(X)$.
\end{teilaufgaben}

\begin{loesung}
\begin{teilaufgaben}
\item
Man muss überprüfen, dass die Normierungsbedingung
\[
\int_{-\infty}^\infty \varphi(x)\, dx = 1
\]
erfüllt ist.
Die rote Fläche unter dem Graphen von $\varphi(x)$ kann in das flächengleiche
Einheitsquadrat umgewandelt werden, indem der Teil der Fläche zwischen
$x=-1$ und $x=0$ vertikal gespiegelt und als die grüne Fläche
zwischen $x=0$ und $x=1$ wieder eingesetzt wird.
\item
Der Erwartungswert ist
\begin{align*}
E(X)
&=
\int_{-\infty}^\infty x\varphi(x)\,dx
=
\int_{-1}^0 x(x+1)^2\,dx
+
\int_0^1 x(1-x^2)\,dx
\\
&=
\int_{-1}^0 x^3+2x^2+x \,dx
+
\int_0^1 x-x^3\,dx
\\
&=
\left[
\frac{x^4}{4} + \frac{2x^3}{3} + \frac{x^2}2
\right]_{-1}^0
+
\left[
\frac{x^2}{2} - \frac{x^4}{4}
\right]_0^1
\\
&=
-\frac14+\frac23-\frac12+\frac12-\frac14
=
\frac16.
\end{align*}
\item
Die Varianz kann mit der Formel $\operatorname{var}(X)=E(X^2)-E(X)^2$
bestimmt werden.
Dazu rechnen wir
\begin{align*}
E(X^2)
&=
\int_{-\infty}^\infty x^2\varphi(x)\,dx
=
\int_{-1}^0 x^2(x+1)^2\,dx
+
\int_0^1 x^2(1-x^2)\,dx
\\
&=
\int_{-1}^0 x^4+2x^3 + x^2\,dx
+
\int_0^1 x^2-x^4\,dx
\\
&=
\left[
\frac{x^5}{5} + \frac{2x^4}{4} + \frac{x^3}{3}
\right]_{-1}^0
+
\left[
\frac{x^3}{3}-\frac{x^5}{5}
\right]_0^1
\\
&=
\frac15-\frac12+\frac13+\frac13-\frac15
=\frac16
\\
\operatorname{var}(X)
&=
E(X^2) - E(X)^2
=
\frac16-\frac{1}{36}
=
\frac{5}{36}
\approx
0.13888.
\end{align*}
\end{teilaufgaben}
Der Erwartunswert wie auch das Interval zwischen
$E(X) - \sqrt{\operatorname{var}(X)}$
und
$E(X) + \sqrt{\operatorname{var}(X)}$
sind in Abbildung~\ref{50000051:fig} blau eingezeichnet.
\end{loesung}

\begin{bewertung}
Normierungsbedingung ({\bf N}) 1 Punkt,
Integral für Erwartungswert ({\bf I}) 1 Punkt,
Wert des Erwartungswertes ({\bf E}) 1 Punkt,
Formel für die Varianz ({\bf V}) 1 Punkt,
Erwartungswert von $X^2$ ({\bf E}2) 1 Punkt,
Wert der Varianz ({\bf W}) 1 Punkt.
\end{bewertung}
