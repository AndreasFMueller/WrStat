Welche der folgenden Funktionen k"onnen keine Verteilungsfunktionen sein
und warum?
\begin{teilaufgaben}
\item $F_1(x)=\tanh x$
\item $F_2(x)=1+\tanh x$
\item $F_3(x)=\frac2{\pi}(1+\arctan x)$
\item $F_4(x)=\begin{cases}
0&\qquad x\le -\frac{\pi}2\\
\frac12(1+\sin x)&\qquad-\frac{\pi}2\le x\le \frac{\pi}2\\
1&\qquad x\ge \frac{\pi}2
\end{cases}$
\item $F_5(x)=\frac1{2\pi\sigma^2}e^{-\frac{(x-\mu)^2}{2\sigma^2}}$
\item $F_6(x)=\begin{cases}0&\qquad x<0\\1-e^{-x}&\qquad x\ge 0\end{cases}$
\end{teilaufgaben}

\begin{loesung}
Eine Verteilungsfunktion $F(x)$ ist monoton und erf"ullt
$\lim_{x\to-\infty}F(x)=0$
und
$\lim_{x\to\infty}F(x)=1$. Diese Eigenschaften sind f"ur die
Funktionen $F_1$ bis $F_6$ zu testen.
\begin{teilaufgaben}
\item Da $F_1(x)=\tanh x<0$ f"ur $x<0$ kann $F_1$ keine Verteilungsfunktion sein.
\item Wegen $\lim_{x\to\infty}F_2(x)=2>1$ kann $F_2$ keine Verteilungsfunktion sein.
\item $F_3$ wird negativ f"ur $x$ gen"ugend klein, denn wenn der Winkel
$\alpha$ gegen $-\frac{\pi}2$ strebt, strebt $\tan\alpha$ gegen $-\infty$.
Eine Verteilungsfunktion darf aber keine negativen Werte annehmen.
\item $F_4$ erf"ullt alle Bedingungen einer Verteilungsfunktion.
\item Die Funktion $F_5$ ist keine Verteilungsfunktion, weil
\[
\lim_{x\to-\infty}F_5(x)=\lim_{x\to\infty}F_5(x)=0.
\]
$F_5$ sieht auf den ersten
Blick aus wie die Wahrscheinlichkeitsdichte der Normalverteilung, doch ist
der Normierungsfaktor falsch, er m"usste $\frac1{\sqrt{2\pi\sigma^2}}$ lauten.
Doch "andert dies nat"urlich auch nichts daran, dass $F_5$ nicht monoton und damit
keine Verteilungsfunktion ist.
\item Dies ist die Verteilungsfunktion der Exponentialverteilung mit Parameter $a=1$.
\qedhere
\end{teilaufgaben}
\end{loesung}

