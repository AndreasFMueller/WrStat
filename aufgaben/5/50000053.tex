In grossen Teilen der Welt beschenken sich die Menschen zur Zeit
der Wintersonnenwende gegenseitig, ein Brauch, der auf das römische
Fest der Saturnalien zurückgeht.
In der westlichen Welt machen sich Eltern zusätzlich einen Spass 
daraus, den Kindern zu erzählen, die Geschenke würden von einem
übergewichtigen älteren Herrn namens Santa gebracht, der zu diesem Zweck
unter Missachtung aller Vorschriften zur Arbeitsplatzsicherheit
in völlig ungeigneter Kleidung durch Kamine weltweit in Häuser eindringt.
Gerade in Zeiten einer Pandemie entsteht für Paketlieferdienste
ein grosser zusätzlicher Aufwand durch die vielen Geschenkpakete.
Viele Leute senden ihre Geschenkpakete rechtzeitig, 25\% der Pakete
kommen schon am 18.~Dezember an.
Für einige wird es aber knapp, 5\% der Pakete kommen erst nach dem
28.~Dezember an.
\begin{teilaufgaben}
\item
An welchem Tag ist Peak Santa, d.~h.~an welchem Tag werden
am meisten Pakete ausgeliefert?
\item
Wie gross ist die Wahrscheinlichkeit, dass ein Paket erst im neuen
Jahr ausgeliefert wird?
\end{teilaufgaben}

\begin{hinweis}
Nehmen sie an, dass die Paketverteilung stetig ist, wie wenn die
Paketedienste ohne freie Tage rund um die Uhr arbeiten würden.
\end{hinweis}

\begin{loesung}
Wir nehmen an, dass der Auslieferungszeitpunkt $X$ eines Paketes
eine normalverteilte Zufallsvariable mit Erwartungswert $\mu$ und
Standardabweichung $\sigma$ ist.
Die beiden Bedingungen besagen
\[
\begin{aligned}
P(X<18)&=0.25&&            &         &       \\
P(X>28)&=0.05 &&\Rightarrow& P(X<28) &= 0.95
\end{aligned}
\]
oder nach Standardisierung
\[
\begin{aligned}
P\biggl(\frac{X-\mu}{\sigma} < \frac{18-\mu}{\sigma}\biggr) &= 0.25 \\
P\biggl(\frac{X-\mu}{\sigma} < \frac{28-\mu}{\sigma}\biggr) &= 0.95.\\
\end{aligned}
\]
Da $Z=(X-\mu)/\sigma$ standardnormalverteilt ist, folgt
\[
\left.
\begin{aligned}
\frac{18-\mu}{\sigma} &=          - 0.6745 \\
\frac{28-\mu}{\sigma} &= \phantom{-}1.6449 
\end{aligned}
\;
\right\}
\qquad\Rightarrow\qquad
\left\{\;
\begin{linsys}{3}
18 &=& \mu &-& 0.6745\sigma\phantom{.}\\
28 &=& \mu &+& 1.6449\sigma.
\end{linsys}
\right.
\]
Dieses lineare Gleichungssystem kann zum Beispiel mit einer der vielen
Methoden gelöst werden, man findet
\begin{align*}
\mu &= 20.9081 \\
\sigma &= \phantom{0}4.3115.
\end{align*}
Damit kann man jetzt die gestellten Fragen beantworten.
\begin{teilaufgaben}
\item
Peak Santa ist $\mu$, also am 20.~Dezember.
\item
Die Wahrscheinlichkeit, dass ein Paket erst im neuen Jahr ausgeliefert
wird, ist
\[
P(X>32)
=
1-P\biggl(Z<\frac{32-\mu}{\sigma}\biggr)
=
1-P(Z < 2.5727) 
=
0.00504,
\]
also etwa 0.5\% der Pakete kommen erst im neuen Jahr an.
\qedhere
\end{teilaufgaben}
\end{loesung}

\begin{bewertung}
Normalverteilung ({\bf N}) 1 Punkt,
Standardisierung ({\bf S}) 1 Punkt,
Quantilen ({\bf Q}) 1 Punkt,
Gleichungssystem für $\mu$ und $\sigma$ ({\bf G}) 1 Punkt,
Wert für $\mu$ und Peak Santa ({\bf P}) 1 Punkt,
Wahrscheinlichkeit für Januarlieferung ({\bf J}) 1 Punkt.
\end{bewertung}

