Bei der Abstimmung zur Minarett-Initiative fiel eine grosse Abweichung zwischen
den Vorhersagen und dem tatsächlichen Resultat auf. Um abzuschätzen,
wie Wahrscheinlich eine solche Abweichung ist, gehen wir von folgenden
Annahmen aus. Wir nehmen an, dass 55\% der Beölkerung der Initative
zustimmen. Jetzt werden zufällig 1000 Leute befragt, und die Anzahl der
Ja-Stimmen ermittelt.
\begin{teilaufgaben}
\item
Wie gross ist die Wahrscheinlichkeit, dass die
gewählte Stichprobe die Initiative ablehnt, das Resultat der
Meinungsumfrage sich also am Abstimmungstag als falsch erweist?
\item Wie gross müsste der Anteil der Beführworter in der Bevölkerung
sein, damit die Wahrscheinlichkeit für eine Fehlprognose 10\% ist?
\end{teilaufgaben}

\thema{Binomialverteilung}
\thema{Normalapproximation}
\thema{Standardisierung}

\begin{loesung}
Offenbar stimmt ein zufällig ausgewählter Stimmbürger mit
Wahrscheinlichkeit $p=0.55$ der Initiative zu. Die Zahl $X$ der Zustimmenden
in einer Stichprobe von 1000 zufällig ausgewählten Stimmbürgern ist
binomialverteilt. Wegen der beträchtlichen Grösse der Stichprobe können
wir näherungsweise von einer Normalverteilung mit Mittelwert $\mu=np=550$ und
Standardabweichung $\sigma=\sqrt{np(1-p)}
% sqrt(1000 * 0.55 * 0.45)
=15.732
$.
Damit lassen sich jetzt die Wahrscheinlichkeiten berechnen
\begin{teilaufgaben}
\item
Gesucht ist die Wahrscheinlichkeit, dass die Zahl $X$ kleiner als 500 ist,
\begin{align*}
P(X < 500)&=P\left(\frac{X-\mu}{\sigma}<\frac{500-\mu}{\sigma}\right)\\
        &=P(Z<
%-50/15.732
-3.178
)=0.000741
\end{align*}
Diese Wahrscheinlichkeit ist also so klein, dass man schliessen muss,
dass die Stichprobe nicht wirklich zufällig ausgewählt war.
\item
Offenbar muss $p$ noch wesentlich näher bei $0.5$ liegen.  Wir suchen also
$p>0.5$ so, dass
\begin{align*}
0.1=P(X<500)&=P\left(\frac{X-\mu}{\sigma}<\frac{500-\mu}{\sigma}\right)\\
&=P\left(Z<\frac{n(\frac12-p)}{\sqrt{n(1-p)}}\right)\\
-1.281552=
z&=\frac{n(\frac12-p)}{\sqrt{n(1-p)}}
\end{align*}
Darin ist $p$ die Unbekannte, man kann nach $p$ auflösen
\begin{align*}
z^2np(1-p)&=n^2\biggl(\frac12-p\biggr)^2\\
z^2np-z^2np^2
&=\frac{n^2}4-n^2p+n^2p^2
\\
0&=n(z^2+n)p^2
-n(z^2+n)p
+\frac{n^2}4
\\
p&=\frac{-(z^2+n)\pm\sqrt{(z^2+n)^2-n(z^2+n)n^2}}{2n^2(z^2+n)}
\qedhere
\end{align*}
\end{teilaufgaben}
\end{loesung}

