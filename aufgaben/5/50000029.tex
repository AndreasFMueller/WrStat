Eine Zufallsvariable $X$ ist auf dem Interval $[0,\pi]$ so verteilt, dass die
Wahrscheinlichkeitsdichte
\[
\varphi_X(x)=\begin{cases}
0\qquad&x < 0\\
a(1-\cos x)\qquad&0\le x\le\pi\\
0\qquad&x>\pi
\end{cases}
\]
ist wie im folgenden Graphen.
%\begin{figure}[h]
%\centering
\begin{center}
\includeagraphics[]{graph-1.pdf}
\end{center}
%\caption{Wahrscheinlichkeitsdichte der Verteilung von $X$ in Aufgabe
%\ref{50000029}.
%\label{50000029:phi}}
%\end{figure}
\begin{teilaufgaben}
\item
Wie muss man $a$ wählen, damit diese Funktion tatsächlich eine
Wahrscheinlichkeitsdichte ist?
\item
Berechnen Sie $E(X)$ und $\operatorname{var}(X)$.
\end{teilaufgaben}

\thema{Wahrscheinlichkeitsdichte}
\thema{Verteilungsfunktion}
\thema{Erwartungswert}
\thema{Varianz}

%\begin{hinweis}
%\begin{align*}
%\int x^{\phantom{1}}\cos x\,dx
%&=x\sin x +\cos x
%\\
%\int x^2\cos x\,dx
%&=
%(x^2-2)\sin x+2x\cos x
%\\
%\int x^3\cos x\,dx
%&=
%(x^3-6x)\sin x+(3x^2-6)\cos x
%\end{align*}
%\end{hinweis}

\begin{loesung}
\begin{teilaufgaben}
\item
Die Konstante $a$ muss so gewählt werden, dass das Integral von $\varphi_X$
über $\mathbb R$ den Wert $1$ hat:
\begin{align*}
\int_{-\infty}^{\infty}\varphi_X(x)\,dx
&=
\int_0^\pi a(1-\cos x)\,dx
=
a\left[
(x-\sin x)
\right]_0^\pi
=a\pi= 1
&&\Rightarrow&a&=\frac1{\pi}.
\end{align*}
Dieses Resultat kann man auch ohne Rechnung direkt aus der Graphik
ablesen.
Da der Graph zentralsymmetrisch ist bezüglich des Punktes $(\frac{\pi}2,a)$,
ist der Teil der rosaroten Fläche mit $\varphi_X(x)>a$ kongruent
zu dem fehlenden Teil zwischen der roten Kurve und $a$ über dem
Interval $[0,\frac{\pi}2]$.
Schneidet man den Teil oberhalb $a$ weg und fügt in links ein, wird
die rosarote Fläche zu einem Rechteck mit Seiten $\pi$ und $a$.
Diese Fläche muss $1$ sein, also $a\pi=1\Rightarrow a=\frac1{\pi}$.
\item
Für den Erwartungswert $E(X)$ müssen wir $x\varphi_X(x)$ integrieren.
Dazu brauchen auch die Stammfunktionen von $x\cos x$ und $x^2\cos x$,
welche wir zunächst separat berechnen:
\begin{align*}
\int \underset{\displaystyle\downarrow}{x}\,\underset{\displaystyle\uparrow}{\cos x}\,dx
&=
x\sin x -\int \sin x\,dx
=
x\sin x +\cos x
\\
\int \underset{\displaystyle\downarrow}{x^2}\,\underset{\displaystyle\uparrow}{\cos x}\,dx
&=
x^2\sin x-\int \underset{\displaystyle\downarrow}{2x}\,\underset{\displaystyle\uparrow}{\sin x}\,dx
=
x^2\sin x+2x\cos x-\int 2\cos x\,dx
\\
&=
x^2\sin x+2x\cos x-2\sin x.
\end{align*}
Damit können wir jetzt die Erwartungswerte berechnen:
\begin{align*}
E(X)
&=
\int_{-\infty}^{\infty}x\varphi_X(x)\,dx
=
\int_0^\pi ax(1-\cos x)\,dx
=
a\biggl[\frac{x^2}2
-x\sin x-\cos x
\biggr]_0^\pi
\\
&=
\frac1{\pi}\biggl(
\frac{\pi^2}{2}+2
\biggr)
=
\frac{\pi}{2}+\frac{2}{\pi}=2.2074161
\\
E(X^2)
&=
\int_{-\infty}^{\infty}x^2\varphi_X(x)\,dx
=
\int_0^\pi ax^2(1-\cos x)\,dx
=
a\biggl[\frac{x^3}3
-x^2\sin x-2x\cos x +2\sin x
\biggr]_0^\pi
\\
&=
\frac1{\pi}\biggl(
\frac{\pi^3}{3} +2\pi
\biggr)
=\frac{\pi^2}{3}+2=5.2898681
\end{align*}
Und schliesslich die Varianz
\begin{align*}
\operatorname{var}(X)
&=
E(X^2)-E(X)^2
=
\frac{\pi^2}3+2-\biggl(\frac{\pi}{2}+\frac{2}{\pi}\biggr)^2
=
\frac{\pi^2}3+2-\frac{\pi^2}{4}-2-\frac{4}{\pi^2}
\\
&=
\frac{4\pi^4-3\pi^4-48}{12\pi^2}
=
\frac{\pi^4-48}{12\pi^2}=\frac{\pi^2}{12}-\frac{4}{\pi^2}=0.4171823
\qedhere
\end{align*}
\end{teilaufgaben}
\begin{figure}
\centering
\includeagraphics[]{graph-2.pdf}
\caption{Wahrscheinlichkeitsdichte zu Aufgabe~\ref{50000029} mit
eingezeichnetem Erwartungs und Varianz.
\label{50000029:phievar}}
\end{figure}
\end{loesung}


\begin{bewertung}
Dichtefunktion, Wert von $a$ ({\bf D}) 1 Punkt,
Integralformel für Erwartungswerte ({\bf I}) 1 Punkt,
Erwartungswert $E(X)$ ({\bf E}) 1 Punkt,
Erwartungswert $E(X^2)$ ({\bf E2)} 1 Punkt
Varianzformel $\operatorname{var}(X)=E(X^2)-E(X)^2$ ({\bf F}) 1 Punkt,
Varianz $\operatorname{var}(X)$ ({\bf V}) 1 Punkt.
\end{bewertung}




