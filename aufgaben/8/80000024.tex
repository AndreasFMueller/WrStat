Ein Würfelhersteller möchte einen neuen 32er-Würfel auf den Markt
bringen.
Die Form dazu hat er sich bei einem Hersteller von ``Globen'' abgeschaut: 
\begin{center}
\includeagraphics[width=0.5\hsize]{globus.jpg}
\end{center}
Diese Geschenkartikel werden aus acht Kartonteilen zusammengesteckt, es
entsteht eine Approximation einer Kugel, mit insgesamt 32 Flächen.
Für einen ersten Test zählt er beim ``würfeln'' nur, in welcher Zone
der Würfel landet: Nordpolarregion, Tropen nördlich des "Aquators,
Tropen südlich des "Aquators, Südpolarregion.
Dabei erhält der die folgenden Daten:
\begin{center}
\begin{tabular}{|l|c|}
\hline
Zone       &Anzahl\\
\hline
Nordpol    & 10\\
Tropen nord& 20\\
Tropen süd& 14\\
Südpol    & 6\\
\hline
\end{tabular}
\end{center}
Soll er das Projekt weiter verfolgen?

\thema{Hypothesentest}
\themaL{chi2Test}{$\chi^2$-Test}

\begin{loesung}
Wenn das Projekt Erfolg haben soll, müssen die vier Zonen gleich häufig
vorkommen, wir führen daher einen $\chi^2$-Test auf Gleichverteilung
durch.
Die Nullhypothese ist
\begin{quote}
Alle Zonen sind gleich wahrscheinlich.
\end{quote}
Dazu berechnen wir die Diskrepanz:
\begin{center}
\begin{tabular}{|l|c|c|c|c|}
\hline
Zone       & $n_i$ & $p_i$ & $np_i$ & $(n_i-np_i)^2/np_i$ \\
\hline
Nordpol    &  10   & 0.25  & 12.5   & 0.50 \\
Tropen nord&  20   & 0.25  & 12.5   & 4.50 \\
Tropen süd&  14   & 0.25  & 12.5   & 0.18 \\
Südpol    &   6   & 0.25  & 12.5   & 3.38 \\
\hline
           &$n=50$ &       &        & 8.56 \\
\hline
\end{tabular}
\end{center}
Wir erhalten also als Diskrepanz $D=8.56$.
Für $\alpha$ wählen wir den Wert $\alpha=5\%$.
Der kritische Wert für $\chi^2$ für 3 Freiheitsgrade ist 
$D_{\text{krit}}=7.815$, wird also von der gefundenen Diskrepanz
überschritten.
Die Nullhypothese muss also verworfen werden, auf der Basis dieser
Form ist nicht damit zu rechnen, dass man einen fairen Würfel
erhalten wird, das Projekt hat wenig Aussicht auf Erfolg.
\end{loesung}

\begin{bewertung}
$\chi^2$-Test ({\bf X}) 1 Punkt,
Nullhypothese ({\bf N}) 1 Punkt,
Berechnung der Diskrepanz ({\bf D}) 1 Punkt,
Wahl von $\alpha$ ({\bf A}) 1 Punkt,
Korrekte Zahl der Freiheitsgrade und Bestimmung der kritische Diskrepanz
({\bf K}) 1 Punkt,
Schlussfolgerung ({\bf S}) 1 Punkt.
\end{bewertung}

