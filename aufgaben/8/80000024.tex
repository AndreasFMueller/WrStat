Ein W"urfelhersteller m"ochte einen neuen 32er-W"urfel auf den Markt
bringen.
Die Form dazu hat er sich bei einem Hersteller von ``Globen'' abgeschaut: 
\begin{center}
\includeagraphics[width=0.5\hsize]{globus.jpg}
\end{center}
Diese Geschenkartikel werden aus acht Kartonteilen zusammengesteckt, es
entsteht eine Approximation einer Kugel, mit insgesamt 32 Fl"achen.
F"ur einen ersten Test z"ahlt er beim ``w"urfeln'' nur, in welcher Zone
der W"urfel landet: Nordpolarregion, Tropen n"ordlich des "Aquators,
Tropen s"udlich des "Aquators, S"udpolarregion.
Dabei erh"alt der die folgenden Daten:
\begin{center}
\begin{tabular}{|l|c|}
\hline
Zone       &Anzahl\\
\hline
Nordpol    & 10\\
Tropen nord& 20\\
Tropen s"ud& 14\\
S"udpol    & 6\\
\hline
\end{tabular}
\end{center}
Soll er das Projekt weiter verfolgen?


\begin{loesung}
Wenn das Projekt Erfolg haben soll, m"ussen die vier Zonen gleich h"aufig
vorkommen, wir f"uhren daher einen $\chi^2$-Test auf Gleichverteilung
durch.
Die Nullhypothese ist
\begin{quote}
Alle Zonen sind gleich wahrscheinlich.
\end{quote}
Dazu berechnen wir die Diskrepanz:
\begin{center}
\begin{tabular}{|l|c|c|c|c|}
\hline
Zone       & $n_i$ & $p_i$ & $np_i$ & $(n_i-np_i)^2/np_i$ \\
\hline
Nordpol    &  10   & 0.25  & 12.5   & 0.50 \\
Tropen nord&  20   & 0.25  & 12.5   & 4.50 \\
Tropen s"ud&  14   & 0.25  & 12.5   & 0.18 \\
S"udpol    &   6   & 0.25  & 12.5   & 3.38 \\
\hline
           &$n=50$ &       &        & 8.56 \\
\hline
\end{tabular}
\end{center}
Wir erhalten also als Diskrepanz $D=8.56$.
F"ur $\alpha$ w"ahlen wir den Wert $\alpha=5\%$.
Der kritische Wert f"ur $\chi^2$ f"ur 3 Freiheitsgrade ist 
$D_{\text{krit}}=7.815$, wird also von der gefundenen Diskrepanz
"uberschritten.
Die Nullhypothese muss also verworfen werden, auf der Basis dieser
Form ist nicht damit zu rechnen, dass man einen fairen W"urfel
erhalten wird, das Projekt hat wenig Aussicht auf Erfolg.
\end{loesung}

\begin{bewertung}
$\chi^2$-Test ({\bf X}) 1 Punkt,
Nullhypothese ({\bf N}) 1 Punkt,
Berechnung der Diskrepanz ({\bf D}) 1 Punkt,
Wahl von $\alpha$ ({\bf A}) 1 Punkt,
Korrekte Zahl der Freiheitsgrade und Bestimmung der kritische Diskrepanz
({\bf K}) 1 Punkt,
Schlussfolgerung ({\bf S}) 1 Punkt.
\end{bewertung}

