Es wird oft behauptet, dass Frauen einen höheren Körperfettanteil
haben als Männer.
Zur Untersuchung dieser Behauptung wurde bei 13 Männern und 10 Frauen
der Körperfettanteil gemessen mit den folgenden Resultaten:
\begin{center}
\begin{tabular}{|l|>{$}r<{$}>{$}r<{$}>{$}r<{$}|}
\hline
&\text{Stichprobengrösse}&\text{Mittelwert}&\text{Standardabweichung}\\
\hline
Männer &     13&    14.95&     6.48\\
Frauen &     10&    22.29&     5.32\\
\hline
\end{tabular}
\end{center}
Kann man aus diesen Daten eine Aussage über die ursprüngliche Behauptung
ableiten?

\begin{loesung}
Die beiden Mittelwerte können mit einem $t$-Test verglichen werden.
Die Null-Hypothese dafür lautet
\begin{quote}
Der Körperfettanteil von Männern und Frauen ist gleich.
\end{quote}
Angesichts der kleinen Datenmenge ist $\alpha = 0.05$ angemessen.
Die Zahl der Freiheitsgrade ist $13+10-2=21$, dies führt auf einen
kritischen Wert $t_{\text{krit}}=2.0796$.
Die Berechnung der Grösse $T$ ergibt
\begin{align*}
T
&=
\frac{\overline{X}-\overline{Y}}{\sqrt{(n-1)S_X^2+(m-1)S_Y^2}}
\sqrt{\frac{nm(n+m-2)}{n+m}}
\\
&=
\frac{14.95-22.29}{\sqrt{12\cdot 6.48^2 + 9\cdot 5.32^2}}
\sqrt{\frac{10\cdot 13\cdot 21}{23}}
=
-2.9034
\end{align*}
Da $|T|>t_{\text{krit}}$ ist, kann man die Nullhypothese verwerfen und 
folgern, dass Frauen tatsächlich einen höheren Körperfettanteil haben.
\end{loesung}

\begin{bewertung}
$t$-Test ({\bf T}) 1 Punkt,
Nullhypothese ({\bf N}) 1 Punkt,
Wahl von $\alpha$ ({\bf A}) 1 Punkt,
Freiheitsgrade und kritischer Wert ({\bf K}) 1 Punkt,
Berechnung von $T$ ({\bf W}) 1 Punkt,
Schlussfolgerung von $S$ ({\bf S}) 1 Punkt.
\end{bewertung}



