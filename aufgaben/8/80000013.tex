In j"ungster Zeit ist die Diskussion dar"uber erneut entbrannt,
ob Frauenquoten n"otig sind. Verwendet man Quoten, um den Frauenanteil
zu heben, dann muss man der Gleichberechtigung halber auch M"annerquoten
definieren. Nat"urlich m"ussten sich solche Quoten nach der Zahl der
verf"ugbaren Kandidaten beider Geschlechter richten, man kann ja nur
schwer Positionen mit Frauen besetzen, wenn es keine qualifizierten
Kandidatinnen daf"ur gibt.
\begin{teilaufgaben}
\item Das Problem beginnt nat"urlich schon in der Ausbildung. Im Modul
WrStat an der HSR sind 74 Personen eingeschrieben, davon 4 Frauen.
W"urde man aus
der Bev"olkerung zuf"allig 74 Personen ausw"ahlen, wie wahrscheinlich
w"are es, vier Frauen oder weniger zu w"ahlen?
\item Eine Firma muss $n$ Stellen neu besetzen und wir nehmen an, dass die
Bef"ahigung vom Geschlecht nicht abh"angt,
und dass unter den f"ahigen Kandidaten gleich viele Frauen wie M"anner
vorhanden sind. Wie gross muss die Abweichung der Anzahl Frauen von
$\frac{n}2$ sein, damit wir der Firma mit ziemlicher Sicherheit ($\alpha=1\%$)
Sexismus vorwerfen k"onnen?
\item
Wieviele Frauen m"ussen in ein $n=10$-k"opfiges
Gremium gew"ahlt werden, damit sich die Firma vor dem Vorwurf des
Seximus sch"utzen kann?
\item Eine Quote ist nur sinnvoll, wenn sie den tats"achlich zur
Verf"ugung stehenden Kandidaten Rechnung tr"agt.
Sei $p$ die Wahrscheinlichkeit, dass ein in Frage kommender Kandidate
eine Frau ist, und seien $n$ Stellen zu besetzen. Sei $F$ die
Zahl der eingestellten Frauen.
Damit die Quote weder f"ahige M"anner noch f"ahige Frauen bestraft,
muss die Quotenregelung so formuliert sein, dass der Frauenanteil nicht
``zu weit'' von $np$ entfernt ist. Wieviel gr"osser als $np$ muss $F$
sein, damit man der Firma vorwerfen kann, sie diskriminiere die M"anner?
Wieviel kleiner als $np$ muss $F$  sein, damit man ihr vorwerfen kann,
sie diskriminiere die Frauen? Berechnen Sie diese Schranken f"ur
$n=10$ und $p=\frac13$.
\end{teilaufgaben}
{\it Bemerkung:} Schon die Fragestellung macht die m"oglicherweise falsche
Annahme, dass die Wahrscheinlichkeit
$p$ konstant ist. Dies bedeutet n"amlich, dass man ein im Prinzip
unbegrenztes Reservoir an Kandidaten hat, aus denen man Mitarbeiter
rekrutieren kann. Oft ist aber bereits diese Annahme falsch: es ist
denkbar, dass es f"ur eine Spezialistenfunktion nur wenige Kandidaten
gibt.

\begin{loesung}
Die Anzahl Frauen ist in jedem Fall binomialverteilt.
F"ur grosse $n$ kann die Binomialverteilung durch eine Normalverteilung
approximiert werden.
\begin{teilaufgaben}
\item
Wir approximieren die Binomialverteilung durch eine Normalverteilung
mit Erwartungswert $\mu=np=\frac{n}2$ und Varianz $\sigma^2=np(1-p)=\frac{n}4$.
\begin{align*}
P(F\le 4)
&=
P\biggl( \frac{F-\mu}{\sigma}\le \frac{4-\mu}{\sigma} \biggr)
=
P\biggl( Z\le \frac{4-\mu}{\sigma} \biggr)
=
P\biggl( Z\le \frac{4-\frac{n}2}{\sqrt{\frac{n}4}}\biggr)
\\
&=
P\biggl( Z\le \frac{8-n}{\sqrt{n}}\biggr)
=
P\biggl( Z\le -\frac{70}{\sqrt{74}}\biggr)
\\
&=
P(Z\le -7.6723)= 8.444\cdot 10^{-15}
\end{align*}
Vier Frauen auf 74 Studenten ist also h"ochstwahrscheinlich keine
zuf"allige Auswahl.
\item
Wir verlangen, dass die Abweichung von einer parit"atischen
Geschlechterverteilung so gross sein muss, dass sie zuf"allig nur
mit Wahrscheinlichkeit $\alpha = 0.01$ eingetreten w"are, also
$P(|F-\frac{n}2| >\Delta)=\alpha$, und verwenden wieder eine
Normalapproximation mit $\mu=np$ und $\sigma^2=np(1-p)$.
\begin{align*}
P\biggl(\frac{n}2-\Delta\le F\le\frac{n}2+\Delta\biggr)
&=
P\biggl(\frac{\frac{n}2-\Delta-\mu}{\sigma}\le \frac{F-\mu}{\sigma}\le \frac{\frac{n}2+\delta-\mu}{\sigma}\biggr)
\\
&=
P\biggl(
\frac{-\Delta}{\sigma}\le Z\le \frac{\Delta}{\sigma}
\biggr)
=
F\biggl(\frac{\Delta}{\sigma}\biggr)
-
F\biggl(-\frac{\Delta}{\sigma}\biggr)
\\
&=
2F\biggl(\frac{\Delta}{\sigma}\biggr)-1=0.99
\\
F\biggl(\frac{\Delta}{\sigma}\biggr)&=0.995\\
\Rightarrow \Delta&\simeq 2.5758\sigma=2.5758\frac{ \sqrt{n}}2=1.2879\sqrt{n}
\end{align*}
\item
F"ur $n=10$ folgt, dass die Abweichung gr"osser als $\Delta = 4.0477$ sein
muss, es reicht also, eine Frau in das Gremium zu w"ahlen, die ber"uhmte
Alibi-Frau. Allerdings ist die Verwendung der Normalapproximation f"ur
$n=10$ etwas gewagt. Man kann aber mit R auch die Wahrscheinlichkeit
$P(F\le 1)$ f"ur $n=10$ ausrechnen. Man erh"alt $P(F\le 1)=0.01$, mit
einer Alibi-Frau schafft man es also auch in diesem Fall, sich vom
Vorwurf des Seximus reinzuwaschen.
\item
Wir m"ussen zun"achst einen sinnvollen Wert von $\alpha$ festlegen. $\alpha$
ist die Wahrscheinlichkeit, dass eine Firma, die bei der Stellenbesetzung
versucht hat, m"oglichst nur auf die fachliche Qualifikation zu achten,
per Zufall doch eine ``sexistische'' Besetzung erhalten hat. W"ahlt man
also $\alpha$ gross, werden ``rechtschaffenen'' Firmen mit hoher
Wahrscheinlichkeit ungerechtfertigterweise als sexistisch gebrandmarkt.
W"ahlt man $\alpha$ allerdings zu klein, werden die Schranken derart weit,
dass die Quote jeden Sinn verliert.

\begin{figure}
\begin{center}
\begin{tabular}{cc}
\includeagraphics[width=0.49\hsize]{frauen.pdf}&%
\includeagraphics[width=0.49\hsize]{frauen-all.pdf}
\end{tabular}
\end{center}
\caption{Frauenquoten f"ur $\alpha\in\{0.05,0.01, 0.001\}$.\label{frauen}}
\end{figure}

\begin{figure}
{\small
\verbatimainput{frauen.R.expanded}
}
\caption{R-Programm zur Berechnung der Frauenquoten\label{quoten}}
\end{figure}

Wir suchen also Schranken $F_-$ und $F_+$ f"ur die Frauenzahl so, dass
$P(F<F_-)=\alpha$ und $P(F>F_+)=\alpha$. Dies kann man am einfachsten
mit R durchf"uhren. Das Programm in Abbildung~\ref{quoten} berechnet die
Graphiken in Abbildung~\ref{frauen}. Man sieht, dass f"ur kleine
Gremien und hohe Anforderungen daran, dass keine unschuldigen Firmen
angeschw"arzt werden nicht einmal die Alibi-Frau vorgeschrieben
werden kann.
\qedhere
\end{teilaufgaben}
\end{loesung}

