Ein grosses Unternehmen möchte die Ausfälle durch Grippeerkrankungen
vermeiden und bietet daher allen Mitarbeitern rechtzeitig eine Impfung
an.
Zu Beginn der Grippe-Saison sind mehrere Impfstoffe verfügbar, es ist
aber nicht bekannt, welcher am besten wirken wird.
Nach der Saison kann man allerdings feststellen, wie erfolgreich
die Impfung war.
Die Tabelle zeigt einerseits die verzeichneten Grippefälle trotz
Impfung mit Impfstoff $i$ und
andererseits den Anteil der Mitarbeiter, die mit dem Impfstoff $i$
geimpft worden waren.
\begin{center}
\begin{tabular}{|>{$}r<{$}|>{$}r<{$}>{$}r<{$}|}
\hline
\text{Impfstoff}&\text{Grippefälle}&\text{Anteil der MA}\\
\hline
1&  43&    20.74\%\\
2&  52&    18.52\%\\
3&  25&    20.00\%\\
4&  48&    19.26\%\\
5&  57&    21.48\%\\
\hline
\end{tabular}
\end{center}
Kann man daraus einen Schluss ziehen, ob ein Impfstoff besonders
wirksam war?

\begin{loesung}
Wir führen einen $\chi^2$-Gest für die Nullhypothese
\begin{quote}
Die Fallzahlen passen zur Verteilung der Impfstoff in der
Belegschaft.
\end{quote}
durch.
\ainput{d.tex}
Wegen der geringen Fallzahlen ist ein $\alpha=5\%$ angemessen.
Die Fallzahlen $n_i$ entnehmen wir der zweiten Spalte der Tabelle,
die Wahrscheinlichkeiten $p_i$ der dritten Spalte.
Damit lässt sich jetzt die Diskrepanz berechnen:
\begin{center}
\begin{tabular}{|>{$}r<{$}|>{$}r<{$}>{$}r<{$}>{$}r<{$}>{$}r<{$}>{$}r<{$}|}
\hline
i&     n_i&       p_i & np_i & n_i - np_i & (n_i-np_i)^2/np_i\\
\hline
\tabelle
\hline
\end{tabular}
\end{center}
Der kritische Wert der Diskrepanz für 4 Freiheitsgrade ist
$D_{\text{krit}}=9.488$.
Da $D=\D>D_{\text{krit}}$ ist, kann man die Nullhypothese verwerfen
und folgern, dass es tatsächlich einen Unterschied zwischen den 
verschiedenen Impfungen gibt.
Es scheint, dass Impfung 3 besonders wirkungsvoll war.
\end{loesung}

\begin{bewertung}
Nullhypothese ({\bf H$\mathstrut_0$}) 1 Punkt,
$\chi^2$-Test ({\bf X}) 1 Punkt,
Berechnung der Diskrepanz ({\bf D}) 1 Punkt,
Wahl von $\alpha$ ({\bf A}) 1 Punkt,
Freiheitsgrade und kritischer Wert ({\bf K}) 1 Punkt,
Schlussfolgerung ({\bf S}) 1 Punkt.
\end{bewertung}

