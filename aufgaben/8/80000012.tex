In einem Sack M-Budget Gummibonbons befinden sich weisse, gr"une und orange
Gummi-Ms. Es scheint, dass gewisse Farben bevorzugt vorkommen.
\begin{teilaufgaben}
\item Formulieren Sie eine Hypothese und beschreiben Sie einen Test, mit
dem die Hypothese getestet werden kann.
\item Zur Durchf"uhrung des Tests wurde ein Sack M-Budget Gummibonbons
beschafft und folgende Anzahlen der verschiedenen Farben von Gummibonbons
gez"ahlt:
\begin{center}
\begin{tabular}{lr}
Farbe&Anzahl\\
\hline
weiss&16\\
gr"un&22\\
orange&27
\end{tabular}
\end{center}
F"uhren Sie den Test mit $\alpha = 0.1$ durch.
\end{teilaufgaben}

\begin{loesung}
\begin{teilaufgaben}
\item Die zu testende Hypthese ist ``die Farben sind gleichverteilt'', jede Farbe
hat also die Wahrscheinlichkeit
$p_{\text{weiss}}=p_{\text{gr"un}}=p_{\text{orange}}=\frac13$. Dies kann
mit einem $\chi^2$-Test getestet werden.
\item Die Diskrepanz wird wie folgt berechnet:
\begin{center}
\begin{tabular}{lrrr}
Farbe&$n_i$&$np_i$&Diskrepanz\\
\hline
weiss&16&21.66&1.479\\
gr"un&22&21.66&0.005\\
orange&27&21.66&1.317\\
\hline
&65&&2.801
\end{tabular}
\end{center}
Die Verteilungsfunktion der $\chi^2$-Verteilung mit 2 Freiheitsgraden nimmt
den Wert $0.9=1-0.1$ bei 4.605 an. Somit geben die erhobenen Daten keinen
Hinweis darauf, dass die Hypothese der Gleichverteilung unzutreffend w"are.
\qedhere
\end{teilaufgaben}
\end{loesung}

