Während der Covid-19 Pandemie im Jahre 2020 ging in den sozialen Medien
und den Fake News die Behauptung um, es gäbe keine Veränderung der
Sterblichkeit unter jüngeren Menschen.
In der folgenden Tabelle sind die Todesfallzahlen für
20--39-jährige in der Schweiz für
das Jahr 2020 und die Todesfallhäufigkeit in den Vorjahren 2015-2019
zusammengestellt.
\begin{center}
\begin{tabular}{|c|c|c|}
\hline
Wochen         &Todesfälle 2020&Todesfallhäufikeit Vorjahre\\
\hline
\phantom{0}1--10&            141&         0.2093199 \\
          11--20&            183&         0.1874055 \\
          21--30&            150&         0.1997481 \\
          31--40&            170&         0.2095718 \\
          41--50&            133&         0.1939547 \\
\hline
\end{tabular}
\end{center}
Was kann man auf Grund dieser Zahlen über obige Behauptung sagen?

\thema{Hypothesentest}

\begin{loesung}
Man kann die folgende Nullhypothese testen:
\begin{quote}
Die Verteilung der Todesfallzahlen unter den 20--39-jährigen im Jahr
2020 unterscheidet sich nicht von der langjährigen Todesfallhäufigkeit.
\end{quote}
Diese Aussage kann mit einem $\chi^2$-Test getestet werden.
Für einen Test auf dem Niveau $\alpha=5\%$ ist die kritische 
Diskrepanz $D_{\text{krit}}=9.488$.
Die Berechnung der Diskrepanz liefert:
\begin{center}
\begin{tabular}{|c|rr|rr|}
\hline
Wochen           &  $n_i$& $p_i$     & $np_i$   & $(n_i-np_i)^2/np_i$  \\
\hline
\phantom{0}1--10 &   141 & 0.2093199 & 162.6416 &  2.87968 \\
11--20           &   183 & 0.1874055 & 145.6141 &  9.59869 \\
21--30           &   150 & 0.1997481 & 155.2043 &  0.17450 \\
31--40           &   170 & 0.2095718 & 162.8373 &  0.31506 \\
41--50           &   133 & 0.1939547 & 150.7028 &  2.07951 \\
\hline
                 &$n=777$&           &          &$D= 15.04747$\\
\hline
\end{tabular}
\end{center}
Da die Diskrepanz deutlich grösser ist als die kritische Diskrepanz,
also $D>D_{\text{krit}}$, muss die Nullhypothese verworfen werden.
Die Behauptung, die Pandemie hätte keine Auswirkungen auf die
20--39-jährigen, ist also falsch.

Man kann in der Tabelle auch ablesen, dass der wesentliche Beitrag
zur Diskrepanz in den Wochen 11--20 zustande kam, also währen der
ersten Welle.
Man kann auch erkennen, dass sogar ein Test mit $\alpha=0.01$ 
einen signifikanten Effekt nachweist.
\end{loesung}


