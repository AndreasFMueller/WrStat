Nach einer Prüfung, in der Studenten zweier Studiengänge $A$ und $B$
geprüft wurden, möchten die Studiengangleiter wissen, ob man zwischen
den beiden Studiengängen einen Unterschied sehen kann.
Der prüfungsverantwortliche Dozent teilt den Studiengangleitern
folgende Daten mit:
\begin{center}
\begin{tabular}{c|>{$}c<{$}|>{$}c<{$}|>{$}c<{$}}
Studiengang
	&\text{Teilnehmer}
		&\text{Mittelwert}
			&\text{emprische Standardabweichung}
\\
\hline
A	&60	&4.45	&1.1
\\
B	&15	&4.11	&0.8
\\
\end{tabular}
\end{center}
und überlässt ihnen die Auswertung.
Ist der Leiter von Studiengang $A$ auf Grund dieser Daten berechtigt,
sich mit der Überlegenheit seines Studiengangs zu brüsten?

\thema{Hypothesentest}
\themaL{tTest}{$t$-Test}

\begin{loesung}
Wir testen dies mit einem $t$-Test.
Die Nullhypothese $H_0$ ist: ``es gibt keinen Unterschied zwischen den
beiden Mittelwerten''.
Wir führen einen Test mit $\alpha=0.05$ durch.
Die Messgrösse ist
\begin{align*}
T
&=
\frac{\overline{X}-\overline{Y}}{\sqrt{S_X^2(n_X-1)+S_Y^2(n_Y-1)}}
\sqrt{\frac{n_Xn_Y(n_X+n_Y-2)}{n_X+n_Y}}
\\
&=(\overline{X}-\overline{Y})
\frac{1}{\sqrt{1.1^2\cdot 59+0.8^2\cdot 14}}
\sqrt{\frac{60\cdot15(60+15-2)}{60+15}}
\\
&=(\overline{X}-\overline{Y})
\frac{\sqrt{876}}{8.9638}
=(\overline{X}-\overline{Y})\cdot 3.302
\\
&=0.34\cdot 3.302
=1.123
\end{align*}
Der kritische Wert für $\alpha=0.05$ und
$n_X+n_Y-2=60+15-2=73$
Freiheitsgrade ist
$t_{\text{krit}}=1.66$ für $p=1-\alpha=0.95$.
Da $T<t_{\text{krit}}$ gibt es keinen Grund an der Nullhypothese zu zweifeln.
Der Leiter von Studiengang $A$ handelt trotz des grossen Unterschieds der
Mittelwerte nicht rational, wenn er seinen Studiengang für besser ansieht.
\end{loesung}

\begin{bewertung}
Nullhypothese ({\bf N}) 1 Punkt,
Verwendung des studentschen $t$-Tests ({\bf S}) 1 Punkt,
Wahl eines geeigneten $\alpha$ ({\bf A}) 1 Punkt,
Berechnung der Testgrösse $T$ ({\bf T}) 1 Punkt,
Berechnung der Freiheitsgrade für die $t$-Verteilung und kritischer Wert
({\bf K}) 1 Punkt,
Entscheidung, Nullhypothese wird nicht verworfen ({\bf E}) 1 Punkt.
\end{bewertung}


