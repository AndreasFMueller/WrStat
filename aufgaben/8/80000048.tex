\bgroup
\ainput{data.tex}
Die Sternenflotte rekrutiert Personal für ihre Raumschiffe für 
die vier Bereiche {\em Command}, {\em Science}, {\em Engineering} und
{\em Security}.
Die anteilsmässige Grösse dieser Bereiche ist in
Tabelle~\ref{80000048:data} festgehalten.
Auf der Mission der Enterpreise NCC-1701 in den Jahren 2265--2269 kamen 
insgesamt 40 Besatzungsmitglieder um.
Die Aufteilung der Todesfälle auf die Bereich ist ebenfalls in der
Tabelle zusammengestellt.
Es wird behauptet, nicht in allen Bereichen seien die Überlebenschancen
gleich gut.
Können Sie dies beweisen oder wiederlegen?

\begin{table}[h]
\centering
\begin{tabular}{|l|rrr|}
\hline
            &Shirtfarbe&Anteil [\%]&Todesfälle\\
\hline
Command     &      gold&      \pone&         9\\
Science     &      blau&      \ptwo&         7\\
Engineering &       rot&    \pthree&         6\\
Security    &       rot&     \pfour&        18\\
\hline
\end{tabular}
\caption{Rekrutierungsanteile für die vier Bereiche und Anzahl der
Todesfälle auf der Mission des Raumschiffs NCC-1701 Enterprise 
in den Jahren 2265--2269.
\label{80000048:data}}
\end{table}

\begin{hinweis}
Die Zahlen in Tabelle~\ref{80000048:data} weichen von den Zahlen
über Todesfälle in Star Trek ab, mit denen der Begriff der bedingten
Wahrscheinlichkeit in der Vorlesung eingeführt worden ist.
\end{hinweis}

\begin{loesung}
Wir testen die Nullhypothese
\begin{quote}
Die Häuffigkeit von Todesfällen in den einzelnen Bereichen entspricht
den Anteilen, die die Bereich an der Besatzung haben.
\end{quote}
mit Hilfe eines $\chi^2$-Tests.
Wir verwenden $\alpha=5\%$, also $p=0.95$.
Da es $4$ mögliche Ausgänge gibt, müssen wir den kritischen Wert
für $3$ Freiheitsgrade aus der $\chi^2$-Tabelle zu lesen, wir
erhalten den Wert $D_{\text{krit}} = 7.815$.
Die Diskrepanz kann mit der folgenden Tabelle berechnet werden:
\begin{center}
\begin{tabular}{|>{$}c<{$}|>{$}r<{$}>{$}r<{$}|>{$}r<{$}>{$}r<{$}|}
\hline
i&    p_i& n_i&np_i&(n_i-np_i)^2/np_i\\
\hline
\tabledata
\hline
 &       &n=40&    & D=\D\\
\hline
\end{tabular}
\end{center}
Da $D>D_{\text{krit}}$ ist, muss die Nullhypothese verworfen werden.
\end{loesung}
\egroup

\begin{bewertung}
Nullhypothese ({\bf H}) 1 Punkt,
Wahl von $\alpha$ ({\bf A}) 1 Punkt,
Freiheitsgrade und kritische Diskrepanz ({\bf F}) 1 Punkt,
Berechnung der Diskrepanz ({\bf D}) 2 Punkt,
Entscheid und Schlussfolgerung ({\bf S}) 1 Punkt.
\end{bewertung}

