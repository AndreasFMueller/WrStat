Testen Sie mit dem Kolmogorov-Smirnov-Test, ob die Daten von Aufgabe~\ref{80000061}
eine Stichprobe einer standardnormalverteilten Zufallsvariable ist.

\begin{loesung}
\ainput{data.tex}
Die Nullhypothese ist
\begin{center}
$H_0$: die Zahlen bilden eine Stichprobe einer standardnormalverteilten Zufallsvariable.
\end{center}
In Aufgabe~\ref{80000063} wurden die Werte 
\[
K_n^+
=
\pgfmathparse{sqrt(10)*\Kplus}
\pgfmathprintnumber[fixed,precision=4]{\pgfmathresult}
\qquad\text{und}\qquad
K_n^-
=
\pgfmathparse{sqrt(10)*\Kminus}
\pgfmathprintnumber[fixed,precision=4]{\pgfmathresult}
\]
bestimmt.
Aufgrund der geringen Datenmenge wählen wir $\alpha=5\%$.
Aus der K-S-Tabelle liest man für $n=10$ den zugehörigen kritischen Wert
für $p=0.95$ als $K_\text{krit}=1.16582$ ab.
Da weder $K_n^+$ noch $K_n^-$ den kritischen Wert überschreitet, gibt es keinen
Grund, an der Nullhypothese zu zweifeln.
\end{loesung}
