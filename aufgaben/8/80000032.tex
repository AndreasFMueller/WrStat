Ein Maschinenbauer bezieht Schrauben von drei verschiedenen Lieferanten,
die nicht den gleichen Anteil an seinem Gesamtbedarf decken.
Er führt Statistik darüber, wie oft Schrauben jedes Lieferanten
zu Defekten führen:
\begin{center}
\begin{tabular}{|c|c|c|}
\hline
Lieferant&Anteil&Schadenfälle
\\
\hline
1&25\%&13
\\
2&25\%&27
\\
3&50\%&40
\\
\hline
\end{tabular}
\end{center}
Auf den ersten Blick scheint Lieferant~2 im Vergleich zu seinem 
Volumen einen besonders hohen Anteil an den Schadenfällen zu haben,
die Geschäftsleitung beschliesst daher, sich nach einem anderen
Lieferanten umzusehen.
Handelt sie rational?

\thema{Hypothesentest}
%\thema{$\chi^2$-Test}

\begin{loesung}
Die Nullhypothese $H_0$ ist: Die Häufigkeit der Schadenfälle
entspricht dem Liefervolumen eines Lieferanten.

Mit einem $\chi^2$-Test mti $\alpha=0.05$ lässt sich diese Hypothese
testen.
Wir berechnen die Diskrepanz:
\begin{center}
\begin{tabular}{|c|>{$}c<{$}|>{$}r<{$}|>{$}c<{$}|>{$}r<{$}|>{$}r<{$}|}
\hline
Lieferant& p_i& n_i&np_i&n_i-np_i&(n_i-np_i)^2/np_i
\\
\hline
1        &0.25&  13&  20&      -7&  2.45
\\
2        &0.25&  27&  20&       7&  2.45
\\
3        &0.50&  40&  40&       0&  0.00
\\
\hline
         &    &n=80&    &        &D=4.90
\\
\hline
\end{tabular}
\end{center}
Der kritische Wert für die Diskrepanz bei $3-1=2$ Freiheitsgraden ist
$D_{\text{krit}}=5.991$.
Da $D<D_{\text{krit}}$ kann der Test die Nullhypothese nicht verwerfen,
die Geschäftsleitung handelt irrational.
\end{loesung}

\begin{bewertung}
Nullhypothese ({\bf H}) 1 Punkt,
$\chi^2$-Test ({\bf$\chi$}) 1 Punkt,
Wahl von $\alpha$ ({\bf A}) 1 Punkt,
Freiheitgrade und kritischer Wert der Diskrepanz ({\bf K})  1 Punkt,
Berechnung der Diskrepanz ({\bf D}) 1 Punkt,
Schlussfolgerung ({\bf S}) 1 Punkt.
\end{bewertung}

