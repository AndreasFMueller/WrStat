Am 14. Oktober 2020 erschien in der Zeitschrift {\em Blood Advances}
ein Artikel über die Resultate einer dänischen Studie mit über 7000
Subjekten über einen Zusammenhang zwischen Blutgruppe und Covid-19-Risiko,
deren Resultate graphisch wie folgt zusammengefasst wurden:
\begin{center}
\includeagraphics[width=8cm]{summary.png}
\end{center}
In der Studie wurden Daten von immerhin 473654 Corona-Tests ausgewertet,
wovon 7422 positiv waren.
Die Studie behauptet, dass die Blutgruppe ein Risikofaktor sei für einen
positiven Test, aber nichts aussagt über das Risiko für einen schweren
Verlauf der Krankheit.

Formulieren Sie einen Test, mit dem man allein auf der
Basis der dargestellten Informationen entscheiden kann, ob
die Resultate dieser Studie aussagekräftig sind.

\begin{hinweis}
Im Orginalartikel stehen auch die beobachteten Fallzahlen in Abhängigkeit
von der Blutgruppe, für die Graphik wurden diese in Prozentzahlen umgerechnet,
die sich aber nicht einmal zu 100\% summieren.
Um diese Schwierigkeit müssen Sie sich natürlich auch kümmern.
\end{hinweis}

\begin{loesung}
Die Studie untersucht die Nullhypothese
\begin{quote}
Die Wahrscheinlichkeit einer Person mit Blutgruppe
$i\in\{\text{0},\text{A},\text{B},\text{AB}\}$ an Covid-19 zu erkranken
entspricht der Wahrscheinlichkeitsverteilung der Blutgruppen in der
Bevölkerung.
\end{quote}
Die Hypothese kann mit einem $\chi^2$-Test getestet werden, dieser braucht
aber nicht Prozentzahlen für die $n=7422$ Test-Subjekte, sondern Anzahlen.
Diese kann man aber durch Multiplikation der Wahrscheinlichkeiten mit $n$
erhalten.
Wir wählen in diesem Fall $\alpha=0.01$, die Zahl der Freiheitsgrade
ist $4-1=3$, der kritische Wert der Diskrepanz ist
$D_{\text{krit}} = 11.345$.
Wir berechnen die Diskrepanz:
\begin{center}
\begin{tabular}{|c|r>{$}r<{$}|>{$}r<{$}>{$}r<{$}>{$}r<{$}|}
\hline
Blutgruppe
  &      &  n_i   &  p_i &   np_i & (n_i-np_i)^2/np_i \\
\hline
0 & 38\% & 2820.4 & 0.42 & 3086.1 &  22.876 \\
A & 44\% & 3265.7 & 0.42 & 3086.1 &  10.452 \\
B & 12\% &  890.6 & 0.11 &  808.7 &   8.294 \\
AB&  5\% &  371.1 & 0.04 &  293.9 &  20.278 \\
\hline
  &      & 7347.8 &      &        &D=61.900 \\
\hline
\end{tabular}
\end{center}
Die Diskrepanz ist also deutlich grösser als $D_{\text{krit}}$,
die Nullhypothese muss daher verworfen werden und man darf nach heutigem
Wissensstand tatsächlich davon ausgehen, dass die Blutgruppe einen Einfluss
auf das Ansteckungsrisiko hat.
\end{loesung}

\begin{bewertung}
$\chi^2$-Text ({\bf X}) 1 Punkt,
Wahl von $\alpha$ ({\bf A}) 1 Punkt,
Bestimmung des kritischen $D$ ({\bf K}) 1 Punkt,
Ermittelung der $n_i$ aus den vorhandene Daten ({\bf I}) 1 Punkt,
Berechnung der Diskrepanz ({\bf D}) 1 Punkt,
Schlussfolgerung ({\bf S}) 1 Punkt.
\end{bewertung}

\begin{diskussion}
Die Studie macht viele weitere detailliertere Aussagen, die in
\url{https://ashpublications.org/bloodadvances/article/4/20/4990/463793/Reduced-prevalence-of-SARS-CoV-2-infection-in-ABO}
nachgelesen werden können.
\end{diskussion}

