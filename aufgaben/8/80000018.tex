Ein Fünferwürfel kann aus einem dreiseitgen Prisma wie folgt
konstruiert werden:
\begin{center}
\includeagraphics[width=\hsize]{5wuerfel.jpg}
\end{center}
Während bei den üblichen Würfeln mit Zahlen 1 bis 6 die Symmetrie
des Würfels suggeriert, dass die Zahlen gleich wahrscheinlich sind,
fehlt eine solche Garantie im abgebildeten Fünferwürfel gänzlich.
Daher wurde er 100 mal geworfen, mit folgenden Resultaten:
\begin{center}
\begin{tabular}{|c|r|}
\hline
Augenzahl&Anzahl\\
\hline
1&22\\
2&16\\
3&21\\
4&15\\
5&26\\
\hline
\end{tabular}
\end{center}
Ist dieser Würfel fair?

\begin{loesung}
Wir machen einen $\chi^2$-Test mit der Nullhypothese: ``alle Augenzahlen
sind gleich wahrscheinlich''. Die fünf möglichen Versuchsausgänge
haben die Wahrscheinlichkeiten $p_1=p_2=\dots=p_5=\frac15$. Die 
Berechnung der Diskrepanz ergibt:
\begin{center}
\begin{tabular}{|>{$}c<{$}|>{$}c<{$}|>{$}r<{$}|>{$}c<{$}|>{$}r<{$}|>{$}r<{$}|}
\hline
i&p_i&n_i&np_i&n_i-np_i&(n_i-np_i)^2/np_i\\
\hline
1&0.2&     22&20& 2&0.20\\
2&0.2&     16&20&-4&0.80\\
3&0.2&     21&20& 1&0.05\\
4&0.2&     15&20&-5&1.25\\
5&0.2&     26&20& 6&1.80\\
\hline
 &   &n = 100&  &  &D=4.10\\
\hline
\end{tabular}
\end{center}
Wir wählen für den Test $\alpha=5\%$, der kritische Wert der
$\chi^2$-Verteilung mit $4$ Freiheitsgraden ist $D_{\text{krit}}=7.81$.
Da $D<D_{\text{krit}}$ gibt es keinen Grund, an der Hypothese zu
zweifeln. Nach aktuellem Wissenstand ist der Fünferwürfel also fair.
\end{loesung}

\begin{bewertung}
Hypothesentest mit $\chi^2$ ({\bf X}) 1 Punkt,
Hypothese/Nullhypthese ({\bf H}) 1 Punkt,
Wahl eines geeigneten $\alpha$ ({\bf A}) 1 Punkt,
Berechnung der Diskrepanz ({\bf D}) 1 Punkt,
Ermittlung der kritischen Diskrepanz aus der $\chi^2$-Tabelle ({\bf K})
1 Punkt (richtige Anzahl Freiheitsgrade),
Schlussfolgerung ({\bf S}) 1 Punkt.
\end{bewertung}



