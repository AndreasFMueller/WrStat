In einem Projekt für ein Gerät mit digitaler Steuerung, welches
seine Aufgabe mit möglichst geringem Stromverbrauch erfüllen soll, stehen
drei Algorithmen zur Auswahl, die alle nur mehr oder weniger gute
Approximationen liefern. Keiner der Algorithmen ist klar besser wie
die anderen. Weder Rechenzeit noch vorhandener Speicherplatz
erlauben, alle drei Algorithmen zu implementieren. Die Ingenieure
müssen sich daher für einen entscheiden. Sie unterwerfen die drei
Algorithmen daher zufällig generierten Testszenarien, und zählen,
wie oft jeder Algorithmus das beste Resultat liefert. Sie erhalten
folgendes Resultat:
\begin{center}
\begin{tabular}{c|c}
Algorithmus&Anzahl Bestresultate\\
\hline
A&12\\
B&15\\
C&9\\
\end{tabular}
\end{center}
Kann man behaupten, dass Algorithmus B klar besser sei als die beiden
anderen?

\thema{Hypothesentest}
\themaL{chi2Test}{$\chi^2$-Test}

\begin{loesung}
Offenbar geht es hier darum, die Hypothese zu testen, dass alle
drei Algorithmen gleich gut sind. Kann man diese Hypothese
verwerfen, bedeutet dies, dass einer der Algorithmen klar besser ist,
und das ist natürlich der Algorithmus B, da er am meisten Bestresultate
erzielt hat.

Um die Hypothese zu testen, verwendet man den $\chi^2$-Test mit
den Wahrscheinlichkeiten $p_1=p_2=p_3=\frac13$. Die Berechnung
der Diskrepanz kann mit folgender Tabelle durchgeführt werden:
\begin{center}
\begin{tabular}{|c|c|c|c|}
\hline
$i$&$n_i$&$n_i-np_i$&$D$\\
\hline
A&12&0&$0$\\
B&15&3&$\frac{9}{12}=\frac34$\\
C& 9&$-3$&$\frac{9}{12}=\frac34$\\
\hline
&36&&$\frac{3}{2}=1.5$\\
\hline
\end{tabular}
\end{center}
Für zwei Freiheitsgrade und $\alpha=0.05$ ist der kritische Wert der
Diskrepanz $D_{\text{krit}}=5.991$. Da $D<D_{\text{krit}}$ gibt es
auf Grund der Datenlage keinen Grund daran zu zweifeln, dass alle drei
Algorithmen gleich gut sind.
\end{loesung}

