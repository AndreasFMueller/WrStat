Sternen wird auf Grund ihrer Farbe die sogenannte Spektralklasse zugeordnet,
eine Buchstabenkennzeichnung, die Information "uber Gr"osse und Temperatur
enth"alt.
In der Milchstrasse beobachtet man die folgende Verteilung
der Spektralklassen:
\begin{center}
\begin{tabular}{|c|l|l|}
\hline
Klasse&Farbe     &H"aufigkeit\\
\hline
O     &blau      &0.0000003\\
B     &blau-weiss&0.0013\\
A     &weiss     &0.006\\
F     &gelb-weiss&0.03\\
G     &gelb      &0.076\\
K     &orange    &0.121\\
M     &rot       &0.7645\\
\hline
\end{tabular}
\end{center}
Die spektrale Verteilung der Sterne einer Population "andert sich mit der
Zeit, da die blauen Sterne sehr viel weniger lange leben.

In einem neu entdeckten Kugelsternhaufen werden $100$ Sterne spektral
analysiert und dabei folgende Anzahlen gefunden:
\begin{center}
\begin{tabular}{|c|c|}
\hline
Klasse&Anzahl\\
\hline
G&9\\
K&22\\
M&69\\
\hline
\end{tabular}
\end{center}
Es wird behauptet, diese Sterne des Kugelsternhaufens entstammen einer
anderen Population als die Sterne der Milchstrasse.
Ist diese Aussage gerechtfertigt?

\begin{loesung}
Die Nullhypothese ist, dass die ausgemessenen Sterne zur selben
Population wie die Milchstrassensterne geh"oren.
Wir m"ussen die beobachtete Verteilung mit der theoretischen Verteilung
der Milchstrasse vergleichen.
Dazu eignet sich ein $\chi^2$-Test mit $\alpha=5\%$.
\begin{center}
\begin{tabular}{|c|l|r|l|l|}
\hline
Klasse&$p_i$    &$n_i$&$np_i$            &$(n_i-np_i)/\sqrt{np_i}$\\
\hline
   O  &0.0000003&    0&\phantom{0}0.00003& 0.000\phantom{03003603}\\
   B  &0.0013   &    0&\phantom{0}0.13   & 0.130\phantom{1561    }\\
   A  &0.006    &    0&\phantom{0}0.6    & 0.600\phantom{7207    }\\
   F  &0.03     &    0&\phantom{0}3      & 3.003\phantom{603     }\\
   G  &0.076    &    9&\phantom{0}7.6    & 0.254\phantom{2371    }\\
   K  &0.121    &   22&12.1              & 8.066\phantom{546     }\\
   M  &0.7645   &   69&76.45             & 0.743\phantom{1121    }\\
\hline
      &         &     &                  &$D=12.79$\\
\hline
\end{tabular}
\end{center}
Der kritische Wert f"ur $D$ f"ur $\alpha = 0.05$ und $6$ Freiheitsgrade
ist $D_{\text{krit}}=12.592$. 
Da $D>D_{\text{krit}}$ muss die Nullhypothese verworfen werden.
\end{loesung}

\begin{diskussion}
Man k"onnte argumentieren, dass dieser $\chi^2$-Test nicht zul"assig ist,
weil in den Spektralkalssen O bis F gar keine Sterne beobachtet wurden.
Man k"onnte dem oben durchgef"uhrten Test daher auch einen Test
gegen"uberstellen, in dem man nur die Spektralklassen G bis M ber"ucksichtigt.
Dann muss man aber die Wahrscheinlichkeiten neu skalieren, so dass sie
zusammen 1 ergeben.
Aus der obigen Tabelle zur Berechnung der Diskrepanz wird dann:
\begin{center}
\begin{tabular}{|c|l|r|l|l|}
\hline
Klasse&$p_i$    &$n_i$&$np_i$            &$(n_i-np_i)/\sqrt{np_i}$\\
\hline
   G  &0.0790432&    9&\phantom{0}7.9043 & 0.151\phantom{882     }\\
   K  &0.1258450&   22&12.5845           & 7.044\phantom{503     }\\
   M  &0.7951118&   69&79.5112           & 1.389\phantom{552     }\\
\hline
      &         &     &                  &$D=8.58$\\
\hline
\end{tabular}
\end{center}
%[1] 0.07904316 0.12584503 0.79511180
%> nikurz = ni[teil]
%> nikurz
%[1]  9 22 69
%> npi = 100 * pkurz
%> npi
%[1]  7.904316 12.584503 79.511180
%> dkurz = (nikurz - npi)^2 / npi
%> dkurz
%[1] 0.151882 7.044503 1.389552
%> Dkurz = sum(dkurz)
%> Dkurz
%[1] 8.585937
%> 
%> chisq.test(nikurz, p = pkurz)
%
%        Chi-squared test for given probabilities
%
%data:  nikurz
%X-squared = 8.5859, df = 2, p-value = 0.01366
Der kritische Wert f"ur zwei Freiheitsgrade und $\alpha=0.05$ ist
$D_{\text{krit}}=5.991$,
so dass erneut die Nullhypothese verworfen werden kann.
\end{diskussion}

\begin{bewertung}
$\chi^2$-Test ({\bf X}) 1 Punkt,
Hypothese/Nullhypothese ({\bf H}) 1 Punkt,
Berechnung der Diskrepanz ({\bf D}) 1 Punkt,
Wahl eines geeigneten $\alpha$ ({\bf A}) 1 Punkt,
Freiheitsgrad und Bestimmung des kritischen Wertes der Diskrepanz
$D_{\text{krit}}$ ({\bf F}) 1 Punkt,
Schlussfolgerung ({\bf S}) 1 Punkt.
\end{bewertung}


