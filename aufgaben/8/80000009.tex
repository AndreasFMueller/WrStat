Wir haben von den Mythbusters leider keine exakten Daten über den
Benzinverbrauch des Testautos mit und ohne ``Golfball''-Dellen.
Nehmen Sie an, es seien für den Verbrauch die folgenden Werte in ml/km
gemessen worden:
\begin{center}
\begin{tabular}{ccc}
\hline
Fahrt&Verbrauch mit Dellen&Verbrauch ohne Dellen\\
\hline
1&86&103\\
2&97&97\\
3&102&104\\
\hline
\end{tabular}
\end{center}
Kann man daraus schliessen, dass die Dellen eine Wirkung haben?

\thema{$t$-Test}
\thema{Hypothesentest}

\begin{loesung}
Wir testen die Nullhypothese, dass der mittlere Verbrauch in den
beiden Situationen gleich ist.
Können wir die Nullhypothese verwerfen, schliessen wir, dass die
Dellen tatsächlich einen Einfluss auf den Verbraucht haben.

Für den Test verwenden wir einen $t$-Test, ein Alpha von
$\alpha=0.05$ scheint angebracht.

Die mittleren Verbrauchswerte sind 95ml/km mit und 101.3ml/km ohne
Dellen.
Allerdings ist die Testgrösse für den $t$-Test
$T=1.21635$ und diese ist grösser als der kritische Wert für
einen einseitigen, und damit leichter zu bestehenden $t$-Test mit
$3+3-2=4$ Freiheitsgraden und
$\alpha=0.05$: $t_{\text{krit}}=2.1318$
%[1]  86.13225  97.83724 102.09296
%> y <- rnorm(m = 110, sd = 12, m);
%> y
%[1] 102.75746  97.15253 104.07669
%> mean(x)
%[1] 95.35415
%> var(x)
%[1] 68.31045
%> mean(y)
%[1] 101.3289
%> var(y)
%[1] 13.51658
%> 
%> sx2 = var(x) 
%> sy2 = var(y)
%> 
%> sp2 = ((n- 1) * sx2 + (m-1) * sy2)/(n+m-2)
%> sp2
%[1] 40.91352
%> 
%> t = sqrt(n * m * (n+m-2)/(n+m)) * (mean(x) - mean(y)) / sqrt(sp2)
%> t
%[1] -2.288029
%> 
\end{loesung}

