Wir haben von den Mythbusters leider keine exakten Daten über den
Benzinverbrauch des Testautos mit und ohne ``Golfball''-Dellen.
Nehmen Sie an, es seien für den Verbrauch die folgenden Werte in ml/km
gemessen worden:
\begin{center}
\begin{tabular}{ccc}
\hline
Fahrt&Verbrauch mit Dellen&Verbrauch ohne Dellen\\
\hline
1&86&103\\
2&97&97\\
3&102&104\\
\hline
\end{tabular}
\end{center}
Kann man daraus schliessen, dass die Dellen eine Wirkung haben?

\thema{$t$-Test}
\thema{Hypothesentest}

\begin{loesung}
Wir testen die Nullhypothese, dass der mittlere Verbrauch in den
beiden Situationen gleich ist.
Können wir die Nullhypothese verwerfen, schliessen wir, dass die
Dellen tatsächlich einen Einfluss auf den Verbrauch haben.

Für den Test verwenden wir einen $t$-Test, ein Alpha von
$\alpha=0.05$ scheint angebracht.

Die mittleren Verbrauchswerte sind $\overline X=95\,\text{ml/km}$ mit und
$\overline{Y}=101.3\,\text{ml/km}$ ohne Dellen.
Die empirische Varianz ist $S_X^2 = 67$ bzw.~$S_Y^2 = 14.3$.
Daraus berechnet sich die Testgrösse für den $t$-Test nach
\[
T
=
\frac{95-101.3}{\sqrt{2\cdot 67+2\cdot 14.3}}
\sqrt{\frac{3\cdot 3\cdot(3+3-2)}{3+3}}
=
\frac{-6.3}{12.75408}\sqrt{6}
=
-1.21.
\]
Allerdings ist diese Abweichung kleiner als der kritische Wert $t_\text{krit}$
für einen (einseitigen und damit leichter zu bestehenden) $t$-Test mit
$3+3-2=4$ Freiheitsgraden und
$\alpha=0.05$: $t_{\text{krit}}=2.1318$
%[1]  86.13225  97.83724 102.09296
%> y <- rnorm(m = 110, sd = 12, m);
%> y
%[1] 102.75746  97.15253 104.07669
%> mean(x)
%[1] 95.35415
%> var(x)
%[1] 68.31045
%> mean(y)
%[1] 101.3289
%> var(y)
%[1] 13.51658
%> 
%> sx2 = var(x) 
%> sy2 = var(y)
%> 
%> sp2 = ((n- 1) * sx2 + (m-1) * sy2)/(n+m-2)
%> sp2
%[1] 40.91352
%> 
%> t = sqrt(n * m * (n+m-2)/(n+m)) * (mean(x) - mean(y)) / sqrt(sp2)
%> t
%[1] -2.288029
%> 
Die Daten zeigen also keinen genügend grossen Unterschied, dass man die
Nullhypothese verwerfen können.
Das Experiment der Mythbusters ist nicht aussagekräftig genug um zu
beweisen, dass Dellen im Auto den Benzinverbrauch verändern.
\end{loesung}

