Wir haben von den Mythbusters leider keine exakten Daten "uber den
Benzinverbrauch des Testautos mit und ohne ``Golfball''-Dellen.
Nehmen Sie an, es seien f"ur den Verbrauch die folgenden Werte in ml/km
gemessen worden:
\begin{center}
\begin{tabular}{ccc}
\hline
Fahrt&Verbrauch mit Dellen&Verbrauch ohne Dellen\\
\hline
1&86&103\\
2&97&97\\
3&102&104\\
\hline
\end{tabular}
\end{center}
Kann man daraus schliessen, dass die Dellen eine Wirkung haben?

\begin{loesung}
Die mittleren Verbrauchswerte sind 95ml/km mit und 101.3ml/km ohne
Dellen. Allerdings ist die Testgr"osse f"ur den $t$-Test
$T=1.21635$ und die ist gr"osser als der kritische Wert f"ur
einen einseitigen, und damit leichter zu bestehenden $t$-Test mit
$3+3-2=4$ Freiheitsgraden und
$\alpha=0.05$: $t_{\text{krit}}=2.1318$
%[1]  86.13225  97.83724 102.09296
%> y <- rnorm(m = 110, sd = 12, m);
%> y
%[1] 102.75746  97.15253 104.07669
%> mean(x)
%[1] 95.35415
%> var(x)
%[1] 68.31045
%> mean(y)
%[1] 101.3289
%> var(y)
%[1] 13.51658
%> 
%> sx2 = var(x) 
%> sy2 = var(y)
%> 
%> sp2 = ((n- 1) * sx2 + (m-1) * sy2)/(n+m-2)
%> sp2
%[1] 40.91352
%> 
%> t = sqrt(n * m * (n+m-2)/(n+m)) * (mean(x) - mean(y)) / sqrt(sp2)
%> t
%[1] -2.288029
%> 
\end{loesung}

