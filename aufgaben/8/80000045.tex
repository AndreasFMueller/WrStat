Ein Psychiater möchte wissen, ob die sozioökonomische Klasse einer
Person einen Einfluss auf die Wahrscheinlichkeit einer Selbstvergiftung
hat, wie er sie in seiner Praxis gezählt hat.
Er vergleicht dazu die Anzahl der Fälle
mit der Wahrscheinlichkeit einer Hospitalisierung auf Grund
eines Magen-Darm-Problems, wie sie von einem grossen Spital
ermittelt worden ist.
\begin{center}
\begin{tabular}{|>{$}c<{$}|>{$}c<{$}|>{$}c<{$}|}
\hline
\text{Klasse} &  \text{Selbstvergiftungen} & \text{Hospitalisierungswahrscheinlichkeit} \\
\hline
\text{I}      &           17         & \phantom{0}3.7\% \\
\text{II}     &           25         &           15.7\% \\
\text{III}    &           39         &           25.4\% \\
\text{IV}     &           52         &           36.6\% \\
\text{V}      &           32         &           18.6\% \\
\hline
\text{Total}  &          165         &                  \\
\hline
\end{tabular}
\end{center}
Kann man daraus schliessen, dass eine Abhängigkeit zwischen
sozioökonomischer Klasse und Selbstvergiftungswahrscheinlichkeit
besteht?

\begin{loesung}
Die Nullhypothese, die wir testen müssen ist:
\begin{quote}
Die beobachteten Selbstvergiftungszahlen weichen nicht wesentlich
von der Verteilung der Hospitalisierungen ab.
\end{quote}
Wir verwenden einen $\chi^2$-Test mit 4 Freiheitsgraden.
Ein sinnvoller Wert $\alpha=5\%$ scheint angemessen.
Der kritische Wert der Diskrepanz für $p=0.95$ und $k=4$ ist
$D_{\text{krit}}=9.488$. 
Die Nullhypothese wird verworfen, wenn die aus obigen Daten
berechnete Diskrepanz grösser ist als $D_{\text{krit}}$.

Berechnung der Diskrepanz:
\begin{center}
\begin{tabular}{|>{$}c<{$}|>{$}c<{$}|>{$}r<{$}>{$}r<{$}>{$}r<{$}>{$}r<{$}|}
\hline
i &  p_i   &   n_i  & np_i   &n_i-np_i& (n_i-npi)^2/np_i \\
\hline
1 & 0.037  &   17   &  6.105 & 10.895 &   19.44324 \\
2 & 0.157  &   25   & 25.905 & -0.905 &    0.03161 \\
3 & 0.254  &   39   & 41.910 & -2.910 &    0.20205 \\
4 & 0.366  &   52   & 60.390 & -8.390 &    1.16562 \\
5 & 0.186  &   32   & 30.690 &  1.310 &    0.05591 \\
\hline
  &        & n=165  &        &        & D=20.89846 \\
\hline
\end{tabular}
\end{center}
Da die Diskrepanz $D>D_{\text{krit}}$ klar grösser ist als die kritische
Diskrepanz $D_{\text{krit}}$ muss die Nullhypothese verworfen werden.
Es folgt also, dass es einen Zusammenhang zwischen sozioökonomischer 
Klasse und Selbstvergifgungen gibt.
\end{loesung}

\begin{bewertung}
Nullhypothese ({\bf H}) 1 Punkt,
Wahl von $\alpha$ ({\bf A}) 1 Punkt,
Freiheitsgrade und kritische Diskrepanz ({\bf F}) 1 Punkt,
Berechnung der Diskrepanz ({\bf D}) 2 Punkt,
Entscheid und Schlussfolgerung ({\bf S}) 1 Punkt.
\end{bewertung}

