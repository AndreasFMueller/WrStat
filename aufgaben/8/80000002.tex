Drei Spitzensportler, die besten ihrer Disziplin, liefern sich erbitterte
K"ampfe. Jeder ist ausserordentlich, aber keiner dominiert wirklich,
so wie es Roger Federer im Tennis oder Tiger Woods im Golf tun. Jeder
kann einige Siege f"ur sich verbuchen, muss aber auch Niederlagen
einstecken. Am Ende der Saison sieht der Medallienspiegel der drei
Sportler wie folgt aus
\begin{center}
\begin{tabular}{|c|c|}
\hline
Sportler&Anzahl Siege\\
\hline
A&12\\
B&14\\
C&7\\
\hline
\end{tabular}
\end{center}
Kann man behaupten, dass B besser ist als die beiden anderen?

\begin{loesung}
Wir testen die Hypothese, dass die Siegeswahrscheinlichkeit bei allen
drei Sportlern gleich gross also $p=\frac13$ ist. Der Test kann mit
dem $\chi^2$-Test durchgef"uhrt werden. Zur Berechnung der Diskrepanz
verwenden wir die Tabelle
\begin{center}
\begin{tabular}{|c|c|c|c|}
\hline
$i$&$n_i$&$n_i-np_i$&$D$\\
\hline
A&12&1&$\frac{1}{11}$\\
B&14&3&$\frac{9}{11}$\\
C& 7&-4&$\frac{16}{11}$\\
\hline
&33&&$2.363636$\\
\hline
\end{tabular}
\end{center}
F"ur zwei Freiheitsgrade und $\alpha=0.05$ ist der kritische Wert der
Diskrepanz $D_{\text{krit}}=5.991$. Da $D<D_{\text{krit}}$ gibt es
auf Grund der Datenlage keinen Grund daran zu zweifeln, dass alle drei
Sportler gleich stark sind.
\end{loesung}

