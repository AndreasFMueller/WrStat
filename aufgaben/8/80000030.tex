Eine Hochschule, die sich sehr um die Gesundheit ihrer Studenten sorgt,
befragte ihre Neueintretenden danach, wie häufig sie irgend eine Form von
Fitness-Training absolvierten:
\begin{center}
\begin{tabular}{|l|r|}
\hline
Trainingshäufigkeit&Studenten\\
\hline
kaum               &    60\% \\
gelegentlich       &    25\% \\
regelmässig        &    15\% \\
\hline
\end{tabular}
\end{center}
Etwas alarmiert hat die Hochschule daher eine Werbeinitiative gestartet, 
um den Studenten die Sportangebote näher zu bringen.
Um die Wirkung dieser Initative zu beurteilen, wurden die Studenten erneut
befragt, und es ergaben sich die folgenden Zahlen:
\begin{center}
\begin{tabular}{|l|r|}
\hline
Trainingshäufigkeit&Studenten\\
\hline
kaum               &    255  \\
gelegentlich       &    125  \\
regelmässig        &     90  \\
\hline
\end{tabular}
\end{center}
Hat die Werbeinitiative gewirkt?

\thema{Hypothesentest}

\begin{loesung}
Die Nullhypothese ``die Trainingshäufigkeit hat sich nicht verändert'' kann
mit einem $\chi^2$-Test getestet werden.
Dazu berechnen wir die Diskriminante mit folgender Tabelle
\begin{center}
\begin{tabular}{|l|>{$}r<{$}|>{$}r<{$}|>{$}r<{$}|>{$}r<{$}|>{$}r<{$}|}
\hline
$i$                &p_i &  n_i& np_i&n_i - np_i&(n_i-np_i)^2/np_i \\
\hline
\text{kaum}        &0.60&  255&282.0&     -27.0&   2.585\\
\text{gelegentlich}&0.25&  125&117.5&       7.5&   0.479\\
\text{regelmässig} &0.15&   90& 70.5&      19.5&   5.394\\
\hline
                   &    &n=470&     &          & D=8.457\\
\hline
\end{tabular}
\end{center}
Für $\alpha=5\%$ und die Anzahl $2$ der Freiheitsgrade finden wir für
den kritischen Wert der Diskrepanz: $D_{\text{krit}}=5.991$. 
Da $D>D_{\text{krit}}$ ist verwerfen wir die Nullhypothese und schliessen,
dass die Werbeinitiative erfolgreich war.
\end{loesung}

\begin{bewertung}
Hypothese ({\bf H}) 1 Punkt,
Wahl eines geeigneten $\alpha$ ({\bf A}) 1 Punkt,
Anzahl Freiheitsgrade ({\bf F}) 1 Punkt,
Berechnung der Diskrepanz ({\bf D}) 1 Punkt,
kritischer Wert für die Diskrepanz ({\bf K}) 1 Punkt,
Schlussfolgerung ({\bf S}) 1 Punkt.
\end{bewertung}


