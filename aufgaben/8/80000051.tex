Eine Marketingorganisation untersucht die Wirksamkeit verschiedener
Werbematerialien.
Zu diesem Zweck werden Kunden mit verschiedenen Arten von Werbematerialien
beschenkt und es wird gezählt,
wie viele Kunden sich zu einem Meeting mit
einem Verkäufer bewegen lassen.
In der folgenden Tabelle ist zusammengestellt, welcher Anteil der
Kunden mit welcher Art von Werbematrial beschenkt wurde und wieviele
sich zu einem Meeting bewegen liessen.
\begin{center}
\begin{tabular}{|l|>{$}r<{$}>{$}r<{$}|}
\hline
Werbematrial  &\text{Anteil der Kunden}&\text{Meeting}\\
\hline
Logo T-Shirt                          &      0.3342&     61\\
Geschenkkarte                         &      0.2426&     32\\
Flyers                                &      0.1806&     18\\
Deko-Gegenstand für den Schreibtisch  &      0.2426&     42\\
\hline
\end{tabular}
\end{center}
Kann man daraus einen Schluss darüber ziehen, ob eine Art von Werbematerial
besonders wirksam war?

\begin{loesung}
Wir führen einen $\chi^2$-Test für die Nullhypothese
\begin{quote}
Die Zahl der Meetings passt zur Verteilung der Werbematerialien in der
Kundschaft.
\end{quote}
durch.
\ainput{d.tex}
Wegen der geringen Zahl der Kunden ist ein $\alpha=5\%$ angemessen.
Die Anzahl der Meetings $n_i$ entnehmen wir der zweiten Spalte der Tabelle,
die Wahrscheinlichkeiten $p_i$ der dritten Spalte.
Damit lässt sich jetzt die Diskrepanz berechnen:
\begin{center}
\begin{tabular}{|>{$}r<{$}|>{$}r<{$}>{$}r<{$}>{$}r<{$}>{$}r<{$}>{$}r<{$}|}
\hline
i&     n_i&       p_i & np_i & n_i - np_i & (n_i-np_i)^2/np_i\\
\hline
\tabelle
\hline
\end{tabular}
\end{center}
Der kritische Wert der Diskrepanz für 3 Freiheitsgrade ist
$D_{\text{krit}}=7.815$.
Da $D=\D<D_{\text{krit}}$ ist, kann man die Nullhypothese nicht
verwerfen, es gibt keinen signifikanten Unterschied in der Wirksamkeit
der verschiedenen Werbematerialien.
\end{loesung}

\begin{bewertung}
Nullhypothese ({\bf H$\mathstrut_0$}) 1 Punkt,
$\chi^2$-Test ({\bf X}) 1 Punkt,
Berechnung der Diskrepanz ({\bf D}) 1 Punkt,
Wahl von $\alpha$ ({\bf A}) 1 Punkt,
Freiheitsgrade und kritischer Wert ({\bf K}) 1 Punkt,
Schlussfolgerung ({\bf S}) 1 Punkt.
\end{bewertung}

