In seinem Buch ``Analyzing Linguistic Data'' untersucht R.~H.~Baayen,
ob Männer und Frauen das {\em ont-}Präfix im Holländischen verschieden
lang aussprechen.
Als Grundlagen stellt er die folgenden Messresultate zusammen:
\begin{center}
\begin{tabular}{|l|>{$}r<{$}>{$}r<{$}>{$}r<{$}|}
\hline
&\text{Stichprobengrösse}&\text{Mittelwert [ms]}&\text{Standardabweichung [ms]}\\
\hline
Männer &     53&    144.362&     42.63\\
Frauen &     49&    153.657&     34.47\\
\hline
\end{tabular}
\end{center}
Kann man daraus schliessen, dass Frauen das Präfix länger aussprechen?

\begin{loesung}
Die beiden Mittelwerte können mit einem $t$-Test verglichen werden.
Die Nullhypothese dafür lautet
\begin{quote}
Männer und Frauen sprechen das Präfix gleich lang aus.
\end{quote}
Angesichts der kleinen Datenmenge ist $\alpha = 0.05$ angemessen.
Die Zahl der Freiheitsgrade ist $53+49-2=100$, dies führt auf einen
kritischen Wert $t_{\text{krit}}=1.9840$ für einen zweiseitigen Test passend
zur Formulierung der Nullhypothese.
Die Berechnung der Grösse $T$ ergibt
\begin{align*}
T
&=
\frac{\overline{X}-\overline{Y}}{\sqrt{(n-1)S_X^2+(m-1)S_Y^2}}
\sqrt{\frac{nm(n+m-2)}{n+m}}
\\
&=
\frac{144.362-153.657}{\sqrt{52\cdot 42.63^2 + 48\cdot 34.47^2}}
\sqrt{\frac{53\cdot 49\cdot 100}{102}}
=
-1.20484
\end{align*}
Da $|T|<t_{\text{krit}}$ ist, kann man die Nullhypothese nicht verwerfen und 
hat keine ausreichende Grundlage um zu schliessen, dass Männer und Frauen
das Präfix verschieden lang aussprechen.
\end{loesung}

\begin{bewertung}
$t$-Test ({\bf T}) 1 Punkt,
Nullhypothese ({\bf N}) 1 Punkt,
Wahl von $\alpha$ ({\bf A}) 1 Punkt,
Freiheitsgrade und kritischer Wert ({\bf K}) 1 Punkt,
Berechnung von $T$ ({\bf W}) 1 Punkt,
Schlussfolgerung von $S$ ({\bf S}) 1 Punkt.
\end{bewertung}



