Ein Wirebonder ist eine Maschine, die die Anschluss-Pads eines Chips
mit den Anschluss-Pins des Chip-Gehäuses verbindet.
Sie kann bis zu 40 Verbindungen pro Sekunde aus einem dünnen Golddraht
herstellen.
Leider sind nicht alle Verbindungen immer gut, manchmal haftet der
Draht nicht auf dem Pad oder er bricht.
Ein Chiphersteller verwendet Wirebonder von drei verschiedenen
Herstellern, die jedoch einen unterschiedlichen Anteil an der Produktion
haben.
Über die Fälle fehlerhafter Verbindungen wird Buch geführt:
\begin{center}
\begin{tabular}{|c|c|c|}
\hline
Wirebonder&Anteil&fehlerhafte Chips\\
\hline
$A$&40\%&92
\\
$B$&40\%&118
\\
$C$&20\%&60
\\
\hline
\end{tabular}
\end{center}
Auf Grund der hohen Zahl von fehlerhaften Chips, die der Wirebonder des
Herstellers $B$ produziert hat, beschliesst das Management, diese
Wirebonder zu ersetzen.
Ist diese Entscheidung rational gerechtfertigt?

\begin{loesung}
Wir überprüfen mit Hilfe eines $\chi^2$-Tests, ob die Anzahl der fehlerhaften
Chips ungewöhnlich von der auf Grund der Produktionsanteile zu erwartenden
Verteilung abweicht.
Die Nullhypothese lautet daher, dass die Anzahlen der fehlerhaften Chips
den Anteilen an der Produktion entsprechen.
Jetzt muss die Diskrepanz berechnet werden:
\begin{center}
\begin{tabular}{|l|>{$}r<{$}|>{$}r<{$}|>{$}r<{$}|>{$}r<{$}|>{$}r<{$}|}
\hline
Wirebonder&p_i&n_i&np_i&n_i-np_i&(n_i-np_i)^2/np_i
\\
\hline
$A$&0.4&   92&108&-16& 2.37
\\
$B$&0.4&  118&108& 10& 0.92
\\
$C$&0.2&   60& 54&  6& 0.66
\\
\hline
   &   &n=270&   &   &D=3.95
\\
\hline
\end{tabular}
\end{center}
Der kritische Wert für die Diskrepanz bei $3-1=2$ Freiheitsgraden ist
$D_{\text{krit}}=5.991$.
Da $D<D_{\text{krit}}$ kann der Test die Nullhypothese nicht verwerfen,
das Management hat also nicht rational entschieden.
\end{loesung}

