Das Universum dehnt sich aus mit einer Geschwindigkeit, welche
durch die sogenannte Hubble-Konstante $H_0$ gegeben wird.
Ein allgemein üblicher Wert ist etwa $70 \text{km}\,\text{s}^{-1}/\text{Mpc}$.
Ein Problem der modernen Kosmologie ist jedoch, dass es zwei verschiedene
Methoden gibt, $H_0$ zu messen.
Eine Methode verwendet Entfernungsmessungen zu weit entfernten Galaxien,
die sogenannte Entfernungsleiter.
Es ist aber auch möglich, $H_0$ aus Messungen des kosmischen
Mikrowellenhintergrundes zu bestimmen.
\begin{teilaufgaben}
\item
Alte Messungen basierend auf den Satelliten Hipparcos und WMAP ergaben
die Werte
\begin{center}
\begin{tabular}{|l|r|r|}
\hline
Methode               &Wert für $H_0\,[\text{km}\,\text{s}^{-1}/\text{Mpc}]$&Messfehler $\sigma\,[\text{km}\,\text{s}^{-1}/\text{Mpc}]$\\
\hline
Entfernungsleiter     &74.2&2.4\\
Mikrowellenhintergrund&70.5&2.5\\
\hline
\end{tabular}
\end{center}
Wie wahrscheinlich ist es, dass dieser Unterschied rein zufällig durch
die Messfehler verursacht worden ist?
Ist der Unterschied signifikant?
\item
Für lange Zeit gaben die unterschiedlichen Resultate, die man aus diesen
beiden Methoden erhalten hat, nicht zu Bedenken Anlass.
Neuere Präzisionsmessungen zum Beispiel mit dem GAIA-Satelliten für 
die Entfernungsmessung oder dem Planck-Satelliten für die kosmische
Hintergrundstrahlung haben die folgenden Werte ergeben:
\begin{center}
\begin{tabular}{|l|r|r|}
\hline
Methode               &Wert für $H_0\,[\text{km}\,\text{s}^{-1}/\text{Mpc}]$&Messfehler $\sigma\,[\text{km}\,\text{s}^{-1}/\text{Mpc}]$\\
\hline
Entfernungsleiter     &73\phantom{.0}&1.8\\
Mikrowellenhintergrund&67.3&1.2\\
\hline
\end{tabular}
\end{center}
Wie gross ist die Wahrscheinlichkeit dafür, dass ein derart grosser
Unterschied rein zufällig entstehen könnte?
\end{teilaufgaben}

\thema{Hypothesentest}
\thema{Normalverteilung}
\thema{Standardisierung}

\begin{loesung}
In beiden Teilaufgaben führen wir die gleiche Rechnung durch mit
verschiedenen Zahlen.
Die beiden Messwerte sind normalverteilte Zufallsgrössen $X_1$ und $X_2$ 
mit bekannten Erwartungswerten $\mu_1$ und $\mu_2$ und
mit Standardabweichungen $\sigma_1$ und $\sigma_2$.

Die zugrundeliegende Hypothese ist, dass die beiden Messprozesse den
gleichen Wert für die Hubble-Konstante ergeben, dass also 
$E(X_1-X_2)=0$ sein sollten.
Die Differenz $Y=X_1-X_2$ ist ebenfalls normalverteilt, die Standardabweichung
ist 
$\sigma=\sqrt{\sigma_1^2+\sigma_2^2}$.

In den Daten beobachten wir, dass $Y=X_1-X_2$ mindestens in b) relativ 
gross ist.
Es stellt sich die Frage, wie wahrscheinlich die beobachteten grossen Werte
von $Y$ unter diesen Hypothesen sind.
Gesucht ist also die Wahrscheinlichkeit $ P(Y > \mu_1-\mu_2).  $
Zur Abkürzung schreiben wir $\mu=\mu_1-\mu_2$ und Formen um
\begin{align*}
P(Y> \mu)
&=
1-P(Y\le \mu)
\\
&=
1-P\biggl(\frac{X-0}{\sigma} < \frac{\mu-0}{\sigma}\biggr)
\\
&=
1-P(Z < \mu/\sigma)
\end{align*}
In der zweiten Zeile haben wir standardisiert und Verwendet, dass
$E(Y)=0$ ist.
\begin{teilaufgaben}
\item
In diesem Fall ist $\mu_1=74.2$, $\mu_2=70.5$, $\mu=3.7$,
$\sigma_1=2.4$, $\sigma_2=2.5$, $\sigma=3.465$ und damit die
gesuchte Wahrscheinlichkeit
\[
P(Y>\mu)
=1-P(Z<1.50) = 1-0.933 = 0.067.
\]
Die Wahrscheinlichkeit ist also $>5\%$, der Unterschied ist damit
nicht signifikant.
\item
In diesem Fall ist $\mu = 5.4$ und $\sigma=2.163$ und damit
\[
P(Y>\mu)
=1-P(Z<2.50) = 1-0.9938 = 0.0062.
\]
Die Wahrscheinlichkeit ist also sehr klein, dass dieser Unterschied
zufällig ist.
Dieser Unterschied ist signifikant.
\qedhere
\end{teilaufgaben}
\end{loesung}

\begin{diskussion}
Mehr Information über das Problem der Diskrepanz der Hubble-Konstanten,
die sogenannte {\em Hubble-Tension}, kann man im
Youtube-Video \url{https://www.youtube.com/watch?v=uoAkFq-KIrk}
finden.
\end{diskussion}

\begin{bewertung}
Normalverteilung ({\bf N}) 1 Punkt,
Standardisierung ({\bf S}) 1 Punkt,
Differenz hat Erwartungswert 0 ({\bf D}) 1 Punkt,
Varianz der Differenz ({\bf V}) 1 Punkt,
Wahrscheinlichkeit in a) ({\bf A}) 1 Punkt,
Wahrscheinlichkeit in b) ({\bf B}) 1 Punkt.
\end{bewertung}
