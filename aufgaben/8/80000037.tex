Beim Roulette wird die Null speziell behandelt, weil sie zum Beispiel
weder als gerade noch als ungerade gilt und auch kein Gewinn über
die Einsätze auf rot oder schwarz möglich sind.
Ohne die Null wäre die Gewinnerwartung im Roulette-Spiel 0,
dank der Null wird die Gewinnerwartung für den Mitspieler negativ
und für das Casino positiv. 
In einigen Varianten des Spiels wird diese Balance durch die Doppelnull
noch weiter zu Gunsten des Casinos verschoben.
Ein betrügerisches Casino könnte also versucht sein, durch Manipulation
des Roulettekessels die Wahrscheinlichkeit für die Null zu erhöhen.
Was muss ein Ankläger tun, wenn er eine solche Manipulation
aus Daten über die Häufigkeit der gespielten Zahlen nachweisen möchte?
Nehmen Sie an, dass dem Ankläger die Resultate von $n=10000$ Spielen
zur Verfügung stehen.
Wir gross ist die Wahrscheinlichkeit, dass der Beweis des Anklägers
ein Zufallsresultat ist?

\begin{loesung}
Der Ankläger muss einen Hypothesentest für die Anzahl $k$ der gespielten
Nullen in $n$ Spielen durchführen.
Die Nullhypothese besagt, dass die Null Wahrscheinlichkeit $p=1/37$ hat.
Die Zahl der Nullen ist binomialverteilt.
Der erwartete Wert für die Anzahl der Nullen ist $\mu=np$.
Für eine grosse Anzahl $n$ von Versuchen kann man die Binomialverteilung
mit der Normalverteilung Approximation mit $\sigma = \sqrt{np(1-p)}$ 
approximieren.

Wählt man $\alpha=0.01$, dann ist diese auch gleichzeitig die
Wahrscheinlichkeit dafür, dass der Ankläger einen Fehler erster Art
begeht.
Eine Zahl $k$ ausserhalb des Intervals
\[
[ \mu - 2.5758\,\sigma, \mu + 2.5758\,\sigma]
=
[228.50, 312.04]
\]
ist ein Beweis im Sinne eines Hypothesentests.
\end{loesung}


