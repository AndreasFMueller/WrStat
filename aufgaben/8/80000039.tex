In einem Projekt muss zwischen zwei Methoden der Positionsbestimmung
entschieden werden.
Die beiden Methoden sind verschieden genau, man kennt allerdings nur
empirische Werte für den Fehler.
Bei der genaueren Methode gibt es Hinweise darauf, dass sie möglicherweise einen
noch nicht verstandenen systematischen Fehler aufweist.
Um diese Frage zu klären, werden zwei Versuchsreihen von je 10 Messungen
gemacht.
Die gemessenen Mittelwerte und empirischen Standardabweichungen sind
\begin{center}
\begin{tabular}{|c|cc|}
\hline
Methode&Mittelwert&Standardabweichung\\
\hline
A      & 3.1469   &   0.8002         \\
B      & 4.0772   &   1.4323         \\
\hline
\end{tabular}
\end{center}
Es wird erkennbar, dass A genauer sein dürfte, der Unterschied zwischen
den mittleren Positionen ist allerdings nicht sehr gross.
Kann man auf Grund dieser Daten schliessen, dass es tatsächlich einen
Unterschied der ermittelten Positionen gibt?

% m1 =  3.1469
% s1 =  0.80018
% m2 =  4.0772
% s2 =  1.4323
% sp =  1.1601
% k =  18
% t = -1.7930
% tkrit =  1.7341
% tkrittwosided =  2.1009
% s =  0.51881
% nkrit =  1.0169

\begin{loesung}
Die gestellte Frage ist ein Test für die Nullhypothese 
\begin{quote}
Es gibt keinen Unterschied zwischen den Mittelwerten der
beiden Messmethoden.
\end{quote}
Diese Hypothese kann mit Hilfe eines $t$-Tests getestet werden.
Dazu muss die Teststatistik
\begin{align*}
T
&=
\frac{\bar{X}-\bar{Y}}{\sqrt{(n-1)S_X^2 + (m-1)S_Y^2}}
\sqrt{\frac{nm(n+m-2)}{n+m}}
=-1.7930
\end{align*}
berechnet werden.
Die Anzahl der Freiheitsgrade ist $k=n+m-2 = 18$, der zugehörige kritische
Wert für $\alpha=5\%$ ist $t_{\text{krit}}=1.7341$ für einen
einseitigen Test und $t_{\text{krit}}=2.1009$ für einen zweiseitigen Test.
Im vorliegenden Fall ist $|T|<t_{\text{krit}}$ für den zweiseitigen Test,
die verfügbaren Daten gestatten also nicht, die Nullhypothese zu verwerfen.

Für den einseitigen Test ist der kritische Wert überschritten, man könnte
also versucht sein, die Nullhypothese zu verwerfen.
Dies ist aber nicht korrekt, weil die Nullhypothese nichts darüber
sagt, ab $T$ positiv oder negativ sei, der Test muss also beide
Vorzeichen berücksichtigen.

Im Falle des einseitigen Tests ist $|T|$ zwar grösser als der kritische
Wert, aber nur wenig.
Dies ist ein Hinweis darauf, dass das Resultat möglicherweise ohnehin
ein Zufallsresultat ist.
\end{loesung}

\begin{bewertung}
Nullhypothese ({\bf N}) 1 Punkt,
$t$-Test ({\bf T}) 1 Punkt,
Freiheitsgrade ({\bf F}) 1 Punkt,
Wahl eines $\alpha$ und kritischer Wert für $T$ ({\bf K}) 1 Punkt,
Berechnung des $T$-Wertes ({\bf D}) 1 Punkt,
Schlussfolgerung ({\bf S}) 1 Punkt.
\end{bewertung}


