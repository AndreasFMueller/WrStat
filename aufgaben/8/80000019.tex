Indem man eine Kugel an sieben ``gleichmässig'' verteilten Seiten
abflacht, kann man einen ``Siebnerwürfel'' herstellen:
\begin{center}
\includeagraphics[width=0.5\hsize]{wuerfel.jpg}
\end{center}
Auf \url{http://www.thingiverse.com:93539} findet man ein File, mit dessen
Hilfe man diesen Würfel mit dem 3D-Drucker herstellen kann.
Beim Ausprobieren des Würfels in 210 Würfen wurden folgende Resultate
gezählt:
\begin{center}
\begin{tabular}{|l|rrrrrrr|}
\hline
Augenzahl    &  1&  2&  3&  4&  5&  6&  7\\
\hline
Anzahl Würfe& 26& 30& 29& 34& 31& 30& 30\\
\hline
\end{tabular}
\end{center}
Ist dieser Würfel fair?

\begin{loesung}
Wir machen einen $\chi^2$-Test mit der Nullhypothese: ``alle Augenzahlen
sind gleich wahrscheinlich''. Die sieben möglichen Versuchsausgänge
haben die Wahrscheinlichkeiten $p_1=p_2=\dots=p_7=\frac17$. Die 
Berechnung der Diskrepanz ergibt:
\begin{center}
\begin{tabular}{|>{$}c<{$}|>{$}c<{$}|>{$}r<{$}|>{$}c<{$}|>{$}r<{$}|>{$}r<{$}|}
\hline
i&   p_i&    n_i& np_i&n_i-np_i&(n_i-np_i)^2/np_i\\
\hline
1&0.0333&     26&   30&      -4&    0.5333\\
2&0.0333&     30&   30&       0&    \\
3&0.0333&     29&   30&      -1&    0.0333\\
4&0.0333&     34&   30&       4&    0.5333\\
5&0.0333&     31&   30&       1&    0.0333\\
6&0.0333&     30&   30&       0&    \\
7&0.0333&     30&   30&       0&    \\
\hline
 &      &n = 210&     &        &D=   1.1333\\
\hline
\end{tabular}
\end{center}
Wir wählen für den Test $\alpha=5\%$, der kritische Wert der
$\chi^2$-Verteilung mit $7-1=6$ Freiheitsgraden ist $D_{\text{krit}}=12.592$.
Da $D<D_{\text{krit}}$ gibt es keinen Grund, an der Hypothese zu
zweifeln. Nach aktuellem Wissenstand ist der Siebnerwürfel also fair.

Die Diskrepanz ist in diesem Fall sogar kleiner als die $p=0.05$ Quantile,
der kritische Wert dafür ist $1.635$. Wegen $D<1.635$ kann man also
mit Wahrscheinlichkeit $>0.95$ sogar vermuten, dass die Zahlen
fabriziert waren. Das waren sie aber nicht, die Zahlen sind mit einem
eigens zu diesem Zweck ausgedruckten Würfel erwürfelt worden.
\end{loesung}

\begin{bewertung}
Hypothesentest mit $\chi^2$ ({\bf X}) 1 Punkt,
Hypothese/Nullhypthese ({\bf H}) 1 Punkt,
Wahl eines geeigneten $\alpha$ ({\bf A}) 1 Punkt,
Berechnung der Diskrepanz ({\bf D}) 1 Punkt,
Ermittlung der kritischen Diskrepanz aus der $\chi^2$-Tabelle ({\bf K})
1 Punkt (richtige Anzahl Freiheitsgrade),
Schlussfolgerung ({\bf S}) 1 Punkt.
\end{bewertung}


