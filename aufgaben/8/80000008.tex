Ein Zufallszahlgenerator f"ur im Interval $[0,1]$ gleichverteilte
Zufallszahlen hat die Zahlen
\begin{center}
\begin{tabular}{ccc}
0.27667& 0.00040& 0.43987\\
0.12173& 0.00412& 0.03695\\
0.00741& 0.20949& 0.00398
\end{tabular}
\end{center}
erzeugt.
\begin{teilaufgaben}
\item Ist dies ein guter Zufallszahlgenerator?
\item Angenommen, der Zufallszahlgenerator w"are von h"ochster
Qualit"at, dann kann es trotzdem vorkommen, dass er beim oben
skizzierten Tests mit neun Zufallszahlen durchf"allt.
Wie oft erwarten Sie dies bei $n$ Durchf"uhrungen des Tests?
\end{teilaufgaben}

\begin{loesung}
\begin{teilaufgaben}
\item
Wir testen die Zahlen mit einem Kolmogorov-Smirnov-Test.
F"ur $n=9$ und $\alpha=0.01$ ist der kritische Wert
$K_{\text{krit}}=1.43878$. Die Werte von
$K_n^-$ und $K_n^+$ sind:
\begin{center}
\begin{tabular}{|rrrr|}
\hline
$j$&$x_j$&$\frac{j}n-F(x_j)$&$F(x_j)-\frac{j-1}n$\\
\hline
$1$&$0.00040$&$0.1107111$&$ 0.0004000$\\
$2$&$0.00398$&$0.2182422$&$-0.1071311$\\
$3$&$0.00412$&$0.3292133$&$-0.2181022$\\
$4$&$0.00741$&$0.4370344$&$-0.3259233$\\
$5$&$0.03695$&$0.5186056$&$-0.4074944$\\
$6$&$0.12173$&$0.5449367$&$-0.4338256$\\
$7$&$0.20949$&$0.5682878$&$-0.4571767$\\
$8$&$0.27667$&$0.6122189$&$-0.5011078$\\
$9$&$0.43987$&$0.5601300$&$-0.4490189$\\
\hline
&&$\operatorname{max}=0.61222$&$\operatorname{max}= 0.00040$\\
&&$K_n^+=1.83666$&$K_n^-= 0.00127$\\
\hline
\end{tabular}
\end{center}
Da $K_n^+>K_{\text{krit}}$ muss die Hypothese verworfen werden, dass
die Zufallszahlen gleichverteilt seien.
\item Die Wahrscheinlichkeit f"ur ein negatives Testresultat unter der
Annahme von Gleichverteilung ist $n\alpha$.
\end{teilaufgaben}
Die Berechnung kann nat"urlich auch mit R erfolgen. Wenn man die Daten
als CSV File der Form
\verbatimainput{ks.csv}
vorliegen hat, kann man die Berechnung der $K$-Werte mit folgendem
R-Skript durchf"uhren:
\verbatimainput{ks.R}
Der Output ist:
\verbatimainput{ks.log}
\end{loesung}

