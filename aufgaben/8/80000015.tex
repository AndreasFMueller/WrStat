Das schweizer Zahlenlotto publiziert die Häufigkeit der einzelnen
Zahlen auf seiner Website \url{http://www.swisslos.ch}. Zum Beispiel
wurden die Glückszahlen mit folgender Häufigkeit gezogen:
\begin{center}
\begin{tabular}{|c|c|}
\hline
Zahl&Häufkeit\\
\hline
1&11\\
2&12\\
3&11\\
4&13\\
5& 9\\
6&18\\
\hline
\end{tabular}
\end{center}
Man stellt fest, dass $6$ doppelt so häufig ist wie $5$. Kann man schliessen,
dass die Ziehungen unfair sind?

\begin{loesung}
Man kann die beobachtete Verteilung mit einem $\chi^2$-Test
gegen die Nullhypothese testen, die besagt,
dass alle Zahlen die gleiche Wahrscheinlichkeit $p=\frac16$ haben.
Die Berechnung der Diskrepanz ergibt
\begin{center}
\begin{tabular}{|>{$}c<{$}|>{$}r<{$}|>{$}c<{$}|>{$}c<{$}|>{$}r<{$}|>{$}r<{$}|}
\hline
i&n_i&   p_i&   np_i&n_i - np_i&(n_i-np_i)^2/np_i\\
\hline
1& 11&0.1666&12.3333&   -1.3333&      0.144144144\\
2& 12&0.1666&12.3333&   -0.3333&      0.009009009\\
3& 11&0.1666&12.3333&   -1.3333&      0.144144144\\
4& 13&0.1666&12.3333&    0.3333&      0.036036036\\
5&  9&0.1666&12.3333&   -3.3333&      0.900900900\\
6& 18&0.1666&12.3333&    5.6666&      2.603603604\\
\hline
 &74&      &        &          & D=   3.837837837\\
\hline
\end{tabular}
\end{center}
Die kritische Diskrepanz für $6-1=5$ Freiheitsgrade und $\alpha = 0.05$
ist $D_{\text{krit}}=11.070$. Da $D<D_{\text{krit}}$ gibt es keinen Grund,
an der Nullhypothese zu zweifeln.
\end{loesung}

\begin{diskussion}
Der analoge Test lässt sich auch für die Lottozahlen durchführen.
Als Diskrepanz findet man $D=47.3473$, aber der kritische Wert der
Diskrepanz für 41 Freiheitsgrade ist $D_{\text{krit}}=56.9424$.
Auch für die Lottozahlen gibt es keinen Grund, an der Hypothese zu
zweifeln, dass diese gleichverteilt sind.
\end{diskussion}
