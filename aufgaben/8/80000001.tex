Ein Forscher behauptet, vom Menschen verbreitete Umweltgifte h"atten die
Tierpopulationen dreier Arten $A$, $B$ und $C$ unterschiedlich stark
dezimiert. Er pr"asentiert dazu
folgende Daten, einerseits die relativen H"aufigkeiten der Arten,
wie sie aus langj"ahrigen Statistiken bekannt sind, und
andereseits seine neuesten Z"ahlungen:
\begin{center}
\begin{tabular}{|c|c|c|}
\hline
Art&rel.~H"aufigkeit&gez"ahlt\\
\hline
$A$&0.3&47\\
$B$&0.5&75\\
$C$&0.2&17\\
\hline
\end{tabular}
\end{center}
Er sagt, man s"ahe den Zahlen sofort an, dass die Art $C$ "uberproportional
stark betroffen sei.
Hat er recht?

\begin{loesung}
Hier geht es um die Hypothese ``Tierpopulationen wurden unterschiedlich
stark zerst"ort''. Die Nullhypothese dazu ist ``die Tierpopulationen
haben sich in Ihren H"aufigkeiten nicht ver"andert''. Die Hypothese
besagt also, dass die relativen H"aufigkeiten immer noch die gleichen
sind, $0.3$ f"ur $A$, $0.5$ f"ur $B$ und $0.2$ f"ur $C$.
Wir testen diese Hypothese auf dem Niveau $\alpha=0.05$ mit Hilfe eines
$\chi^2$-Testes.

Dazu ist die Diskrepanz zu berechnen, offenbar wurden $n=139$ Exemplare
gez"ahlt:
\begin{center}
\begin{tabular}{|c|c|c|r|r|r|}
\hline
Art&$p_i$&   $n_i$&$np_i$&$n_i-np_i$&$(n_i-np_i)^2/np_i$\\
\hline
$A$& 0.3 &     47 & 41.7 &    5.3   &   0.673621103     \\
$B$& 0.5 &     75 & 69.5 &    5.5   &   0.435251799     \\
$C$& 0.2 &     17 & 27.8 & $-10.8$  &   4.195683453     \\
\hline
   &     &$n=139$ &      &          &$D=5.304556355$    \\
\hline
\end{tabular}
\end{center}
Das Problem hat $k=2$ Freiheitsgrade, die
kritische Diskrepanz $D_{\text{krit}}$, bei der die Hypothese
verworfen werden m"usste, ist $D_{2,\alpha}=5.991$. Da die
beobachtete Diskrepanz kleiner ist, kann man die Nullhypothese
nicht verwerfen. Es gibt also noch keinen Beweis f"ur die
Behauptung des Forschers.
\end{loesung}
