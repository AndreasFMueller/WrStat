Nach dem Sugus-Experiment hat ein Student festgestellt, dass bei den
Fizzers m"oglicherweise Diskriminierung vorliegt. Inzwischen wurden
$n=855$ Fizzers auszgez"ahlt mit folgendem Resultat
\begin{center}
\begin{tabular}{|l|r|}
\hline
Farbe&Anzahl\\
\hline
weiss&187\\
gelb&141\\
gr"un&136\\
orange&167\\
pink&128\\
rot&96\\
\hline
\end{tabular}
\end{center}
Was schliessen Sie daraus?

\begin{loesung}
Die Durchf"uhrung des $\chi^2$-Test f"ur diese Daten ergibt
einen Diskrepanz-Wert von $D=35.0702$. Der kritische Diskrepanz-Wert
f"ur $\alpha = 0.05$ und $5$ Freiheitsgrade ist $11.070$. Da
$D>D_{\text{krit}}$ muss die Hypothese verworfen werden.

Mit R kann der $\chi^2$-Test mit der Funktion {\tt chisq.test} durchgef"uhrt
werden:
\verbatimainput{fizzers.txt}
Da der $p$-Value Wert kleiner als $\alpha$ ist, muss die
Null-Hypothese verworfen werden.
\end{loesung}
