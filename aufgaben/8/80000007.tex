Nach dem Sugus-Experiment hat ein Student festgestellt, dass bei den
Fizzers möglicherweise Diskriminierung vorliegt. Inzwischen wurden
$n=855$ Fizzers auszgezählt mit folgendem Resultat
\begin{center}
\begin{tabular}{|l|r|}
\hline
Farbe&Anzahl\\
\hline
weiss&187\\
gelb&141\\
grün&136\\
orange&167\\
pink&128\\
rot&96\\
\hline
\end{tabular}
\end{center}
Was schliessen Sie daraus?

\thema{Hypothesentest}
%\thema{$\chi^2$-Test}

\begin{loesung}
Die Durchführung des $\chi^2$-Test für diese Daten ergibt
einen Diskrepanz-Wert von $D=35.0702$. Der kritische Diskrepanz-Wert
für $\alpha = 0.05$ und $5$ Freiheitsgrade ist $11.070$. Da
$D>D_{\text{krit}}$ muss die Hypothese verworfen werden.

Mit R kann der $\chi^2$-Test mit der Funktion {\tt chisq.test} durchgeführt
werden:
\verbatimainput{fizzers.txt}
Da der $p$-Value Wert kleiner als $\alpha$ ist, muss die
Null-Hypothese verworfen werden.
\end{loesung}
