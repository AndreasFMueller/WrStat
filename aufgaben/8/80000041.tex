Eine vorläufige chinesische Studie stellte bei 2000 CoVID-19 Patienten
die folgende Verteilung der Blutgruppen $A$, $B$, $AB$ und $0$ fest:
\begin{center}
\begin{tabular}{|>{$}c<{$}|>{$}r<{$}|>{$}c<{$}|}
\hline
\text{Blutgruppe}&\text{Häufigkeit}&\text{CoVID-19 Fälle}\\
\hline
A &           32.16\,\% &  755 \\
B &           24.90\,\% &  528 \\
AB& \phantom{0}9.10\,\% &  201 \\
0 &           33.84\,\% &  516 \\
\hline
\end{tabular}
\end{center}
In der Spalte ``Häufigkeit'' stehen die Häufigkeiten der Blutgruppen
in der Gesamtbevölkerung.
Wie aussagekräftig ist dieses Resultat?

\thema{Hypothesentest}
\themaL{chi2Test}{$\chi^2$-Test}

\begin{loesung}
Die Nullhypothese ist, dass sich die Blutgruppenzahlen der CoVID-19-Fälle
nicht von denen unterscheiden, die man aufgrund der Blutgruppenhäufigkeit
in der Bevölkerung erwarten würde.
Diese Hypothese kann man mit einem $\chi^2$-Test testen.
Da es sich um eine vorläufige Studie mit einer relativ geringen Zahl
von Fällen ($n=2000$) handelt, ist $\alpha=5\,\%$ angemessen.
Dazu berechnet man die Diskrepanz
\begin{center}
\begin{tabular}{|>{$}c<{$}|>{$}r<{$}|>{$}r<{$}|>{$}r<{$}|>{$}r<{$}|}
\hline
\text{Blutgruppe $i$} & n_i   & p_i    & np_i   & (n_i-np_i)^2/np_i \\
\hline
A &    755 & 0.3216 & 643.2 & 19.43 \\
B &    528 & 0.2490 & 498.0 &  1.81 \\
AB&    201 & 0.0910 & 182.0 &  1.98 \\
0 &    516 & 0.3384 & 676.8 & 38.20 \\
\hline
  & n=2000 &        &       & D = 61.43 \\
\hline
\end{tabular}
\end{center}
Für $k=3$ Freiheitsgrade und $\alpha=5\,\%$ ist der kritische Wert der 
Diskrepanz $D_{\text{krit}}=7.815$.
Da $D_{\text{krit}}$ viel kleiner ist als $D$, muss man schliessen, dass
die Nullhypothese verworfen werden muss.
Es scheint also einen Zusammenhang zwischen Blutgruppenverteilung und
Corona-Anfälligkeit zu geben.
\end{loesung}

\begin{bewertung}
Nullhypothese ({\bf N}) 1 Punkt,
$\chi^2$-Text ({\bf X}) 1 Punkt,
Wahl von $\alpha=0.05$ ({\bf A}) 1 Punkt,
Freiheitsgrade und kritische Diskrepanz ({\bf F}) 1 Punkt,
Berechnung der Diskrepanz ({\bf D}) 1 Punkt,
Schlussfolgerung ({\bf F}) 1 Punkt.
\end{bewertung}
