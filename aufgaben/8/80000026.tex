In einer kleinen Gemeinde waren bei den vorletzten Wahlen die
Stimmen f"ur die f"unf Parteien wie folgt wie folgt verteilt:
\begin{center}
\begin{tabular}{ll|r}
Partei                   &    &Prozentsatz\\
\hline
Extrem linke Partei      &ELP & 10\%\\
Nicht so linke Partei    &NLP & 25\%\\
Genau in der Mitte Partei&GIMP& 13\%\\
Eher rechte Partei       &ERP & 21\%\\
Total rechte Partei      &TRP & 31\%\\
\hline
\end{tabular}
\end{center}
Bei den n"achsten Wahlen wurden folgende Stimmen gez"ahlt:
\begin{center}
\begin{tabular}{ll|rr}
Partei                   &    &Stimmen&Anteil\\
\hline
Extrem linke Partei      &ELP &  6    & 8.4\%\\
Nicht so linke Partei    &NLP & 14    &19.7\%\\
Genau in der Mitte Partei&GIMP&  8    &11.3\%\\
Eher rechte Partei       &ERP & 18    &25.4\%\\
Total rechte Partei      &TRP & 25    &35.2\%\\
\hline
\end{tabular}
\end{center}
Offenbar haben die rechten Parteien dazugewonnen, und die linken
eher verloren.
Kann man auf Grund dieser Zahlen von einem Rechtsrutsch sprechen?

\begin{loesung}
Die erste Tabelle liefert die Wahrscheinlichkeit, dass ein Stimmb"urger
f"ur eine gewisse Partei stimmt.
Es wird die Frage gestellt, ob die letzten Wahlen signifikant von dieser
Wahrscheinlichkeitsverteilung abweichen.
Tun sie es, kann man tats"achlich davon sprechen, dass unter den
Stimmb"urgern ein Sinneswandel stattgefunden hat, andernfalls handelt
es sich um eine zuf"allige Verschiebung.
Ob die Stimmenzahlen zur Verteilung passen, kann mit einem $\chi^2$-Test
beantwortet werden.
Dazu berechnen wir die Diskrepanz:
\begin{center}
\begin{tabular}{|l|>{$}r<{$}>{$}r<{$}>{$}r<{$}>{$}r<{$}|>{$}r<{$}|}
\hline
$i$ &p_i & n_i&np_i &n_i-np_i&(n_i-np_i)^2/np_i\\
\hline
ELP &0.10&   6& 7.10&   -1.10& 0.170423        \\
NLP &0.25&  14&17.75&   -3.75& 0.792254        \\
GIMP&0.13&   8& 9.23&   -1.23& 0.163911        \\
ERP &0.21&  18&14.91&    3.09& 0.640382        \\
TRP &0.31&  25&22.01&    2.99& 0.406184        \\
\hline
    &    &n=71&     &        &D=2.173151       \\
\hline
\end{tabular}
\end{center}
F"ur $\alpha=5\%$ liest man aus der Tabelle der $\chi^2$-Verteilung
f"ur $k=5-1=4$ Freiheitsgrade einen kritischen Wert $D_{\text{krit}}=9.488$
f"ur die Diskrepanz ab.
Da $D<D_{\text{krit}}$ folgt, dass es keinen Hinweis darauf gibt, dass
sich die Wahrscheinlichkeiten seit den vorletzten Wahlen ver"andert haben.
Von einem Rechtsrutsch kann also keinesfalls die Rede sein.
\end{loesung}
