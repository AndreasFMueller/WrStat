In einer kleinen Gemeinde waren bei den vorletzten Wahlen die
Stimmen für die fünf Parteien wie folgt verteilt:
\begin{center}
\begin{tabular}{ll|r}
Partei                   &    &Prozentsatz\\
\hline
Extrem linke Partei      &ELP & 10\%\\
Nicht so linke Partei    &NLP & 25\%\\
Genau in der Mitte Partei&GIMP& 13\%\\
Eher rechte Partei       &ERP & 21\%\\
Total rechte Partei      &TRP & 31\%\\
\hline
\end{tabular}
\end{center}
Bei den nächsten Wahlen wurden folgende Stimmen gezählt:
\begin{center}
\begin{tabular}{ll|rr}
Partei                   &    &Stimmen&Anteil\\
\hline
Extrem linke Partei      &ELP &  6    & 8.4\%\\
Nicht so linke Partei    &NLP & 14    &19.7\%\\
Genau in der Mitte Partei&GIMP&  8    &11.3\%\\
Eher rechte Partei       &ERP & 18    &25.4\%\\
Total rechte Partei      &TRP & 25    &35.2\%\\
\hline
\end{tabular}
\end{center}
Offenbar haben die rechten Parteien dazugewonnen, und die linken
eher verloren.
Kann man auf Grund dieser Zahlen von einem Rechtsrutsch sprechen?

\thema{Hypothesentest}
\themaL{chi2Test}{$\chi^2$-Test}

\begin{loesung}
Die erste Tabelle liefert die Wahrscheinlichkeit, dass ein Stimmbürger
für eine gewisse Partei stimmt.
Es wird die Frage gestellt, ob die letzten Wahlen signifikant von dieser
Wahrscheinlichkeitsverteilung abweichen.
Tun sie es, kann man tatsächlich davon sprechen, dass unter den
Stimmbürgern ein Sinneswandel stattgefunden hat, andernfalls handelt
es sich um eine zufällige Verschiebung.
Ob die Stimmenzahlen zur Verteilung passen, kann mit einem $\chi^2$-Test
beantwortet werden.
Dazu berechnen wir die Diskrepanz:
\begin{center}
\begin{tabular}{|l|>{$}r<{$}>{$}r<{$}>{$}r<{$}>{$}r<{$}|>{$}r<{$}|}
\hline
$i$ &p_i & n_i&np_i &n_i-np_i&(n_i-np_i)^2/np_i\\
\hline
ELP &0.10&   6& 7.10&   -1.10& 0.170423        \\
NLP &0.25&  14&17.75&   -3.75& 0.792254        \\
GIMP&0.13&   8& 9.23&   -1.23& 0.163911        \\
ERP &0.21&  18&14.91&    3.09& 0.640382        \\
TRP &0.31&  25&22.01&    2.99& 0.406184        \\
\hline
    &    &n=71&     &        &D=2.173151       \\
\hline
\end{tabular}
\end{center}
Für $\alpha=5\%$ liest man aus der Tabelle der $\chi^2$-Verteilung
für $k=5-1=4$ Freiheitsgrade einen kritischen Wert $D_{\text{krit}}=9.488$
für die Diskrepanz ab.
Da $D<D_{\text{krit}}$ folgt, dass es keinen Hinweis darauf gibt, dass
sich die Wahrscheinlichkeiten seit den vorletzten Wahlen verändert haben.
Von einem Rechtsrutsch kann also keinesfalls die Rede sein.
\end{loesung}

\begin{bewertung}
$\chi^2$-Test ({\bf X}) 1 Punkt,
Hypothese ({\bf H}) 1 Punkt,
Wahl eines zweckmässigen $\alpha$ ({\bf A}) 1 Punkt,
Freiheitsgrade und korrekte kritische Diskrepanz ($\text{\bf D}_{\text{krit}}$)
1 Punkt,
Berechnung der Diskrepanz ({\bf D}) 2 Punkte,
Schlussfolgerung ({\bf S}) 1 Punkt.
\end{bewertung}

