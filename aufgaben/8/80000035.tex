Eine Zeitung hat im Dezember 2018 eine Leserumfrage darüber durchgeführt,
welche Politiker bei den Lesern am beliebtesten sind.
Dabei hat sich zum Erstaunen der Redaktion herausgestellt, dass Kim Yong Un
beliebter ist als Donald Trump.
Leider wurden die Rohdaten nicht veröffentlicht, so dass man die
Schlussfolgerung der Redaktion, obwohl naheliegend und verständlich,
nicht wirklich nachvollziehen kann.
Nehmen wir daher an, dass Leute befragt wurden, verschiedenen Politikern 
Noten auf einer Skala von 1--10 zu geben. 
Die Redaktion hat dann Mittelwerte der Punktzahlen verglichen und eine
Rangliste der Politiker erstellt.
Die (angenommenen) Resultate sind in der folgenden Tabelle zusammengefasst.
$n$ ist die Anzahl der Antworten, $\bar X$ die gemittelte Note und 
$S_X^2$ die empirische Varianz.
\begin{center}
\begin{tabular}{l|>{$}r<{$}>{$}r<{$}>{$}r<{$}}
Kandidat    & n&\bar X& S_X \\
\hline
Donald Trump&11&   1.6& 0.23\\
Kim Yong Un &13&   1.8& 0.24\\
\hline
\end{tabular}
\end{center}
Kann man aus den (angenommenen) Noten
schliessen, dass Donald Trump tatsächlich unbeliebter ist?

\thema{Hypothesentest}
\thema{$t$-Test}

\begin{loesung}
Wir führen eine $t$-Test durch mit $\alpha=0.95$ durch.
Die Nullhypothese ist, dass die beiden Mittelwerte zur gleichen Verteilung
gehören.
Die Anzahl der Freiheitsgrade ist $k=11+13-2=22$, dazu gehört der kritische
Wert $t_{\text{krit}}=1.7171$ (oder $t_{\text{krit}}=2.0739$
für einen beidseitigen Test).
Die Testgrösse ist
\begin{align*}
T
&=
\frac{\bar X -\bar Y}{\sqrt{(n-1)S_X^2 + (m-1)S_Y^2}}
\sqrt{\frac{nm(n+m-2)}{n+m}}
\\
&=
\frac{1.6-1.8}{10\cdot0.23^2 + 12\cdot 0.24^2}\sqrt{\frac{11\cdot13\cdot22}{24}}
=
-2.072947
\end{align*}
Daraus liest man ab, dass die Nullhypothese verworfen werden muss, dass
also ein signifikanter Unterschied zwischen den beiden Bewertungen besteht,
dass also Donald Trump tatsächlich unbeliebter ist.

Bei einem zweiseitigen Test sieht man dagegen, dass der Unterschied zu
gering ist, man also nicht schliessen kann, dass Trump weniger bliebt ist.
\end{loesung}

\begin{bewertung}
Verwendung des studentschen $t$-Tests ({\bf S}) 1 Punkt,
Wahl eines geeigneten $\alpha$ ({\bf A}) 1 Punkt,
Berechnung der Testgrösse $T$ ({\bf T}) 1 Punkt,
Berechnung der Freiheitsgrade für die $t$-Verteilung und kritischer Wert
({\bf K}) 1 Punkt,
Entscheidung, Nullhypothese wird nicht verworfen ({\bf E}) 1 Punkt.
\end{bewertung}

