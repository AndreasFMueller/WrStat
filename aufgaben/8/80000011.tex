F"ur eine Website wurde die Besucherstatistik
in Tabelle~\ref{80000011:besucher}
erhoben. Am 24.~Oktober 2006 war die Website in der Zeitung erw"ahnt worden,
am 25. wurden einige Dinge auf der Website umgestellt, und auf zwei externen
Websites wurden Links hierhin hinzugef"ugt.
Nun ist unter den Marketingverantwortlichen der Streit entbrannt,
ob der pl"otzliche Anstieg der Besucherzahlen am 24.~Oktober 2006
auf den Zeitungsartikel zur"uckzuf"uhren war, oder eher eine zuf"allige
Schwankung war.
Das Hauptargument f"ur das Lager derjeningen, die an eine zuf"allige Schwankung
glauben ist der Einbruch der Besucherzahlen am 23.~Oktober 2006.
K"onnen Sie dieses Argument widerlegen, wobei Sie sich
h"ochstens mit einer Wahrscheinlichkeit von 10\% einen Fehler gestatten?

\begin{table}
\begin{center}
\begin{tabular}{lcl}
Datum&Besucher&\\
16. 10.&41&\rule{41mm}{5pt}\\
17. 10.&40&\rule{40mm}{5pt}\\
18. 10.&39&\rule{39mm}{5pt}\\
19. 10.&44&\rule{44mm}{5pt}\\
20. 10.&36&\rule{36mm}{5pt}\\
21. 10.&42&\rule{42mm}{5pt}\\
22. 10.&45&\rule{45mm}{5pt}\\
23. 10.&33&\rule{33mm}{5pt}\\
24. 10.&50&\rule{50mm}{5pt}\\
25. 10.&48&\rule{48mm}{5pt}\\
26. 10.&39&\rule{39mm}{5pt}\\
27. 10.&42&\rule{42mm}{5pt}\\
28. 10.&53&\rule{53mm}{5pt}\\
29. 10.&45&\rule{45mm}{5pt}\\
30. 10.&39&\rule{39mm}{5pt}\\
31. 10.&37&\rule{37mm}{5pt}\\
\end{tabular}
\end{center}
\caption{Besucherzahlen auf einer Website\label{80000011:besucher}}
\end{table}

\begin{loesung}
Wir testen die Hypothese, dass der Wert vom 24. Oktober der gleichen
Normalverteilung folgt wie die acht Werte davor. Die Werte nach
dem 24.~Oktober d"urfen nicht einbezogen werden, denn durch die
Website-Ver"anderungen und die externen Links werden sich die
Besucherzahlen ganz bestimmt ver"andert haben.

Sei $b_i$ die Besucherzahl am $i$-ten Oktober,
Dann haben die ersten acht Besucherzahlen als Mittelwert
$$\mu=\frac18\sum_{i=16}^{23}b_i=\frac18(41+40+39+44+36+42+45+33)=40$$
und als empirische Varianz
\begin{align*}
\sigma^2&=\frac17\sum_{i=16}^{23}(b_i-\mu)^2\\
&=\frac17(1^2+0^2+1^2+4^2+4^2+2^2+5^2+7^2)\\
&=\frac17(1 + 0 + 1 + 16 + 16 + 4 + 25 + 49)\\
&=\frac{112}{7}=16.\\
\sigma&=\sqrt{16}=4
\end{align*}
Auf dem Niveau $\alpha=0.1$ verwerfen wir die Hypothese, wenn der gemessene
neue Wert gr\"osser ist als $\mu + 1.6449\sigma$
(siehe Normalverteilungstabelle). F"ur die Beobachtung
vom 24. Oktober gilt $b_{24}-\mu=10 > 1.6449\cdot 4$, die Hypothese
muss also verworfen werden, mit einer Fehlerwahrscheinlichkeit von 10\%
ist der Anstieg der Besucherzahlen am 24.~Oktober also nicht zuf"allig,
und damit liegt ein urs"achlicher Zusammenhang mit dem Zeitungsartikel nahe.

Betrachtet man nur Abweichungen auf eine Seite, w"are ein einseitiger Test
ebenso zweckm"assig. Dazu ist eine Abweichung von $1.2816\cdot 4=5.1264$,
was nat"urlich zur gleichen Schlussfolgerung f"uhrt.
\end{loesung}

