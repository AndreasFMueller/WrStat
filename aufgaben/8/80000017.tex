Bei der Produktion von Wein ist der Zuckergehalt des unvergorenen
Traubensaftes ein entscheidender Indikator f"ur die Qualit"at des
hergestellten Weines. Der Zuckergehalt h"angt von der Traubensorte,
der Bodenqualit"at, der Besonnung und nat"urlich vom Wetter ab.
Um die Qualit"at zweier Weinberge zu vergleichen, werden jeweils
f"unf Proben
des frisch gepressten Traubensaftes auf ihren Zuckergehalt hin
untersucht. Dabei ergaben sich folgende Messwerte

\begin{center}
\begin{tabular}{|c|cc|}
\hline
Messung&Weinberg A&Weinberg B\\
\hline
1&134.0&195.0\\
2&181.0&168.0\\
3&167.0&206.0\\
4&192.0&184.0\\
5&169.0&178.0\\
\hline
$\mu$ &168.6&186.2\\
\hline
\end{tabular}
\end{center}
Besteht ein signifkanter Unterschied zwischen den zwei Weinbergen?

\begin{loesung}
Die Frage kann mit einem $t$-Test beantwortet werden.
Die Nullhypothese ist, dass sich die beiden Erwartungswerte
nicht unterscheiden. F"ur den $t$-Test muss die Statistik $T$
berechnet werden, die aus den Varianzen der Messwerte 
bestimmt werden k"onnen. Mit dem Sch"atzer f"ur die Varianz
bekommt man
\begin{align*}
\sigma_A^2&=475.3&
\sigma_B^2&=218.2.
\end{align*}
Die Statistik ist
\begin{align*}
T&=\frac{\mu_A-\mu_B}{\sqrt{4\sigma_A^2+4\sigma_B^2}}\sqrt{\frac{5\cdot5\cdot 8}{10}}
=\frac{168.6-186.2}{2\sqrt{475.3+218.2}}2\sqrt{5}
=\frac{-17.6\sqrt{5}}{\sqrt{693.5}}
\\
&=-1.4944
\end{align*}
Die Testgr"osse $T$ kann sowohl positiv oder negativ einen kritischen
Betrag "uberschreiten.
F"ur $\alpha=5\%$ sind die kritischen Schranken bei $p=0.025$ und $p=0.975$
und bei $8$ Freiheitsgraden zu suchen, der $t$-Tabelle entnimmt man
bei $t_{\text{krit}}=2.3060$. Da der beobachtete $t$-Wert den kritischen
nicht "uberschreitet, gibt es keinen Grund, an der Nullhypothese
zu zweifeln.

Obwohl der Unterschied zwischen den beiden Mittelwerten mehr als 10\%
des kleineren Mittelwertes ausmacht, reicht dies nicht. Die Streuung
der Werte ist zu gross.
\end{loesung}
