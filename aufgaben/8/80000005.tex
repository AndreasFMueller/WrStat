Unzählige ``Experten'' versuchen immer wieder, uns zu erklären,
wie sich die diversen Krisen unserer Zeit entwickeln werden. Nur
selten wird dabei zurückgeschaut, und nachgesehen, wie die Auguren
sich in der Vergangenheit vertippt haben. Nehmen wir an, einer
dieser Propheten behaupte von sich, am häufigsten eine am besten
zutreffende
Prognose gemacht zu haben, basierend auf folgenden Zahlen:
\begin{center}
\begin{tabular}{cc}
\hline
Experte&Anzahl am besten zutreffende Prognosen\\
\hline
A&17\\
B&10\\
C&8\\
D&6\\
\hline
\end{tabular}
\end{center}
Tatsächlich sieht seine Bilanz beeindruckend aus. Aber
kann man daraus wirklich schliessen, dass er der Beste ist?

\thema{Hypothesentest}
%\thema{$\chi^2$-Test}


\begin{loesung}
Man kann die Hypothese, alle Experten hätten die gleiche Wahrscheinlichkeit,
die am besten zutreffende Prognose zu veröffentlichen, mit einem $\chi^2$-Test
testen. Dazu berechnet man die Diskrepanz mit Hilfe folgender Tabelle
\begin{center}
\begin{tabular}{|c|crrrr|}
\hline
Experte&
$i$& $n_i$& $p_i$&$n_i-np_i$&$(n_i-np_i)^2/np_i$\\
\hline
A&  1&    17&$0.25$&$6.75$ &$4.4451$\\
B&  2&    10&$0.25$&$-0.25$&$0.0061$\\
C&  3&     8&$0.25$&$-2.25$&$0.4939$\\
D&  4&     6&$0.25$&$-4.25$&$1.7622$\\
\hline
 &   &$n=41$&      &      &$D=6.7073$\\
\hline
\end{tabular}
\end{center}
Die kritische Diskrepanz für $\alpha=0.05$ und drei Freiheitsgrade ist
$D_{\text{krit}}=7.815$, die beobachtete Diskrepanz ist also nicht
gross genug, um die Nullhypothese zu verwerfen. Damit gibt es keinen
Beweis, dass Experte $A$ besser ist also die anderen drei.
\end{loesung}

