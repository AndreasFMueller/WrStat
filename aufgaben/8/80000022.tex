In der Vorlesung wurden Mandarinen ausgemessen, die Resultate dieser
Messkampagne sind im File \texttt{mandarinen.csv} zusammengestellt.
Die Theorie lässt vermuten, dass die Durchmesser normalverteilt sein
sollten. Allerdings werden Landwirtschaftsprodukt häufig selektiert,
was deren Grössenverteilung verzerrt, extrem grosse und kleine
Exemplare werden aussortiert. Sind die Mandarinen durchmesser normalverteilt?

\begin{loesung}
Die Nullhypothese
\begin{quote}
Die Mandarinendurchmesser sind normalverteilt.
\end{quote}
lässt sich mit einem Kolmogorov-Smirnov-Test überprüfen. Diesen führt
man am bequemsten mit Hilfe von $R$.

Es stellt sich jedoch das Problem,
dass der Kolmogorov-Smirnov-Test von $R$ nicht zulässt, dass Werte
mehrfach vorkommen. Man muss also die Daten leicht modifizieren, so
dass gleiche Werte unwahrscheinlich werden. Eine einfache Methode ist,
zu den Daten einen kleinen Wert hinzuzuaddieren, so dass sie garantiert
verschieden werden, aber so klein, dass es die zu untersuchenden Aussagen
nicht wesentlich verändert. Diese Werte können zufällig gewählt werden,
dann empfiehlt es sich, sie normalverteilt zu wählen, da damit der Test
nicht beeinflusst wird. Man kann sie auch einer arithmetischen Folge
entnehmen, was den Test ebenfalls nicht stört, wenn die Rohdaten in
zufälliger Reihenfolge vorliegen.

\begin{figure}
\begin{center}
\includeagraphics[width=\hsize]{mandarinen.pdf}
\end{center}
\caption{Empirische Verteilungsfunktion (blau) der Mandarinendurchmesser
sowie Normalverteilung (rot) mit gleichem Erwartungswert und Varianz.
\label{80000022:ecdf}}
\end{figure}
\begin{figure}
\begin{center}
\includeagraphics[width=\hsize]{mandarinenqq.pdf}
\end{center}
\caption{Q-Q-Plot für die Verteilung der Mandarinendurchmesser
\label{80000022:qq}}
\end{figure}

Das folgende Skript löst das Problem. Es erzeugt auch PDF Files mit
graphischen Darstellungen der empirischen Verteilungsfunktion
(Abbildung~\ref{80000022:ecdf}) sowie mit einem Q-Q-Plot
(Abbildung~\ref{80000022:qq}).
{\small
\verbatimainput{mandarinen.R}
}
Der Output des Skripts liefert die Resultate des Tests:
\verbatimainput{mandarinen.ks}
Der $p$-Wert ist $0.2822$, also immer noch grösser als der Wert
$\alpha = 0.05$, den wir bräuchten, um die Hypothese verwerfen zu können.

Es gibt einen weiteren, nicht in der Vorlesung behandelten Test, mit dem
man Datensätze darauf testen kann, ob sie normalverteilt sind. Dieser
sogenannte Shapiro-Wilk-Test liefert 
\verbatimainput{mandarinen.shapiro}
Hier ist der $p$-Wert nur $2.845\cdot 10^{-5}$, die Hypothese muss
als für ziemlich jede Wahl von $\alpha$ verworfen werden.
\end{loesung}

