Das Organisationskomite eines jährlich wiederholten Anlasses muss
eines von vier Wochenenden dafür auswählen.
Das Wetter ist ein wichtiger, kaum vorhersagbarer Faktor.
Darüber, welches der vier zur Auswahl stehenden Wochenenden
in der Vergangenheit jeweils das beste war, wurde eine Statistik
erstellt:
\begin{center}
\begin{tabular}{ccc}
\hline
Wochenende&Anzahl ``bestes''&Anzahl ``schlechtestes''\\
\hline
1&5&11\\
2&10&8\\
3&11&9\\
4&10&8\\
\hline
\end{tabular}
\end{center}
Im OK entbrennt jetzt ein Streit. Die einen sagen, das erste Wochenende
musse man gar nicht mehr in Betracht ziehen, es sei kaum brauchbar.
Die anderen sagen, der Unterschied sei zu klein, und führen ins Feld,
dass der etwas frühere Termin für die Besucherzahlen besser ist.
Wer hat recht?

\begin{loesung}
Die Datenmenge ist knapp genügend, um die Hypothese
dass die Wahrscheinlichkeit, das beste oder schlechteste Wochenende zu sein,
für alle vier Wochenenden gleich gross ist.
Dazu muss die Diskrepanz berechnet werden.
\begin{center}
\begin{tabular}{|crrrr|}
\hline
$i$& $n_i$& $p_i$&$n_i-np_i$&$(n_i-np_i)^2/np_i$\\
\hline
  1&     5&$0.25$&$ 4$&$1.7777$\\
  2&    10&$0.25$&$-1$&$0.1111$\\
  3&    11&$0.25$&$-2$&$0.4444$\\
  4&    10&$0.25$&$-1$&$0.1111$\\
\hline
   &$n=36$&      &      &$D=2.4444$\\
\hline
\end{tabular}
\end{center}
Für drei Freiheitsgrade und $\alpha=0.05$ müsste die Diskrepanz
aber den kritischen Wert $D_{\text{krit}}=7.815$ übersteigen, damit
man die Nullhypothese verwerfen kann.

Verwendet man statt der Statistik für das ``beste'' Wochenende
die Statistik für das ``schlechteste'' Wochenende, findet man
\begin{center}
\begin{tabular}{|crrrr|}
\hline
$i$& $n_i$& $p_i$&$n_i-np_i$&$(n_i-np_i)^2/np_i$\\
\hline
  1&    11&$0.25$&$-2$&$0.4444$\\
  2&     8&$0.25$&$1$&$0.1111$\\
  3&     9&$0.25$&$0$&$0.0000$\\
  4&     8&$0.25$&$1$&$0.1111$\\
\hline
   &$n=36$&      &      &$D=0.6666$\\
\hline
\end{tabular}
\end{center}
was noch weiter davon entfernt ist, signifikant zu sein.
\end{loesung}

