Eine wichtige Grösse zur Beurteilung eines Winters ist die Schneemenge.
Natürlich schwankt sie, sie ist eine Zufallsvariable mit
Erwartungswert $\mu$ und Varianz $\sigma^2$.
Ein Jahrhundert-Winter ist ein Winter mit einer Schneemenge, die nur
mit Wahrscheinlichkeit 1\% fällt. Wieviel mehr Schnee fällt in einem
Jahrhundertwinter als in einem Durchschnittswinter? Wieviel mehr
Schnee fällt in einem Jahrtausendwinter als einem Jahrhundertwinter?

\thema{Hypothesentest}
\thema{Normalverteilung}
\thema{Standardisierung}

\begin{loesung}
Die Schneemenge $X$ ist eine normalverteilte Zufallsvariable mit
Erwartungswert $\mu$ und Varianz $\sigma^2$. Gesucht ist $x$ so,
dass
\[ 
P(X > x+\mu) = 0.01\quad\Rightarrow\quad
P(X\le x+\mu) =0.99.
\]
Standardisierung liefert
\[
P\biggl(\frac{X-\mu}{\sigma}\le \frac{x}{\sigma}\biggr)=F\biggl(\frac{x}{\sigma}\biggr)=0.99
\quad\Rightarrow\quad \frac{x}{\sigma}=2.3264.
\]
In einem Jahrhundertwinter fällt also $2.3264\sigma$ mehr Schnee als
in einem Durchschnittswinter.

In einem Jahrtausendwinter muss statt $p=0.99$ der Wert $p=0.999$
verwendet werden. Anstelle von $\frac{x}{\sigma}=2.3264$ muss entsprechend
$\frac{x}{\sigma}=3.0902$ verwendet
werden. Es fällt also $(3.0902-2.3264)\sigma = 0.7638\sigma$
mehr Schnee in einem Jahrtausendwinter.
\end{loesung}

