Der Ertrag von zwei Arten von Kautschukbäumen wird untersucht.
%\ainput{t.tex} %
Folgende Erträge wurden von 16 bzw.~12 Bäumen der zwei Arten gemessen:
\begin{center}
\begin{tabular}{|l|>{$}r<{$}>{$}r<{$}>{$}r<{$}|}
\hline
&\text{Stichprobengrösse}&\text{Mittelwert}&\text{Standardabweichung}\\
\hline
Art I  &    16 &    6.25 &     0.885\\
Art II &    12 &    6.96 &     0.994\\
\hline
\end{tabular}
\end{center}
Kann man daraus ableiten, dass eine der Arten ertragreicher ist?

\begin{loesung}
Die beiden Mittelwerte können mit einem $t$-Test verglichen werden.
Die Null-Hypothese dafür lautet
\begin{quote}
Der Ertrag der beiden Arten von Kautschukbäumen ist gleich.
\end{quote}
Angesichts der kleinen Datenmenge ist $\alpha = 0.05$ angemessen.
Die Zahl der Freiheitsgrade ist $16+12-2=26$, dies führt auf einen
kritischen Wert $t_{\text{krit}}=2.0555$.
Die Berechnung der Grösse $T$ ergibt
\begin{align*}
T
&=
\frac{\overline{X}-\overline{Y}}{\sqrt{(n-1)S_X^2+(m-1)S_Y^2}}
\sqrt{\frac{nm(n+m-2)}{n+m}}
\\
&=
\frac{6.25-6.96}{\sqrt{15\cdot 0.885^2 + 11\cdot 0.994^2}}
\sqrt{\frac{16\cdot 12\cdot 26}{28}}
=
-1.993
\end{align*}
Da $|T|<t_{\text{krit}}$ ist, kann man die Nullhypothese nicht
verwerfen und auch nicht folgern, dass keine der Arten einen
signifikant höheren Ertrag hat.
\end{loesung}

\begin{diskussion}
Das negative Vorzeichen von $T$ kann als Anzeichen dafür gesehen
werden, dass zweiseitig getestet werden muss, dass also der kritische
Wert in der Spalte $p=0.975$ der $t$-Tabelle abgelesen werden muss.

Der Wert $\alpha=0.01$ ist für die gegebene Datenmenge zu gering.
\end{diskussion}

\begin{bewertung}
$t$-Test ({\bf T}) 1 Punkt,
Nullhypothese ({\bf N}) 1 Punkt,
Wahl von $\alpha$ ({\bf A}) 1 Punkt,
Freiheitsgrade und kritischer Wert ({\bf K}) 1 Punkt,
Berechnung von $T$ ({\bf W}) 1 Punkt,
Schlussfolgerung von $S$ ({\bf S}) 1 Punkt.
\end{bewertung}



