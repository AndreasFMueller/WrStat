Eine Untersuchung möchte Wissen, ob das Sternzeichen einer
Personen einen Einfluss auf die Persönlichkeit hat.
Dazu wurden 256 Künstler nach ihrem Sternzeichen befragt,
es ergaben sich die Resultate der folgenden Tabelle:
\begin{center}
\begin{tabular}{|l|r|}
\hline
Sternzeichen & Anzahl \\
\hline
Widder     & 29 \\
Stier      & 24 \\
Zwillinge  & 22 \\
Krebs      & 19 \\
Löwe       & 21 \\
Jungfrau   & 18 \\
Waage      & 19 \\
Skorpion   & 20 \\
Schütze    & 23 \\
Steinbock  & 18 \\
Wassermann & 20 \\
Fische     & 23 \\
\hline
\end{tabular}
\end{center}
Es scheint, dass Künstler häufiger das Sternzeichen Widder
haben als andere, doch reicht das?
Kann damit eine Aussage darüber machen, ob das Sternzeichen einen
Einfluss darauf hat, ob eine Person ein Künstler ist?

\begin{hinweis}
Um den Rechenaufwand für die Handrechnung zu reduzieren, dürfen Sie
auf Wunsch die rechnerische Auswertung auch nur mit der ersten Hälfte
der Tabelle durchführen.
\end{hinweis}

\begin{loesung}
Die Nullhypothese, die wir testen müssen ist
\begin{quote}
Die Eigenschaft, ein Künstler zu sein, ist unabhängig vom 
Sternzeichen.
\end{quote}
Wir verwenden einen $\chi^2$-Test mit 11 Freiheitsgraden.
Ein sinnvoller Wert $\alpha=5\%$ scheint angemessen.
Der kritische Wert der Diskrepanz für $p=0.95$ und $k=11$ ist
$D_{\text{krit}}=19.675$. 
Die Nullhypothese wird verworfen, wenn die aus obigen Daten
berechnete Diskrepanz grösser ist als $D_{\text{krit}}$.
\begin{center}
\begin{tabular}{|r|l|>{$}r<{$}>{$}r<{$}>{$}r<{$}>{$}r<{$}>{$}r<{$}|}
\hline
   & Sternzeichen    & n_i&        p_i &     np_i &  n_i-np_i  &(n_i-np_i)^2/np_i \\
\hline
1  & Aries           & 29 & 0.08333333 & 21.33333 &  7.66666 & 2.75520 \\
2  & Taurus          & 24 & 0.08333333 & 21.33333 &  2.66666 & 0.33333 \\
3  & Gemini          & 22 & 0.08333333 & 21.33333 &  0.66666 & 0.02083 \\
4  & Cancer          & 19 & 0.08333333 & 21.33333 & -2.33333 & 0.25520 \\
5  & Leo             & 21 & 0.08333333 & 21.33333 & -0.33333 & 0.00520 \\
6  & Virgo           & 18 & 0.08333333 & 21.33333 & -3.33333 & 0.52083 \\
7  & Libra           & 19 & 0.08333333 & 21.33333 & -2.33333 & 0.25520 \\
8  & Scorpio         & 20 & 0.08333333 & 21.33333 & -1.33333 & 0.08333 \\
9  & Sagittarius     & 23 & 0.08333333 & 21.33333 &  1.66666 & 0.13020 \\
10 & Capricorn       & 18 & 0.08333333 & 21.33333 & -3.33333 & 0.52083 \\
11 & Aquarius        & 20 & 0.08333333 & 21.33333 & -1.33333 & 0.08333 \\
12 & Pisces          & 23 & 0.08333333 & 21.33333 &  1.66666 & 0.13020 \\
\hline
   &            & n = 256 &            &          &            & 5.09375 \\
\hline
\end{tabular}
\end{center}
Da die Diskrepanz $D<D_{\text{krit}}$ klar kleiner ist als die kritische
Diskrepanz $D_{\text{krit}}$, gibt es keinen Grund, an der Nullhypothese
zu zweifeln.
Das Sternzeichen hat keinen Einfluss darauf, ob jemand ein Künstler ist.

Führt man die Rechnung nur mit der ersten Hälfte der Tabelle durch,
erhält man:
\begin{center}
\begin{tabular}{|r|l|>{$}r<{$}>{$}r<{$}>{$}r<{$}>{$}r<{$}>{$}r<{$}|}
\hline
  & Sternzeichen & n_i &        p_i &     np_i & n_i-np_i &(n_i-np_i)^2/np_i \\
\hline
1 & Aries        &  29 & 0.1666667 & 22.16667 &  6.833333 & 2.106511 \\
2 & Taurus       &  24 & 0.1666667 & 22.16667 &  1.833333 & 0.151623 \\
3 & Gemini       &  22 & 0.1666667 & 22.16667 & -0.166667 & 0.001253 \\
4 & Cancer       &  19 & 0.1666667 & 22.16667 & -3.166667 & 0.452382 \\
5 & Leo          &  21 & 0.1666667 & 22.16667 & -1.166667 & 0.061409 \\
6 & Virgo        &  18 & 0.1666667 & 22.16667 & -4.166667 & 0.783208 \\
\hline
   &         & n = 133 &           &          &         & D=3.556391 \\
\hline
\end{tabular}
\end{center}
Die Anzahl der Freiheitsgrade reduziert sich auf $5$ und die kritische
Diskrepanz auf $D_{\text{krit}}=11.070$.
Da die Diskrepanz $D<D_{\text{krit}}$ auch in diesem Fall klar kleiner
ist als die kritische Diskrepanz $D_{\text{krit}}$, gibt es auch mit
der kleineren Datenmenge keinen Grund, an der Nullhypothese zu zweifeln.
\end{loesung}

\begin{bewertung}
Nullhypothese ({\bf Z}) 1 Punkt,
Wahl von $\alpha$ ({\bf A}) 1 Punkt,
Freiheitsgrade und kritische Diskrepanz ({\bf F}) 1 Punkt,
Berechnung der Diskrepanz ({\bf D}) 2 Punkt,
Entscheid und Schlussfolgerung ({\bf S}) 1 Punkt.
\end{bewertung}

