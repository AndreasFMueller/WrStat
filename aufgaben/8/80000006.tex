Zur Vers"ussung der Pr"ufungsvorbereitung habe ich eine Rolle Smarties
Schokobonbons genascht. Dabei kam die Farbe rot deutlich am h"aufigsten vor.
Genauer: folgende Farbverteilung wurde gez"ahlt. 
\begin{center}
\begin{tabular}{|l|cccccccc|}
\hline
Farbe&rot&oragen&gelb&gr"un&blau&violett&pink&braun\\
\hline
Anzahl&22&    15&  15&   13&  12&     13&  13&   14\\
\hline
\end{tabular}
\end{center}
Kann man daraus schliessen, dass die roten Smarties generell am h"aufigsten
sind?

\begin{loesung}
Wir f"uhren einen $\chi^2$-Test auf dem Niveau $\alpha=0.05$
f"ur die Nullhypothese 
``{\it alle Farben sind gleich wahrscheinlich}'' durch.
Die Nullhypothese bedeutet $p_i=\frac18$ f"ur alle $i$.
Dazu ist die Diskrepanz zu berechnen:
\begin{center}
\begin{tabular}{|l|r|r|r|r|}
\hline
Farbe  &$n_i$   &$np_i$&$np_i-n_i$&$(np_i-n_i)^2/(np_i)$\\
\hline
rot    &   22   &14.625&     7.375&              3.71902\\
oragen &   15   &14.625&     0.375&              0.00962\\
gelb   &   15   &14.625&     0.375&              0.00962\\
gr"un  &   13   &14.625&    -1.625&              0.18056\\
blau   &   12   &14.625&    -2.625&              0.47115\\
violett&   13   &14.625&    -1.625&              0.18056\\
pink   &   13   &14.625&    -1.625&              0.18056\\
braun  &   14   &14.625&    -0.625&              0.02671\\
\hline
       &$n=117$ &      &          &$D= 4.77778         $\\
\hline
\end{tabular}
\end{center}
Die Zahl der Freiheitsgrade ist $8-1=7$, die kritische Diskrepanz
$D_{\text{krit}}$ ist die Quantile der $\chi^2$-Verteilung 
f"ur $p=1-\alpha=0.95$, aus der $\chi^2$-Tabelle finden wir den
Wert $D_{\text{krit}}=14.067$. Da das berechnete $D$ nicht gr"osser
ist als $D_{\text{krit}}$ gibt es keinen Hinweis darauf, dass
tats"achlich eine Farbe h"aufiger vorkommt.
\end{loesung}
