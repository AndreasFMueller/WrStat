Der Eötvös-Effekt ist die vertikale Ablenkung eines Körpers, der sich
auf der rotierenden Erde parallel zu einem Breitenkreis bewegt.
Bewegt er sich in Richtung der Erddrehung (von Westen nach Osten),
führt dies zu einer grösseren Winkelgeschwindigkeit um die Erdachse
und damit zu höherer Zentrifugalkraft, der Körper scheint leichter
zu werden.
Bewegt er sich gegen die Erddrehung (von Osten nach Westen), führt
dies zu einer geringeren Winkelgeschwindigkeit um die Erdachse und
damit zu geringerer Zentrifugalkraft, der Körper scheint leichter
zu werden.
Youtuber Wolfie6020 hat eine Präzisionswaage in ein Flugzeug
mitgenommen und bei breitenkreisparallelen Flügen über Australien
das Gewicht einer Testmasse gemessen:
\begin{center}
\includeagraphics[width=10cm]{eotvos.jpg}
\end{center}
Dabei stellte er einen kleinen Unterschied zwischen dem Flug in
Ost-West- und West-Ost-Richtung fest:
\begin{center}
\begin{tabular}{|l|rrr|}
\hline
        &Mittelwert [g]&Standardabweichung [g]&Messungen\\
\hline
Ost-West&        499.62&                  1.38&       83\\
West-Ost&        495.23&                  1.06&       82\\
\hline
\end{tabular}
\end{center}
Allerdings sind die Messfehler relativ gross, zeitweise kreuzen sich
die Kurven der Messdaten, wie man in der Abbildung sehen kann.
Kann man aus den Messdaten mit hoher Sicherheit den Schluss ziehen,
dass der Eötvös-Effekt tatsächlich vorhanden ist?

%
%Eotvos-Effekt: 07:09, zwei Datensätz mit Mittelwerten und
%Varianzen
%https://www.youtube.com/watch?v=ELPRxCtuhGs
%

\begin{loesung}
Die Nullhypothese
\begin{quote}
Die beiden Werte sind Mittelwerte von Stichproben von normalverteilten
Zufallsvariablen mit dem gleichen Erwartungswert
\end{quote}
lässt sich mit dem $t$-Test testen.
Um eine hohe Sicherheit zu erreichen, verwenden wir $\alpha=1\%$.
Die Anzahl der Freiheitsgrade ist $f=n+m-2= 163$.
Der kritische Wert für $p=1-\frac{\alpha}2=0.995$ und die genannte
Zahl der Freiheitsgrade ist nicht in der $t$-Tabelle im Skript,
aber er muss zwischen $t_{\text{krit}}=2.6259$ und $2.5857$ sein.
Um auf der sicheren Seite zu liegen, verwenden wir den grösseren Wert.
Für den Test wird die Grösse
\[
T = \frac{\bar{X}-\bar{Y}}{\sqrt{(n-1)S_X^2 + (m-1)S_Y^2}}
\sqrt{\frac{nm(n+m-2)}{n+m}}
\]
für 
\begin{equation*}
\begin{aligned}
\bar{X}&=499.62,&&& S_X&=1.38,&&&  n&=83\\
\bar{Y}&=495.23,&&& S_Y&=1.06,&&&  m&=82
\end{aligned}
\end{equation*}
berechnet, es ergibt sich $T=22.896$.
Da $T\gg t_{\text{krit}}$, muss die Nullhypothese verworfen werden.
Damit ist der Eötvös-Effekt nachgewiesen.
\end{loesung}

\begin{bewertung}
$t$-Test ({\bf T}) 1 Punkt,
Wahl von $\alpha\le 1\%$ ({\bf A}) 1 Punkt,
Freiheitsgrade ({\bf F}) 1 Punkt,
Kritischer Wert für $t$ ({\bf K}) 1 Punkt,
Berechnung des Wertes für $T$ ({\bf B}) 1 Punkt,
Schlussfolgerung ({\bf S}) 1 Punkt.
\end{bewertung}
