David Jones hat in seinem EEVblog auf Youtube 400 1k$\Omega$-Widerstände
nachgemessen um zu sehen, ob die Widerstandswerte tatsächlich normalverteilt
sind. Seine Rohdaten hat er für diese "Ubungsaufgabe im File
\verb+resistors.csv+ zur Verfügung gestellt.
\begin{teilaufgaben}
\item Welchen mittleren Widerstandswert und welche Streuung ist auf Grund
der Messungen zu erwarten?
\item Wie gross ist die Wahrscheinlichkeit, dass der Wert eines Widerstandes
aus dieser Charge mehr als 0.5\% vom Nennwert abweicht?
\item Wieviele von 1000000 Widerständen weichen um mehr als 1\% vom
Nennwert ab?
\item Kann man davon ausgehen, dass die Widerstandswerte normalverteilt sind?
\end{teilaufgaben}

\thema{Hypothesentest}
\thema{Kolmogorov-Smirnov-Test}

\begin{loesung}
Die Lösung dieser Aufgabe mit {\tt R} ist im File {\tt eevblogresistors.R}
zu finden, die einzelnen Schritte werden im Folgenden diskutiert.
\begin{figure}
\begin{center}
\includeagraphics[width=\hsize]{ecdf.pdf}
\end{center}
\caption{Empirische Verteilungsfunktion der Widerstandsdaten\label{80000020-ecdf}}
\end{figure}
\begin{figure}
\begin{center}
\includeagraphics[width=\hsize]{qq.pdf}
\end{center}
\caption{Q-Q-Plot der Widerstandsdaten\label{80000020-qq}}
\end{figure}
\begin{teilaufgaben}
\item
Mit Hilfe der Funktionen {\tt mean} und {\tt var} von {\tt R} findet man
$\mu= 999.8746$ und $\sigma= 1.695783$.
\item
Unter der Annahme einer Normalverteilung mit obigem $\mu$ und $\sigma$
ist die Wahrscheinlichkeit
$1-P(995 < X \le 1005)$ gesucht.
Diese kann mit der Funktion {\tt pnorm} berechnet werden:
\verbatimainput{pnorm.txt}
Die Wahrscheinlichkeit für dieses Ereignis ist also $0.0032768$.
\item
Die Wahrscheinlichkeit $1-P(990 < X \le 1010)$ ist 
\verbatimainput{pnorm2.txt}
die erwartete Anzahl von Widerständen ist also
$1000000 \cdot 4.9\cdot 10^{-9}=0.004$, eine Abweichung von mehr als 1\%
kommt also praktisch nicht vor.
\item
Der Kolmogorov-Smirnov-Test auf den Widerstandsdaten liefert:
{\small
\verbatimainput{ks.txt}
}
Der $p$-Wert ist die Wahrscheinlichkeit, unter der Annahme der Null-Hypothese
zu einer derart grossen Abweichung zwischen empirischer Verteilungsfunktion
und Verteilungsfunktion der Normalverteilung zu kommen. Da der $p$-Wert
so klein ist, muss die Nullhypothese verworfen werden, man muss also davon
ausgehen, dass keine Normalverteilung vorliegt.

Der Shapiro-Wilk-Test, der darauf spezialisiert ist, Daten auf Normalverteilung
zu testen, bestätigt dies mit einem ebenfalls sehr kleinen $p$-Wert.
\verbatimainput{shapiro.txt}

Die Ursache für die Abweichung von der Normalverteilung ist in den Plots
\ref{80000020-ecdf} und \ref{80000020-qq} zu
sehen. In der Mitte der empirischen Verteilungsfunktion sieht man, dass
ziemlich genau eine Normalverteilung vorliegt, aber mit deutlich
geringerer Varianz.
Für diese ``schmalere'' Normalverteilung gibt es aber in den Daten
zu viele stärker abweichenden Werte.
\end{teilaufgaben}
\end{loesung}

