Es wird manchmal vermutet, dass Studierende verschiedener Studiengänge in
der WrStat-Prüfung unterschiedlich gut abschneiden.
Die relativ hohen Anmeldezahlen für die Prüfung im HS 2025 ermöglichen,
dies zu testen.
Dazu werden Durchschnittsnoten $m_i$ pro Studiengang berechnet, wobei $i$ den
Studiengang bezeichnet.
Ausserdem sind die Teilnehmerzahlen $n_i$ pro Studiengang bekannt.
\begin{teilaufgaben}
\item Wie lautet die Nullhypothese?
\item Welches $\alpha$ ist angemessen?
\item Welche Teststatistik ist zu verwenden?
\item Wie ist der kritische Wert zu bestimmen?
\end{teilaufgaben}

\begin{loesung}
\begin{teilaufgaben}
\item Nullhypothese:
\begin{quote}
Die Durschnittsnoten unterscheiden sich nicht.
\end{quote}
\item $\alpha=0.05$ ist angemessen.
\item Zum Vergleich von Mittelwerten wird der $t$-Test verwendet.
Beim Vergleich der Mittelwerte $m_i$ und $m_k$ ist daher der Wert
der $t$-Statistik zu berechnen.
\item Beim Vergleich der Mittelwerte $m_i$ und $m_k$ ist zunächst mit
der Zahl $n_i+n_k-2$ der Freiheitsgrade die Quantile 97.5\%-Quantile
der $t$-Verteilung bestimmt werden.
\qedhere
\end{teilaufgaben}
\end{loesung}
