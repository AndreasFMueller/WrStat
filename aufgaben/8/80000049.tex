Eine Untersuchung geht einem möglichen Zusammenhang zwischen 
Schulbildung und Zivilstand nach.
In einer Umfrage ergaben sich für alle Teilnehmer die folgenden Resultate
\begin{equation}
\begin{tabular}{|>{$}c<{$}|l|>{$}r<{$}|}
\hline
&Zivilstand&\text{Häufigkeit}\\
\hline
1&ledig       &  30\%\\
2&verheiratet &  50\%\\
3&geschieden  &  10\%\\
4&verwittwet  &  10\%\\
\hline
\end{tabular}
\label{80000049:tab1}
\end{equation}
Die Untersuchung vermutet, dass Teilnehmer, die nur die
obligatorischen Schulen besucht haben, geringere Chancen haben,
verheiratet zu sein.
Die Umfrage fand für diese Untergruppe die folgenden Zahlen:
\begin{equation}
\begin{tabular}{|>{$}c<{$}|l|>{$}r<{$}|}
\hline
&Zivilstand&\text{Anzahl}\\
\hline
1&ledig       &  18\\
2&verheiratet &  12\\
3&geschieden  &   6\\
4&verwittwet  &   3\\
\hline
\end{tabular}
\label{80000049:tab2}
\end{equation}
Können Sie die Vermutung bestätigen oder widerlegen?

\begin{loesung}
Wir führen einen $\chi^2$-Test für die Nullhypothese
\begin{quote}
Die beobachteten Zahlen des Zivilstands weichen nicht wesentlich von
den Häufigkeiten der Zivilstände in der gesamten Untersuchung ab.
\end{quote}
durch.
Die erwarteten Wahrscheinlichkeiten $p_i$ entnehmen wir der dritten
Spalte der Tabelle~\eqref{80000049:tab1}.
Wir berchnen die Diskrepanz zu den Zahlen $n_i$ aus
Tabelle~\eqref{80000049:tab2}.
Dazu dient die folgende Tabelle:
\begin{center}
\begin{tabular}{|>{$}c<{$}|>{$}r<{$}|>{$}r<{$}>{$}r<{$}|>{$}r<{$}>{$}r<{$}|}
\hline
i&  n_i& p_i& np_i& n_i-np_i& (n_i-np_i)^2/np_i\\
\hline
1&   18& 0.3& 11.7&      6.3&            3.3923\\
2&   12& 0.5& 19.5&     -7.5&            2.8846\\
3&    6& 0.1&  3.9&      2.1&            1.1308\\
4&    3& 0.1&  3.9&     -0.9&            0.2077\\
\hline
 & n=39&    &     &         &        D = 7.6153\\
\hline
\end{tabular}
\end{center}
Aufgrund der geringen Datenmenge ist $\alpha=5\%$ angemessen, der zugehörige
kritische Wert der Diskrepanz für 3 Freiheitsgrade ist $D_{\text{krit}}=7.815$.
Da $D < D_{\text{krit}}$ ist, erlauben die Daten nicht, die Nullhypothese
zu verwerden, man kann also nicht schliessen, dass weniger gebildete
Leute weniger oft heiraten.
\end{loesung}

\begin{bewertung}
Nullhypothese ({\bf H$\mathstrut_0$}) 1 Punkt,
$\chi^2$-Test ({\bf X}) 1 Punkt,
Berechnung der Diskrepanz ({\bf D}) 1 Punkt,
Wahl von $\alpha$ ({\bf A}) 1 Punkt,
Freiheitsgrade und kritischer Wert ({\bf K}) 1 Punkt,
Schlussfolgerung ({\bf S}) 1 Punkt.
\end{bewertung}

