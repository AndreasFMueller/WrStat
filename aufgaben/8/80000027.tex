Mit dem folgenden Experiment soll herausgefunden werden, ob vier Münzen
fair sind.
Die vier Münzen werden 500 mal geworfen, und es wird gezählt, wie
wie oft die fünf möglichen Anzahlen von Kopf auftreten.
Die Resultate sind
\begin{center}
\begin{tabular}{|c|c|}
\hline
Anzahl Kopf&Anzahl Würfe\\
\hline
0& \phantom{0}43\\
1& 127\\
2& 168\\
3& 121\\
4& \phantom{0}41\\
\hline
\end{tabular}
\end{center}
Sind die Münzen fair?

\thema{Hypothesentest}
\themaL{chi2Test}{$\chi^2$-Test}

\begin{loesung}
Es geht hier um die Nullhypothese, dass die vier Münzen fair sind,
also Kopf und Zahl mit der gleichen Wahrscheinlichkeit von 0.5 zeigen.
Wenn dem so ist, dann ist die Anzahl Kopf in einem Wurf von vier Münzen
binomailverteilt mit $p=0.5$ und $n=4$.

Wir können diese Hypothese  mit einem $\chi^2$-Test mit 4 Freiheitsgraden
testen.
Wir wählen $\alpha=0.05$, der zugehörige kritische Wert für die Diskrepanz
ist $D_{\text{krit}}=9.488$.
Wir berechnen die Diskrepanz der Daten:
\begin{center}
\begin{tabular}{|>{$}c<{$}|>{$}r<{$}|>{$}r<{$}|>{$}r<{$}|>{$}r<{$}|>{$}r<{$}|}
\hline
i&n_i&   p_i&  np_i& n_i-np_i&(n_i-np_i)^2/np_i\\
\hline
0& 43&0.0625& 31.25& 11.75&  4.418\\
1&127&0.2500&125.00&  2.00&  0.032\\
2&168&0.3750&187.50&-19.50&  2.028\\
3&121&0.2500&125.00& -4.00&  0.128\\
4& 41&0.0625& 31.25&  9.75&  3.042\\
\hline
 &500&1.0000&      &      &D=9.648\\
\hline
\end{tabular}
\end{center}
Da $D>D_{\text{krit}}$ ist, muss die Hypothese verworfen werden, die Münzen
sind also {\em nicht} fair.
\end{loesung}

\begin{bewertung}
Hypothese ({\bf H}) 1 Punkt,
Wahl eines geeigneten $\alpha$ ({\bf A}) 1 Punkt,
Anzahl Freiheitsgrade und kritischer Diskrepanzwert ({\bf K}) 1 Punkt.
Wahrscheinlichkeiten $p_i$ ({\bf P}) 1 Punkt,
Berechnung der Diskrepanz ({\bf D}) 1 Punkt,
Schlussfolgerung ({\bf S}) 1 Punkt.
\end{bewertung}

