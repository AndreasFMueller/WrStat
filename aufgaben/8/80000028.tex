Im Jahre 2002 hat das amerikanische National Center for Health Statistics
die folgenden
Informationen über die Verteilung des Körpergewichtes in der
Bevölkerung publiziert:
\begin{center}
\begin{tabular}{l|cccc}
Kategorie&untergewichtig&normal&übergewichtig&fettleibig\\
         &$\text{BMI}<18.5$&$18.5\le\text{BMI}<25$&$25\le\text{BMI}<30$&$30\le\text{BMI}$\\
\hline
Häufigkeit&0.01&0.39&0.36&0.23
\end{tabular}
\end{center}
In einer Studie über den Zusammenhang zwischen Körpergewicht und
Herzkrankheiten wurden $3326$ Teilnehmer untersucht.
Die Resultate sind natürlich nur dann für die Gesamtbevölkerung 
aussagekräftig, wenn die Stichprobe die Körpergewicht-Verteilung in
der Bevölkerung repräsentiert. 
Es wurden folgende Anzahlen gefunden
\begin{center}
\begin{tabular}{l|r}
Kategorie      &Anzahl\\
\hline
untergewichtig &    20\\
normal         &   932\\
übergewichtig &  1374\\
fettleibig     &  1000
\end{tabular}
\end{center}
Ist diese Stichprobe für die Gesamtbevölkerung repräsentativ?

\thema{Hypothesentest}
%\thema{$\chi^2$-Test}

\begin{loesung}
Man muss prüfen, ob die Zahlen zu den Wahrscheinlichkeiten aus der
ersten Tabelle passen, was man mit einem $\chi^2$-Test machen kann.
Die zu testende Nullhypothese ist also:
Die beobachteten Anzahlen sind Beobachtungen der Verteilung der
Körpergewichte gemäss National Center for Health Statistics.

Für den Test ist zunächst ein $\alpha$ zu wählen, weil es sich
um eine medizinische Anwendung handelt, müssten wir $\alpha=0.001$
wählen, doch dieser Wert kommt in unserer Tabelle nicht vor, also
begnügen wir uns mit $\alpha=0.01$.
Wir haben $4$ Klassen, also $3$ Freiheitsgrade, und finden in der
Tabelle den zugehörigen kritischen Wert der Diskrepanz als
$D_{\text{krit}}=11.345$.

Wir berechnen die Diskrepanz:
\begin{center}
\begin{tabular}{|l|>{$}r<{$}|>{$}r<{$}|>{$}r<{$}|}
\hline
Kategorie     & p_i& n_i&(n_i-np_i)^2/np_i\\
\hline
untergewichtig&0.02&  20&   32.533\\
normal        &0.39& 932&  102.786\\
übergewichtig&0.36&1374&   26.059\\
fettleibig    &0.23&1000&   72.204\\
\hline
              &    &3325&D=233.581\\
\hline
\end{tabular}
\end{center}
Da $D>D_{\text{krit}}$ müssen wir die Nullhypothese
verwerfen und können schliessen, dass diese Verteilung
nicht repräsentativ ist.
\end{loesung}

\begin{bewertung}
$\chi^2$-Test ({\bf X}) 1 Punkt,
Nullhypothese ({\bf N}) 1 Punkt,
Wahl von $\alpha$ ({\bf A}) 1 Punkt,
Anzahl Freiheitsgrade und kritische Diskrepanz ({\bf K}) 1 Punkt,
Berechnung der Diskrepanz ({\bf D}) 1 Punkt,
Schlussfolgerung ({\bf S}) 1 Punkt.
\end{bewertung}


