Im Jahre 2002 hat das amerikanische National Center for Health Statistics
die folgenden
Informationen "uber die Verteilung des K"orpergewichtes in der
Bev"olkerung publiziert:
\begin{center}
\begin{tabular}{|l|cccc|}
Kategorie&untergewichtig&normal&"ubergewichtig&fettleibig\\
         &$\text{BMI}<18.5$&$18.5\le\text{BMI}<25$&$15\le\text{BMI}<30$&$30\le\text{BMI}$\\
\hline
H"aufigkeit&0.01&0.39&0.36&0.23
\end{tabular}
\end{center}
In einer Studie "uber den Zusammenhang zwischen K"orpergewicht und
Herzkrankheiten wurden $3326$ Teilnehmer untersucht.
Die Resultate sind nat"urlich nur dann f"ur die Gesamtbev"olkerung 
aussagekr"aftig, wenn die Stichprobe die K"orpergewicht-Verteilung in
der Bev"olkerung repr"asentiert. 
Tut sie dies?

\begin{loesung}
\end{loesung}

\begin{bewertung}
\end{bewertung}


