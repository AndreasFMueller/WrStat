Im Jahre 2002 hat das amerikanische National Center for Health Statistics
die folgenden
Informationen "uber die Verteilung des K"orpergewichtes in der
Bev"olkerung publiziert:
\begin{center}
\begin{tabular}{l|cccc}
Kategorie&untergewichtig&normal&"ubergewichtig&fettleibig\\
         &$\text{BMI}<18.5$&$18.5\le\text{BMI}<25$&$15\le\text{BMI}<30$&$30\le\text{BMI}$\\
\hline
H"aufigkeit&0.01&0.39&0.36&0.23
\end{tabular}
\end{center}
In einer Studie "uber den Zusammenhang zwischen K"orpergewicht und
Herzkrankheiten wurden $3326$ Teilnehmer untersucht.
Die Resultate sind nat"urlich nur dann f"ur die Gesamtbev"olkerung 
aussagekr"aftig, wenn die Stichprobe die K"orpergewicht-Verteilung in
der Bev"olkerung repr"asentiert. 
Es wurden folgende Anzahlen gefunden
\begin{center}
\begin{tabular}{l|r}
Kategorie      &Anzahl\\
\hline
untergewichtig &    20\\
normal         &   932\\
"ubergewichtig &  1374\\
fettleibig     &  1000
\end{tabular}
\end{center}
Sind diese Zahlen f"ur die Gesamtbev"olkerung repr"asentativ?

\begin{loesung}
Man muss pr"ufen, ob die Zahlen zu den Wahrscheinlichkeiten aus der
ersten Tabelle passen, was man mit einem $\chi^2$-Test machen kann.
Dazu berechnet man die Diskrepanz:
\begin{center}
\begin{tabular}{|l|>{$}r<{$}|>{$}r<{$}|>{$}r<{$}|}
\hline
Kategorie     & p_i& n_i&(n_i-np_i)^2/np_i\\
\hline
untergewichtig&0.02&  20&   32.533\\
normal        &0.39& 932&  102.786\\
"ubergewichtig&0.36&1374&   26.059\\
fettleibig    &0.23&1000&   72.204\\
\hline
              &    &3325&D=233.581\\
\hline
\end{tabular}
\end{center}
Da es sich um eine medizinische Anwendung handelt, sollten wir eigentlich
$\alpha=0.001$ verwenden, aber unsere $\chi^2$-Tabelle enth"alt das nicht,
wir begn"ugen uns daher mit $\alpha=0.01$, und bekommen f"ur den kritischen
$D$-Wert bei drei Freiheitsgraden
$D_{\text{krit}}=11.345$.
Da $D>D_{\text{krit}}$ k"onnen wir schliessen, dass diese Verteilung
nicht repr"asentativ ist.
\end{loesung}

\begin{bewertung}
\end{bewertung}


