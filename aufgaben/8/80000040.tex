MrThriveAndSurvive, Flacherdler und Youtuber, versucht zu beweisen,
dass Mondlicht etwas grundsätzlich anderes sei als Sonnenlicht, weil
Mondlicht im Gegensatz zum Sonnenlicht nicht wärme, sondern kühle.
Er führt fünf Messungen durch, in denen er im übrigen identische Gegenstände
mit einem Infrarot-Thermometer misst.
Die einen sind direkt dem Mondlicht
ausgesetzt, die anderen werden von einem Schirm geschützt.
Natürlich ist er nicht in der Lage zu verstehen, dass der von dieser
Messanordnung möglicherweise gemessene Temperaturunterschied eher von
der sogenannten radiativen Kühlung kommt, der Tatsache, dass Gegenstände
unter freiem Himmel durch Abstrahlung von Wärme ins Weltall schneller
abkühlen als Gegenstände unter einem Schirm.
Er bekommt die folgenden, nicht sehr unterschiedlichen Temperaturen:
\begin{center}
\begin{tabular}{l >{$}c<{$} >{$}c<{$}}
\hline
Messung&\text{Temperatur in $\mathstrut^\circ$F mit Mondlicht}&\text{Temperatur in $\mathstrut^\circ$F im Schatten}\\
\hline
1&71.7&74.1\\
2&71.6&72.5\\
3&69.9&70.5\\
4&69.9&69.2\\
5&71.9&72.3\\
\hline
Mittelwert&71&71.72\\
Varianz&1.02&3.612\\
\hline
\end{tabular}
\end{center}
Kann man aus diesen Daten schliessen, dass es tatsächlich einen Unterschied
zwischen Mond und Schatten gibt?

\thema{Hypothesentest}
\themaL{chi2Test}{$\chi^2$-Test}

\begin{loesung}
Es geht darum, den Mittelwert der Beiden Datensätze zu vergleichen.
Dies ist mit einem $t$-Test möglich.
Die Nullhypothese ist
\begin{quote}
Es gibt keinen Unterschied in der Verteilung der Messwerte zwischen
Mondlicht und Schatten.
\end{quote}
Ein Test mit $\alpha=0.05$ und $f=n+m-2 = 5+5-2=8$ Freiheitsgraden 
verwendet einen kritischen Wert $t_{\text{krit}}= 2.3060$ für einen
zweiseitigen Test.
Der $t$-Wert kann mit Hilfe der Formel 
\begin{align*}
T
&=
\frac{\mu_X-\mu_Y}{\sqrt{(n-1)S_X^2+(m-1)S_Y^2}}\sqrt{\frac{nm(n+m-2)}{n+m}}
\\
&=
\frac{71-71.72}{\sqrt{4\cdot 1.02 + 4\cdot 3.612}}\sqrt{\frac{25\cdot 8}{10}}
=
\frac{-0.72}{4.3044}\sqrt{20}
=
-0.74806.
\end{align*}
Da dieser Unterschied viel kleiner ist als der kritische Wert, kann man
die Nullhypothese nicht verwerfen.
Das Experiment beweist also nichts.
\end{loesung}

\begin{diskussion}
Das Experiment wurde von Youtuber AB science in einem Video gehörig zerzaust: 
\url{https://www.youtube.com/watch?v=5J7ovRNbBcU}
\end{diskussion}

\begin{bewertung}
$t$-Test ({\bf T}) 1 Punkt,
Nullhypothese ({\bf N}) 1 Punkt,
Wahl von $\alpha$ ({\bf A}) 1 Punkt,
Freiheitsgrade und kritischer $t$-Wert ({\bf F}) 1 Punkt,
Berechnung von $t$ ({\bf T}) 1 Punkt,
Schlussfolgerung ({\bf S}) 1 Punkt.
\end{bewertung}

