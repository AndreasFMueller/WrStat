Bei einer Untersuchung der Häufigkeit von Verkehrsunfällen in
verschiedenen Altersgruppen wurden folgende Zahlen für die Anzahl der 
Unfälle verursacht durch eine standardisierte Anzahl Lenker
in jeder Kategorie gefunden:
\begin{center}
\begin{tabular}{l|cccccc}
Altersgruppe   & 17--20 & 21--30 & 31--40 & 41--50 & 51--60 & über 60 \\
\hline
Anzahl Unfälle &   45   &   39&      25   &   29   &   26   &   25    \\
\end{tabular}
\end{center}
Kann man daraus ableiten, dass Junglenker mehr Unfälle verursachen?

\begin{loesung}
Die Nullhypothese, die wir testen wollen ist
\begin{quote}
Unfälle sind in allen Alterskategorien gleich häufig.
\end{quote}
Wir testen diese Hypothese mit einem $\chi^2$-Test für $\alpha=0.05$ und
$\alpha=0.01$.
Die Zahl der Freiheitsgrade ist 5, die kritischen Werte für die Diskrepanz
sind $D_{0.05}=11.070$ und $D_{0.01}=15.086$.
Wir berechnen die Diskrepanz mit der folgenden Tabelle:
\begin{center}
\begin{tabular}{|c|>{$}r<{$}|>{$}c<{$}>{$}c<{$}|>{$}r<{$}|}
\hline
Altersgruppe&   n_i&p_i    &np_i& (n_i-np_i)^2/np_i \\
\hline
17--20      &    45&\frac16&31.5&           5.78571 \\
21--30      &    39&\frac16&31.5&           1.78571 \\
31--40      &    25&\frac16&31.5&           1.34127 \\
41--50      &    29&\frac16&31.5&           0.19841 \\
51--60      &    26&\frac16&31.5&           0.96032 \\
über 60     &    25&\frac16&31.5&           1.34127 \\
\hline
            &n=189&       &    &        D = 11.413 \\
\hline
\end{tabular}
\end{center}
Daraus schliesst man wegen $D>D_{0.05}$, dass sich für $\alpha=0.05$
die Nullhypothese verwerfen lässt, wegen $D< D_{0.01}$ aber nicht
für $\alpha=0.01$. 
Es gibt in den Daten also einen Hinweis darauf, dass Junglenker
vielleicht mehr Unfälle verursachen, aber wirklich sicher sein kann
man da nicht.
\end{loesung}


% Quelle: https://www.open.ac.uk/socialsciences/spsstutorial/files/tutorials/chi-square.pdf
\begin{bewertung}
Hypothese ({\bf H}) 1 Punkt,
$\chi^2$-Test ({\bf X}) 1 Punkt,
Wahl von $\alpha$ ({\bf A}) 1 Punkt,
Freiheitsgrade und kritische Diskrepanz ({\bf F}) 1 Punkt,
Berechnung der Diskrepanz ({\bf D}) 1 Punkt,
Schlussfolgerung ({\bf S}) 1 Punkt.
\end{bewertung}



