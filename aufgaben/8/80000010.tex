Wünschelrutengänger behaupten, sie könnten mit Hilfe eines Astes
oder eines anderen mechanischen Werkzeugs erfühlen, wo sich Wasser
im Boden befindet. Wie könnte man einen Wünschelrutengänger testen?

\thema{Hypothesentest}

\begin{loesung}
Man könnte $n$ Flaschen im Boden vergraben, wovon die Hälfte
mit Sand die andere Hälfte mit Wasser gefüllt sind.
Der Wünschelrutengänger muss dann herausfinden, an welchen Stellen
sich Wasserflaschen befinden.
Natürlich darf die Person, die den
Wünschelrutengänger anleitet, auch nicht wissen, wo sich die
Wasserflaschen befinden (Doppelblindversuch).

Die Nullhypothese ist die Annahme, dass der Wunschelrutengänger mit
seinem Resultat mit
gleicher Wahrscheinlichkeit richtig oder falsch liegt.
Die Wahrscheinlichkeit, dass er  in $n$ Versuchen
$k$ Flaschen richtig detektiert, ist
binomialverteilt mit $p=\frac12$.
Zweifel an der Nullhypothese bekommen wir erst
dann, wenn die Anzahl $K$ der richtigen grösser als $K_+$ ist, wobei
$K_+$ so gewählt werden muss, dass unter der Nullhypothese
$P(K\le K_+)=1-\alpha$ ist. Für $\alpha=0.05$ und $n=10$ bedeutet
dies, dass er mindestens 8 mal richtig liegen muss. Diese Zahl
kann man mit der Funktion {\tt qbinom} in R berechnen:
\verbatimainput{aufg4.txt}
\end{loesung}

