W"unschelruteng"anger behaupten, sie k"onnten mit Hilfe eines Astes
oder eines anderen mechanischen Werkzeugs erf"uhlen, wo sich Wasser
im Boden befindet. Wie k"onnte man einen W"unschelruteng"anger testen?

\begin{loesung}
Man k"onnte $n$ Wasserflaschen und Flaschen voller Sand im Boden vergraben,
und den W"unschelruteng"anger herausfinden lassen, an welchen Stellen
sich Wasserflaschen befinden. Nat"urlich darf die Person, die den
W"unschelruteng"anger anleitet, auch nicht wissen, wo sich die
Wasserflaschen befinden (Doppelblindversuch).

Die Nullhypothese ist die Annahme, dass der Wunschelruteng"anger mit
seinem Resultat mit
gleicher Wahrscheinlichkeit richtig oder falsch liegt.
Die Wahrscheinlichkeit, dass er  in $n$ Versuchen
$k$ Flaschen richtig detektiert, ist
binomialverteilt mit $p=\frac12$.
Zweifel an der Nullhypothese bekommen wir erst
dann, wenn die Anzahl $K$ der richtigen gr"osser als $K_+$ ist, wobei
$K_+$ so gew"ahlt werden muss, dass unter der Nullhypothese
$P(K>K_+)=1-\alpha$ ist. F"ur $\alpha=0.05$ und $n=10$ bedeutet
dies, dass er mindestens 8 mal richtig liegen muss. Diese Zahl
kann man mit der Funktion {\tt pbinom} in R berechnen:
\verbatimainput{aufg4.txt}
\end{loesung}

