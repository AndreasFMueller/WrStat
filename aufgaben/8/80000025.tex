Zwei Mathematiker beginnen beim Skifahren eine Diskussion dar"uber, 
wer von den beiden der schnellere Fahrer sei.
Nat"urlich k"onnten Sie einfach an einem Rennen teilnehmen, doch
als Mathematiker sind sie sich der Tatsache bewusst, dass eine
Einzelmessung immer mit Fehlern behaftet ist, und dass es daher
zu einer reinen Zufallsentscheidung kommen k"onnte. Sie einigen sich daher
auf folgendes Protokoll. Jeder f"ahrt einen vorgegebenen Parcours
drei Mal, w"ahrend ihn der andere mit der Handykamera filmt.
Danach werden die Filme ausgewertet, um die Zeit f"ur den Parcours
zu ermitteln. Sie messen folgende Fahrzeiten:
\begin{center}
\begin{tabular}{|l|c|c|}
\hline
        &Fahrer A&Fahrer B\\
\hline
1.~Fahrt&  31.6  &  32.5  \\
2.~Fahrt&  27.8  &  32.9  \\
3.~Fahrt&  31.9  &  31.3  \\
\hline
\end{tabular}
\end{center}
Offenbar war das knapp: in den beiden ersten Fahrten ist Fahrer A schneller,
in der letzten aber Fahrer B.
Welcher Fahrer ist besser?

\begin{loesung}
Wir machen einen $t$-Test f"ur die Mittelwerte.
Die Nullhypothese lautet
\begin{quote}
Die beiden Fahrer haben normalverteilte Fahrzeiten mit gleichem
Mittelwert und gleicher Varianz.
\end{quote}
Der kritische Wert f"ur $3+3-2=4$ Freiheitsgrade und $\alpha=5\%$
ist $t_{\text{krit}}=2.1318$.
Die Berechnung von Mittelwert und Varianz sowie der Teststatistik $T$
liefert
\begin{align*}
\mu_A&=30.43
&
\mu_B&=32.23
\\
\sigma_A&=2.29
&
\sigma_B&=0.83
\\
T&=-1.28
\end{align*}
Dies ist betragsm"assig kleiner als der kritische Wert,
man kann also nicht sagen, dass einer der Skifahrer schneller sei.
\end{loesung}

\begin{bewertung}
$t$-Test ({\bf T}) 1 Punkt,
Nullhypothese ({\bf N}) 1 Punkt,
Berechnung des $t$-Wertes ({\bf T}) 1 Punkt,
Anzahl Freiheitsgrade ({\bf F}) 1 Punkt,
Wahl von $\alpha$ und Bestimmung des kritischen $t$-Wertes ({\bf A}),
Schlussfolgerung ({\bf S}) 1 Punkt.
\end{bewertung}

