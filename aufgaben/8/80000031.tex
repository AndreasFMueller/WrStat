In der Rassismus-Diskussion in den USA wird oft behauptet, dass
Schwarze besonders häufig Opfer von Gewaltverbrechen durch Weisse werden.
Youtuber Thunderf00t hat sich dieser Behauptung angenommen und die
Behauptung mit ziemlich eindeutigen Daten des FBI ganz offensichtlich
wiederlegt.

Mit den folgenden (fiktiven) Daten ist jedoch nicht mehr so eindeutig
erkennbar, ob Schwarze häufiger Opfer von Gewaltverbrechen werden als
auf Grund ihrer Häufigkeit in der Bevölkerung zu erwarten wäre:
\begin{center}
\begin{tabular}{|
>{$}c<{$}|
>{$}l<{$}|
>{$}c<{$}|
>{$}r<{$}|}
\hline
i&\text{Farbe}  &\text{Bevölkerungsanteil}&\text{Opfer}\\
\hline
1&\text{Weiss}  &  70\%                   &    1888    \\
2&\text{Schwarz}&  15\%                   &     387    \\
3&\text{andere} &  15\%                   &     342    \\
\hline
\end{tabular}
\end{center}
Können Sie darüber eine Entscheidung fällen?

\thema{Hypothesentest}
\thema{$\chi^2$-Test}

\begin{loesung}
Wir testen die Nullhypothese ``Die Anzahl der Opfer für jede Hauptfarbe
entspricht der Verteilung der Hauptfarben in der Bevölkerung''.
Wir testen diese Hypothese mit einem $\chi^2$-Test.
\begin{center}
\begin{tabular}{|
>{$}c<{$}|
>{$}r<{$}
>{$}r<{$}|
>{$}r<{$}
>{$}r<{$}
|}
\hline
i& p_i&   n_i&n_i-np_i&(n_i-np_i)^2/np_i\\
\hline
1&0.70&  1888& 56.10&1.7180\\
2&0.15&   387& -5.55&0.0785\\
3&0.15&   342&-50.55&6.5095\\
\hline
 &    &n=2617&      &D=8.3060\\
\hline
\end{tabular}
\end{center}
Der kritische Wert für die Diskrepanz bei zwei Freiheitsgraden und
$\alpha = 5\%$ ist $D_{\text{krit}}=5.991$.
Da $D>D_{\text{krit}}$ müssen wir die Nullhypothese verwerfen und schliessen
daher, dass die Hautfarben der Opfer anders verteilt sind als die Hauptfarben
in der Bevölkerung.
\end{loesung}

\begin{bewertung}
Nullhypothese ({\bf H}) 1 Punkt,
$\chi^2$-Test ({\bf$\chi$}) 1 Punkt,
Wahl von $\alpha$ ({\bf A}) 1 Punkt,
Freiheitgrade und kritischer Wert der Diskrepanz von ({\bf K})  1 Punkt,
Berechnung der Diskrepanz ({\bf D}) 1 Punkt,
Schlussfolgerung ({\bf S}) 1 Punkt.
\end{bewertung}

