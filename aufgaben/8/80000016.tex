In der Stadt Zürich waren gemäss Volkszählung 2010 die Religionen
wie folgt verteilt:
\begin{center}
\begin{tabular}{|l|r|r|}
\hline
Religion&Einwohner&prozentual\\
\hline
Christen&   250535&68.9\%\\
Muslime &    20888& 5.8\%\\
Juden   &     4903& 1.4\%\\
Andere  &     6859& 1.9\%\\
Keine   &    80088&22.1\%\\
\hline
\end{tabular}
\end{center}
Ein Politiker stellt in der Schulklasse seines Sohnes folgende
Verteilung fest
\begin{center}
\begin{tabular}{|l|r|r|}
\hline
Religion&Schüler&prozentual\\
\hline
Christen&      13&    59.1\%\\
Muslime &       4&    18.2\%\\
Juden   &       1&     4.5\%\\
Andere  &       0&     0.0\%\\
Keine   &       4&    18.2\%\\
\hline
\end{tabular}
\end{center}
Ihm fällt vor allem die grosse Abweichung bei den Muslimen
zu Lasten der Christen ins Auge, was er in seiner Wahlkampagne
sofort ausschlachtet und den Niedergang der christlichen Kultur
bejammert.
Er behauptet sogar, die Zahlen aus der Volkszählung würden die
Realität nicht richtig wiedergeben. Hat er recht?

\thema{Hypothesentest}
%\thema{$\chi^2$-Test}

\begin{loesung}
Es geht um die Frage, ob die in der Schulklasse beobachtete Religionsverteilung
stark genug von der in der Volkszählung erhobenen Verteilung abweicht,
dass man behaupten kann, dass die Schulklasse eine andere Religionsverteilung
besitzt.
Diese Frage kann mit Hilfe eines Hypothesentests basierend auf der
$\chi^2$-Verteilung beantwortet werden. Die Nullhypothese ist:
\begin{quotation}
\parindent 0pt
Die Religionsverteilung in der Schulklasse ist die gleiche wie in
der Gesamtbevölkerung gemäss Volkszählung.
\end{quotation}
Wir müssen die Diskrepanz berechnen:
\begin{center}
\begin{tabular}{|c|l|r|r|r|r|r|r|}
\hline
$i$&Religion&Einwohner&$p_i$    &$n_i$&$np_i$    &$n_i-np_i$&$(n_i-np_i)^2/np_i$\\
\hline
1  &Christen&   250535&0.6897   &   13&15.1725   &-2.1725   &0.3111\\
2  & Muslime&    20888&0.0575   &    4& 1.2650   & 2.7350   &5.9133\\
3  &   Juden&     4903&0.0135   &    1& 0.2969   & 0.7030   &1.6647\\
4  &  Andere&     6859&0.0189   &    0& 0.4154   &-0.4154   &0.4154\\
5  &   Keine&    80088&0.2205   &    4& 4.8502   &-0.8502   &0.1490\\
\hline
   &        &         &         &$n=22$&         &          &$D=8.4536$\\
\hline
\end{tabular}
\end{center}
Wir verlangen für unseren Test $\alpha=5\%$, die $\chi^2$-Tabelle
für 4 Freiheitsgrade liefert einen kritischen Wert der Diskrepanz
von $D_{\text{krit}}=9.488$. Da der beobachtete Wert $D=8.4536$
nicht grösser ist als der kritische Wert, gibt es keinen Grund an
der Nullhypothese zu zweifeln.

Allerdings gibt es auch Bedenken bezüglich der Zulässigkeit des
Testes.
Die Faustregel, dass für jeden Ausgang mindestens
fünf Beobachtungen vorliegen müssen, ist nicht erfüllt.
\end{loesung}

\begin{bewertung}
Chi-Quadrat-Test ({\bf X}) 1 Punkt,
Hypothese ({\bf H}) 1 Punkt,
Freiheitsgrade ({\bf F}) 1 Punkt,
Berechnung der Diskrepanz ({\bf D}) 1 Punkt,
Kritische Diskrepanz ({\bf K}) 1 Punkt,
Schlussfolgerung ({\bf S}) 1 Punkt.
\end{bewertung}
