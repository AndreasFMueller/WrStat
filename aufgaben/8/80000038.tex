Im Jahre 1996 wurde in Bahia in Südamerika untersucht, ob besser
klassifizierte Hotels auch mehr Personal für die Betreuung ihrer
Gäste anstellen.
Folgende Resultate wurde dabei gefunden:
\begin{center}
\begin{tabular}{|c|r|r|r|r|}
\hline
Sterne&Hotels&Zimmer&Anteil&Angestellte\\
\hline
     5&     5&  1205&0.1869&        769\\
     4&    16&  1747&0.2710&       1189\\
     3&    32&  1865&0.2893&        945\\
     2&    10&   303&0.0470&        123\\
     1&    41&  1327&0.2058&        418\\
\hline
Total&    104&  6447&      &       3444\\
\hline
\end{tabular}
\end{center}
In der vierten Spalte steht der Anteil der Zimmer dieser Klasse an der
Gesamtzahl der Zimmer.
Auf den ersten Blick sieht es so aus, als würden in allen Kategorieren
ungefähr ähnlich viele Angestellte pro Zimmer beschäftigt.
Können Sie dies bestätigen?

\thema{Hypothesentest}
%\thema{$\chi^2$-Test}

\begin{loesung}
Wir führen einen $\chi^2$-Test dafür durch, ob die Angestelltenverteilung
zur Verteilung der Hotelzimmer auf die Sterne-Klassen passt.
Die Nullhypothese is
\begin{quote}
Die Verteilung der Angestellten auf die Hotelklassen entspricht den Anteilen
der Zimmer an den Hotelklassen.
\end{quote}
Wir wählen $\alpha=5\%$, die Anzahl der Freiheitsgrade ist 4, der kritische
Wert der Diskrepanz ist $D_{\text{krit}}=9.488$.
Die Diskrepanz wird mit der folgenden Tabelle berechnet:
\begin{center}
\begin{tabular}{|>{$}c<{$}|>{$}r<{$}|>{$}r<{$}|>{$}r<{$}|>{$}r<{$}|}
\hline
i&p_i   &  np_i& n_i& (np_i-n_i)^2/np_i \\
\hline
5&0.1869& 815.6&   769&     24.38         \\
4&0.2710&1182.6&  1189&     70.09         \\
3&0.2893&1272.5&   945&      2.64         \\
2&0.0470& 205.1&   123&      9.33         \\
1&0.2058& 898.1&   418&    119.36         \\
\hline
 &      &      &n=3444& D= 225.80         \\
\hline
\end{tabular}
\end{center}
Die Diskrepanz ist offenbar viel grösser als der kritische Wert, die
Nullhypothese muss verworfen werden.

Man kann allerdings nicht schliessen, dass die billigeren Kategorien
weniger Angestellte pro Zimmer beschäftigen, denn das Maximum der
Anzahl Beschäftigte pro Zimmer wird für 4-Stern-Hotels erreicht.
\end{loesung}

\begin{bewertung}
Nullhypothese ({\bf N})  1 Punkt,
Berechnung der Diskrepanz ({\bf D}) 1 Punkt,
Anzahl Freiheitsgrade ({\bf F}) 1 Punkt,
Wahl eines $\alpha$ ({\bf A}) 1 Punkt,
kritischer Wert für  die Diskrepanz ({\bf K}) 1 Punkt,
Schlussfolgerung ({\bf S}) 1 Punkt.
\end{bewertung}



