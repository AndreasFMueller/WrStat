Zufallszahlen. Von einer Statistik-Software erwartet man, dass Sie auch
Simulationen durchführen kann, wozu Zufallszahlen unerlässlich sind.
Die Funktion {\tt runif} erzeugt einen Vektor von ``gleichverteilte''
Zufallszahlen zwischen 0 und 1.
\begin{teilaufgaben}
\item Erzeugen Sie einen Vektor von 100 Zufallszahlen zwischen 0 und 1.
\item Erzeugen Sie einen Vektor von 100 Zahlen, die Quadrate
von Zufallszahlen zwischen 1 und 2 sind
\end{teilaufgaben}

\thema{R}

\begin{loesung}
Nach dem Start von R auf dessen Kommandozeile:
\verbatimainput{aufg3.script}
\end{loesung}

