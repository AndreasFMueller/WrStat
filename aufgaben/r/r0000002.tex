Vektoren. Oft muss man Rechnungen mit vielen Zahlenwerten durchführen,
und dann Mittelwerte oder ähnliche Dinge berechnen. R bietet spezielle
Funktionen zum erzeugen von Vektoren, und erlaubt auch mit Vektoren
zu rechnen, so wie man es in der linearen Algebra gelernt hat.
\begin{teilaufgaben}
\item Mit der Funktion {\tt c} können einzelne Elemente oder
Vektoren verkettet werden. Erzeugen Sie ein Paar bestehend aus
den Zahlen $\pi$ und $\sqrt{2}$.
\item Erzeugen Sie mit der Funktion {\tt 1:10} einen Vektor bestehend
aus den Zahlen von 1 bis 10.
\item Erzeugen Sie einen Vektor aller geraden Zahlen von 2 bis 20
\item Erzeugen Sie einen Vektor mit allen Quadratwurzeln der natürlichen
Zahlen von 1 bis 100
\item Berechnen Sie die Summe und den Mittelwert aller Zahlen von 1 bis 100
\quad
{\it Hinweis\/:} Funktionen {\tt sum} und {\tt mean}
\item Berechnen Sie die Summe und den Mittelwert aller Quadratzahlen von 1 bis 100
\end{teilaufgaben}

\begin{loesung}
Nach dem Start von R auf dessen Kommandozeile:
\verbatimainput{aufg2.script}
\end{loesung}

