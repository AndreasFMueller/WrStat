Graphik. Ein Bild sagt mehr als tausend Worte. Die von R erzeugten
Graphiken haben eine hohe Qualit"at, erfordern jedoch etwas "Ubung.
Meist beginnt man damit, ein leeres Koordinatennetz zu zeichen, in welches
man dann die Daten einf"ugt.
\begin{teilaufgaben}
\item Erzeugen Sie mit dem Befehl \verb+plot(0, 0, type = "0")+ eine
leeres Koordinatenkreuz.
\item Lesen Sie in der Hilfefunktion nach, welche zus"atzlichen
Argumente die Funktion {\tt plot} versteht, und setzen sie Titel,
Untertitel, Bezeichnung der Achsen.
\item Lesen Sie in der Hilfefunktion unter {\tt plot.defaults} die
weiteren m"oglichen Einstellungen nach, und f"ugen Sie ihrem Plot-Befehl
Argumente hinzu, damit nur der $t$-Bereich zwischen 1 und 10 und der
$v$-Bereich zwischen $-10$ und $10$ geplottet wird.
\item F"ugen Sie jetzt die Punkte aus dem Frame {\tt f} mit Hilfe des
Plot-Befehls {\tt points} in roter Farbe hinzu.
\item Erzeugen Sie einen Vektor von ganzzahligen $x$-Werten zwischen 1
und 10. Berechnen Sie die dazugeh"origen $y$-Werte nach der Formel $y=2x-11$.
Zeichnen Sie die Linien, die die aus {\tt x} und {\tt y} gebildeten
Punkte verbinden mit Hilfe der Funktion {\tt points}.
\item Die so erzeugte Graphik hat noch nicht Publikationsqualit"at, da
sie bisher nur als Bitmap-Bild auf dem Bildschirm ausgegeben wurde.
Mit dem Befehl {\tt pdf} k"onnen Sie den Namen eines PDF-Files angeben,
in welches die Graphik ausggeben werden soll. Erzeugen sie die eben
generierte Graphik erneut, diesmal aber in einem PDF-File.
\end{teilaufgaben}

\begin{loesung}
Die PDF-Graphik kann mit folgendem Script erzeugt werden:
\verbatimainput{aufg7.R}
Es erzeugt als Resultat Abbildung~\ref{r0000007:aufg7}.

Auf einigen Plattformen muss das Graphik-Device explizit geschlossen
werden, damit das PDF-File ge"offnet werden kann, dies erreicht man
mit dem Befehl {\tt dev.off()}.
\begin{figure}
\begin{center}
\includeagraphics[width=\hsize]{aufg7.pdf}
\end{center}
\caption{L"osung zu Aufgabe 7\label{r0000007:aufg7}}
\end{figure}
\end{loesung}

