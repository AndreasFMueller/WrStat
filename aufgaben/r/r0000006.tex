Daten. F"ur statistische Datenanalysen m"ussen Daten zun"achst eingelesen
oder auch in Files geschrieben werden k"onnen.
R kann mit vielen verschiedenen Datenquellen umgehen
(plattformabh"angig), auf jeden Fall aber mit CSV Files.
"Ublicherweise gruppiert man die Daten nicht einfach nur in Vektoren, sondern
in einem sogenannten Frame. Dies sieht aus wie ein Tabelle, enth"alt aber
zus"atzlichen Informationen "uber Namen und Datentyp.
\begin{teilaufgaben}
\item Verwenden Sie den Befehl
\begin{verbatim}
f <- data.frame(t = x, v = y)
\end{verbatim}
ein Datenframe, welches die Daten aus der vorangegangenen
Aufgabe enth"alt. Die erste Spalte enth"alt die Werte aus dem Vektor
{\tt x} unter dem Namen {\tt t}.
\item Schreiben Sie die Daten in ein File {\tt daten.csv} mit Hilfe der
Funktion {\tt write.csv}.
\item Lesen Sie die Daten mit der Funktion {\tt read.csv} wieder ein.
\end{teilaufgaben}

\begin{loesung}
Fortsetzung des Protokolls aus der Aufgabe 6:
\verbatimainput{aufg6.script}

Sie k"onnen die Daten jederzeit durch Aufrufen des Frames {\tt f}
bzw.~{\tt f2} anschauen. Auf eine einzelne Spalte k"onnen Sie "uber
den Namen der Spalte zugreifen. Die $t$-Spalte ist {\tt f\$t},
die $v$-Spalte ist {\tt f\$v}.
\end{loesung}

