Starten von R, Benutzung als ``Taschenrechner''.
R bietet eine interaktive Arbeitsumgebung, in der Berechnungen angestellt
werden können. Insbesondere können beliebige arithmetische Operationen
und die üblichen Funktionen verwendet werden, wobei natürlich eine
Computertaugliche Notation verwendet werden muss, zum Beispiel
{\tt sqrt()} für $\sqrt{\mathstrut\;}$ oder {\tt pi} für $\pi$.
Starten Sie das Programm~R, und berechnen Sie folgende Ausdrücke
\begin{teilaufgaben}
\item $1+2\cdot 3$
\item $(1+2)\cdot 3$
\item $\sqrt{2}\sqrt{3}\sqrt{6}$
\item $\sin \frac{\pi}4$
\item $170!\qquad${\it Hinweis\/}: Funktion {\tt factorial}
\item $2\binom{99}{9}-\binom{98}{8}$ \qquad {\it Hinweis\/:} Funktion {\tt choose}
\end{teilaufgaben}
{\it Hinweis\/:} Zusätzliche Information können Sie mit der Help-Funktion
finden. So zeigt zum Beispiel {\tt help(factorial)} Details zur Funktion
{\tt factorial} an. Verwenden Sie diese Funktion auch in allen folgenden
Aufgaben, um mehr Information über die zur Lösung vorgeschlagenen
Funktionen zu erfahren.

\begin{loesung}
Nach dem Start von R auf dessen Kommandozeile:
\verbatimainput{aufg1.script}
\end{loesung}

