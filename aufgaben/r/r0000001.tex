Starten von R, Benutzung als ``Taschenrechner''.
R bietet eine interaktive Arbeitsumgebung, in der Berechnungen angestellt
werden k"onnen. Insbesondere k"onnen beliebige arithmetische Operationen
und die "ublichen Funktionen verwendet werden, wobei nat"urlich eine
Computertaugliche Notation verwendet werden muss, zum Beispiel
{\tt sqrt()} f"ur $\sqrt{\mathstrut\;}$ oder {\tt pi} f"ur $\pi$.
Starten Sie das Programm~R, und berechnen Sie folgende Ausdr"ucke
\begin{teilaufgaben}
\item $1+2\cdot 3$
\item $(1+2)\cdot 3$
\item $\sqrt{2}\sqrt{3}\sqrt{6}$
\item $\sin \frac{\pi}4$
\item $170!\qquad${\it Hinweis\/}: Funktion {\tt factorial}
\item $2\binom{99}{9}-\binom{98}{8}$ \qquad {\it Hinweis\/:} Funktion {\tt choose}
\end{teilaufgaben}
{\it Hinweis\/:} Zus"atzliche Information k"onnen Sie mit der Help-Funktion
finden. So zeigt zum Beispiel {\tt help(factorial)} Details zur Funktion
{\tt factorial} an. Verwenden Sie diese Funktion auch in allen folgenden
Aufgaben, um mehr Information "uber die zur L"osung vorgeschlagenen
Funktionen zu erfahren.

\begin{loesung}
Nach dem Start von R auf dessen Kommandozeile:
\verbatimainput{aufg1.script}
\end{loesung}

