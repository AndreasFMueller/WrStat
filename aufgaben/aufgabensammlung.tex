%
% aufgabensammlung.tex -- Aufgabensammlung zu WrStat
%
% (c) 2009 Prof. Dr. Andreas Mueller, HSR
%
\documentclass[a4paper,12pt]{book}
\usepackage[utf8]{inputenc}
\usepackage[T1]{fontenc}
\usepackage{CJKutf8}
\usepackage{times}
\usepackage{geometry}
\geometry{papersize={210mm,297mm},total={160mm,240mm},top=31mm,bindingoffset=15mm,marginparwidth=9mm}
\usepackage{alltt}
\usepackage{verbatim}
\usepackage{fancyhdr}
\usepackage{amsmath}
\usepackage{amssymb}
\usepackage{amsfonts}
\usepackage{amsthm}
\usepackage{textcomp}
\usepackage{graphicx}
\usepackage{array}
\usepackage{ifthen}
\usepackage{multirow}
\usepackage{txfonts}
%\usepackage[basic]{circ}
\usepackage[all]{xy}
\usepackage{algorithm}
\usepackage{algorithmic}
\usepackage{makeidx}
\usepackage{paralist}
\usepackage{color}
\usepackage{xcolor}
\usepackage{epsdice}
\usepackage{mathtools}
\usepackage{german}
\usepackage[colorlinks=true]{hyperref} % changed
\hypersetup{
    linktoc=all,
    linkcolor=blue
}
\usepackage{environ}
\usepackage{nicefrac}
\usepackage{etoolbox}
\usepackage{tikz}
\usetikzlibrary{arrows,scopes}
\usepackage{bm}

\setlength{\headheight}{15pt} % fix for headheight warning

\newcommand\myatop[2]{\genfrac{}{}{0pt}{}{#1}{#2}}

%%%%%%%%%%%%%%%%%%%%%%%
%% Copyleft
%% Walter A. Kehowski
%% Department of Mathematics
%% Glendale Community College
%% walter.kehowski@gcmail.maricopa.edu
%% \begin{linsys}{2}
%% -x & + & 4y & = & 8\\
%% -3x & - & 2y & = & 6
%% \end{linsys}
%%%%%%%%%%%%%%%%%%%%%%%
%\makeatletter
%% math-mode column types ------------------
\newcolumntype{\linsysR}{>{$}r<{$}}
\newcolumntype{\linsysL}{>{$}l<{$}}
\newcolumntype{\linsysC}{>{$}c<{$}}
\newenvironment{linsys}[1]{%
\begin{tabular}{*{#1}{\linsysR@{\;}\linsysC}@{\;}\linsysR}}%
{\end{tabular}}
%\makeatother
\endinput

\makeindex
\begin{document}
\pagestyle{fancy}
\lhead{Aufgabensammlung}
\rhead{}
\frontmatter
\newcommand\HRule{\noindent\rule{\linewidth}{1.5pt}}
\begin{titlepage}
\vspace*{\stretch{1}}
\HRule
\vspace*{2pt}
\begin{flushright}
{\Huge
Wahrscheinlichkeit und Statistik:\\
\bigskip
Aufgabensammlung}
\end{flushright}
\HRule
\begin{flushright}
\vspace{30pt}
\LARGE
Andreas Müller
\end{flushright}
\vspace*{\stretch{2}}
\begin{center}
HSR Hochschule für Technik, Rapperswil, 2011-2020\\
OST Ostschweizer Fachhochschule, Rapperswil, 2020-2021
\end{center}
\end{titlepage}
\hypersetup{
        linktoc=all,
        linkcolor=blue
}
\rhead{Inhaltsverzeichnis}
\tableofcontents
\newenvironment{beispiel}[1][Beispiel]{%
\begin{proof}[#1]%
\renewcommand{\qedsymbol}{$\bigcirc$}
}{\end{proof}}
\mainmatter
%
% uebung.tex -- gemeinsame Makros fuer Uebungsblaetter
%
% (c) 2006-2017 Prof. Dr. Andreas Mueller, HSR
%
\newcounter{uebungsaufgabe}
\newboolean{loesungen}
% environment fuer uebungsaufgaben
\newenvironment{uebungsaufgaben}{
\begin{list}{\arabic{uebungsaufgabe}.}
  {\usecounter{uebungsaufgabe}
  \setlength{\labelwidth}{2cm}
  \setlength{\leftmargin}{0pt}
  \setlength{\labelsep}{5mm}
  \setlength{\rightmargin}{0pt}
  \setlength{\itemindent}{0pt}
}}{\end{list}\vfill\pagebreak}
% Teilaufgaben
\newenvironment{teilaufgaben}{
\begin{enumerate}
\renewcommand{\labelenumi}{\alph{enumi})}
}{\end{enumerate}}
% Loesung
\NewEnviron{loesung}{%
\begin{proof}[Lösung]%
\renewcommand{\qedsymbol}{$\bigcirc$}
\BODY
\end{proof}}
\NewEnviron{diskussion}{
\BODY
\bigskip
}
\def\keineloesungen{%
\RenewEnviron{loesung}{\relax}
\RenewEnviron{diskussion}{\relax}
\setboolean{loesungen}{false}
}
% Hinweis
\newenvironment{hinweis}{%
\renewcommand{\qedsymbol}{}
\begin{proof}[Hinweis]}{\end{proof}}
% Aufgabe aus der Sammlung wiedergeben
\newboolean{themastatus}
\setboolean{themastatus}{false}
\newcounter{problemcounter}[chapter]
\def\aufgabepath{./}
\def\ainput#1{\input\aufgabepath/#1}
\def\verbatimainput#1{\expandafter\verbatiminput{\aufgabepath/#1}}
\def\aufgabetoplevel#1{%
\expandafter\def\expandafter\inputpath{#1}%
\let\aufgabepath=\inputpath
}
\def\includeagraphics[#1]#2{\expandafter\includegraphics[#1]{\aufgabepath#2}}
% \aufgabe
\newcommand{\aufgabe}[2]{%
  \expandafter\def\csname themalist\endcsname{}
  \setboolean{themastatus}{false}
  \refstepcounter{problemcounter}%
  \label{#2}
  \bigskip{\parindent0pt\strut}\hbox{\bf\theproblemcounter. }%
  \marginpar{\raggedright\tiny #2}%
  \expandafter\def\csname currentaufgabe\endcsname{#2}%
  \expandafter\def\csname aufgabepath\endcsname{\inputpath/#1/#2/}%
  \expandafter\input{\inputpath#1/#2.tex}
  %\medskip
  \ifthenelse{\boolean{themastatus}}{
    \parindent 0pt
    {\sc Thema:} {\small \themalist.}}{%
  }
  \bigskip

}
\renewcommand\theproblemcounter{\thechapter.\arabic{problemcounter}}
% Bewertung
\NewEnviron{bewertung}{\footnotesize
\renewcommand{\qedsymbol}{}
\begin{proof}[Bewertung]
\BODY
\end{proof}}
% oft benutzte Macros
\def\blank{\text{\textvisiblespace}}
%
% macros fuer den thema-Index
%
\newcommand{\openthemaindex}{%
  \newwrite\themaindex
  \immediate\openout\themaindex=thema.tix
}
%
\newcommand{\closethemaindex}{\immediate\closeout\themaindex}
%
\def\themalink#1#2{\hyperref[thema:#1]{#2}}
\def\themaL#1#2{%
  \ifthenelse{\boolean{themastatus}}{%
    \xappto{\themalist}{, \noexpand\themalink{#1}{#2}}
  }{%
    \xdef\themalist{\noexpand\themalink{#1}{#2}}
    \setboolean{themastatus}{true}
  }
  \immediate\write\themaindex%
  {{\thechapter}{#1}{#2}{\arabic{problemcounter}}{\thechapter.\arabic{problemcounter}}{\currentaufgabe}}%
}
\def\thema#1{\themaL{#1}{#1}}
% Thema-Information anzeigen
\def\themasection#1#2{ \item[#2]\label{thema:#1} }
\newcommand{\printthemata}{
  \IfFileExists{./thema.tex}{
    \chapter*{Aufgaben nach Themen}
    \begin{description}
    \input{thema.tex}
    \end{description}
  }{}
}

\openthemaindex
\definecolor{darkgreen}{rgb}{0,0.6,0}
\setboolean{loesungen}{true}
\newboolean{pruefung}
\setboolean{pruefung}{false}
\aufgabetoplevel{./}

\chapter{Kombinatorik}
\lhead{Kapitel~\thechapter}
\rhead{Kombinatorik}
\input{1.tex}
\chapter{Ereignisse und ihre Wahrscheinlichkeit}
\lhead{Kapitel~\thechapter}
\rhead{Ereignisse und ihre Wahrscheinlichkeit}
\input{2.tex}
%\chapter{Wahrscheinlichkeit und Kombinatorik}
%\rhead{Wahrscheinlichkeit Kombinatorik}
%\input 3.tex
\chapter{Erwartungswert und Varianz}
\lhead{Kapitel~\thechapter}
\rhead{Erwartungswert und Varianz}
\input{4.tex}
\chapter{Wahrscheinlichkeitsverteilung}
\lhead{Kapitel~\thechapter}
\rhead{Wahrscheinlichkeitsverteilung}
\input{5.tex}
\chapter{Ein Katalog von Wahrscheinlichkeitsverteilungen}
\lhead{Kapitel~\thechapter}
\rhead{Ein Katalog von Wahrscheinlichkeitsverteilungen}
\input{6.tex}
\chapter{Schätzen}
\lhead{Kapitel~\thechapter}
\rhead{Schätzen}
\input{7.tex}
\chapter{Testen}
\lhead{Kapitel~\thechapter}
\rhead{Testen}
\input{8.tex}
\chapter{Filtern}
\lhead{Kapitel~\thechapter}
\rhead{Filtern}
\input{9.tex}
\chapter{Einführung in die Software R}
\rhead{Einführung in R}
Die Aufgaben in diesem Kapitel sind als Lernaufgabe gedacht, mit deren
Hilfe man sich mit der Bedienung der Software R vertraut machen kann.
R ist freie Software, Sie können Sie von \url{www.r-project.org}
herunterladen und kostenfrei benutzen.

Die Versionen von R für Windows und Mac OS X haben ein GUI und eine darin
integrierte Hilfefunktion, welche die ersten Schritte vereinfacht. Alle
Versionen haben eine Kommandozeilen-Version, welche genügt, die nachstehend
beschriebenen Probleme zu lösen.

Natürlich kann diese "Ubungsserie nur einen ersten Einblick in die
Funktionen von R bieten. Tausende von Funktionen sind in dem über 3000
Seiten starken Benutzerhandbuch dokumentiert, ausserdem kann R durch
zugeladene ``packages'' erweitert werden. Es existiert ein grosses Netzwerk
von Zusatz-Packages, die aus dem CRAN, dem Comprehensive R Archive
Network (nach dem Muster des CPAN für Perl) heruntergeladen werden
können.

\bigskip
\input{r.tex}
\closethemaindex
\printthemata
%
% aufgabensammlung.tex -- Aufgabensammlung zu WrStat
%
% (c) 2009 Prof. Dr. Andreas Mueller, HSR
%
\documentclass[a4paper,12pt]{book}
\usepackage{german}
\usepackage[utf8]{inputenc}
\usepackage[T1]{fontenc}
\usepackage{times}
\usepackage{geometry}
\geometry{papersize={210mm,297mm},total={160mm,240mm},top=31mm,bindingoffset=15mm,marginparwidth=9mm}
\usepackage{alltt}
\usepackage{verbatim}
\usepackage{fancyhdr}
\usepackage{amsmath}
\usepackage{amssymb}
\usepackage{amsfonts}
\usepackage{amsthm}
\usepackage{textcomp}
\usepackage{graphicx}
\usepackage{array}
\usepackage{ifthen}
\usepackage{multirow}
\usepackage{txfonts}
%\usepackage[basic]{circ}
\usepackage[all]{xy}
\usepackage{algorithm}
\usepackage{algorithmic}
\usepackage{makeidx}
\usepackage{paralist}
\usepackage{color}
\usepackage[colorlinks=true]{hyperref} % changed
\hypersetup{
    linktoc=all,
    linkcolor=blue
}
\usepackage{environ}
\usepackage{nicefrac}
\usepackage{etoolbox}
\usepackage{tikz}
\usetikzlibrary{arrows,scopes}
\usepackage{bm}

\setlength{\headheight}{15pt} % fix for headheight warning

\newcommand\myatop[2]{\genfrac{}{}{0pt}{}{#1}{#2}}

\input linsys.tex
\makeindex
\begin{document}
\pagestyle{fancy}
\lhead{Aufgabensammlung}
\rhead{}
\frontmatter
\newcommand\HRule{\noindent\rule{\linewidth}{1.5pt}}
\begin{titlepage}
\vspace*{\stretch{1}}
\HRule
\vspace*{2pt}
\begin{flushright}
{\Huge
Wahrscheinlichkeit und Statistik:\\
\bigskip
Aufgabensammlung}
\end{flushright}
\HRule
\begin{flushright}
\vspace{30pt}
\LARGE
Andreas Müller
\end{flushright}
\vspace*{\stretch{2}}
\begin{center}
Hochschule für Technik, Rapperswil, 2011-2019
\end{center}
\end{titlepage}
\hypersetup{
        linktoc=all,
        linkcolor=blue
}
\rhead{Inhaltsverzeichnis}
\tableofcontents
\newenvironment{beispiel}[1][Beispiel]{%
\begin{proof}[#1]%
\renewcommand{\qedsymbol}{$\bigcirc$}
}{\end{proof}}
\mainmatter
\input uebungen.tex
\openthemaindex
\setboolean{loesungen}{true}
\newboolean{pruefung}
\setboolean{pruefung}{false}
\aufgabetoplevel{./}

\chapter{Kombinatorik}
\lhead{Kapitel~\thechapter}
\rhead{Kombinatorik}
\input 1.tex
\chapter{Ereignisse und ihre Wahrscheinlichkeit}
\lhead{Kapitel~\thechapter}
\rhead{Ereignisse und ihre Wahrscheinlichkeit}
\input 2.tex
%\chapter{Wahrscheinlichkeit und Kombinatorik}
%\rhead{Wahrscheinlichkeit Kombinatorik}
%\input 3.tex
\chapter{Erwartungswert und Varianz}
\lhead{Kapitel~\thechapter}
\rhead{Erwartungswert und Varianz}
\input 4.tex
\chapter{Wahrscheinlichkeitsverteilung}
\lhead{Kapitel~\thechapter}
\rhead{Wahrscheinlichkeitsverteilung}
\input 5.tex
\chapter{Ein Katalog von Wahrscheinlichkeitsverteilungen}
\lhead{Kapitel~\thechapter}
\rhead{Ein Katalog von Wahrscheinlichkeitsverteilungen}
\input 6.tex
\chapter{Schätzen}
\lhead{Kapitel~\thechapter}
\rhead{Schätzen}
\input 7.tex
\chapter{Testen}
\lhead{Kapitel~\thechapter}
\rhead{Testen}
\input 8.tex
\chapter{Filtern}
\lhead{Kapitel~\thechapter}
\rhead{Filtern}
\input 9.tex
\chapter{Einführung in die Software R}
\rhead{Einführung in R}
Die Aufgaben in diesem Kapitel sind als Lernaufgabe gedacht, mit deren
Hilfe man sich mit der Bedienung der Software R vertraut machen kann.
R ist freie Software, Sie können Sie von \url{www.r-project.org}
herunterladen und kostenfrei benutzen.

Die Versionen von R für Windows und Mac OS X haben ein GUI und eine darin
integrierte Hilfefunktion, welche die ersten Schritte vereinfacht. Alle
Versionen haben eine Kommandozeilen-Version, welche genügt, die nachstehend
beschriebenen Probleme zu lösen.

Natürlich kann diese "Ubungsserie nur einen ersten Einblick in die
Funktionen von R bieten. Tausende von Funktionen sind in dem über 3000
Seiten starken Benutzerhandbuch dokumentiert, ausserdem kann R durch
zugeladene ``packages'' erweitert werden. Es existiert ein grosses Netzwerk
von Zusatz-Packages, die aus dem CRAN, dem Comprehensive R Archive
Network (nach dem Muster des CPAN für Perl) heruntergeladen werden
können.

\bigskip
\input{r.tex}
\closethemaindex
\printthemata
\input aufgabensammlung.ind
\end{document}

\end{document}
