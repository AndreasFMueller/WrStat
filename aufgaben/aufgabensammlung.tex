%
% aufgabensammlung.tex -- Aufgabensammlung zu WrStat
%
% (c) 2009 Prof. Dr. Andreas Mueller, HSR
%
\documentclass[a4paper,12pt]{book}
\usepackage{german}
\usepackage{times}
\usepackage{geometry}
\geometry{papersize={210mm,297mm},total={160mm,240mm},top=31mm,bindingoffset=15mm}
\usepackage{alltt}
\usepackage{verbatim}
\usepackage{fancyhdr}
\usepackage{amsmath}
\usepackage{amssymb}
\usepackage{amsfonts}
\usepackage{amsthm}
\usepackage{textcomp}
\usepackage{graphicx}
\usepackage{array}
\usepackage{ifthen}
\usepackage{multirow}
\usepackage{txfonts}
%\usepackage[basic]{circ}
\usepackage[all]{xy}
\usepackage{algorithm}
\usepackage{algorithmic}
\usepackage{makeidx}
\usepackage{paralist}
\usepackage{color}
\usepackage[colorlinks=true]{hyperref} % changed
\usepackage{environ}
\usepackage{nicefrac}

\setlength{\headheight}{15pt} % fix for headheight warning

\newcommand\myatop[2]{\genfrac{}{}{0pt}{}{#1}{#2}}

\input linsys.tex
\makeindex
\begin{document}
\pagestyle{fancy}
\lhead{Aufgabensammlung}
\rhead{}
\frontmatter
\newcommand\HRule{\noindent\rule{\linewidth}{1.5pt}}
\begin{titlepage}
\vspace*{\stretch{1}}
\HRule
\vspace*{2pt}
\begin{flushright}
{\Huge
Wahrscheinlichkeit und Statistik:\\
\bigskip
Aufgabensammlung}
\end{flushright}
\HRule
\begin{flushright}
\vspace{30pt}
\LARGE
Andreas M"uller
\end{flushright}
\vspace*{\stretch{2}}
\begin{center}
Hochschule f"ur Technik, Rapperswil, 2011-2016
\end{center}
\end{titlepage}
\hypersetup{
        linktoc=all,
        linkcolor=blue
}
\rhead{Inhaltsverzeichnis}
\tableofcontents
\newenvironment{beispiel}[1][Beispiel]{%
\begin{proof}[#1]%
\renewcommand{\qedsymbol}{$\bigcirc$}
}{\end{proof}}
\mainmatter
\input uebungen.tex
\setboolean{loesungen}{true}
\aufgabetoplevel{./}

\chapter{Kombinatorik}
\rhead{Kombinatorik}
\input 1.tex
\chapter{Ereignisse und ihre Wahrscheinlichkeit}
\rhead{Ereignisse und ihre Wahrscheinlichkeit}
\input 2.tex
%\chapter{Wahrscheinlichkeit und Kombinatorik}
%\rhead{Wahrscheinlichkeit Kombinatorik}
%\input 3.tex
\chapter{Erwartungswert und Varianz}
\rhead{Erwartungswert und Varianz}
\input 4.tex
\chapter{Wahrscheinlichkeitsverteilung}
\rhead{Wahrscheinlichkeitsverteilung}
\input 5.tex
\chapter{Ein Katalog von Wahrscheinlichkeitsverteilungen}
\rhead{Ein Katalog von Wahrscheinlichkeitsverteilungen}
\input 6.tex
\chapter{Sch"atzen}
\rhead{Sch"atzen}
\input 7.tex
\chapter{Testen}
\rhead{Testen}
\input 8.tex
\chapter{Filtern}
\rhead{Filtern}
\input 9.tex
\chapter{Einf"uhrung in die Software R}
\rhead{Einf"uhrung in R}
Die Aufgaben in diesem Kapitel sind als Lernaufgabe gedacht, mit deren
Hilfe man sich mit der Bedienung der Software R vertraut machen kann.
R ist freie Software, Sie k"onnen Sie von \url{www.r-project.org}
herunterladen und kostenfrei benutzen.

Die Versionen von R f"ur Windows und Mac OS X haben ein GUI und eine darin
integrierte Hilfefunktion, welche die ersten Schritte vereinfacht. Alle
Versionen haben eine Kommandozeilen-Version, welche gen"ugt, die nachstehend
beschriebenen Probleme zu l"osen.

Nat"urlich kann diese "Ubungsserie nur einen ersten Einblick in die
Funktionen von R bieten. Tausende von Funktionen sind in dem "uber 3000
Seiten starken Benutzerhandbuch dokumentiert, ausserdem kann R durch
zugeladene ``packages'' erweitert werden. Es existiert ein grosses Netzwerk
von Zusatz-Packages, die aus dem CRAN, dem Comprehensive R Archive
Network (nach dem Muster des CPAN f"ur Perl) heruntergeladen werden
k"onnen.

\bigskip
\input r.tex
\input aufgabensammlung.ind
\end{document}
