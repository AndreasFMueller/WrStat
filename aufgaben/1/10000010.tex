Aus einem Artikel von der Website der Zeitung 20min vom 11.~September~2013:
\begin{quotation}
Der erste Kandidat glaubte, mit einem Holzstück ``St"orfelder'' im Raum
orten und abwehren zu k"onnen.
Bei 50 Versuchen war er aber bloss 30 Mal erfolgreich. Um statistisch
ein relevantes Ergebnis zu liefern, h"atte er mindestens 40 Treffer gebraucht.
\end{quotation}
Auf wieviele Arten kann man diesen Test bestehen, wieviele Versuchsausg"ange
gabe es "uberhaupt.

\begin{loesung}
Wer gehen davon aus, dass der Kandidat in jedem Versuch entscheiden musste,
ob St"orfelder anwesend waren oder nicht. Es gibt also in jedem Versuch
zwei m"ogliche Ausg"ange. In einer Versuchsreihe von 50 Versuchen gibt
es also $2^{50}$ m"ogliche Ausg"ange.

$k$ Mal richtig tippen kann man auf
$\binom{50}{k}$ Arten. Um in dem Test erfolgreich zu sein, muss man also
\begin{align*}
\sum_{k=40}^{50}\binom{50}{k}
&=
\binom{50}{40}
+
\binom{50}{41}
+
\binom{50}{42}
+
\binom{50}{43}
+
\binom{50}{44}
+
\binom{50}{45}
+
\binom{50}{46}
+
\binom{50}{47}
+
\binom{50}{48}
+
\binom{50}{49}
+
\binom{50}{50}
\\
&=
10272278170
+
2505433700
+
536878650
+
99884400
+
15890700
\\
&\qquad+
2118760
+
230300
+
19600
+
1225
+
50
+
1
%\\
%&=
%10272278170
%+
%2505433700
%+
%536878650
%+
%118145036
\\
&=13432735556.
\end{align*}
Der Anteil der F"alle, in denen man ohne besondere F"ahigkeiten 
erfolgreich sein kann, ist also $13432735556/2^{50}=0.0000119$,
oder einer von 100000 F"allen. Wer den Test besteht, kann also ziemlich
sicher etwas, wenngleich nicht ganz klar ist, was eigentlich.
\end{loesung}
