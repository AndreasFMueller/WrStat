Aus einem Artikel von der Website der Zeitung 20min vom 11.~September~2013:
\begin{quotation}
Der erste Kandidat glaubte, mit einem Holzstück ``Störfelder'' im Raum
orten und abwehren zu können.
Bei 50 Versuchen war er aber bloss 30 Mal erfolgreich. Um statistisch
ein relevantes Ergebnis zu liefern, hätte er mindestens 40 Treffer gebraucht.
\end{quotation}
Auf wieviele Arten kann man diesen Test bestehen, wieviele Versuchsausgänge
gabe es überhaupt.

\thema{Kombinatorik}

\begin{loesung}
Wer gehen davon aus, dass der Kandidat in jedem Versuch entscheiden musste,
ob Störfelder anwesend waren oder nicht. Es gibt also in jedem Versuch
zwei mögliche Ausgänge. In einer Versuchsreihe von 50 Versuchen gibt
es also $2^{50}$ mögliche Ausgänge.

$k$ Mal richtig tippen kann man auf
$\binom{50}{k}$ Arten. Um in dem Test erfolgreich zu sein, muss man also
\begin{align*}
\sum_{k=40}^{50}\binom{50}{k}
&=
\binom{50}{40}
+
\binom{50}{41}
+
\binom{50}{42}
+
\binom{50}{43}
+
\binom{50}{44}
+
\binom{50}{45}
+
\binom{50}{46}
+
\binom{50}{47}
+
\binom{50}{48}
+
\binom{50}{49}
+
\binom{50}{50}
\\
&=
10272278170
+
2505433700
+
536878650
+
99884400
+
15890700
\\
&\qquad+
2118760
+
230300
+
19600
+
1225
+
50
+
1
%\\
%&=
%10272278170
%+
%2505433700
%+
%536878650
%+
%118145036
\\
&=13432735556.
\end{align*}
Der Anteil der Fälle, in denen man ohne besondere Fähigkeiten 
erfolgreich sein kann, ist also $13432735556/2^{50}=0.0000119$,
oder einer von 100000 Fällen. Wer den Test besteht, kann also ziemlich
sicher etwas, wenngleich nicht ganz klar ist, was eigentlich.
\end{loesung}
