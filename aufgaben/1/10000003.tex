Für ein Klassenfoto wird die Klasse in grosse, mittlere und kleine
Schüler eingeteilt. Es gibt 6 grosse, 9 mittlere und 7 kleine
Schüler. Die grossen Schüler sowie der Lehrer stehen in der
hintersten Reihe, der Lehrer ganz links oder rechts aussen. Die mittleren
Schüler sitzen in der mittleren Reihe, und die kleinen in der
vordersten Reihe am Boden. Wieviele verschiedene Aufstellungen für das
Klassenfoto sind möglich?

\thema{Kombinatorik}

\begin{loesung}
Der Platz des Lehrers ist bis auf die Seite (links oder rechts)
festgelegt, zu jeder Aufstellung der Klasse ohne den Lehrer, gibt
es genau zwei Möglichkeiten, wo sich der Lehrer hinstellen kann.

Für die kleinen Schüler gibt es $7!$ Möglichkeiten, sich in
der ersten Reihe zu platzieren. Zu jeder dieser Möglichkeiten
gibt es für die mittleren Schüler $9!$ Möglichkeiten, sich in
der zweiten Reihe zu platzieren. Insgesamt sind dies also $7! \cdot 9!$
Möglichkeiten. Zu jeder solchen Möglichkeit gibt es jetzt noch $6!$
Möglichkeiten, wie sich die grossen Schüler in der letzten Reihe
aufstellen können. Somit können sich die Schüler auf $6!\cdot 7!\cdot 9!$
Arten aufstellen. Da es zu jeder Schüleranordnung zwei mögliche
Lehrerpositionen gibt, ist die Gesamtzahl der Aufstellungen
\[
2\cdot 6!\cdot 7!\cdot 9!=
2\cdot 720 \cdot 5040 \cdot 362880=2633637888000.
\qedhere
\]
\end{loesung}

