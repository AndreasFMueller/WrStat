In der Zeitung liest man, dass an einem Hochzeitsfest 10
von 100 Gästen eine Lebensmittelvergiftung erlitten haben.
\begin{teilaufgaben}
\item
Auf wieviele Arten (Szenarien) ist dies möglich?
\item
In wievielen Szenarien
davon ist der Bräutigam ebenfalls krank geworden?
\item In wievielen Szenarien sind beide Brautleute krank
geworden?
\item In wievielen Szenarien ist Braut oder Bräutigam
krank geworden?
\end{teilaufgaben}

\begin{loesung}
\begin{teilaufgaben}
\item Die Lebensmittelvergiftung muss sich 10 Opfer aus 100 Gästen
auswählen, dies ist auf $ \binom{100}{10}\simeq 1.731\cdot10^{13}$
Arten möglich.
\item
In einem Szenario, in welchem der Bräutigam krank wird, hat die
Lebensmittelvergiftung den Bräutigam bereits als Opfer ausgewählt,
und muss nun unter den verbleibenden 99 Gästen nur noch 9 Opfer
wählen, was auf $\binom{99}{9}\simeq1.7310\cdot10^{12}$ Arten
möglich ist.
\item
Wenn beide Brautleute krank werden, dann hat danach die Lebensmittelvergiftung
nur noch 8 Opfer aus 98 verbleibenden Gästen auszuwählen, was auf
$\binom{98}{8}$ Arten möglich ist.
\item
Es gibt nach b) $\binom{99}{9}$ Szenarien, in denen der Bräutigam
krank wird, und ebensoviele, in denen die Braut krank wird. Davon
sind aber nach c) $\binom{98}{8}$ Szenarien, wo beide krank werden.
Insgesamt ist also die Zahl der Szenarien, mindestens eine der Brautleute
krank wird
\[
2\binom{99}{9}-\binom{98}{8}
\qedhere
\]
\end{teilaufgaben}
\end{loesung}

