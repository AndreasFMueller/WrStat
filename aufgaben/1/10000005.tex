In der Zeitung liest man, dass an einem Hochzeitsfest 10
von 100 G"asten eine Lebensmittelvergiftung erlitten haben.
\begin{teilaufgaben}
\item
Auf wieviele Arten (Szenarien) ist dies m"oglich?
\item
In wievielen Szenarien
davon ist der Br"autigam ebenfalls krank geworden?
\item In wievielen Szenarien sind beide Brautleute krank
geworden?
\item In wievielen Szenarien ist Braut oder Br"autigam
krank geworden?
\end{teilaufgaben}

\begin{loesung}
\begin{teilaufgaben}
\item Die Lebensmittelvergiftung muss sich 10 Opfer aus 100 G"asten
ausw"ahlen, dies ist auf $ \binom{100}{10}\simeq 1.731\cdot10^{13}$
Arten m"oglich.
\item
In einem Szenario, in welchem der Br"autigam krank wird, hat die
Lebensmittelvergiftung den Br"autigam bereits als Opfer ausgew"ahlt,
und muss nun unter den verbleibenden 99 G"asten nur noch 9 Opfer
w"ahlen, was auf $\binom{99}{9}\simeq1.7310\cdot10^{12}$ Arten
m"oglich ist.
\item
Wenn beide Brautleute krank werden, dann hat danach die Lebensmittelvergiftung
nur noch 8 Opfer aus 98 verbleibenden G"asten auszuw"ahlen, was auf
$\binom{98}{8}$ Arten m"oglich ist.
\item
Es gibt nach b) $\binom{99}{9}$ Szenarien, in denen der Br"autigam
krank wird, und ebensoviele, in denen die Braut krank wird. Davon
sind aber nach c) $\binom{98}{8}$ Szenarien, wo beide krank werden.
Insgesamt ist also die Zahl der Szenarien, mindestens eine der Brautleute
krank wird
\[
2\binom{99}{9}-\binom{98}{8}
\]
\end{teilaufgaben}
\end{loesung}

