Gegeben ist der Stapel der Spielkarten vom Ass bis zur Nummer 10,
das Ass zählt dabei als $1$.
\begin{center}
\includeagraphics[width=0.6\hsize]{tencards.png}
\end{center}

\begin{teilaufgaben}
\item 
Auf wieviele Arten kann man zwei Karten aus dem Stapel ziehen?
\item
Auf wieviele Arten kann man zwei Karten aus dem Stapel ziehen derart,
dass die Summe der Zahlen gerade ist?
\item
Auf wieviele Arten kann man zwei Karten aus dem Stapel ziehen derart,
dass die Summe der Zahlen ungerade ist?
\item
Auf wievele Arten kann man einen Stapel von Karten ziehen, der
nur aus geraden Karten besteht, wenn man eine gerade Anzahl Karten
zieht und nur aus ungeraden, wenn man eine ungerade Anzahl Karten zieht?
\end{teilaufgaben}

\begin{loesung}
\begin{teilaufgaben}
\item
Dies ist ein Auswahlproblem, es gibt $\binom{10}{2}=45$ solche Auswahlen.
\item
Wenn die Summe gerade sein soll, dann müssen beide Zahlen gerade oder beide
ungerade sein, sie müssen also die gleiche Parität haben.
Wählt man also irgend eine Karte, dann gibt es noch vier weitere der gleichen
Parität, die man als zweite Karten auswählen kann.
Es gibt also $10\cdot 4$ solche Auswahlen.
Dabei hat man aber auch alle Reihenfolgen doppelt gewählt, man muss also
noch durch 2 teilen.
Es gibt also 20 Arten.
\item
Wenn die Summe ungerade sein soll, dann müssen die beiden Zahlen verschiedene
Partität haben.
Zu den zehn Möglichkeiten der ersten gewählten Karte gibt es 5 weitere
Karten, die entgegengesetzte Parität haben, also $10\cdot 5$ Arten.
Dabei hat man aber wieder die Reihenfolgen doppelt erzeugt, man muss
also noch durch 2 teilen und erhält $25$ Arten.
\item
Man zieht also immer Karten gleicher Parität, und ausserdem der gleichen
Parität wie die Anzahl der Karten.
Sei $k$ die Anzahl der gezogenen Karten, dann muss man $k$ Karten der gleichen
Parität wie $k$ ziehen.
Da es gleich viele gerade Karten  wie ungerade Karten gibt, geht das immer
auf $\binom{5}{k}$ Arten.
Dies muss man für alle möglichen Anzahlen $k$ von 0 bis 5 durchführen, man
erhält
\[
\binom{5}{0}
+
\binom{5}{1}
+
\binom{5}{2}
+
\binom{5}{3}
+
\binom{5}{4}
+
\binom{5}{5}
=
2^5
=
32.
\qedhere
\]
\end{teilaufgaben}
\end{loesung}

\begin{bewertung}
\begin{teilaufgaben}
\item ({\bf A}) 1 Punkt.
\item ({\bf B}) 1 Punkt.
\item ({\bf C}) 1 Punkt.
\item Aufteilung in je drei Fälle für $k$ ({\bf K}) 1 Punkt,
Möglichkeiten für jeden Fall $k$ ({\bf M}) 1 Punkt,
Summe ({\bf S}) 1 Punkt.
\end{teilaufgaben}
\end{bewertung}
