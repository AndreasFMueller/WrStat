Ein Wünschlerutengänger soll mit folgendem Test getestet werden:
Unter zehn Behältern soll er den einen Behälter finden, unter dem
ein Glas Wasser versteckt ist. Dreizehn mal darf er versuchen, wir
glauben ihm aber seine Fähigkeit erst, wenn er davon sieben mal
erfolgreich war. Auf wieviele Arten können sich mindestens sieben
Erfolge ergeben, und wieviele Versuchsausgänge gibt es ingesamt?

\thema{Kombinatorik}

\begin{loesung}
In einem Versuch hat der Wünschelrutengänger genau 10 Wahlmöglichkeiten,
aber nur genau eine Möglichkeit, erfolgreich zu sein.
Insgesamt gibt es also $10^{13}$ Möglichkeiten. 

Genau sieben richtige kann man erreichen, indem man zuerst 7 Positionen
für die erfolgreichen Versuche auswählt, was auf $\binom{13}{7}$
Arten möglich ist. Zu jeder dieser Erfolgsmöglichkeiten kann man jetzt
noch auf $9^6$ Arten die falschen Behälter auswählen. Entsprechend
kann man genau $k$ von $n$ richtige auf 
\[
\binom{n}{k}9^k
\]
Arten erreichen. Um die Anzahl Möglichkeiten für mindestens $k$ richtige
zu bestimmen, muss man jetzt nur noch die Summe
\[
\sum_{k=7}^{n}
\binom{n}{k}9^{n-k}
\]
ausrechnen. Im vorliegenden Fall mit $n=13$ bekommt man
\begin{align*}
&\binom{13}{7}9^6
+
\binom{13}{8}9^5
+
\binom{13}{9}9^4
+
\binom{13}{10}9^3
+
\binom{13}{11}9^2
+
\binom{13}{12}9^1
+
\binom{13}{13}9^0
\\
&=
1716\cdot 531441
+
1287\cdot 59049
+
715\cdot 6561
+
286\cdot 729
+
78\cdot 81
+
13\cdot 9
+
1\cdot 1
\\
&=
911952756
+
75996063
+
4691115
+
208494
+
6318
+
117
+
1
\\
&=
992854864
\end{align*}
Nur in 992854864 von 10000000000000 Fällen, oder in 
$0.0099285\%$ der Fälle kann man also diesen Test bestehen, wenn man
nicht über die behauptete Fähigkeit verfügt.
\end{loesung}
