Ein W"unschleruteng"anger soll mit folgendem Test getestet werden:
Unter zehn Beh"altern soll er den einen Beh"alter finden, unter dem
ein Glas Wasser versteckt ist. Dreizehn mal darf er versuchen, wir
glauben ihm aber seine F"ahigkeit erst, wenn er davon sieben mal
erfolgreich war. Auf wieviele Arten k"onnen sich mindestens sieben
Erfolge ergeben, und wieviele Versuchsausg"ange gibt es ingesamt?

\begin{loesung}
In einem Versuch hat der W"unschelruteng"anger genau 10 Wahlm"oglichkeiten,
aber nur genau eine M"oglichkeit, erfolgreich zu sein.
Insgesamt gibt es also $10^13$ M"oglichkeiten. 

Genau sieben richtige kann man erreichen, indem man zuerst 7 Positionen
f"ur die erfolgreichen Versuche ausw"ahlt, was auf $\binom{13}{7}$
Arten m"oglich ist. Zu jeder dieser Erfolgsm"oglichkeiten kann man jetzt
noch auf $9^6$ Arten die falschen Beh"alter ausw"ahlen. Entsprechend
kann man genau $k$ von $n$ richtige auf 
\[
\binom{n}{k}9^k
\]
Arten erreichen. Um die Anzahl M"oglichkeiten f"ur mindestens $k$ richtige
zu bestimmen, muss man jetzt nur noch die Summe
\[
\sum_{k=7}^{n}
\binom{n}{k}9^{n-k}
\]
ausrechnen. Im vorliegenden Fall mit $n=13$ bekommt man
\begin{align*}
&\binom{13}{7}9^6
+
\binom{13}{8}9^5
+
\binom{13}{9}9^4
+
\binom{13}{10}9^3
+
\binom{13}{11}9^2
+
\binom{13}{12}9^1
+
\binom{13}{13}9^0
\\
&=
1716\cdot 531441
+
1287\cdot 59049
+
715\cdot 6561
+
286\cdot 729
+
78\cdot 81
+
13\cdot 9
+
1\cdot 1
\\
&=
911952756
+
75996063
+
4691115
+
208494
+
6318
+
117
+
1
\\
&=
992854864
\end{align*}
Nur in 992854864 von 10000000000000 F"allen, oder in 
$0.0099285\%$ der F"alle kann man also diesen Test bestehen, wenn man
nicht "uber die behauptete F"ahigkeit verf"ugt.
\end{loesung}
