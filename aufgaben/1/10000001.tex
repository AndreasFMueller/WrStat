Drei Mathematiker und drei Ingenieure treffen sich zu einem
geselligen Nachtessen an einem runden Tisch.
Es wird angenommen, dass die Mathematiker, wenn sie nebeneinander
sitzen, sofort mit einem Fachgespr"ach beginnen, dem
die Ingenieure nicht mehr folgen k"onnen und wollen. Umgekehrt
sind die Ingenieure auch nicht besser, wenn sie alle nebeneinander
s"assen, w"are es um die Geselligkeit des Abends ebenfalls weitgehend
geschehen.
\begin{teilaufgaben}
\item
Auf wieviele Arten kann dieser Fall eintreten (dass drei Mathematiker
nebeneinander sitzen),
wenn die Teilnehmer zuf"allig ihre Pl"atze einnehmen?
\item
Auf wieviele Arten kann man die Teilnehmer platzieren, dass nie zwei oder
mehr Ingenieure oder Mathematiker nebeneinander sitzen?
\end{teilaufgaben}

\begin{loesung}
Es muss gez"ahlt werden, auf wieviele Arten die Ingenieure und Mathematiker
ihre Pl"atze einnehmen k"onnen, und auf wieviele Arten dies mit den
jeweils genannten Einschr"ankungen m"oglich ist. Offenbar gibt es $6!=720$
M"oglichkeiten, die Leute zu platzieren.
\begin{teilaufgaben}
\item Dieser Fall tritt immer dann ein, wenn drei benachbarte Pl"atze
mit Mathematikern besetzt sind.
Es gibt 6 M"oglichkeiten, drei Pl"atze
nebeneinander f"ur die Mathematiker auszuw"ahlen, bestimmt durch den
in Uhrzeigersinn jeweils ersten Mathematiker-Platz.
Sobald die Mathematiker-Pl"atze festgelegt sind, m"ussen wir 3 Mathematiker
auf 3 Pl"atze verteilen, das geht auf $3!$ Arten. Ebenso m"ussen wir
3 Ingenieure auf 3 Pl"atze verteilen, was nochmals $3!$ M"oglichkeiten
beitr"agt. Insgesamt haben wir also $6\cdot 3!\cdot 3!=6\cdot 6\cdot 6=216$
F"alle, in denen diese unerw"unschte Situation eintritt.
\item In diesem Fall m"ussen die Mathematiker und Ingenieure abwechselnd
Platz nehmen. Es gibt nur zwei Arten, wie man die drei Pl"atze f"ur die
Mathematiker und die Ingenieure ausw"ahlen kann. Man kann entweder die
Pl"atze mit geraden Nummern den Mathematikern geben, oder den Ingenieuren.
Dann muss man die drei Mathematiker auf die drei Pl"atze platzieren,
was auf $3!$ Arten m"oglich ist, und ausserdem die drei Ingenieure auf
ihre drei Pl"atze, was nochmals $3!$ M"oglichkeiten gibt. Insgesamt bekommen
wir also $2\cdot 3!\cdot 3!=2\cdot 6\cdot 6=72$ M"oglichkeiten.
\end{teilaufgaben}
\end{loesung}
