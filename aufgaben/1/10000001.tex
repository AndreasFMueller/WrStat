Drei Mathematiker und drei Ingenieure treffen sich zu einem
geselligen Nachtessen an einem runden Tisch.
Es wird angenommen, dass die Mathematiker, wenn sie nebeneinander
sitzen, sofort mit einem Fachgespräch beginnen, dem
die Ingenieure nicht mehr folgen können und wollen. Umgekehrt
sind die Ingenieure auch nicht besser, wenn sie alle nebeneinander
sässen, wäre es um die Geselligkeit des Abends ebenfalls weitgehend
geschehen.
\begin{teilaufgaben}
\item
Auf wieviele Arten kann dieser Fall eintreten (dass drei Mathematiker
nebeneinander sitzen),
wenn die Teilnehmer zufällig ihre Plätze einnehmen?
\item
Auf wieviele Arten kann man die Teilnehmer platzieren, dass nie zwei oder
mehr Ingenieure oder Mathematiker nebeneinander sitzen?
\end{teilaufgaben}

\begin{loesung}
Es muss gezählt werden, auf wieviele Arten die Ingenieure und Mathematiker
ihre Plätze einnehmen können, und auf wieviele Arten dies mit den
jeweils genannten Einschränkungen möglich ist. Offenbar gibt es $6!=720$
Möglichkeiten, die Leute zu platzieren.
\begin{teilaufgaben}
\item Dieser Fall tritt immer dann ein, wenn drei benachbarte Plätze
mit Mathematikern besetzt sind.
Es gibt 6 Möglichkeiten, drei Plätze
nebeneinander für die Mathematiker auszuwählen, bestimmt durch den
in Uhrzeigersinn jeweils ersten Mathematiker-Platz.
Sobald die Mathematiker-Plätze festgelegt sind, müssen wir 3 Mathematiker
auf 3 Plätze verteilen, das geht auf $3!$ Arten. Ebenso müssen wir
3 Ingenieure auf 3 Plätze verteilen, was nochmals $3!$ Möglichkeiten
beiträgt. Insgesamt haben wir also $6\cdot 3!\cdot 3!=6\cdot 6\cdot 6=216$
Fälle, in denen diese unerwünschte Situation eintritt.
\item In diesem Fall müssen die Mathematiker und Ingenieure abwechselnd
Platz nehmen. Es gibt nur zwei Arten, wie man die drei Plätze für die
Mathematiker und die Ingenieure auswählen kann. Man kann entweder die
Plätze mit geraden Nummern den Mathematikern geben, oder den Ingenieuren.
Dann muss man die drei Mathematiker auf die drei Plätze platzieren,
was auf $3!$ Arten möglich ist, und ausserdem die drei Ingenieure auf
ihre drei Plätze, was nochmals $3!$ Möglichkeiten gibt. Insgesamt bekommen
wir also $2\cdot 3!\cdot 3!=2\cdot 6\cdot 6=72$ Möglichkeiten.
\qedhere
\end{teilaufgaben}
\end{loesung}
