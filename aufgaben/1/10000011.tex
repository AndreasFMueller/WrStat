Der Kabarettist Ursus Wehrli ist 2011 mit seiner Aktion,
einen Parkplatz in Zürich Oerlikon aufzuräumen, in die Schlagzeilen
gelangt.
Der Prozess des Aufräumens stellt aus einer beliebigen Anordnung der
Autos eine ganz bestimmte her:
\begin{center}
\includeagraphics[width=\hsize]{parkplatz.jpg}
\end{center}
Die Aufräumregel scheint zu sein, dass die Autos nach Fraben in drei Gruppen
aufgeteilt werden: weiss und grau links (A),
blau bis grün in der Mitte (B), rot und gelb rechts (C). Innerhalb der Gruppen
wird dann der Farbkreis zur Hilfe genommen, um die richtige Anordnung
der Autos zu finden.

Wir untersuchen das Parkplatzaufräumproblem für einen kleineren
Parkplatz mit 25 Parkfeldern: 5 für Gruppe A und je 10 für Gruppe B und C.
Auf dem Parkplatz werden 3 Autos der Gruppe A, 8 der Gruppe B und 7 der
Gruppe C abgestellt.
\begin{teilaufgaben}
\item Auf wieviele Arten können die Autos auf dem Parkplatz abgestellt
werden, wenn die Aufräumregeln nicht befolgt werden.
\item Auf wieviele Arten kann man die Autos abstellen, wenn man nur
die Zuordnung der Autos zu den Gruppen einhält, aber innerhalb 
der Gruppen Unordnung zulässt?
\item Selbst wenn die Autos jeder Gruppe in der richtigen farblichen
Reihenfolge aufgestellt werden, gibt es immer noch verschiedene
Möglichkeiten, je nach dem, welche Parkfelder leer bleiben. Auf
wieviele Arten kann man die Autos in richtiger farblicher Reihenfolge,
aber mit möglicherweise verschiedenen leeren Parkfeldern, aufstellen?
\end{teilaufgaben}

\thema{Kombinatorik}

\begin{loesung}
\begin{teilaufgaben}
\item Es müssen Parkfelder für $3 + 8 + 7 = 18$ Autos auf einem
Parkplatz mit $25$ Parkfeldern ausgewählt werden
werden, was auf $\binom{25}{18}=480700$ Arten möglich ist.
Auf den $18$ ausgewählten Parkfeldern sind $18!=6402373705728000$
verschiedene Anordnungen möglich.
Insgesamt ergeben sich also
\[
\binom{25}{18}\cdot 18!=480700 \cdot 6402373705728000=3077621040343449600000
=3.0776\cdot10^{21}
\]
Möglichkeiten.

Man kann zu diesem Resultat auch wie folgt gelangen.
Für das erste Auto stehen $n$ Parkplätze zur Verfügung, für das zweite
$n-1$ usw. bis zum $k$-ten Auto, für welches noch $n-k+1$ Parkplätze
zur Auswahl stehen. Die Autos können daher auf
\[
25\cdot
24\cdot
23\cdot
\dots
\cdot
8
=
3077621040343449600000
=3.0776\cdot10^{21}
\]
Arten abgestellt werden.
\item Man kann $k$ Autos auf $n$ Parkfelder auf $\binom{n}{k}k!$ Arten
abstellen. Für jede der $\binom{n_A}{k_A}k_A!$ Anordnungen in Gruppe A gibt es 
$\binom{n_B}{k_B}k_B!$ Anordnungen in Gruppe B, und analog für Gruppe C.
Die Gesamtzahl der Anordungen ist also
\begin{align*}
\binom{n_A}{k_A}k_A!
\binom{n_B}{k_B}k_B!
\binom{n_C}{k_C}k_B!
&=
\binom{5}{3}3!
\binom{10}{8}8!
\binom{10}{7}7!
\\
&=10\cdot 6\cdot 45\cdot 40320\cdot 120\cdot 5040=65840947200000=6.584\cdot10^{13}.
\end{align*}
\item Innerhalb der Gruppen ist die Reihenfolge eindeutig bestimmt, aber
die Platzierung der leeren Felder ist frei.
Wenn in einer Gruppe mit $k$ Autos und $n$ Parkfeldern deren $n-k$
leer bleiben müssen,
dann kann man diese auf $\binom{n}{n-k}=\binom{n}{k}$ Arten auswählen.
Für jede der $\binom{5}{2}=\binom{5}{3}$ Wahlen der leeren Parkfelder
für Gruppe A gibt es $\binom{10}{2}=\binom{10}{8}$ Wahlen der
leeren Parkfelder für Gruppe B, und analog für Gruppe C.
Die Gesamtzahl der Anordnungen ist also
\[
\binom{5}{2}
\binom{10}{2}
\binom{10}{3}
=
\binom{5}{3}
\binom{10}{8}
\binom{10}{7}
=10\cdot 45\cdot 120=54000.
\qedhere
\]
\end{teilaufgaben}
\end{loesung}

\begin{bewertung}
In jeder Teilaufgabe: Lösungsweg ({\bf L}) 1 Punkt, korrektes numerisches
Resultat ({\bf R}) 1 Punkt. Speziell:
\begin{teilaufgaben}
\item Häufig wurde die Möglichkeit der Vertauschung, also der Faktor
$18!$ vergessen.
\item Wesentlich für den Punkt ({\bf L}) ist die Kombination mit Hilfe
der Produktregel.
\item Auch hier ist wesentlich die Verwendung der Produktregel.
\end{teilaufgaben}
\end{bewertung}
