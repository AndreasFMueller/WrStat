Ein Kartenspiel mit 52 Karten besteht aus 26 roten und 26 schwarzen
Karten.
\begin{teilaufgaben}
\item
Auf wieviele Arten kann man das Kartenspiel mischen?
\item
Auf wieviele Arten kann man das Kartenspiel mischen, so dass die oberste
und die unterste Karte die gleiche Farbe haben?
\item 
Welcher Prozentsatz der möglichen Kartenstapel hat gleichfarbige
oberste und unterste Karten?
\item
Jetzt werden dem Spiel noch drei Joker-Karten hinzugefügt, die sowohl
als rot als auch als schwarz betrachtet werden können.
Auf wieviele Arten kann man das erweiterte Kartenspiel mischen so, 
dass die oberste und unterste Karte die gleiche Farbe haben?
\end{teilaufgaben}

\thema{Kombinatorik}

\begin{loesung}
\begin{teilaufgaben}
\item
Dies ist ein Anordnungsproblem, es gibt $52!$ solche Reihenfolgen.
\item
Wir können alle möglichen Kartenstapel wie folgt erzeugen.
Zunächst wählen wir eine von 52 Karten, dies wird die oberste
Karte.
Dann wählen wir eine Karte der gleichen Farbe, davon gibt es noch 25,
dies wird die unterste Karte.
Die verbleibenden 50 Karten können in jeder beliebigen Reihenfolge
dazwischen gelegt werden, was auf $50!$ Arten möglich ist.
Somit gibt es insgesamt
\[
52\cdot 25\cdot 50!
\]
\item
Dies ist der Quotient
\[
\frac{52\cdot 25\cdot 50!}{52!}
=
\frac{25\cdot 50!}{51!}
=
\frac{25\cdot 50!}{51\cdot 50!}
=
\frac{25}{51}
=
0.490196%07843137254901
\]
\item
Wir verwenden den gleichen Prozess, um die Karten zu mischen, wie oben in 
b).
Allerdings gibt es jetzt zwei Situtationen für die erste Karte: diese
erste Karte könnte eine Farbkarte sein oder eine Jokerkarte.
Wir zählen diese beiden Arten separat.

Es gibt drei möglich Jokerkarten für die erste Karte, da zu passt dann jede
beliebige andere Karte als unterste Karte, von denen gibt es 54.
Die verbleibenden 53 Karten können in jeder beliebigen Reihenfolge dazwischen 
gelegt werden. 
Insgesamt erhalten wir $3\cdot 54 \cdot 53!$ Arten.

Ist die erste Karte eine Farbkarte, was auf 52 Arten möglich ist, dann
gibt es $25+3$ mögliche Farb- bwz.~Jokerkarten, die als gleichfarbig
gelten.
Die verbleibenden 53 Karten können dann wieder dazwischen gelegt werden.
Wir erhalten also 
$52\cdot 28\cdot 53!$.

Die beiden Fälle sind disjunkt, so dass die gesamte Zahl die Summe ist,
also
\[
3\cdot 54 \cdot 53! + 52 \cdot 28 \cdot 53!
=
(3\cdot 54 + 52\cdot 28)\cdot 53!
\qedhere
\]
\end{teilaufgaben}
\end{loesung}

\begin{diskussion}
Man kann nachrechnen, dass die Gesamtzahl der verschiedenen Stapel (55!)
nur noch
\[
\frac{ (3\cdot 54 + 52\cdot 28)\cdot 53!}{55!}
=
\frac{3\cdot 54+52\cdot 28}{55\cdot 54}
=
\frac{809}{1485}
=
0.544781\dots
\]
mal mehr ist also die Zahl der Stapel mit gleicher
oberster und unterster Karte, so dass die Wahrscheinlichkeit eines
Stapels mit gleichfarbiger oberster und unterster Karte leicht
gestiegen ist.
\end{diskussion}

\begin{bewertung}
\begin{teilaufgaben}
\item ({\bf A}) 1 Punkt.
\item ({\bf B}) 1 Punkte.
\item ({\bf C}) 1 Punkt.
\item Fallunterscheidung ({\bf U}) 1 Punkt
Berechnung Fall Farbkarte ({\bf F}) 1 Punkt,
Berechnung Fall Joker ({\bf J} 1 Punkt.
\end{teilaufgaben}
\end{bewertung}
