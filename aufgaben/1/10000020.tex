An einem Workshop nehmen 24 Teilnehmer teil.
\begin{teilaufgaben}
\item
Für Gruppenarbeiten sollen 4er Teams gebildet werden.
Auf wieviele Arten ($n_1$) ist dies möglich?
\item
Für Gruppenarbeiten sollen 4er Teams gebildet werden, aber die
sechs Teamleiter sind schon vorgängig bestimmt worden.
\item
Der Workshopleiter kommt auf die Idee, die Teams wie folgt zu
bestimmen.
Er wirft eine grosse Zahl von farbigen Zetteln mit sieben verschiedenen
Farben in einen Topf, jedem Team entspricht eine Farbe.
Die Teilnehmer müssen jetzt jeweils einen Zettel ziehen, was deren
Team-Zugehörigkeit festlegt.
Auf wieviele Arten ($n_3$) kann man auf diese Weise Teams bilden?
Warum ist diese Zahl viel grösser als $n_1$?
\end{teilaufgaben}

\thema{Kombinatorik}

\begin{loesung}
\begin{teilaufgaben}
\item 
Um ein Vierer-Team zu bilden, müssen $4$ Personen aus $24$ ausgewählt werden.
Dies ist auf $\binom{24}{4}$ Arten möglich.
Von den verbleibenden $20$ Teilnehmern müssen jetzt für das zweite Vierer-Team
wieder $4$ Personen ausgewählt werden.
Insgesamt ist dies auf 
\[
\binom{24}{4}
\binom{20}{4}
\binom{16}{4}
\binom{12}{4}
\binom{8}{4}
\binom{4}{4}
\]
Arten möglich.
Allerdings entstehen auf diese Weise die Teams in jeder denkbaren
Reihenfolge, von denen es $6!$ gibt.
Die gesucht Anzahl $n$ ist daher
\begin{align*}
n_1
&=
\frac{1}{6!}
\binom{24}{4}
\binom{20}{4}
\binom{16}{4}
\binom{12}{4}
\binom{8}{4}
\binom{4}{4}
=
\frac{1}{6!}
\frac{24!}{20!\,4!}
\frac{20!}{16!\,4!}
\frac{16!}{12!\,4!}
\frac{12!}{8!\,4!}
\frac{8!}{4!\,4!}
\frac{4!}{0!\,4!}
\\
&=
\frac1{6!}\cdot \frac{24!}{(4!)^6}
=
4509264634875
.
\end{align*}
\item 
Mit der gleichen Methode wie in Teilaufgaben a), jetzt aber mit 18 Teilnehmern
finden wir
\begin{align*}
n_2
&=
\frac1{6!}
\binom{18}{3}
\binom{15}{3}
\binom{12}{3}
\binom{9}{3}
\binom{6}{3}
\binom{3}{3}
=
\frac{1}{6!}
\frac{18!}{15!\,3!}
\frac{15!}{12!\,3!}
\frac{12!}{9!\,3!}
\frac{9!}{6!\,3!}
\frac{6!}{3!\,3!}
\frac{3!}{0!\,3!}
\\
&=
\frac{1}{6!}
\frac{18!}{(3!)^6}
=
190590400
.
\end{align*}
\item
Dies ist das Perlenketten-Problem, 
also
\[
n_3
=
7^{24}
=
191581231380566414401.
\]
Auf diese Weise entstehen 7 verschiedene Teams statt 6, und es entstehen
auch Teams mit einer anderen Anzahl
Mitgliedern als $4$.
\qedhere
\end{teilaufgaben}
\end{loesung}

\begin{bewertung}
\begin{teilaufgaben}
\item[a) und b)]
Auswahl ({\bf A}) 1 Punkt,
wiederholte Auswahl (Produktregel) ({\bf W}) 1 Punkt,
Permutationen der Teams ({\bf P}) 1 Punkt,
Resultat ($\text{\bf N}$) 1 Punkt,
\item[c)]
Perlenkettenproblem ({\bf V}) 1 Punkt,
Begründung für die grosse Anzahl ({\bf B}) 1 Punkt.
\end{teilaufgaben}
\end{bewertung}




