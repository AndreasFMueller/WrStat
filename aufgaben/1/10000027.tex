Zum Schutz der eigenen Daten wird empfohlen, Passwörter zu wählen,
die nicht so leicht erraten werden können.
Meist wird dazu eine minimale Passwortlänge verlangt, oft werden
aber zusätzliche Forderungen gestellt, wie zum Beispiel Gross-
und Kleinbuchstaben, Ziffern und Sonderzeichen, die im Passwort
vorkommen müssen.
In dieser Aufgabe soll untersucht werden, welche Anforderungen an
ein Passwort die Wahrscheinlichkeit, das Passwort zu erraten, 
tatsächlich senken und damit den Schutz des Benutzers verbessern.
Nehmen Sie an, dass das zur Verfügung stehende Alphabet aus den
folgenden Arten Zeichen besteht:
\begin{center}
\begin{tabular}{|lr|}
\hline
Zeichenklasse                 & Anzahl Zeichen \\
\hline
Buchstaben                    &             52 \\
\qquad davon Kleinbuchstaben  &             26 \\
\qquad oder Grossbuchstaben   &             26 \\
Ziffern                       &             10 \\
Sonderzeichen                 &             39 \\
\hline
Total                         &            101 \\
\hline
\end{tabular}
\end{center}
\begin{teilaufgaben}
\item
Wieviele verschiedene Passwörter mit $n$ Zeichen gibt es?
\item
Wieviele verschiedene Passwörter gibt es, die nur Buchstaben verwenden?
\item
Oft wird versucht, die Robustheit der Passwörter dadurch zu erhöhen, dass
man von den Benutzern verlangt, dass sie in ihren Passwörtern gewisse
Zeichen verwenden müssen.
Vergleichen Sie die Anzahlen verschiedener Passwörter der Länge $n$
unter folgenden Einschränkungen:
\begin{enumerate}[i)]
\item
Passwörter aus $n$ Kleinbuchstaben.
\item
Passwörter der Länge $n$, die genau
eine Ziffer, ein Sonderzeichen und ein Grossbuchstaben enthalten
und sonst aus nur Kleinbuchstaben bestehen.
\item
Um wieviele Zeichen längere Passwörter muss man verlangen, wenn die
Benutzer nur Kleinbuchstaben verwenden, um mehr verschiedene
Passwörter als in (ii) zu bekommen.
\end{enumerate}
\end{teilaufgaben}


\begin{loesung}
\begin{teilaufgaben}
\item Dies ist ein Perlenkettenproblem, es gibt daher $101^n$
Passwörter.
\item Desgleichen: $52^n$
\item
\begin{enumerate}[i)]
\item Es handelt sich um ein Perlenkettenproblem, die Anzahl ist $26^n$.
\item
Zunächst müssen die Plätze für die besonderen Zeichen ausgewählt
werden: es gibt $n$ mögliche Plätze für die Ziffer, $n-1$ Plätze
für das Sonderzeichen und $n-2$ für den Grossbuchstaben.
Mit den Anzahlen für die einzelnen Zeichen in der Tabelle folgt,
dass die Anzahl der Passwörter
\begin{equation}
\underbrace{\mathstrut n\cdot 10}_{\text{Ziffer}}
\cdot
\underbrace{(n-1)\cdot 39}_{\text{Sonderzeichen}}
\cdot
\underbrace{(n-2)\cdot 26}_{\text{Grossbuchstabe}}
\cdot
26^{n-3}
\label{10000027:komplex}
\end{equation}
\item
Statt $n$ Kleinbuchstaben haben wir jetzt $n+k$ Kleinbuchstaben, also
$26^{n+k}$ verschiedene Passwörter.
Es muss jetzt $k$ so gefunden werden, dass
diese Anzahl grösser wird als \eqref{10000027:komplex}:
\begin{align*}
n(n-1)(n-2)\cdot 10 \cdot 39 \cdot 26^{n-2}
&<
26^{n+k}
\\
n(n-1)(n-2)\cdot 10 \cdot 39
&<
26^{k+2}
\intertext{Da wir uns für den Exponenten interessieren,
nehmen wir den Logarithmus zur Basis 26 und bekommen}
\log_{26} \bigl(n(n-1)(n-2)\cdot 10\cdot 39\bigr)
&<
k+2.
\\
k&> \frac{\log\bigl(n(n-1)(n-2)\cdot 10 \cdot 39\bigr)}{\log 26} -2
\intertext{Für $n=6$ ergibt sich}
k&>
\frac{\log (6\cdot 5 \cdot 4\cdot 10 \cdot 39)}{\log 26} -2
=
1.300589
\end{align*}
Wenn Benutzer also die Komplexitätsanforderungen nur minimal erfüllen,
ist es leichter mit etwas längeren Passwörtern mehr Sicherheit zu erhalten
als mit komplizierten Komplexitätsanforderungen.
\qedhere
\end{enumerate}
\end{teilaufgaben}
\end{loesung}

\begin{diskussion}
Man kann in c) auc die Frage stellen, wie lang Passwörter sein
müssen, damit es mindestens um zwei Zeichen längere Passwörter braucht,
um die Wahrscheinlichkeit, dass ein Passwort erraten wird, gleichermassen
zu erhöhen wie durch die in (ii) verlangten Komplexitätsanforderungen.
Dazu muss
\begin{align*}
\log_{26}(n(n-1)(n-2)\cdot10\cdot 39) -2 &> 2 
\intertext{sein}
n(n-1)(n-2)\cdot 10 \cdot 39 &> 26^{4}
\\
n(n-1)(n-2) &> \frac{26^4}{390} \approx 1171.733
\end{align*}
Dies ist ein kubisches Polynom und es ist etwas umständlich, $n$ direkt
zu bestimmen. 
Indem man die dritte Wurzel zieht, kann man ungefähr abschätzen, wie
gross $n$ sein muss, man erhält $n\approx 10.543$.
Man kann jetzt mit verschiedenen $n$ probieren:
\begin{center}
\begin{tabular}{|>{$}c<{$}|>{$}r<{$}|}
\hline
n&n(n-1)(n-2)
\\
\hline
10 & 720 \\
11 & 990 \\
12 & 1320 \\
\hline
\end{tabular}
\end{center}
Erst für Passwörter mit einer Länge von mindestens 12 Zeichen braucht
es mindestens zwei zusätzliche Zeichen.
Man kann dies so interpretieren: wenn man davon ausgehen muss, dass
Benutzer nur die minimalen Anforderungen an die Passwortkomplexität
erfüllen, dann gewinnt man mehr Sicherheit dadurch, dass man die minimale
Passwortlänge um ein einziges Zeichen vergrössert.
\end{diskussion}
