Auf wieviele Arten kann man die Schuhe binden?
Diese Frage stellt sich Youtuber Mathologer in einem Video, das er am
20.~Juni 2020 veröffentlich hat\footnote{\url{https://youtu.be/CSw3Wqoim5M}}.
Bei einer Schuhbindung wird der Bändel abwechselnd durch zwei Reihen von
Ösen gefädelt bis alle Ösen benutzt sind.
Dabei kommt es offenbar nicht darauf an, ob man den Bändel von innen oder
von aussen durch die Öse fädelt.
\begin{center}
\def\eyelet#1{
	\draw[color=white,line width=0.18cm] #1 circle[radius=0.2];
	\draw[color=black,line width=0.12cm] #1 circle[radius=0.2];
}
\definecolor{darkred}{rgb}{0.8,0,0}
\def\segment#1#2{
	%\draw[color=white,line width=0.12cm] #1 -- #2;
	\draw[round cap-round cap,color=darkred,line width=0.1cm] #1 -- #2;
}
\begin{tikzpicture}[>=latex,thick]

\begin{scope}[xshift=-4cm]
	\segment{(-1,-1)}{(1,-1)}
	\eyelet{(1,-1)}
	\eyelet{(-1,-1)}
	\segment{(1,-1)}{(-1,0)}
	\segment{(-1,-1)}{(1,0)}
	\eyelet{(1,0)}
	\eyelet{(-1,0)}
	\segment{(-1,0)}{(1,1)}
	\segment{(1,0)}{(-1,1)}
	\eyelet{(1,1)}
	\eyelet{(-1,1)}
	\segment{(1,1)}{(-1,1)}
\end{scope}

\begin{scope}[xshift=0cm]
	\segment{(-1,-2)}{(1,-2)}
	\eyelet{(-1,-2)}
	\eyelet{(1,-2)}
	\segment{(-1,-2)}{(1,2)}
	\eyelet{(1,2)}
	\segment{(1,-2)}{(-1,-1)}
	\eyelet{(-1,-1)}
	\segment{(-1,-1)}{(1,-1)}
	\eyelet{(1,-1)}
	\segment{(1,-1)}{(-1,0)}
	\eyelet{(-1,0)}
	\segment{(-1,0)}{(1,0)}
	\eyelet{(1,0)}
	\segment{(1,0)}{(-1,1)}
	\eyelet{(-1,1)}
	\segment{(-1,1)}{(1,1)}
	\eyelet{(1,1)}
	\segment{(1,1)}{(-1,2)}
	\eyelet{(-1,2)}
	\segment{(-1,2)}{(1,2)}
\end{scope}

\begin{scope}[xshift=5cm]
\draw[<->] (-1.5,-2) -- (-1.5,2);
\draw[line width=0.2pt] (-1.7,-2)--(-1,-2);
\draw[line width=0.2pt] (-1.7,2)--(-1,2);
\node at (-1.5,0) [above,rotate=90] {$n$ Ösenpaare};

\draw[densely dotted,densely dotted,color=darkred,line width=0.1cm] (0,-1.5) -- (-0.5,-1.75);
\segment{(-1,-2)}{(-0.5,-1.75)}
\eyelet{(-1,-2)}
\segment{(-1,-2)}{(1,-2)}
\eyelet{(1,-2)}
\segment{(1,-2)}{(0.75,-1.5)}
\draw[densely dotted,densely dotted,color=darkred,line width=0.1cm] (0.5,-1) -- (0.75,-1.5);

%\draw[densely dotted,densely dotted,color=darkred,line width=0.1cm] #1 -- #2;

\draw[color=darkred,line width=0.1cm,densely dotted] (0,-0.5) -- (0.5,-0.25);

\segment{(0.5,-0.25)}{(1,0)}
\eyelet{(1,0)}
\segment{(1,0)}{(-1,2)}
\eyelet{(-1,2)}
\segment{(-1,2)}{(1,1)}
\eyelet{(1,1)}
\segment{(1,1)}{(-1,0)}
\eyelet{(-1,0)}
\segment{(-1,0)}{(1,2)}
\eyelet{(1,2)}
\segment{(1,2)}{(-1,1)}
\eyelet{(-1,1)}
\segment{(-1,1)}{(-0.5,0)}
\draw[color=darkred,line width=0.1cm,densely dotted] (-0.5,0) -- (0,-1);

\end{scope}

\end{tikzpicture}
\end{center}

\begin{teilaufgaben}
\item
Auf wieviele Arten kann man einen Schuh mit $2n$ Ösen binden, wenn der
Bändel abwechselnd durch Ösen der beiden Reihen gefädelt werden soll?
\item
Auf wieviele Arten kann man einen Schuh mit $2n$ Ösen binden, wenn
es zusätzlich darauf ankommt, ob der Bändel von innen oder von aussen
durch die Öse gefädelt wird?
\item
Auf wieviele Arten kann man einen Schuh mit $2n$ Ösen binden,
wenn es nicht einmal darauf ankommt, dass der Bändel abwechselnd
durch Ösen beider Reihen gefädelt wird?
\end{teilaufgaben}

\thema{Kombinatorik}

\begin{loesung}
\begin{teilaufgaben}
\item
Da der Bändel durch alle Ösen gefädelt werden muss, können wir, um die
Schuhbindungen zu zählen, in der linken oberen Ecke beginnen.
Der Bändel muss anschliessend durch eine der $n$ Ösen in der rechten
Reihe gefädelt werden.
Von dort muss er wieder zurück gefädelt werden durch eine der
verbleibenden $n-1$ Ösen der linken Reihe.
Auf der rechten Seite bleiben dann $n-1$ Ösen.
Die Anzahl der möglichen Bindungen, die in der linken oberen Ecke beginnen,
ist daher
\[
n\cdot (n-1) \cdot (n-1) \cdot \dots \cdot 3\cdot 2 \cdot 2 \cdot 1 \cdot 1
=
n! \cdot (n-1)!
\]
In diesem Prozess haben wir aber alle Bindungen doppelt gezählt.
Da der Bändel auch wieder zur ersten Öse zurück führt, können wir die
Bindung auch ``rückwärts'' durchlaufen und erhalten so nochmals dieselbe
Bindung, sie wurde aber separat gezählt.
Die tatsächliche Anzahl der Bindungen ist daher
\[
\frac{n!(n-1)!}2.
\]
\item 
Die Anzahl der Fädelungen ist immer noch dieselbe, aber wir müssen
jetzt bei jeder Öse zulassen, dass der Bändel von oben nach unten
oder von unten nach oben durch die Öse gefädelt wird.
Die Auswahl der Fädelungsrichtung ist auf $2^{2n}$ Arten möglich.
Die Anzahl der Bindungen ist damit
\[
2^{2n-1}\, n!\, (n-1)!
\]
\item
Wenn es nicht darauf ankommt, ob dass der Bändel abwechselnd durch
Ösen dir beiden Reihen gefädelt wird, dann wird die Zählung etwas
einfacher.
Nach der ersten Öse in der linken oberen Ecke gibt es $2n-1$ Möglichkeiten
für die zweite Öse, $2n-2$ Möglichkeiten für die dritte Öse usw.
Da auch in diesem Fall jede Bindung doppelt gezählt wurde, ist
die Anzahl der Bindungen
\[
\frac{(2n-1)!}{2}
\] 
bei Vernachlässigung der Fädelungsrichtung (wie in a)\,) und
\[
2^{2n-1} \cdot (2n-1)!
\]
wenn die Fädelungsrichtung durch die Öse wie in (wie in c)\,)
berücksichtigt werden soll.
\qedhere
\end{teilaufgaben}
\end{loesung}

\begin{bewertung}
Für jede Teilaufgabe 1 Punkt für einen erfolgversprechenden Lösungsweg
und 1 Punkt für das korrekte Resultat.
\end{bewertung}
