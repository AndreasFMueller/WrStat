\begin{teilaufgaben}
\item
Wieviele verschiedene Wörter kann man aus den Buchstaben des
Wortes {\texttt{MISSISSIPI}} bilden?
\item
In einem Setzkasten sind alle Buchstaben des Alphabetes in grosser 
Zahl vorhanden.
Wieviele Wörter der Länge $10$ kann man bilden, die nur die Buchstaben
\texttt{M}, \texttt{S}, \texttt{I} und \texttt{P} verwenden?
\item 
Jetzt wird mit den Wörtern der Teilaufgabe a) ein Ratespiel gespielt.
Der Spieler muss die Plätze der Buchstaben \texttt{S} vorhersagen.
Dann wird das Wort aufgedeckt und kontrolliert, ob die Buchstabenplätze
richtig vorhergesagt worden sind.
Wieviele verschiedene Wörter kann man aus den Buchstaben des Wortes
{\texttt{MISSISSIPI}} bilden, wenn die Plätze der Buchstaben \texttt{S}
zum vornherein festgelegt sind?
\item
Wie gross ist die Wahrscheinlichkeit, im Spiel von Teilaufgabe c)
vier richtige Plätze vorherzusagen?
\end{teilaufgaben}

\begin{loesung}
\begin{teilaufgaben}
\item
Es stehen 10 Buchstaben zur Verfügung, die in $10!$ Anordnungen
zu einem Wort zusammengefügt werden können.
Dabei werden in jedem solchen Wort auch die Anordnungen
der vier \texttt{S} und der vier \texttt{I} mitgezählt, die
Anzahl der Wörter wird also um den Faktor $4!\cdot 4!$ überschätzt.
Die gesuchte Anzahl ist daher
\[
\frac{10!}{4!\,4!} = \frac{3628800}{24\cdot 24}= 6300.
\]
\item
Dies ist ein Perlenkettenproblem für eine Perlenkette der Länge 10
aus Perlen in 4 Farben.
Die Anzahl der Wörter ist daher $4^{10} = 2^{20} = 1048576$.
\item
In diesem Fall müssen nur noch die 6 verbleibenden Buchstaben auf
die 6 verbleibenden Plätze verteilt werden, dies ist auf $6!$ Arten
möglich.
Dabei wurden aber die $4!$ verschiedenen Permutationen von \texttt{I}
mehrfach gezählt, die tatsächliche Anzahl Wörter ist also
$6!/4!=6\cdot 5=30$.
\item
Da $30$ der $6300$ möglichen Wörter die Buchstaben \texttt{S} an der
richtigen Stelle haben, ist die gesuchte Wahrscheinlichkeit 
$30/6300 = 1/210 \approx 0.00476$.
\qedhere
\end{teilaufgaben}
\end{loesung}

\begin{bewertung}
\begin{teilaufgaben}
\item Permutationen ({\bf P}) 1 Punkt, wiederholte Buchstaben ({\bf W}) 1 Punkt,
\item Variation ({\bf V}) 1 Punkt,
\item Auswahl der Plätze der {\tt I} ({\bf I}) 1 Punkt, Permutation der
übrigen Buchstaben ({\bf U}) 1 Punkt (alternativ: Vorgehen wie in a),
total 2 Punkte),
\item Wahrscheinlichkeit ({\bf P}) 1 Punkt (alternativ, Verwendung
der hypergeometrischen Verteilung ({\bf H}) 1 Punkt.
\end{teilaufgaben}
\end{bewertung}



