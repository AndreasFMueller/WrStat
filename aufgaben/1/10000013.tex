In einer etwas limitierten Programmiersprache können Variablennamen
mit maximal 5 Zeichen gebildet werden, wobei das erste Zeichen nur
einer der 26 Kleinbuchstaben sein darf, die folgenden hingegen Kleinbuchstaben
oder Ziffern (Zeichenwiederholung ist erlaubt).
\begin{teilaufgaben}
\item
Wie viele Variablennamen mit genau 5 Zeichen können gebildet werden?
\item
Wie viele Möglichkeiten bestehen insgesamt, Variablennamen zu bilden?
\item
Wie ändert sich die Variablenanzahl in Frage a),
wenn Zeichenwiederholungen nicht erlaubt sind?
\end{teilaufgaben}

\begin{loesung}
\begin{teilaufgaben}
\item Die hinteren vier Zeichen können aus insgesamt 36 Zeichen ausgewählt
werden, dies ist eine wiederholte Auswahl, daher ist die gesuchte Anzahl
\[
26\cdot 36^4=43670016.
\]
\item 
Wir müssen das Resultat von a) für alle Variablenlängen verallgemeinern
und addieren:
\[
26
+
26\cdot 36
+
26\cdot 36^2
+
26\cdot 36^3
+
26\cdot 36^4
=
26 \cdot (1 + 36 + 36^2 + 36^3 + 36^4)=26\frac{1-36^5}{1-36}
=
44917730
\]
\item
In diesem Fall handelt es sich nicht mehr um eine wiederholte Auswahl,
da in jedem Schritt ein Zeichen weniger zur Auswahl steht. Es bleibt
aber
\[
26 \cdot 35 \cdot 34 \cdot 33 \cdot 32=32672640.
\qedhere
\]
\end{teilaufgaben}
\end{loesung}

\begin{bewertung}
Jede Teilaufgabe zwei Punkte.
\end{bewertung}

\begin{diskussion}
In den ersten Unix-Versionen war die Länge von Identifiern im Assembler
auf 6 Zeichen limitiert. Der C-Compiler hat zudem allen von ihm 
erzeugten Symbolen einen Unterstrich vorangestellt, so dass nur fünf
Zeichen tatsächlich benutzt wurden. Im Unterschied zur Aufgabe
waren die Namen in Unix aber case sensitive, Grossbuchstaben waren erlaubt,
und wurden unterschieden.
\end{diskussion}

