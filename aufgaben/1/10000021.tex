Ein Kind hat 2 gerade und 8 gekrümmte Schienenstücke ($45^\circ$) für
ein Holzspielbahn.
Die gekrümmten Stücke können sowohl als Rechts- als auch als
Links-Kurven eingesetzt werden (siehe Abbildung~\ref{10000021:bild}).
\begin{figure}[h]
\centering
\includeagraphics[width=0.65\hsize]{brio.jpg}
\quad
\includeagraphics[width=0.25\hsize]{BRI-33710.jpg}
\caption{Schienenstücke für eine Holzspielbahn
\label{10000021:bild}}
\end{figure}

\begin{teilaufgaben}
\item
Wie viele verschiedene Bahnanlagen lassen sich aus diesen Schienenstücken
bauen, wenn die Bahn nicht geschlossen sein muss und nicht darauf geachtet
werden muss, ob sich die Schienen kreuzen?
\item
Was ändert sich, wenn eines der gekrümmten Schienenstücke durch ein
gekrümmtes Stück aus Kunststoff ersetzt wird, welches nur als Linkskurve
eingesetzt werden kann, dafür aber einen Torbogen über der Bahn aufbaut
(Abbildung~\ref{10000021:bild} rechts)?
\item
Jetzt sollen nur noch geschlossene Bahnanlagen gebaut werden.
Wieviele verschiedene solche Anlagen können aus 2 geraden und 8
gekrümmten Schienenstücken gebauten werden?
Wieviele verschiedene solche Anlagen sind möglich, wenn statt der
8 nur 7 gekrümmte Schienen und der Torbogen zur Verfügung stehen?
\end{teilaufgaben}

\begin{loesung}
\begin{teilaufgaben}
\item
Es sind Folgen von 10 Schienenstücken zu bilden.
Zwei davon müssen gerade sein, es müssen also zwei von 10 Plätzen für die
geraden Schienen ausgewählt werden.
Dies ist auf $\binom{10}{2}=45$ Arten möglich.
Die gekrümmten Schienen können jeweils in zwei Richtungen eingesetzt werden,
für die 8 gekrümmten Schienen bedeutet das $2^8$ verschiedene Möglichkeiten.
Die Gesamtzahl der möglichen Bahnanlagen ist daher
\[
\binom{10}{2}\cdot 2^8 = 45\cdot 256=11520.
\]
\item
Wie in a) müssen erst die zwei Plätze für die geraden Schienenstücke 
ausgewählt werden.
Einer der verbleibenden acht Plätze wird jetzt für den Torbogen verwendet,
dies ist auf 8 Arten möglich.
Die 7 Holzschienen können wieder als Rechts- oder als Links-Kurven eingesetzt
werden, was auf $2^7$ Arten möglich ist.
Es gibt also insgesamt
\[
\binom{10}{2}\cdot 8 \cdot 2^7
=
\binom{10}{2}\cdot 2^{10}
=
\binom{10}{2}2^{8}\cdot 4
=
46080.
\]
\item
Eine geschlossen Bahn lässt sich nur bauen, wenn aus jeweils vier 
gekrümmten Schienenstücken $180^\circ$-Kurven gebaut werden, die
mit den zwei geraden Teilstücken verbunden werden.
Es entsteht ein Oval mit zwei $180^\circ$-Kurven, die mit geraden
Schienenstücken verbunden sind.

Ein solches Oval ist auf zwei Arten möglich, indem die Kurvenstücke
alle als Rechts- oder alle als Links-Kurven eingesetzt werden.
Beim Einsatz des Torbogens müssen alle Kurven als Linkskurven eingesetzt
werden.
Jedes der 8 Kurvenstücke kann durch den Torbogen ersetzt werden, aber
die beiden $180^\circ$-Kurven gehen durch Drehung der ganzen Anlage 
auseinander hervor, es sind also nur 4 tatsächlich verschiedene Bahnanlagen
möglich.

Insgesamt erhalten wir 6 Möglichkeiten, geschlossene Bahnanlagen zu bauen.
\qedhere
\end{teilaufgaben}
\end{loesung}

\begin{bewertung}
Jede Teilaufgabe 2 Punkte.
\end{bewertung}




