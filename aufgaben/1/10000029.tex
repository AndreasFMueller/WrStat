Ein Skigebiet bietet den Skifahrern 15 verschiedene Liftanlagen an, die
man mit einer Tageskarte alle benutzen kann.
Ein Tag dauert aber nicht beliebig lang, man kann also nur eine beschränkte
Anzahl Liftfahrten tatsächlich fahren.
Wir nehmen an, dass 13 Fahren möglich sind.
\begin{teilaufgaben}
\item
Auf wieviele verschiedene Arten kann man die Abfolge von Liftfahrten
in einem Skitag wählen, wenn es möglich ist von jeder Bergstation mit einer
Abfahrt jede Talstation zu erreichen?
Es ist erlaubt, mehrmals mit dem gleichen Lift zu fahren.
\item
Das Skigebiet hat 4 Sessellifte und 11 Skilifte.
Auf wieviele Arten kann man, unter denselben Voraussetzungen wie
in Teilaufgabe a), die Fahrten planen, wenn man genau 6 der
Fahrten mit einem Sessellift durchführen will?
\item
Die Annahme in Teilaufgabe a), dass man von jeder Bergstation in einer
Abfahrt zu jeder Talstation gelangen kann, ist nicht ganz richtig.
Vielmehr ist das Skigebiet zusammengewachsen aus zwei ursprünglich
getrennten Gebieten, einem mit 6 Liften und einem mit 8 Liften.
In jedem dieser Gebiete ist es möglich, jede Talstation mit einer
Abfahrt von jeder Bergstation aus zu erreichen.
Dazwischen wurde ein neuer Lift gebaut, dessen Talstation von allen
Liften erreichbar ist.
Von der Bergstation des neuen Lifts sind alle Talstationen erreichbar,
von den Bergstationen der Teilgebiet nur die eigenen Talstationen und 
die des Verbindungsliftes.
Leider muss man am Verbindungslift ziemlich lange anstehen, was man nur
einmal an einem Tag machen will, damit immer noch 12 Fahrten über den
ganzen Tag möglich bleiben (inkl.~Verbindungslift).
Auf wieviele Arten kann man Fahrten eines Tages planen, wenn man im ersten
Teilgebiet $5$ Fahrten (ohne die Fahrt auf dem Verbindungslift)
und 6 im zweiten fahren will (ohne Verbindungslift).
\ifthenelse{\boolean{pruefung}}{}{
\item 
Stellen Sie eine Formel auf für die Anzahl der Möglichkeiten eines
Szenarios wie in Teilaufgabe b) mit $n$ Fahrten im ersten und dem
Rest der Fahrten im zweiten Teilgebiet.
\item 
Nehmen Sie jetzt an, dass man mindestens eine Fahrt im ersten Teilgebiet
durchführt, dann wechselt und die restlichen Fahrten im zweiten Teilgebiet
durchführt.
Stellen Sie eine Formel auf, mit der man die Anzahl der Arten
berechnen kann, auf die dies möglich ist.
}
\end{teilaufgaben}

\ifthenelse{\boolean{pruefung}}{}{
\begin{hinweis}
Die Teilaufgaben d) und e) sind in der Prüfung nicht gestellt worden.
\end{hinweis}
}

\begin{loesung}
\begin{teilaufgaben}
\item
Für jede der 12 Fahrten kann kann ein beliebiger der 15 Lifte Lift
gewählt werden, dies ist auf
\[
15^{13} = 1946195068359375
\]
Arten möglich
(Perlenkettenproblem).
\item
6 Fahrten mit 4 Sesselliften sind auf $4^6$ Arten möglich,
7 Fahrten mit 11 Skiliften auf $11^7$ Arten (Perlenkettenproblem).
Zusätzlich ist jetzt auszuwählen, welche der 13 Fahrten
des Tages Sesselliftfahrten sein sollen.
Diese Auswahl ist auf $\binom{13}{6}$ Arten möglich (Auswahlproblem).
Insgesamt gibt es also
\[
\binom{13}{6}\cdot 4^6\cdot 11^7
=
1716\cdot 4096 \cdot 19487171
=
136970180345856
\]
\item
$5$ Fahrten mit den $6$ Liften des ersten Teilgebietes kann man auf
$6^5$ Arten planen, die $6$ Fahrten auf den $8$ Liften im zweiten auf
$8^6$ Arten.
Es gibt aber noch die Wahlmöglichkeit, in welchem Teilgebiet man 
beginnen will, was noch einen zusätzlichen Faktor $2$ einführt.
Die Gesamtzahl der Möglichkeiten ist daher
\[
2\cdot 6^5\cdot 8^6
=
2\cdot 7776\cdot 262144
=
4076863488
\]
\item
$n$ Fahrten kann man auf $6^n$ Arten planen, die verbleibenden Fahrten
auf $8^{12-n}$ Arten.
Die Gesamtzahl der Möglichkeiten ist
\[
6^n\cdot 8^{12-n}.
\]
\item
Man muss die Summe über alle möglichen $n$ bilden, also
\[
\sum_{n=1}^{11} 6^n \cdot 8^{12-n}.
\qedhere
\]
\end{teilaufgaben}
\end{loesung}

\begin{bewertung}
\begin{teilaufgaben}
\item Perlenkettenproblem ({\bf P}) 1 Punkt,
\item Auswahlproblem ({\bf A}) 1 Punkt,
Perlenkettenprobleme für Teilgebiet ({\bf T}) 1 Punkt,
Multiplikation ({\bf M}) 1 Punkt,
\item
Startgebiet ({\bf S}) 1 Punkt,
Resultat ({\bf R}) 1 Punkt.
\end{teilaufgaben} 
\end{bewertung}
