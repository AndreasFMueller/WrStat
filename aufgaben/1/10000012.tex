Ein Selecta-Automat enth"alt sechs Reihen von F"achern, in denen in einer
Transportspirale Artikel eingef"ullt sind. Der Kunde bezahlt, w"ahlt ein
Fach, und die Transportspirale bef"ordert ein St"uck in ein Fach, aus dem
der Kunde den Artikel entnehmen kann.
\begin{center}
\includeagraphics[width=0.3\hsize]{selecta.jpg}
\end{center}
Ein Mathematik-Student ergattert einen Ferien-Job als
Selecta-Automaten-Nachf"uller. Er findet es aber langweilig, die Artikel
immer wieder in die gleichen F"acher nachzuf"ullen. Er stellt fest,
dass es drei Gr"ossen von F"achern gibt: breite (9 in den obersten zwei Reihen),
normale (40, in den unteren vier Reihen), kleine (4, oben in der Mitte).
Das bringt ihn auf neue Ideen. Man k"onnte doch beim Auff"ullen auch die 
Anordnung der Produkte im Automaten etwas umstellen, so dass zum Beispiel
die Paprika-Chips nach dem Auff"ullen nicht mehr oben links, sondern im
Fach ganz rechts in der zweiten Reihe zu finden sind. Nat"urlich soll
pro Fach immer noch nur ein Produkt eingef"ullt sein, und das Fach muss
in der Gr"osse passend sein.
\begin{teilaufgaben}
\item
Auf wieviele Arten kann man die Produkte im Automaten anordnen?
\item
Die normalen F"acher erm"oglichen sehr viele verschiedenen Anordnungen.
Weil das zu lange dauert, entschliesst sich der Student, nur noch die
grossen und kleinen F"acher umzusortieren. Wieviele M"oglichkeiten
gibt es jetzt noch?
\item
Wieviele M"oglichkeiten gibt es, die normal grossen Produkte
umzuordnen, wenn es auf die Reihenfolge innerhalb einer Reihe des
Automaten nicht ankommt?
\item 
Ein Kunde hat Kleingeld, welches ihm gestattet, genau drei Produkte
aus den Reihen 3 oder 4 zu kaufen. Auf wieviele Arten kann er
das tun, wenn er sich zus"atzlich die Bedinung auferlegt, dass alle
drei Produkte aus der gleichen Reihe stammen?
\item
Auf wieviele Arten kann man an diesem Automaten drei Produkte einkaufen, wenn
jedes Produkt aus einer anderen Reihe stammen muss, aber nicht aus
der obersten Reihe kommen darf?
\end{teilaufgaben}

\begin{loesung}
\begin{teilaufgaben}
\item Es gibt 9 grosse F"acher, 40 normale und 4 kleine. In jeder Gruppe
k"onnen die $k$ Produkte auf $k!$ Arten angeordnet, werden. Insgesamt 
bekommt man
\begin{align*}
9!\cdot 40!\cdot 4!&=
362880\cdot 815915283247897734345611269596115894272000000000 \cdot 24
\\
&=7105904111639931116144050020264924857122160640000000000.
\end{align*}
\item $9!\cdot 4!=8709120$.
\item
F"ur jedes normale Fach muss ein Produkt ausgew"ahlt werden, dies ist
auf $10!$ Arten m"oglich. Allerdings spielen die $10!$ Reihenfolgen
in jeder Reihe keine Rolle, also ist die Anzahl der M"oglichkeiten
\[
\frac{40!}{(10!)^4}=4705360871073570227520.
\]
\item
Es gibt $\binom{10}{3}$ M"oglichkeiten, $3$ Produkte aus Reihe 3
auszuw"ahlen, und nochmals $\binom{10}{3}$ M"oglichkeiten, $3$ Produkte
aus Reihe 4 auszuw"ahlen. Insgesamt gibt es also 
\[
2\cdot\binom{10}{3}=2\cdot 120=240.
\]
\item
Zun"achst sind drei Reihen auszuw"ahlen, aus denen ein Produkt stammen
kann.
Dies ist auf $\binom{5}{3}=10$ Arten m"oglich.
Da die zweite Reihe aber weniger verschiedene Artikel enth"alt, m"ussen
wir unterscheiden zwischen Auswahlen, die die zweite Reihe beinhalten,
und anderen.
Es gibt $\binom{4}{2}=6$ M"oglichkeiten, eine Auswahl mit der zweiten
Reihe zu treffen, und $\binom{4}{3}=4$ M"oglichkeiten ohne die zweite Reihe.
Zu jeder Auswahl ohne die zweite Reihe gibt es $10^3=1000$ M"oglichkeiten,
Produkte aus den gew"ahlten Reihen auszuw"ahlen.
Zu jeder Auswahl mit der zweiten Reihe gibt es $8\cdot 10^2=800$ M"oglichkeiten.
Insgesamt gibt es 
\[
6\cdot 1000
+
4\cdot 800
=6000 + 3200=9200
\]
Einkaufsm"oglichkeiten.
\end{teilaufgaben}
\end{loesung}

\begin{bewertung}
\begin{teilaufgaben}
\item 1 Punkt
\item 1 Punkt
\item 1 Punkt
\item 1 Punkt
\item 1 Punkt
\item Fallunterscheidung (\textbf{F}) 1 Punkt,
Anzahl M"oglichkeiten (\textbf{M}) 1 Punkt.
\end{teilaufgaben}
\end{bewertung}

