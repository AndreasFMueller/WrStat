Ein Selecta-Automat enthält sechs Reihen von Fächern, in denen in einer
Transportspirale Artikel eingefüllt sind. Der Kunde bezahlt, wählt ein
Fach, und die Transportspirale befördert ein Stück in ein Fach, aus dem
der Kunde den Artikel entnehmen kann.
\begin{center}
\includeagraphics[width=0.3\hsize]{selecta.jpg}
\end{center}
Ein Mathematik-Student ergattert einen Ferien-Job als
Selecta-Automaten-Nachfüller. Er findet es aber langweilig, die Artikel
immer wieder in die gleichen Fächer nachzufüllen. Er stellt fest,
dass es drei Grössen von Fächern gibt: breite (9 in den obersten zwei Reihen),
normale (40, in den unteren vier Reihen), kleine (4, oben in der Mitte).
Das bringt ihn auf neue Ideen. Man könnte doch beim Auffüllen auch die 
Anordnung der Produkte im Automaten etwas umstellen, so dass zum Beispiel
die Paprika-Chips nach dem Auffüllen nicht mehr oben links, sondern im
Fach ganz rechts in der zweiten Reihe zu finden sind. Natürlich soll
pro Fach immer noch nur ein Produkt eingefüllt sein, und das Fach muss
in der Grösse passend sein.
\begin{teilaufgaben}
\item
Auf wieviele Arten kann man die Produkte im Automaten anordnen?
\item
Die normalen Fächer ermöglichen sehr viele verschiedenen Anordnungen.
Weil das zu lange dauert, entschliesst sich der Student, nur noch die
grossen und kleinen Fächer umzusortieren. Wieviele Möglichkeiten
gibt es jetzt noch?
\item
Wieviele Möglichkeiten gibt es, die normal grossen Produkte
umzuordnen, wenn es auf die Reihenfolge innerhalb einer Reihe des
Automaten nicht ankommt?
\item 
Ein Kunde hat Kleingeld, welches ihm gestattet, genau drei Produkte
aus den Reihen 3 oder 4 zu kaufen. Auf wieviele Arten kann er
das tun, wenn er sich zusätzlich die Bedinung auferlegt, dass alle
drei Produkte aus der gleichen Reihe stammen?
\item
Auf wieviele Arten kann man an diesem Automaten drei Produkte einkaufen, wenn
jedes Produkt aus einer anderen Reihe stammen muss, aber nicht aus
der obersten Reihe kommen darf?
\end{teilaufgaben}

\thema{Kombinatorik}

\begin{loesung}
\begin{teilaufgaben}
\item Es gibt 9 grosse Fächer, 40 normale und 4 kleine. In jeder Gruppe
können die $k$ Produkte auf $k!$ Arten angeordnet, werden. Insgesamt 
bekommt man
\begin{align*}
9!\cdot 40!\cdot 4!&=
362880\cdot 815915283247897734345611269596115894272000000000 \cdot 24
\\
&=7105904111639931116144050020264924857122160640000000000.
\end{align*}
\item $9!\cdot 4!=8709120$.
\item
Für jedes normale Fach muss ein Produkt ausgewählt werden, dies ist
auf $10!$ Arten möglich. Allerdings spielen die $10!$ Reihenfolgen
in jeder Reihe keine Rolle, also ist die Anzahl der Möglichkeiten
\[
\frac{40!}{(10!)^4}=4705360871073570227520.
\]
\item
Es gibt $\binom{10}{3}$ Möglichkeiten, $3$ Produkte aus Reihe 3
auszuwählen, und nochmals $\binom{10}{3}$ Möglichkeiten, $3$ Produkte
aus Reihe 4 auszuwählen. Insgesamt gibt es also 
\[
2\cdot\binom{10}{3}=2\cdot 120=240.
\]
\item
Zunächst sind drei Reihen auszuwählen, aus denen ein Produkt stammen
kann.
Dies ist auf $\binom{5}{3}=10$ Arten möglich.
Da die zweite Reihe aber weniger verschiedene Artikel enthält, müssen
wir unterscheiden zwischen Auswahlen, die die zweite Reihe beinhalten,
und anderen.
Es gibt $\binom{4}{2}=6$ Möglichkeiten, eine Auswahl mit der zweiten
Reihe zu treffen, und $\binom{4}{3}=4$ Möglichkeiten ohne die zweite Reihe.
Zu jeder Auswahl ohne die zweite Reihe gibt es $10^3=1000$ Möglichkeiten,
Produkte aus den gewählten Reihen auszuwählen.
Zu jeder Auswahl mit der zweiten Reihe gibt es $5\cdot 10^2=500$ Möglichkeiten.
Insgesamt gibt es 
\[
4\cdot 1000
+
6\cdot 500
=4000 + 3000=7000
\]
Einkaufsmöglichkeiten.
\qedhere
\end{teilaufgaben}
\end{loesung}

\begin{bewertung}
\begin{teilaufgaben}
\item 1 Punkt
\item 1 Punkt
\item 1 Punkt
\item 1 Punkt
\item Fallunterscheidung (\textbf{F}) 1 Punkt,
Anzahl Möglichkeiten (\textbf{M}) 1 Punkt.
\end{teilaufgaben}
\end{bewertung}

