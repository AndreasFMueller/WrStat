Schreiben Sie die erzeugenden Funktionen zu den folgenden Zahlenfolgen auf:
\begin{teilaufgaben}
\item $(a_k)_{k\ge 1} = 1,1,1,1,\dots$ oder allgemein $a_k = 1$.
\item $(b_k)_{k\ge 1} = 1,2,3,4,\dots$ oder allgemein $b_k = k$.
\item $(c_k)_{k\ge 1} = 1,\frac12,\frac13,\frac14,\dots$ oder allgemein $c_k = \frac1k$.
\end{teilaufgaben}
Können Sie die Funktionen in geschlossener Form schreiben?

\begin{loesung}
\begin{teilaufgaben}
\item Die erzeugende Funktion ist die geometrische Reihe
\[
a(z)
=
\sum_{k=1}^\infty a_kz^k = \sum_{k=1}^\infty z^k = \frac{1}{1-z}.
\]
\item Die erzeugende Funktion ist 
\begin{align*}
b(z)
&=
\sum_{k=1}^\infty b_kz^k
=
\sum_{k=1}^\infty kz^k
\intertext{Durch Ausklammern eines Faktors $z$ erhält man die folgende
Reihe, die man als Ableitung von $a(z)$ schreiben kann:}
&=
z
\sum_{k=1}^\infty kz^{k-1}
=
z\frac{d}{dz}
\sum_{k=1}^\infty z^{k-1}
=
z\frac{d}{dz}
\sum_{k=0}^\infty z^k.
\intertext{Bis auf den ersten Term der Reihe ist die Summe auf der rechten
Seite die Funktion $a(z)$ von Teilaufgabe a):}
&=
z\frac{d}{dz}\bigl(a(z)-1\bigr)
=
za'(z)
=
z\frac{d}{dz}
\frac{1}{1-z}
=
\frac{z}{(1-z)^2}.
\end{align*}
\item Die erzeugende Funktion ist die Stammfunktion der Funktion $a(z)$:
\begin{align*}
c(z)
&=
\sum_{k=1}^\infty c_kz^k
=
\sum_{k=1}^\infty \frac{z^k}{k}
=
\sum_{k=1}^\infty \int z^{k-1}\,dz
=
\int
\sum_{k=1}^\infty z^{k-1}\,dz
=
\int
\sum_{k=0}^\infty z^k
\,dz
\\
&=
\int a(z)\,dz
=
\int \frac{dz}{1-z}
=
-\log(1-z).
\qedhere
\end{align*}
\end{teilaufgaben}
\end{loesung}
