Komplexe Passwörter erhöhen die Datensicherheit, es ist aber auch
entsprechend schwieriger, sich an sie zu erinnern.
Eine Hilfe für Benutzer kann sein, nicht nur zufällige Zeichenfolgen
sondern aussprechbare Passwörter zu erzeugen.
Die Forderung der Aussprechbarkeit hat zur Folge, dass weniger
Passwörter bei gleicher Passwortlänge zur Verfügung stehen.
In dieser Aufgabe soll untersucht werden, wieviel länger die Passwörter
sein müssen, um die gleiche Sicherheit zu erreichen.
Es soll angenommen werden, dass nur $a=26$ Buchstaben zur Verfügung
stehen, wovon 5 Vokale sind.
\begin{teilaufgaben}
\item
Wieviele verschiedene Passwörter der Länge $2n$ gibt es?
\item
Wieviele Silben aus genau einem Konsonanten und einem Vokal
gibt es?
Solche Kombinationen werden auch {\em Digraphen} genannt.
\item
Wieviele Passwörter der Längen $2n$ gibt es, die nur aus
Digraphen wie in Teilaufgaben b) zusammengesetzt sind?
\item
Wieviele Passwörter der Länge $2n$ gibt es, die aus genau
$n$ Vokalen und $n$ Konsonanten bestehen?
\item
Deutsch und Englisch lassen sich besser approximieren, wenn man
Silben aus drei Buchstaben verwendet, die {\em Trigraphen} heissen.
Wieviele Trigraphen gibt es, die mit einem Konsonanten beginnen,
an der zweiten Stelle einen Vokal haben und mit einem beliebigen
Buchstaben Enden?
\item
Wieviele Passwörter der Länge $12$ lassen sich aus Trigraphen
zusammensetzen?
Wieviel mehr Trigraphen-Passwörter als Digraphen-Passwörter gibt es?
\end{teilaufgaben}

\begin{loesung}
\begin{teilaufgaben}
\item
Perlenkettenproblem: $26^{2n}$.
\item
$21$ Konsonanten und $5$ Vokale ergibt $21\cdot 5=105$ verschiedene 
Digraphen.
\item
Passwortlänge $2n$ bedeutet, dass sich das Passwort aus $n$ Digraphen
zusammensetzt.
Daher gibt es $101^n$ verschiedene Passwörter.
\item
Dazu müssen erst die Stellen ausgewählt werden, an denen Vokale stehen
sollen.
Dies ist ein Auswahlproblem: es gibt $\binom{2n}{n}$ Möglichkeiten.
Dann müssen die $n$ Vokale bestimmt werden, das geht auf $5^n$ Arten.
Für die Konsonanten gibt es entsprechend $21^n$ Arten.
Die Gesamtzahl der Passwörter ist damit
\[
\binom{2n}{n}\cdot 5^n 21^n 
=
\binom{2n}{n} \cdot 105^n
\]
\item
$5\cdot 21\cdot 26=2730$
\item
Ein Trigraphen-Passwort der Länge 12 besteht aus 4 Trigraphen, also
gibt es $2730^4$ solche Passwörter.
Da es $105^6$ Digraphen-Passwörter gibt, gibt es
\[
\frac{2730^4}{105^6}
=
\frac{5^4\cdot 21^4\cdot 26^4}{21^6\cdot 5^6}
=
\frac{26^4}{21^2\cdot 5^2}
=
\biggl(
\frac{676}{105}
\biggr)^2
\approx
41.44907
\]
mal mehr Trigraphen-Passwörter.
Trotz der auf den ersten Blick einschränkenderen Struktur mit
weniger Elementen gibt es vor allem wegen des dritten, freien Buchstabens
in jedem Trigraphen über 40 mal mehr Trigraphen-Passwörter.
\qedhere
\end{teilaufgaben}
\end{loesung}

\begin{diskussion}
Die Erzeugung aussprechbarer Passwörter mit Digraphen und Trigraphen
wurde in den Sechzigerjahren gut untersucht.
Der FIPS-181-Standard beschreibt eine noch viel ausgefeiltere Methode,
englisch aussprechbare Passwörter zu erzeugen.
Sie leitet sich ab aus einem Algorithmus von Morrie Gasser, der für
das Multics-Projekt geschrieben wurde.
Eine Adaption an moderne Systeme ist das Programm \texttt{gpw} von
Tom Van Vleck vom Projekt \url{multicians.org}.
Das Programm ist auch für Linux verfügbar, auf Debian im Packet 
\texttt{gpw}.
Eine Javascript-basiert Web-Implementation findet man unter
\url{https://multicians.org/thvv/gpw-js.html}.
\end{diskussion}
