Eine Skiverleihfirma besitzt 6 Paare weisser und 7 Paare schwarzer Skis
einer bestimmten Grösse. Die Skis sind asymmetrisch, man kann linke
und rechte Skis unterscheiden.
Nach dem Sommer werden die Skis aus dem Lagercontainer geholt, 
wo sie ziemlich durcheinander geraten sind.
\begin{teilaufgaben}
\item Auf wieviele Arten können die Skis zu Paaren zusammengefügt werden?
\item Der Marketingchef findet, es dürften durchaus auch gemischfarbige Paare
vermietet werden, wieviele mögliche Paarungen gibt es in diesem
Fall?
\item Bei der Revision stellt sich heraus, das zwei schwarze linke Skis
und eine linker weisser Ski nicht mehr zu reparieren sind. Wieviele
mögliche Paarungen gibt es jetzt noch?
\end{teilaufgaben}

\thema{Kombinatorik}

\begin{loesung}
\begin{teilaufgaben}
\item Stellt man sich vor, dass die linken Skier einer bestimmen
Farbe in einer bestimmten
Reihenfolge angeordnet sind, dann bekommt man alle möglichen Paarungen,
indem man die rechten Skis der gleichen Farbe in allen möglichen
Reihenfolgen anordnet.
Für $n_c$ Paare der Farbe $c$ gibt es also $n_c!$ mögliche
Paarungen für Skis der Farbe $c$. Für die beiden Farben Schwarz
und weiss ergibt sich:
\[
n_s!\cdot n_w!=7!\cdot 6!=5040\cdot 720=3628800
\]
\item
Falls die Farbe keine Rolle spielt, ist das gleiche Argument wie vorhin
anwendbar, wobei jetzt alle Skis die gleiche Farbe haben. Die Anzahl
der möglichen Paarungen ist $13!=6227020800$.
\item
Hat man nur $n$ linke und $m$ rechte Skis einer gewissen Farbe, mit $n<m$,
dann muss man offenbar zuerst aus den $m$ Skis deren $n$ auswählen,
die man tatsächlich für Paarungen verwenden will. Das geht auf $\binom{m}{n}$
Arten. Dann kann man wie vorhin $n!$ Anordnungen bilden. Die Gesamtzahl
der möglichen Paarungen ist also 
\[
\binom{m}{n}n!=\frac{m!}{n!\,(m-n)!}\cdot n!=\frac{m!}{(m-n)!}
=
m\cdot(m-1)\cdot\dots\cdot(m-n+1).
\]
Im vorliegenden Fall ist für die schwarzen Skis $n=5$ und $m=7$,
für die weissen Skis ist dagegen $n=5$ und $m=6$.
Die Gesamtzahl der möglichen Paarungen ist daher
\[
\frac{7!}{2!}\cdot \frac{6!}{1!}
=\frac12\cdot 7!\,6!=1814400
\]
\end{teilaufgaben}
Falls man wie in b) auch gemischtfarbige Paare zulassen will, muss
man nur zählen, auf wieviele Arten man die 10 vorhandenen
linken Skis mit den 13 rechten Skis paaren kann. Mit der gleichen
Methode wie vorhin findet man
\[
\binom{13}{10}10!=\frac{13!}{3!}=1037836800.
\qedhere
\]
\end{loesung}

%\begin{bewertung}
%a) Permutationen ({\bf P}) 1 Punkt, Produkt ({\bf M}) 1 Punkt.
%b) Permutationen ohne Farbeinschränkung ({\bf F}) 1 Punkt.
%c) Auswahl von passenden Skiern ({\bf A}) 1 Punkt,
%Reihenfolgen ({\bf R}) 1 Punkt, Produkt ({\bf M2}) 1 Punkt.
%\end{bewertung}
