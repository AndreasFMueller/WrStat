Beim Wichtelspiel werden die Namen aller Teilnehmer auf Zettel geschrieben
und in einen Korb getan.
Die Teilnehmer ziehen der Reihe nach einen Zettel.
Zieht jemand seinen eigenen Namen, legt er ihn zur"uck und probiert
nochmals, bis er einen anderen als den eigenen Namen zieht.
Es kann nat"urlich passieren, dass dies beim letzten Teinehmer
nicht mehr klappt, dann ist die Namens-Zuteilung fehlgeschlagen und
muss erneut begonnen werden.

\begin{hinweis}
Es gibt keine ``leichte'' Methode, Formeln f"ur die gesuchten
Wahrscheinlichkeiten anzugeben.
\end{hinweis}

\begin{teilaufgaben}
\item Auf wieviele Arten k"onnen $n$ Teilnehmer die Namen zuteilen,
wenn Sie nicht auf die Zettel schauen und zur"ucklegen, wenn sie ihren
eigenen Namen gezogen haben?
\item Auf wieviele Arten k"onnen drei Teilnehmer die Namen zuteilen
so dass keiner seinen eigenen Namen zieht?
\item Wie gross ist die Wahrscheinlichkeit dass drei Teilnehmer
bei einer ``blinden Ziehung'' wie in a) eine Zuteilung erreichen, bei
der niemand seinen eigenen Namen gezogen hat?
\item Auf wieviele Arten k"onnen drei Teilnehmer die Namen zuteilen,
so dass der letzte Teilnehmer seinen eigenen Namen zugeteilt bekommt?
\item Beantworten Sie b) und d) f"ur 4 statt 3 Teilnehmer.
\item Wie gross ist die Wahrscheinlichkeit, dass bei 4 Teilnehmern
der Zuteilungsalgorithmus fehlschl"agt?
\end{teilaufgaben}

\begin{loesung}
\begin{teilaufgaben}
\item 
In diesem Fall werden einfach nur alle Permutationen erzeugt,
das geht auf $n!$ Arten.
\item
Es gibt 6 Permutation der drei Namen,  n"amlich
\begin{align*}
1:\qquad&{\color{red}1}\;{\color{red}2}\;{\color{red}3}\\
2:\qquad&{\color{red}1}\;            3 \;            2 \\
3:\qquad&            2 \;            1 \;{\color{red}3}\\
4:\qquad&            2 \;            3 \;            1 \\
5:\qquad&            3 \;            1 \;            2 \\
6:\qquad&            3 \;{\color{red}2}\;            1 
\end{align*}
Rote Ziffern zeigen eine Verletzung der Regeln an.
Nur die Permutationen 4 und 5 erf"ullen also alle Regeln.
\item
Nach b) geht das in 2 von 6 F"allen, also mit Wahrscheinlichkeit $\frac13$.
\item
Nur die Permutation 3 ist m"oglich.
\item
Die gleiche L"osung kann auch auf 4 Personen angewendet werden:
\begin{align*}
 1:\qquad&{\color{red}1}\;{\color{red}2}\;{\color{red}3}\;{\color{red}4}\\
 2:\qquad&{\color{red}1}\;{\color{red}2}\;            4 \;            3 \\
 3:\qquad&{\color{red}1}\;            3 \;            2 \;{\color{red}4}\\
 4:\qquad&{\color{red}1}\;            3 \;            4 \;            2 \\
 5:\qquad&{\color{red}1}\;            4 \;            2 \;            3 \\
 6:\qquad&{\color{red}1}\;            4 \;{\color{red}3}\;            2 \\
%
 7:\qquad&            2 \;            1 \;{\color{red}3}\;{\color{red}4}\\
 8:\qquad&            2 \;            1 \;            4 \;            3 \\
 9:\qquad&            2 \;            3 \;            1 \;{\color{red}4}\\
10:\qquad&            2 \;            3 \;            4 \;            1 \\
11:\qquad&            2 \;            4 \;            1 \;            3 \\
12:\qquad&            2 \;            4 \;{\color{red}3}\;            1 \\
%
13:\qquad&            3 \;            1 \;            2 \;{\color{red}4}\\
14:\qquad&            3 \;            1 \;            4 \;            2 \\
15:\qquad&            3 \;{\color{red}2}\;            1 \;{\color{red}4}\\
16:\qquad&            3 \;{\color{red}2}\;            4 \;            1 \\
17:\qquad&            3 \;            4 \;            1 \;            2 \\
18:\qquad&            3 \;            4 \;            2 \;            1 \\
%
19:\qquad&            4 \;            1 \;            2 \;            3 \\
20:\qquad&            4 \;            1 \;{\color{red}3}\;            2 \\
21:\qquad&            4 \;{\color{red}2}\;            1 \;            3 \\
22:\qquad&            4 \;{\color{red}2}\;{\color{red}3}\;            1 \\
23:\qquad&            4 \;            3 \;            1 \;            2 \\
24:\qquad&            4 \;            3 \;            2 \;            1
\end{align*}
Nur die 9 Permutationen 8, 10, 11, 14, 17, 18, 19, 23 und 24 erf"ullen
alle Bedingungen.
Es gibt genau zwei Permutationen, bei denen dem vierten Teilnehmer der
eigene Name zugeteilt wird, n"amlich 9 und 13.
\item
Die Wahrscheinlichkeit, dass im Wichtelspiel die Ziehung wiederholt 
werden muss ist $2/(9 + 2)=18.18\%$.
\end{teilaufgaben}
\end{loesung}

\begin{bewertung}
Jede Teilaufgabe 1 Punkt.
\end{bewertung}

