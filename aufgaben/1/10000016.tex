Am Bahnhof auf dem Perron stehen zwei B"anke, auf denen je vier
Personen Platz nehmen k"onnen.
\begin{teilaufgaben}
\item
Auf wieviele Arten k"onnen sieben Reisende auf den beiden B"anken Platz
nehmen?
\item
Vier Nichtraucher und drei Raucher m"ochten sich setzen.
Auf wieviele Arten ist es m"oglich, so dass auf jeder Bank mindestens
ein Raucher sitzt.
\item
Wie gross ist die Wahrscheinlichkeit, dass auf jeder Bank ein Raucher
sitzt?
\item
Den Raucher sieht man ihr Laster nicht an, mit Wahrscheinlichkeit $0.5$
beginnen Sie jedoch sofort zu rauchen, wenn sie sich gesetzt haben, was
f"ur die Nichtraucher auf der gleichen Bank unertr"aglich ist.
Wie gross ist die Wahrscheinlichkeit, dass alle Nichtraucher bel"astigt
werden?
\end{teilaufgaben}

\begin{loesung}
\begin{teilaufgaben}
\item
Es m"ussen auf jeden Fall 4 Personen auf der einen Bank Platz nehmen,
und wir haben zwei B"anke, auf denen dies geschehen kann. 
Damit haben wir $2\cdot\binom{7}{4})=2\cdot 35=70$ M"oglichkeiten.
\item
Dies bedeutet, dass auf der einen Bank $1$ Raucher, und auf der anderen
$2$ Raucher sitzen.
Es gibt $3$ M"oglichkeiten, einen Raucher auf die erste Bank zu setzen,
und die Raucher auf der zweiten Bank sind damit auch bereits festgelegt.
Auf der ersten Bank m"ussen jetzt noch zwei oder drei Nichtraucher platziert
werden.
Es gibt $\binom{4}{2}=6$ M"oglichkeiten, 2 Nichtraucher auszw"ahlen, und
$\binom{4}{3}=4$ M"oglichkeiten, 3 Nichtraucher auszw"ahlen.
Insgesamt gibt es also $10$ M"oglichkeiten f"ur die Auswahl der Raucher,
die sich mit dem einen Nichtraucher die ersten Bank teilen.
Wir k"onnen den einen Nichtraucher aber auf zwei B"anke platzieren, so dass
wir insgesamt
\[
2\cdot 3\cdot 10 = 60
\]
M"oglichkeiten haben.
\item
Diese Situation tritt in $60$ von $70$ F"allen ein, also mit
Wahrscheinlicheit $\frac{6}{7}$.
\item
Dieser Fall tritt ein, wenn auf beiden B"anken Raucher sitzen,
und ein Raucher zu rauchen beginnt.
Die Wahrscheinlichkeit, dass auf der Bank mit einem Raucher geraucht
wird, ist $0.5$.
Die Wahrscheinlichkeit, dass auf der Bank mit zwei Rauchern geraucht
wird, ist dagegen
\begin{align*}
&
P(\text{mindestens ein Raucher raucht})
\\
&=
P(\text{Raucher 1 raucht}\quad\vee\quad\text{Raucher 2 raucht})
\\
&=
P(\text{Raucher 1 raucht})
+
P(\text{Raucher 2 raucht})
-
P(\text{beide rauchen})
\\
&=
0.5+0.5-0.5^2 = 0.75.
\end{align*}
Die Wahrscheinlichkeit, dass auf beiden B"anken geraucht wird, wenn auf
beiden B"anken Raucher sitzen, ist daher $0.5\cdot 0.75 = 0.375$.
Da nur mit Wahrscheinlichkeit $\frac{6}{7}$ auf beiden B"anken
Raucher sitzen, ist die Wahrscheinlichkeit, dass alle Nichtraucher
bel"astigt werden $\frac{6}{7}\cdot 0.375=0.3214$
\end{teilaufgaben}
\end{loesung}

\begin{bewertung}
\end{bewertung}


