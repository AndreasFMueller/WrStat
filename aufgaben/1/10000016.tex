Am Bahnhof auf dem Perron stehen zwei Bänke, auf denen je vier
Personen Platz nehmen können.
\begin{teilaufgaben}
\item
Auf wieviele Arten können sieben Reisende auf den beiden Bänken Platz
nehmen?
\item
Vier Nichtraucher und drei Raucher möchten sich setzen.
Auf wieviele Arten ist es möglich, so dass auf jeder Bank mindestens
ein Raucher sitzt.
\item
Wie gross ist die Wahrscheinlichkeit, dass auf jeder Bank ein Raucher
sitzt?
\item
Den Raucher sieht man ihr Laster nicht an, mit Wahrscheinlichkeit $0.5$
beginnen Sie jedoch sofort zu rauchen, wenn sie sich gesetzt haben, was
für die Nichtraucher auf der gleichen Bank unerträglich ist.
Wie gross ist die Wahrscheinlichkeit, dass alle Nichtraucher belästigt
werden?
\end{teilaufgaben}

\begin{loesung}
Leider ist die Formulierung der Aufgabe nicht ganz eindeutig.
Es ist nicht klar, ob in den Teilaufgaben a) und b) die verschiedenen
Anordnungen auf den beiden Bänken gezählt werden sollen, oder ob es
nur darauf ankommt, wie die Personen auf die beiden Bänke verteilt
sind.

Ausserdem ist in beiden Interpretationen ist nicht ganz klar, ob man mitzählen
will, welcher der Plätze leer bleibt.
Die Anzahl der Anordnungen wird aber immer um den Faktor $4$ 
vervielfältigt, wenn man die Position des leeren Platzes berücksichtigt.

Wenn man in beiden Teilaufgaben a) und b) die gleiche Interpretation verwendet,
erhält man in c) auch die gleiche Wahrscheinlichkeit, die Wahl der
Interpretation hat also keinen Einfluss auf die Resultate in c) und d).
\begin{teilaufgaben}
\item
Wenn die Anordnung berücksichtigt werden muss, dann geht es nur
darum, $7$ Personen zu platzieren.
Es geht also um das Problem, $7$ Personen und einen leeren Platz,
also $8$ Objekte auf total $8$ Plätze zu verteilen, was auf 
$8!=40320$ Arten möglich ist.

Man könnte aber auch verstehen, dass es darum geht, $7$ Personen
auf zwei Gruppen von Plätzen (die beiden Bänke) zu verteilen,
wobei es nicht auf die Anordnung innerhalb einer Bank ankommt.
In diesem Fall
müssen auf jeden Fall 4 Personen auf der einen Bank Platz nehmen,
und wir haben zwei Bänke, auf denen dies geschehen kann. 
Damit haben wir $2\cdot\binom{7}{4}=2\cdot 35=70$ Möglichkeiten.

\item
Wie in Teilaufgabe a) kann man auch diese Aufgabe auf zwei Arten verstehen.
Gemeinsam ist, dass sich auf einer Bank immer genau ein Raucher befinden
muss, auf der anderen zwei.

Achtet man auf die Anordnung auf den Bänken,
hat man $3$ Raucher, die man auf einen der $4$ Plätze der ersten Bank setzen
kann, und $4\cdot 3=12$ Möglichkeiten, die zwei anderen Raucher auf die
zweite Bank zu platzieren.
Jetzt muss man noch die $4$ Nichtraucher platzieren,
Da die Anordnung eine Rolle spielt, muss man $4$ Personen auf und
einen leeren Platz auf fünf Plätze verteilen, was auf $5!=120$
Arten geht.
Man hat also
\[
2\cdot 3\cdot 4\cdot 12\cdot 5!=34560
\]
Möglichkeiten.

Wenn man die Anordnungen auf den Bänken nicht berücksichtigt,
gibt es $3$ Möglichkeiten, einen der Raucher auf die erste Bank zu setzen,
und die Raucher auf der zweiten Bank sind damit auch bereits festgelegt.
Auf der ersten Bank müssen jetzt noch zwei oder drei Nichtraucher platziert
werden.
Es gibt $\binom{4}{2}=6$ Möglichkeiten, 2 Nichtraucher auszwählen, und
$\binom{4}{3}=4$ Möglichkeiten, 3 Nichtraucher auszwählen.
Insgesamt gibt es also $10$ Möglichkeiten für die Auswahl der Raucher,
die sich mit dem einen Nichtraucher die ersten Bank teilen.
Wir haben zu Beginn aber zwei Bänke, auf denen wir den einen Raucher
platzieren können, so dass wir insgesamt
\[
2\cdot 3\cdot 10 = 60
\]
Möglichkeiten haben.
\item
Mit den Resultaten der Teilaufgaben a) und b) können wir diese
Wahrscheinlichkeit als Quotient bestimmen:
\[
p= \frac{34560}{40320}=\frac{6}{7}=\frac{60}{70}.
\]
\item
Dieser Fall tritt ein, wenn auf beiden Bänken Raucher sitzen,
und ein Raucher zu rauchen beginnt.
Die Wahrscheinlichkeit, dass auf der Bank mit einem Raucher geraucht
wird, ist $0.5$.
Die Wahrscheinlichkeit, dass auf der Bank mit zwei Rauchern geraucht
wird, ist dagegen
\begin{align*}
&
P(\text{mindestens ein Raucher auf Bank 2 raucht})
\\
&=
P(\text{Raucher 1 raucht}\quad\vee\quad\text{Raucher 2 raucht})
\\
&=
P(\text{Raucher 1 raucht})
+
P(\text{Raucher 2 raucht})
-
P(\text{beide rauchen})
\\
&=
0.5+0.5-0.5^2 = 0.75.
\end{align*}
Die Wahrscheinlichkeit, dass auf beiden Bänken geraucht wird, wenn auf
beiden Bänken Raucher sitzen, ist daher $0.5\cdot 0.75 = 0.375$.
Da nur mit Wahrscheinlichkeit $\frac{6}{7}$ auf beiden Bänken
Raucher sitzen, ist die Wahrscheinlichkeit, dass alle Nichtraucher
belästigt werden $\frac{6}{7}\cdot 0.375=0.3214$
\qedhere
\end{teilaufgaben}
\end{loesung}

\begin{bewertung}
\begin{teilaufgaben}
\item
({\bf A}) 1 Punkt,
\item
Zerlegung des Problems in Platzierung der Raucher und der Nichtraucher
({\bf Z}) 1 Punkt,
Resultat b) ({\bf B}) 1 Punkt,
\item
Wahrscheinlichkeit ({\bf W}) 1 Punkt,
\item
Ein-/Ausschalt-Formel ({\bf E}) 1 Punkt,
Wahrscheinlichkeit als Produkt ({\bf P}) 1 Punkt.
\end{teilaufgaben}
\end{bewertung}


