20 Stundenten wollen zusammen Skifahren gehen.
\begin{teilaufgaben}
\item
Die Studenten rechnen aus, dass sie mit fünf Autos zu je vier Plätzen
den Transport ins Skigebiet bewerkstelligen könnten.
Auf wieviele Arten können sich die Studenten auf die Autos
aufteilen, wenn es auf die Platzierung innerhalb der Autos
ankommt?
\item
Fünf Studenten bieten an, mit ihren Autos die Studenten ins Skigebiet
zu fahren.
Auf wieviele Arten können die übrigen Studenten auf die Autos verteilt werden,
wenn es auf die Anordnung der Studenten (ausser dem Fahrer natürlich)
innerhalb eines Autos {\em nicht} ankommt?
\item
Bei der weiteren Planung stellt sich heraus, dass fünf der Nicht-Fahrer
unter Reisekrankheit leiden und daher in einem Auto vorne sitzen
müssen.
Auf wievele Arten können die Studenten unter dieser Einschränkung noch
den Autos zugeordnet werden?
Auf die Anordnung der Teilnehmer auf den hinteren Plätzen im Auto
soll es nicht ankommen.
\end{teilaufgaben}

\thema{Kombinatorik}

\begin{loesung}
\begin{teilaufgaben}
\item
Es stehen 20 Plätze für 20 Studenten zur Verfügung, dies ist ein
Anordnungsproblem, die Anzahl der Anordnungen ist
\[
20!
=
2432902008176640000
\]
\item
Die Fahrer sind den Autos bereits zugeordnet, es geht also nur noch darum,
die 15 anderen Studenten zu verteilen.
Für das erste Auto müssen 3 aus den 15 ausgewählt werden, für das zweite
3 aus den verbleibenden 12, etc.
Dies ist auf
\[
\binom{15}{3}
\binom{12}{3}
\binom{9}{3}
\binom{6}{3}
\binom{3}{3}
=
455\cdot 220\cdot 84 \cdot 20\cdot 1
=
168168000.
\]
Übrigens kann man diese Zahl auch mit Hilfe der Fakultätsformel ausrechnen:
\begin{align*}
\binom{15}{3}
\binom{12}{3}
\binom{9}{3}
\binom{6}{3}
\binom{3}{3}
&=
\frac{15!}{3!\,12!}
\frac{12!}{3!\,9!}
\frac{9!}{3!\,6!}
\frac{6!}{3!\,3!}
\frac{3!}{3!\,0!}
=
\frac{15!\,12!\,9!\,6!\,3!}{(3!)^5\,12!\,9!\,6!\,3!}
=
\frac{15!}{(3!)^5}
=
168168000.
\end{align*}

Alternativ kann man auch wie folgt argumentieren.
15 Studenten sind auf die 15 ``Nichtfahrer''-Plätze zu verteilen.
Damit sind aber alle möglichen Anordnungen innerhalb der Autos mitgezählt.
In jedem Auto können die ``Nichtfahrer'' auf $3!$ Arten angeordnet werden.
Es gibt also zu einer Zuteilen zu den Autos jeweils $3!^5$ Möglichkeiten,
wie sich die zugeteilten innerhalb der Autos anordnen können.
Die Anzahl Möglichkeiten ist daher
\[
\frac{15!}{3!^5}
=
168168000.
\]
\item
Die fünf Reisekranken müssen je einem Auto zugeteilt werden, dies ist
ein Anordnungsproblem, welches auf $5!=120$ Arten gelöst werden kann.
Die verbleibenden 10 Studenten müssen je zu zweit auf die fünf Autos 
verteilt werden. Dies kann nach dem gleichen Verfahren wie in
Teilaufgabe~b) gemacht werden.
Die gesamte Zahl der Zuordnungen wird
\[
5!\cdot
\binom{10}{2}
\binom{8}{2}
\binom{6}{2}
\binom{4}{2}
=
5!\cdot
\frac{10!}{2^5}
=
120\cdot\frac{3628800}{32}
=
13608000.
\]
Alternativ kann man wie folgt argumentieren.
Die 5 Beifahrer können auf $5!$ Arten den Autos zugeordnet werden.
Die 10 ``Nichtbeifahrer'' können auf $10!$ Arten den Autos zugeordnet
werden, aber sie können in jedem Auto die Plätze auf der Rückbank auf
$2!=2$ Arten einnehmen, insgesamt also $2^5$ Möglichkeiten.
Diese Möglichkeiten sind in $10!$ mitgezählt, um die gesuchte Anzahl
zu erhalten, müssen wir also noch durch $2^5$ dividieren:
\[
5!\cdot
\frac{10!}{2^5}
=
120\cdot\frac{3628800}{32}
=
13608000.
\qedhere
\]
\end{teilaufgaben}
\end{loesung}

\begin{bewertung}
Jede Teilaufgabe 2 Punkte.
Teilpunkte für Binomialkoeffizienten-Faktoren in den Produkten in Teilaufgaben 
b) und c) oder für den Faktor $5!$ in c).
\end{bewertung}
