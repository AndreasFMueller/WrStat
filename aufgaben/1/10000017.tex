Ein Restaurant hat 8 Vierertische und 6 Zweiertische.
An einem Abend besuchen 5 Vierergruppen und 5 Zweiergruppen das Restaurant.
Jede Gruppe m"ochte als ganze an einem Tisch sitzen, und ausserdem
keine G"aste anderer Gruppen am eigenen Tisch haben.
\begin{teilaufgaben}
\item Auf wieviele Arten k"onnen die Gruppen den Tischen zugeordnet
werden?
\item Auf wieviele Arten k"onnen die G"aste platziert werden?
\item Wie gross ist die Wahrscheinlichkeit, dass mindestens ein Vierertisch
leer bleibt?
\end{teilaufgaben}

\begin{loesung}
\begin{teilaufgaben}
\item
Die Vierergruppen k"onnen nur auf Vierertische verteilt werden,
daher m"ussen wir dies als erstes tun.
Man muss daher $5$ Vierertische aus $8$ ausw"ahlen.
Dann bleiben $9$ Tische, die f"ur alle Zweiergruppen geeignet sind,
es m"ussen also $5$ Zweiertische aus diesen $9$ zur Verf"ugung stehenden
angeboten ausgew"ahlt werden. 
Die Gesamtzahl der M"oglichkeiten ist daher
\[
\binom{8}{5}\cdot\binom{9}{5}=
56\cdot 126=7056.
\]
\item
In jeder Vierergruppe sind $4!=24$ Permutationen m"oglich, in jeder
Zweiergruppe deren $2$.
Die Gesamtzahl der M"oglichkeiten, die G"aste zu platzieren, ist daher
\[
7056\cdot 4!^5 \cdot 2^5=1797896798208.
\]
\item
Wir m"ussen die Belegungen z"ahlen, in denen alle Vierertische
besetzt sind. 
Dazu sind immer noch zun"achst $5$ Vierertische auszw"ahlen f"ur
die Vierergruppen, doch dann werden $3$ aus den $5$ Zweiergruppen ausgew"ahlt,
die an den verbleibenden Vierertischen platziert werden.
Die letzten $2$ Zweiergruppen w"ahlen dann unter den $6$ Zweiertischen.
Insgesamt bekommen wir f"ur die Anzahl:
\[
\binom{8}{5}\cdot \binom{5}{3}\cdot \binom{6}{2}
\]
Um die Wahrscheinlichkeit daf"ur zu bekommen, dass alle Vierertische
besetzt sind, teilen wir diese Zahl durch das Resultat von a), und
erhalten
\[
P(\text{Vierertische besetzt})
=
\frac{\binom{8}{5}\cdot\binom{9}{5}}{
\binom{8}{5}\cdot \binom{5}{3}\cdot \binom{6}{2}}
=
\frac{\binom{9}{5}}{\binom{5}{3}\cdot \binom{6}{2}}
=
\frac{126}{10\cdot 15}=\frac{126}{150}=0.84.
\]
Die Wahrscheinlichkeit, dass mindestens ein Vierertisch frei bleibt ist
daher $1-0.84=0.16$.
\qedhere
\end{teilaufgaben}
\end{loesung}

\begin{bewertung}
\begin{teilaufgaben}
\item
Auswahl ({\bf A}) 1 Punkt, Produktregel ({\bf F}) 1 Punkt,
\item
Permutationen ({\bf P}) 1 Punkt, Variationen ({\bf V}) 1 Punkt,
\item
Anzahl Belegungen, bei denen alle Vierertische besetzt sind
({\bf 4}) 1 Punkt,
Wahrscheinlichkeit ({\bf W}) 1 Punkt.
\end{teilaufgaben}
\end{bewertung}



