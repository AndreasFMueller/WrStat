Ein Restaurant hat 8 Vierertische und 6 Zweiertische.
An einem Abend besuchen 5 Vierergruppen und 5 Zweiergruppen das Restaurant.
Jede Gruppe möchte als ganze an einem Tisch sitzen, und ausserdem
keine Gäste anderer Gruppen am eigenen Tisch haben.
\begin{teilaufgaben}
\item Auf wieviele Arten können die Gruppen den Tischen zugeordnet
werden?
\item Auf wieviele Arten können die Gäste platziert werden?
\item Wie gross ist die Wahrscheinlichkeit, dass mindestens ein Vierertisch
leer bleibt?
\end{teilaufgaben}

\thema{Kombinatorik}

\begin{loesung}
\begin{teilaufgaben}
\item
Die Vierergruppen können nur auf Vierertische verteilt werden,
daher müssen wir dies als erstes tun.
Man muss daher $5$ Vierertische aus $8$ auswählen.
Dann bleiben $9$ Tische, die für alle Zweiergruppen geeignet sind,
es müssen also $5$ Zweiertische aus diesen $9$ zur Verfügung stehenden
angeboten ausgewählt werden. 
Die Gesamtzahl der Möglichkeiten ist daher
\[
\binom{8}{5}\cdot\binom{9}{5}=
56\cdot 126=7056.
\]
\item
In jeder Vierergruppe sind $4!=24$ Permutationen möglich, in jeder
Zweiergruppe deren $2$.
Die Gesamtzahl der Möglichkeiten, die Gäste zu platzieren, ist daher
\[
7056\cdot 4!^5 \cdot 2^5=1797896798208.
\]
\item
Nachdem die fünf Vierergruppen an Vierertischen platziert sind, bleiben
noch neun Tische für die Zweiergruppen.
Man muss jetzt also herausfinden, wie wahrscheinlich es ist, dass die
drei Vierertische besetzt sind, wenn man fünf der Tische auswählt.
Dann sind zwei der Zweiertische besetzt.
Dies ist ein Lottoproblem, dessen Wahrscheinlichkeit mit der
hypergeometrischen Verteilung berechnet werden kann:
\[
P(\text{alle Vierertische besetzt})
=
\frac{\binom{3}{3}\binom{6}{2}}{\binom{9}{5}}
=
\frac{1\cdot 15}{126}
=
\frac1{42}=0.11905
\]
\end{teilaufgaben}
Die gesuchte Wahrscheinlichkeit ist jetzt
\[
P(\text{ein Vierertisch leer})
=
1-P(\text{alle Vierertische besetzt})
=
0.88095.
\qedhere
\]
\end{loesung}

\begin{diskussion}
Man kann Teilaufgabe b) auch so verstehen, dass die zwei Personen an
den Dreiertischen alle denkbaren Platzierungen einnehmen können.
Dies wird in Restaurants allerdings normalerweise nicht so praktiziert.
Wenn man dies aber tut, dann sind an einem Vierertisch nicht $2$ sondern
$2 \binom{4}{2}=12$ Platzierungen möglich.
Natürlich kommt es jetzt auch darauf an, wieviele Vierertische von 
Zweiergruppen besetzt sind.
Die Anzahl Platzierungen der Gäste der Zweiergruppen ist dann
\begin{align*}
\sum_{k=0}^3 \binom{3}{k}12^k\binom{6}{5-k}2^{5-k}
&=
1\cdot 12^0\cdot 6\cdot 2^5
+
3\cdot 12^1\cdot 15\cdot 2^4
+
3\cdot 12^2\cdot 20\cdot 2^3
+
1\cdot 12^3\cdot 15\cdot 2^2
\\
&=
192 + 8640 + 69120 + 311040
=
388992.
\end{align*}

\end{diskussion}

\begin{bewertung}
\begin{teilaufgaben}
\item
Auswahl ({\bf A}) 1 Punkt, Produktregel ({\bf F}) 1 Punkt,
\item
Permutationen ({\bf P}) 1 Punkt, Variationen ({\bf V}) 1 Punkt,
\item
Anzahl Belegungen, bei denen alle Vierertische besetzt sind
({\bf 4}) 1 Punkt,
Wahrscheinlichkeit ({\bf W}) 1 Punkt.
\end{teilaufgaben}
\end{bewertung}



