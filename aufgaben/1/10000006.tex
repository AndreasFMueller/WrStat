Wieviele zehnstellige Zahlen gibt es, die genau dreimal die Ziffer
3 enthalten?

\begin{loesung}
Die Aufgabe spezifiziert nicht, ob f"uhrende Nullen bei einer solchen
10-stelligen Zahl
vorkommen d"urfen. Wir berechnen daher die Anzahl von $n$-stelligen
Zahlen, die genau $k$ Dreier enth"alt zun"achst unter der Annahme,
dass f"uhrende Nullen erlaubt sind (Anzahl $a(n,k)$) und dann auch
noch unter der Annahme, dass f"uhrende Nullen nicht erlaubt sind.

Um eine solche Zahl zu bilden, m"ussen zun"achst die $k$ Pl"atze
f"ur die Ziffern 3 gew"ahlt werden, dies ist auf $\binom{n}{k}$
Arten m"oglich. Zu jeder solchen Wahl m"ussen jetzt noch die
verbleibenden Pl"atze mit Ziffern $\ne 3$ besetzt werden, dies
geht auf $9^{n-k}$ Arten. Somit gibt es
$$a(n,k)=\binom{n}{k}\cdot9^{n-k}$$
Zahlen der genannten Art.

Wenn wir jetzt f"uhrende Nullen verbieten, gibt es zwei F"alle zu
betrachten. Die erste Ziffer ist eine $3$, dann muss f"ur die
verbleibenden $n-1$ Stellen eine Zahl m"oglicherweise mit f"uhrender
Null gew"ahlt werden, die genau $k-1$ Dreier erh"alt, das ist auf $a(n-1,k-1)$
Arten m"oglich. Die erste Stelle k"onnte aber auch etwas anderes
als eine Drei sein, das geht auf $8$ verschiedene Arten, f"ur den
Rest m"ussen dann $n-1$ Stellen mit genau $k$ Dreiern gew"ahlt
werden, was auf $a(n-1,k)$ Arten geht. Die gesuchte Anzahl ist also
\[
1\cdot a(n-1,k-1) + 8\cdot a(n-1,k)=
\binom{n-1}{k-1}\cdot 9^{n-k}+8\cdot\binom{n-1}{k}\cdot 9^{n-1-k}
\qedhere
\]
\end{loesung}

