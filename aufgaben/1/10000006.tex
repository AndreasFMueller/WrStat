Wieviele zehnstellige Zahlen gibt es, die genau dreimal die Ziffer
3 enthalten?

\thema{Kombinatorik}

\begin{loesung}
Die Aufgabe spezifiziert nicht, ob führende Nullen bei einer solchen
10-stelligen Zahl
vorkommen dürfen. Wir berechnen daher die Anzahl von $n$-stelligen
Zahlen, die genau $k$ Dreier enthält zunächst unter der Annahme,
dass führende Nullen erlaubt sind (Anzahl $a(n,k)$) und dann auch
noch unter der Annahme, dass führende Nullen nicht erlaubt sind.

Um eine solche Zahl zu bilden, müssen zunächst die $k$ Plätze
für die Ziffern 3 gewählt werden, dies ist auf $\binom{n}{k}$
Arten möglich. Zu jeder solchen Wahl müssen jetzt noch die
verbleibenden Plätze mit Ziffern $\ne 3$ besetzt werden, dies
geht auf $9^{n-k}$ Arten. Somit gibt es
$$a(n,k)=\binom{n}{k}\cdot9^{n-k}$$
Zahlen der genannten Art.

Wenn wir jetzt führende Nullen verbieten, gibt es zwei Fälle zu
betrachten. Die erste Ziffer ist eine $3$, dann muss für die
verbleibenden $n-1$ Stellen eine Zahl möglicherweise mit führender
Null gewählt werden, die genau $k-1$ Dreier erhält, das ist auf $a(n-1,k-1)$
Arten möglich. Die erste Stelle könnte aber auch etwas anderes
als eine Drei sein, das geht auf $8$ verschiedene Arten, für den
Rest müssen dann $n-1$ Stellen mit genau $k$ Dreiern gewählt
werden, was auf $a(n-1,k)$ Arten geht. Die gesuchte Anzahl ist also
\[
1\cdot a(n-1,k-1) + 8\cdot a(n-1,k)=
\binom{n-1}{k-1}\cdot 9^{n-k}+8\cdot\binom{n-1}{k}\cdot 9^{n-1-k}
\qedhere
\]
\end{loesung}

