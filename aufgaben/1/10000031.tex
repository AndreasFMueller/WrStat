In einem Zugwagen gibt es 16 Plätze, davon 8 Fenster- und 8 Gangplätze.
Je die Hälfte der Plätze ist in Fahrtrichtung gerichtet, die anderen
fahren mit dem Rücken zur Fahrtrichtung.
Eine Gesellschaft von zwölf Personen will im Zug Platz nehmen.
\begin{teilaufgaben}
\item 
Auf wieviele Arten können die Reisenden die Plätze einnehmen?
\item
Vier der Personen möchten unbedingt einen Fensterplatz haben, den anderen
spielt es keine Rolle.
Auf wieviele Arten können die Reisenden Platz nehmen?
\item
Einer Person wird übel, wenn sie nicht in Fahrtrichtung schauen kann.
Sie verlangt nicht unbedingt nach einem Fensterplatz.
Auf wieviele Arten können sich die Reisenden setzen, wenn diese
Person einen Fensterplatz einnimmt?
\item
Auf wieviele Arten können sich die Fahrgäste setzen, wenn die Person,
der übel wird, keinen Fensterplatz einnimmt?
\item
Auf wieviele Arten können sich die Reisenden setzen, wenn beide Bedingungen
(Fensterplätze und Fahrtrichtung) erfüllt sein sollen?
\end{teilaufgaben}

\begin{loesung}
\begin{teilaufgaben}
\item
Zunächst müssen 12 der 16 Plätze ausgewählt werden, dies ist auf
$\binom{16}{12}$ Arten möglich.
Auf den zwölf gewählten Plätzen können die Fahrgäste auf $12!$
Arten angeordnet werden.
Die Reisenden können die Plätze daher auf
\[
\binom{16}{12}\cdot 12!
=
\frac{16!}{12!\cdot 4!}\cdot 12!
=
\frac{16!}{4!}
=
16\cdot 15\cdot 14\cdot\ldots\cdot 6\cdot 5
=
871782912000
\]
Arten Platz nehmen.
\item
Zunächst werden 4 der 8 Fensterplätze auf $\binom{8}{4}$ Arten gewählt
und die Fensterplatzfahrer in $4!$ möglichen Anordnungen platziert.
Dann werden 8 von 12 Plätzen für die verbleibenden Gäste gewählt und
die Gäste auf $8!$ Arten angeordnet.
So ergeben sich
\[
\binom{8}{4}\cdot 4! \cdot \binom{12}{8}\cdot 8!
=
70 \cdot 24 \cdot 495 \cdot 40320
=
33530112000
\]
mögliche Platzierungen.
\item
Die Person $U$, der übel wird, nimmt einen von 4 Fensterplätzen ein,
die auch in Fahrtrichtung schauen.
Dann werden von den 7 verbleibenden Fensterplätzen 4 für die 
Fensterplatzfahrer gewählt.
Schliesslich werden 7 der verbleibenden 11 Plätze gewählt.
So ergeben sich
\[
4\cdot\binom{7}{4}\cdot 4! \cdot\binom{11}{7}\cdot 7!
=
4\cdot 35 \cdot 24\cdot 330\cdot 5040
=
5588352000
\]
Anordnungen.
\item
Die Person $U$ nimmt einen von 4 Nicht-Fensterplätzen in Fahrtrichtung
ein.
Dann werden 4 Gäste auf die 8 vorhandenen Fensterplätze platziert und
schliesslich die 7 verbleibenden Gäste auf die 11 übrigen Plätze.
Die gesuchte Anzahl ist
\[
4\cdot \binom{8}{4}\cdot 4!\cdot\binom{11}{7}\cdot 7!
=
4\cdot 70\cdot 24\cdot 330\cdot 5040
=
11176704000
\]
\item
Die beiden Situationen der vorgegangenen zwei Teilaufgaben schliessen sich
aus, sie können daher addiert werden.
Es gibt also
\[
5588352000
+
11176704000
=
16765056000
\]
Platzierungen der verlangten Art.
\qedhere
\end{teilaufgaben}
\end{loesung}

\begin{bewertung}
Auswahl-Problem (Binomialkoeffizient) ({\bf W}) 1 Punkt,
jede Teilaufgabe ein Punkt ({\bf A} bis {\bf E}) ein Punkt.
\end{bewertung}
