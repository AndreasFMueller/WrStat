Ein Setzer hat die folgenden sieben Buchstaben zur Verfügung
\[
\texttt{O},\;
\texttt{O},\;
\texttt{T},\;
\texttt{T},\;
\texttt{T},\;
\texttt{W},\;
\texttt{X}
\]
\begin{teilaufgaben}
\item Auf wieviele Arten kann man diese sieben Buchstaben anordnen?
\item Wieviele verschiedene Wörter mit zwei Buchstaben kann man bilden?
\item Wieviele Möglichkeiten gibt es, die sieben Buchstaben in eine Reihe
zu legen, so dass die drei \texttt{T} direkt nebeneinander liegen?
\end{teilaufgaben}

\begin{loesung}
\begin{teilaufgaben}
\item
Es gibt $7!$ Anordnungen von $7$ Buchstaben, aber davon unterscheiden sich
jeweils $2!$ nur durch Vertauschung der beiden \texttt{O} und $3!$ nur
durch Permutation der drei \texttt{T}.
Die Zahl der verschiedenen Reihenfolgen ist daher
\[
\frac{7!}{2!\cdot 3!}=\frac{5040}{2\cdot 6} = 420.
\]
\item
Verlangt man, dass die zwei Buchstaben verschieden sind, dann lassen sich
$\binom{4}{2}=6$ verschiedene Wörter bilden.
Dazu kommen noch die Wörter \texttt{OO} und \texttt{TT} mit zwei gleichen
Buchstaben.
Insgesamt kann man also 8 verschiedene Wörter mit zwei Zeichen bilden.
\item
Es kommt offenbar darauf an, wo die die Folge der \texttt{T} beginnt.
Dies kann beim schon beim ersten Zeichen sein, spätestens aber beim vierten.
Es gibt also vier Möglichkeiten, wie die \texttt{T} platziert sein können.
Die verbleibenden 4 Zeichen können auf $4!$ Arten platziert werden.
Doch davon sind jeweils zwei identisch, da sie durch Vertauschung der
beiden \texttt{O} auseinander hervorgehen.
Somit bleiben 
\[
4\cdot\frac{4!}{2!}=4\cdot \frac{24}{2}=4\cdot 12=48
\]
Möglichkeiten.
\end{teilaufgaben}
\end{loesung}


