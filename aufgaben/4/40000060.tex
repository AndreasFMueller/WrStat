Aus einem Kartenspiel mit 52 Karten wird eine Karte gezogen und dann
die Auszahlung $X(\omega)$ wie folgt berechnet:
\[
X = m(\omega)\cdot w(\omega),
\]
wobei $w(\omega)$ der Wert der Karte ist (Das Ass hat den Wert 1, der König
den Wert 13), und $m(\omega)$ ein Multiplikator, der von der Farbe abhängt:
\def\V#1{\raisebox{-0.1cm}{\includeagraphics[]{#1}}}
\begin{center}
\begin{tabular}{|>{$}l<{$}|cccc|}
\hline
\omega & \V{heartsuit.pdf} & \V{spadesuit.pdf} & \V{diamondsuit.pdf} & \V{clubsuit.pdf} \\
\hline
m      &     1     &          2      &     3      &       4       \\
\hline
\end{tabular}
\end{center}
Bestimmen Sie das Ereignis
\[
A
=
\{ X \ge 17 \}
\]
und seine Wahrscheinlichkeit.

\begin{loesung}
Für eine \V{heartsuit.pdf}-Karte wird der Wert nie mindestens 17, diese
Karten kommen also in $A$ nicht vor.
Die \V{spadesuit.pdf}-Kartenwerte werden mit 2 multipliziert, der Kartenwert
muss daher mindestens 9 sein.
Bei den \V{diamondsuit.pdf}-Karten ist der minimale Wert 6 und bei
den \V{clubsuit.pdf}-Karten ist er 5.
Das Ereignis ist daher:
\begin{center}
\begin{tabular}{c|>{\tt}c>{\tt}c>{\tt}c>{\tt}c>{\tt}c>{\tt}c>{\tt}c>{\tt}c>{\tt}c>{\tt}c>{\tt}c>{\tt}c>{\tt}c|}
\hline
\V{heartsuit.pdf}   &  &  &  &  &  &  &  &  &   &    &   &   &   \\
\V{spadesuit.pdf}   &  &  &  &  &  &  &  &  & 9 & 10 & J & Q & K \\
\V{diamondsuit.pdf} &  &  &  &  &  & 6& 7& 8& 9 & 10 & J & Q & K \\
\V{clubsuit.pdf}    &  &  &  &  & 5& 6& 7& 8& 9 & 10 & J & Q & K \\
\hline
\end{tabular}
\end{center}
Dies sind insgesamt  $0 + 5 + 8 + 9 = 22$ Karten.
Das Ereignis hat also die Wahrscheinlichkeit $P(A)=22/52 = 0.4231$.
\end{loesung}


