Ein Gl"ucksspiel wird wie folgt gespielt. Aus einem Kartenspiel
mit 52 Karten werden zuf"allig zwei Karten gezogen. Ist eine
davon eine Bildkarte (K"onig, Dame, Bube oder Ass), oder sind die
Karten von verschiedener Farbe, hat der Spieler verloren. Nur wenn
die Karten die gleiche Farbe haben, und ausserdem beide Zahlkarten
sind, wird als Gewinn die Summe der von den Karten gezeigten Zahlen
ausbezahlt.
\begin{teilaufgaben}
\item Wie gross ist die Wahrscheinlichkeit, etwas zu gewinnen.
\item Wie gross ist der erwartete Gewinn?
\end{teilaufgaben}

\begin{loesung}
\begin{teilaufgaben}
\item
Elementarereignisse sind Paare von Karten, wobei Paare mit zwei
gleichen Karten nicht m"oglich sind. Es gibt $52^2-52=2652$ solche
Paare, sie sind alle gleich wahrscheinlich. Nur die Paare mit gleicher
Farbe gewinnen, von jeder Farbe sind dies $9^2-9=72$ Paare, also
insgesamt $4\cdot 72=288$ Paare. Die Wahrscheinlichkeit zu gewinnen
ist daher
\[
p=\frac{288}{2652}=\frac{24}{221}=0.108597.
\]

Etwas formaler kann man auch wie folgt argumentieren.
Sei $G$ das Ereignis, zu gewinnen, und $Z$ das Ereignis, als erste
Karte eine Zahlkarte zu ziehen.
Die Wahrscheinlichkeit,
im ersten Zug eine Zahlkarte zu ziehen, ist $P(Z)=\frac{36}{52}$.
Die Gewinnwahrscheinlichkeit ist
\[
P(G)=
P(G|Z)P(Z)
+P(G|\overline{Z})P(\overline{Z}).
\]
Da man aber bestimmt nicht mehr gewinnt, wenn man im ersten Zug
keine Zahlkarte zieht, ist $P(G|\overline{Z})=0$, man muss
also nur den ersten Term weiter betrachten.
Damit man
gewinnt, muss man aber im zweiten Zug ebenfalls eine Zahlkarte ziehen,
die Wahrscheinlichkeit zu gewinnen (Ereigniss $G$) ist also
\[
P(G)=P(\text{Zahlkarte gleicher Farbe}|Z)P(Z)
\]
Wenn bereits eine Zahlkarte gezogen ist, bleiben nur noch $8$
Zahlkarten der gleichen Farbe im Spiel. Nur bei $8$ der
verbleiben $51$ Karten kann man also gewinnen, $P(G|Z)=\frac{8}{51}$.
Die Gewinnwahrscheinlichkeit wird damit
\[
P(G)=
P(G|Z)P(Z)
=\frac{36}{52}\cdot\frac{8}{51}= \frac{288}{2652}
\]
wie vorhin.
\item
Der erwartete Gewinn ist
\begin{align}
E(X)&=4\sum_{\myatop{i,j=2}{i \ne j}}^{10} \text{``Gewinn bei Karten $(i,j)$''}
\cdot\text{``Wahrscheinlichkeit von $(i,j)$''}
\notag
\\
&=
4\sum_{\myatop{i,j=2}{i\ne j}}^{10}(i+j)\cdot\frac1{2652}
=
\frac{1}{663}\sum_{\myatop{i,j=2}{i\ne j}}^{10}(i+j)
\label{gewinn}
\end{align}
W"aren die F"alle $i=j$ nicht ausgeschlossen, w"are die Summe
\begin{align}
\sum_{i,j=2}^{10}(i+j)
&=
\sum_{i=2}^{10}\sum_{j=2}^{10}(i+j)
\notag
\\
&=
\sum_{i=2}^{10}\biggl(9i + \sum_{j=2}^{10}j\biggr)
=
9\sum_{i=2}^{10}i +9\sum_{j=2}^{10}j
\label{umformung}
\\
&=
18 \sum_{i=2}^{10}
\notag
=
18 \biggl(\sum_{i=1}^{10}-1\biggr)
=18\cdot 54= 972
\notag
\end{align}
Die Umformungen in (\ref{umformung}) verwenden, dass der eine
Summand nicht Summationsindex abh"angt, also einfach so oft
addiert wird, wie die Summe Terme hat, also $9$ mal.
Aus den F"allen $i=j$ sind darin aber noch
m"ogliche Gewinne im Umfang von
\[
\sum_{i=2}^{10}2i=2\biggl(\sum_{i=1}^{10}i-1\biggr)=2\cdot 54=108
\]
eingeschlossen, die tats"achlich m"ogliche Gewinnsumme ist
f"ur eine Farbe ist also $972-108=864$. F"ur alle vier Farben
wird der erwartete Gewinn also durch Einsetzen in (\ref{gewinn})
\[
E(X)=\frac{864}{663}=
1.30316742081447963800
.
\]

Man kann diese Gewinne auch etwas weniger abstrakt mit einer Tabelle
addieren. Tabelle~\ref{tabelle} zeigt die Gewinnsituationen f"ur die
Karten einer einzelnen Farbe. Jedes Feld der Tabelle steht f"ur eine gewinnendes
Paar. Ist das Feld leer, gibt es nichts zu gewinnen.
\begin{table}
\begin{center}
\begin{tabular}{c|c|c|c|c|c|c|c|c|c|c|c|c|c|}
  & 2& 3& 4& 5& 6& 7& 8& 9&10& J& D& K& A\\
\hline
2 &  & 5& 6& 7& 8& 9&10&11&12&  &  &  &  \\
3 & 5&  & 7& 8& 9&10&11&12&13&  &  &  &  \\
4 & 6& 7&  & 9&10&11&12&13&14&  &  &  &  \\
5 & 7& 8& 9&  &11&12&13&14&15&  &  &  &  \\
6 & 8& 9&10&11&  &13&14&15&16&  &  &  &  \\
7 & 9&10&11&12&13&  &15&16&17&  &  &  &  \\
8 &10&11&12&13&14&15&  &17&18&  &  &  &  \\
9 &11&12&13&14&15&16&17&  &19&  &  &  &  \\
10&12&13&14&15&16&17&18&19&  &  &  &  &  \\
J &  &  &  &  &  &  &  &  &  &  &  &  &  \\
D &  &  &  &  &  &  &  &  &  &  &  &  &  \\
K &  &  &  &  &  &  &  &  &  &  &  &  &  \\
A &  &  &  &  &  &  &  &  &  &  &  &  &  \\
\hline
\end{tabular}
\end{center}
\caption{Summierung der Gewinne f"ur eine einzelne Kartenfarbe
in Aufgabe 3\label{tabelle}}
\end{table}
Zur Summation der Gewinne kann man z"ahlen, wie oft jeder
m"ogliche Gewinn vorkommt, wie dies in Tabelle~\ref{haeufigkeiten}
durchgef"uhrt ist.
\begin{table}
\begin{center}
\begin{tabular}{|l|c|c|c|c|c|c|c|c|c|c|c|c|c|c|c|}
\hline
Gewinn    &5&6&7&8&9&10&11&12&13&14&15&16&17&18&19\\
\hline
H"aufikeit&2&2&4&4&6& 6& 8& 8& 8& 6& 6& 4& 4& 2& 2\\
\hline
\end{tabular}
\end{center}
\caption{H"aufigkeiten der einzelnen Summen\label{haeufigkeiten}}
\end{table}
Die gesuchte Summe ist
\[
2\cdot(5+6+18+19)
+
4\cdot(7+8+16+17)
+
6\cdot(9+10+14+15)
+
8\cdot(11+12+13)
\]
In den ersten drei Klammerausdr"ucken ergeben symmetrisch
angeordnete Terme jeweils die Summe $24$, jeder dieser
drei Klammerausdr"ucke hat also den Wert $48$. Die Summe ist
damit insgesamt
\[
(2+4+6)\cdot 48+8\cdot 36=12\cdot 48+8\cdot 36
=
%12*48+8*36
864,
\]
wie oben bereits gefunden.

Will man nur die Gewinnerwartung wissen f"ur den Fall, dass
man tats"achlich gewonnen hat, muss man noch durch die
Wahrscheinlichkeit dividieren, "uberhaupt zu gewinnen:
\[
E(X)=E(X|G)P(G)\quad\Rightarrow\quad E(X|G)=\frac{E(X)}{P(G)}.
\]
Durch Einsetzen der bekannten Werte f"ur $E(X)$ und $P(G)$
ergibt sich:
\[
E(X|G)
=
\frac{864}{663}\cdot\frac{2652}{288}
=
\frac{864}{663}\cdot\frac{663}{72}
=
\frac{864}{72}=
%864/72
12.
\]
Dies ist nat"urlich genau der mittlere Wert der m"oglichen
Gewinne, wie man aus der Tabelle~\ref{haeufigkeiten}
sehr sch"on ablesen kann. Dies ist aber nicht weiter
"uberraschend, denn bei der Berechnung von $E(X|G)$
schliesst man ja alle F"alle aus, in denen man nicht
gewinnt, es bleiben nur die F"alle "ubrig, die durch
die Tabelle \ref{haeufigkeiten} beschrieben werden.

Relativ h"aufig war der Fehler, mit den Zahlen $1$ bis $9$
zu arbeiten statt mit $2$ bis $10$. Die Gewinnerwartung
im Falle eines Gewinns verschiebt sich dadurch zu
$E(X|G)=10$, der erwartet Gewinn ist also in diesem Fall
\[
E(X)=E(X|G)P(G)=10\cdot\frac{24}{221}=\frac{240}{221}=
1.08597285067873303167.
\]
\end{teilaufgaben}
\end{loesung}

