Wann startet das James Webb Space Telescope?
Im Laufe des Projekts des James Webb Space Telescope kam es immer 
wieder zu Verzögerungen.
Das geplante Startdatum wurde im Laufe der Zeit immer später, wie dies
die in der folgenden Graphik zusammengefasst ist
(Quelle \url{https://xkcd.com/2014/}):
\begin{center}
\includeagraphics[width=10cm]{jwstdelays.png}
\end{center}
\begin{teilaufgaben}
\item
Verwenden Sie die folgenden Daten, um ein Modell für das geplante
Startdatum in Abhängigkeit von der Zeit aufzustellen.
\begin{center}
\begin{tabular}{|>{$}c<{$}|>{$}r<{$}|>{$}r<{$}|}
\hline
i&\text{Prognosedatum}&\text{Startdatum}\\
\hline
1&2002& 2010\\
2&2006& 2014\\
3&2010& 2016\\
4&2014& 2019\\
5&2018& 2020\\
\hline
\end{tabular}
\end{center}
\item 
Beurteilen Sie die Qualität Ihres Modells.
\item
Wann wird nach Ihrem Modell das JWST gestartet?
\end{teilaufgaben}

%https://youtu.be/OVjbelWgGkw

\begin{hinweis}
Ignorieren Sie bei Ihrer Lösung, dass das JWST am 25.~Dezember tatsächlich
gestartet und inzwischen auch im Lagrange-Punkt $L_2$ angekommen ist.
\end{hinweis}

\begin{loesung}
\begin{teilaufgaben}
\item
Zur Berechnung der Parameter eines linearen Modells verwendet man die Tabelle
\begin{center}
\hspace*{-1cm}
\begin{tabular}{|>{$}r<{$}|>{$}r<{$}>{$}r<{$}|>{$}r<{$}>{$}r<{$}|>{$}r<{$}|}
\hline
i&      x_i&      y_i\phantom{.0}&    x_i^2&     y_i^2\phantom{.0}&   x_iy_i \\
\hline
1&     2002&     2010\phantom{.0}&  4008004&   4040100\phantom{.0}&   4024020\\
2&     2006&     2014\phantom{.0}&  4024036&   4056196\phantom{.0}&   4040084\\
3&     2010&     2016\phantom{.0}&  4040100&   4064256\phantom{.0}&   4052160\\
4&     2014&     2019\phantom{.0}&  4056196&   4076361\phantom{.0}&   4066266\\
5&     2018&     2020\phantom{.0}&  4072324&   4080400\phantom{.0}&   4076360\\
\hline
 &\sum x_i = 10050& \sum y_i = 10079\phantom{.0}& \sum x_i^2 = 20200660& \sum y_i^2 = 20317313\phantom{.0}& \sum x_iy_i = 20258890\\
\hline
 & E(X)= 2010& E(Y)= 2015.8& E(X^2) = 4040132& E(Y^2) =4063462.6& E(XY) = 4051778\\
\hline
\end{tabular}
\end{center}
Die Kovarianzen und Varianzen sind
\[
\begin{aligned}
\operatorname{cov}(X,Y)
&=
20,
&
&&
\operatorname{var}(X)
&=
32
\\
&&
&\text{und}&
\operatorname{var}(Y)
&=
12.96.
\end{aligned}
\]
Daraus kann man jetzt die Parameter der Regressionsgeraden ableiten:
\begin{align*}
a
&=
\frac{\operatorname{cov}(X,Y)}{\operatorname{var}(X)}
=
\frac{20}{32}
=
\frac{5}{8} 
=
0.625,
\\
b
&=
E(Y) - aE(X)
=
2015.8 - 0.625 \cdot 2010
=
759.55.
\end{align*}
\item
Zur Beurteilung des Modells muss der Regressionskoeffizient
\[
r
=
\frac{
\operatorname{cov}(X,Y)
}{
\sqrt{\operatorname{var}(X)\operatorname{var}(Y)}
}
=
\frac{20}{\sqrt{32\cdot 0.625}}
\approx
0.9821
\]
bestimmen.
Da $r$ sehr nahe bei $1$ ist, kann man das Modell als gut beurteilen.
\item
Gesucht ist der Zeitpunkt, wo sich die Vorhersage des Modells und die
aktuelle Zeit treffen, also eine Lösung der Gleichung
\[
\begin{aligned}
&&ax+b&=x
\\
&\Rightarrow&(a-1)x&=-b
\\
&\Rightarrow& x&=\frac{b}{1-a}.
\end{aligned}
\]
Durch Einsetzen der oben ermittelten Zahlen findet man für den zu
erwartenden Startzeitpunkt
\[
x
=
\frac{b}{1-a}
=
\frac{759.55}{0.375}
=
2025.4666.
\]
Nach diesem Modell wird das JWST also im Jahr 2025 gestartet.
\qedhere
\end{teilaufgaben}
\end{loesung}

\begin{diskussion}Mehr zu den Verzögerungen des
James Webb Space Telescope im Sixty-Symbols Video
\url{https://youtu.be/OVjbelWgGkw} mit Prof.~Mike Merryfield.
\end{diskussion}

\begin{bewertung}
Lineare Regression ({\bf LR}) 1 Punkt,
Berechnungsmethode ({\bf M}) 1 Punkt,
Wert für $a$ ({\bf A}) 1 Punkt,
Wert für $b$ ({\bf B}) 1 Punkt,
Regressionskoeffizient $r^2$ und Beurteilung ({\bf R}) 1 Punkt,
Schätzung für Startzeitpunkt ({\bf S}) 1 Punkt.
\end{bewertung}

%t =
%
%   2002   2010
%   2006   2014
%   2010   2016
%   2014   2019
%   2018   2020
%
%t =
%
%      2002      2010   4008004   4040100   4024020
%      2006      2014   4024036   4056196   4040084
%      2010      2016   4040100   4064256   4052160
%      2014      2019   4056196   4076361   4066266
%      2018      2020   4072324   4080400   4076360
%
%s =
%
%      10050      10079   20200660   20317313   20258890
%
%e =
%
% Columns 1 through 3:
%
%   2.010000000000000e+03   2.015800000000000e+03   4.040132000000000e+06
%
% Columns 4 and 5:
%
%   4.063462600000000e+06   4.051778000000000e+06
%
%covXY = 20
%varX = 32
%varY = 12.96000000042841
%a = 0.625000000000000
%b = 759.5500000000000
%r = 0.982092751631750
%JWST = 2025.466666666666
%
