Ein autonomer Sensor soll w"ahrend des Tages
in regelm"assigen Zeitabst"anden von einer
Stunde Messdaten an einen zentralen Server senden.
Damit die Zeitabst"ande gleich bleiben, hat der Sensor einerseits
eine Uhr, andererseits erf"ahrt er bei jeder Daten"ubermittlung
die als zuverl"assig angesehene Zeit des
Servers\footnote{Die Zeit wird im Date-Feld einer HTTP-Antwort "ubermittelt.}.
Durch Vergleich mit seiner eigenen Uhr kann der Sensor nicht nur die
die Zeit ermitteln, sondern auch ausrechnen, ob seine Uhr zu schnell oder
zu langsam l"auft. W"ahrend des Nachmittags stellt der Sensor folgende
Zeitdifferenzen fest:
\begin{center}
\begin{tabular}{|l|c|c|c|c|c|}
\hline
Zeit (Sensoruhr)&14:00:00&15:00:00&16:00:00&17:00:00&18:00:00\\
\hline
Serverzeit&14:00:23&15:01:03&16:01:29&17:01:58&18:02:21\\
Abweichung [s]&23&63&89&118&141\\
\hline
\end{tabular}
\end{center}
Offenbar l"auft die Sensor-Uhr etwas zu langsam.
\begin{teilaufgaben}
\item Welche Zeit hat der Server in dem Moment, da der Sensor 07:00 hat.
\item Die n"achste Messung soll m"oglichst genau um 07:00 stattfinden.
Zu welcher Sensorzeit muss sie durchgef"uhrt werden?
\item Beurteilen Sie die Qualit"at Ihrer Approximation.
\item L"osen Sie das Problem auch unter Zuhilfenahme von R.
\end{teilaufgaben}

\begin{figure}
\begin{center}
\includeagraphics[width=0.6\hsize]{sensor.pdf}
\end{center}
\caption{Lineare Regression f"ur das Sensorproblem}
\end{figure}
\begin{loesung}
Wir f"uhren eine lineare Regression durch. Dazu muss zun"achst eine
geeignete Skala f"ur $X$ und $Y$ gew"ahlt werden. Dies ist auf
verschiedene Weise m"oglich, die unten dargestellte L"osung ist nur
eine von vielen m"oglichen L"osungen.

Als $X$-Variable verwenden wir
die Sensorzeit in Stunden, als $Y$-Variable die Abweichung in Sekunden.
\begin{align*}
\sum X_i&=80
\\
\sum X_i^2&=1290
\\
\sum Y_i&=434
\\
\sum Y_i^2&=46224
\\
\sum X_iY_i&=7235
\end{align*}
Nach den Formeln zur Regression  im Skript (Abschnitt 4.3.1) ist
\begin{align*}
a&=\frac{n\sum X_iY_i-\sum X_i\sum Y_i}{n\sum X_i^2-(\sum X_i)^2}
\\
&=\frac{5\cdot 7235-80\cdot 434}{5\cdot 1290-80^2}=\frac{1455}{50}=29.1
\\
b&=\frac1n\sum Y_i -a\frac1n\sum X_i
\\
&=\frac15\cdot 434-a\frac15\cdot 80=-378.8
\end{align*}
Der Zusammenhang zwischen Sensorzeit $X$ und Abweichung $Y$ ist also
\[
Y=-378.8+29.1 X
\]
\begin{teilaufgaben}
\item Zur Sensorzeit 07:00 am n"achsten Morgen entspricht $X=31$, die
Abweichung ist dann
$Y=523.3$. Dies entspricht einer Serverzeit von 07:08:43.
\item
Um die richtige Sensorzeit f"ur eine Messung um 07:00 Serverzeit zu bekommen,
brauchen wir die Serverzeit in Abh"angigkeit von der Sensorzeit. Diese erhalten
wir, indem wir die Abweichung in Stunden umrechnen und zur Sensorzeit
hinzuaddieren:
\[
t=X+\frac1{3600}(-378.8+29.1X)
\]
Nun suchen wir $X$ so, dass $t=31$ wird:
\begin{align*}
31&=X+\frac1{3600}(-378.8+29.1X)
\\
31\cdot 3600 +378.8&=(3600 + 29.1)X
\\
X&=\frac{31\cdot 3600 + 378.8}{3629.1}=30.8558044694
\end{align*}
Dies entspricht einer Sensorzeit von 06:51:21.
\item
Um die Qualit"at der Approximation beurteilen zu k"onnen, muss
der Regressionskoeffizient bestimmt werden und man muss pr"ufen,
ob er nahe bei $1$ ist. Der Regressionskoeffizient ist
\begin{align*}
r&=\frac{\operatorname{cov}(X,Y)}{\sqrt{\operatorname{var}(X)\operatorname{var}(Y)}}
=a\frac{\sqrt{\operatorname{var}(X)}}{\sqrt{\operatorname{var}(Y)}}
\\
&=
29.1 \sqrt{\frac{\frac{1290}{5}-\frac{80^2}{5^2}}{\frac{46224}{5}-\frac{434^2}{5^2}}}
=29.1 \sqrt{\frac{2}{1710.56}}= 0.995
\end{align*}
Da $r$ sehr nahe bei $1$ ist, liegt eine sehr gute Approxmation vor.
\item
Um das Problem mit R zu l"osen, muss man zun"achst die Daten in
elektronischer Form zur Verfuegung haben, zum Beispiel in Form
eines Textfiles {\tt sensor.csv} mit dem Inhalt
\verbatimainput{sensor.csv}
Dann kann man diese Daten in ein Datenframe einlesen und das lineare
Modell erzeugen. Bei der Berechnung des linearen Modelles muss als
erstens Argument angegeben werden, welche Variable im Datenframe von
welcher anderen Variable abh"angt:
\verbatimainput{sensor.R}
\qedhere
\end{teilaufgaben}
\end{loesung}

