Ein autonomer Sensor soll während des Tages
in regelmässigen Zeitabständen von einer
Stunde Messdaten an einen zentralen Server senden.
Damit die Zeitabstände gleich bleiben, hat der Sensor einerseits
eine Uhr, andererseits erfährt er bei jeder Datenübermittlung
die als zuverlässig angesehene Zeit des
Servers\footnote{Die Zeit wird im Date-Feld einer HTTP-Antwort übermittelt.}.
Durch Vergleich mit seiner eigenen Uhr kann der Sensor nicht nur die
die Zeit ermitteln, sondern auch ausrechnen, ob seine Uhr zu schnell oder
zu langsam läuft. Während des Nachmittags stellt der Sensor folgende
Zeitdifferenzen fest:
\begin{center}
\begin{tabular}{|l|c|c|c|c|c|}
\hline
Zeit (Sensoruhr)&14:00:00&15:00:00&16:00:00&17:00:00&18:00:00\\
\hline
Serverzeit&14:00:23&15:01:03&16:01:29&17:01:58&18:02:21\\
Abweichung [s]&23&63&89&118&141\\
\hline
\end{tabular}
\end{center}
Offenbar läuft die Sensor-Uhr etwas zu langsam.
\begin{teilaufgaben}
\item Welche Zeit hat der Server in dem Moment, da der Sensor 07:00 hat.
\item Die nächste Messung soll möglichst genau um 07:00 stattfinden.
Zu welcher Sensorzeit muss sie durchgeführt werden?
\item Beurteilen Sie die Qualität Ihrer Approximation.
\item Lösen Sie das Problem auch unter Zuhilfenahme von R.
\end{teilaufgaben}

\thema{lineare Regression}

\begin{figure}
\begin{center}
\includeagraphics[width=0.6\hsize]{sensor.pdf}
\end{center}
\caption{Lineare Regression für das Sensorproblem}
\end{figure}
\begin{loesung}
Wir führen eine lineare Regression durch. Dazu muss zunächst eine
geeignete Skala für $X$ und $Y$ gewählt werden. Dies ist auf
verschiedene Weise möglich, die unten dargestellte Lösung ist nur
eine von vielen möglichen Lösungen.

Als $X$-Variable verwenden wir
die Sensorzeit in Stunden, als $Y$-Variable die Abweichung in Sekunden.
\begin{align*}
\sum X_i&=80
\\
\sum X_i^2&=1290
\\
\sum Y_i&=434
\\
\sum Y_i^2&=46224
\\
\sum X_iY_i&=7235
\end{align*}
Nach den Formeln zur Regression  im Skript (Abschnitt 4.3.1) ist
\begin{align*}
a&=\frac{n\sum X_iY_i-\sum X_i\sum Y_i}{n\sum X_i^2-(\sum X_i)^2}
\\
&=\frac{5\cdot 7235-80\cdot 434}{5\cdot 1290-80^2}=\frac{1455}{50}=29.1
\\
b&=\frac1n\sum Y_i -a\frac1n\sum X_i
\\
&=\frac15\cdot 434-a\frac15\cdot 80=-378.8
\end{align*}
Der Zusammenhang zwischen Sensorzeit $X$ und Abweichung $Y$ ist also
\[
Y=-378.8+29.1 X
\]
\begin{teilaufgaben}
\item Zur Sensorzeit 07:00 am nächsten Morgen entspricht $X=31$, die
Abweichung ist dann
$Y=523.3$. Dies entspricht einer Serverzeit von 07:08:43.
\item
Um die richtige Sensorzeit für eine Messung um 07:00 Serverzeit zu bekommen,
brauchen wir die Serverzeit in Abhängigkeit von der Sensorzeit. Diese erhalten
wir, indem wir die Abweichung in Stunden umrechnen und zur Sensorzeit
hinzuaddieren:
\[
t=X+\frac1{3600}(-378.8+29.1X)
\]
Nun suchen wir $X$ so, dass $t=31$ wird:
\begin{align*}
31&=X+\frac1{3600}(-378.8+29.1X)
\\
31\cdot 3600 +378.8&=(3600 + 29.1)X
\\
X&=\frac{31\cdot 3600 + 378.8}{3629.1}=30.8558044694
\end{align*}
Dies entspricht einer Sensorzeit von 06:51:21.
\item
Um die Qualität der Approximation beurteilen zu können, muss
der Regressionskoeffizient bestimmt werden und man muss prüfen,
ob er nahe bei $1$ ist. Der Regressionskoeffizient ist
\begin{align*}
r&=\frac{\operatorname{cov}(X,Y)}{\sqrt{\operatorname{var}(X)\operatorname{var}(Y)}}
=a\frac{\sqrt{\operatorname{var}(X)}}{\sqrt{\operatorname{var}(Y)}}
\\
&=
29.1 \sqrt{\frac{\frac{1290}{5}-\frac{80^2}{5^2}}{\frac{46224}{5}-\frac{434^2}{5^2}}}
=29.1 \sqrt{\frac{2}{1710.56}}= 0.995
\end{align*}
Da $r$ sehr nahe bei $1$ ist, liegt eine sehr gute Approxmation vor.
\item
Um das Problem mit R zu lösen, muss man zunächst die Daten in
elektronischer Form zur Verfuegung haben, zum Beispiel in Form
eines Textfiles {\tt sensor.csv} mit dem Inhalt
\verbatimainput{sensor.csv}
Dann kann man diese Daten in ein Datenframe einlesen und das lineare
Modell erzeugen. Bei der Berechnung des linearen Modelles muss als
erstens Argument angegeben werden, welche Variable im Datenframe von
welcher anderen Variable abhängt:
\verbatimainput{sensor.R}
\qedhere
\end{teilaufgaben}
\end{loesung}

