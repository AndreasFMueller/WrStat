Ein System soll Laboranten ihre Arbeit erleichtern und Pulver auf mg genau
automatisch abwägen.
Zur Verfügung steht ein steuerbares Pulverförderungsgerät und eine Waage.
Allerdings lässt sich das Pulverförderungsgerät nicht sofort stoppen,
es läuft immer noch genau eine Sekunde weiter.
Die Förderungsgeschwindigkeit ist jedoch in jedem Fördervorgang
konstant, kann aber zwischen Vorgängen und zwischen verschiedenen
Pulvern varieren. Das Abwägesystem misst daher laufend die bereits 
geförderte Menge. Folgende Massen wurden gemessen:
\begin{center}
\begin{tabular}{|>{$}c<{$}|>{$}c<{$}|}
\hline
t\text{[s]}&m\text{[mg]}\\
\hline
0&1\\
1&12\\
2&20\\
3&31\\
4&39\\
\hline
\end{tabular}
\end{center}
Wann soll das Fördergerät angehalten werden, um möglichst genau
60\,mg des Pulvers abzuwägen?

\thema{lineare Regression}

\begin{loesung}
Mit linearer Regression kann der lineare Zusammenhang zwischen der Förderzeit
$t$ und der Fördermenge $m$ gefunden werden. Dazu müssen verschiedene Summen
von Werten von $t$ und $m$ und ihren Produkten gebildet werden.
\begin{center}
\begin{tabular}{|>{$}r<{$}>{$}r<{$}|>{$}r<{$}>{$}r<{$}|>{$}r<{$}|}
\hline
 t&  m&t^2& m^2& tm\\
\hline
 0&  1&  0&   1&  0\\
 1& 12&  1& 144& 12\\
 2& 20&  4& 400& 40\\
 3& 31&  9& 961& 93\\
 4& 39& 16&1521&156\\
\hline
10&103& 30&3027&301\\
\hline
\end{tabular}
\end{center}
Daraus kann man Steigung einer optimalen linearen Approximation $M=aT+b$
bestimmen mit Hilfe der bekannten Formeln:
\begin{align*}
a&=\frac{n\sum tm -\sum t\sum m}{n\sum t^2-(\sum t)^2}
=\frac{5\cdot 301-10\cdot 103}{5\cdot 30-10^2}
=\frac{1505-1030}{150-100}
=\frac{475}{50}=9.5\\
b&=\frac15\sum m-a\frac15\sum t
=\frac15\cdot 103-9.5\cdot \frac15\cdot 10=1.6
\end{align*}
Die beste Approximation für $m$ ist also $m=9.5t+1.6$. Um die Masse
60mg zu erreichen, muss der Fördermechanismus zur Zeit $t$ aufhören
zu fördern, also wenn
\begin{align*}
60&=9.5t+1.6\\
t&=6.147
\end{align*}
Da der Fördermechanismus jeweils noch eine Sekunde weiterläuft, muss
er zur Zeit $t=5.147\text{s}$ angehalten werden.
\end{loesung}


\begin{bewertung}
Lineare Regression (\textbf{L}) 1 Punkt,
Berechnung von Summen (\textbf{R}) 2 Punkte,
Berechung von $a$ (\textbf{A}) und $b$ (\textbf{B}) je 1 Punkt,
Berechnung der Stoppzeit (\textbf{T}) 1 Punkt.
\end{bewertung}



