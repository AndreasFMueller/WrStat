Ein System soll Laboranten ihre Arbeit erleichtern und Pulver auf mg genau
automatisch abw"agen.
Zur Verf"ugung steht ein steuerbares Pulverf"orderungsger"at und eine Waage.
Allerdings l"asst sich das Pulverf"orderungsger"at nicht sofort stoppen,
es l"auft immer noch genau eine Sekunde weiter.
Die F"orderungsgeschwindigkeit ist jedoch in jedem F"ordervorgang
konstant, kann aber zwischen Vorg"angen und zwischen verschiedenen
Pulvern varieren. Das Abw"agesystem misst daher laufend die bereits 
gef"orderte Menge. Folgende Massen wurden gemessen:
\begin{center}
\begin{tabular}{|>{$}c<{$}|>{$}c<{$}|}
\hline
t\text{[s]}&m\text{[mg]}\\
\hline
0&1\\
1&12\\
2&20\\
3&31\\
4&39\\
\hline
\end{tabular}
\end{center}
Wann soll das F"orderger"at angehalten werden, um m"oglichst genau
60\,mg des Pulvers abzuw"agen?

\begin{loesung}
Mit linearer Regression kann der lineare Zusammenhang zwischen der F"orderzeit
$t$ und der F"ordermenge $m$ gefunden werden. Dazu m"ussen verschiedene Summen
von Werten von $t$ und $m$ und ihren Produkten gebildet werden.
\begin{center}
\begin{tabular}{|>{$}r<{$}>{$}r<{$}|>{$}r<{$}>{$}r<{$}|>{$}r<{$}|}
\hline
 t&  m&t^2& m^2& tm\\
\hline
 0&  1&  0&   1&  0\\
 1& 12&  1& 144& 12\\
 2& 20&  4& 400& 40\\
 3& 31&  9& 961& 93\\
 4& 39& 16&1521&156\\
\hline
10&103& 30&3027&301\\
\hline
\end{tabular}
\end{center}
Daraus kann man Steigung einer optimalen linearen Approximation $M=aT+b$
bestimmen mit Hilfe der bekannten Formeln:
\begin{align*}
a&=\frac{n\sum tm -\sum t\sum m}{n\sum t^2-(\sum t)^2}
=\frac{5\cdot 301-10\cdot 103}{5\cdot 30-10^2}
=\frac{1505-1030}{150-100}
=\frac{475}{50}=9.5\\
b&=\frac15\sum m-a\frac15\sum t
=\frac15\cdot 103-9.5\cdot \frac15\cdot 10=1.6
\end{align*}
Die beste Approximation f"ur $m$ ist also $m=9.5t+1.6$. Um die Masse
60mg zu erreichen, muss der F"ordermechanismus zur Zeit $t$ aufh"oren
zu f"ordern, also wenn
\begin{align*}
60&=9.5t+1.6\\
t&=6.147
\end{align*}
Da der F"ordermechanismus jeweils noch eine Sekunde weiterl"auft, muss
er zur Zeit $t=5.147\text{s}$ angehalten werden.
\end{loesung}


\begin{bewertung}
Lineare Regression (\textbf{L}) 1 Punkt,
Berechnung von Summen (\textbf{R}) 2 Punkte,
Berechung von $a$ (\textbf{A}) und $b$ (\textbf{B}) je 1 Punkt,
Berechnung der Stoppzeit (\textbf{T}) 1 Punkt.
\end{bewertung}



