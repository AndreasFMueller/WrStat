Die Zeitschrift {\em The Economist} hat 1986 den Big Mac Index geschaffen,
angeblich um die Theorie der Wechselkurse ebenso ``leicht verdaulich''
zu machen wie das namensgebende Produkt.
Unter anderem wird regelm"assig die Abh"angigkeit des Big Mac Preises
vom Bruttoinlandprodukt (BIP) pro Person ermittelt und publiziert.
\begin{center}
\begin{tabular}{|l|r|r|}
\hline
Land&BIP pro Person USD&Big Mac Preis USD\\
\hline
Norwegen   &96930&5.21\\
USA        &54370&4.93\\
Euro Raum  &40001&4.00\\
Indien     & 1608&1.90\\
\hline
\end{tabular}
\end{center}
Es wird postuliert, dass es einen Zusammenhang zwischen 
BIP pro Person und Big Mac Preis g"abe.

\begin{teilaufgaben}
\item Finden Sie eine m"oglichst gute Approximation f"ur so einen Zusammenhang.
\item Die Schweiz hat eine BIP pro Kopf von 86468USD, ein Big Mac kostet
in der Schweiz 6.44USD.
Zahlt man in der Schweiz zu viel f"ur einen Big Mac?
\item Wie gut ist die Approxmation?
\end{teilaufgaben}

\begin{loesung}
\begin{teilaufgaben}
\item
Wir bestimmen einen linearen Zusammenhang zwischen dem BIP pro Person $x$ 
und dem Preis $y$ eines Big Mac mit Hilfe von linearer Regression.
Dazu brauchen wir die Summen von $x$- und $y$-Werten, von Quadraten
und Produkten:
\begin{center}
\begin{tabular}{|>{$}c<{$}|>{$}r<{$}>{$}r<{$}|>{$}r<{$}>{$}r<{$}|>{$}r<{$}|}
\hline
   i&    x_i&   y_i&      x_i^2 &  y_i^2 &  x_iy_i\\
\hline
   1&  96930&  5.21&  9395424900& 27.1441& 505005.3\\
   2&  54370&  4.93&  2956096900& 24.3049& 268044.1\\
   3&  40001&  4.00&  1600080001& 16.0000& 160004.0\\
   4&   1608&  1.90&     2585664&  3.6100&   3055.2\\
\hline
\sum& 192909& 16.04& 13954187465& 71.0590& 936108.6\\
\hline
\end{tabular}
\end{center}
Daraus k"onnen wir jetzt die Koeffizienten der Regressionsgeraden $y=ax+b$
mit den bekannten Formeln ermitteln:
\begin{align*}
\textstyle
\sum x_i^2-\frac1n\bigl(\sum x_i\bigr)^2
&=13954187465-\frac14\cdot 192909^2
=
4650716894.75
\\
\textstyle
\sum y_i^2-\frac1n\bigl(\sum y_i\bigr)^2
&=
6.7386
\\
\textstyle
\sum x_iy_i-\frac1n\bigl(\sum x_i\bigr)\bigl(\sum y_i\bigr)
&=936108.6 -\frac14\cdot 192909\cdot 16.04
=
162543.51
\\
a
&=
\frac{E(XY)-E(X)E(Y)}{E(X^2)-E(X^2)}
=\frac{\frac1n\sum x_iy_i - \bigl(\frac1n\sum x_i\bigr)\bigl(\frac1n\sum y_i\bigr)}{\frac1n \sum x_i^2-\bigl(\frac1n\sum x_i\bigr)^2}
\\
&=
\frac{936108.6 -\frac14\cdot 192909\cdot 16.04}{13954187465-\frac14\cdot 192909^2}
=0.000034950
\\
b
&=
\frac1n\sum y_i-a\frac1n\sum x_i
=
\frac14\cdot16.04-a \frac14\cdot 192909
=
2.3244
\end{align*}
\item
F"ur das Schweizer BIP pro Person bekommt man als prognostizierten
Big Mac Preis
\[
y=2.3244 + 0.00003495\cdot 86468=5.3465,
\]
der Schweizer Big Mac ist also deutlich teurer, als die Prognose aus
dem BIP pro Kopf vermuten l"asst.
\item
Die Qualit"at der Approximation kann mit dem Regressionskoeffizienten
gemessen werden:
\begin{align*}
r
&=
\frac{E(XY)-E(X)(Y)}{\sqrt{\operatorname{var}(X)\operatorname{var}(Y)}}
=\frac{162543.51}{\sqrt{
4650716894.75
\cdot
6.7386
}}
=0.9182
\end{align*}
Die Abweichung von $1$ ist ziemlich gross, von einer guten Approximation
kann nicht die Rede sein.
Dies wird auch best"atigt durch die graphische Darstellung in
Abbildung~\ref{40000037:lin}.
\begin{figure}
\centering
\includeagraphics[]{graph-1.pdf}
\caption{Lineare Regression am Beispiel des Big Mac Index
\label{40000037:lin}}
\end{figure}
\end{teilaufgaben}
\end{loesung}

\begin{bewertung}
Ansatz Lineare Regression ({\bf L}) 1 Punkt,
Berechnung der Summen ({\bf T}) 1 Punkt,
Berechnung von $a$ ({\bf A}) 1 Punkt,
Berechnung von $b$ ({\bf B}) 1 Punkt,
Berechnung Big Mac Preis ({\bf P}) 1 Punkt,
Regressionskoeffizient als Mass f"ur Qualit"at ({\bf R}) 1 Punkt.
\end{bewertung}



