Welche Auszahlung ist zu erwarten, wenn nach dem Ziehen einer Karte aus
einem Kartenspiel mit 52 Karten die folgenden Gewinne ausbezahlt werden:
\begin{center}
\begin{tabular}{|l>{$}l<{$}|>{$}r<{$}|}
\hline
Ereignis&&\text{Gewinn}\\
\hline
Zahlkarte (2--9)             & A & 1\\
Bildkarte (J, Q, K) rot      & B & 2\\
Bildkarte (J, Q, K) schwarz  & C & 3\\
sonst                        &   & 0\\
\hline
\end{tabular}
\end{center}

\begin{loesung}
Zunächst müssen die Wahrscheinlichkeiten der Ereignisse $A$, $B$ und $C$ ermittel werden.
Sie sind
\begin{align*}
P(A) &= \frac{4\cdot 9}{52},
&
P(B) &= \frac{3\cdot 2}{52},
&
P(C) &= \frac{3\cdot 2}{52}.
\end{align*}
Damit kann jetzt der Erwartungswert ermittelt werden:
\begin{align*}
E(X)
&=
X(A)\cdot P(A)
+
X(B)\cdot P(B)
+
X(C)\cdot P(C)
+
0\cdot P(\overline{A\cup B \cup C})
\intertext{Da der Wert auf dem Ereignis $\overline{A\cup B\cup C}$ Null ist,
wird die Wahrscheinlichkeit gar nicht benötigt und der Erwartungswert
vereinfacht sich zu}
&=
1\cdot \frac{36}{52}
+
2\cdot \frac{6}{52}
+
3\cdot \frac{6}{52}
\\
&=
\frac{36 + 12 + 18}{52}
=
\frac{66}{52}
\approx
1.2692.
\qedhere
\end{align*}
\end{loesung}
