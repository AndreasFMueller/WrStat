Ein instabiler Farbstoff zerfällt mit der Zeit, die Konzentration
ist von der Zeit abhängig nach einem Gesetz von der Form
\[
c(t)=c_0e^{-kt}.
\]
Die Konzentration kann zum Beispiel durch die Absorption gemessen
werden, die proportional zur Konzentration ist. Gemessen wurden
folgende Werte:
\begin{center}
\begin{tabular}{|l|ccccc|}
\hline
$t$&1&2&3&4&5
\\
\hline
$c$&0.806&0.679&0.543&0.448&0.370
\\
\hline
\end{tabular}
\end{center}
Bestimmen Sie die Halbwertszeit des Stoffes, also die Zeit, nach
der die Hälfte des anfänglich vorhandenen Stoffes zerfallen ist.

\begin{hinweis}
Betrachten Sie $x(t)=\log c(t)$.
\end{hinweis}

\begin{loesung}
Die Abhängigkeit zwischen $t$ und $c$ ist nicht linear, aber
\[
\log c(t)=\log(c_0e^{-kt})=\log(c_0)-kt
\]
ist linear. Man kann also $k$ bestimmen, indem man eine lineare
Regression für $z=\log c(t)$ durchführt\footnote{Dabei wird
unter $\log c(t)$ der natürliche Logarithmus von $c(t)$
verstanden.}:
\begin{center}
\begin{tabular}{rccrcc}
$t$&$c$&$z=\log c$&$t^2$&$z^2$&$tz$\\
\hline
1&0.806&$-0.21567154$& 1&0.04651421&$-\phantom{0}0.21567154$\\
2&0.679&$-0.38713415$& 4&0.14987285&$-\phantom{0}0.77426830$\\
3&0.543&$-0.61064596$& 9&0.37288849&$-\phantom{0}1.83193788$\\
4&0.448&$-0.80296205$&16&0.64474805&$-\phantom{0}3.21184819$\\
5&0.370&$-0.99425227$&25&0.98853758&$-\phantom{0}4.97126137$\\
\hline
15&    &$-3.01066597$&55&2.20256118&$-11.00498727$\\
\end{tabular}
\end{center}
Mit den Formeln für die Regressionsgerade lässt sich daraus
die Steigung und der Achsabschnitt ermitteln.
\begin{align*}
-k=a&=\frac{E(TZ)-E(T)E(Z)}{E(T^2)-E(T)^2}
\\
&=\frac{\frac15\cdot(-)-\frac1{25}\cdot 15\cdot(-)}{
\frac15\cdot 55-\frac1{25}\cdot 15^2
}
\\
&=\frac{5\cdot(-11.00498727)- 15\cdot(-3.01066597)}{
5\cdot 55- 15^2
}
\\
&=\frac{-11.00498727- 3\cdot(-3.01066597)}{
55- 3\cdot 15
}
\\
&=\frac{-11.00498727- 3\cdot(-3.01066597)}{
10
}
\\
&=-0.197299
\\
\log c_0=b&=E(Z)-E(T)a
\\
&=\frac15(
-3.01066597+15 \cdot 0.19729894
)
\\
&=-0.01023638
\\
c_0&=0.9898158
\end{align*}
Die Halbwertszeit kann aus $k$ ermittelt werden: gesucht ist
die Zeit, zu der $e^{-kt}=\frac12$, also
\begin{align*}
-kt&=\log\frac12
\\
t&=-\frac1k\log\frac12=
-\frac1{0.19729894}\cdot(-0.69314718)
\\
&=\frac{0.69314718}{0.19729894}
=3.513182
\end{align*}
Die Halbwertszeit ist also $t=3.51$.

Verwendet man stattdessen den Zehnerlogarithmus, wird die Abhängigkeit
zu
$$
\log_{10} c(t)=-kt\log_{10}e+\log_{10}c_0
$$
Stellen wir die Tabelle wie vorhin auf, erhalten wir
\begin{center}
\begin{tabular}{rccrcc}
$t$&$c$&$z=\log_{10} c$&$t^2$&$z^2$&$tz$\\
\hline
1&0.806&$-0.09366496$&1&0.00877312&$-0.09366496$\\
2&0.679&$-0.16813023$&4&0.02826777&$-0.33626045$\\
3&0.543&$-0.26520017$&9&0.07033113&$-0.79560051$\\
4&0.448&$-0.34872199$&16&0.12160702&$-1.39488794$\\
5&0.370&$-0.43179828$&25&0.18644975&$-2.15899138$\\
\hline
15&    &$-1.30751562$&55&0.41542880&$-4.77940524$\\
\end{tabular}
\end{center}
Daraus folgt jetzt
\begin{align*}
-k\log_{10}e=a&=-0.08568584\\
-k&=\frac{-0.085685884}{\log_{10}e}=-0.085685884\cdot \log 10
=
-0.197299
\\
\log_{10}c_0=b&=-0.00444560\\
c_0&=0.9898158
\end{align*}
Die Berechnung der Halbwertszeit ändert sich dadurch jedoch
nicht.
\end{loesung}

