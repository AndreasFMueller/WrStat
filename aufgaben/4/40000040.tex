Ein Webshop verkauft Schuhe.
Ein Teil der Kunden weiss jedoch die eigene Schuhnummer nicht, 
daher implementiert der Webshop versuchshalber die folgende Idee.
Grosse Menschen haben auch eher grosse F"usse, es wird daher vermutet,
dass es zwischen K"orpergr"osse und Schuhgr"osse einen linearen
Zusammenhang gibt. 
Die K"orpergr"osse findet man im Pass.
Der Webshop versucht sein Modell mit den folgenden Daten zu
eichen:
\begin{center}
\begin{tabular}{lrrrrrrr}
K"orpergr"osse&165&170&175&180  &185&190&195\\
\hline
Schuhnummer   & 38& 39& 42& 44.5& 43& 45&46
\end{tabular}
\end{center}
\begin{teilaufgaben}
\item Finden sie ein lineares Modell, welches die Schuhnummer in
Abh"angigkeit von der K"orpergr"osse vorhersagt.
\item Welche Schuhr"osse h"atte eine 2m grosse Person nach diesem Modell?
\item Geben Sie eine Masszahl f"ur die Qualt"at dieses Modells.
\end{teilaufgaben}

\begin{loesung}
Ein solches Modell kann mit linearer Regression gefunden werden.
\begin{center}
\begin{tabular}{|>{$}c<{$}|>{$}c<{$}>{$}c<{$}|>{$}c<{$}>{$}c<{$}|>{$}c<{$}|}
\hline
i&K_i&S_i           &K_i^2&S_i^2&K_iS_i\phantom{.000}\\
\hline
1&165&38\phantom{.5}& 27225&1444.00&6270\phantom{.000}\\
2&170&39\phantom{.5}& 28900&1521.00&6630\phantom{.000}\\
3&175&42\phantom{.5}& 30625&1764.00&7350\phantom{.000}\\
4&180&44.5          & 32400&1980.25&8010\phantom{.000}\\
5&185&43\phantom{.5}& 34225&1849.00&7955\phantom{.000}\\
6&190&45\phantom{.5}& 36100&2025.00&8550\phantom{.000}\\
7&195&46\phantom{.5}& 38025&2116.00&8970\phantom{.000}\\
\hline
E&180&42.5          & 32500&1814.18&7676.429\\
\hline
\end{tabular}
\end{center}
\begin{teilaufgaben}
\item
Daraus kann man jetzt mit den bekannten Formeln die Steigung und den
Achsabschnitt berechnen:
\begin{align*}
a
&=
\frac{E(KS)-E(K)E(S)}{E(K^2)-E(K)^2}
=
\frac{7676.429 - 180\cdot 42.5}{32500-180^2}
\\
&=
0.2642857
\\
b
&=
E(S)-E(K)a
=
42.5 - 180\cdot 0.2642857
\\
&=
-5.071429
\end{align*}
\item Wir setzen $K=200$ in das Modell ein:
\[
200\cdot 0.2642857 -5.071429
=
47.79.
\]
\item
Als Masszahl f"ur die Qualit"at der Approximation kann der
Regressionskoeffizient verwendet werden:
\begin{align*}
r^2 
&=\frac{E(KS) - E(K)E(S)}{\sqrt{ (E(K^2)-E(K)^2) (E(S^2)-E(S)^2) }}
=0.9385906,
\end{align*}
was doch schon relativ weit von $1$ entfernt ist, und daher nicht unbedingt
auf gute Qualit"at hinweist.
\qedhere
\end{teilaufgaben}
\end{loesung}

\begin{bewertung}
\end{bewertung}

