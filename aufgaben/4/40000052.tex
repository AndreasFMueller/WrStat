Um Twitter für einen überhöhten Preis zu kaufen, musste Elon Musk
Tesla Aktien im grossen Stil verkaufen, was deren Wert zusammenbrechen
liess:
\begin{center}
\begin{tabular}{|l|r|r|}
\hline
Datum&Monat&Kurs\\
\hline
26.~August 2022   & 8&288.09\\
27.~September 2022& 9&282.94\\
27.~Oktober 2022  &10&225.09\\
28.~November 2022 &11&182.92\\
27.~Dezember 2022 &12&109.10\\
\hline
\end{tabular}
\end{center}
\begin{teilaufgaben}
\item
Finden Sie ein einfaches Modell, mit dem sich die Entwicklung des
Kurses der Tesla-Aktie modellieren lässt.
\item
Wann sagt Ihr Modell voraus, dass die Tesla-Aktie wertlos wird?
\item
Beurteilen Sie die Qualität Ihres Modells.
\end{teilaufgaben}

\ainput{tesla.tex}

\begin{loesung}
\begin{figure}
\centering
\begin{tikzpicture}[>=latex,thick]
\def\dx{1}
\def\ox{0}
\def\dy{0.01}
\draw[color=blue] ({7.5*\dx+\ox},{(7.5*\teslaa+\teslab)*\dy})
	-- ({15.5*\dx+\ox},{(15.5*\teslaa+\teslab)*\dy});
\draw[->] (7.45,0) -- (15.5,0) coordinate[label={Zeit $X$}];
\draw[->] (7.5,-0.05) -- (7.5,4) coordinate[label={right:$Y = \text{TSLA}$}];
\foreach \x in {8,...,15}{
	\draw ({\x*\dx+\ox},-0.05) -- ({\x*\dx+\ox},0.05);
	\node at ({\x*\dx+\ox},-0.05) [below] {\x};
}
\teslapunkte
\fill[color=darkgreen] ({\teslax*\dx+\ox},0) circle[radius=0.08];
\end{tikzpicture}
\caption{Der Kurs der Tesla-Aktie im zweiten Halbjahr 2022
\label{40000052:kurs}}
\end{figure}
\begin{teilaufgaben}
\item
Wir verwenden ein lineares Modell, führen also lineare Regression durch.
Dazu berechnen wir:
\begin{center}
\begin{tabular}{|>{$}c<{$}|>{$}r<{$}>{$}r<{$}|>{$}r<{$}>{$}r<{$}|>{$}r<{$}|}
\hline
i&x_i&x_i^2&y_i&y_i^2&x_iy_i\\
\hline
\tesladata
\hline
 & E(X)= \teslaEX & \operatorname{var}(X) = \teslavarX & E(Y) = \teslaEY & \operatorname{var}(Y)=\teslavarY & \operatorname{cov}(X,Y)=\teslacovXY \\
\hline
\end{tabular}
\end{center}
Daraus lassen sich jetzt Steigung $a$ und Achsabschnitt $b$ ermitteln:
\begin{align*}
a &= \frac{\operatorname{cov}(X,Y)}{\operatorname{var}(X)} = \teslaa\\
b &= E(Y) - a E(X) = \teslab
\end{align*}
\item
Die Aktie erreicht den Wert $0$ wenn
\[
ax+b=0
\qquad
\text{ oder}
\qquad
x  = -\frac{b}{a} = \teslax
\]
ist.
Damit wäre nach diesem Modell im März 2023 zu rechnen.
\item
Die Qualität des Modells lässt sich mit dem Regressionskoeffizienten
\[
r^2 = \teslarr
\qquad\Rightarrow\qquad
r = \teslar
\]
beurteilen.
Da $r$ relativ nahe an $-1$ ist, kann man die Genauigkeit als ausreichen
betrachten.
\qedhere
\end{teilaufgaben}
\end{loesung}

\begin{bewertung}
Lineare Regression ({\bf L}) 1 Punkt,
Berechnung von $a$ ({\bf A}) 1 Punkt,
Berechnung von $b$ ({\bf B}) 1 Punkt,
Vorhersage für Wert 0 ({\bf V}) 1 Punkt,
Qualitätskriterium $r^2\approx 1$ ({\bf Q}) 1 Punkt,
Berechnung von $r^2$ und Qualitätsurteil ({\bf R}) 1 Punkt.
\end{bewertung}
