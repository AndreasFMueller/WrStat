Aus theoretischen "Uberlegungen folgt, dass die Laufzeit eines bestimmten
Algorithmus proportional zur dritten Potenz $n^3$ der
Gr"osse $n$ des Problems sein m"usste (der Gauss-Algorithmus
f"ur lineare Gleichungssystem ist von dieser Art). Daher m"usste
der Logarithmus der Laufzeit 
\[
\log_2 t(n)=3\log_2(n) +C
\]
also proportional zum Logarithmus von $n$ sein.
Hier werden Logarithmen zur Basis 2 genommen.
Performance-Messungen geben folgendes Resultat:
\begin{center}
\begin{tabular}{|r|r|}
\hline
$n$&Laufzeit\\
\hline
$ 2$&$  0.1$\\
$ 4$&$  2.7$\\
$ 8$&$ 13.4$\\
$16$&$173.2$\\
\hline
\end{tabular}
\end{center}
\begin{teilaufgaben}
\item Welche Laufzeit ist f"ur $n=32$ zu erwarten?
\item Wie gut ist das Potenzgesetz $n^3$ f"ur die Messdaten erf"ullt?
\end{teilaufgaben}

\begin{loesung}
Hier wird offenbar lineare Regression f"ur die Logartihmen von $n$
und der Laufzeit gemacht, wir schreiben $X=\log_2 n$ und $Y=\log_2 t$.
Damit haben wir die Daten
\begin{center}
\begin{tabular}{| >{$}r<{$} >{$}r<{$}| >{$}r<{$} >{$}r<{$}| >{$}r<{$} >{$}r<{$} >{$}r<{$}|}
\hline
 n&    t& X&        Y&X^2&      Y^2&       XY\\
\hline
 2&  0.1& 1&-3.321928&  1&11.0352&-3.32192\\
 4&  2.7& 2& 1.432959&  4& 2.0533& 2.86591\\
 8& 13.4& 3& 3.744161&  9&14.0187&11.23248\\
16&173.2& 4& 7.436295& 16&55.2984&29.74518\\
\hline
  &     &10& 9.291488& 30&82.4058&40.52165\\
\hline
\end{tabular}
\end{center}
Nach den Formeln f"ur die lineare Regression sind Steigung
und Achsabschnitt der Regressionsgereaden
\begin{align*}
a&=
\frac{E(XY)-E(X)E(Y)}{E(X^2)-E(X)^2}
=3.458587
\\
b&=
E(Y)-aE(X)
=-6.323596
\end{align*}
\begin{teilaufgaben}
\item 
Mit Hilfe der Koeffizienten $a$ und $b$ kann man jetzt auch die
Laufzeit f"ur $n=32$, d.~h.~f"ur $X=5$ vorhersagen:
\begin{align*}
\log_2 t(5)&=a\cdot 5+b= 10.96934
\\
t(5)&=2004.935
\end{align*}
\item 
Die Qualit"at der Approximation wird durch den Regressionskoeffizienten
gemessen:
\[
r^2=\frac{(E(XY)-E(X)E(Y))^2}{
(E(X^2)-E(X)^2)
(E(Y^2)-E(Y)^2)
}=0.9833328
\]
Da $r^2$ ziemlich nahe bei $1$ liegt, kann man davon ausgehen, dass
das Gesetz gut erf"ullt ist, allerdings eher f"ur den Exponenten
$a=3.458587$ statt $3$. Dies zeigt auch die Abbildung~\ref{400000025:log}.
\begin{figure}
\begin{center}
\includeagraphics[width=\hsize]{log.pdf}
\end{center}
\caption{Logarithmische Darstellung der Abh"angigkeit der Laufzeit von
$n$\label{40000026:log}}
\end{figure}
\qedhere
\end{teilaufgaben}
\end{loesung}

\begin{bewertung}
Lineare Regression ({\bf L}) 1 Punkt,
Formeln f"ur ({\bf A}) und ({\bf B}) je 1 Punkt,
Berechnung der Koeffizienten ({\bf C}) 1 Punkt,
Sch"atzung von $t(32)$ ({\bf T}) 1 Punkt,
Regressionskoeffizient als Qualit"atskriterium ({\bf R}) 1 Punkt.
\end{bewertung}
