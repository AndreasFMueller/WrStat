Im deutschen Sprachraum ist im Weinbau das Mass ``Oechslegrad'', auch
$\mathstrut^\circ\text{Oe}$,
f"ur die
Angabe des Zuckergehaltes von Traubensaft "ublich. Je mehr 
$\mathstrut^\circ\text{Oe}$
ein Traubensaft hat, desto gr"osser ist der Gehalt an Zucker, und desto
gr"osser kann der Alkoholgehalt des fertigen Weines werden.

Der Zusammenhang zwischen Oechslegraden und Zuckergehalt in Volumen-Prozenten
ist linear. Doch Juristen und Gesetzgeber scheinen selbst von so einer einfachen
Formel "uberfordert, denn in der deutschen Weinverordnung von 1995 steht
in Paragraph 17: 
\begin{quote}
Die Ermittlung des nat"urlichen Alkoholgehalts in Volumenprozent (\%vol)
aus den Oechslegraden (Grad Oe) erfolgt nach der in der Anlage 8
aufgef"uhrten Tabelle. F"ur andere Umrechnungen ist die Tabelle nicht
anzuwenden.
\end{quote}
Es folgt dann eine umfangreiche
Tabelle, die den Zuckergehalt f"ur jeden Oechslegrad von 40 bis
150 angibt. Dies ist umso ungl"ucklicher, als schon viel h"ohere
Werte gemessen wurden, im Rekordsommer 2003 zum Beispiel $331^\circ\text{Oe}$.
Es gibt also keine gesetzteskonforme Zuckergehaltangabe f"ur diesen Saft.
Wenn man die Werte der Tabelle plottet, scheinen
sie relativ stark um die erwartete Gerade zu streuen. Finden Sie die
urspr"ungliche Gerade m"oglichst genau aus den unten ausgew"ahlten
Datenpunkten.
\begin{center}
\begin{tabular}{|r|r|}
\hline
$\mathstrut^\circ$Oe&\%vol\\
\hline
 40& 4.4\\
 80&10.6\\
120&16.9\\
150&21.5\\
\hline
\end{tabular}
\end{center}
\begin{teilaufgaben}
\item
Wieviele Oechslegrade hat ein Traubensaft mit 15 Volumenprozent Zuckergehalt?
\item
Was m"usste bei einem Wert von $160^\circ\text{Oe}$ f"ur ein Zuckergehalt 
herauskommen?
\item 
Welchen Zuckergehalt hat der im Text genannte Saft aus dem Rekordsommer 2003?
\end{teilaufgaben}

\begin{loesung}
Wir bezeichnen die Oechslegrad-Werte mit $x$ und den Alkoholgehalt in
Volumenprozent mit $y$.
Die optimal passende Gerade kann mit linearer Regression gefunden werden.
Dazu berechnen wir die Summen der Werte, Quadrate und Produkte der Zahlen
aus der Tabelle
\begin{center}
\begin{tabular}{|>{$}c<{$}|>{$}r<{$}>{$}r<{$}|>{$}r<{$}>{$}r<{$}|>{$}r<{$}|}
\hline
i     &x_i& y_i&x_i^2& y_i^2&x_iy_i\\
\hline
1     & 40& 4.4& 1600& 19.36&   176\\
2     & 80&10.6& 6400&112.36&   848\\
3     &120&16.9&14400&285.61&  2028\\
4     &150&21.5&22500&462.25&  3225\\
\hline
\Sigma&390&53.4&44900&879.58&  6277\\
\hline
\end{tabular}
\end{center}
Mit den Formeln f"ur die lineare Regression k"onnen jetzt die Koeffizienten
f"ur den linearen Zusammenhang zwischen $x$ und $y$ gefunden werden.
\begin{align*}
a
&=
\frac{4\cdot 6277 - 390 \cdot 53.4}{4\cdot 44900 - 390^2}
=
\frac{4282}{27500}
\simeq
0.1557090909
\\
b
&= \frac{53.4}{4} - a\frac{390}{4}
=
13.35 -a \cdot 37.5
\simeq
-1.831636363
\end{align*}
Wir finden daher die urspr"ungliche Umrechnungsformel
\[
y=0.1557090909x-1.831636363
\]
Damit k"onnen jetzt die Urspr"unglichen Fragen beantwortet werden:
\begin{teilaufgaben}
\item 
Ein Saft mit 15 Volumenprozent Zuckergehalt hat $108.1^\circ\text{Oe}$.
\item 
$160^\circ\text{Oe}$ enspricht ein Zuckergehalt von 23.08 Volumenprozent.
\item 
$331^\circ\text{Oe}$ enspricht ein Zuckergehalt von 49.71 Volumenprozent.
\qedhere
\end{teilaufgaben}
\end{loesung}

\begin{diskussion}
Es ist interessant, dass weder das Gesetz noch die zugeh"orige Verordnung
die im Vergleich zur Masszahlumrechnung viel schwierigere Frage "uberhaupt
tangiert, wie der Zuckergehalt gemessen werden muss.
Immerhin k"onnte zum Beispiel die Temperatur einen merklichen Einfluss
auf solche Messungen haben.
\end{diskussion}

\begin{bewertung}
Lineare Regression ({\bf L}) 1 Punkt,
Berechnung der Steigung $a$ ({\bf S}) 1 Punkt,
Berechnung des Abschnittes $b$ ({\bf I}) 1 Punkt,
L"osung f"ur die Teilaufgaben
a) ({\bf A}),
b) ({\bf B}) und
c) ({\bf C}) je 1 Punkt.
\end{bewertung}

