Bei einem Positionsbestimmungssystem eines Roboters sendet ein rotierender
Laserstrahl regelm"assig einen Impuls zu einem am Rande des Spielfeldes
stehenden Empf"anger, der daraus die $x$-Koordinaten bestimmen kann,
wobei nat"urlich auch ein Messfehler entsteht.
Da sich der Roboter auch drehen kann, ist die Zeit zwischen den Impulsen
nicht immer gleich, und manchmal wird der Laser auch von Hindernissen, zum
Beispiel von anderen Robotern verdeckt. Folgende Werte wurden gemessen
\begin{center}
\begin{tabular}{|r|r|}
\hline
$t$&$x$\\
\hline
0&-3\\
1&-3\\
3&-1\\
5&2\\
\hline
\end{tabular}
\end{center}
Geben sie die bestm"ogliche Sch"atzung an f"ur den Zeitpunkt,
zu dem der Roboter die Marke $x=0$ "uberquert hat.

\begin{loesung}
Die bestm"ogliche Sch"atzung f"ur die Bewegung des Roboters erh"alt
man mit linearer Regression.
\begin{center}
\begin{tabular}{|c|rr|rr|r|}
\hline
$i$&$t_i$&$x_i$&$t_i^2$&$x_i^2$&$t_ix_i$\\
\hline
1&0&$-3$& 0&9&$ 0$\\
2&1&$-3$& 1&9&$-3$\\
3&3&$-1$& 9&1&$-3$\\
4&5&$ 2$&25&4&$10$\\
\hline
&9&$-5$&35&23&4\\
\hline
\end{tabular}
\end{center}
Aus den Formeln f"ur die Koeffizienten der Regressionsgeraden finden
wir
\begin{align*}
a&=\frac{4\cdot 4 -9\cdot(-5)}{4\cdot 35-9^2}=\frac{61}{59}
\simeq 1.03389830508474576271
\\
b&=\frac14\cdot (-5)-a\frac14\cdot 9
=-\frac54-\frac{61\cdot 9}{59\cdot 4}
=-\frac{5\cdot 59+61\cdot 9}{4\cdot 59}
=-\frac{844}{236}
=-\frac{211}{59}
\simeq
-3.57627
\end{align*}
Die beste Sch"atzung ist also der Wert von $t$, der
$at+b=0$ erf"ullt, also
\[
t=-\frac{b}{a}=\frac{844}{236}\cdot\frac{59}{61}
=\frac{844}{4\cdot 61}=\frac{844}{244}
=\frac{211}{61}
\simeq 3.459016
\]
Man kann die Regression auch mit vertauschten Rollen von $t$ und $x$
durchf"uhren, was f"ur die sp"ater verlangte Berechnung von $t$
einfacher ist, aber ein leicht anderes Resultat ergibt. In diesem
Fall sind die Koeffizienten $a$ und $b$:
\begin{align*}
a&=0.910448\\
b&=3.38806
\end{align*}
Daraus ergibt sich der Wert von $t$ durch Einsetzen von $x$=0, ist
also identisch mit $b$: $t=b$.
\end{loesung}

