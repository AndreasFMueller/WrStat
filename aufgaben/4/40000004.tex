Ein Glücksspiel wird wie folgt gespielt. Der Spieler bezahlt seinen
Einsatz $a$. Dann wir einerseits ein
Rouletterad gedreht, andererseits ein Würfel geworfen. Falls
die beiden Zahlen teilerfremd sind, verliert man den Einsatz, falls
sie einen gemeinsamen Teiler $>1$ haben, gewinnt man so viel, wie
der Teiler angibt.
\begin{teilaufgaben}
\item Wie gross ist der grösstmögliche Gewinn?
\item Wie gross ist der wahrscheinlichste Gewinn?
\item Wie gross darf $a$ höchstens sein, damit sich das Spiel langfristig
für den Spieler lohnt (oder anders herum, welchen minimalen Einsatz $a$
muss das Spielcasino verlangen, damit es nicht ``drauflegt'')?
\end{teilaufgaben}

\begin{loesung}
\begin{teilaufgaben}
\item
Die grösstmögliche Augenzahl ist $6$, die kommt auch als gemeinsamer
Teiler vor, also ist  der grösstmögliche Gewinn $6$.
\item
Die möglichen Gewinne sind $2$, $3$ und $6$. Wir müssen abzählen,
wie häufig jeder dieser Gewinne vorkommt. Dies geschieht am einfachsten
mit folgender Tabelle, in der vertikal die Würfelzahl, horizontal die
Roulettradzahl und in den Zellen der Gewinn dargestellt ist.
\begin{center}
\begin{tabular}{|c|c|c|c|c|c|c|c|c|c|c|c|c|c|c|c|c|c|c|c|c|c|c|c|c|c|c|c|c|c|c|c|c|c|c|c|c|}
\hline
&&&&&&&&&&&&&&&&&&&&&&&&&&&&&&&&&&&&\\
\hline
&&2&&2&&2&&2&&2&&2&&2&&2&&2&&2&&2&&2&&2&&2&&2&&2&&2&&2\\
\hline
&&&3&&&3&&&3&&&3&&&3&&&3&&&3&&&3&&&3&&&3&&&3&&&3\\
\hline
&&2&&4&&2&&4&&2&&4&&2&&4&&2&&4&&2&&4&&2&&4&&2&&4&&2&&4\\
\hline
&&&&&5&&&&&5&&&&&5&&&&&5&&&&&5&&&&&5&&&&&5&\\
\hline
&&2&3&2&&6&&2&3&2&&6&&2&3&2&&6&&2&3&2&&6&&2&3&2&&6&&2&3&2&&6\\
\hline
\end{tabular}
\end{center}
Daraus liest man die folgenden Häufigkeiten für die einzelnen
Gewinne ab:
\begin{center}
\begin {tabular}{|c|c|c|}
\hline
Gewinn&Häufigkeit&Wahrscheinlichkeit\\
\hline
2&39&$\frac{39}{222}=\frac{13}{37}$\\
3&18&$\frac{18}{222}=\frac9{111}=\frac3{37}$\\
4&9&$\frac{9}{222}=\frac9{111}=\frac3{37}$\\
5&7&$\frac7{222}$\\
6&6&$\frac{6}{222}=\frac3{111}=\frac1{37}$\\
\hline
\end {tabular}
\end{center}
\item Mit den in der vorangegangen Teilaufgabe berechneten Wahrscheinlichkeiten
kann man jetzt auch den erwarteten Gewinn ausrechnen, er ist der
Erwartungswert der Zufallsvariable $X$, die den Gewinn angibt:
\begin{align*}
a=E(X)&=2\cdot \frac{39}{222}+3\cdot\frac{18}{222}+4\cdot\frac{9}{222}
+5\cdot\frac{7}{222}+6\cdot\frac6{222}
\\
&=\frac{
2\cdot 39+3\cdot 18+4\cdot 9+5\cdot 7 +6\cdot 6
}{222}
\\
&=\frac{ 239 }{222}
\simeq 1.07657657657657657657
\end{align*}
\end{teilaufgaben}
Leider hatte sich in der Aufgabenstellung ein Druckfehler eingeschlichen,
es war von gemeinsamem Teiler $> 0$ die Rede, was natürlich unsinnig ist,
denn Teiler sind immer $> 0$. Die meisten Kandidaten haben dies stillschweigend
in Teiler $>1$ übersetzt, was auch richtig ist, weil ausdrücklich
gesagt wird, dass für teilerfremde Zahlen (als gemeinsamer Teiler $= 1$)
kein Gewinn ausbezahlt wird.

Auch wurde nicht festgehalten,
dass die Nullen keinen Gewinn bringen, wie das beim Roulette-Rad so
üblich ist. Die Null ist durch jede Zahl teilbar. Wenn also das Roulette-Rad
eine Null zeigt, ist der Gewinn gleich gross wie die Augenzahl des Würfels,
in der Häufigkeitentabelle kommt also in jeder Zeile noch ein zusätzliches
Elementarereignis dazu. Der Erwartungswert wird damit zu
\[
a=E(X)=
\frac{2\cdot 40+3\cdot 19+4\cdot 10+5\cdot 8+6\cdot 7}{222}
=
\frac{259}{222}
\simeq
1.1666666
\qedhere
\]
\end{loesung}

