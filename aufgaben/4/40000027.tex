Im Spiel {\it Dungeons and Dragons} werden K"ampfe zwischen Spielern und
Monstern wie folgt ausgetragen.
Der Spieler w"urfelt mit einem Ikosaederw"urfel eine Zahl.
Falls sie mindestens den Wert
$a$ erreicht, der im wesentlichen die Panzerungsst"arke
des Monsters bestimmt ist, ist sein Angriff erfolgreich. Dann muss aber noch der
Schaden ermittelt werden, der dem Monster durch den Angriff zugef"ugt wird.
Dazu w"urfelt der Spieler nochmals mit einem 10er-W"urfel, und addiert die
eine Konstante $b$ dazu, die durch die Wahl der Angriffswaffe bestimmt ist.
Dies ergibt die Punktzahl, die dem Monster abgezogen wird.
Falls der Ikosaederw"urfel die Zahl 20 zeigt, ist das Erw"urfeln des
Schadens nicht n"otig, in diesem Fall wird automatisch der maximale
Schaden angerichtet. Wie gross ist der erwartete Schaden?

\begin{loesung}
Es sind offenbar drei F"alle zu unterscheiden:
\begin{enumerate}
\item Der Ikosaederw"urfel zeigt eine Zahl unter $a$, dann ist der Schaden
$S=0$. Dieser Fall tritt mit Wahrscheinlichkeit $\frac{a-1}{20}$ ein.
\item Der Ikosaederw"urfel zeigt eine Zahl zwischen $a$ und $19$,
dann ist der Schaden $W+b$, wobei $W$ das Resultat des 10er-W"urfels ist.
Der erwartete Schaden in diesem Fall ist $E(W+b)=E(W)+b=\frac{11}2 + b$.
Dieser Fall tritt mit Wahrscheinlichkeit $\frac{20-a}{20}$ ein.
\item Der Ikosaederw"urfel zeigt 20, dann ist der Schaden $S=b + 10$,
dieser Fall tritt mit Wahrscheinlichkeit $\frac1{20}$ ein.
\end{enumerate}
Damit kann man den erwarteten Schaden berechnen:
\begin{align*}
E(S)&=\frac{a-1}{20}\cdot 0 + \frac{20-a}{20}\cdot \biggl(\frac{11}2+b\biggr)
+\frac1{20}\cdot (b+10)
\\
&=
\frac1{20}\biggl(
(20-a)\biggl(\frac{11}2+b\biggr)+b+10
\biggr)
\\
&=
\frac1{20}\biggl(
110 -\frac{11a}2+20b-ab + b + 10
\biggr)
\\
&=
\frac1{20}\biggl(
120 -\frac{11a}2+21b-ab
\biggr).
\end{align*}
Insbesondere h"angt der Schaden nicht linear von den Werten $a$ und $b$ ab.
\end{loesung}
