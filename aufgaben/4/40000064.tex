Für einen Widerstand $R$ wurde bei der Spannung $U$ der Strom $I$ gemäss der
Tabelle
\begin{center}
\begin{tabular}{>{$}c<{$}|rrrrr|l}
U & 20\phantom{.0} & 40\phantom{.0} & 60\phantom{.0} & 80\phantom{.0} & 100\phantom{.0} & [mV] \\
\hline
I & 18.4 & 36.9 & 54.9 & 72.9 & 91.5 & [\textmu A]\\
\end{tabular}
\end{center}
gemessen.
Bestimmen Sie den bestmöglichen Wert des Widerstandes $R$.

\begin{hinweis}
Verwenden Sie den Regressions-Rechner \url{https://wrstat.ch/apps/lr}
\end{hinweis}

\begin{loesung}
Die Rechnung mit dem Regressionsrechner ergibt
\begin{center}
\includeagraphics[width=\textwidth]{lr.jpg}
\end{center}
Daraus kann man die Steigung $a=0.911$ ablesen, was eine Widerstand
von 
\[
R=\frac{1}{0.911}\,\text{k$\Omega$} \approx 1.1\,\text{k$\Omega$}
\]
entspricht.
\end{loesung}

