Auf einer indischen Website wurden folgende Information über den
Zusammenhang zwischen Autoverkaufszahlen in einem Jahr und
dem Bruttosozialprodukt (GDP, in nicht spezifizierten Einheiten)
des Vorjahres gefunden:
\begin{center}
\begin{tabular}{|l|r|r|}
\hline
Jahr&GDP&Verkäufe (Mio)\\
\hline
2011&6.2\phantom{0}&2.63\phantom{0}\\
2012&6.2\phantom{0}&2.665\\
2013&5.48&2.503\\
2014&6.54&2.601\\
2015&7.18&2.79\phantom{0}\\
2016&7.93&3.047\\
\hline
\end{tabular}
\end{center}
\begin{teilaufgaben}
\item 
Im aktuellen Jahr erreichte des Bruttosozialprodukt den Wert $9$.
Welche Anzahl von Autoverkäufen kann man im nächsten Jahr erwarten?
\item
Bei welchem Bruttosozialprodukt kann man damit rechnen, dass die
Zahl von 3 Millionen verkaufter Autos erreicht wird?
\item
Wie zuverlässig ist diese Prognose?
\end{teilaufgaben}

\thema{lineare Regression}

\begin{loesung}
Dieses Problem kann mit linearer Regression gelöst werden.
Wir bezeichnen das Bruttosozialprodukt mit $x$, die Zahl der
Autoverkäufe mit $y$.
Wir müssen die Summen der Werte, Quadrate und Produkte der Datenpunkte
berechnen:
%1 1 6.20 2.630 38.4400 6.916900 16.30600
%2 2 6.20 2.665 38.4400 7.102225 16.52300
%3 3 5.48 2.503 30.0304 6.265009 13.71644
%4 4 6.54 2.601 42.7716 6.765201 17.01054
%5 5 7.18 2.790 51.5524 7.784100 20.03220
%6 6 7.93 3.047 62.8849 9.284209 24.16271
\begin{center}
\begin{tabular}{|>{$}c<{$}|>{$}r<{$}>{$}r<{$}|>{$}r<{$}>{$}r<{$}|>{$}r<{$}|}
\hline
i& x_i & y_i  & x_i^2  & y_i^2   & x_iy_i  \\
\hline
1& 6.20& 2.630& 38.4400& 6.916900& 16.30600\\
2& 6.20& 2.665& 38.4400& 7.102225& 17.32250\\
3& 5.48& 2.503& 30.0304& 6.265009& 13.71644\\
4& 6.54& 2.601& 42.7716& 6.765201& 17.01054\\
5& 7.18& 2.790& 51.5524& 7.784100& 20.03220\\
6& 7.93& 3.047& 62.8849& 9.284209& 24.16271\\
\hline
 &39.53&16.236&264.1193&44.11764\phantom{0}&107.75089\\
\hline
\end{tabular}
\end{center}
Daraus kann man Steigung und Achsbschnitt der Regressionsgeraden
ermitteln:
\begin{align*}
a
&=
\frac{6\cdot 107.75089 - 39.53\cdot 16.236}{6\cdot 264.1193-39.53^2}
=
\frac{4.62246}{21.1469}
=
0.21255
\\
b
&=
\frac{16.236}{6} - a\cdot \frac{39.53}{6}
=
1.30565
\end{align*}
\begin{teilaufgaben}
\item
Mit der Regressionsgeraden kann man die zu erwartenden Verkäufe
prognostizieren:
\[
y = ax+b = a \cdot 9 + b
=
3.2186
\]
\item
Gesucht ist $x$ derart, dass
\[
3=ax+b
\qquad\Rightarrow\qquad
x
=
\frac{3-b}{a}
=
7.971536
\]
Wenn das Bruttosozialprodukt von 7.983329 erreicht wird kann man davon
ausgehen, dass die Verkaufszahlen 3 Mio überschreiten werden.
\item
Über die Qualität der Approximation gibt der Regressionskoeffizient
Auskunft.
Im vorliegenden Fall ist er
\[
r^2
=
\frac{
(6\cdot 107.75089 - 39.53\cdot 16.236)^2
}{
(6\cdot 264.1193-39.53^2)(6\cdot44.11764-16.236^2)
}
=
0.909
\]
Da $r^2$ nicht so wahnsinnig nahe bei $1$ ist, ist die Zuverlässigkeit
der Prognose fraglich.
\end{teilaufgaben}
\begin{figure}
\centering
\begin{tikzpicture}[>=latex,scale=3.7]
\draw[->,line width=0.7pt] (4.98,2)--(9.3,2) coordinate[label={$x$}];
\draw[->,line width=0.7pt] (5,1.98)--(5,3.5) coordinate[label={right:$y$}];
\def\punkt#1{
\fill[color=red] #1 circle[radius=0.02];
}
\punkt{(6.20,2.630)}
\punkt{(6.50,2.665)}
\punkt{(5.48,2.503)}
\punkt{(6.54,2.601)}
\punkt{(7.18,2.790)}
\punkt{(7.93,3.047)}
\def\a{0.218588}
\def\b{1.25494}
\draw[line width=1pt,color=blue] ({4.9},{\a*4.9+\b})--(9.3,{\a*9.3+\b});

\draw[line width=0.3pt] (9,2)--(9,{\a*9+\b});
\draw[line width=0.3pt] (5,{\a*9+\b})--(9,{\a*9+\b});
%\draw[line width=0.7pt] (4.97,{\a*9+\b})--(5.03,{\a*9+\b});
%\node at (4.97,{\a*9+\b}) [left] {$3.222$};
\fill[color=blue] (9,{\a*9+\b}) circle[radius=0.02];
\node at (9,{\a*9+\b}) [above left] {$y=3.222$};

\draw[line width=0.3pt] (5,3)--({(3-\b)/\a},3);
\draw[line width=0.3pt] ({(3-\b)/\a},3)--({(3-\b)/\a},2);
\node at ({(3-\b)/\a},3) [rotate=90,below left] {$x=7.983$};
\fill[color=blue] ({(3-\b)/\a},3) circle[radius=0.02];

\draw[line width=0.7pt] (6,1.98)--(6,2.02);
\draw[line width=0.7pt] (7,1.98)--(7,2.02);
\draw[line width=0.7pt] (8,1.98)--(8,2.02);
\draw[line width=0.7pt] (9,1.98)--(9,2.02);
\node at (6,1.97) [below] {$6$};
\node at (7,1.97) [below] {$7$};
\node at (8,1.97) [below] {$8$};
\node at (9,1.97) [below] {$9$};
\draw[line width=0.7pt] (4.98,3)--(5.02,3);
\node at (4.97,3) [left] {$3$};
\node at (5,2) [left] {$2$};
\node at (5,2) [below] {$5$};
\end{tikzpicture}
\caption{Abhängigkeit zwischen Autoverkäufen und Bruttosozialprodukt
(Aufgabe~\ref{40000045}).
\label{40000045:graph}}
\end{figure}
Der Zusammenhang zwischen Brutosozialprodukt und Autoverkäufen auf der
Basis obiger Daten ist in Abbildung~\ref{40000045:graph} dargestellt.
\end{loesung}

\begin{diskussion}
Die Daten dieser Aufgabe stammen von der Website
\url{https://towardsdatascience.com/linear-regression-with-example-8daf6205bd49}.
\end{diskussion}

\begin{bewertung}
Lineare Regression Steigung ({\bf A}) 1 Punkt,
Achsabschnitt ({\bf B}) 1 Punkt,
a) Verkäufe ({\bf V}) 1 Punkt,
b) GDB ({\bf G}) 1 Punkt,
c) Berechnung von $r^2$ ({\bf R}) 1 Punkt,
Qualitätskriterium ({\bf Q}) 1 Punkt.
\end{bewertung}
