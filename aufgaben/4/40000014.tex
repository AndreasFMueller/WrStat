Am Samstag, 7.~11.~2009 wurde in Kaltbrunn die
Bilderfolge in Abbildung~1 fotographiert. Sie zeigen den Start
einer 75cm (im Bild 48Pixel) langen Rakete mit einer Bildrate von
30 Bildern pro Sekunde.
Durch Ausmessen des gut definierten Anfangspunktes der Flamme findet
man die folgenden Punkte in Pixel-Koordinaten:
\begin{center}
\begin{tabular}{l|ccccc}
Horizontal&47&146&246&345&443\\
\hline
Vertikal (von oben)&438&411&356&273&161
\end{tabular}
\end{center}
\begin{figure}
\begin{center}
\includeagraphics[width=0.8\hsize]{stummel.jpg}
\end{center}
\caption{Bildfolge eines Raketenstarts mit 30 Bildern pro Sekunde}
\end{figure}
\begin{teilaufgaben}
\item Rechnen Sie die Datenpunkte Meter f"ur die vertikale Achse
und Sekunden f"ur die horizontale Achse, speichern sie diese Daten
im gleichen Datenframe unter den Namen {\tt t} und {\tt y}.
\item Unter der Annahme konstanter Beschleunigung (die im vorliegenden Fall
ziemlich genau zutrifft) bewegt sich die Rakete nach einem Gesetz
der Form
\[
y=\frac12at^2+bt+c,
\]
die Konstanten $a$, $b$ und $c$ sind noch zu bestimmen. Diese Aufgabe
kann sehr effizient mit Hilfe eines nichtlinearen Modells in R erledigt
werden. Dazu dient die Funktion {\tt nls}.
\item Verwenden Sie den Ansatz der Methode der kleinsten Quadrate, um
eine m"oglichst gute Approximation einer quadratischen Funktion
$c+bt+\frac12at^2$ an die Datenpunkte zu finden.
\item Mit welcher Beschleunigung startete die Rakete im photographierten
Zeitinterval?
\end{teilaufgaben}

\begin{figure}
\begin{center}
\includeagraphics[width=0.8\hsize]{rakete.pdf}
\end{center}
\caption{Nichtlineare Regression f"ur das Raketenproblem}
\end{figure}
\begin{loesung}
\begin{teilaufgaben}
\item
Die Daten werden zun"achst in ein File {\tt raw.csv}
geschrieben, welches folgenden Inhalt hat:
\verbatimainput{raw.csv}
Dann kann man diese Daten in eine Daten-Frame einlesen,
und die n"otigen Umrechnungen gleich durch neue Spalten im
Datenframe abbilden:
\verbatimainput{raw.R}
\item
Die Daten im Datenframe k"onnen jetzt verwendet werden, um
das Polynom anzupassen. Dazu muss man die Formel, die angepasst
werden soll, in f"ur R verst"andlicher Form formulieren.
\verbatimainput{formula.R}
Diese Formel ist das erste Argument der Funktion {\tt nls}, welche
das nichtlineare Modell anpasst:
\verbatimainput{fit.R}
Aus dem Output der {\tt summary}-Funktion
kann man ablesen, dass $a=398=50.6g$ ist.
\item Wir gehen davon aus, dass $X$ und $Y$ Zufallsvariablen sind, und
suchen die Werte $a$, $b$ und $c$ derart, dass die mittlere quadratische
Abweichung, also
$\Delta(a,b,c)=E((Y-(aX^2+bX+c))^2)$ so klein wie m"oglich ist.
Die Extremstellen finden wir aus den Ableitungen von $\Delta(a,b,c)$ nach
den Parametern:
\begin{align*}
\frac{\partial \Delta}{\partial a}
&=
E(2(Y-aX^2-bX-c)X^2)
\\
&=
2E(YX^2)-2aE(X^4)-2bE(X^3)-2cE(X^2)
\\
\frac{\partial \Delta}{\partial b}
&=
E(2(Y-aX^2-bX-c)X)
\\
&=
2E(YX)-2aE(X^3)-2bE(X^2)-2cE(X)
\\
\frac{\partial \Delta}{\partial c}
&=E(2(Y-aX^2-bX-c))
\\
&=
2E(Y)-2aE(X^2)-2bE(X)-2c
\\
\end{align*}
F"ur die Werte $a$, $b$, $c$ finden wir also das lineare Gleichungssystem
\[
\begin{pmatrix}
E(X^4)&E(X^3)&E(X^2)\\
E(X^3)&E(X^2)&E(X)\\
E(X^2)&E(X)&1
\end{pmatrix}
\begin{pmatrix}
a\\b\\c
\end{pmatrix}
=
\begin{pmatrix}
E(YX^2)\\E(YX)\\E(Y)
\end{pmatrix}
\]
Zur Berechnung der Funktion rechnen wir die Daten erst auf SI Einheiten
um. Aus der L"ange der Rakete bekommen wir die Umrechnungsformel
$Y=\frac{438}{48} - \frac{0.75}{48}y$ f"ur die H"ohe.
Da der Bildabstand 100 Pixel
betr"agt, zwischen den Bildern aber $\frac{1}{30}=0.03333$ Sekunden verstreichen,
ist die Umrechung f"ur die Zeit $X=\frac{1}{3000}=0.00033333$. Damit
lassen sich die Zeiten und H"ohen ebenso wie die ben"otigten Erwartungswerte
berechnen:
\begin{center}
\begin{tabular}{rrrrrrrr}
\hline
$X$&$Y$&$X^2$&$X^3$&$X^4$&$XY$&$X^2Y$\\
\hline
0.015667&2.2812&0.000245&0.0000038&0.00000006&0.035740&0.00055\\
0.048667&2.7031&0.002368&0.0001152&0.00000561&0.131552&0.00640\\
0.082000&3.5625&0.006724&0.0005513&0.00004521&0.292125&0.02395\\
0.115000&4.8594&0.013225&0.0015208&0.00017490&0.558828&0.06426\\
0.147667&6.6094&0.021805&0.0032199&0.00047548&0.975984&0.14412\\
\hline
0.081800&4.0031&0.008874&0.0010822&0.00014025&0.398845&0.04786\\
\hline
\end{tabular}
\end{center}
Daraus kann man jetzt die L"osung f"ur das Gleichungssystem ablesen:
\[
\begin{pmatrix}
a\\b\\c
\end{pmatrix}
=\begin{pmatrix}
0.0001402&0.0010823&0.0088737\\
0.0010823&0.0088737&0.0818000\\
0.0088737&0.0818000&1.0000000
\end{pmatrix}^{-1}
\begin{pmatrix}
0.047860\\
0.398846\\
4.003125
\end{pmatrix}
\]
Seine L"osung ist der Vektor
\[
\begin{pmatrix}
   205.35561\\
    -0.82335\\
     2.24822
\end{pmatrix}
\]
Zu seiner Berechnung kann das folgende Matlab-Programm verwendet werden:
\verbatimainput{qregr.m}
\item
Unter der Annahme konstanter Beschleunigung folgt,  dass sich die
Rakete nach einem quadratischen $s=s_0+v_0t+\frac12a_0t^2$ bewegt.
Das Resultat von Teilaufgabe a) zeigt, dass die Beschleunigng
\[
a_0=2a\simeq410.71122\frac{\text{m}}{\text{s}^2}\simeq42g
\]
betr"agt.
\end{teilaufgaben}
In der folgenden Abbildung wurde die in Teilaufgabe b) gefunden Kurve
in die Graphik hineingezeichnet:
\qedhere
\begin{figure}
\begin{center}
\includeagraphics[width=0.8\hsize]{kurve.jpg}
\end{center}
\caption{Bildsequenz des Raketenstarts mit Parabel\label{bildmitkurve}}
\end{figure}
\end{loesung}

