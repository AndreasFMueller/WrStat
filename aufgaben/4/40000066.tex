Beim radioaktiven Zerfall nimmt die Konzentration eines Isotops
exponentiell ab.
\begin{teilaufgaben}
\item
Finden Sie ein Modell für den Zusammenhang zwischen der Zeit $T$
und der Konzentration $X$ des Isotops.
\item
Konstruieren Sie ein lineares Regressionsproblem, um die Parameter
Ihres Modells zu bestimmen.
\end{teilaufgaben}

\begin{loesung}
\begin{teilaufgaben}
\item
Exponentieller Zerfall wird durch ein exponentielles Gesetz
\[
X
=
{\color{darkred}C}e^{-{\color{darkred}k}T}
\]
modelliert.
Zu bestimmen sind die Parameter ${\color{darkred}C}$ und
${\color{darkred}k}$.
\item
Durch Logarithmieren entsteht ein lineare Zusammenhang
\[
\log X
=
\log{\color{darkred}C} - {\color{darkred}k} T.
\]
Mit einer linearen Regression für den Zusammenhang zwischen $T$ und $\log X$
kann die Steigung $a=-k$ und der Achsabschnitt $b=\log C$ bestimmt werden.
\qedhere
\end{teilaufgaben}
\end{loesung}

