\newboolean{lang}
\setboolean{lang}{true}
In den Anfangsjahren des italienischen Bekleidungsherstellers Benetton
ergaben sich für die Ausgaben für Werbung und die Einnahmen aus
Verkäufen folgende Zahlen:
\begin{center}
\begin{tabular}{|>{$}r<{$}|>{$}r<{$}|}
\hline
\text{Werbung (Mio)}&\text{Verkäufe (Mio)}\\
\hline
23&  651\\
26&  762\\
30&  856\\
34& 1063\\
43& 1190\\
48& 1298\\
52& 1421\\
57& 1440\\
58& 1518\\
\hline
\end{tabular}
\end{center}
\begin{teilaufgaben}
\item
Können Sie ein einfaches Modell für den Zusammenhang zwischen Werbung
und Verkäufen aufstellen?
\item
Wieviel muss für Werbung ausgegeben werden, damit die Verkäufe
\ifthenelse{\boolean{lang}}{
1600 Mio
}{
1300 Mio
}
erreichen?
\item 
Wie gut ist das Modell?
\end{teilaufgaben}

\begin{loesung}
\begin{teilaufgaben}
\item
Man kann ein lineares Modell der Form $Y=aX+b$ verwenden, wobei $X$
für die Werbeausgaben und $Y$ für die Verkäufe steht.
Zur Berechnung der Koeffizienten $a$ und $b$ verwenden wir die Tabelle
\begin{center}
\def\p{\phantom{.00}}
\begin{tabular}{|>{$}c<{$}|>{$}r<{$}>{$}r<{$}|>{$}r<{$}>{$}r<{$}|>{$}r<{$}|}
\hline
i&x_i\p&     y_i\p& x_i^2\p&   y_i^2\p& x_iy_i\p\\
\hline
1& 23\p&     651\p&   529\p&  423801\p&  14973\p\\
2& 26\p&     762\p&   676\p&  580644\p&  19812\p\\
3& 30\p&     856\p&   900\p&  732736\p&  25680\p\\
4& 34\p&    1063\p&  1156\p& 1129969\p&  36142\p
\ifthenelse{\boolean{lang}}{\\
5& 43\p&    1190\p&  1849\p& 1416100\p&  51170\p\\
6& 48\p&    1298\p&  2304\p& 1684804\p&  62304\p\\
7& 52\p&    1421\p&  2704\p& 2019241\p&  73892\p\\
8& 57\p&    1440\p&  3249\p& 2073600\p&  82080\p\\
9& 58\p&    1518\p&  3364\p& 2304324\p&  88044\p
}{}\\
\hline
\ifthenelse{\boolean{lang}}{
 &41.22&   1133.22& 1859.00&1373913.22& 50455.22
}{
 &28.25&    833.00&  815.25& 716787.50& 24151.75
}
\\
 &E(X)\p&   E(Y)\p&E(X^2)\p&  E(Y^2)\p&  E(XY)\p\\
\hline
\end{tabular}
\end{center}
Daraus kann man jetzt die Koeffizienten des Modells ableiten:
\begin{align*}
a
&=
\frac{E(XY)-E(X)E(Y)}{E(X^2)-E(X)^2}
=
\ifthenelse{\boolean{lang}}{
\frac{50455.22 - 41.22 \cdot 1133.22}{1859.00 - 41.22^2} = 23.423
}{
\frac{24151.75 - 28.25 \cdot 833}{815.25 - 28.25^2} = 36.044
}
\\
b
&=
E(Y) - aE(X)
=
\ifthenelse{\boolean{lang}}{
1133.22 - 23.423 \cdot 41.22
}{
833 - 36.044 \cdot 28.25
}
=
\ifthenelse{\boolean{lang}}{
167.68
}{
-185.23
}.
\end{align*}
\item
Die Gleichung
\[
Y=aX +b 
\qquad\Rightarrow\qquad
X
=
\frac{Y-b}{a}
=
\ifthenelse{\boolean{lang}}{
\frac{1600 - 167.68}{23.423} = 61.151
}{
\frac{1300 + 185.23}{36.044} = 41.207
}
\]
liefert den Wert
\ifthenelse{\boolean{lang}}{
61.151
}{
41.207
}
Mio
als Werbebudget, um Verkäufe im Umfang von
\ifthenelse{\boolean{lang}}{
1600
}{
1300
}
Mio
zu erreichen.
\item
Die Qualität der linearen Regression kann man beurteilen mit Hilfe 
des Regressionskoeffizienten:
\begin{align*}
r
&=
\frac{
\operatorname{cov}(X,Y)
}{
\sqrt{\operatorname{var}(X)\operatorname{var}(Y)}
}
=
\ifthenelse{\boolean{lang}}{
\frac{3741.3}{\sqrt{159.73\cdot 89720.61728}} = 0.98829
}{
\frac{619.50}{\sqrt{17.188\cdot 22898.5}} = 0.98749
}.
\end{align*}
Da $r$ nahe bei $1$ ist, kann man die von einer guten Approximation sprechen.
\begin{center}
\begin{tikzpicture}[>=latex,thick,scale=0.2]
\ifthenelse{\boolean{lang}}{
\def\a{23.423} \def\b{167.68}
}{
\def\a{36.044} \def\b{-185.23}
}
\def\s{0.01}
\def\punkt#1#2{
\fill[color=red] (#1,{#2*\s}) circle[radius=0.3];
}
\draw[->] (22,0) -- (60,0) coordinate[label={$X$}];
\draw[->] (22,0) -- (22,20) coordinate[label={right:$Y$}];
\draw[color=blue,line width=1pt]
	(22,{(\a*22+(\b))*\s})
	--
	(58,{(\a*58+(\b))*\s});
\punkt{23}{651}
\punkt{26}{762}
\punkt{30}{856}
\punkt{34}{1063}
\ifthenelse{\boolean{lang}}{
\punkt{43}{1190}
\punkt{48}{1298}
\punkt{52}{1421}
\punkt{57}{1440}
\punkt{58}{1518}
}{}
\def\strich#1{
	\draw (#1,{-0.1/0.2}) -- (#1,{0.1/0.2});
	\node at (#1,0) [below] {$#1$};
}
\strich{22}
\strich{30}
\strich{40}
\strich{50}
\strich{58}
\def\hstrich#1{
	\draw ({22-0.1/0.2},{#1*\s}) -- ({22+0.1/0.2},{#1*\s});
	\node at (22,{#1*\s}) [left] {$#1$};
}
\hstrich{400}
\hstrich{800}
\hstrich{1200}
\hstrich{1600}
\end{tikzpicture}
\end{center}
\end{teilaufgaben}
\end{loesung}

\begin{bewertung}
Methode der linearen Regression ({\bf L}) 1 Punkt,
Wert von $a$ ({\bf A}) 1 Punkt,
Wert von $b$ ({\bf B}) 1 Punkt,
Berechnung von $x$ ({\bf X}) 1 Punkt,
Wert von $r$ ({\bf R}) 1 Punkt,
Beurteilung mit Kriterium ``$r^2$ nahe bei 1'' ({\bf U}) 1 Punkt.
\end{bewertung}


% Quelle: https://www.displayr.com/what-is-linear-regression/#:~:text=Linear%20regression%20quantifies%20the%20relationship,s)%20and%20one%20outcome%20variable.&text=For%20example%2C%20it%20can%20be,height%20(the%20outcome%20variable).
