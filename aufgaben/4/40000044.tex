{\em The Slow Mo Guys} haben am 13.~Dezember 2018 ein Video auf Youtube
veröffentlicht, in dem sie mit einer Hochgeschwindigkeitskamera die
Ausbreitung eines Bruchs in getempertem Glas aufgezeichnet haben.
Mit Hilfe einer Skala auf dem Glas konnten sie anschliessend die 
Ausbreitungsgeschwindigkeit des Bruchs bestimmen.
Die Auswertung des Films ergab, dass der Bruch zur Zeit $t$ die folgende
$x$-Position erreicht hatte:
\begin{center}
\begin{tabular}{ll|rrrrrrr}
%Zeit   &[$\mu$s]&3941.62&4157.65&4360.98&4577.01&4780.33&4996.36&5225.10\\
Zeit    &[$\mu$s]&3942&4158&4361&4577&4780&4996&5225\\
\hline
Position&[ft]    &1      &2      &3      &4     &5     &6     &7
\end{tabular}
\end{center}
\begin{teilaufgaben}
\item
Bestimmen Sie den bestmöglichen Wert für die Ausbreitungsgeschwindigkeit
des Bruchs in $[\text{ft}/\text{ms}]$.
\item
Breitet sich der Bruch wirklich mit konstanter Geschwindigkeit durch das
Glas aus?
\end{teilaufgaben}

\thema{lineare Regression}

\begin{loesung}
Wenn sich der Bruch tatsächlich mit konstanter Geschwindigkeit im Glas 
ausbreitet, dann gibt es einen linearen Zusammenhang zwischen Zeit $t$ und
Position $x$ des Bruchs in der Form $x=at+b$, wobei $a$ die
Ausbreitungsgeschwindigkeit des Bruchs ist.
Lineare Regression kann daher die bestmögliche Schätzung von $a$ liefern wie
auch eine Aussage über die Qualität der Approximation.

Zur Durchführung der Berechnung benötigen wir die Mittelwerte von der
der Messwerte $t_i$ und $x_i$ sowie ihrer Quadrate und Produkte:
\begin{center}
\begin{tabular}{|>{$}r<{$}|>{$}r<{$} >{$}r<{$}| >{$}r<{$} >{$}r<{$}|>{$}r<{$}|}
\hline
i& t_i&x_i&   t_i^2&x_i^2&t_ix_i\\
\hline
1&3942&  1&15539364&    1&  3942\\
2&4158&  2&17288964&    4&  8316\\
3&4361&  3&19018321&    9& 13083\\
4&4577&  4&20948929&   16& 18308\\
5&4780&  5&22848400&   25& 23900\\
6&4996&  6&24960016&   36& 29976\\
7&5225&  7&27300625&   49& 36575\\
\hline
E&4577&  4&21129231&   20& 19157\\
\hline
\end{tabular}
\end{center}

\begin{teilaufgaben}
\item
Wir bestimmen die Ausbreitungsgeschwindigkeit $a$ mit Hilfe der Formel
\begin{align*}
a
&=
\frac{\displaystyle\frac1n\sum_{i=1}^n t_ix_i - \frac1n\sum_{i=1}^n t_i\cdot \frac1n\sum_{i=1}^n x_i}{\displaystyle\frac1n\sum_{i=1}^n t_i^2 - \biggl(\frac1n\sum_{i=1}^nt_i\biggr)^2}
\\
&=
\frac{19157-4577\cdot 4}{21129231 - 4577^2}
=
\frac{849}{180302}
=
0.00471
=
4.71 [\text{ft}/\text{ms}].
\end{align*}
\item
Die Qualität der Approximation wird mit dem Regressionskoeffizienten
gemessen.
Er ist
\begin{align*}
r^2
&=
\frac{\displaystyle
\biggl(
\frac1n\sum_{i=1}^n t_ix_i
-
\frac1n\sum_{i=1}^nt_i
\cdot
\frac1n\sum_{i=1}^nx_i
\biggr)^2
}{\displaystyle
\biggl(
\frac1n\sum_{i=1}^nt_i^2 - \biggl(\frac1n\sum_{i=1}^nt_i\biggr)^2
\biggr)
\biggl(
\frac1n\sum_{i=1}^nx_i^2 - \biggl(\frac1n\sum_{i=1}^nx_i\biggr)^2
\biggr)
}
\\
&=
\frac{(19157 - 4577\cdot 4)^2}{
(21129231-4577^2)(20-4^2)
}
=
\frac{849^2}{721208}
=
\frac{720801}{721208}
=
0.999436.
\end{align*}
Da $r^2$ sehr nahe bei $1$ liegt, ist die lineare Approximation sehr gut,
die Ausbreitungsgeschwindigkeit des Bruchs ist also in sehr guter Näherung
konstant.
\qedhere
\end{teilaufgaben}
\end{loesung}

\begin{diskussion}
Das Video der Slow Mo Guys:
\url{https://www.youtube.com/watch?v=GIMVge5TYz4}
\end{diskussion}

\begin{bewertung}
Lineare Regression ({\bf L}) 1 Punkt,
Berechnung von $a$ ({\bf A}) 1 Punkt,
Wert von $a$ ({\bf W}) 1 Punkt,
korrekte Masseinheit von $a$ ({\bf M}) 1 Punkt,
Berechnung von $r^2$ ({\bf R}) 1 Punkt,
Qualitätsbeurteilung mit Hilfe von $r^2$ ({\bf Q}) 1 Punkt.
\end{bewertung}


