In einer Simulation wird ein Würfel 47000 mal geworfen, und der Mittelwert
der Augenzahlen sowie die Wahrscheinlichkeit für einen Sechser empirisch
ermittelt.

\begin{teilaufgaben}
\item Wie gross ist die Wahrscheinlichkeit, dass der Mittelwert um mehr
als 0.01 vom Erwartungswert abweicht?
\item Wie gross ist die Wahrscheinlichkeit, dass die relative Häufigkeit
für einen Sechser um mehr als 0.01 von $\frac16$ abweicht?
\item Die Tschebyscheff-Ungleichung ist nur eine Abschätzung, die
tatsächlichen Wahrscheinlichkeiten können wesentlich geringer sein.
Führen Sie selbst eine Simulation und ermitteln Sie empirisch die
Wahrscheinlichkeit für eine Abweichung der genannten Grösse. 
Stimmt die Grössenordung der in a) und b) gefundenen Wahrscheinlichkeit?
\end{teilaufgaben}


\begin{loesung}
Offenbar geht es um ein wiederholtes Experiment mit $n=47000$ Versuchen
und eine Abweichung der interessierenden Grösse von $\varepsilon=0.01$.
\begin{teilaufgaben}
\item
Das Gesetz der grossen Zahlen von Bernoulli sagt, dass 
\[
P(|M_n-E(X)|>\varepsilon) \le
\frac{\operatorname{var}(X)}{n\varepsilon^2}.
\]
Die Varianz für die Augenzahl eines Würfels wurde in der Vorlesung berechnet,
sie ist
$\operatorname{var}(X)=\frac{35}{12}.$
Einsetzen der Werte ergibt
\[
P(|M_n-E(X)|>\varepsilon)\le
\frac{\operatorname{var}(X)}{n\varepsilon^2}
=\frac{35}{12\cdot 47000\cdot 0.01^2}=0.62
\]
Im worst case erwartet man also in mehr als der Hälfte der Durchführungen
dieses Experimentes eine Abweichung von mehr als  0.01.
\item
Aus dem Gesetz der grossen Zahlen von Bernoulli haben wir folgende Formel
für die Wahrscheinlichkeit einer Abweichung der relativen Häufigkeit von
der Wahrscheinlichkeit hergeleitet:
\[
P(|h_n-p(A)|>\varepsilon)\le \frac{P(A)(1-P(A))}{n\varepsilon^2}.
\]
Im vorliegenden Fall ist $P(A)=\frac16$, Einsetzen der Werte ergibt
\[
P(|h_n-p(A)|>\varepsilon)\le \frac{P(A)(1-P(A))}{n\varepsilon^2}
=\frac{\frac16(1-\frac16)}{47000\cdot 0.01^2}
=\frac{5}{36\cdot 47000\cdot 0.01^2}
=0.03
\]
Nur in drei Prozent der Versuche wird man eine Abweichung relativen
Häufigkeit von der tatsächlichen Wahrscheinlichkeit von mehr als 0.01
finden.
\item Eine Simulation mit 100000 Wiederholungen des in der Aufgabe
beschriebenen Experimentes ergibt, eine Wahrscheinlichkeit von 0.204
für eine Abweichung des Mittelwertes von mehr als 0.01 vom tatsächlichen
Wert. In 100000 Versuchen wich die relative Häufigkeit kein einziges
Mal mehr als 0.01 vom theoretischen Wert ab.
\qedhere
\end{teilaufgaben}
\end{loesung}


