In einer Simulation wird ein W"urfel 47000 mal geworfen, und der Mittelwert
der Augenzahlen sowie die Wahrscheinlichkeit f"ur einen Sechser empirisch
ermittelt.

\begin{teilaufgaben}
\item Wie gross ist die Wahrscheinlichkeit, dass der Mittelwert um mehr
als 0.01 vom Erwartungswert abweicht?
\item Wie gross ist die Wahrscheinlichkeit, dass die relative H"aufigkeit
f"ur einen Sechser um mehr als 0.01 von $\frac16$ abweicht?
\item Die Tschebyscheff-Ungleichung ist nur eine Absch"atzung, die
tats"achlichen Wahrscheinlichkeiten k"onnen wesentlich geringer sein.
F"uhren Sie selbst eine Simulation und ermitteln Sie empirisch die
Wahrscheinlichkeit f"ur eine Abweichung der genannten Gr"osse. 
Stimmt die Gr"ossenordung der in a) und b) gefundenen Wahrscheinlichkeit?
\end{teilaufgaben}


\begin{loesung}
Offenbar geht es um ein wiederholtes Experiment mit $n=47000$ Versuchen
und eine Abweichung der interessierenden Gr"osse von $\varepsilon=0.01$.
\begin{teilaufgaben}
\item
Das Gesetz der grossen Zahlen von Bernoulli sagt, dass 
\[
P(|M_n-E(X)|>\varepsilon) \le
\frac{\operatorname{var}(X)}{n\varepsilon^2}.
\]
Die Varianz f"ur einen M"unzwurf wurde in der Vorlesung berechnet, sie ist
$\operatorname{var}(X)=\frac{35}{12}.$
Einsetzen der Werte ergibt
\[
P(|M_n-E(X)|>\varepsilon)\le
\frac{\operatorname{var}(X)}{n\varepsilon^2}
=\frac{35}{12\cdot 47000\cdot 0.01^2}=0.62
\]
Im worst case erwartet man also in mehr als der H"alfte der Durchf"uhrungen
dieses Experimentes eine Abweichung von mehr als  0.01.
\item
Aus dem Gesetz der grossen Zahlen von Bernoulli haben wir folgende Formel
f"ur die Wahrscheinlichkeit einer Abweichung der relativen H"aufigkeit von
der Wahrscheinlichkeit hergeleitet:
\[
P(|h_n-p(A)|>\varepsilon)\le \frac{P(A)(1-P(A))}{n\varepsilon^2}.
\]
Im vorliegenden Fall ist $P(A)=\frac16$, Einsetzen der Werte ergibt
\[
P(|h_n-p(A)|>\varepsilon)\le \frac{P(A)(1-P(A))}{n\varepsilon^2}
=\frac{\frac16(1-\frac16)}{47000\cdot 0.01^2}
=\frac{5}{36\cdot 47000\cdot 0.01^2}
=0.03
\]
Nur in drei Prozent der Versuche wird man eine Abweichung relativen
H"aufigkeit von der tats"achlichen Wahrscheinlichkeit von mehr als 0.01
finden.
\item Eine Simulation mit 100000 Wiederholungen des in der Aufgabe
beschriebenen Experimentes ergibt, eine Wahrscheinlichkeit von 0.204
f"ur eine Abweichung des Mittelwertes von mehr als 0.01 vom tats"achlichen
Wert. In 100000 Versuchen wich die relative H"aufigkeit kein einziges
Mal mehr als 0.01 vom theoretischen Wert ab.
\end{teilaufgaben}
\end{loesung}


