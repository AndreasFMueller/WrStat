In einem Spiel werden zwei Würfel geworfen und als Gewinn
der grösste gemeinsame Teiler der Augenzahlen ausbezahlt.
\begin{teilaufgaben}
\item
Welchen Gewinn erwarten Sie?
\item
Welche Varianz hat der Gewinn?
\end{teilaufgaben}

\thema{Laplace-Experiment}
\thema{Erwartungswert}
\thema{Varianz}

\begin{loesung}
\begin{figure}
\centering
\includeagraphics[]{graph-1.pdf}
\caption{Histogramm der Gewinne in Aufgabe~\ref{40000023}
\label{40000023:histogram}}
\end{figure}
Sei $X$ die Zufallsvariable des Gewinns.
Es führt wohl nichts daran vorbei, alle möglichen Gewinne aufzulisten.
Diese sind
\[
\begin{tabular}{|cccccc|}
\hline
1&1&1&1&1&1\\
1&2&1&2&1&2\\
1&1&3&1&1&3\\
1&2&1&4&1&2\\
1&1&1&1&5&1\\
1&2&3&2&1&6\\
\hline
\end{tabular}
\]
Daraus kann man abzählen, dass die Werte 1 bis 6 folgende
Wahrscheinlichkeiten haben, ein Histogram der Gewinne ist in 
Abbildung~\ref{40000023:histogram} dargestellt.
\[
\setlength\extrarowheight{3pt}
\begin{tabular}{>{$}c<{$}|>{$}c<{$}}
\text{Wert}&\text{Wahrscheinlichkeit}\\
\hline
1& \frac{23}{36} \\
2& \frac{ 7}{36} \\
3& \frac{ 3}{36} \\
4& \frac{ 1}{36} \\
5& \frac{ 1}{36} \\
6& \frac{ 1}{36} 
\end{tabular}
\]
Daraus kann man die benötigten Erwartungswerte ermitteln.
\begin{teilaufgaben}
\item
Der Erwartungswert des Gewinns ist
\begin{align*}
E(X)&=
1\cdot\frac{23}{36}+
2\cdot\frac{7}{36}+
3\cdot\frac{3}{36}+
4\cdot\frac{1}{36}+
5\cdot\frac{1}{36}+
6\cdot\frac{1}{36}
\\
&=\frac{23 + 2\cdot 7 + 3\cdot 3+4\cdot 1+5\cdot 1+6\cdot 1}{36}\\
&=\frac{23 + 14 + 9 + 4+5+6}{36}
=\frac{61}{36}=1.69444.
\end{align*}
\item
Für die Varianz brauchen wir zunächst den Erwartungswert $E(X^2)$:
\begin{align*}
E(X^2)&=
1^2\cdot\frac{23}{36}+
2^2\cdot\frac{7}{36}+
3^2\cdot\frac{3}{36}+
4^2\cdot\frac{1}{36}+
5^2\cdot\frac{1}{36}+
6^2\cdot\frac{1}{36}
\\
&=\frac{23 + 4\cdot 7 + 9\cdot 3+16\cdot 1+25\cdot 1+36\cdot 1}{36}\\
&=\frac{23 + 28 + 27 + 16+25+36}{36}
=\frac{155}{36}.
\end{align*}
Jetzt kann man daraus die Varianz bestimmen:
\begin{align*}
\operatorname{var}(X)
&=
E(X^2)-E(X)^2
=
\frac{155}{36}-\biggl(\frac{61}{36}\biggr)^2
=
\frac{155\cdot 36-61^2}{36^2}=\frac{1859}{1296}\simeq 1.4344.
\qedhere
\end{align*}
\end{teilaufgaben}
\end{loesung}

