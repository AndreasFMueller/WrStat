Eine faire Münze wird vier mal geworfen und die Anzahl ``Kopf'' gezählt.
Wenn $k$ mal Kopf erscheint, wird ein Gewinn ausbezahlt, der der $k+1$-ten
Primzahl entspricht. Wie gross ist die Varianz des Gewinns?

\begin{loesung}
$E(X)$ ist bereits berechnet worden, wir brauchen nur noch den
Erwartungswert der quadrierten Werte:
\begin{center}
\begin{tabular}{|c|c|c|c|c|}
\hline
Anzahl Kopf&Gewinn $g$&$g^2$&$P(X=g)$&$P(X=g)\cdot g^2$\\
\hline
$0$&$ 2$&$  4$&$\frac1{16}$&$\frac{  4}{16}$\\
$1$&$ 3$&$  9$&$\frac4{16}$&$\frac{ 36}{16}$\\
$2$&$ 5$&$ 25$&$\frac6{16}$&$\frac{150}{16}$\\
$3$&$ 7$&$ 49$&$\frac4{16}$&$\frac{196}{16}$\\
$4$&$11$&$121$&$\frac1{16}$&$\frac{121}{16}$\\
\hline
&&&&$\frac{507}{16}=31.6875$\\
\hline
\end{tabular}
\end{center}
Die Varianz ist jetzt
\[
\operatorname{var}(X)
=E(X^2)-E(X)^2
=\frac{507}{16}-\left(\frac{83}{16}\right)^2
=\frac{507\cdot 16 - 83^2}{16^2}=\frac{1223}{256}=4.77734375
\]
Wir können also sagen, dass die Gewinne im Mittel
etwa um $\pm2.186$ Einheiten um den Mittelwert $5.188$
streuen.
\end{loesung}

