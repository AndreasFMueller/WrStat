Eine faire M"unze wird vier mal geworfen und die Anzahl ``Kopf'' gez"ahlt.
Wenn $k$ mal ``Kopf'' erscheint, wird ein Gewinn ausbezahlt, der der $(k+1)$-ten
Primzahl entspricht. Wenn also zum Beispiel kein ``Kopf'' erscheint ist $k=0$,
$k+1=1$, die erste Primzahl ist $2$, es wird also $2$ als Gewinn ausbezahlt. 
Oder bei zwei mal ``Kopf'' wird die dritte Primzahl, also $5$ als Gewinn
ausbezahlt.
\begin{teilaufgaben}
\item
Welchen Gewinn k"onnen Sie erwarten?
\item
Welche Varianz hat der Gewinn?
\end{teilaufgaben}

\begin{loesung}
Wir erhalten die folgenden Wahrscheinlichkeiten und Gewinne
\begin{center}
\renewcommand{\arraystretch}{1.3}
\begin{tabular}{|>{$}c<{$}|>{$}c<{$}|>{$}c<{$}|>{$}c<{$}|>{$}c<{$}|}
\hline
\text{Anzahl Kopf}&\text{Gewinn $g$}&P(X=g)&P(X=g)\cdot g&P(X=g)\cdot g^2\\
\hline
0 &  2 & \frac1{16} & \frac{ 2}{16} &\frac{  4}{16}\\
1 &  3 & \frac4{16} & \frac{12}{16} &\frac{ 36}{16}\\
2 &  5 & \frac6{16} & \frac{30}{16} &\frac{150}{16}\\
3 &  7 & \frac4{16} & \frac{28}{16} &\frac{196}{16}\\
4 & 11 & \frac1{16} & \frac{11}{16} &\frac{121}{16}\\
\hline
  &    &            & \frac{83}{16} &\frac{507}{16}\\
\hline
\end{tabular}
\end{center}
Die gesuchten Erwartungswerte kann man daraus ablesen:
\begin{teilaufgaben}
\item
Der erwartete Gewinn ist $E(X)=\frac{83}{16}=5.1875$.
\item
Die Varianz ist
$\operatorname{var}(X)=E(X^2)-E(X)^2=\frac{507}{16}-\frac{83^2}{16^2}
=\frac{507\cdot 16-83^2}{16^2}=\frac{1223}{256}\simeq 4.777.$
\qedhere
\end{teilaufgaben}
\end{loesung}

