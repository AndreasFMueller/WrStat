Auf einer Waage werden 2000\,mg einer Flüssigkeit abgewogen.
Mit dem Ende der Wägung beginnt die Flüssigkeit auch sofort zu
verdampfen, wie die untenstehenden Messwerte zeigen.
\begin{teilaufgaben}
\item Bestimmen Sie den bestmöglichen Wert für die Verdunstungsrate.
\item Zu welcher Zeit war die Masse genau 2000\,mg?
\item Beurteilen Sie die Qualität der Approximation.
\end{teilaufgaben}

\begin{center}
\begin{tabular}{ll}
\begin{minipage}{5cm}
\begin{center}
\begin{tabular}{|>{$}r<{$}|>{$}r<{$}|}
\hline
\text{Zeit [s]}&\text{Masse [mg]}\\
\hline
  0.0&1999.49\\
 20.0&1999.23\\
 40.0&1998.99\\
 60.0&1998.72\\
 80.0&1998.48\\
100.0&1998.21\\
120.0&1997.96\\
140.0&1997.70\\
\hline
\end{tabular}
\end{center}
\end{minipage}&%
\begin{minipage}{9cm}
\begin{center}
\includeagraphics[]{graph.pdf}
\end{center}
\end{minipage}
\end{tabular}
\end{center}


\begin{loesung}
Es ist eine lineare Regression durchzuführen, dazu berechnen wir die
nachfolgende Tabelle.
Damit nicht so grosse Zahlen auftreten, verwenden wir den Excess der 
Masse über $1997\,\text{mg}$.
Bei der Berechnung mit der ganzen Masse können kleine Abweichungen auftreten,
die damit zu tun haben, dass die Berechnung der Varianz numerisch
problematisch ist, da in $E(X^2)-E(X)^2$ eine Differenz fast gleich
grosser Zahlen gebildet, ist das Resultat starker Auslöschung unterworfen.
wird
\begin{center}
\begin{tabular}{|>{$}r<{$}|>{$}r<{$}>{$}r<{$}|>{$}r<{$}>{$}r<{$}|>{$}r<{$}|}
\hline
i&  t_i&    m_i&  t_i^2&    m_i^2& t_im_i\\
\hline
1&    0&   2.49&      0&   6.2001&    0.0\\
2&   20&   2.23&    400&   4.9729&   44.6\\
3&   40&   1.99&   1600&   3.9601&   79.6\\
4&   60&   1.72&   3600&   2.9584&  103.2\\
5&   80&   1.48&   6400&   2.1904&  118.4\\
6&  100&   1.21&  10000&   1.4641&  121.0\\
7&  120&   0.96&  14400&   0.9216&  115.2\\
8&  140&   0.70&  19600&   0.4900&   98.0\\
\hline
 &  560&  12.78&  56400&  23.1576&  680.0\\
\hline
\end{tabular}
\end{center}
Die Parameter der Regressionsgerade werden damit:
\begin{align*}
a&= \frac{8\cdot 680.0 - 560\cdot 12.78}{8\cdot 56400-560^2}
=
\frac{-1716.80}{137600}=-0.0124767
\\
b&=\frac18 \cdot 12.78 -a\frac18\cdot 560=2.47087
\end{align*}
Damit können jetzt die einzelnen Fragen beantwortet werden:
\begin{teilaufgaben}
\item
Der bestmögliche Wert für die Verdunstungsrate ist $a=-0.0124767\text{[mg/s]}$.
\item
Gesucht ist die Zeit $t$ so, dass $at+b=3$.
Es folgt
\[
t=\frac{3-b}{a}=-42.409133\,\text{[s]}.
\]
\item
Die Qualität der Approximation kann mit dem Regressionskoeffizienten
beurteilt werden.
\begin{align*}
r
&=
\frac{
8\cdot 680 - 560\cdot 12.78
}{
\sqrt{8\cdot 56400-560^2}\sqrt{8\cdot 23.1576-12.78^2}
}
=
-0.988251
\end{align*}
Da $r$ sehr nahe bei $-1$ kann man die Qualität der Approximation als
sehr gut betrachten.
\qedhere
\end{teilaufgaben}
\end{loesung}

\begin{bewertung}
Lineare Regression ({\bf LR}) 1 Punkt,
Berechnungsmethode Tabelle von Summen ({\bf M}) 1 Punkt,
Formeln für $a$ ({\bf A}) 1 Punkt,
Formel für $b$ und Berechnungsmethode für Zeitpunkt bin b)  ({\bf B}) 1 Punkt,
Qualitätskriterium mit $r^2$ ({\bf Q}) 1 Punkt,
Folgerung für die Qualität in c) ({\bf C}) 1 Punkt.
\end{bewertung}
