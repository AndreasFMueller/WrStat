Auf einer Website findet man die folgende Tabelle über RAM-Preise
im Jahr 2023:
\begin{center}
\begin{tabular}{|>{$}c<{$}|>{$}r<{$}>{$}r<{$}|}
\hline
 i &\text{Kapazität\,[GB]}&\text{Preis\,[USD]}\\
\hline
 1 &  2 & 12 \\
 2 &  4 & 16 \\
 3 &  8 & 28 \\
 4 & 16 & 62 \\
\hline
\end{tabular}
\end{center}
\begin{teilaufgaben}
\item
Finden Sie eine einfaches Modell für die Abhängigkeit des Preises von
der RAM-Kapazität.
\item
Was für einen Preis prognostiziert Ihr Modell für einen 32\,GB RAM-Riegel?
\item
Welche RAM-Kapazität wäre für 100\,USD zu bekommen?
\item
Beurteilen Sie die Qualität Ihres Modells.
\end{teilaufgaben}

\begin{loesung}
Die Punkte und die Regressionsgerade sind in Abbildung~\ref{40000055:fig}
dargestellt.
\def\dx{0.45}
\def\dy{0.05}
\ainput{daten.tex}
\definecolor{darkred}{rgb}{0.8,0,0}
\def\p#1#2{ ({\dx*(#1)},{\dy*(#2)}) }
\def\punkt#1#2{ \fill[color=darkred] \p{#1}{#2} circle[radius=0.08]; }
\begin{figure}
\centering
\begin{tikzpicture}[>=latex,thick]
\draw[line width=0.3pt] \p{32}{0} -- \p{32}{\uy} -- \p{0}{\uy};
\draw[line width=0.3pt] \p{\h}{0} -- \p{\h}{100} -- \p{0}{100};
\draw[->] (-0.1,0) -- \p{34}{0} coordinate[label={$x\text{\,[GB]}$}];
\draw[->] (0,-0.1) -- \p{0}{128} coordinate[label={right:$y$}];
\foreach \x in {2,4,8,16,32}{
	\draw \p{\x}{-1} -- \p{\x}{1};
	\node at \p{\x}{-0.1} [below] {$\x$};
}
\foreach \y in {10,20,...,120}{
	\draw \p{-0.1}{\y} -- \p{0.1}{\y};
	\node at \p{-0.1}{\y} [left] {$\y$};
}
\gerade{blue}
\punkte
\end{tikzpicture}
\caption{Datenpunkte und Regressionsgerade für Aufgabe~\ref{40000055}.
\label{40000055:fig}}
\end{figure}
\begin{teilaufgaben}
\item
Mit der Berechnungstabelle
\begin{center}
\begin{tabular}{|>{$}c<{$}|>{$}r<{$}>{$}r<{$}|>{$}r<{$}>{$}r<{$}|>{$}r<{$}|}
\hline
i&x_i&y_i&x_i^2&y_i^2&x_iy_i\\
\hline
\tabelle
\hline
\summen
\hline
\mittelwerte
\hline
\end{tabular}
\end{center}
lassen sich die Koeffizienten eines linearen Modells wie folgt
berechnen:
\begin{align*}
a
&=
\frac{E(XY)-E(X)E(Y)}{E(X^2)-E(X)^2}
=
\steigung,
\\
b&=
E(Y)-aE(X)
=
\achsabschnitt.
\end{align*}
\item
Aus den Koeffizienten des Modells ergibt sich für die Kapazität $x_5=32$
der Wert
\[
y_5 = ax_5+b = \steigung\cdot 32 + \achsabschnitt = \uy\text{\,USD}.
\]
\item
Auflösen der Gleichung $ax+b=100$ nach $x$ ergibt
\[
x = \frac{100-b}{a} = \h\text{\,GB}.
\]
\item
Mit $r^2=\rr$ ist die Qualität recht gut.
Allerdings ist der Datensatz auch sehr klein und möglicherweise sogar
selektiert.
\qedhere
\end{teilaufgaben}
\end{loesung}

\begin{bewertung}
Lineare Regression ({\bf LR}) 1 Punkt,
Berechnung von $a$ ({\bf A}) 1 Punkt,
Berechnung von $b$ ({\bf B}) 1 Punkt,
Berechnung von $y_u$ in Teilaufgabe b) ({\bf Y}) 1 Punkt,
Berechnung von $h$ in Teilaufgabe c) ({\bf H}) 1 Punkt,
Beurteilung mit dem Regressionskoeffizienten in Teilaufgabe d)
({\bf R}) 1 Punkt.
\end{bewertung}


