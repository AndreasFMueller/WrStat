Die Zufallsvariable $X$ hat Erwartungswert $\mu$ und Varianz $\sigma^2$.
Welchen Erwartungswert und welche Varianz hat die Zufallsvariable
\[
Z=\frac{X-\mu}{\sigma}?
\]

\thema{Rechenregeln für Erwartungswert}
\thema{Standardisierung}

\begin{loesung}
Mit den Rechenregeln bekommt man
\begin{align*}
E(Z)&=
\frac1{\sigma}(E(X)-\mu)=0,\\
\operatorname{var}(Z)&=
\frac1{\sigma^2}(\operatorname{var}(X)+\operatorname{var}(\mu))
=
\frac{\operatorname{var}(X)}{\sigma^2}=1.
\end{align*}
Man kann also jede Zufallsvariable in eine transformieren, die Erwartungswert $0$
und Varianz $1$ hat.
\end{loesung}

