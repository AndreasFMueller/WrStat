Ein App f"ur Skifahrer soll mit Hilfe von GPS alle Bewegungen
eines Skifahrers auf Pisten und Skiliften aufzeichnen.
Auch die Neigung der Pisten sollen aufgezeichnet werden.
Dabei muss ber"ucksichtigt werden, dass sowohl die
Positionsmessungen als auch die H"ohe nicht ohne Messfehler
bestimmt werden k"onnen.  Die beste L"osung f"ur diese Art
von Problem w"are nat"urlich ein Kalmanfilter, dazu w"aren
aber Informationen "uber Systemfehler n"otig, die stark vom Fahrstil
eines Skisportlers abh"angt.  Daher wird folgende einfachere
Strategie verwendet: Aus den jeweils 4 letzten Positionsmessungen
wird ein m"oglichst guter Steigungswert bestimmt.
\begin{center}
\begin{tabular}{cc}
Distanz [m]&H"ohenunterschied [m]\\
\hline
 0&    0\\
50&$-27$\\
92&$-47$\\
130&$-67$\\
\hline
\end{tabular}
\end{center}
\begin{teilaufgaben}
\item
Welche Gel"andeneigung hat die Piste an dieser Stelle?
\item
Wie weit unter dem Ausgangspunkt befindet sich der Skifahrer
voraussichtlich nach 150m Fahrt?
\end{teilaufgaben}

\begin{loesung}
Die Steigung kann mit Hilfe von linearer Regression ermittelt werden.
\begin{center}
\begin{tabular}{|c|rr|rr|r|}
\hline
$i$&$x_i$&$x_i^2$&$y_i$&$y_i^2$&$x_iy_i$\\
\hline
1&  0&    0&$   0$&   0&$    0$\\
2& 50& 2500&$ -27$& 729&$ -1350$\\
3& 92& 8464&$ -47$&2209&$ -4324$\\
4&130&16900&$ -67$&4489&$ -8710$\\
\hline
 &272&27864&$-141$&7427&$-14384$\\
\hline
\end{tabular}
\end{center}
Die Formeln f"ur $a$ und $b$ liefern
\begin{align*}
a
&=\frac{4\cdot(-14384)-272\cdot (-141)}{4\cdot 27864 - 272^2}
\\
&=
\frac{-57536+38352}{111456-73984}
=
\frac{19184}{37472}=-0.51195559350982066609
\\
b
&=
\frac14(-141)-a\frac14272
=-35.25 + 0.51195559\cdot 68=-0.43701964133219470588
\end{align*}
\begin{teilaufgaben}
\item Der beste Steigungswert ist also $-0.51195559$, oder etwa 51\%.
\item Setzt man in den linearen Ausdruck $ax+b$ f"ur $x$ den Wert
150 ein, erh"alt man den erwarteten H"ohenunterschied
\[
a\cdot 150 + b= 
-77.2303586678052946193.
\qedhere
\]
\end{teilaufgaben}
\end{loesung}

