Ein App für Skifahrer soll mit Hilfe von GPS alle Bewegungen
eines Skifahrers auf Pisten und Skiliften aufzeichnen.
Auch die Neigung der Pisten sollen aufgezeichnet werden.
Dabei muss berücksichtigt werden, dass sowohl die
Positionsmessungen als auch die Höhe nicht ohne Messfehler
bestimmt werden können.  Die beste Lösung für diese Art
von Problem wäre natürlich ein Kalmanfilter, dazu wären
aber Informationen über Systemfehler nötig, die stark vom Fahrstil
eines Skisportlers abhängt.  Daher wird folgende einfachere
Strategie verwendet: Aus den jeweils 4 letzten Positionsmessungen
wird ein möglichst guter Steigungswert bestimmt.
\begin{center}
\begin{tabular}{cc}
Distanz [m]&Höhenunterschied [m]\\
\hline
 0&    0\\
50&$-27$\\
92&$-47$\\
130&$-67$\\
\hline
\end{tabular}
\end{center}
\begin{teilaufgaben}
\item
Welche Geländeneigung hat die Piste an dieser Stelle?
\item
Wie weit unter dem Ausgangspunkt befindet sich der Skifahrer
voraussichtlich nach 150m Fahrt?
\end{teilaufgaben}

\thema{lineare Regression}

\begin{loesung}
Die Steigung kann mit Hilfe von linearer Regression ermittelt werden.
\begin{center}
\begin{tabular}{|c|rr|rr|r|}
\hline
$i$&$x_i$&$x_i^2$&$y_i$&$y_i^2$&$x_iy_i$\\
\hline
1&  0&    0&$   0$&   0&$    0$\\
2& 50& 2500&$ -27$& 729&$ -1350$\\
3& 92& 8464&$ -47$&2209&$ -4324$\\
4&130&16900&$ -67$&4489&$ -8710$\\
\hline
 &272&27864&$-141$&7427&$-14384$\\
\hline
\end{tabular}
\end{center}
Die Formeln für $a$ und $b$ liefern
\begin{align*}
a
&=\frac{4\cdot(-14384)-272\cdot (-141)}{4\cdot 27864 - 272^2}
\\
&=
\frac{-57536+38352}{111456-73984}
=
\frac{19184}{37472}=-0.51195559350982066609
\\
b
&=
\frac14(-141)-a\frac14272
=-35.25 + 0.51195559\cdot 68=-0.43701964133219470588
\end{align*}
\begin{teilaufgaben}
\item Der beste Steigungswert ist also $-0.51195559$, oder etwa 51\%.
\item Setzt man in den linearen Ausdruck $ax+b$ für $x$ den Wert
150 ein, erhält man den erwarteten Höhenunterschied
\[
a\cdot 150 + b= 
-77.2303586678052946193.
\qedhere
\]
\end{teilaufgaben}
\end{loesung}

