Eine Zufallsvariable $X$ mit Erwartungswert $E(X)=1291$ und
Varianz $\operatorname{var}(X)=47$ wird $n$-mal gemessen.
Wieviele Messwerte müssen gemittelt werden, um der quadratische
Messfehler in den einstelligen Bereich zu bringen?

\begin{loesung}
Damit der Fehler einstellig wird, muss die Varianz des Mittelwertes
$<10$ sein.
Die Varianz des Mittelwertes von $n$ Messungen ist
\[
\operatorname{var}(M_n)
=
\frac{\operatorname{var}(X)}{n}.
\]
Die Bedingung
\[
\operatorname{var}(M_n)
=
\frac{\operatorname{var}(X)}{n}
=
\frac{47}{n}
<
10
\]
wird für $n>4.7$ erfüllt.
Es braucht also mindestens 5 Messungen, um die Varianz des Mittelwertes
in den den einstelligen Bereich zu senken.
\end{loesung}
