Eine Zufallsvariable nimmt gnazzahlige Werte mit den Wahrscheinlichkeiten
in der folgenden Tabelle an, die auch im Histogramm rechts visualisiert
sind:
\bigskip

\hbox to\textwidth{\hfill
\raisebox{2.0cm}{\begin{tabular}{>{$}c<{$}>{$}r<{$}}
\hline
x&      p_x\\
\hline
1& 0.142889\\
2& 0.352938\\
3& 0.302219\\
4& 0.162637\\
5& 0.039317\\
\hline
\end{tabular}}
\hfill
\def\s{6}
\def\balken#1#2{
	\fill[color=darkred!20] ({#1-0.5},0) rectangle ++(1,{\s*#2});
	\draw[color=darkred] ({#1-0.5},0) rectangle ++(1,{\s*#2});
}
\begin{tikzpicture}[>=latex,thick]
\balken{1}{0.142889}
\balken{2}{0.352938}
\balken{3}{0.302219}
\balken{4}{0.162637}
\balken{5}{0.039317}

\draw[->] (-0.1,0) -- (5.8,0) coordinate[label={$x$}];
\draw[->] (0,-0.1) -- (0,2.8) coordinate[label={right:$p$}];
\foreach \x in {1,...,5}{
	\draw (\x,-0.05) -- ++(0,0.1);
	\node at (\x,0) [below] {$\x\mathstrut$};
}
\end{tikzpicture}
\hfill}

\noindent
Visualisieren Sie die Erwartungswert und Streuung im Histogramm.

\begin{loesung}
Wir müssen zuerst Erwartungswerte und Varianz berechnen:
\begin{align*}
E(X)
&=
\sum_{x=1}^5 x\cdot p_x
=
2.6026
\\
E(X^2)
&=
\sum_{x=1}^5 x^2\cdot p_x
=
7.8597
\\
\operatorname{var}(X)
&=
E(X^2)-E(X)^2
=
7.8597
-
6.7735
=
1.0864.
\intertext{Für die Visualisierung muss die Wurzel gezogen werden:}
\sqrt{\operatorname{var}(X)}
&=
1.0423.
\end{align*}
Den Erwartungswert $E(X)$ visualisieren wir als blaue vertikale Linie
bei $x=E(X)$, die Streuung als grüne vertikale Linien bei
$x=E(X)\pm\sqrt{\operatorname{var}(X)}:$
\begin{equation*}
\def\s{6}
\def\balken#1#2{
	\fill[color=darkred!20] ({#1-0.5},0) rectangle ++(1,{\s*#2});
	\draw[color=darkred] ({#1-0.5},0) rectangle ++(1,{\s*#2});
}
\begin{tikzpicture}[>=latex,thick]
\balken{1}{0.142889}
\balken{2}{0.352938}
\balken{3}{0.302219}
\balken{4}{0.162637}
\balken{5}{0.039317}

\draw[->] (-0.1,0) -- (5.8,0) coordinate[label={$x$}];
\draw[->] (0,-0.1) -- (0,3.3) coordinate[label={right:$p$}];
\foreach \x in {1,...,5}{
	\draw (\x,-0.05) -- ++(0,0.1);
	\node at (\x,0) [below] {$\x\mathstrut$};
}
\def\ex{2.6026}
\def\var{1.0423}
\def\h{2.4}
\draw[color=blue,line width=1.2pt] (\ex,-0.1) -- ++(0,{\h+0.3});
\draw[<->,color=darkgreen,line width=1pt] (\ex,\h) -- ++(-\var,0);
\draw[<->,color=darkgreen,line width=1pt] (\ex,\h) -- ++(\var,0);
\node[color=blue] at (\ex,{\h+0.5}) {$E(X)\mathstrut$};
\draw[color=darkgreen,line width=1.2pt] ({\ex-\var},-0.1) -- ++(0,{\h+0.3});
\draw[color=darkgreen,line width=1.2pt] ({\ex+\var},-0.1) -- ++(0,{\h+0.3});
\draw[color=darkgreen,line width=0.4pt] ({\ex+0.5*\var},\h) -- ++(0.7,0.5);
\node[color=darkgreen] at ({\ex+0.5*\var+0.7},{\h+0.5})
	[right] {$\!\sqrt{\operatorname{var}(X)}$};
\end{tikzpicture}
\qedhere
\end{equation*}
\end{loesung}
