Im Spiel {\it Dungeons and Dragons} werden viele verschiedene Würfel
verwendet, die Zahlen zwischen $1$ und $n$ anzeigen können. Sei $X_n$ 
die Augenzahl eines solchen Würfels mit maximaler Augenzahl $n$. 
Berechnen Sie $E(X_n)$ und $\operatorname{var}(X_n)$.

\begin{hinweis}
Verwenden Sie
\begin{align*}
\sum_{k=1}^nk&=\frac{n(n+1)}2,&
\sum_{k=1}^nk^2&=\frac{n(n+1)(2n+1)}6.
\end{align*}
\end{hinweis}

\thema{Laplace-Experiment}
\thema{Erwartungswert}
\thema{Varianz}

\begin{loesung}
Für den Erwartungswert gilt
\begin{align*}
E(X_n)&=\sum_{i=1}^n i\cdot \frac1n=\frac1n\sum_{i=1}^ni
=\frac1n\cdot\frac{n(n+1)}2=\frac{n+1}2.
\end{align*}
Für die Varianz muss zusätzlich $E(X_n^2)$ ermittelt werden.
Unter Verwendung des Hinweises findet man
\begin{align*}
E(X_n^2)&=\sum_{i=1}^n i^2\cdot \frac1n=\frac1n\sum_{i=1}^ni^2
=\frac1n\cdot\frac{n(n+1)(2n+1)}6
=\frac{(n+1)(2n+1)}6,\\
\operatorname{var}(X_n)&=E(X_n^2)-E(X_n)^2
=\frac{(n+1)(2n+1)}6-\frac{(n+1)^2}4\\
&=\frac{(n+1)(4n + 2 - 3n - 3)}{12}
=\frac{(n+1)(n - 1)}{12}
=\frac{n^2-1}{12}.
\end{align*}
Dieses Resultate stimmt für $n=6$ mit dem bekannten Resultate für einen
``gewöhnlichen'' Würfel überein.
\end{loesung}
