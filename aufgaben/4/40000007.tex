Beim Spiel ``Eile mit Weile'' wird in jeder Runden nach folgenden Regeln
verfahren.
W"urfelt man eine Zahl zwischen 1 und 5, darf man mit der Spielfigur
um eine entsprechende Zahl Felder vorw"arts. W"urfelt man dagegen
eine 6, darf man 12 Felder vorr"ucken, und nochmals nach den
gleichen Regeln w"urfeln. W"urfelt man aber dreimal hintereinander
eine 6, wird man damit bestraft, dass man auf die Startposition
zur"uck muss, man kommt also nicht weiter.
\begin{teilaufgaben}
\item Sei $N$ das Ereignis, ``normal'' zu W"urfeln, also alles ausser drei
Sechser. $Z$ sei dagegen das Ereignis, 3 Sechser zu w"urfeln. Wie gross
sind die Wahrscheinlichkeiten von $N$ und $Z$.
\item Wie weit kommt man pro Runde im Mittel?
\item Wie wahrscheinlich ist, dass man in 10 Runden mindestens einmal
ganz an den Anfang zur"uck musste?
\item Wie viele Runden m"ussen gespielt werden, damit die Wahrscheinlichkeit,
dass man im Laufe des Spieles auf den Anfang zur"uck musste, gr"osser als 50\%
wird?
\item Wie weit ist man im Schnitt nach vier Runden?
\item Bonusfrage: Wie weit ist man im Schnitt nach $n$ Runden?
\end{teilaufgaben}

\begin{loesung}
Es sind folgende Ereignisse m"oglich
\begin{center}
\begin{tabular}{|ccc|c|r|r|r|}
\hline
1.&2.&3.&$p_i$&$x_i$&$p_i\cdot x_i$&$p_i\cdot x_i^2$\\
\hline
1& & &              &$         1$&$   \frac{1}{6}=\frac{ 36}{6^3}$&$\frac{   36}{6^3}$\\
2& & &              &$         2$&$   \frac{2}{6}=\frac{ 72}{6^3}$&$\frac{  144}{6^3}$\\
3& & &$    \frac16 $&$         3$&$   \frac{3}{6}=\frac{108}{6^3}$&$\frac{  324}{6^3}$\\
4& & &              &$         4$&$   \frac{4}{6}=\frac{144}{6^3}$&$\frac{  576}{6^3}$\\
5& & &              &$         5$&$   \frac{5}{6}=\frac{180}{6^3}$&$\frac{  900}{6^3}$\\
\hline
6&1& &              &$   12+1=13$&$\frac{13}{6^2}=\frac{ 78}{6^3}$&$\frac{ 1014}{6^3}$\\
6&2& &              &$   12+2=14$&$\frac{14}{6^2}=\frac{ 84}{6^3}$&$\frac{ 1176}{6^3}$\\
6&3& &$\frac1{6^2} $&$   12+3=15$&$\frac{15}{6^2}=\frac{ 90}{6^3}$&$\frac{ 1350}{6^3}$\\
6&4& &              &$   12+4=16$&$\frac{16}{6^2}=\frac{ 96}{6^3}$&$\frac{ 1536}{6^3}$\\
6&5& &              &$   12+5=17$&$\frac{17}{6^2}=\frac{102}{6^3}$&$\frac{ 1734}{6^3}$\\
\hline
6&6&1&              &$12+12+1=25$&$               \frac{ 25}{6^3}$&$\frac{  625}{6^3}$\\
6&6&2&              &$12+12+1=26$&$               \frac{ 26}{6^3}$&$\frac{  676}{6^3}$\\
6&6&3&$\frac1{6^3} $&$12+12+1=27$&$               \frac{ 27}{6^3}$&$\frac{  729}{6^3}$\\
6&6&4&              &$12+12+1=28$&$               \frac{ 28}{6^3}$&$\frac{  784}{6^3}$\\
6&6&5&              &$12+12+1=29$&$               \frac{ 29}{6^3}$&$\frac{  841}{6^3}$\\
6&6&6&              &zur"uck     &                                &                   \\
\hline
 & & &              &            &$\frac{1125}{6^3}=\frac{1125}{216}$
                                                                  &$\frac{12445}{6^3}=\frac{12445}{216}$\\
\hline
\end{tabular}
\end{center}
\begin{teilaufgaben}
\item Aus der Tabelle liest man die Wahrscheinlichkeit f"ur $Z$ in der letzten
Zeile ab: $P(Z)=\frac1{216}$. Da $Z\cup N=\Omega$ ist
$P(N)=1-P(Z)=\frac{215}{216}=0.9953704$.
\item
Erwartungswert und Varianz der Anzahl Felder, um die man in einer Runde vorr"ucken kann,
kann man wie folgt daraus berechnen:
\begin{align*}
E(X)
&=\frac{1125}{216}\simeq 5.208333
\\
\operatorname{var}(X)
&= E(X^2)-E(X)^2\\
&=\frac{12445}{216}-\left(\frac{1125}{216}\right)^2\\
&=\frac{12445\cdot 216  - 1125^2}{216^2}\\
&=\frac{1422495}{46656}\simeq 30.489\\
\sqrt{\operatorname{var}(X)}&\simeq 5.522
\end{align*}
\item
Die Wahrscheinlichkeit, in einer Runde nicht an den Anfang zur"uck
zu m"ussen ist $P(N)$. Die Wahrscheinlichkeit, dass man dieses
Gl"uck in 10 Runden jedes mal hat, ist $P(N)^{10}$. Die Wahrscheinlichkeit,
mindestens einmal zur"uck zu m"ussen ist also $1-P(N)^{10}=0.04534360167$.
\item
Man muss $n$ finden so dass $1-P(N)^n>\frac12$, also
\begin{align*}
P(N)^n<\frac12\\
n\log P(N)&<\log\frac12\\
n>\frac{\log\frac12}{\log P(N)}\simeq 149.3729
\end{align*}
Man beachte, dass die Richtung des Vergleichszeichens im zweitletzten Schritt
in der zweitletzten  Schritt kehrt, weil $\log P(N) < 0$. Man muss also mindestens
150 Runden spielen.
\item
G"abe es die Regel nicht, dass man nach drei Sechsern zur"uck muss,
w"are die gesuchte Gr"osse einfach $4 E(X)$. Da man aber
jederzeit ``zur"uckgeworfen'' werden kann, kommt es nur darauf an,
wie viele ``ungest"orte'' Runden man spielen konnte.
Wir unterscheiden die Ereignisse $Z$, drei Sechser, zur"uck auf die
Anfangsposition und $N$, normaler Wurf. Die Wahrscheinlichkeiten f"ur
die Ereignisse sind
\begin{align*}
P(Z)&= \frac1{216}\\
P(N)&= 1-P(Z)=\frac{6^3-1}{6^3}=\frac{215}{216}
\end{align*}
Dann sind folgende Szenarien m"oglich:
\begin{center}
\begin{tabular}{|cccc|c|c|c|}
\hline
1&2&3&4&$p$&$y$&$p\cdot y$\\
\hline
 * & * & * &$Z$&$P(Z)$&$0$&$0$\\
 * & * &$Z$&$N$&$P(Z)P(N)$&$E(X)$&$P(Z)P(N)$\\
 * &$Z$&$N$&$N$&$P(Z)P(N)^2$&$2E(X)$&$2P(Z)P(N)^2$\\
$Z$&$N$&$N$&$N$&$P(Z)P(N)^3$&$3E(X)$&$3P(Z)P(N)^3$\\
$N$&$N$&$N$&$N$&$P(N)^4$&$4E(X)$&$4    P(N)^4$\\
\hline
\end{tabular}
\end{center}
Die Summe der Werte in der mit $p$ angeschriebenen Spalte sind
\begin{align*}
P(Z)\biggl(\sum_{k=0}^3 P(N)^k\biggr) + P(N)^4
&=
(1-P(N))\biggl(\sum_{k=0}^3 P(N)^k\biggr) + P(N)^4\\
&=
(1-P(N)^4) + P(N)^4
=1
\end{align*}
Der Erwartungswert ist die Summe der Werte in der letzten
Spalte, also
\begin{align*}
E(S_4)
&=
P(Z)\biggl(\sum_{k=0}^3kP(N)^k\biggr)+4P(N)^4
\end{align*}
Allgemein f"ur $n$ Runden wird analog gelten
\begin{align*}
E(S_n)&=P(Z)\biggl(\sum_{k=0}^{n-1} kP(N)^k\biggr) + nP(N)^n
\end{align*}
F"ur die Reihe gilt mit der Abk"urzung $p=P(N)$
\begin{align*}
\sum_{k=1}^{n-1}kp^k
&=p\frac{d}{dp}\sum_{k=0}^{n-1}p^k
 =p\frac{d}{dp}\frac{1-p^n}{1-p}\\
&= \frac{(n-1) p^{n+1}-n p^n+p}{(1-p)^2}
\end{align*}
Somit ist der Erwartungswert $E(S_n)$
\begin{equation}
E(S_n)
=
\frac{(n-1) p^{n+1}-n p^n+p}{1-p}+np^n
=\frac{p(1-p^n)}{1-p}
\label{40000007:eile}
\end{equation}
F"ur $n=4$ und $p=\frac{215}{216}$ folgt
$
E(S_4)=
3.9539175
$.
\item Wurde in der vorangegangenen Teilaufgabe bereits
beantwortet.
\end{teilaufgaben}
Die Formel (\ref{40000007:eile}) f"ur $E(S_n)$ ist etwas "uberraschend,
sie bedeutet,
dass man im ``Eile mit Weile'' nur schwer sehr grosse Distanzen
erreichen kann. F"ur $n\to\infty$ strebt $E(S_n)$
wegen $p^n\to 0$ gegen den Wert $p/(1-p)=215$.  Und darin ist die
M"oglichkeit, dass andere Mitspieler eine Spielfigur wieder auf den
Anfang zur"ucksetzen k"onnen, noch gar nicht ber"ucksichtigt.
\end{loesung}

