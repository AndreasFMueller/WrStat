Ein Venustransit ist ein seltenes astronomisches Ereignis, bei dem
der Planet Venus vor der Sonnescheibe durchzieht.
Aus Beobachtungen der Dauer eines Transits von verschiedenen Standorten
auf der Erde aus konnte im 18.~Jahrhundert erstmals der Abstand
zwischen Erde und Sonne mit hoher Genauigkeit (Fehler $< 2\%$) ermittelt
werden. Solche Transite sind sehr selten, in keinem Jahrhundert gibt es
mehr als deren zwei. Viel h"aufiger sind Merkurtransite, die aber wegen
der Kleinheit des Planeten Merkur schwieriger zu beobachten sind.
In einem kleinen Teleskop ist Merkur als einziger schwarzer Pixel
sichtbar. Auch ist der Zeitpunkt, zu dem Merkur die Sonnescheibe
verl"asst, weniger genau beobachtbar. Daher wurde folgendes Messverfahren
vorgeschlagen: in regelm"assigen Zeitabstanden wird der Abstand des
Merkur vom Rand der Sonne gemessen, und dann extrapoliert, wann genau
Merkus die Sonne verl"asst. Die Messresultate sind:
\begin{center}
\begin{tabular}{cc}
Zeit [min]&Abstand [Pixel]\\
\hline
0&17\\
1&13\\
2&10\\
3&8\\
4&3\\
\hline
\end{tabular}
\end{center}
Finden sie eine bestm"ogliche Sch"atzung f"ur den Zeitpunkt, zu dem
Merkur die Sonnenscheibe verl"asst.

\begin{loesung}
Die bestm"ogliche Sch"atzung kann mittels linearer Regression
gefunden werden. Sei $T$ die Zeit, und $X$ der Abstand in Pixeln,
dann m"ussen empirischen Erwartungswerte f"ur $T$, $T^2$, $X$ und $X^2$
ermittelt werden. Dazu verwendet man folgende Tabelle:
\begin{center}
\begin{tabular}{|c|rr|rr|r|}
\hline
$i$&$t_i$&$t_i^2$&$x_i$&$x_i^2$&$t_ix_i$\\
\hline
0& 0& 0&17&289& 0\\
1& 1& 1&13&169&13\\
2& 2& 4&10&100&20\\
3& 3& 9& 8& 64&24\\
4& 4&16& 3&  9&12\\
\hline
 &10&30&51&631&69\\
\hline
\end{tabular}
\end{center}
Die Formeln f"ur die Koeffizienten der Regressionsgeraden liefern
mit $n=5$:
\begin{align*}
a&=
\frac{n\sum_i t_ix_i -\sum_it_i\sum_ix_i}{n\sum_it_i^2-(\sum_i t_i)^2}
\\
&=\frac{5\cdot 69 - 10\cdot 51}{5\cdot 30 - 10^2}=\frac{-165}{50}=-3.3
\\
b&=\frac1n\sum_ix_i-a\frac1n\sum_it_i\\
&=\frac15\cdot 51-(-3.3)\cdot \frac15\cdot 10=\frac{51+3.3 \cdot 10}5= 16.8
\end{align*}
Die beste Approximation ist also $x(t)=16.8-3.3t$.
Jetzt muss der Zeitpunkt $t$ gefunden werden, zu dem $x(t)=0$ gilt,
also
\[
16.8-3.3t=0\qquad\Rightarrow\qquad t=16.8/3.3=5.090909
\]
Merkur verl"asst also nach 5 Minuten und 5 Sekunden die Sonnescheibe.

Man kann die Aufgabe auch mit vertauschten Rollen von $t$ und $x$
l"osen. Dann braucht man am Ende nur noch $x=0$ einzusetzen, wegen
$t(x)=ax+b$ ist also $b$ genau das gesuchte Resultat. Die Formeln
f"ur $a$ und $b$ werden in diesem Fall zu
\begin{align*}
a&=
\frac{n\sum_i x_it_i -\sum_it_i\sum_it_i}{n\sum_ix_i^2-(\sum_i x_i)^2}
\\
&=\frac{5\cdot 69 - 10\cdot 51}{5\cdot 631 - 51^2}=\frac{-165}{554}=-0.297833935
\\
b&=\frac1n\sum_it_i-a\frac1n\sum_ix_i\\
&=\frac15\cdot 10-(-0.297833935)\cdot\frac15\cdot 51=\frac{10+ 0.297833935\cdot 51}5= 5.0379.
\end{align*}
In diesem Fall verl"asst Merkur die Sonnenscheibe also schon nach 5 Minuten
und 3 Sekunden. Aus diesem Unterschied kann man auch ablesen, dass man
mehr Genauigkeit als ein paar Sekunden mit diesem Verfahren niemals
wird erhalten k"onnen.
\end{loesung}

