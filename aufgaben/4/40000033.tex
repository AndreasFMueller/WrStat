Ein bekanntes Spielsystem im Roulette ist die Verdoppelung: man beginnt
mit dem Minimaleinsatz auf Rot, und solange man verliert,
was mit Wahrscheinlichkeit $p=\frac{19}{37}$ passiert,
verdoppelt man den Einsatz.
So holt man dann das verlorene Geld wieder zurück,
wenn der Gewinn endlich eintritt, wobei der
doppelte Einsatz ausbezahlt wird.
Die Casinos begegnen dieser Strategie mit der Limitierung der Einsätze.
Ausserdem ist das Verfahren ziemlich langwierig,
denn man kann jedesmal nur den Minimaleinsatz gewinnen.

Ein Spieler möchte das Verfahren etwas beschleunigen, und verdreifacht
den Einsatz jedesmal, wenn er verliert. So macht er seine Verluste mehr
als wett, sollte also schneller zu Geld kommen. Da er so auch schneller
den Maximaleinsatz erreicht, muss er aber auch häufiger damit rechnen,
den gesamten Einsatz zu verlieren.
Nehmen Sie an, dass der Maximaleinsatz das Zehnfache des Minimaleinsatzes ist.
\begin{teilaufgaben}
\item
Wie gross ist die Wahrscheinlichkeit, dass der Spieler seine Strategie
bis zu einem Gewinn durchziehen kann?
\item
Mit welchem mittleren Gewinn kann der Spieler rechnen?
\item
Ist diese Strategie besser oder schlechter als die klassische
Verdoppelungsstrategie?
\end{teilaufgaben}

\thema{Binomialverteilung}
\thema{Erwartungswert}

\begin{loesung}
\begin{teilaufgaben}
\item
Der Spieler kann seine Strategie wegen des Maximaleinsatzes maximal 
über drei Runden durchziehen. Wenn er in der dritten Runde nicht gewinnt,
hat er seinen Einsatz verloren.
Die Wahrscheinlichkeit nichts zu gewinnen, ist die Wahrscheinlichkeit,
dreimal hintereinander zu verlieren, also $p^3$.
Die Wahrscheinlichkeit zu gewinnen, ist also $1-p^3=0.86459$.
\item
Es gibt vier Szenarien: entweder
gewinnt er in Runde 1, 2 oder 3, oder er verliert alles.
Die Wahrscheinlichkeiten dafür und die möglichen Gewinne sind in 
der folgenden Tabelle zusammengestellt
\begin{center}
\begin{tabular}{|l|>{$}c<{$}|>{$}c<{$}|>{$}c<{$}|>{$}c<{$}|}
\hline
Szenario         &\text{Wahrscheinlichkeit}&\text{Aufwand}&\text{Ertrag}&\text{Gewinn}\\
\hline
Gewinn in Runde 1& (1-p)            &1      &2     &1\\
Gewinn in Runde 2& p(1-p)           &1+3    &6     &2\\
Gewinn in Runde 3& p^2(1-p)         &1+3+9  &18    &5\\
Verlust          & p^3              &1+3+9  &0     &-13\\
\hline
\end{tabular}
\end{center}
Zur Kontrolle können wir nachrechnen, dass sich die Wahrscheinlichkeiten
aller vier Szenarien tatsächlich  zu $1$ summieren:
\[
(1-p) + p(1-p) + p^2(1-p) + p^3
=
1-p + p - p^2 +p^2 - p^3 + p^3 = 1.
\]
Der Erwartungswert für den Gewinn ist also
\begin{align*}
E(G)
&=
(1-p) + 2 p(1-p) + 5p^2(1-p)-13p^3
=
1-p+2p -2p^2+5p^2-5p^3-13p^3
\\
&=
1+p+3p^2-18p^3
=
f(p)
\end{align*}
Setzt man darin den Wert $p=\frac{19}{37}$ ein, erhält man
\[
E(G)=f\biggl(\frac{19}{37}\biggr)
=
-\frac{6727}{50653}
=
-0.13281.
\]
Im Mittel verliert der Spieler also, wofür natürlich die niedrige
Einsatzlimite verantwortlich ist.
\item
Wir müssen dieselbe Rechnung für die Verdoppelungsstrategie durchführen.
Diese gestattet uns aber, eine Runde länger im Spiel zu bleiben.
Wir erhalten:
\begin{center}
\begin{tabular}{|l|>{$}c<{$}|>{$}c<{$}|>{$}c<{$}|>{$}c<{$}|}
\hline
Szenario&\text{Wahrscheinlichkeit}&\text{Aufwand}&\text{Ertrag}&\text{Gewinn}\\
\hline
Gewinn in Runde 1& (1-p)          &1      &2     &1\\
Gewinn in Runde 2& p(1-p)         &1+2    &4     &1\\
Gewinn in Runde 3& p^2(1-p)       &1+2+4  &8     &1\\
Gewinn in Runde 4& p^3(1-p)       &1+2+4+8&16    &1\\
Verlust          & p^4            &1+2+4+8&0     &-16\\
\hline
\end{tabular}
\end{center}
Der erwartete Gewinn ist daher
\[
E(G)
=
(1-p)
+p(1-p)
+p^2(1-p)
+p^3(1-p)
-16p^4
=1-17p^4
=
-\frac{341296}{1874161}=-0.18211.
\]
Die neue Strategie ist also etwas weniger verlustreich, aber nicht
viel besser.
\qedhere
\end{teilaufgaben}
\end{loesung}

\begin{bewertung}
Teilaufgabe a) ({\bf A}) 1 Punkt,
Zerlegung in 4 Szenarien ({\bf S}) 1 Punkt,
Wahrscheinlichkeiten ({\bf W}) 1 Punkt,
Erträge/Gewinne für die verschiedenen Szenarion ({\bf G}) 1 Punkt,
erwarteter Gewinn ({\bf E}) 1 Punkt,
erwarteter Gewinn für die Verdoppelungsstrategie ({\bf V}) 1 Punkt,
\end{bewertung}

\begin{diskussion}
Man kann die Gewinnerwartung bei der Verdoppelungsstrategie bei limitierten
Einsätzen explizit ausrechnen. Wenn die Verdoppelungsstrategie über
maximal $n$ Runden gespielt werden kann, dann ist der erwartete Gewinn
\begin{align*}
E(G)
&=
(1-p) + p(1-p) + \dots + p^{n-1}(1-p) - 2^np^n
=1-(2^n+1)p^n
=1-(2p)^n-p^n.
\end{align*}
Darin kann der letzte Term vernachlässigt werden, da er wegen $p<1$
gegen $0$ strebt.
Der zweitletzte Term dagegen strebt gegen $-\infty$, denn 
\[
2p=2\frac{19}{37}=\frac{38}{37}>1.
\]
Die Verdoppelungsstrategie ist also in real existierenden Casinos ruinös,
je länger man die Strategie spielen kann, desto grösser wird der zu
erwartende Verlust, ja er wächst sogar exponentiell mit der Limite an.
Ein Spielkasino ist also nicht grosszügig, wenn es eine hohe Limite erlaubt,
sondern gierig.
\end{diskussion}
