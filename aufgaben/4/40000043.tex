Ein Schmidt-Cassegrain-Teleskop hat einen gefalteten Strahlengang,
man kann daher die Brennweite nicht direkt messen.
Um die Brennweite $f$ zu bestimmen, verwendet man folgende Methode.
Man richtet das Teleskop auf einen Stern, durch die Erddrehung bewegt
er sich mit der Winkelgeschwindigkeit
$\omega=360^\circ/24\,h=2\pi/86400\,s$.
Mit einem Okular mit einer eingebauten Messplatte
\begin{center}
\begin{tabular}{ccc}
\begin{minipage}{0.25\hsize}
\includeagraphics[width=\hsize]{20040512Micro3.jpg}
\end{minipage}&\qquad\qquad\qquad&
\begin{minipage}{0.2\hsize}
\begin{tabular}{>{$}r<{$}|>{$}c<{$}}
 t_i\,[\text{s}]&x_i\,[\text{mm}]\\
\hline
 0&1.32\\
10&2.24\\
20&3.21\\
30&4.02\\
40&5.03
\end{tabular}
\end{minipage}
\end{tabular}
\end{center}
beobachtet man die Bewegung des Sterns entlang der horizontalen Skala
in der Mitte.
Er bewegt sich mit der Geschwindigkeit $f\omega$.
Ein Teilstrich dieser Skala ist 0.1\,mm.
In regelmässigen Zeitabständen liest man die Position des Sterns ab
und erhält die Datenpunkte gemäss obiger Tabelle.
\begin{teilaufgaben}
\item
Bestimmen Sie den bestmöglichen Wert für die Brennweite $f$ des Teleskops.
\item
Bestimmen Sie den Zeitpunkt, zu dem der Stern sich genau in der Mitte der
Skala befand.
\item
Wie gut ist diese Approximation?
\end{teilaufgaben}

\thema{lineare Regression}

\begin{loesung}
Die Position hängt linear von der Zeit ab, es ist $x=x_0 + vt$.
Um die Brennweite $f$ zu bestimmen, müssen wir aus den Daten
mit Hilfe von linearer Regression den Wert $v$ bestimmen und
daraus die Brennweite $f=v/\omega$ ablesen.
\begin{center}
\begin{tabular}{|
>{$}c<{$}|
>{$}r<{$}|
>{$}r<{$}|
>{$}r<{$}|
>{$}r<{$}|
>{$}r<{$}|}
\hline
i&t_i&  x_i&t_i^2&  x_i^2&t_ix_i\\
\hline
1&  0& 1.32&    0& 1.7424&    0.0\\
2& 10& 2.24&  100& 5.0176&   22.4\\
3& 20& 3.21&  400&10.3041&   64.2\\
4& 30& 4.02&  900&16.1604&  120.6\\
5& 40& 5.03& 1600&25.3009&  201.2\\
\hline
 &100&15.82& 3000&58.4254&  408.4\\
\hline
\end{tabular}
\end{center}
Daraus kann man die gesuchten Grössen wie folgt bekommen:
\begin{align*}
v
&=
\frac{5\cdot 408.4 - 100\cdot 15.82}{5\cdot 3000-100^2}=0.092
\\
x_0
&=
\frac15\cdot 15.82 - 0.092\cdot \frac15\cdot 100
=
1.324
\\
r^2
&=
0.999183
\end{align*}
Damit lassen sich die Teilaufgaben jetzt wie folgt lösen.
\begin{teilaufgaben}
\item
Die Brennweite ist $f=v/\omega=1265\,\text{mm}$.
\item
Wir suchen den Zeitpunkt $t$ mit der Eigenschaft
\[
x_0+vt=3.0
\qquad\Rightarrow\qquad
t = \frac{3.0-x_0}{v} = \frac{3.0-1.324}{0.092}=18.22.
\]
\item
Der Regressionskoeffizient $r^2$ ist sehr nahe bei $1$, es handelt sich
also um eine sehr gute Approximation.
\qedhere
\end{teilaufgaben}
\end{loesung}

\begin{bewertung}
Lineare Regression ({\bf L}) 1 Punkt,
Berechnung von $x_0$ und $v$ ({\bf V}) 1 Punkt,
Bestimmung der Brennweite ({\bf F}) 1 Punkt,
Bestimmung des Zeitpunkts ({\bf Z}) 1 Punkt,
Berechnung von $r^2$ ({\bf R}) 1 Punkt,
Qualitätsbeurteilung mit Hilfe von $r^2$ ({\bf Q}) 1 Punkt.
\end{bewertung}


