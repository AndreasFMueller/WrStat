Bei einem Wahrscheinlichkeitsexperiment tritt das Ereignis $A$
mit einer Wahrscheinlichkeit $p=P(A)$ ein. Dieses Experiment wird
$n$ mal wiederholt, die Zufallsvariable $X$ ist die Anzahl der
Versuche, bei denen $A$ eingetreten ist.
\begin{teilaufgaben}
\item Berechnen Sie die Wahrscheinlichkeit, dass $A$ genau $k$
mal eingetreten ist: $P(X=k)$.
\item Berechnen Sie den Erwartungswert $E(X)$ für $n=1,2$ und $3$.
\item Stellen Sie eine Formel für $E(X)$ auf.
\item Können Sie $E(X)$ auch für $n=1000$ berechnen?
\item Kann man das Resultat auch direkt aus den Rechenregeln bekommen?
\end{teilaufgaben}
{\it Hinweis:} Für d) betrachten Sie
\[
(x+y)^n=\sum_{k=0}^n\binom{n}{k}x^ky^{n-k},
\]
und berechnen sie die Ableitung beider Seiten nach $x$.
Setzen Sie dann $x=p$ und $y=1-p$ ein. Alternativ können Sie die
Anzahl als die Summe von $n$ Instanzen der Zufallsvariable $\chi_A$
betrachten.

\begin{loesung}
\begin{teilaufgaben}
\item
Man kann $k$ Versuche, bei denen $A$ eingetreten ist, auf
$\binom{n}{k}$ Arten aus der Versuchsreihe von $n$ Versuchen
auswählen. Eine bestimmte Reihenfolge von Versuchen hat die
Wahrscheinlichkeit $p^k(1-p)^{n-k}$, also ist die gesuchte
Wahrscheinlichkeit
\[
P(X=k)=\binom{n}{k}p^k(1-p)^{n-k}.
\]
\item
Für $n=2$ gilt
\[
E(X)=
0\cdot (1-p)^2+1\cdot 2\cdot p(1-p)+2p^2=2p-2p^2+2p^2=2p.
\]
Für $n=3$ gilt
\begin{align*}
E(X)&=0\cdot (1-p)^3+1\cdot 3p(1-p)^2+2\cdot 3p^2(1-p)+3\cdot p^3\\
&=3p-6p^2+3p^3+6p^2-6p^3+3p^3=3p.
\end{align*}
\item
Aus der Wahrscheinlichkeit des Wertes $k$ folgt
\[
E(X)=\sum_{k=0}^n\underbrace{\phantom{\binom{n}{k}}k}_{\text{Wert}}\cdot
\underbrace{\binom{n}{k}p^k(1-p)^{n-k}}_{\text{Wahrscheinlichkeit}}.
\]
\item Mit dem Hinweis berechnet man
\begin{align*}
(x+y)^n&=\sum_{k=0}^n\binom{n}{k}x^ky^{n-k},\\
\frac{d}{dx}
(x+y)^n&= \sum_{k=0}^n \frac{d}{dx} \binom{n}{k}x^ky^{n-k},\\
n(x+y)^{n-1}&=\sum_{k=0}^nk\binom{n}{k}x^{k-1}y^{n-k}\\
n(p+1-p)^{n-1}&=\sum_{k=0}^nk\binom{n}{k}p^{k-1}(1-p)^{n-k}
\end{align*}
Der Ausdruck auf der rechten Seite ist fast der Erwartungswert, nur
der Exponent von $p$ ist um eins zu klein. Man kann aber die ganze
Gleichung mit $x$ bzw.~mit $p$ multiplizieren:
\begin{align*}
nx(x+y)^{n-1}&=\sum_{k=0}^nk\binom{n}{k}x^ky^{n-k}\\
np(p+1-p)^{n-1}&=\sum_{k=0}^nk\binom{n}{k}p^k(1-p)^{n-k}
\end{align*}
Ersetzt man jetzt $x=p$ und $y=1-p$
entsteht auf der rechten
Seite der Erwartungswert, also
\[
np=E(X).
\]
\item
Betrachtet man $X$ als Summe von $n$ Instanzen der Zufallsvariable
$\chi_A$, die wir einzeln mit $Y_i$ bezeichnen, dann gilt
\[
E(X)=E(Y_1+\dots+Y_n)=E(\chi_A)+\dots +E(\chi_A)=nP(A).
\qedhere
\]
\end{teilaufgaben}
\end{loesung}

