Im Monty-Hall Problem k"onnen Autos und Ziegen gewonnen werden, wobei
die Ziegen als wertlos betrachtet werden. In diese Variante betrachten
wir die Ziegen nicht als wertlos, sondern bewerten Sie mit CHF~300.00,
die Autos haben einen Wert von CHF~30000.00.
\begin{teilaufgaben}
\item Wie gross ist der erwartete Gewinn, wenn man die Wechselstrategie
anwendet?
\item Wie gross ist der erwartete Gewinn, wenn man jeweils bei der Wahl
der ersten T"ur bleibt?
\end{teilaufgaben}

\begin{loesung}
Die Wahrscheinlichkeiten f"ur den Gewinn bei Anwendung einer der
genannten Strategien wurden bereits fr"uher berechnet.
\begin{teilaufgaben}
\item
Die Wahrscheinlichkeit, mit der Wechselstrategie das Auto zu gewinnen
ist $\frac23$, mit Wahrscheinlichkeit $\frac13$ wird eine Ziege gewonnen.
Der erwartete Gewinn ist daher:
\[
E(X)=\frac23\cdot 30000+\frac13\cdot 300.00=20100.00
\]
\item
Die Wahrscheinlichkeit, mit der treuen Strategie das Auto zu
gewinnen, ist $\frac13$, entsprechend ist der erwartete Gewinn
\[
E(X)=\frac13\cdot 30000+\frac23\cdot 300=10200.00
\]
\end{teilaufgaben}
\end{loesung}

