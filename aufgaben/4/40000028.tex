Im Spiel {\it Dungeons and Dragons} kann ein Spieler ein Monster mit
einem {\it Twin Strike} bekämpfen, und dabei noch die Fähigkeit
{\it Hunter's Quarry} einsetzen, die den am Monster angerichteten Schaden
weiter erhöht. Die Ermittlung des Schadens geschieht wie folgt.
Der Spieler führt zwei Schläge aus, in jedem würfelt er mit einem
Ikosaederwürfel und trifft das Monster, wenn der Ikosaederwürfel mindestens
den Wert $c$ zeigt. Falls er trifft, wird der angerichtete Schaden
mit dem 10er-Würfel ermittelt. Falls der Ikosaederwürfel 20 anzeigt,
entfällt das Erwürfeln des Schadens, der Schaden ist in diesem Fall 10.
Trifft einer der beiden Schläge, kann der Spieler den von
{\it Hunter's Quarry} versursachten zusätzlichen Schaden mit dem
6er-Würfel ermitteln. Welcher Schaden hat das Monster zu erwarten?


\begin{loesung}
Seien $w_1$ und $w_2$ die Resultate der beiden Würfe des Ikosaederwürfels.
Dann können wir neuen Fälle unterscheiden, mit folgenden Wahrscheinlichkeiten:
\begin{center}
\begin{tabular}{|>{$\displaystyle }c<{$}|>{$\displaystyle }c<{$}|>{$\displaystyle}c<{$}|>{$\displaystyle}c<{$}|}
\hline
&w_1 < c&w_1\le c<20&w_1=20\\
\hline
w_2<c&\frac{(c-1)^2}{400}&\frac{(c-1)(20-c)}{400}&\frac{c-1}{400}\\
\hline
w_2\le c< 20&\frac{(c-1)(20-c)}{400}&\frac{(20-c)^2}{400}&\frac{20-c}{400}\\
\hline
w_2=20&\frac{c-1}{400}&\frac{20-c}{400}&\frac1{400}\\
\hline
\end{tabular}
\end{center}
Der in den einzelnen Fällen zu erwartende Schaden setzt sich zusammen aus
dem Schaden des 10er-Würfels, dessen Erwartungswert 5.5 ist, und dem
zusätzlichen Schaden des 6er-Würfels mit Erwartungswert 3.5.
\begin{center}
\begin{tabular}{|>{$\displaystyle }c<{$}|>{$\displaystyle }c<{$}|>{$\displaystyle}c<{$}|>{$\displaystyle}c<{$}|}
\hline
&w_1 < c&w_1\le c<20&w_1=20\\
\hline
w_2<c&0&5.5+3.5 = 9&10+3.5 = 13.5\\
\hline
w_2\le c< 20&5.5+3.5=9&5.5 + 5.5 + 3.5=14.5&10 + 5.5 + 3.5 = 19\\
\hline
w_2=20&10+3.5=13.5&10 + 5.5 + 3.5=19&10 + 10 + 3.5=23.5\\
\hline
\end{tabular}
\end{center}
Daraus kann man jetzt den erwarteten Schaden ausrechnen:
\begin{align*}
E(S)
&=
\frac{(c-1)^2}{400}\cdot 0
+
2\frac{(c-1)(20-c)}{400}\cdot 9
+
2\frac{c-1}{400}\cdot 13.5
+
\frac{(20-c)^2}{400}\cdot 14.5
\\
&\quad
+
2\frac{20-c}{400}\cdot 19
+
\frac{1}{400}\cdot 23.5
\qedhere
\end{align*}
\end{loesung}
