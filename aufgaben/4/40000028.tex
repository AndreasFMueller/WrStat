Im Spiel {\it Dungeons and Dragons} kann ein Spieler ein Monster mit
einem {\it Twin Strike} bek"ampfen, und dabei noch die F"ahigkeit
{\it Hunter's Quarry} einsetzen, die den am Monster angerichteten Schaden
weiter erh"oht. Die Ermittlung des Schadens geschieht wie folgt.
Der Spieler f"uhrt zwei Schl"age aus, in jedem w"urfelt er mit einem
Ikosaederw"urfel und trifft das Monster, wenn der Ikosaederw"urfel mindestens
den Wert $c$ zeigt. Falls er trifft, wird der angerichtete Schaden
mit dem 10er-W"urfel ermittelt. Falls der Ikosaederw"urfel 20 anzeigt,
entf"allt das Erw"urfeln des Schadens, der Schaden ist in diesem Fall 10.
Trifft einer der beiden Schl"age, kann der Spieler den von
{\it Hunter's Quarry} versursachten zus"atzlichen Schaden mit dem
6er-W"urfel ermitteln. Welcher Schaden hat das Monster zu erwarten?


\begin{loesung}
Seien $w_1$ und $w_2$ die Resultate der beiden W"urfe des Ikosaederw"urfels.
Dann k"onnen wir neuen F"alle unterscheiden, mit folgenden Wahrscheinlichkeiten:
\begin{center}
\begin{tabular}{|>{$\displaystyle }c<{$}|>{$\displaystyle }c<{$}|>{$\displaystyle}c<{$}|>{$\displaystyle}c<{$}|}
\hline
&w_1 < c&w_1\le c<20&w_1=20\\
\hline
w_2<c&\frac{(c-1)^2}{400}&\frac{(c-1)(20-c)}{400}&\frac{c-1}{400}\\
\hline
w_2\le c< 20&\frac{(c-1)(20-c)}{400}&\frac{(20-c)^2}{400}&\frac{20-c}{400}\\
\hline
w_2=20&\frac{c-1}{400}&\frac{20-c}{400}&\frac1{400}\\
\hline
\end{tabular}
\end{center}
Der in den einzelnen F"allen zu erwartende Schaden setzt sich zusammen aus
dem Schaden des 10er-W"urfels, dessen Erwartungswert 5.5 ist, und dem
zus"atzlichen Schaden des 6er-W"urfels mit Erwartungswert 3.5.
\begin{center}
\begin{tabular}{|>{$\displaystyle }c<{$}|>{$\displaystyle }c<{$}|>{$\displaystyle}c<{$}|>{$\displaystyle}c<{$}|}
\hline
&w_1 < c&w_1\le c<20&w_1=20\\
\hline
w_2<c&0&5.5+3.5 = 9&10+3.5 = 13.5\\
\hline
w_2\le c< 20&5.5+3.5=9&5.5 + 5.5 + 3.5=14.5&10 + 5.5 + 3.5 = 19\\
\hline
w_2=20&10+3.5=13.5&10 + 5.5 + 3.5=19&10 + 10 + 3.5=23.5\\
\hline
\end{tabular}
\end{center}
Daraus kann man jetzt den erwarteten Schaden ausrechnen:
\begin{align*}
E(S)
&=
\frac{(c-1)^2}{400}\cdot 0
+
2\frac{(c-1)(20-c)}{400}\cdot 9
+
2\frac{c-1}{400}\cdot 13.5
+
\frac{(20-c)^2}{400}\cdot 14.5
\\
&\quad
+
2\frac{20-c}{400}\cdot 19
+
\frac{1}{400}\cdot 23.5
\qedhere
\end{align*}
\end{loesung}
