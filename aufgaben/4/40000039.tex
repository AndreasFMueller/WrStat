Auf einer amerikanischen Website gefunden: Studenten haben untersucht,
wie die Prüfungsnote vom Lernaufwand abhängt.
Die Umfrage unter Kollegen hat ergeben:
\begin{center}
\begin{tabular}{c|c}
Aufwand (Stunden)&SAT Score Mathematik\\
\hline
4&390\\
10&650\\
13&700\\
22&790
\end{tabular}
\end{center}
\begin{teilaufgaben}
\item
Angenommen, man kann die Prüfung mit eine Score von 500 bestehen, 
wieviele Stunden muss man in die Vorbereitung investieren?
\item
Wie zuverlässig ist Ihre Methode?
\end{teilaufgaben}

\begin{hinweis}
Es wird nicht unbedingt empfohlen, die Erkenntnisse aus der
Lösung dieser Aufgabe auf die Prüfungssituation an der HSR anzwenden.
\end{hinweis}

\begin{loesung}
\begin{teilaufgaben}
\item
Wir vermuten einen linearen Zusammenhang und bestimmen daher eine
Regressionsgerade zwischen Aufwand $A$ und Score $S$ in der
Form $S=aA+b$.
Dazu brauchen wir die Summen der Werte:
\begin{center}
\begin{tabular}{|>{$}c<{$}|>{$}r<{$}>{$}r<{$}|>{$}r<{$}>{$}r<{$}>{$}r<{$}|}
\hline
i&a_i& s_i&a_i^2&  s_i^2&a_is_i\\
\hline
1&  4& 390&   16& 152100&  1560\\
2& 10& 650&  100& 422500&  6500\\
3& 13& 700&  169& 490000&  9100\\
4& 22& 790&  484& 624100& 17380\\
\hline
 & 49&2530&  769&1688700& 34540\\
\hline
\end{tabular}
\end{center}
Daraus kann man mit Hilfe der Formeln im Skript die Koeffizienten $a$ und
$b$ berechnen:
\begin{align*}
a
&=
\frac{n\sum a_is_i - \sum a_i\sum s_i}{n\sum a_i^2-(\sum a_i)^2}
=
\frac{4\cdot 34540 - 49\cdot 2530}{4\cdot 769-49^2}
=
\frac{14190}{675}=21.022
\\
b
&=
\frac1n\sum s_i - a\cdot \frac1n\sum a_i
=
\frac{2530}{4}-a\frac{49}{4}=374.98
\end{align*}
Wir suchen jetzt die Punktzahl $a_0$, mit der man das Resultat 500 erreichen
kann:
\[
500=aa_0+b
\qquad\Rightarrow\qquad
a_0
=
\frac{500-b}{a}
=5.9470
\]
Die Prüfung sollte also mit einer Vorbereitungszeit von etwa 6 Stunden
zu schaffen sein.
\item
Zum Zweck der Beurteilung der Qualität der Approximation beechnen
wir den Regressionsoeffizienten
\begin{align*}
r^2
&=
\frac{(n\sum a_is_i - \sum a_i\sum s_i)^2}{(n\sum a_i^2-(\sum a_i)^2)
(n\sum s_i^2-(\sum s_i)^2)}
=
\frac{14190^2}{675\cdot 353900}
=
0.84291.
\end{align*}
Die Approximation ist gut, wenn $r^2$ nahe bei $1$ ist, tatsächlich ist
$|r|=0.9181$ recht nahe bei $1$, so dass wir einigermassen Vertrauen
in diese Lösung haben können.
\qedhere
\end{teilaufgaben}
\end{loesung}

\begin{bewertung}
Lineare Regression ({\bf L}) 1 Punkt,
Koeffizient $a$ ({\bf A}) 1 Punkt,
Koeffizient $b$ ({\bf B}) 1 Punkt,
Aufwand $a_0$ ({\bf A$\mathstrut_0$}) 1 Punkt,
Regressionskoeffizient $r^2$ ({\bf  R}) 1 Punkt,
Beurteilung (nahe bei $1$) ({\bf N}) 1 Punkt.
\end{bewertung}


