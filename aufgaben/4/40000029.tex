Das Auslesen der Pixel eines CCD-Chips einer Digitalkamera funktioniert
wie folgt: Der Inhalt aller Pixel werden gemeinsam eine Zeile nach unten
geschoben,
die untersten Pixel werden dadurch in eine spezielle Auslesezeile geschoben,
die dann digitalisiert wird. Insbesondere wandern also alle Pixel vertikal
durch die Spalte, bis sie in der Auslesezeile ankommen.
Während dieses Prozesses, der ja auch Zeit braucht, akkumuliert sich
ein zusätzliches Störsignal. Je länger der Pixelinhalt auf dem Chip
bleibt, desto mehr zusätzliches Signal akkumuliert sich:
\begin{center}
\includeagraphics[width=0.8\hsize]{Biasthermaly.png}
\end{center}
%(von \url{http://canburytech.net/QSI532/BiasFrames.html}).
Misst man in einer Spalte eines dunklen Bildes die Pixelwerte $v$ an
einzelnen Positionen $y$, und erhält folgende Werte:
\begin{center}
\begin{tabular}{|r|r|}
\hline
$y$&$v$\\
\hline
   0& 28\\
 500& 26\\
1000& 25\\
1500& 22\\
\hline
\end{tabular}
\end{center}
\begin{teilaufgaben}
\item Welchen Wert erwarten Sie in Zeile 800?
\item Um wieviel nimmt der Pixelwert pro Zeile im Mittel zu?
\item Wie gut ist die Approximation, die Sie in a) verwendet haben?
\end{teilaufgaben}

\begin{loesung}
Man kann das zusätzliche Signal als lineare Funktion $v=ay+b$ betrachten.
Gesucht ist also Werte für $a$ und $b$ so, dass die Gerade möglichst
gut zu den vorgegebenen Werten passt. Diese Aufgabe wird von der linearen
Regression gelöst.

Damit die Zahlen nicht so gross werden, verwenden wir als Massenheit entlang
der $y$-Achse Hunderter.

\begin{center}
\begin{tabular}{|>{$}r<{$}|>{$}r<{$}>{$}r<{$}|>{$}r<{$}>{$}r<{$}|>{$}r<{$}|}
\hline
i&y&v&y^2&v^2&yv\\
\hline
1& 0& 28&  0& 784&  0\\
2& 5& 26& 25& 676&130\\
3&10& 25&100& 625&250\\
4&15& 22&225& 484&330\\
\hline
 &30&101&350&2569&710\\
\hline
\end{tabular}
\end{center}
Die Formeln für die lineare Regression liefern jetzt für die Koeffizienten
$a$ und $b$
\begin{align*}
a&=\frac{n\sum yv -\sum y\sum v}{n\sum y^2-(\sum y)^2}
=\frac{4\cdot 710- 30\cdot 101}{4\cdot 350 - 30^2}
=-\frac{190}{500}=-0.38\\
b&=\frac1n\sum v-a\frac1n\sum y
=\frac14\cdot101 +0.38\cdot\frac14\cdot 30
=8.35
\end{align*}
Damit lassens ich jetzt die Teilaufgaben beantworten:
\begin{teilaufgaben}
\item Wir berechnen den Pixelwert für $y=8$:
\[
v=ay+b=-0.38\cdot 8 + 28.1=25.06.
\]
\item 
Die Zunahme pro 100 Zeilen wird durch den Wert $-a$ angegeben, pro
Zeile nimmt das Signal also um $0.0038$.
\item 
Die Qualität kann mit dem Regressionskoeffizienten beurteilt werden.
Dieser ist:
\begin{align*}
r&=\frac{n\sum yv-\sum y\sum v}{\sqrt{n\sum y^2-(\sum y)^2}\sqrt{n\sum v^2-(\sum v)^2}}
\\
&=\frac{4\cdot 710-30\cdot 101}{\sqrt{4\cdot 350-30^2}\sqrt{4\cdot 2569 - 101^2}}
=\frac{-190}{\sqrt{500}\sqrt{75}}=-0.981155781.
\end{align*}
Da dieser Wert sehr nahe bei $-1$ ist, kann man von einer guten Approximation
ausgehen.
\qedhere
\end{teilaufgaben}
\end{loesung}

\begin{bewertung}
Lösungsansatz mit Linearer Regression ({\bf L}) 1 Punkt,
Berechnung der Koeffizienten $a$ und $b$ ({\bf A} und {\bf B}) 2 Punkte,
jede Teilaufgabe 1 Punkt.
\end{bewertung}
