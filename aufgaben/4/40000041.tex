Die Zustimmung zur Politik des amerikanischen Präsidenten Donald Trump
hat seit seiner Amtseinführung ständig abgenommen:
\begin{center}
\begin{tabular}{cc}
\hline
Monat seit Amtseinführung&Zustimmung in \%\\
\hline
1& 28\\
2& 27\\
3& 24\\
4& 22\\
\hline
\end{tabular}
\end{center}
\begin{teilaufgaben}
\item
Wenn wird die Zustimmung auf $0$ gesunken sein?
\item
Ist dies eine zuverlässige Schätzung?
\end{teilaufgaben}

\thema{lineare Regression}

\begin{loesung}
Wir nehmen an, dass die Zustimmung linear fällt, und berechnen daher eine
Regressionsgerade für diese Daten.
Dazu besimmen wir 
\begin{center}
\begin{tabular}{|>{$}r<{$}|>{$}r<{$}>{$}r<{$}|>{$}r<{$}>{$}r<{$}|>{$}r<{$}|}
\hline
i&x_i&y_i&x_i^2&y_i^2&x_iy_i\\
\hline
1&  1& 28&    1&  784&    28\\
2&  2& 27&    4&  729&    54\\
3&  3& 24&    9&  576&    72\\
4&  4& 22&   16&  484&    88\\
\hline
 & 10&101&   30& 2573&   242\\
\hline
\end{tabular}
\end{center}
Die Formeln für die Regressionsgerade liefern $a$ und $b$ wie folgt:
\begin{align*}
a
&=
\frac{
\frac14\cdot 242 -\frac14\cdot 10\cdot\frac14\cdot 101
}{
\frac14\cdot 30-(\frac14\cdot 10)^2
}
=
\frac{4\cdot 242-10\cdot 101}{4\cdot 30-10^2}
=
\frac{968-1010}{120-100}
=
-\frac{42}{20}=-2.1
\\
b&=E(Y)-E(X)a = \frac14\cdot 101 +2.1\cdot\frac14\cdot 10
=
30.5
\end{align*}
Die Regressionsgerade hat also die Gleichung
\[
y=-2.1x+30.5.
\]
\begin{teilaufgaben}
\item
Der Wert $y=0$ wird erreicht für $x=-30.5/(-2.1)=14.52$, also
erst im April 2018.
\item
Der Regressionskoeffizient ist
\begin{align*}
r
&=
\frac{4\cdot 242 - 10\cdot 101}{\sqrt{4\cdot 30-10^2}\sqrt{4\cdot 2573-101^2}}
=
\frac{-42}{\sqrt{20}\sqrt{91}}
=
-0.9899.
\end{align*}
Da dieser Wert sehr nahe bei $\pm 1$ liegt, kann man dieser Prognose eine
hohe Vertrauenswürdigkeit beimessen.
\qedhere
\end{teilaufgaben}
\end{loesung}

\begin{bewertung}
Lineare Regression ({\bf L}) 1 Punkt,
Berechnung der Summen ({\bf S}) 1 Punkt,
Berechnung von $a$ ({\bf A}) 1 Punkt,
Berechnung von $b$ ({\bf B}) 1 Punkt,
Regressionsgerade und Prognose ({\bf P}) 1 Punkt,
Regressionskoeffizient und Beurteilung ({\bf R}) 1 Punkt.
\end{bewertung}
