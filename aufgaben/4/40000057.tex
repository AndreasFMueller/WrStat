Gibt es einen Zusammenhang zwischen dem Budget für TV-Werbung und
Verkäufen?
Die Tabelle
\begin{center}
\begin{tabular}{|>{$}c<{$}|>{$}r<{$}>{$}r<{$}|}
\hline
   &\text{TV-Werbeausgaben}&\text{Verkäufe}\\
\hline
 0 & 230.1 & 22.1 \\
 1 &  44.5 & 10.4 \\
 2 &  17.2 & 12.0 \\
 3 & 151.5 & 16.5 \\
 4 & 180.8 & 17.9 \\
\hline
\end{tabular}
\end{center}
gibt Auskunft.
\begin{teilaufgaben}
\item
Finden Sie eine einfaches Modell für die Abhängigkeit der Verkäufe vom
TV-Werbebudget.
\item
Wieviele Verkäufe prognostiziert Ihr Modell für ein TV-Werbebudget von
von $100$?
\item
Welches Werbebudget ist nötig, damit die Verkäufe den Wert
$20$ erreichen.
\item
Beurteilen Sie die Qualität Ihres Modells.
\end{teilaufgaben}

\begin{loesung}
Die Punkte und die Regressionsgerade sind in Abbildung~\ref{40000057:fig}
dargestellt.
\def\dx{0.05}
\def\dy{0.25}
\ainput{daten.tex}
\definecolor{darkred}{rgb}{0.8,0,0}
\def\p#1#2{ ({\dx*(#1)},{\dy*(#2)}) }
\def\punkt#1#2{ \fill[color=darkred] \p{#1}{#2} circle[radius=0.08]; }
\begin{figure}
\centering
\begin{tikzpicture}[>=latex,thick]
\draw[line width=0.3pt] \p{\ux}{0} -- \p{\ux}{\uy} -- \p{0}{\uy};
\draw[line width=0.3pt] \p{\hx}{0} -- \p{\hx}{\hy} -- \p{0}{\hy};
\draw[->] \p{-1}{0} -- \p{270}{0} coordinate[label={$x$}];
\draw[->] \p{0}{-0.5} -- \p{0}{25} coordinate[label={right:$y$}];
\foreach \x in {50,100,...,250}{
	\draw \p{\x}{-0.3} -- \p{\x}{0.3};
	\node at \p{\x}{-0.3} [below] {$\x$};
}
\foreach \y in {0,5,...,20}{
	\draw \p{-1}{\y} -- \p{1}{\y};
	\node at \p{-1}{\y} [left] {$\y$};
}
\begin{scope}
\clip \p{0}{0} rectangle \p{250}{23.3};
\gerade{blue}
\end{scope}
\punkte
\end{tikzpicture}
\caption{Datenpunkte und Regressionsgerade für Aufgabe~\ref{40000057}.
\label{40000057:fig}}
\end{figure}
\begin{teilaufgaben}
\item
Mit der Berechnungstabelle
\begin{center}
\begin{tabular}{|>{$}c<{$}|>{$}r<{$}>{$}r<{$}|>{$}r<{$}>{$}r<{$}|>{$}r<{$}|}
\hline
i&x_i&y_i&x_i^2&y_i^2&x_iy_i\\
\hline
\tabelle
\hline
\summen
\hline
\mittelwerte
\hline
\end{tabular}
\end{center}
lassen sich die Koeffizienten eines linearen Modells wie folgt
berechnen:
\begin{align*}
a
&=
\frac{E(XY)-E(X)E(Y)}{E(X^2)-E(X)^2}
=
\steigung,
\\
b&=
E(Y)-aE(X)
=
\achsabschnitt.
\end{align*}
\item
Aus den Koeffizienten des Modells ergibt sich für das Budget $x_5=\ux$
der Wert
\[
y_5 = ax_5+b = \steigung\cdot \ux + \achsabschnitt = \uy.
\]
\item
Auflösen der Gleichung $ax+b=100$ nach $x$ ergibt
\[
x = \frac{100-b}{a} = \hx.
\]
\item
Mit $r^2=\rr$ ist die Qualität nicht besonders gut.
\qedhere
\end{teilaufgaben}
\end{loesung}

\begin{bewertung}
Lineare Regression ({\bf LR}) 1 Punkt,
Berechnung von $a$ ({\bf A}) 1 Punkt,
Berechnung von $b$ ({\bf B}) 1 Punkt,
Berechnung von $y_u$ in Teilaufgabe b) ({\bf Y}) 1 Punkt,
Berechnung von $h$ in Teilaufgabe c) ({\bf H}) 1 Punkt,
Beurteilung mit dem Regressionskoeffizienten in Teilaufgabe d)
({\bf R}) 1 Punkt.
\end{bewertung}


