Gemäss der Daten der Tabelle
\begin{center}
\begin{tabular}{|>{$}c<{$}|l|>{$}r<{$}>{$}r<{$}|}
\hline
i&Wasserfall&x_i&y_i\\
\hline
1&Fort Greeley      &    5 &   0.03 \\
2&Firehole          &   13 &   0.05 \\
3&Canadian Niagara  &   50 &   0.16 \\
4&Yosemite          &   72 &   0.25 \\
\hline
\end{tabular}
\end{center}
scheint es einen Zusammenhang zwischen der Fallhöhe $x$ eines Wasserfalls
und der Wellenlänge $y$ der dominierenden Bodenfrequenz zu geben, die der
Aufprall des Wasserfalls erzeugt.
\begin{teilaufgaben}
\item
Bestimmen Sie ein lineares Modell für diesen Zusammenhang.
\item
Welche Wellenlänge erwarten Sie für den {\em Upper Yellowstone},
der eine Fallhöhe von 35\,m aufweist?
\item
Wie hoch müsste nach diesem Modell die Fallhöhe eines Wasserfalls sein,
dessen dominierende Frequenz die Wellenlänge 0.4 hat?
\item
Beurteilen Sie die Qualität Ihres Modells.
\end{teilaufgaben}

\begin{loesung}
Die Punkte und die Regressionsgerade sind in Abbildung~\ref{40000054:fig}
dargestellt.
\ainput{daten.tex}
\definecolor{darkred}{rgb}{0.8,0,0}
\def\dx{0.1}
\def\dy{10}
\def\p#1#2{ ({\dx*(#1)},{\dy*(#2)}) }
\def\punkt#1#2{ \fill[color=darkred] \p{#1}{#2} circle[radius=0.08]; }
\begin{figure}
\centering
\begin{tikzpicture}[>=latex,thick]
\draw[line width=0.3pt] \p{35}{0} -- \p{35}{\uy} -- \p{0}{\uy};
\draw[line width=0.3pt] \p{\h}{0} -- \p{\h}{0.4} -- \p{0}{0.4};
\draw[->] (-0.1,0) -- \p{130}{0} coordinate[label={$x$}];
\draw[->] (0,-0.1) -- \p{0}{0.42} coordinate[label={right:$y$}];
\foreach \x in {10,20,...,120}{
	\draw \p{\x}{-0.01} -- \p{\x}{0.01};
	\node at \p{\x}{-0.01} [below] {$\x$};
}
\foreach \y in {1,...,4}{
	\draw \p{-1}{0.1*\y} -- \p{1}{0.1*\y};
	\node at \p{-1}{0.1*\y} [left] {$0.\y$};
}
\gerade{blue}
\punkte
\end{tikzpicture}
\caption{Datenpunkte und Regressionsgerade für Aufgabe~\ref{40000054}.
\label{40000054:fig}}
\end{figure}
\begin{teilaufgaben}
\item
Mit der Berechnungstabelle
\begin{center}
\begin{tabular}{|>{$}c<{$}|>{$}r<{$}>{$}r<{$}|>{$}r<{$}>{$}r<{$}|>{$}r<{$}|}
\hline
i&x_i&y_i&x_i^2&y_i^2&x_iy_i\\
\hline
\tabelle
\hline
\summen
\hline
\mittelwerte
\hline
\end{tabular}
\end{center}
lassen sich die Koeffizienten eines linearen Modells wie folgt
berechnen:
\begin{align*}
a
&=
\frac{E(XY)-E(X)E(Y)}{E(X^2)-E(X)^2}
=
\steigung,
\\
b&=
E(Y)-aE(X)
=
\achsabschnitt.
\end{align*}
\item
Aus den Koeffizienten des Modells ergibt sich für die Höhe $x_u=35$
der Wert
\[
y_u = ax_u+b = \steigung\cdot 35 + \achsabschnitt = \uy.
\]
\item
Auflösen der Gleichung $ah+b=0.4$ nach $h$ ergibt
\[
h = \frac{0.4-b}{a} = \h.
\]
\item
Mit $r^2=\rr$ ist die Qualität recht gut.
Allerdings ist der Datensatz auch sehr klein und möglicherweise sogar
selektiert.
\qedhere
\end{teilaufgaben}
\end{loesung}

\begin{bewertung}
Lineare Regression ({\bf LR}) 1 Punkt,
Berechnung von $a$ ({\bf A}) 1 Punkt,
Berechnung von $b$ ({\bf B}) 1 Punkt,
Berechnung von $y_u$ in Teilaufgabe b) ({\bf Y}) 1 Punkt,
Berechnung von $h$ in Teilaufgabe c) ({\bf H}) 1 Punkt,
Beurteilung mit dem Regressionskoeffizienten in Teilaufgabe d)
({\bf R}) 1 Punkt.
\end{bewertung}


