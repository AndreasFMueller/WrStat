Ein Ingenieur hat die Idee, ein Autofocus-System für unendlich weit
entfernte punktförmige
Lichtquellen (Sterne) nach folgendem Prinzip zu bauen.
Solange das System nicht scharf gestellt ist,
ist das Bild des Sterns eine Kreisfläche.
Mit einem Schrittmotor kann das Objektiv bewegt werden, wodurch der Kreis
seinen Durchmesser verändert. Die Messung des Durchmessers ist 
natürlich mit einem Fehler behaftet (Pixelgrösse des CCD Sensors,
Luftunruhe).
Für verschiedene Positionen des Schrittmotors misst man folgende
Durchmesser:
\begin{center}
\begin{tabular}{|c|c|}
\hline
Position&Durchmesser\\
\hline
10&22\\
20&17\\
30&12\\
40&5\\
\hline
\end{tabular}
\end{center}
Welche Schrittmotorposition soll angefahren werden, um Scharfstellung
(Focus) zu erreichen?

\thema{lineare Regression}

\begin{loesung}
Man kann davon ausgehen, dass der Zusammenhang zwischen Durchmesser
des Bildes und der Position des Objektivs linear ist, dies ist der
Strahlensatz. Wir bezeichnen die Position mit $x$ und den Durchmesser
mit $y$ und suchen eine bestmögliche Approximation $y=ax+b$ mit
Hilfe linearer Regression. Dazu muss man die Summen der Messwerte und
ihrer Quadrate und Produkte ermitteln.
\begin{center}
\begin{tabular}{|c|rr|rrr|}
\hline
$i$&$x_i$&$y_i$&$x_i^2$&$y_i^2$&$x_iy_i$\\
\hline
1& 10&22& 100&484& 220\\
2& 20&17& 400&289& 340\\
3& 30&12& 900&144& 360\\
4& 40& 5&1600& 25& 200\\
\hline
 &100&56&3000&942&1120\\
\hline
\end{tabular}
\end{center}
Aus den Formeln für die lineare Regression findet man jetzt die
Koeffizienten $a$ und $b$:
\begin{align*}
a&=\frac{
%\displaystyle
4\sum_{i=1}^4 x_iy_i-\sum_{i=1}^4x_i\sum_{i=1}^4y_i
}{
%\displaystyle
4\sum_{i=1}^4x_i^2-\bigl(\sum_{i=1}^4 x_i\bigr)^2
}
=
\frac{4\cdot 1120-100\cdot 56}{4\cdot 3000-100^2}=\frac{-1120}{2000}=-0.56
\\
b&=\frac14\sum_{i=1}^4y_i-a\frac14\sum_{i=1}^4x_i
=14 - a\cdot 25=28
\end{align*}
Die optimale lineare Approximation ist also $y=28-0.56x$.
Fokus bedeutet Durchmesser $y=0$, dies wird für die Schrittmotorposition
$x_f$ erreicht:
\[
ax_f+b=0\quad\Rightarrow\quad x_f=-\frac{b}{a}=\frac{28}{0.56}=50
\qedhere
\]
\end{loesung}
