Je w"armer es ist, desto schneller zirpen die Grillen.
Folgende Daten wurden erhoben:
\begin{center}
\begin{tabular}{>{$}c<{$}|>{$}c<{$}}
T\text{[$\mathstrut^\circ$C]}&N \text{[Zirpen/15 Sekunden]}\\
\hline
%14.44&18\\
%15.55&20\\
15&20\\
%17.77&21\\
%18.88&23\\
19&23\\
%20&27\\
%21.66&30\\
22&30\\
%23.88&34\\
25&39\\
\hline
\end{tabular}
\end{center}
Es wird vermutet, dass die Anzahl der Zirplaute pro 15 Sekunden
linear von der Temperatur abh"angt.
Finden Sie eine solche Gesetzm"assigkeit und beurteilen Sie ihre Qualit"at.

\begin{loesung}
Ein lineare Abh"angigkeit kann mit der Methode der linearen Regression
gefunden werden.
Dazu m"ussen die Summen der Messwerte und ihrer Produkte berechnet 
werden:
\begin{center}
\begin{tabular}{|>{$}r<{$}|>{$}r<{$}>{$}r<{$}|>{$}r<{$}>{$}r<{$}|>{$}r<{$}|}
\hline
     i&t_i&n_i&t_i^2& n_i^2&t_in_i\\
\hline
     1& 15& 20&  225&  400&    300\\
     2& 19& 23&  361&  529&    437\\
     3& 22& 30&  484&  900&    660\\
     4& 25& 39&  625& 1521&    975\\
\hline
\Sigma& 81&112& 1695& 3350&   2372\\
\hline
\end{tabular}
\end{center}
Mit diesen Daten kann man jetzt die Koeffizienten des besten linearen
Zusammenhangs zwischen $T$ und $N$ ermitteln:
\begin{align*}
a&=\frac{E(TN)-E(T)E(N)}{E(T^2)-E(T)^2}
=\frac{\frac14\cdot 2372-\frac14\cdot81\cdot\frac14\cdot112}{\frac14\cdot1695-\frac1{4^2}81^2}
=\frac{4\cdot 2372-81\cdot 112}{4\cdot 1695-81^2}
=\frac{416}{219}=1.89954337899
\\
b&=E(N)-aE(T)=\frac14\cdot 112 - a \frac14\cdot 81=-10.46575.
\end{align*}
Man kommt so auf einen linearen Zusammenhang der Form
\[
N=1.8995\cdot T-10.465
\]
Die Qualit"at kann man mit Hilfe des Regressionskoeffizienten $r$ beurteilen.
Es gilt
\begin{align*}
r
&=
\frac{\operatorname{cov}(T,N)}{\sqrt{\operatorname{var}(T)\operatorname{var}(N)}}
\frac{ E(NT) - E(N)E(T) }{\sqrt{E(N^2)-E(N)^2}\sqrt{E(T^2)-E(T)^2}}
=
\frac{4\cdot 2372-112\cdot 81}{\sqrt{4\cdot 3350-112^2}\sqrt{4\cdot 1695-81^2}}
\\
&=
\frac{416}{\sqrt{856}\sqrt{219}}
=
0.96080
\end{align*}
Dies zeigt, dass die lineare Approximation recht gut ist.
Dies best"atigt auch die Abbildung~\ref{40000034:linear}.
\begin{figure}
\centering
\includeagraphics[]{graph-1.pdf}
\caption{Linearer Zusammenhang zwischen Temperatur und Zirph"aufigkeit
\label{40000034:linear}}
\end{figure}
\end{loesung}

\begin{diskussion}
Dies ist eine Adaption einer Aufgabe aus der zweiten Auflage von 
{\em Statistics for Dummies}.
\end{diskussion}

\begin{bewertung}
Ansatz Lineare Regression ({\bf LR}) 1 Punkt,
Berechnung der Summen ({\bf T}) 2 Punkte,
Berechnung von $a$ ({\bf A}) 1 Punkt,
Berechnung von $b$ ({\bf B}) 1 Punkt,
Regressionskoeffizient als Mass f"ur Qualit"at ({\bf R}) 1 Punkt.
\end{bewertung}


