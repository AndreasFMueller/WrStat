Im Youtube-Video
\url{https://www.youtube.com/watch?v=lU-hIq2STHk}
wird ein Spiel mit einem Glücksrad beschrieben.
Ein Glücksrad hat drei gleich grosse Sektoren, die beschriftet sind mit
den Gewinnsummen $1$ und $2$, und mit {\em Ende}.
Man spielt so lange, bis zum ersten Mal {\em Ende} erscheint, und gewinnt 
die bis zu diesem Zeitpunkt angesammelte Summe.
Wie gross ist der erwartete Gewinn?

\thema{Laplace-Experiment}
\thema{Erwartungswert}

\begin{loesung}
Sei $E_k$ das Ereignis, dass das Spiel genau im $k$-ten Versuch endet.
Seine Wahrscheinlichkeit ist
\[
P(E_k)=\biggl(\frac23\biggr)^k\frac13.
\]
In jeder Runde, die das Spiel nicht beendet, erwartet man einen Gewinn von $1.5$.
Wenn das Spiel beim $k$-ten Versuch endet, ist der zu erwartende Gewinn also
$\frac32\cdot k$.
Der erwartete Gewinn ist daher
\begin{equation}
E(G)
=
\sum_{k=1}^\infty \frac32k\biggl(\frac23\biggr)^k\frac13
=
\frac13\sum_{k=1}^\infty kq^{k-1}
=
\frac13\sum_{k=0}^\infty kq^{k-1}
\label{40000035:erwartung}
\end{equation}
mit $q=\frac23$.
Die Summe kann mit dem Ableitungstrick wie folgt berechnet werden
\begin{align*}
\sum_{k=1}^\infty kq^{k-1}
&=
\frac{d}{dq} \sum_{k=0}^\infty q^k
=
\frac{d}{dq} \frac{1}{1-q}
=
\frac1{(1-q)^2}.
\end{align*}
Einsetzen in (\ref{40000035:erwartung}) ergibt
\[
E(G)=\frac13\frac1{(1-\frac23)^2}=3.
\]

Das Video schlägt eine andere Lösung vor.  
Es nennt den erwarteten Gewinn $g$, und analysiert dann den Spiel-Verlauf mit
Hilfe der bedingten Wahrscheinlichkeit.
Sie $X$ die Zufallsvariable, die den Ausgang des nächsten Spieles beschreibt.
Der erwartete Gewinn nach dem nächsten Spiel ist dann
\begin{align*}
E(G)
&=
E(G+1|X=1)+E(G+2|X=2)+E(G|X=\text{\em Ende})
\\
&=
\frac13(E(G)+1) + \frac13(E(G)+2)
\\
\frac13E(G)
&=\frac13(1+2)=1
\\
\Rightarrow\qquad E(G)&=3.
\qedhere
\end{align*}
\end{loesung}

