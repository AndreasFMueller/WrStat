Die einzelnen Pixel eines CCD-Chips einer Digitalkamera wandeln einfallende
Photonen in Elektronen um, jedoch nicht perfekt, nur ein Teil der Photonen
wird tats"achlich umgesetzt. Die sogenannte {\it Quanteneffizienz} eines
CCD-Chips ist die Wahrscheinlichkeit, dass ein einfallendes Photon
auch tats"achlich zu einem Elektron wird.

Um die Quanteneffizienz eines Chips zu messen, wird folgendes Experiment
durchgef"uhrt. Der Chip wird mit eine Sekunde lang mit Licht bekannter
Intensit"at bestrahlt und dann ausgelesen. Dies wiederholt man f"ur
verschiedene Intensit"aten. Die gemessene Elektronenzahl sollte
linear von der Intensit"at abh"angen. Aus Aufgabe 2 wissen wir allerdings,
dass da noch Fehler hinzukommen, die Linearit"at wird also nicht
exakt erf"ullt sein. Gemessen wurden bei einem Pixel folgende Daten
\begin{center}
\begin{tabular}{|c|c|}
\hline
Photonen (in Tausend)&Elektronen\\
\hline
0&$-5$\\
1&601\\
2&1209\\
3&1795\\
4&2373\\
\hline
\end{tabular}
\end{center}
\begin{teilaufgaben}
\item
Wie gross ist die Quanteneffizienz dieses Pixels?
\item
Welche Elektronenzahl erwarten Sie bei einer Belichtung mit
100000 Photonen?
\item
Hat dieser Pixel einen noch nicht korrigierten systematischen Fehler?
\end{teilaufgaben}

\ifthenelse{\boolean{loesungen}}{
\begin{loesung}
Da die Zahl der Elektronen linear von der Zahl der Photonen abh"angt,
ist die Quanteneffizienz die Steigung der Geraden, die Elektronenzahl $Y$
in Abh"angigkeit von Photonenzahl $X$ zeigt. Diese Steigung findet man
optimal mit einer linearen Regression. Dazu sind die Summen der gemessenen
$x$ und $y$ Werte sowie deren Quadrate und Produkte zu berechnen:
\begin{center}
\begin{tabular}{|r|rr|rr|r|}
\hline
$i$&$x_i$&$y_i$&$x_i^2$&$y_i^2$&$x_iy_i$\\
\hline
0&0&   -5&  0&     25&    0\\
1&1&  601&  1& 361201&  601\\
2&2& 1209&  4&1461681& 2418\\
3&3& 1795&  9&3222025& 5385\\
4&4& 2373& 16&5631129& 9492\\
\hline
$E$&2& 1194.6&6&2135212&3579.2\\
\hline
\end{tabular}
\end{center}
Aus den Formeln f"ur die Regressionsgerade kann man jetzt die
Steigung ausrechnen:
\begin{align*}
a&=\frac{E(XY)-E(X)E(Y)}{E(X^2)-E(X)^2}=
\frac{3579.2-2 \cdot 1194.6}{6-2^2}=\frac{1190}{2}=595
\\
b&=E(Y)-E(X)a=1194.6-2 \cdot 595=4.6
\end{align*}
Mit diesem Resultat kann man jetzt die einzelnen Fragen beantworten.
\begin{teilaufgaben}
\item Die Quanteneffizienz ist $595$ Elektronen pro 1000 Photonen, also
$0.595$.
\item
Aus den Koeffizienten $a$ und $b$ bekomen wir f"ur $x=100000$
$y=ax+b=595\cdot 100+4.6=59504.6$.
\item
Ja, es werden im Mittel 4.6 Elektronen zu viel gemessen. Im Vergleich
zur Gesamtzahl der Elektronen ist dies jedoch sehr wenig.
\end{teilaufgaben}
\end{loesung}
}{}
