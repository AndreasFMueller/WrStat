Die einzelnen Pixel eines CCD-Chips einer Digitalkamera wandeln einfallende
Photonen in Elektronen um, jedoch nicht perfekt, nur ein Teil der Photonen
wird tatsächlich umgesetzt. Die sogenannte {\it Quanteneffizienz} eines
CCD-Chips ist die Wahrscheinlichkeit, dass ein einfallendes Photon
auch tatsächlich zu einem Elektron wird.

Um die Quanteneffizienz eines Chips zu messen, wird folgendes Experiment
durchgeführt.
Der Chip wird eine Sekunde lang mit Licht bekannter
Intensität bestrahlt und dann ausgelesen.
Dies wiederholt man für verschiedene Intensitäten.
Die gemessene Elektronenzahl sollte linear von der Intensität abhängen.
Technisch bedingt werden da noch Fehler hinzukommen
(Rauschen, nicht optimale Transfer-Effizienz im CCD-Chip,
Nichtlinearität für stark ``gefüllte'' Pixel),
die Linearität wird also nicht exakt erfüllt sein.
Gemessen wurden bei einem Pixel folgende Daten
\begin{center}
\begin{tabular}{|c|c|}
\hline
Photonen (in Tausend)&Elektronen\\
\hline
0&$-5$\\
1&601\\
2&1209\\
3&1795\\
4&2373\\
\hline
\end{tabular}
\end{center}
\begin{teilaufgaben}
\item
Wie gross ist die Quanteneffizienz dieses Pixels?
\item
Welche Elektronenzahl erwarten Sie bei einer Belichtung mit
100000 Photonen?
\item
Hat dieser Pixel einen noch nicht korrigierten systematischen Fehler?
\item
Wie gut ist die Approximation?
\end{teilaufgaben}

\thema{lineare Regression}

\begin{loesung}
Da die Zahl der Elektronen linear von der Zahl der Photonen abhängt,
ist die Quanteneffizienz die Steigung der Geraden, die Elektronenzahl $Y$
in Abhängigkeit von Photonenzahl $X$ zeigt. Diese Steigung findet man
optimal mit einer linearen Regression. Dazu sind die Summen der gemessenen
$x$ und $y$ Werte sowie deren Quadrate und Produkte zu berechnen:
\begin{center}
\begin{tabular}{|r|rr|rr|r|}
\hline
$i$&$x_i$&$y_i$&$x_i^2$&$y_i^2$&$x_iy_i$\\
\hline
0&0&   -5&  0&     25&    0\\
1&1&  601&  1& 361201&  601\\
2&2& 1209&  4&1461681& 2418\\
3&3& 1795&  9&3222025& 5385\\
4&4& 2373& 16&5631129& 9492\\
\hline
$E$&2& 1194.6&6&2135212&3579.2\\
\hline
\end{tabular}
\end{center}
Aus den Formeln für die Regressionsgerade kann man jetzt die
Steigung ausrechnen:
\begin{align*}
a&=\frac{E(XY)-E(X)E(Y)}{E(X^2)-E(X)^2}=
\frac{3579.2-2 \cdot 1194.6}{6-2^2}=\frac{1190}{2}=595
\\
b&=E(Y)-E(X)a=1194.6-2 \cdot 595=4.6
\end{align*}
\begin{figure}
\centering
\includeagraphics[]{graph-1.pdf}
\caption{Regressionsgerade zur Bestimmung der Quanteneffizienz eines CCD-Chips
in Aufgabe~\ref{40000001}.
\label{40000001:graphik}}
\end{figure}
Die Regressionsgerade ist in Abbildung~\ref{40000001:graphik} graphisch
dargestellt.
Mit diesem Resultat kann man jetzt die einzelnen Fragen beantworten.
\begin{teilaufgaben}
\item Die Quanteneffizienz ist $595$ Elektronen pro 1000 Photonen, also
$0.595$.
\item
Aus den Koeffizienten $a$ und $b$ bekomen wir für $x=100000$
$y=ax+b=595\cdot 100+4.6=59504.6$.
\item
Ja, es werden im Mittel 4.6 Elektronen zu viel gemessen. Im Vergleich
zur Gesamtzahl der Elektronen ist dies jedoch sehr wenig.
\item
Die Frage nach der Qualität der Approximation kann mit dem
Regressionskoeffizienten beantwortet werden.
Es ist
\begin{align*}
r
&=
\frac{\operatorname{cov}(X,Y)}{\sqrt{\operatorname{var}(X)\operatorname{var}(Y)}},
\\
\operatorname{var}(X)
&=
\frac1n\sum_{i=1}^nx_i^2-\biggl(\frac1n\sum_{i=1}^nx_i\biggr)^2
=
6-2^2=2
\\
\operatorname{var}(Y)
&=
\frac1n\sum_{i=1}^ny_i^2-\biggl(\frac1n\sum_{i=1}^ny_i\biggr)^2
=
2135212-1194.6^2=708142.84
\\
\operatorname{cov}(X,Y)
&=
E(XY)-E(X)E(Y)=3579.2-2\cdot 2294.6=1190.0
\\
r&=\frac{1190.0}{\sqrt{2\cdot 708142.84}}=0.999934
\end{align*}
Dieser Wert von $r$ liegt extrem nahe bei $1$, die Anpassung ist also von
hervorragender Qualität, was auch die graphische Darstellung
in Abbildung~\ref{40000001:graphik} bestätigt.
\qedhere
\end{teilaufgaben}
\end{loesung}
