Zwei faire Würfel werden geworfen. Die Würfel, die eine Sechs
zeigen, bleiben liegen, mit den anderen wird ein zweites Mal gewürfelt.
Welche Gesamtpunktzahl kann man erwarten?
Wie ändert sich das Resultat, wenn man statt der Sechser die Einer
behält?

\begin{loesung}
Die Aufgabe spricht zwar von zwei Würfeln, aber eigentlich kann man
jeden dieser Würfel für sich betrachten, denn was mit einem Würfel
im Laufe des Spieles passiert hängt nur davon ab, was er selbst
anzeigt.

Wir haben also zwei Fälle zu unterscheiden: entweder zeigt der
Würfel im ersten Wurf eine Sechs (Ereignis $S$), dann bleibt er so liegen,
oder er zeigt keine Sechs, dann wird er nochmals geworfen.
Der erste Fall tritt mit Wahrscheinlichkeit $P(S)=\frac16$ ein,
der zweite mit Wahrscheinlicheit $P(\bar{S})=\frac56$. Im ersten Fall
erziehlt man die Augenzahl $6$, im zweiten die erwartete
Augenzahl eines einzelnen Würfels, also $3.5$. Der erwartete
Wert ist daher
\[
E(X)=6\cdot P(S) + 3.5 \cdot P(\bar S)=
\frac16\cdot 6 +\frac56\cdot\frac72=\frac{12+35}{12}=\frac{47}{12}=3.916666
\]
Nun ist aber der Erwartungswert zweier solcher Würfel
gefragt, also das Doppelte: $7.833333$.

Wenn man die Einer statt die Sechser behalten will, könnte man
auch das bisherige Spiel spielen, dann aber die Augenzahl
$x$ durch $7-x$ ersetzen. Dann wird für jeden Würfel auch
die erwartete Augenzahl durch $7-E(X)$ ersetzt, für zwei
Würfel erhalten wir also $12-7.833333 = 6.166666$.

Will man trotzdem die drei Fälle gesondert untersuchen (wie
viele Prüfungsteilnehmer es versucht haben), kann man wie
folgt vorgehen.
Als Resultat des ersten Wurfes sind drei Fälle zu betrachten, die
der Anzahl der Sechser in diesem Wurf entsprechen.
\begin{itemize}
\item[0:] Kein Sechser, Ereignis $S_0$, dieser Fall tritt mit Wahrscheinlichkeit
$\frac{25}{36}$ ein.  In diesem Fall wirft man beide Würfel nochmals,
dabei ist die erwartete Augenzahl $7$.
\item[1:] Genau ein Sechser, Ereignis $S_1$,
dieser Fall tritt mit Wahrscheinlichkeit
$\frac{10}{36}$ ein.  In diesem Fall wirft man einen Würfel nochmals,
wobei man im Mittel die Augenzahl $3.5$ erwartet.  Zusammen mit dem
bereits erziehlten Sechser ist die erwartete Gesamtpunktzahl in diesem
Fall also $9.5$.
\item[2:] Genau zwei Sechser, Ereignis $S_2$,
dieser Fall tritt mit Wahrscheinlichkeit
$\frac{1}{36}$ ein.  In diesem Fall wirft man nicht nochmals, die
erziehlte Punktzahl ist $12$.
\end{itemize}
Aus diesen Fällen kann man jetzt die erwartete Punktzahl zusammensetzen:
\begin{align*}
E(X)&=
7 \cdot P(S_0) + (6 + 3.5)\cdot P(S_1) + 12\cdot P(S_2)
=
7\cdot \frac{25}{36}
+
9.5\cdot \frac{10}{36}
+
12\cdot\frac{1}{36}
\\
&=
\frac{175+95+12}{36}=\frac{282}{36}=\frac{47}{6}
=7.8333333
\end{align*}
Für die Einer ändert sich der erwartete Wert wie folgt:
\[
7\cdot \frac{25}{36}
+
4.5\cdot \frac{10}{36}
+
2\cdot\frac{1}{36}
=
\frac{175+45+2}{36}=\frac{222}{36}=\frac{37}{6}
=6.1666666.
\]
In beiden Fällen verschiebt sich das Resultat in Richtung der Zahlen,
die auf dem Tisch liegen bleiben. Wenn man alles ausser Sechser nochmals
würfeln darf, dann steigt die Wahrscheinlichkeit, Sechser zu erreichen,
und der Erwartungswert steigt von $7$ auf $7.83$. Darf man umgekehrt alles
ausser Einer nochmals würfeln, dann steigt die Wahrscheinlichkeit,
Einer zu erreichen, und der Erwartungswert sinkt um den symmetrischen Betrag
von $7$ auf $6.17$.
\end{loesung}

