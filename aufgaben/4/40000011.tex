Zwei faire W"urfel werden geworfen. Die W"urfel, die eine Sechs
zeigen, bleiben liegen, mit den anderen wird ein zweites Mal gew"urfelt.
Welche Gesamtpunktzahl kann man erwarten?
Wie "andert sich das Resultat, wenn man statt der Sechser die Einer
beh"alt?

\begin{loesung}
Die Aufgabe spricht zwar von zwei W"urfeln, aber eigentlich kann man
jeden dieser W"urfel f"ur sich betrachten, denn was mit einem W"urfel
im Laufe des Spieles passiert h"angt nur davon ab, was er selbst
anzeigt.

Wir haben also zwei F"alle zu unterscheiden: entweder zeigt der
W"urfel im ersten Wurf eine Sechs, dann bleibt er so liegen,
oder er zeigt keine Sechs, dann wird er nochmals geworfen.
Der erste Fall tritt mit Wahrscheinlichkeit $\frac16$ ein,
der zweite mit Wahrscheinlicheit $\frac56$. Im ersten Fall
erziehlt man die Augenzahl $6$, im zweiten die erwartete
Augenzahl eines einzelnen W"urfels, also $3.5$. Der erwartete
Wert ist daher
\[
\frac16\cdot 6 +\frac56\cdot\frac72=\frac{12+35}{12}=\frac{47}{12}=3.916666
\]
Nun ist aber der Erwartungswert zweier solcher W"urfel
gefragt, also das Doppelte: $7.833333$.

Wenn man die Einer statt die Sechser behalten will, k"onnte man
auch das bisherige Spiel spielen, dann aber die Augenzahl
$x$ durch $7-x$ ersetzen. Dann wird f"ur jeden W"urfel auch
die erwartete Augenzahl durch $7-E(X)$ ersetzt, f"ur zwei
W"urfel erhalten wir also $12-7.833333 = 6.166666$.

Will man trotzdem die drei F"alle gesondert untersuchen (wie
viele Pr"fungsteilnehmer es versucht haben), kann man wie
folgt vorgehen.
Als Resultat des ersten Wurfes sind drei F"alle zu betrachten, die
der Anzahl der Sechser in diesem Wurf entsprechen.
\begin{itemize}
\item[0:] Kein Sechser, dieser Fall tritt mit Wahrscheinlichkeit
$\frac{25}{36}$ ein.  In diesem Fall wirft man beide W"urfel nochmals,
dabei ist die erwartete Augenzahl $7$.
\item[1:] Genau ein Sechser, dieser Fall tritt mit Wahrscheinlichkeit
$\frac{10}{36}$ ein.  In diesem Fall wirft man einen W"urfel nochmals,
wobei man im Mittel die Augenzahl $3.5$ erwartet.  Zusammen mit dem
bereits erziehlten Sechser ist die erwartete Gesamtpunktzahl in diesem
Fall also $9.5$.
\item[2:] Genau zwei Sechser, dieser Fall tritt mit Wahrscheinlichkeit
$\frac{1}{36}$ ein.  In diesem Fall wirft man nicht nochmals, die
erziehlte Punktzahl ist $12$.
\end{itemize}
Aus diesen F"allen kann man jetzt die erwartete Punktzahl zusammensetzen:
\[
7\cdot \frac{25}{36}
+
9.5\cdot \frac{10}{36}
+
12\cdot\frac{1}{36}
=
\frac{175+95+12}{36}=\frac{282}{36}=\frac{47}{6}
=7.8333333
\]
F"ur die Einer "andert sich der erwartete Wert wie folgt:
\[
7\cdot \frac{25}{36}
+
4.5\cdot \frac{10}{36}
+
2\cdot\frac{1}{36}
=
\frac{175+45+2}{36}=\frac{222}{36}=\frac{37}{6}
=6.1666666.
\]
Offenbar ist die Resultatverbesserung bei den Einern viel dramatischer
als bei den Sechsern.
\end{loesung}

