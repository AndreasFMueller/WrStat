In Aufgabe~\ref{40000024} wurde ein Experiment betrachtet, welches
$n$ mal durchgef"uhrt wird, wobei gez"ahlt wurde, wie oft das Ereignis
$A$, welches Wahrscheinlichkeit $P(A)=p$ hat, eingetreten ist. Die
betrachtete Zufallsvariable $X$ war die Anzahl ``Kopf''. Es wurde
mit Hilfe eines Tricks gefunden, dass $E(X)=np$.
\begin{teilaufgaben}
\item
K"onnen Sie diesen Trick verallgemeinern, so dass sich damit auch $E(X^2)$
berechnen l"asst?
\item Wie gross ist $\operatorname{var}(X)$?
\end{teilaufgaben}

\begin{loesung}
\begin{teilaufgaben}
\item
Man musste dort
\[
E(X)=\sum_{k=0}^nk\cdot \binom{n}{k}p^k(1-p)^{n-k}
\]
berechnen.
Der Trick bestand darin, die binomische Formel abzuleiten:
\[
\frac{d}{dx}(x+y)^n=n(x+y)^{n-1}
=\frac{d}{dx}\sum_{k=0}^n\binom{n}{k}x^ky^{n-k}
=\sum_{k=0}^nk\binom{n}{k}x^{k-1}y^{n-k}
\]
Bis auf einen fehlenden Faktor $x$ ergibt die rechte Seite genau
die gesuchten Ausdruck, wenn man $x=p$ und $y=1-p$ setzt. Multipliziert
man mit $x$ bekommt man
\begin{equation}
nx(x+y)^{n-1}
=
\sum_{k=0}^nk\binom{n}{k}x^ky^{n-k}.
\label{formel}
\end{equation}

Um jetzt die $E(X^2)$ zu berechnen, muss die Summe
\[
E(X^2)=\sum_{k=0}^n\underbrace{\phantom{\binom{n}{k}}k^2}_{\text{Wert}}\cdot\underbrace{\binom{n}{k}p^k(1-p)^{n-k}}_{\text{Wahrscheinlichkeit}}
\]
ausgewertet werden. Dazu leitet man den Ausdruck (\ref{formel}) nochmals
ab, und bekommt
\[
n(x+y)^{n-1}+n(n-1)x(x+y)^{n-2}
=
\sum_{k=0}^nk^2\cdot \binom{n}{k}x^{k-1}y^{n-k}.
\]
Wieder steht auf der rechten Seite bis auf einen Faktor $x$ der gesuchte
Ausdruck, oder
\[
nx(x+y)^{n-1}+n(n-1)x^2(x+y)^{n-2}
=
\sum_{k=0}^nk^2\cdot \binom{n}{k}x^{k}y^{n-k}.
\]
Ersetzt man jetzt wieder $x=p$ und $y=1-p$, bekommt man $x+y=1$ und daher
\[
E(X^2)=np+n(n-1)p^2.
\]
\item
Die Varianz kann man jetzt mit der Definition $\operatorname{var}(X)=E(X^2)-E(X)^2$ ausrechnen:
\[
\operatorname{var}(X)=np+n(n-1)p^2-(np)^2=np +n^2p^2-np^2-n^2p^2=np(1-p).
\qedhere
\]
\end{teilaufgaben}
\end{loesung}

