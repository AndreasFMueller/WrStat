Bei einem Würfelspiel wird mit zwei Würfeln gewürfelt. Wenn
auf beiden Würfeln die gleiche Augenzahl erscheint, man nennt dies
einen Pasch, darf
der Spieler noch ein zweites Mal würfeln.
\begin{teilaufgaben}
\item Wie gross ist die Wahrscheinlichkeit, ein zweites Mal würfeln
zu dürfen (Ereignis $P$)?
\item Wie gross ist die Wahrscheinlichkeit, 10 zu erreichen,
wenn man im ersten Wurf zwei gleiche gewürfelt hat?
\item Wie gross ist die
Wahrscheinlichkeit, mehr als 10 zu würfeln (Ereignis $Z$)?
\end{teilaufgaben}

\thema{Wahrscheinlichkeit}
\thema{Laplace-Experiment}

\begin{loesung}
Teilaufgabe b) kann auf zwei
Arten verstanden werden konnte: ``10 erreicht'' konnte
entweder heissen, dass in beiden Würfen mindestens 10 gewürfelt wurde,
oder es konnte heissen, dass in beiden Würfen exakt 10 gewürfelt wurde.
Dies hat Auswirkungen auf b), nicht aber auf c).

\begin{teilaufgaben}
\item von 36 möglichen Ausgängen beim Würfeln mit zwei Würfeln
führen 6 auf einen Pasch, die Wahrscheinlichkeit eines Pasches
ist also $P(P)=\frac16$.
\item
Wie in der Einleitung zur Lösung erwähnt, betrachten wir hier beide
Möglichkeiten, die Aufgabe zu verstehen. Zunächst der Fall, in
dem mindestens Zehn gewürfelt wird.

Bezeichnet man das Ereignis, die totale Augenzahl 10 zu erreichen (womit
wir meinen, dass wir sie auch übertreffen können), mit
$Q$, dann wird die bedingte Wahrscheinlichkeit $P(Q|P)$ gesucht.

Je nach Augenzahl im ersten Wurf ist die Wahrscheinlichkeit
dafür verschieden, zusammen mit dem zweiten Wurf 10 zu erreichen.
Man muss also das Ereignis $Q$ in die verschieedenen Fälle eines
Pasches zerlegen. Sei also $P_k$ das Ereignis, dass im ersten Wurf
ein $k$-er-Pasch geworfen worden ist. Jede Augenzahl ist in diesem
ersten Wurf gleich wahrscheinlich, also
\[
P(P_1|P) = P(P_2|P) = P(P_3|P)
P(P_4|P) = P(P_5|P) = P(P_6|P)=\frac16.
\]
Bei einem $k$-er-Pasch im ersten Wurf muss man im zweiten Wurf die
Augensumme $10-2k$
erreichen, wenn das Ereignis $Q$ eintreten soll.
Sei $A_l$ das Ereignis, im zweiten Wurf die Augensumme $\ge l$
zu erreichen. Dann ist
\begin{align*}
P(Q|P)
&=P(P_1|P) P(A_8) + P(P_2|P) P(A_6) + P(P_3|P) P(A_4) \\
&\qquad + P(P_4|P)P(A_2) + P(P_5|P) P(A_0) + P(P_6|P) P(A_{-2})
\\
&=\frac1{6}(P(A_8) + P(A_6) + P(A_4) + P(A_2) + 1 + 1),
\end{align*}
da $P(P_k|P)=\frac1{6}$. Für die Wahrscheinlichkeiten $P(A_l)$ muss man
jetzt nur noch zählen, wieviele Kombination von Augenzahlen die
verlangte Augensumme ergeben, die folgende Tabelle zeigt die
Resultate:
\begin{center}
\begin{tabular}{|c|c|}
\hline
$l$&$P(A_l)$\\
\hline
$-2$&1\\
0&1\\
2&1\\
4&$\frac{33}{36}$
\\
6&$\frac{26}{36}$
\\
8&$\frac{15}{36}$
\\
\hline
\end{tabular}
\end{center}
Die gesuchte Wahrscheinlichkeit ist daher
\[
P(Q|P)
=
\frac1{6}\biggl(\frac{15}{36} + \frac{26}{36} + \frac{33}{36} + 1 + 1 + 1\biggr)
=
\frac1{216}(15 + 26 + 33 + 36 + 36 + 36)
=
\frac{182}{216}.
\]

Man könnte die Teilaufgabe b) auch so verstehen, dass exakt die
Augenzahl 10 erreicht werden muss. Wir bezeichnen dieses Ereignis
mit $E$. In diesem Fall bleiben obige
Formeln bestehen, wenn $A_l$ jetzt bedeutet, dass man im zweiten
Wurf genau die Augensumme $l$ erreicht. Zur Vermeidung von Verwechslungen
schreiben wir $B_l$ für das Ereignis, exakt die Augensumme $l$ erreicht.
Für diese $B_l$ ergibt sich
folgende modifizierte Tabelle:
\begin{center}
\begin{tabular}{|c|c|}
\hline
$l$&$P(B_l)$\\
\hline
$-2$&0\\
0&0\\
2&$\frac{1}{36}$\\
4&$\frac{3}{36}$\\
6&$\frac{5}{36}$\\
8&$\frac{5}{36}$\\
\hline
\end{tabular}
\end{center}
und die gesuchte Wahrscheinlichkeit wird
\begin{align*}
P(E|P)
&=P(P_1|P) P(B_8) + P(P_2|P) P(B_6) + P(P_3|P) P(B_4) \\
&\qquad + P(P_4|P)P(B_2) + P(P_5|P) P(B_0) + P(P_6|P) P(B_{-2})\\
&=
\frac1{6}\biggl(\frac{5}{36} + \frac{5}{36} + \frac{3}{36} + \frac{1}{36} + 0 + 0\biggr)
\\
&=
\frac1{216}(5 + 5 + 3 + 1 + 0 + 0)
=
\frac{14}{216}=\frac{7}{108}=0.0648148.
\end{align*}

\item
Es wird $P(Z)$ gesucht.
In der Teilaufgabe b) haben wir entweder das Ereignis
$Q$ untersucht, welches auch noch die Fälle umfasst, in denen genau
die Augensumme $10$ erzielt wird, oder das Ereignis $E$, exakt die Augensumme
10 zu erzielen. Es gilt natürlich
\[
Z\cup E = Q,\qquad Z\cap E=\emptyset.
\]
also
\[
P(Z)=P(Q)-P(E).
\]
Auf jeden Fall muss man die Fälle unterschiedlich behandeln, in denen
im ersten Wurf ein Pasch erzielt wurde oder eben nicht. Zusammensetzen
kann man dann das Resultat wieder mit dem Satz über die totale
Wahrscheinlichkeit. Es gilt
\begin{align*}
P(Q)&=P(Q|P)P(P) + P(Q|\bar P)P(\bar P)\\
P(E)&=P(E|P)P(P) + P(E|\bar P)P(\bar P)
\end{align*}
Bekannt sind aus Teilaufgabe a) die Wahrscheinlichkeiten
$P(P)=\frac16$
$P(\bar P)=1-P(P)=1-\frac16=\frac56$. Man muss jetzt also nur noch die
bedingten Wahrscheinlichkeiten ermitteln.

Je nachdem, wie man die Teilaufgabe b) interpretiert hat, hat man $P(Q|P)$
oder $P(E|P)$ bereits bestimmt. Man muss also zunächst nach dem gleichen
Muster die noch fehlende bedingte Wahrscheinlichkeit ausrechnen.

Ausserdem braucht man noch die Wahrscheinlichkeiten $P(Q|\bar P)$ und
$P(E|\bar P)$. Diese ermittelt man durch abzählen der Fälle, in denen
das Ereignis $Q$ bzw.~$E$ ohne Pasch eintritt, also insbesondere in nur
einem Wurf.

Im Falle des Ereignisses $Q$, also mindestens Augenzahl 10, geschieht dies
mit den Elementarereignissen $\{(4,6),(4,5),(5,4),(6,4)\}$, also bei 4 der 30
möglichen Elementarreignisse ohne Pasch\footnote{In $\bar P$ sind tritt
niemals ein Pasch auf, wenn man also Wahrscheinlichkeiten in dieser Menge
berechnet, darf man keinen Pasch mitzählen, weder bei den
günstigen Erreignissen noch bei den Ereignissen der Grundgesamtheit.}.
Im Falle von $E$, also exakt
der Augenzahl $10$ geschieht es hingegen in nur zwei dieser
Fälle $\{(4,6), (6,4)\}$, also mit Wahrscheinlichkeit $\frac{2}{30}$.

Damit kann man jetzt die Wahrscheinlichkeiten zusammensetzen:
\begin{align*}
P(Q)&=P(Q|P)P(P) + P(Q|\bar P)P(\bar P)\\
    &=\frac{182}{216}\cdot\frac16 + \frac4{30}\cdot\frac56\\
    &=\frac{182}{216}\cdot\frac16 + \frac4{6}\cdot\frac16\\
    &=\frac{182}{216}\cdot\frac16 + \frac{24}{216}\\
    &=\frac{206}{216}\\
P(E)&=P(E|P)P(P) + P(E|\bar P)P(\bar P)\\
    &=\frac{14}{216}\cdot\frac16 + \frac2{30}\cdot\frac56\\
    &=\frac{14}{216}\cdot\frac16 + \frac2{6}\cdot\frac16\\
    &=\frac{14}{216}\cdot\frac16 + \frac{12}{216}\\
    &=\frac{14}{1296} + \frac{72}{1296}\\
    &=\frac{86}{1296}
    =\frac{43}{648}
\end{align*}
Damit bekommt man jetzt für $P(Z)$ aus
\[
P(Z)=P(Q)-P(E)
=
    \frac{206}{216}
    -\frac{43}{648}
=\frac{1193}{1296}=0.92052
\qedhere
\]
\end{teilaufgaben}
\end{loesung}

