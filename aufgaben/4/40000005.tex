Bei einem mikromechanischen Sensor wird eine Temperaturabhängigkeit
der Sensorwerte in Neutrallage des Sensors festgestellt,
die wahrscheinlich auf die thermische
Ausdehnung zurückzuführen ist. Man weiss ja, dass sich
ein Stab proportional zu seiner Temperatur verlängert. Folgende
Temperatur- und Sensorwerte sind gemessen worden:
\begin{center}
\begin{tabular}{l|c|c|c|c|c}
Temperatur&270&280&290&300\\
\hline
Sensorwert&3021&3432&3867&4302\\
\end{tabular}
\end{center}
\begin{teilaufgaben}
\item Bestimmen sie möglichst genau den Zusammenhang zwischen
Sensorwerten $S$ und Temperaturwerten $T$.
\item Bei welcher Temperatur überschreitet der erwartete Sensorwert
die Marke $4000$?
\end{teilaufgaben}

\begin{loesung}
\begin{teilaufgaben}
\item Berechnung der Regressionskoeffizienten nach dem bekannten
Schema ergibt
\[
S=-8536.8 + 42.78\cdot T
\]
\item
Der Wert $4000$ wird also bei
\[
T=\frac{4000 + 8536.80}{42.78}=293.1
\]
überschritten.
\end{teilaufgaben}
Die manuelle Berechnung der Koeffizienten kann mit folgender Tabelle
erfolgen:
\begin{center}
\begin{tabular}{|c|r|r|r|r|r|}
\hline
&$T$&$T^2$&$S$&$S^2$&$ST$\\
\hline
1&270&72900& 3021& 9126441& 815670\\
2&280&78400& 3432& 11778624& 960960\\
3&290&84100& 3867& 14953689& 1121430\\
4&300&90000& 4302& 18507204& 1290600\\
\hline
$\sum$&1140& 325400& 14622& 54365958& 4188660\\
\hline
\end{tabular}
\end{center}
Daraus kann die Summen ablesen, mit denen die Koeffizienten berechnet
werden:
\begin{align*}
a&=\frac{n\sum ST-\sum S\sum T}{n\sum T^2-(\sum T)^2}
\\
&=
\frac{
4\cdot 4188660 - 14622\cdot 1140
}{
4\cdot 325400 - 1140^2
}
=
\frac{
85560
}{
2000
}
\frac{85560}{2000}
=42.78
\\
b&=\textstyle \frac1n\sum S-a\frac1n\sum T
\\
&=
\frac14\cdot 14622-42.78\cdot\frac14\cdot 1140
=-8536.8
\end{align*}
Vertauscht man die Rollen von $T$ und $S$ ergibt sich mit der
gleichen Methode die Beziehung
\[
T=0.02337\cdot S +199.56733
\]
Setzt man darin $S=4000$ ein, erhält man direkt
$293.04733$.
\end{loesung}

